\documentclass{article}
\usepackage{cancel}
\usepackage{amsmath, amssymb, amsfonts}
\usepackage[binary-units]{siunitx}
\usepackage{tikz}
\usepackage{float}
\usepackage{pgffor}
\usepackage{import}
\usepackage{vwcol}
\usepackage{fontawesome}
\usepackage{stmaryrd}
\usepackage{multicol}
\usepackage{pdfpages}
\usepackage{transparent}
\usepackage{xcolor}
\usepackage{scalerel}
\usepackage{stackengine}
\usepackage{algpseudocode}
\newcommand{\diag}{\operatorname{diag}}
\newcommand{\card}{\operatorname{card}}
\newcommand{\tr}{\operatorname{tr}}
\newcommand{\rg}{\operatorname{rg}}
\renewcommand{\epsilon}{\varepsilon}
\newcommand{\equivalent}[1]{\underset{#1}{\sim}}
\newcommand{\R}{\mathbb{R}}
\newcommand{\Q}{\mathbb{Q}}
\newcommand{\C}{\mathbb{C}}
\newcommand{\N}{\mathbb{N}}
\newcommand{\Z}{\mathbb{Z}}
\newcommand{\cM}{\mathcal{M}}
\newcommand{\cO}{\mathcal{O}}
\newcommand{\dx}{\mathrm{d}x}
\newcommand{\dy}{\mathrm{d}y}
\newcommand{\dz}{\mathrm{d}z}
\newcommand{\dt}{\mathrm{d}t}
\newcommand{\df}{\mathrm{d}f}
\newcommand{\Sp}{\operatorname{Sp}}
\newcommand{\dangersign}[1][2ex]{%
  \renewcommand\stacktype{L}%
  \scaleto{\stackon[1.3pt]{\color{red}$\triangle$}{\tiny !}}{#1}%
}

\usepackage[a4paper,top=4cm,bottom=4cm,left=3cm,right=3cm,marginparwidth=1.75cm]{geometry}
% \newcommand{\incfig}[2][1]{%
%     \def\svgwidth{#1\columnwidth}
%     \import{./figures/}{#2.pdf_tex}
% }
% 
\newenvironment{theorem}[1][\unskip]{
	\paragraph{Théorème #1}

}{}

\newenvironment{proof}[1][\unskip]{
	\def\temp{#1}\ifx\temp\empty
		\paragraph{Preuve}
	\else
		\paragraph{Preuve \emph{(#1)}}
	\fi

}{}

\newenvironment{definition}[1][\unskip]{
	\paragraph{Définition: #1}

}{}

\newenvironment{warning}[1][\unskip]
{
	\vspace{1cm}
	\begin{minipage}[c]{0.1\linewidth}
	\dangersign[8ex] 
\end{minipage}%
\begin{minipage}[l]{0.9\linewidth}
}
{
	\end{minipage}
	\vspace{1cm}
}

% \pdfsuppresswarningpagegroup=1


\begin{document}
   \section{Adressage IPv4}
   \subsection{Lecture d'adresses}
    
   \begin{description}
       \item[80.2.3.12/16] adresse de classe A\footnote{on prend A par défaut, sans le masque} de machine avec réseau sur 16 bits (80.2.0.0) et broadcast 80.2.255.255
       \item[147.127.2.0/16] classe B\footnote{masque de 2 octets} réseau 147.127.0.0 sur 16 bits , broadcast 147.127.255.255
       \item[1.2.3.4/5] adresse de machine avec réseau sur 5 bits (0.0.0.0/5) et broadcast $(00000111)_2.255.255.255 = 7.255.255.255$
        \item[147.127.0.0] réseau de classe B, broadcast 147.127.255.255
        \item[192.168.0.0] réseau sur 16 bits, broadcast 192.168.255.255 (adresses privées)
        \item[223.4.17.0] réseau 223.4.16.0 sur 21 bits, broadcast: adresse $223.4.(0001 0001)_2.0$, masque $255.255.(1111 1000)_2.0$ donc  $223.4.(0001 0111)_2.255 = 223.4.23.255$
        \item[10.0.0.0] réseau sur 10.0.0.0 sur ? bits (classe A $\implies$ 1 octet $\implies$ 10.0.0.0/8), broadcast: 10.255.255.255 (adresses privées)
        \item[255.255.255.255] broadcast, TOUT INTERNEEEEEEEEEEEEEEEET
        \item[127.0.0.1] there's no place like $\leftarrow$
   \end{description}

   \subsection{Découpage d'une plage d'adresse}
   \begin{align*}
       \underbrace{40.0.0}_{\text{net}}.(\underbrace{000}_{\text{subnet}}\underbrace{0.0000}_{\text{hosts}})_2/24
   \end{align*}

   \begin{description}
       \item[réseau A] 30 terminaux donc $\left\lceil \log_2(30)  \right\rceil = 5$ bits\ldots trop de monde!
   \end{description}

   On va plutôt faire un trie.

   \begin{figure}[H]
       \centering
       
       \digraph[scale=0.75]{trie}{
        root -> "B 
        40.0.0.128/25
        128-2 hosts" [label="1"]
        root -> node0 [label="0"]
        node0 -> "A
        40.0.0.64/26
        64-2 hosts"[label="1"]
        node0 -> node00 [label="0"]
        node00 -> "C
        40.0.0.32/27
        32-2 hosts" [label="1"]
        node00 -> node000 [label="0"]
        node000 -> "D
        40.0.0.16/28
        16-2 hosts" [label="1"]
        node000 -> "I
        40.0.0.0/28
        16-2 hosts"[label="0"]
    }
       \caption{Réseau}
       \label{fig:reseau}
   \end{figure}

    \begin{figure}[H]
        \centering
        \digraph[scale=0.75]{adressage}{
        A -> I [label=".9"]
        I -> A [label=".126"]
        B -> I [label=".10"]
        I -> B
        C -> I [label=".11"]
        I -> C
        D -> I [label=".12"]
        I -> D
        }
        \caption{Plan d'adressage}
        \label{fig:plan-adressage}
    \end{figure}

    Soit $M$ un host source sur le réseau A et $D$ un host destination sur le réseau D

   \begin{table}[H]
       \centering
       \caption{Table de routage de M}
       \label{tab:table-routage}
       \begin{tabular}{llll}
       dest & mask & gateway & interface \\\hline
       40.0.0.64 & & $\ast$\footnote{pas besoin, on reste dans le même réseau} & int0 \\
       0.0.0.0 & & 40.0.0.126 & int0 \\
       \end{tabular}
   \end{table}

   \begin{table}[H]
       \centering
       \caption{Table de routage de $R_{AI}$\footnote{routeur entre A et I}}
       \label{tab:routage-RAI}
       \begin{tabular}{llll}
           dest & mask & gateway & interface \\\hline
           40.0.0.64 & 255.255.255.192 & & int0 \\
           40.0.0.8 & 255.255.255.248 & & int1 \\
           40.0.0.128 & 255.255.255.128 & .10 & int1 \\
           40.0.0.32 & 255.255.255.224 & .11 & int1 \\
           40.0.0.16 & 255.255.255.240 & .12 & int1 \\
       \end{tabular}
   \end{table}

\end{document}
