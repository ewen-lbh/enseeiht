\documentclass{article}

\usepackage{cancel}
\usepackage[pdf]{graphviz}
\usepackage{amsmath, amssymb, amsfonts}
\usepackage[binary-units]{siunitx}
\usepackage{tikz}
\usepackage{float}
\usepackage{pgffor}
\usepackage{import}
\usepackage{vwcol}
\usepackage{fontawesome}
\usepackage{stmaryrd}
\usepackage{multicol}
\usepackage{pdfpages}
\usepackage{transparent}
\usepackage{xcolor}
\usepackage{scalerel}
\usepackage{stackengine}
\usepackage{algpseudocode}
\newcommand{\diag}{\operatorname{diag}}
\newcommand{\argmax}{\operatorname{max}}
\newcommand{\card}{\operatorname{card}}
\newcommand{\tr}{\operatorname{tr}}
\newcommand{\rg}{\operatorname{rg}}
\renewcommand{\epsilon}{\varepsilon}
\newcommand{\equivalent}[1]{\underset{#1}{\sim}}
\newcommand{\R}{\mathbb{R}}
\newcommand{\1}{\mathbb{1}}
\newcommand{\Q}{\mathbb{Q}}
\newcommand{\C}{\mathbb{C}}
\newcommand{\N}{\mathbb{N}}
\newcommand{\Z}{\mathbb{Z}}
\newcommand{\cM}{\mathcal{M}}
\newcommand{\cT}{\mathcal{T}}
\newcommand{\cO}{\mathcal{O}}
\newcommand{\dx}{\mathrm{d}x}
\newcommand{\dy}{\mathrm{d}y}
\newcommand{\dz}{\mathrm{d}z}
\newcommand{\dt}{\mathrm{d}t}
\newcommand{\df}{\mathrm{d}f}
\newcommand{\Sp}{\operatorname{Sp}}
\newcommand{\dangersign}[1][2ex]{%
  \renewcommand\stacktype{L}%
  \scaleto{\stackon[1.3pt]{\color{red}$\triangle$}{\tiny !}}{#1}%
}

\usepackage[a4paper,top=4cm,bottom=4cm,left=3cm,right=3cm,marginparwidth=1.75cm]{geometry}
% \newcommand{\incfig}[2][1]{%
%     \def\svgwidth{#1\columnwidth}
%     \import{./figures/}{#2.pdf_tex}
% }
% 
\newenvironment{theorem}[1][\unskip]{
	\paragraph{Théorème #1}

}{}

\newenvironment{proof}[1][\unskip]{
	\def\temp{#1}\ifx\temp\empty
		\paragraph{Preuve}
	\else
		\paragraph{Preuve \emph{(#1)}}
	\fi

}{}

\newenvironment{definition}[1][\unskip]{
	\paragraph{Définition: #1}

}{}

\newenvironment{warning}[1][\unskip]
{
	\vspace{1cm}
	\begin{minipage}[c]{0.1\linewidth}
	\dangersign[8ex] 
\end{minipage}%
\begin{minipage}[l]{0.9\linewidth}
}
{
	\end{minipage}
	\vspace{1cm}
}

% \pdfsuppresswarningpagegroup=1


\begin{document}
    Après filtrage, le signal décodé en symboles est

    \begin{align*}
        z(t) = \sum_{k=0}^{n} a_k p(t-kT)
    \end{align*}

    Avec $p$ le signal après filtrage.

    Donc les symboles sont obtenus tout les $mT + t_0$:

    \begin{align*}
        z(mT+t_0) &= \sum_{k=0}^{n} a_k p((m-k)T+t_0) \\
                  &= a_m p(t_0) + \underbrace{\sum_{l\neq 0} a_{n-l} p(lT+t_0)}_{\text{interférence entre signaux IES}}  \\
    \end{align*}

    On voudrait juste $a_m p(t_0)$. Et l'IES existe même en l'absence de bruit (comme ici).

    On veut donc
\begin{theorem}[Nyquist-Temporel]


\begin{align*}
    \exists t_0, p(t_0) \neq  0 \land \forall l\neq 0, p(lT + t_0) = 0
\end{align*}
    \end{theorem}

    Dans le domaine fréquentiel:

    \begin{align*}
        \sum_{k} P_{t_0}(f-\frac{k}{T}) \text{   est constante} 
    \end{align*}

    Avec un cas particulier $|P_{t_0}(f)| = 0$ pour $|f| > \frac{1}{T}$, la condition devient:

    \begin{align*}
        P_{t_0}(f) + P_{t_0}(f - \frac{1}{T}) \text{   est constante}
    \end{align*}

    On a un point de symmétrie $f = \frac{1}{2T}$ 

\section{Exercice 1}
\subsection{}
En prenant $t_0\in [T_s/2, T_s]$, le critère et satisfait.

\subsection{}
On prend la forme d'onde $h \ast h_r$ et on la duplique tout les $T_s$ (ça se chevauche).
Ensuite, on obtient le signal final en sommant les courbes.

\subsection{}
On se donne un graphe avec un axe temporel de $0$ à $T_s$. On prend chaque tronçon du signal de longueur  $T_s$, en commençant à $t=T_s$ (le premier symbole y'a l'initialisation).

On obtient la figure \ref{fig:œil}

 \begin{figure}[h]
    \centering
    \begin{tikzpicture}
        \draw[->] (0, -5) -- (0, 5);
        \draw[->] (-1, 0) -- (4, 0);
        \draw[blue] (0, 4) -- (4, 4);
        \draw[blue] (0, -4) -- (4, -4);
        \draw[blue] (0, 4) -- (2, -4);
        \draw[blue] (0, -4) -- (2, 4);
        \draw[dashed] (2, -4) -- (2, 4);
        \draw (2, 0) node[below right]{$\frac{T_s}{2}$};
        \draw[<->] (2, 4.5) -- (4, 4.5) node[above]{valeurs de $t_0$ satisfiant Nyquist}
    \end{tikzpicture}
    \caption{Diagramme de l'œil}
    \label{fig:œil}
\end{figure}

\paragraph{Interprétation}
Avant $t=\frac{T_s}{2}$, il y a plus de deux valeurs, donc on ne peut pas échantilloner avec $t_0 < \frac{T_s}{2}$.

\section{Impact d'un canal de propagation}

Roll-off $\alpha$ d'un racine carrée de cosinus surélevé: dans  $[0, 1]$, tel que la durée où la courbe est non-maximale (``transition") est  $(1+\alpha) \frac{R_s}{2}$.

\subsection{}

On prend le critère en fréquence.

On multiplie les réponses fréquentielles des trois filtres.

Mais on connaît pas $R_s$.

Donc on a deux cas: soit les phases de roll-off sont \emph{totalement} conservées par la multiplication par la porte, i.e. $(1+\alpha) \frac{R_s}{2} < 1200$, et le cas contraire.

Si la conservation n'est que partielle, il y aura des bouts de signal présent en bas des courbes mais pas en haut, et on perd la symmétrie.

\subsection{}

La condition sur $R_s$ est $R_s < 2000$, donc on a maximum  $2000$ symboles.


\subsection{}

Pour $m$ bits, on a $M = 2^{m}$ symboles, donc $m = \log_2(M)$

Donc on a $R_b = m R_s$.

Donc  \begin{align*}
    R_s &\le 2000 \\
    \frac{R_b}{m} &\le  2000 \\
    m &\ge  \frac{R_b}{2000} \\
    M &\ge 4 \quad&\text{car $R_b = 4000$}
\end{align*}

\subsection{}
\subsubsection{}
Fréquentiellement, c'est une porte de largeur $\frac{1}{T_s}$. Une autre porte ne change rien vu qu'elle est de la même largeur.

On a \begin{align*}
    \sum_{k} P(f- \frac{k}{T_s}) \quad\text{est constante}
\end{align*}

Car en sommant chaque porte avec des décalages dans $\frac{1}{T_s} \Z$, la courbe devient constante.

\end{document}
