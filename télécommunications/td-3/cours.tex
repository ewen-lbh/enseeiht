\documentclass{article}
\usepackage{cancel}
\usepackage[pdf]{graphviz}
\usepackage{amsmath, amssymb, amsfonts}
\usepackage[binary-units]{siunitx}
\usepackage{tikz}
\usepackage{float}
\usepackage{pgffor}
\usepackage{import}
\usepackage{vwcol}
\usepackage{fontawesome}
\usepackage{stmaryrd}
\usepackage{multicol}
\usepackage{pdfpages}
\usepackage{transparent}
\usepackage{xcolor}
\usepackage{scalerel}
\usepackage{stackengine}
\usepackage{algpseudocode}
\newcommand{\diag}{\operatorname{diag}}
\newcommand{\argmax}{\operatorname{max}}
\newcommand{\card}{\operatorname{card}}
\newcommand{\tr}{\operatorname{tr}}
\newcommand{\rg}{\operatorname{rg}}
\renewcommand{\epsilon}{\varepsilon}
\newcommand{\equivalent}[1]{\underset{#1}{\sim}}
\newcommand{\R}{\mathbb{R}}
\newcommand{\1}{\mathbb{1}}
\newcommand{\Q}{\mathbb{Q}}
\newcommand{\C}{\mathbb{C}}
\newcommand{\N}{\mathbb{N}}
\newcommand{\Z}{\mathbb{Z}}
\newcommand{\cM}{\mathcal{M}}
\newcommand{\cT}{\mathcal{T}}
\newcommand{\cO}{\mathcal{O}}
\newcommand{\dx}{\mathrm{d}x}
\newcommand{\dy}{\mathrm{d}y}
\newcommand{\dz}{\mathrm{d}z}
\newcommand{\dt}{\mathrm{d}t}
\newcommand{\df}{\mathrm{d}f}
\newcommand{\Sp}{\operatorname{Sp}}
\newcommand{\dangersign}[1][2ex]{%
  \renewcommand\stacktype{L}%
  \scaleto{\stackon[1.3pt]{\color{red}$\triangle$}{\tiny !}}{#1}%
}

\usepackage[a4paper,top=4cm,bottom=4cm,left=3cm,right=3cm,marginparwidth=1.75cm]{geometry}
% \newcommand{\incfig}[2][1]{%
%     \def\svgwidth{#1\columnwidth}
%     \import{./figures/}{#2.pdf_tex}
% }
% 
\newenvironment{theorem}[1][\unskip]{
	\paragraph{Théorème #1}

}{}

\newenvironment{proof}[1][\unskip]{
	\def\temp{#1}\ifx\temp\empty
		\paragraph{Preuve}
	\else
		\paragraph{Preuve \emph{(#1)}}
	\fi

}{}

\newenvironment{definition}[1][\unskip]{
	\paragraph{Définition: #1}

}{}

\newenvironment{warning}[1][\unskip]
{
	\vspace{1cm}
	\begin{minipage}[c]{0.1\linewidth}
	\dangersign[8ex] 
\end{minipage}%
\begin{minipage}[l]{0.9\linewidth}
}
{
	\end{minipage}
	\vspace{1cm}
}

% \pdfsuppresswarningpagegroup=1


\begin{document}
\section{Étude de chaînes de transmission en bande de base sur canal AWGN}

\subsection{}
\begin{description}
    \item[Modulateur] jusqu'à $\oplus$
    \item[Canal de propagation] $\oplus$
    \item[Démodulateur] le reste
\end{description}

\subsection{}
\subsubsection{Chaîne 1}

\begin{itemize}
    \item 
$h, h_r$: carrés de 0 à $T_s$
\item  $g$ : triangle symmétrique de 0 à $2T_s$, pic à $g(T_s) = T_s$.
\end{itemize}

\begin{enumerate}
    \item $\exists t_0 = T_s, g(t_0) \neq 0 \text{ et } g(t_0 + pT_s) = 0 \ \forall \p\in \Z^\ast$
        Donc la chaîne 1 respecte ta daronne.
    \item En l'absence d'interférences entre symboles, l'échantillon $z(t_0+mT_s)$ utilisé pour restimer le symbole $a_m$ émis à $mT_s$ est de la forme

         \begin{align*}
            z(t_0+mT_s) = z_m  = a_m g(t_0) + w_m
        \end{align*}

        où 

        \begin{align*}
            w_m = w(t_0 + mT_s)
        \end{align*}

        \begin{align*}
            \operatorname{SNR} := \frac{E(|a_mg(t_0)|^2)}{\sigma_w^2} = \frac{(\sigma^2a + |m_a|^2) |g(t_0)|^2}{\sigma_w^2} = \frac{2T_s}{N_0}
        \end{align*}

        Avec $\sigma a$ variance des symboles

        et 
\begin{align*}
    \sigma_w^2 = P_w &= \int |H_r(f)|^2 S_m(f)  \\
    &= \int |h(t)|^2 \dt  \times \frac{2N_0}{2} = T_s \frac{N_0}{2}  \\
\end{align*}

\item 

\texttt{Figure: deux gaussiennes avec des pics à $\pm g(t_0)$, se croisent en $t=S=0$}

On choisit le seuil de décision à $S=0$ car pour $z_m < 0$, $p(z_m | a_m = -1) > p(z_m - a_m = 1)$ et, pour $z_m = 0$, $p(z_m | a_m = +1) > p(z_m | a_m = 1)$

\item $P_b = P(a_m = 1) \underbrace{P(\hat{a_m} = -1 | a_m = 1]}_A + P(a_m = -1) \underbrace{P(\hat{a_m} = +1 | a_m = -1)}_B$
    Si les symboles sont iid alors $P(a_m = 1) = P(a_m = -1) = \frac{1}{2}$

    \begin{align*}
        A &= P(z_m < 0 | a_m = 1) \\
          &= P(g(t_0) + w_m < 0 ] \\
          &= P(w_m < g(t_0)]  \\
          &= P(w_m > g(t_0)] \\
          &= \int_{g(t_0)}^{\infty} \frac{1}{\sqrt{2 \pi} \sigma_w} e^{- \frac{x^2}{2 \sigma_w^2}} \dx \\
          &= \int_{\frac{g(t_0)}{\sigma_w}}^{\infty} \frac{1}{\sqrt{2 \pi} \sigma_w} e^{-\frac{u^2}{2}} \sigma_w \du \\
          &= Q\left( \frac{g(t_0)}{\sigma_w} \right)  \\
          &= Q\left( \frac{T_s}{\sigma_w} \right)  \\
    \end{align*}

    \begin{align*}
        B &= P(z_m > 0 | a_m = -1) \\
          &= P(-g(t_0) + w_m > 0 ] \\
          &= P(w_m > g(t_0)]  \\
          &= Q\left( \frac{T_s}{\sigma_w} \right)  \\
    \end{align*}

    Donc \begin{align*}
        P_s = P_b = Q\left( \frac{T_s}{\sigma_w} \right) 
    \end{align*}

\item 
    \begin{align*}
        \sigma_w^2 = \frac{N_0 T_s}{2}
    \end{align*}

    \item \begin{align*}
        E_s = P_x T_s
    \end{align*}

    \begin{align*}
        P_x &= \int S_x(f) \df \\
        &= \frac{\sigma_a^2}{T_s} \int |H(f)|^2 \df \\
        &= \sigma_a^2 \\
        &= 1 \\
    \end{align*}

    \begin{align*}
        E_s = T_s \quad&\text{car $E_s = g(t_0)$}
    \end{align*}

\item 
    \begin{align*}
        E_s = \log_2(M) E_b \quad&\text{$M=2$, $E_s = E_b = T_s$ et $P_b = Q\left( \frac{E_b}{\sqrt{N_0 E_b} } \right) = Q\left( \sqrt{\frac{2 E_b}{N_0}}  \right)  $}
    \end{align*}

\end{enumerate}

\subsubsection{Chaîne 2}

\begin{itemize}
    \item $h$: carré de 0 à $T_s$
    \item $h_r$ : carré de 0 à $\frac{T_s}{2}$
    \item $g$ : 
        \begin{tikzpicture}
            \draw[->] (-0.5, 0) -- (4, 0);
            \draw[->] (0, -0.5) -- (0, 1.5);
            \draw[blue] (0, 0) -- (1, 1) -- (2, 1) -- (3, 0);
            \draw[blue,--] (1, 1) -- (1, 0);
            \draw[blue,--] (2, 1) -- (2, 0);
        \end{tikzpicture}
\end{itemize}

\begin{enumerate}
    \item $\exists t_0 \in [\frac{T_s}{2}, T_s], g(t_0) \neq 0 \text{ et } g(t_0 + pT_s) = 0 \ \forall \p\in \Z^\ast$
        Donc la chaîne 2 respecte ta daronne.
    \item En l'absence d'interférences entre symboles, l'échantillon $z(t_0+mT_s)$ utilisé pour restimer le symbole $a_m$ émis à $mT_s$ est de la forme

         \begin{align*}
            z(t_0+mT_s) = z_m  = a_m g(t_0) + w_m
        \end{align*}

        où 

        \begin{align*}
            w_m = w(t_0 + mT_s)
        \end{align*}

        \begin{align*}
            \operatorname{SNR} := \frac{E(|a_mg(t_0)|^2)}{\sigma_w^2} = \frac{(\sigma^2a + |m_a|^2) |g(t_0)|^2}{\sigma_w^2} = \frac{T_s}{N_0}
        \end{align*}

        Avec $\sigma a$ variance des symboles

        et 
\begin{align*}
    \sigma_w^2 = P_w &= \int |H_r(f)|^2 S_m(f)  \\
    &= \int |h(t)|^2 \dt  \times \frac{2N_0}{2} = T_s \frac{N_0}{4}  \\
\end{align*}

\item 
\texttt{Figure: deux gaussiennes avec des pics à $\pm g(t_0)$, se croisent en $t=S=0$}
On choisit le seuil de décision à $S=0$ car pour $z_m < 0$, $p(z_m | a_m = -1) > p(z_m - a_m = 1)$ et, pour $z_m = 0$, $p(z_m | a_m = +1) > p(z_m | a_m = 1)$

\item $P_b = P(a_m = 1) \underbrace{P(\hat{a_m} = -1 | a_m = 1]}_A + P(a_m = -1) \underbrace{P(\hat{a_m} = +1 | a_m = -1)}_B$
    Si les symboles sont iid alors $P(a_m = 1) = P(a_m = -1) = \frac{1}{2}$

    \begin{align*}
        A &= P(z_m < 0 | a_m = 1) \\
          &= P(g(t_0) + w_m < 0 ] \\
          &= P(w_m < g(t_0)]  \\
          &= P(w_m > g(t_0)] \\
          &= \int_{g(t_0)}^{\infty} \frac{1}{\sqrt{2 \pi} \sigma_w} e^{- \frac{x^2}{2 \sigma_w^2}} \dx \\
          &= \int_{\frac{g(t_0)}{\sigma_w}}^{\infty} \frac{1}{\sqrt{2 \pi} \sigma_w} e^{-\frac{u^2}{2}} \sigma_w \du \\
          &= Q\left( \frac{g(t_0)}{\sigma_w} \right)  \\
          &= Q\left( \frac{T_s}{2 \sigma_w} \right)  \\
    \end{align*}

    \begin{align*}
        B &= P(z_m > 0 | a_m = -1) \\
          &= P(-g(t_0) + w_m > 0 ] \\
          &= P(w_m > g(t_0)]  \\
          &= Q\left( \frac{T_s}{2 \sigma_w} \right)  \\
    \end{align*}

    Donc \begin{align*}
        P_s = P_b = Q\left( \frac{T_s}{2 \sigma_w} \right) 
    \end{align*}

\item 
    \begin{align*}
        \sigma_w^2 = \frac{N_0 T_s}{2}
    \end{align*}

    \item \begin{align*}
        E_s = P_x T_s
    \end{align*}

    \begin{align*}
        P_x &= \int S_x(f) \df \\
        &= \frac{\sigma_a^2}{T_s} \int |H(f)|^2 \df \\
        &= \sigma_a^2 \\
        &= 1 \\
    \end{align*}

    \begin{align*}
        E_s = T_s \quad&\text{car $E_s = g(t_0)$}
    \end{align*}

\item 
    \begin{align*}
        E_s = \log_2(M) E_b \quad&\text{$M=2$, $E_s = E_b = T_s$ et $P_b = Q\left( \frac{E_b}{\sqrt{N_0 E_b} } \right) = Q\left( \sqrt{\frac{2 E_b}{N_0}}  \right)  $}
    \end{align*}

\end{enumerate}

\subsection{}
Le cas de la chaîne 1 correspond à l'utilisation d'un filtre adapté.
Tout changement par rapport à ce cas est donc sous optimal en termes de probabilité d'erreur binaire.

\section{Étude du mapping}


\subsection{}

Le mapping respecte le code de \emph{GRAY}.

\texttt{Figure: 4 gaussiennes avec des pics en $\{\pm 1, \pm 3\} g(t_0)$}

\begin{align*}
    p(\hat{a_m} = -1 | a_m = -3) &= p(z_m < 0 | a_m = -3)  \\
    &= p(z(t_0 + mT_s) < 0 | a_m = -3) \\
    &= p(a_m g(t_0) + w_m < 0 | a_m = -3) \quad&\text{avec $w_m \sim \mathcal{N}(0, \delta_w^2)$ et $\delta_w^2 = \frac{N_0}{2} \int |h_r(t)|^2 \dt$} \\
    &= p(-2g(t_0) < -3g(t_0) + w_m < 0)  \\
    &= p(w_m \in [1, 3]g(t_0)) \\
    &= Q\left( \frac{g(t_0)}{\delta_w} \right) - Q\left( \frac{3g(t_0)}{\delta_w} \right)  \\
\end{align*}

\end{document}
