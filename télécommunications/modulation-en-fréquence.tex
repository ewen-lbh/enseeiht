\documentclass{article}
\usepackage{cancel}
\usepackage{amsmath, amssymb, amsfonts}
\usepackage[binary-units]{siunitx}
\usepackage{tikz}
\usepackage{float}
\usepackage{pgffor}
\usepackage{import}
\usepackage{vwcol}
\usepackage{fontawesome}
\usepackage{stmaryrd}
\usepackage{multicol}
\usepackage{pdfpages}
\usepackage{transparent}
\usepackage{xcolor}
\usepackage{scalerel}
\usepackage{stackengine}
\usepackage{algpseudocode}
\newcommand{\diag}{\operatorname{diag}}
\newcommand{\card}{\operatorname{card}}
\newcommand{\tr}{\operatorname{tr}}
\newcommand{\rg}{\operatorname{rg}}
\renewcommand{\epsilon}{\varepsilon}
\newcommand{\equivalent}[1]{\underset{#1}{\sim}}
\newcommand{\R}{\mathbb{R}}
\newcommand{\Q}{\mathbb{Q}}
\newcommand{\C}{\mathbb{C}}
\newcommand{\N}{\mathbb{N}}
\newcommand{\Z}{\mathbb{Z}}
\newcommand{\cM}{\mathcal{M}}
\newcommand{\cO}{\mathcal{O}}
\newcommand{\dx}{\mathrm{d}x}
\newcommand{\dy}{\mathrm{d}y}
\newcommand{\dz}{\mathrm{d}z}
\newcommand{\dt}{\mathrm{d}t}
\newcommand{\df}{\mathrm{d}f}
\newcommand{\Sp}{\operatorname{Sp}}
\newcommand{\dangersign}[1][2ex]{%
  \renewcommand\stacktype{L}%
  \scaleto{\stackon[1.3pt]{\color{red}$\triangle$}{\tiny !}}{#1}%
}

\usepackage[a4paper,top=4cm,bottom=4cm,left=3cm,right=3cm,marginparwidth=1.75cm]{geometry}
% \newcommand{\incfig}[2][1]{%
%     \def\svgwidth{#1\columnwidth}
%     \import{./figures/}{#2.pdf_tex}
% }
% 
\newenvironment{theorem}[1][\unskip]{
	\paragraph{Théorème #1}

}{}

\newenvironment{proof}[1][\unskip]{
	\def\temp{#1}\ifx\temp\empty
		\paragraph{Preuve}
	\else
		\paragraph{Preuve \emph{(#1)}}
	\fi

}{}

\newenvironment{definition}[1][\unskip]{
	\paragraph{Définition: #1}

}{}

\newenvironment{warning}[1][\unskip]
{
	\vspace{1cm}
	\begin{minipage}[c]{0.1\linewidth}
	\dangersign[8ex] 
\end{minipage}%
\begin{minipage}[l]{0.9\linewidth}
}
{
	\end{minipage}
	\vspace{1cm}
}

% \pdfsuppresswarningpagegroup=1


\begin{document}

\section{Transposition en fréquence}

On prend le signal $m(t)$:

 \begin{align*}
    m(t) \cos(2 \pi f_p t)
\end{align*}

Ça translate la fréquence car 
\begin{align*}
    M(f) \ast \frac{1}{2} \left( \delta(f-f_p) + \delta(f+f_p) \right)  &= \frac{1}{2} \left( M(f-f_p) + M(f+f_p) \right)  \\
\end{align*}

\subsection{Pour $m(t)$ aléatoire}

\begin{align*}
    Rx(\tau) &= E(m(t) \cos(2 \pi f_p + \phi) m^\ast(t-\tau) \cos(2 \pi f_p(t-\tau) + \phi) \\
    &= E(m(t) m^\ast(t-\tau)) E(\frac{1}{2} \cos(2 \pi f_p \tau) + \frac{1}{2} \cos(2 \phi + \ldots)) \\
    &= R_m(\tau) \frac{1}{2} \cos(2 \pi f_p \tau) \\
    S_x(f) &= S_m(f)  \ast \frac{1}{2} \left( \frac{1}{2} \left( \delta(f-f_p) + \delta(f+f_p) \right)  \right)  \\
    &= \frac{1}{4} \left( S_m(f-f_p) + S_m(f+f_p)  \right)  \\
\end{align*}

\subsection{Rajout d'une porteuse}

 \[
     m(t) \cos(2 \pi f_p t) \rightsquigarrow \mathbf{( A + }m(t)\mathbf{)} \cos(2 \pi f_p t)
\] 

Avec $\mathbf{A}$ l'amplitude du signal original

\begin{itemize}
    \item On perd en puissance d'émission  (une partie de la puissance 
    \item Mais on évite une enveloppe erronée lors de passage par zéro dans le signal original
\end{itemize}

\subsection{Modulation numérique}

\subsubsection{Modulation bi-dimensionnelle}

On module en bande de base 2 fois avec deux fréquences, puis on transpose en fréquence en multipliant:

\begin{itemize}
    \item Par un cos
    \item par un sin (avec même fréquence et phase)
\end{itemize}



\subsection{Enveloppe complexe}

\begin{align*}
    x_e(t) = m_1(t) + jm_2(t)  = \sum_{k} \underbrace{(a_k + jb_k)}_{d_k} h(t-kT_s) 
\end{align*}


\end{document}
