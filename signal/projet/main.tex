\documentclass{article}

% Language setting
% Replace `english' with e.g. `spanish' to change the document language
\usepackage[french]{babel}

\usepackage{float}

% Set page size and margins
% Replace `letterpaper' with `a4paper' for UK/EU standard size
\usepackage[a4paper,top=2cm,bottom=2cm,left=3cm,right=3cm,marginparwidth=1.75cm]{geometry}

% Useful packages
\usepackage{siunitx}
\usepackage{amsmath}
\usepackage{pgfplots}
\pgfplotsset{compat=newest}
\usetikzlibrary{plotmarks}
\usetikzlibrary{arrows.meta}
\usepgfplotslibrary{patchplots}
\usepackage{grffile}
\pgfplotsset{plot coordinates/math parser=false}
\newlength\figureheight
\newlength\figurewidth
  
\usepackage{graphicx}
\usepackage{hyperref}
\newcommand{\TF}{\operatorname{TF}}
\newcommand{\sinc}{\operatorname{sinc}}

\title{Rapport de projet -- Traitement du signal}
\author{Florent Puy, Ewen Le Bihan}

\date{ENSEEIHT, département Sciences du Numérique}

\begin{document}
\maketitle



\tableofcontents

\section{Introduction}

Dans ce projet, nous avons eu à implémenter un modem suivant les règles V21 de l'union internationnale des télécommunications (UIT) en Matlab. Nous utiliserons la méthode de la modulation en fréquence numérique.

\section{Modem en fréquence}

Nous allons tout d'abord réaliser un signal NRZ binaire à partir duquel nous construirons ensuite un signal sinusoïdal modulé fréquence $F_0=\SI{1180}{\hertz}$ pour les bits 0 et de fréquence $F_1=\SI{980}{\hertz}$ pour les bits 1.
Nous comparerons ensuite les densités spéctrales de puissance théorique et pratique du signal NRZ et du signal modulé en fréquence.

\subsection{Génération d'un signal NRZ}

On génère tout d'abord un signal NRZ prennant deux valeurs, 0 ou 1, générées aléatoirement d'une durée $T_s=1/300 s$. On effectue cela sur $N_s$ périodes. Voici les résultats ainsi obtenus.

% This file was created by matlab2tikz.
%
%The latest updates can be retrieved from
%  http://www.mathworks.com/matlabcentral/fileexchange/22022-matlab2tikz-matlab2tikz
%where you can also make suggestions and rate matlab2tikz.
%
\definecolor{mycolor1}{rgb}{0.00000,0.44700,0.74100}%
%
\begin{tikzpicture}

\begin{axis}[%
width=4.521in,
height=3.559in,
at={(0.758in,0.488in)},
scale only axis,
xmin=0,
xmax=4000,
xlabel style={font=\color{white!15!black}},
xlabel={temps [s]},
ymin=-0.1,
ymax=1.1,
ylabel style={font=\color{white!15!black}},
ylabel={$\text{\$m}_\text{i}\text{(t)\$}$},
axis background/.style={fill=white},
title style={font=\bfseries},
title={Signal NRZ aléatoire}
]
\addplot [color=mycolor1, forget plot]
  table[row sep=crcr]{%
1	1\\
160	1\\
161	0\\
320	0\\
321	1\\
480	1\\
481	0\\
640	0\\
641	1\\
1440	1\\
1441	0\\
1600	0\\
1601	1\\
1920	1\\
1921	0\\
2240	0\\
2241	1\\
2400	1\\
2401	0\\
2720	0\\
2721	1\\
2880	1\\
2881	0\\
3040	0\\
3041	1\\
3840	1\\
3841	0\\
4000	0\\
};
\end{axis}
\end{tikzpicture}%

On calcule ensuite la densité spectrale de puissance de ce signal NRZ en utilisant la fonction $pwelch$ de Matlab utilisant un périodogramme de Welch.

On calcule ensuite la densité spectrale théorique vue en cours d'un signal NRZ:
\[
S_\text{NRZ}(f)=\frac{1}{4} T_s \sinc^2(\pi f T_s)+\frac{1}{4} \delta(f)
\]

On peut déormais comparer la densité spéctrale de puissance théorique à celle calculée:

% This file was created by matlab2tikz.
%
%The latest updates can be retrieved from
%  http://www.mathworks.com/matlabcentral/fileexchange/22022-matlab2tikz-matlab2tikz
%where you can also make suggestions and rate matlab2tikz.
%
\definecolor{mycolor1}{rgb}{0.00000,0.44700,0.74100}%
%
\begin{tikzpicture}

\begin{axis}[%
width=4.521in,
height=3.548in,
at={(0.758in,0.499in)},
scale only axis,
xmin=-25000,
xmax=25000,
xlabel style={font=\color{white!15!black}},
xlabel={fréquence [Hz]},
ymode=log,
ymin=1e-40,
ymax=1,
yminorticks=true,
ylabel style={font=\color{white!15!black}},
ylabel={Densité spectrale de puissance},
axis background/.style={fill=white},
title style={font=\bfseries},
title={Densité spectrale de puissance},
legend style={legend cell align=left, align=left, draw=white!15!black}
]
\addplot [color=mycolor1]
  table[row sep=crcr]{%
-24000	1.63292967995806e-09\\
-23953.0791788856	6.7966459789558e-09\\
-23906.1583577713	2.04807598616413e-08\\
-23859.2375366569	5.52796503081048e-08\\
-23812.3167155425	3.65762502160344e-08\\
-23765.3958944282	9.92843317004177e-09\\
-23718.4750733138	2.94765032381341e-09\\
-23671.5542521994	3.2300533034679e-09\\
-23624.633431085	1.36175432290685e-08\\
-23577.7126099707	4.48747761915906e-08\\
-23530.7917888563	4.6929820779037e-08\\
-23483.8709677419	1.79389740353176e-08\\
-23436.9501466276	5.0155609529083e-09\\
-23390.0293255132	2.59045700537738e-09\\
-23343.1085043988	7.43596877844657e-09\\
-23296.1876832845	2.85948109585483e-08\\
-23249.2668621701	5.64421629755727e-08\\
-23202.3460410557	2.86395558222412e-08\\
-23155.4252199413	8.25396800803398e-09\\
-23108.504398827	1.94329320187427e-09\\
-23061.5835777126	5.08638943875813e-09\\
-23014.6627565982	1.69458838984734e-08\\
-22967.7419354839	4.914913612614e-08\\
-22920.8211143695	4.503518652506e-08\\
-22873.9002932551	1.22306774439153e-08\\
-22826.9794721408	3.99863230483642e-09\\
-22780.0586510264	2.72394801905653e-09\\
-22733.137829912	1.03643576437291e-08\\
-22686.2170087977	3.68030586025632e-08\\
-22639.2961876833	5.32958240320471e-08\\
-22592.3753665689	2.3321315882338e-08\\
-22545.4545454545	5.83998123450221e-09\\
-22498.5337243402	2.32309500826676e-09\\
-22451.6129032258	6.04640707879494e-09\\
-22404.6920821114	2.31953046216101e-08\\
-22357.7712609971	5.38528183115835e-08\\
-22310.8504398827	3.72297236817084e-08\\
-22263.9296187683	9.99812925265712e-09\\
-22217.008797654	2.81356549578239e-09\\
-22170.0879765396	3.45843657466441e-09\\
-22123.1671554252	1.35129848896611e-08\\
-22076.2463343109	4.46819602560227e-08\\
-22029.3255131965	4.95817318415197e-08\\
-21982.4046920821	1.7975867310242e-08\\
-21935.4838709677	4.63857289232878e-09\\
-21888.5630498534	2.5506581825241e-09\\
-21841.642228739	7.70263256632904e-09\\
-21794.7214076246	3.08571530283524e-08\\
-21747.8005865103	5.33072741998019e-08\\
-21700.8797653959	3.20811100386197e-08\\
-21653.9589442815	7.80251761952471e-09\\
-21607.0381231672	2.48381899561016e-09\\
-21560.1173020528	4.25176887380567e-09\\
-21513.1964809384	1.86618704426435e-08\\
-21466.275659824	5.05406859378726e-08\\
-21419.3548387097	4.50032432930166e-08\\
-21372.4340175953	1.28228118670802e-08\\
-21325.5131964809	4.01734106882577e-09\\
-21278.5923753666	2.87260818030064e-09\\
-21184.7507331378	3.66140604361641e-08\\
-21137.8299120235	5.64460386433229e-08\\
-21090.9090909091	2.41604908774575e-08\\
-21043.9882697947	5.42842595687423e-09\\
-20997.0674486804	2.75708938672057e-09\\
-20950.146627566	5.65530334508873e-09\\
-20903.2258064516	2.6523078908794e-08\\
-20856.3049853372	5.2512551747733e-08\\
-20809.3841642229	3.89295935817746e-08\\
-20762.4633431085	1.0114692007242e-08\\
-20715.5425219941	3.35807679341274e-09\\
-20668.6217008798	3.97348137690216e-09\\
-20621.7008797654	1.38695326077387e-08\\
-20574.780058651	4.49852863181559e-08\\
-20527.8592375367	5.25809008067928e-08\\
-20480.9384164223	1.87211842760278e-08\\
-20434.0175953079	4.61449178592977e-09\\
-20387.0967741935	2.38106076700938e-09\\
-20293.2551319648	3.08320737261894e-08\\
-20246.3343108504	5.78016251029166e-08\\
-20199.4134897361	3.23115875615967e-08\\
-20152.4926686217	7.33244227513031e-09\\
-20105.5718475073	3.13798222617853e-09\\
-20058.651026393	4.34580072080288e-09\\
-20011.7302052786	1.92317840035986e-08\\
-19964.8093841642	5.32767164483021e-08\\
-19917.8885630499	4.5466286884904e-08\\
-19870.9677419355	1.49957385709902e-08\\
-19824.0469208211	3.72557595642003e-09\\
-19777.1260997067	2.98631164668493e-09\\
-19683.284457478	3.96103182384384e-08\\
-19636.3636363636	5.80565009802127e-08\\
-19589.4428152493	2.4856632706025e-08\\
-19542.5219941349	6.47623248152758e-09\\
-19495.6011730205	2.52604352584223e-09\\
-19448.6803519062	6.35936796218229e-09\\
-19401.7595307918	2.5660867585257e-08\\
-19354.8387096774	5.75675234171473e-08\\
-19307.917888563	4.07546022079208e-08\\
-19260.9970674487	1.05714047034758e-08\\
-19214.0762463343	3.59292533606328e-09\\
-19167.1554252199	3.86342636524016e-09\\
-19120.2346041056	1.47576623210414e-08\\
-19073.3137829912	4.78479473805968e-08\\
-19026.3929618768	5.51642986361981e-08\\
-18979.4721407625	1.96390723081506e-08\\
-18932.5513196481	5.15077964557016e-09\\
-18885.6304985337	2.71134938618296e-09\\
-18838.7096774194	8.74087396109374e-09\\
-18791.788856305	3.31771249494441e-08\\
-18744.8680351906	6.13864820937776e-08\\
-18697.9472140762	3.34970399501938e-08\\
-18651.0263929619	8.37072089261064e-09\\
-18604.1055718475	2.87922041076996e-09\\
-18557.1847507331	5.43577748908519e-09\\
-18510.2639296188	1.85661246399783e-08\\
-18463.3431085044	5.82329957814198e-08\\
-18416.42228739	4.90240654430953e-08\\
-18369.5014662757	1.54300743529304e-08\\
-18322.5806451613	3.92992905541118e-09\\
-18275.6598240469	2.96251171265624e-09\\
-18228.7390029326	1.18683618630837e-08\\
-18181.8181818182	4.37292248464953e-08\\
-18134.8973607038	6.03262610286499e-08\\
-18087.9765395894	2.68915781436197e-08\\
-18041.0557184751	7.08178641222401e-09\\
-17994.1348973607	2.68312175901679e-09\\
-17947.2140762463	7.13643194676664e-09\\
-17900.293255132	2.72343762349005e-08\\
-17853.3724340176	6.12812707912036e-08\\
-17806.4516129032	4.4547319307051e-08\\
-17759.5307917889	1.2143119359815e-08\\
-17712.6099706745	3.04866750553495e-09\\
-17665.6891495601	4.08187825655986e-09\\
-17618.7683284457	1.61131559971182e-08\\
-17571.8475073314	5.13912623789616e-08\\
-17524.926686217	6.12153829866175e-08\\
-17478.0058651026	1.95814833371346e-08\\
-17431.0850439883	5.76248727810637e-09\\
-17384.1642228739	3.07933928062263e-09\\
-17337.2434017595	9.01308904767838e-09\\
-17290.3225806452	3.62316319035331e-08\\
-17243.4017595308	6.65923415622805e-08\\
-17196.4809384164	3.60964470944908e-08\\
-17149.5601173021	9.55381024357878e-09\\
-17102.6392961877	2.98305069884662e-09\\
-17055.7184750733	5.71751781462428e-09\\
-17008.7976539589	2.19120690358788e-08\\
-16961.8768328446	6.17628233135847e-08\\
-16914.9560117302	5.37192353350696e-08\\
-16868.0351906158	1.66315504551796e-08\\
-16821.1143695015	4.37698929346378e-09\\
-16774.1935483871	4.10945496359856e-09\\
-16727.2727272727	1.21656068750467e-08\\
-16680.3519061584	4.70922815040199e-08\\
-16633.431085044	6.6709709644672e-08\\
-16586.5102639296	2.98295311518424e-08\\
-16539.5894428152	7.42911763430195e-09\\
-16492.6686217009	2.97384225352324e-09\\
-16445.7478005865	7.68452853010408e-09\\
-16398.8269794721	2.96411382418797e-08\\
-16351.9061583578	6.94504696371247e-08\\
-16304.9853372434	4.75187901595118e-08\\
-16211.1436950147	3.61912196127011e-09\\
-16164.2228739003	4.55837007944116e-09\\
-16117.3020527859	1.84482742237976e-08\\
-16070.3812316716	5.6156739930779e-08\\
-16023.4604105572	6.59832259500854e-08\\
-15976.5395894428	2.39041041783412e-08\\
-15929.6187683284	5.42815660825681e-09\\
-15882.6979472141	3.95503769024861e-09\\
-15835.7771260997	9.30777685952316e-09\\
-15788.8563049853	4.12012458932618e-08\\
-15741.935483871	7.392992538558e-08\\
-15695.0146627566	3.95508853514749e-08\\
-15601.1730205279	3.08977805024238e-09\\
-15554.2521994135	6.04237141339169e-09\\
-15507.3313782991	2.46421578349954e-08\\
-15460.4105571848	6.94656322159664e-08\\
-15413.4897360704	5.95974695238638e-08\\
-15366.568914956	1.8445806984027e-08\\
-15319.6480938416	5.31487976280728e-09\\
-15272.7272727273	4.53600759315367e-09\\
-15225.8064516129	1.37455728625863e-08\\
-15178.8856304985	5.31300638112411e-08\\
-15131.9648093842	7.1873091808668e-08\\
-15085.0439882698	3.64240419489675e-08\\
-15038.1231671554	7.8060731745481e-09\\
-14991.2023460411	3.8376488358571e-09\\
-14944.2815249267	7.6165580743415e-09\\
-14897.3607038123	3.40596751226429e-08\\
-14850.4398826979	7.98501423111432e-08\\
-14803.5190615836	5.19436601900925e-08\\
-14709.6774193548	4.1180578766633e-09\\
-14662.7565982405	5.81767455504117e-09\\
-14615.8357771261	1.86769691798484e-08\\
-14568.9149560117	6.58065886694649e-08\\
-14521.9941348974	7.41189268903372e-08\\
-14475.073313783	2.74713136450293e-08\\
-14428.1524926686	6.29231510552849e-09\\
-14381.2316715543	3.70899477158701e-09\\
-14334.3108504399	1.17371427237958e-08\\
-14287.3900293255	4.85033227484484e-08\\
-14240.4692082111	8.08519531394101e-08\\
-14193.5483870968	4.69464613441758e-08\\
-14146.6275659824	1.17770131258618e-08\\
-14099.706744868	3.92786018675943e-09\\
-14052.7859237537	7.2106903284573e-09\\
-14005.8651026393	2.81017053957105e-08\\
-13958.9442815249	7.77989806975849e-08\\
-13912.0234604106	7.03187062477802e-08\\
-13865.1026392962	2.13566902986816e-08\\
-13818.1818181818	5.49777860911888e-09\\
-13771.2609970674	4.51915248755835e-09\\
-13724.3401759531	1.6159609974527e-08\\
-13677.4193548387	6.04548953109271e-08\\
-13630.4985337243	8.77201048826482e-08\\
-13583.57771261	3.79110767805788e-08\\
-13536.6568914956	9.93727091180737e-09\\
-13489.7360703812	3.85101567695233e-09\\
-13442.8152492669	9.76414496353388e-09\\
-13395.8944281525	3.91982583646509e-08\\
-13348.9736070381	8.98962678095017e-08\\
-13302.0527859238	6.22697731756055e-08\\
-13255.1319648094	1.76194892029727e-08\\
-13208.211143695	4.66131398751507e-09\\
-13161.2903225806	6.91462402234334e-09\\
-13114.3695014663	2.12846157471882e-08\\
-13067.4486803519	7.8689429189867e-08\\
-13020.5278592375	8.61567155568366e-08\\
-12973.6070381232	2.98156991781079e-08\\
-12926.6862170088	9.00375302105324e-09\\
-12879.7653958944	3.47464099338566e-09\\
-12832.8445747801	1.4866585108076e-08\\
-12785.9237536657	5.18670517623186e-08\\
-12739.0029325513	1.02559513270975e-07\\
-12692.082111437	5.2136384514168e-08\\
-12645.1612903226	1.36265351317618e-08\\
-12598.2404692082	4.78304431366226e-09\\
-12551.3196480938	9.35883419258259e-09\\
-12504.3988269795	3.36820543048697e-08\\
-12457.4780058651	8.84493457660992e-08\\
-12410.5571847507	8.48512159952241e-08\\
-12363.6363636364	2.58662440843274e-08\\
-12316.715542522	6.17284104748679e-09\\
-12269.7947214076	5.69531920574659e-09\\
-12222.8739002933	1.93246142530775e-08\\
-12175.9530791789	7.153995371067e-08\\
-12129.0322580645	1.08499616607086e-07\\
-12082.1114369501	4.03376440377801e-08\\
-12035.1906158358	1.34695121618723e-08\\
-11988.2697947214	3.2659689282723e-09\\
-11941.348973607	1.37189058855145e-08\\
-11894.4281524927	4.16014142566862e-08\\
-11847.5073313783	1.12678383440928e-07\\
-11800.5865102639	7.48208869015089e-08\\
-11753.6656891496	2.04119942761367e-08\\
-11706.7448680352	6.1060362975308e-09\\
-11659.8240469208	6.76813166287461e-09\\
-11612.9032258065	2.87161566319852e-08\\
-11565.9824046921	9.50849540337011e-08\\
-11519.0615835777	9.97835167854541e-08\\
-11472.1407624633	3.82964248331909e-08\\
-11425.219941349	1.07786207078719e-08\\
-11378.2991202346	5.630540752072e-09\\
-11331.3782991202	1.62968036777413e-08\\
-11284.4574780059	6.30159336639295e-08\\
-11237.5366568915	1.24850847225958e-07\\
-11190.6158357771	6.35857755418504e-08\\
-11143.6950146628	1.8437259158053e-08\\
-11096.7741935484	4.37655254485961e-09\\
-11049.853372434	1.15852930273047e-08\\
-11002.9325513196	3.88657252650569e-08\\
-10956.0117302053	1.13200692976089e-07\\
-10909.0909090909	1.04111940335329e-07\\
-10862.1700879765	2.84074266710156e-08\\
-10815.2492668622	9.35561728132487e-09\\
-10768.3284457478	6.45188633876638e-09\\
-10721.4076246334	2.47405341318603e-08\\
-10674.4868035191	8.83463245795813e-08\\
-10627.5659824047	1.28410817130616e-07\\
-10580.6451612903	5.64290614222548e-08\\
-10533.724340176	1.42208422085054e-08\\
-10486.8035190616	5.7157086311442e-09\\
-10439.8826979472	1.50344994670452e-08\\
-10392.9618768328	5.8070171498379e-08\\
-10346.0410557185	1.35390001916383e-07\\
-10299.1202346041	9.39609730563783e-08\\
-10252.1994134897	2.53819457408481e-08\\
-10205.2785923754	7.19890368806271e-09\\
-10158.357771261	8.96598196488779e-09\\
-10111.4369501466	3.52942332809654e-08\\
-10064.5161290323	1.1734145487644e-07\\
-10017.5953079179	1.30709453330319e-07\\
-9970.67448680352	4.76179497906183e-08\\
-9923.75366568915	1.23784948898146e-08\\
-9876.83284457478	6.89145093221316e-09\\
-9829.91202346041	2.10110368226544e-08\\
-9782.99120234604	8.47286437581117e-08\\
-9736.07038123167	1.46984434416603e-07\\
-9689.1495601173	8.88371767768556e-08\\
-9642.22873900293	2.17556914107965e-08\\
-9595.30791788856	6.99531160674242e-09\\
-9548.38709677419	1.2124291761384e-08\\
-9501.46627565983	5.36110742451109e-08\\
-9454.54545454546	1.45957787163644e-07\\
-9407.62463343109	1.30498139368501e-07\\
-9360.70381231671	3.73906301549354e-08\\
-9313.78299120235	1.18104772007711e-08\\
-9266.86217008798	8.56862580009428e-09\\
-9173.02052785924	1.10882193349516e-07\\
-9126.09970674487	1.71656105930741e-07\\
-9079.1788856305	7.3850378592307e-08\\
-9032.25806451613	1.67096193144771e-08\\
-8985.33724340176	8.58751784775143e-09\\
-8938.41642228739	1.7839440526982e-08\\
-8891.49560117302	8.43105481678183e-08\\
-8844.57478005865	1.67786793821608e-07\\
-8797.65395894428	1.2493126568205e-07\\
-8750.73313782991	3.26742654755232e-08\\
-8703.81231671554	1.09491938461992e-08\\
-8656.89149560118	1.31562988357027e-08\\
-8609.97067448681	4.6340823174733e-08\\
-8563.04985337243	1.51227426545887e-07\\
-8516.12903225806	1.7755141436175e-07\\
-8469.2082111437	6.35911852136742e-08\\
-8422.28739002933	1.58066975015546e-08\\
-8375.36656891496	8.27684598124778e-09\\
-8281.52492668622	1.09231982021092e-07\\
-8234.60410557185	2.05779852504288e-07\\
-8187.68328445748	1.15622244159888e-07\\
-8140.76246334311	2.64388946811332e-08\\
-8093.84164222874	1.14509889408833e-08\\
-8046.92082111437	1.61069578958849e-08\\
-8000	7.18831210604878e-08\\
-7953.07917888563	2.00374849201041e-07\\
-7906.15835777126	1.71816891198543e-07\\
-7859.23753665689	5.70653525909103e-08\\
-7812.31671554252	1.43149049781667e-08\\
-7765.39589442815	1.16714758906804e-08\\
-7671.55425219941	1.57678655745871e-07\\
-7624.63343108504	2.32307335557649e-07\\
-7577.71260997068	1.00040677870297e-07\\
-7530.79178885631	2.63051861115643e-08\\
-7483.87096774194	1.04111060434204e-08\\
-7436.95014662757	2.6595439046279e-08\\
-7390.0293255132	1.08325349458173e-07\\
-7343.10850439883	2.444767722202e-07\\
-7296.18768328446	1.74024659419371e-07\\
-7249.26686217009	4.54994642803126e-08\\
-7202.34604105572	1.56480642387427e-08\\
-7155.42521994135	1.71447883525274e-08\\
-7108.50439882698	6.61803121691524e-08\\
-7061.58357771261	2.16201876894052e-07\\
-7014.66275659824	2.50644781374838e-07\\
-6967.74193548387	8.98621012918124e-08\\
-6920.8211143695	2.38154113903945e-08\\
-6873.90029325513	1.27668089724597e-08\\
-6826.97947214076	4.17262600248256e-08\\
-6780.05865102639	1.59906727979708e-07\\
-6733.13782991202	2.9770366212917e-07\\
-6686.21700879766	1.6346058002735e-07\\
-6639.29618768329	4.12471540135661e-08\\
-6592.37536656891	1.43935200382511e-08\\
-6545.45454545455	2.77025342492599e-08\\
-6498.53372434018	9.56958473858157e-08\\
-6451.61290322581	3.0233308030269e-07\\
-6404.69208211144	2.56113229014012e-07\\
-6357.77126099707	8.13089205823553e-08\\
-6310.8504398827	2.09619376616614e-08\\
-6263.92961876833	1.6139952808651e-08\\
-6217.00879765396	6.552899802552e-08\\
-6170.08797653959	2.43891720161373e-07\\
-6123.16715542522	3.38703581653104e-07\\
-6076.24633431085	1.52070795930442e-07\\
-6029.32551319648	4.05242171980202e-08\\
-5982.40469208211	1.56385150022363e-08\\
-5935.48387096774	4.23812364205566e-08\\
-5888.56304985337	1.63724533030608e-07\\
-5841.64222873901	3.71180888772674e-07\\
-5794.72140762463	2.71722829745159e-07\\
-5747.80058651026	7.48841669425628e-08\\
-5700.87976539589	1.90771195263387e-08\\
-5653.95894428153	2.61552386320575e-08\\
-5607.03812316716	1.04697123252699e-07\\
-5560.11730205279	3.37275275092751e-07\\
-5513.19648093842	4.04697730602102e-07\\
-5466.27565982405	1.3060141296732e-07\\
-5419.35483870968	3.8959587399495e-08\\
-5372.43401759531	2.13199548693056e-08\\
-5325.51319648094	6.35734129435394e-08\\
-5278.59237536657	2.58766460313171e-07\\
-5231.6715542522	4.79488596551454e-07\\
-5184.75073313783	2.62057530997236e-07\\
-5137.82991202346	7.02693308660746e-08\\
-5090.90909090909	2.23536666542292e-08\\
-5043.98826979472	4.396507517625e-08\\
-4997.06744868035	1.71051028774799e-07\\
-4950.14662756598	4.87089143723891e-07\\
-4903.22580645161	4.27153983931986e-07\\
-4856.30498533724	1.33770207983787e-07\\
-4809.38416422287	3.57696049184982e-08\\
-4762.46334310851	3.45755940334036e-08\\
-4715.54252199414	1.04357642345076e-07\\
-4668.62170087976	4.09147616299963e-07\\
-4621.7008797654	5.84868894954649e-07\\
-4574.78005865103	2.64170525704045e-07\\
-4527.85923753666	6.68557781738678e-08\\
-4480.93841642229	2.74419346757138e-08\\
-4434.01759530792	7.2800534420113e-08\\
-4387.09677419355	2.85541547249204e-07\\
-4340.17595307918	6.76227343203146e-07\\
-4293.25513196481	4.67207975816167e-07\\
-4199.41348973607	3.68624653521296e-08\\
-4152.4926686217	4.80329624949477e-08\\
-4105.57184750733	1.98264575237626e-07\\
-4058.65102639296	6.12320639614469e-07\\
-4011.73020527859	7.26697319859692e-07\\
-3964.80938416422	2.66806198970501e-07\\
-3917.88856304986	6.17196297653782e-08\\
-3870.96774193548	4.65233114751478e-08\\
-3824.04692082111	1.12613361434595e-07\\
-3777.12609970674	5.07276404180957e-07\\
-3730.20527859238	9.21376187305949e-07\\
-3683.28445747801	4.98763998879378e-07\\
-3589.44281524927	4.08397329850772e-08\\
-3542.5219941349	8.29621427430603e-08\\
-3495.60117302053	3.45915037119533e-07\\
-3448.68035190616	9.90623145681188e-07\\
-3401.75953079179	8.60360036623149e-07\\
-3354.83870967742	2.70883872125703e-07\\
-3307.91788856305	8.00525187388106e-08\\
-3260.99706744868	7.14308576769107e-08\\
-3214.07624633431	2.22729099787069e-07\\
-3167.15542521994	8.78167591194005e-07\\
-3120.23460410557	1.20439657229042e-06\\
-3073.3137829912	6.20231798879656e-07\\
-3026.39296187683	1.36318662024007e-07\\
-2979.47214076246	6.9817811925861e-08\\
-2932.55131964809	1.44440025488377e-07\\
-2885.63049853372	6.61712147795903e-07\\
-2838.70967741936	1.57971692261491e-06\\
-2791.78885630499	1.04351703418893e-06\\
-2697.94721407625	8.75500122244917e-08\\
-2651.02639296188	1.30895637839849e-07\\
-2604.10557184751	4.34678996484143e-07\\
-2557.18475073314	1.56649530909088e-06\\
-2510.26392961877	1.79504709402612e-06\\
-2463.3431085044	6.80605310638206e-07\\
-2416.42228739003	1.61114348198402e-07\\
-2369.50146627566	1.00384188670056e-07\\
-2322.58064516129	3.33240833111429e-07\\
-2275.65982404692	1.42309304586673e-06\\
-2228.73900293255	2.42388519006517e-06\\
-2181.81818181818	1.43708694081246e-06\\
-2134.89736070381	3.7321294956581e-07\\
-2087.97653958944	1.30834116981878e-07\\
-2041.05571847507	2.57420186722671e-07\\
-1994.13489736071	1.04671207345637e-06\\
-1947.21407624634	2.98085454347638e-06\\
-1900.29325513196	2.75701681020195e-06\\
-1853.37243401759	8.66490590493051e-07\\
-1806.45161290323	2.33846368022205e-07\\
-1759.53079178886	2.09663610697347e-07\\
-1665.68914956012	3.0891824145614e-06\\
-1618.76832844575	4.61178194249124e-06\\
-1571.84750733138	2.05808116159071e-06\\
-1524.92668621701	5.69005036337926e-07\\
-1478.00586510264	2.39923624491242e-07\\
-1431.08504398827	6.66005378224616e-07\\
-1384.1642228739	2.82910421976991e-06\\
-1337.24340175953	6.74578276554042e-06\\
-1290.32258064516	4.84034363204172e-06\\
-1243.40175953079	1.44862730448142e-06\\
-1196.48093841642	4.14748468724272e-07\\
-1149.56011730205	7.02710132915608e-07\\
-1102.63929618768	2.36523257250344e-06\\
-1055.71847507331	9.22804422469072e-06\\
-1008.79765395894	1.05941632128154e-05\\
-961.876832844573	3.88218066098419e-06\\
-914.956011730206	1.28523964515546e-06\\
-868.035190615836	5.88427548963181e-07\\
-821.114369501465	2.86482473605217e-06\\
-774.193548387098	1.10986869042613e-05\\
-727.272727272728	2.3396542373272e-05\\
-680.351906158357	1.28048966837301e-05\\
-633.431085043987	3.75513216665831e-06\\
-586.51026392962	1.55656131140366e-06\\
-539.58944281525	4.03947609222608e-06\\
-492.668621700879	1.75555355341337e-05\\
-445.747800586509	5.21406116962008e-05\\
-398.826979472142	5.54651214436025e-05\\
-351.906158357771	2.02646641849441e-05\\
-304.985337243401	6.32451119377884e-06\\
-258.064516129034	9.92479796124793e-06\\
-211.143695014664	5.68988762309767e-05\\
-164.222873900293	0.000307621250998231\\
-117.302052785923	0.000616957306531322\\
-70.3812316715557	0.000576453183126342\\
-23.4604105571852	0.00213340607671187\\
23.4604105571852	0.00512914876701211\\
70.3812316715557	0.00213340607671187\\
117.302052785923	0.000576453183126342\\
164.222873900293	0.000616957306531322\\
211.143695014664	0.000307621250998231\\
258.064516129034	5.68988762309767e-05\\
304.985337243401	9.92479796124793e-06\\
351.906158357771	6.32451119377884e-06\\
398.826979472142	2.02646641849441e-05\\
445.747800586509	5.54651214436025e-05\\
492.668621700879	5.21406116962008e-05\\
539.58944281525	1.75555355341337e-05\\
586.51026392962	4.03947609222608e-06\\
633.431085043987	1.55656131140366e-06\\
680.351906158357	3.75513216665831e-06\\
727.272727272728	1.28048966837301e-05\\
774.193548387098	2.3396542373272e-05\\
821.114369501465	1.10986869042613e-05\\
868.035190615836	2.86482473605217e-06\\
914.956011730206	5.88427548963181e-07\\
961.876832844573	1.28523964515546e-06\\
1008.79765395894	3.88218066098419e-06\\
1055.71847507331	1.05941632128154e-05\\
1102.63929618768	9.22804422469072e-06\\
1149.56011730205	2.36523257250344e-06\\
1196.48093841642	7.02710132915608e-07\\
1243.40175953079	4.14748468724272e-07\\
1290.32258064516	1.44862730448142e-06\\
1337.24340175953	4.84034363204172e-06\\
1384.1642228739	6.74578276554042e-06\\
1431.08504398827	2.82910421976991e-06\\
1478.00586510264	6.66005378224616e-07\\
1524.92668621701	2.39923624491242e-07\\
1571.84750733138	5.69005036337926e-07\\
1618.76832844575	2.05808116159071e-06\\
1665.68914956012	4.61178194249124e-06\\
1712.60997067449	3.0891824145614e-06\\
1806.45161290323	2.09663610697347e-07\\
1853.37243401759	2.33846368022205e-07\\
1900.29325513196	8.66490590493051e-07\\
1947.21407624634	2.75701681020195e-06\\
1994.13489736071	2.98085454347638e-06\\
2041.05571847507	1.04671207345637e-06\\
2087.97653958944	2.57420186722671e-07\\
2134.89736070381	1.30834116981878e-07\\
2181.81818181818	3.7321294956581e-07\\
2228.73900293255	1.43708694081246e-06\\
2275.65982404692	2.42388519006517e-06\\
2322.58064516129	1.42309304586673e-06\\
2369.50146627566	3.33240833111429e-07\\
2416.42228739003	1.00384188670056e-07\\
2463.3431085044	1.61114348198402e-07\\
2510.26392961877	6.80605310638206e-07\\
2557.18475073314	1.79504709402612e-06\\
2604.10557184751	1.56649530909088e-06\\
2651.02639296188	4.34678996484143e-07\\
2697.94721407625	1.30895637839849e-07\\
2744.86803519062	8.75500122244917e-08\\
2838.70967741936	1.04351703418893e-06\\
2885.63049853372	1.57971692261491e-06\\
2932.55131964809	6.61712147795903e-07\\
2979.47214076246	1.44440025488377e-07\\
3026.39296187683	6.9817811925861e-08\\
3073.3137829912	1.36318662024007e-07\\
3120.23460410557	6.20231798879656e-07\\
3167.15542521994	1.20439657229042e-06\\
3214.07624633431	8.78167591194005e-07\\
3260.99706744868	2.22729099787069e-07\\
3307.91788856305	7.14308576769107e-08\\
3354.83870967742	8.00525187388106e-08\\
3401.75953079179	2.70883872125703e-07\\
3448.68035190616	8.60360036623149e-07\\
3495.60117302053	9.90623145681188e-07\\
3542.5219941349	3.45915037119533e-07\\
3589.44281524927	8.29621427430603e-08\\
3636.36363636364	4.08397329850772e-08\\
3730.20527859238	4.98763998879378e-07\\
3777.12609970674	9.21376187305949e-07\\
3824.04692082111	5.07276404180957e-07\\
3870.96774193548	1.12613361434595e-07\\
3917.88856304986	4.65233114751478e-08\\
3964.80938416422	6.17196297653782e-08\\
4011.73020527859	2.66806198970501e-07\\
4058.65102639296	7.26697319859692e-07\\
4105.57184750733	6.12320639614469e-07\\
4152.4926686217	1.98264575237626e-07\\
4199.41348973607	4.80329624949477e-08\\
4246.33431085044	3.68624653521296e-08\\
4340.17595307918	4.67207975816167e-07\\
4387.09677419355	6.76227343203146e-07\\
4434.01759530792	2.85541547249204e-07\\
4480.93841642229	7.2800534420113e-08\\
4527.85923753666	2.74419346757138e-08\\
4574.78005865103	6.68557781738678e-08\\
4621.7008797654	2.64170525704045e-07\\
4668.62170087976	5.84868894954649e-07\\
4715.54252199414	4.09147616299963e-07\\
4762.46334310851	1.04357642345076e-07\\
4809.38416422287	3.45755940334036e-08\\
4856.30498533724	3.57696049184982e-08\\
4903.22580645161	1.33770207983787e-07\\
4950.14662756598	4.27153983931986e-07\\
4997.06744868035	4.87089143723891e-07\\
5043.98826979472	1.71051028774799e-07\\
5090.90909090909	4.396507517625e-08\\
5137.82991202346	2.23536666542292e-08\\
5184.75073313783	7.02693308660746e-08\\
5231.6715542522	2.62057530997236e-07\\
5278.59237536657	4.79488596551454e-07\\
5325.51319648094	2.58766460313171e-07\\
5372.43401759531	6.35734129435394e-08\\
5419.35483870968	2.13199548693056e-08\\
5466.27565982405	3.8959587399495e-08\\
5513.19648093842	1.3060141296732e-07\\
5560.11730205279	4.04697730602102e-07\\
5607.03812316716	3.37275275092751e-07\\
5653.95894428153	1.04697123252699e-07\\
5700.87976539589	2.61552386320575e-08\\
5747.80058651026	1.90771195263387e-08\\
5794.72140762463	7.48841669425628e-08\\
5841.64222873901	2.71722829745159e-07\\
5888.56304985337	3.71180888772674e-07\\
5935.48387096774	1.63724533030608e-07\\
5982.40469208211	4.23812364205566e-08\\
6029.32551319648	1.56385150022363e-08\\
6076.24633431085	4.05242171980202e-08\\
6123.16715542522	1.52070795930442e-07\\
6170.08797653959	3.38703581653104e-07\\
6217.00879765396	2.43891720161373e-07\\
6263.92961876833	6.552899802552e-08\\
6310.8504398827	1.6139952808651e-08\\
6357.77126099707	2.09619376616614e-08\\
6404.69208211144	8.13089205823553e-08\\
6451.61290322581	2.56113229014012e-07\\
6498.53372434018	3.0233308030269e-07\\
6545.45454545455	9.56958473858157e-08\\
6592.37536656891	2.77025342492599e-08\\
6639.29618768329	1.43935200382511e-08\\
6686.21700879766	4.12471540135661e-08\\
6733.13782991202	1.6346058002735e-07\\
6780.05865102639	2.9770366212917e-07\\
6826.97947214076	1.59906727979708e-07\\
6873.90029325513	4.17262600248256e-08\\
6920.8211143695	1.27668089724597e-08\\
6967.74193548387	2.38154113903945e-08\\
7014.66275659824	8.98621012918124e-08\\
7061.58357771261	2.50644781374838e-07\\
7108.50439882698	2.16201876894052e-07\\
7155.42521994135	6.61803121691524e-08\\
7202.34604105572	1.71447883525274e-08\\
7249.26686217009	1.56480642387427e-08\\
7296.18768328446	4.54994642803126e-08\\
7343.10850439883	1.74024659419371e-07\\
7390.0293255132	2.444767722202e-07\\
7436.95014662757	1.08325349458173e-07\\
7483.87096774194	2.6595439046279e-08\\
7530.79178885631	1.04111060434204e-08\\
7577.71260997068	2.63051861115643e-08\\
7624.63343108504	1.00040677870297e-07\\
7671.55425219941	2.32307335557649e-07\\
7718.47507331378	1.57678655745871e-07\\
7812.31671554252	1.16714758906804e-08\\
7859.23753665689	1.43149049781667e-08\\
7906.15835777126	5.70653525909103e-08\\
7953.07917888563	1.71816891198543e-07\\
8000	2.00374849201041e-07\\
8046.92082111437	7.18831210604878e-08\\
8093.84164222874	1.61069578958849e-08\\
8140.76246334311	1.14509889408833e-08\\
8187.68328445748	2.64388946811332e-08\\
8234.60410557185	1.15622244159888e-07\\
8281.52492668622	2.05779852504288e-07\\
8328.44574780059	1.09231982021092e-07\\
8422.28739002933	8.27684598124778e-09\\
8469.2082111437	1.58066975015546e-08\\
8516.12903225806	6.35911852136742e-08\\
8563.04985337243	1.7755141436175e-07\\
8609.97067448681	1.51227426545887e-07\\
8656.89149560118	4.6340823174733e-08\\
8703.81231671554	1.31562988357027e-08\\
8750.73313782991	1.09491938461992e-08\\
8797.65395894428	3.26742654755232e-08\\
8844.57478005865	1.2493126568205e-07\\
8891.49560117302	1.67786793821608e-07\\
8938.41642228739	8.43105481678183e-08\\
8985.33724340176	1.7839440526982e-08\\
9032.25806451613	8.58751784775143e-09\\
9079.1788856305	1.67096193144771e-08\\
9126.09970674487	7.3850378592307e-08\\
9173.02052785924	1.71656105930741e-07\\
9219.94134897361	1.10882193349516e-07\\
9313.78299120235	8.56862580009428e-09\\
9360.70381231671	1.18104772007711e-08\\
9407.62463343109	3.73906301549354e-08\\
9454.54545454546	1.30498139368501e-07\\
9501.46627565983	1.45957787163644e-07\\
9548.38709677419	5.36110742451109e-08\\
9595.30791788856	1.2124291761384e-08\\
9642.22873900293	6.99531160674242e-09\\
9689.1495601173	2.17556914107965e-08\\
9736.07038123167	8.88371767768556e-08\\
9782.99120234604	1.46984434416603e-07\\
9829.91202346041	8.47286437581117e-08\\
9876.83284457478	2.10110368226544e-08\\
9923.75366568915	6.89145093221316e-09\\
9970.67448680352	1.23784948898146e-08\\
10017.5953079179	4.76179497906183e-08\\
10064.5161290323	1.30709453330319e-07\\
10111.4369501466	1.1734145487644e-07\\
10158.357771261	3.52942332809654e-08\\
10205.2785923754	8.96598196488779e-09\\
10252.1994134897	7.19890368806271e-09\\
10299.1202346041	2.53819457408481e-08\\
10346.0410557185	9.39609730563783e-08\\
10392.9618768328	1.35390001916383e-07\\
10439.8826979472	5.8070171498379e-08\\
10486.8035190616	1.50344994670452e-08\\
10533.724340176	5.7157086311442e-09\\
10580.6451612903	1.42208422085054e-08\\
10627.5659824047	5.64290614222548e-08\\
10674.4868035191	1.28410817130616e-07\\
10721.4076246334	8.83463245795813e-08\\
10768.3284457478	2.47405341318603e-08\\
10815.2492668622	6.45188633876638e-09\\
10862.1700879765	9.35561728132487e-09\\
10909.0909090909	2.84074266710156e-08\\
10956.0117302053	1.04111940335329e-07\\
11002.9325513196	1.13200692976089e-07\\
11049.853372434	3.88657252650569e-08\\
11096.7741935484	1.15852930273047e-08\\
11143.6950146628	4.37655254485961e-09\\
11190.6158357771	1.8437259158053e-08\\
11237.5366568915	6.35857755418504e-08\\
11284.4574780059	1.24850847225958e-07\\
11331.3782991202	6.30159336639295e-08\\
11378.2991202346	1.62968036777413e-08\\
11425.219941349	5.630540752072e-09\\
11472.1407624633	1.07786207078719e-08\\
11519.0615835777	3.82964248331909e-08\\
11565.9824046921	9.97835167854541e-08\\
11612.9032258065	9.50849540337011e-08\\
11659.8240469208	2.87161566319852e-08\\
11706.7448680352	6.76813166287461e-09\\
11753.6656891496	6.1060362975308e-09\\
11800.5865102639	2.04119942761367e-08\\
11847.5073313783	7.48208869015089e-08\\
11894.4281524927	1.12678383440928e-07\\
11941.348973607	4.16014142566862e-08\\
11988.2697947214	1.37189058855145e-08\\
12035.1906158358	3.2659689282723e-09\\
12082.1114369501	1.34695121618723e-08\\
12129.0322580645	4.03376440377801e-08\\
12175.9530791789	1.08499616607086e-07\\
12222.8739002933	7.153995371067e-08\\
12269.7947214076	1.93246142530775e-08\\
12316.715542522	5.69531920574659e-09\\
12363.6363636364	6.17284104748679e-09\\
12410.5571847507	2.58662440843274e-08\\
12457.4780058651	8.48512159952241e-08\\
12504.3988269795	8.84493457660992e-08\\
12551.3196480938	3.36820543048697e-08\\
12598.2404692082	9.35883419258259e-09\\
12645.1612903226	4.78304431366226e-09\\
12692.082111437	1.36265351317618e-08\\
12739.0029325513	5.2136384514168e-08\\
12785.9237536657	1.02559513270975e-07\\
12832.8445747801	5.18670517623186e-08\\
12879.7653958944	1.4866585108076e-08\\
12926.6862170088	3.47464099338566e-09\\
12973.6070381232	9.00375302105324e-09\\
13020.5278592375	2.98156991781079e-08\\
13067.4486803519	8.61567155568366e-08\\
13114.3695014663	7.8689429189867e-08\\
13161.2903225806	2.12846157471882e-08\\
13208.211143695	6.91462402234334e-09\\
13255.1319648094	4.66131398751507e-09\\
13302.0527859238	1.76194892029727e-08\\
13348.9736070381	6.22697731756055e-08\\
13395.8944281525	8.98962678095017e-08\\
13442.8152492669	3.91982583646509e-08\\
13489.7360703812	9.76414496353388e-09\\
13536.6568914956	3.85101567695233e-09\\
13583.57771261	9.93727091180737e-09\\
13630.4985337243	3.79110767805788e-08\\
13677.4193548387	8.77201048826482e-08\\
13724.3401759531	6.04548953109271e-08\\
13771.2609970674	1.6159609974527e-08\\
13818.1818181818	4.51915248755835e-09\\
13865.1026392962	5.49777860911888e-09\\
13912.0234604106	2.13566902986816e-08\\
13958.9442815249	7.03187062477802e-08\\
14005.8651026393	7.77989806975849e-08\\
14052.7859237537	2.81017053957105e-08\\
14099.706744868	7.2106903284573e-09\\
14146.6275659824	3.92786018675943e-09\\
14193.5483870968	1.17770131258618e-08\\
14240.4692082111	4.69464613441758e-08\\
14287.3900293255	8.08519531394101e-08\\
14334.3108504399	4.85033227484484e-08\\
14381.2316715543	1.17371427237958e-08\\
14428.1524926686	3.70899477158701e-09\\
14475.073313783	6.29231510552849e-09\\
14521.9941348974	2.74713136450293e-08\\
14568.9149560117	7.41189268903372e-08\\
14615.8357771261	6.58065886694649e-08\\
14662.7565982405	1.86769691798484e-08\\
14709.6774193548	5.81767455504117e-09\\
14756.5982404692	4.1180578766633e-09\\
14850.4398826979	5.19436601900925e-08\\
14897.3607038123	7.98501423111432e-08\\
14944.2815249267	3.40596751226429e-08\\
14991.2023460411	7.6165580743415e-09\\
15038.1231671554	3.8376488358571e-09\\
15085.0439882698	7.8060731745481e-09\\
15131.9648093842	3.64240419489675e-08\\
15178.8856304985	7.1873091808668e-08\\
15225.8064516129	5.31300638112411e-08\\
15272.7272727273	1.37455728625863e-08\\
15319.6480938416	4.53600759315367e-09\\
15366.568914956	5.31487976280728e-09\\
15413.4897360704	1.8445806984027e-08\\
15460.4105571848	5.95974695238638e-08\\
15507.3313782991	6.94656322159664e-08\\
15554.2521994135	2.46421578349954e-08\\
15601.1730205279	6.04237141339169e-09\\
15648.0938416422	3.08977805024238e-09\\
15741.935483871	3.95508853514749e-08\\
15788.8563049853	7.392992538558e-08\\
15835.7771260997	4.12012458932618e-08\\
15882.6979472141	9.30777685952316e-09\\
15929.6187683284	3.95503769024861e-09\\
15976.5395894428	5.42815660825681e-09\\
16023.4604105572	2.39041041783412e-08\\
16070.3812316716	6.59832259500854e-08\\
16117.3020527859	5.6156739930779e-08\\
16164.2228739003	1.84482742237976e-08\\
16211.1436950147	4.55837007944116e-09\\
16258.064516129	3.61912196127011e-09\\
16351.9061583578	4.75187901595118e-08\\
16398.8269794721	6.94504696371247e-08\\
16445.7478005865	2.96411382418797e-08\\
16492.6686217009	7.68452853010408e-09\\
16539.5894428152	2.97384225352324e-09\\
16586.5102639296	7.42911763430195e-09\\
16633.431085044	2.98295311518424e-08\\
16680.3519061584	6.6709709644672e-08\\
16727.2727272727	4.70922815040199e-08\\
16774.1935483871	1.21656068750467e-08\\
16821.1143695015	4.10945496359856e-09\\
16868.0351906158	4.37698929346378e-09\\
16914.9560117302	1.66315504551796e-08\\
16961.8768328446	5.37192353350696e-08\\
17008.7976539589	6.17628233135847e-08\\
17055.7184750733	2.19120690358788e-08\\
17102.6392961877	5.71751781462428e-09\\
17149.5601173021	2.98305069884662e-09\\
17196.4809384164	9.55381024357878e-09\\
17243.4017595308	3.60964470944908e-08\\
17290.3225806452	6.65923415622805e-08\\
17337.2434017595	3.62316319035331e-08\\
17384.1642228739	9.01308904767838e-09\\
17431.0850439883	3.07933928062263e-09\\
17478.0058651026	5.76248727810637e-09\\
17524.926686217	1.95814833371346e-08\\
17571.8475073314	6.12153829866175e-08\\
17618.7683284457	5.13912623789616e-08\\
17665.6891495601	1.61131559971182e-08\\
17712.6099706745	4.08187825655986e-09\\
17759.5307917889	3.04866750553495e-09\\
17806.4516129032	1.2143119359815e-08\\
17853.3724340176	4.4547319307051e-08\\
17900.293255132	6.12812707912036e-08\\
17947.2140762463	2.72343762349005e-08\\
17994.1348973607	7.13643194676664e-09\\
18041.0557184751	2.68312175901679e-09\\
18087.9765395894	7.08178641222401e-09\\
18134.8973607038	2.68915781436197e-08\\
18181.8181818182	6.03262610286499e-08\\
18228.7390029326	4.37292248464953e-08\\
18275.6598240469	1.18683618630837e-08\\
18322.5806451613	2.96251171265624e-09\\
18369.5014662757	3.92992905541118e-09\\
18416.42228739	1.54300743529304e-08\\
18463.3431085044	4.90240654430953e-08\\
18510.2639296188	5.82329957814198e-08\\
18557.1847507331	1.85661246399783e-08\\
18604.1055718475	5.43577748908519e-09\\
18651.0263929619	2.87922041076996e-09\\
18697.9472140762	8.37072089261064e-09\\
18744.8680351906	3.34970399501938e-08\\
18791.788856305	6.13864820937776e-08\\
18838.7096774194	3.31771249494441e-08\\
18885.6304985337	8.74087396109374e-09\\
18932.5513196481	2.71134938618296e-09\\
18979.4721407625	5.15077964557016e-09\\
19026.3929618768	1.96390723081506e-08\\
19073.3137829912	5.51642986361981e-08\\
19120.2346041056	4.78479473805968e-08\\
19167.1554252199	1.47576623210414e-08\\
19214.0762463343	3.86342636524016e-09\\
19260.9970674487	3.59292533606328e-09\\
19307.917888563	1.05714047034758e-08\\
19354.8387096774	4.07546022079208e-08\\
19401.7595307918	5.75675234171473e-08\\
19448.6803519062	2.5660867585257e-08\\
19495.6011730205	6.35936796218229e-09\\
19542.5219941349	2.52604352584223e-09\\
19589.4428152493	6.47623248152758e-09\\
19636.3636363636	2.4856632706025e-08\\
19683.284457478	5.80565009802127e-08\\
19730.2052785924	3.96103182384384e-08\\
19824.0469208211	2.98631164668493e-09\\
19870.9677419355	3.72557595642003e-09\\
19917.8885630499	1.49957385709902e-08\\
19964.8093841642	4.5466286884904e-08\\
20011.7302052786	5.32767164483021e-08\\
20058.651026393	1.92317840035986e-08\\
20105.5718475073	4.34580072080288e-09\\
20152.4926686217	3.13798222617853e-09\\
20199.4134897361	7.33244227513031e-09\\
20246.3343108504	3.23115875615967e-08\\
20293.2551319648	5.78016251029166e-08\\
20340.1759530792	3.08320737261894e-08\\
20434.0175953079	2.38106076700938e-09\\
20480.9384164223	4.61449178592977e-09\\
20527.8592375367	1.87211842760278e-08\\
20574.780058651	5.25809008067928e-08\\
20621.7008797654	4.49852863181559e-08\\
20668.6217008798	1.38695326077387e-08\\
20715.5425219941	3.97348137690216e-09\\
20762.4633431085	3.35807679341274e-09\\
20809.3841642229	1.0114692007242e-08\\
20856.3049853372	3.89295935817746e-08\\
20903.2258064516	5.2512551747733e-08\\
20950.146627566	2.6523078908794e-08\\
20997.0674486804	5.65530334508873e-09\\
21043.9882697947	2.75708938672057e-09\\
21090.9090909091	5.42842595687423e-09\\
21137.8299120235	2.41604908774575e-08\\
21184.7507331378	5.64460386433229e-08\\
21231.6715542522	3.66140604361641e-08\\
21325.5131964809	2.87260818030064e-09\\
21372.4340175953	4.01734106882577e-09\\
21419.3548387097	1.28228118670802e-08\\
21466.275659824	4.50032432930166e-08\\
21513.1964809384	5.05406859378726e-08\\
21560.1173020528	1.86618704426435e-08\\
21607.0381231672	4.25176887380567e-09\\
21653.9589442815	2.48381899561016e-09\\
21700.8797653959	7.80251761952471e-09\\
21747.8005865103	3.20811100386197e-08\\
21794.7214076246	5.33072741998019e-08\\
21841.642228739	3.08571530283524e-08\\
21888.5630498534	7.70263256632904e-09\\
21935.4838709677	2.5506581825241e-09\\
21982.4046920821	4.63857289232878e-09\\
22029.3255131965	1.7975867310242e-08\\
22076.2463343109	4.95817318415197e-08\\
22123.1671554252	4.46819602560227e-08\\
22170.0879765396	1.35129848896611e-08\\
22217.008797654	3.45843657466441e-09\\
22263.9296187683	2.81356549578239e-09\\
22310.8504398827	9.99812925265712e-09\\
22357.7712609971	3.72297236817084e-08\\
22404.6920821114	5.38528183115835e-08\\
22451.6129032258	2.31953046216101e-08\\
22498.5337243402	6.04640707879494e-09\\
22545.4545454545	2.32309500826676e-09\\
22592.3753665689	5.83998123450221e-09\\
22639.2961876833	2.3321315882338e-08\\
22686.2170087977	5.32958240320471e-08\\
22733.137829912	3.68030586025632e-08\\
22780.0586510264	1.03643576437291e-08\\
22826.9794721408	2.72394801905653e-09\\
22873.9002932551	3.99863230483642e-09\\
22920.8211143695	1.22306774439153e-08\\
22967.7419354839	4.503518652506e-08\\
23014.6627565982	4.914913612614e-08\\
23061.5835777126	1.69458838984734e-08\\
23108.504398827	5.08638943875813e-09\\
23155.4252199413	1.94329320187427e-09\\
23202.3460410557	8.25396800803398e-09\\
23249.2668621701	2.86395558222412e-08\\
23296.1876832845	5.64421629755727e-08\\
23343.1085043988	2.85948109585483e-08\\
23390.0293255132	7.43596877844657e-09\\
23436.9501466276	2.59045700537738e-09\\
23483.8709677419	5.0155609529083e-09\\
23530.7917888563	1.79389740353176e-08\\
23577.7126099707	4.6929820779037e-08\\
23624.633431085	4.48747761915906e-08\\
23671.5542521994	1.36175432290685e-08\\
23718.4750733138	3.2300533034679e-09\\
23765.3958944282	2.94765032381341e-09\\
23812.3167155425	9.92843317004177e-09\\
23859.2375366569	3.65762502160344e-08\\
23906.1583577713	5.52796503081048e-08\\
23953.0791788856	2.04807598616413e-08\\
24000	6.7966459789558e-09\\
};
\addlegendentry{Pratique}

\addplot [color=red]
  table[row sep=crcr]{%
-24000	1.26631196965301e-36\\
-23953.0791788856	2.94843294621119e-09\\
-23906.1583577713	9.20430684265351e-09\\
-23859.2375366569	1.32244628263452e-08\\
-23812.3167155425	1.14207989811639e-08\\
-23765.3958944282	5.38323193650717e-09\\
-23718.4750733138	4.99335130363089e-10\\
-23671.5542521994	1.16819700264575e-09\\
-23624.633431085	6.85998003453254e-09\\
-23577.7126099707	1.255038148357e-08\\
-23530.7917888563	1.31764034101032e-08\\
-23483.8709677419	8.16225373893329e-09\\
-23436.9501466276	1.96999445480361e-09\\
-23390.0293255132	1.50878548329001e-10\\
-23343.1085043988	4.3912848758425e-09\\
-23296.1876832845	1.0973980823713e-08\\
-23249.2668621701	1.40578011024185e-08\\
-23202.3460410557	1.08826063181257e-08\\
-23155.4252199413	4.2618721256523e-09\\
-23108.504398827	1.12567711644758e-10\\
-23061.5835777126	2.1903706296943e-09\\
-23014.6627565982	8.71430081514548e-09\\
-22967.7419354839	1.39138212944516e-08\\
-22920.8211143695	1.31554323921928e-08\\
-22873.9002932551	7.09467046867304e-09\\
-22826.9794721408	1.1334500672449e-09\\
-22780.0586510264	6.29355507531859e-10\\
-22733.137829912	6.1050974487918e-09\\
-22686.2170087977	1.27412008425165e-08\\
-22639.2961876833	1.46408960085715e-08\\
-22592.3753665689	1.00935731499371e-08\\
-22545.4545454545	3.13909147118045e-09\\
-22498.5337243402	3.53921904939615e-12\\
-22451.6129032258	3.5505082131424e-09\\
-22404.6920821114	1.06924885881024e-08\\
-22357.7712609971	1.51016737666314e-08\\
-22310.8504398827	1.28408117773952e-08\\
-22263.9296187683	5.90390606041906e-09\\
-22217.008797654	4.83278248060095e-10\\
-22170.0879765396	1.46799548364856e-09\\
-22123.1671554252	8.06106129461077e-09\\
-22076.2463343109	1.44438023698404e-08\\
-22029.3255131965	1.49362370380492e-08\\
-21982.4046920821	9.07739000075142e-09\\
-21935.4838709677	2.08211584986131e-09\\
-21888.5630498534	2.26426456790727e-10\\
-21841.642228739	5.24434084630197e-09\\
-21794.7214076246	1.27381730534595e-08\\
-21747.8005865103	1.60583318226674e-08\\
-21700.8797653959	1.22303588679944e-08\\
-21653.9589442815	4.64672567009898e-09\\
-21607.0381231672	8.82582836955969e-11\\
-21560.1173020528	2.68958625721717e-09\\
-21513.1964809384	1.0218496921197e-08\\
-21466.275659824	1.60166114862768e-08\\
-21419.3548387097	1.49153252430842e-08\\
-21372.4340175953	7.87127480657984e-09\\
-21325.5131964809	1.16463880756861e-09\\
-21278.5923753666	8.29434876339403e-10\\
-21231.6715542522	7.25483990276086e-09\\
-21184.7507331378	1.47871751053481e-08\\
-21137.8299120235	1.67327625036502e-08\\
-21090.9090909091	1.13353279077496e-08\\
-21043.9882697947	3.39076808187827e-09\\
-20997.0674486804	1.62501053267456e-11\\
-20950.146627566	4.30539747317656e-09\\
-20903.2258064516	1.25252455180207e-08\\
-20856.3049853372	1.73937068099602e-08\\
-20809.3841642229	1.45608837462681e-08\\
-20762.4633431085	6.52624412416381e-09\\
-20715.5425219941	4.64976215478828e-10\\
-20668.6217008798	1.85253810259587e-09\\
-20621.7008797654	9.55162314954285e-09\\
-20574.780058651	1.67691963336907e-08\\
-20527.8592375367	1.70810751984579e-08\\
-20480.9384164223	1.01816649117118e-08\\
-20434.0175953079	2.21344803691576e-09\\
-20387.0967741935	3.31787369708964e-10\\
-20340.1759530792	6.31367734625314e-09\\
-20293.2551319648	1.49184955227334e-08\\
-20246.3343108504	1.85109451249227e-08\\
-20199.4134897361	1.3868941476874e-08\\
-20152.4926686217	5.10699400944169e-09\\
-20105.5718475073	6.39280456193345e-11\\
-20058.651026393	3.32529820655912e-09\\
-20011.7302052786	1.20907617712596e-08\\
-19964.8093841642	1.86098927922294e-08\\
-19917.8885630499	1.7069134654615e-08\\
-19870.9677419355	8.81094478723086e-09\\
-19824.0469208211	1.20047648353443e-09\\
-19777.1260997067	1.09358041374903e-09\\
-19730.2052785924	8.69845048952297e-09\\
-19683.284457478	1.73260551953019e-08\\
-19636.3636363636	1.93087029308977e-08\\
-19589.4428152493	1.28507306549207e-08\\
-19542.5219941349	3.69137613258697e-09\\
-19495.6011730205	4.23947470129056e-11\\
-19448.6803519062	5.26483820535347e-09\\
-19401.7595307918	1.48150247257085e-08\\
-19354.8387096774	2.02333065351424e-08\\
-19307.917888563	1.66751573579267e-08\\
-19260.9970674487	7.28095341514147e-09\\
-19214.0762463343	4.44023175579727e-10\\
-19167.1554252199	2.35199826980033e-09\\
-19120.2346041056	1.14286742805807e-08\\
-19073.3137829912	1.96680709240235e-08\\
-19026.3929618768	1.97347813139601e-08\\
-18979.4721407625	1.15343449190619e-08\\
-18885.6304985337	4.78805070896611e-10\\
-18838.7096774194	7.67403677455527e-09\\
-18791.788856305	1.76545870726849e-08\\
-18744.8680351906	2.15646203932857e-08\\
-18697.9472140762	1.58922321113116e-08\\
-18651.0263929619	5.66588926955118e-09\\
-18604.1055718475	4.05584053743759e-11\\
-18557.1847507331	4.1466021163547e-09\\
-18510.2639296188	1.4457803771731e-08\\
-18463.3431085044	2.18591810683732e-08\\
-18416.42228739	1.97474195142822e-08\\
-18369.5014662757	9.96611427772032e-09\\
-18322.5806451613	1.24224845904081e-09\\
-18275.6598240469	1.44641995133746e-09\\
-18228.7390029326	1.05400114438394e-08\\
-18181.8181818182	2.05282187311429e-08\\
-18134.8973607038	2.25331956538859e-08\\
-18087.9765395894	1.4730711249046e-08\\
-18041.0557184751	4.05617305845558e-09\\
-17994.1348973607	8.84223138747357e-11\\
-17947.2140762463	6.50380340246039e-09\\
-17900.293255132	1.77238063584767e-08\\
-17853.3724340176	2.38110865541511e-08\\
-17806.4516129032	1.93183639707894e-08\\
-17759.5307917889	8.21166538979011e-09\\
-17712.6099706745	4.1990976618359e-10\\
-17665.6891495601	3.01058284640455e-09\\
-17618.7683284457	1.3832950713401e-08\\
-17571.8475073314	2.33449601308756e-08\\
-17524.926686217	2.30758495135726e-08\\
-17478.0058651026	1.32204283445573e-08\\
-17431.0850439883	2.55786963649986e-09\\
-17384.1642228739	6.85180768683946e-10\\
-17337.2434017595	9.43471829020149e-09\\
-17290.3225806452	2.11496576900453e-08\\
-17243.4017595308	2.54355130176038e-08\\
-17196.4809384164	1.84359451551755e-08\\
-17149.5601173021	6.35665121761528e-09\\
-17102.6392961877	1.98407531714791e-11\\
-17055.7184750733	5.22599047484223e-09\\
-17008.7976539589	1.75053477763237e-08\\
-16961.8768328446	2.60062822483292e-08\\
-16914.9560117302	2.31403303283351e-08\\
-16868.0351906158	1.14127120137596e-08\\
-16821.1143695015	1.29174398062791e-09\\
-16774.1935483871	1.92483586787159e-09\\
-16727.2727272727	1.29332859474327e-08\\
-16680.3519061584	2.46443402363855e-08\\
-16633.431085044	2.66475360700914e-08\\
-16586.5102639296	1.7108125071599e-08\\
-16539.5894428152	4.50715310333232e-09\\
-16492.6686217009	1.64344320310107e-10\\
-16445.7478005865	8.13405887827855e-09\\
-16398.8269794721	2.14916791767361e-08\\
-16351.9061583578	2.84087214914391e-08\\
-16304.9853372434	2.2688954938905e-08\\
-16258.064516129	9.38218418860919e-09\\
-16211.1436950147	3.91992325798081e-10\\
-16164.2228739003	3.89496637686729e-09\\
-16117.3020527859	1.69747116042045e-08\\
-16070.3812316716	2.81043562746152e-08\\
-16023.4604105572	2.73692425401226e-08\\
-15976.5395894428	1.53654950039361e-08\\
-15929.6187683284	2.78978123823597e-09\\
-15882.6979472141	9.78155331374201e-10\\
-15835.7771260997	1.1760457977683e-08\\
-15788.8563049853	2.57085612148816e-08\\
-15741.935483871	3.04469761158933e-08\\
-15695.0146627566	2.17023852228689e-08\\
-15648.0938416422	7.22835604053321e-09\\
-15601.1730205279	4.71078873603341e-12\\
-15554.2521994135	6.67387855185638e-09\\
-15507.3313782991	2.15142862506303e-08\\
-15460.4105571848	3.14157588473143e-08\\
-15413.4897360704	2.75337080341024e-08\\
-15366.568914956	1.32642275720533e-08\\
-15319.6480938416	1.35152708337457e-09\\
-15272.7272727273	2.58584445984407e-09\\
-15225.8064516129	1.61128830946149e-08\\
-15178.8856304985	3.00554058482492e-08\\
-15131.9648093842	3.20174519259081e-08\\
-15085.0439882698	2.01835425967106e-08\\
-15038.1231671554	5.07706708334125e-09\\
-14991.2023460411	2.86187395380787e-10\\
-14944.2815249267	1.03278506209546e-08\\
-14897.3607038123	2.64866235683659e-08\\
-14850.4398826979	3.44566987958346e-08\\
-14803.5190615836	2.70892871866185e-08\\
-14756.5982404692	1.08890216966466e-08\\
-14709.6774193548	3.59454821828681e-10\\
-14662.7565982405	5.10920541171401e-09\\
-14615.8357771261	2.11791241206796e-08\\
-14568.9149560117	3.44155870463105e-08\\
-14521.9941348974	3.30222532590576e-08\\
-14475.073313783	1.81616924549671e-08\\
-14428.1524926686	3.08185523550224e-09\\
-14381.2316715543	1.40113023601688e-09\\
-14334.3108504399	1.490855453272e-08\\
-14287.3900293255	3.18053414137659e-08\\
-14240.4692082111	3.71004851666891e-08\\
-14193.5483870968	2.60042409284694e-08\\
-14146.6275659824	8.35612719061392e-09\\
-14099.706744868	3.60486056272737e-13\\
-14052.7859237537	8.66473671352555e-09\\
-14005.8651026393	2.6925481990283e-08\\
-13958.9442815249	3.86581191950147e-08\\
-13912.0234604106	3.33736339314251e-08\\
-13865.1026392962	1.56967799820872e-08\\
-13818.1818181818	1.42540146204612e-09\\
-13771.2609970674	3.52097605968305e-09\\
-13724.3401759531	2.04507363571638e-08\\
-13677.4193548387	3.73632002373573e-08\\
-13630.4985337243	3.92191627790634e-08\\
-13583.57771261	2.42717699981792e-08\\
-13536.6568914956	5.81660281311295e-09\\
-13489.7360703812	4.80580456965759e-10\\
-13442.8152492669	1.33613436545961e-08\\
-13395.8944281525	3.32955676675219e-08\\
-13348.9736070381	4.26398428464821e-08\\
-13302.0527859238	3.29989568318478e-08\\
-13255.1319648094	1.28841713297172e-08\\
-13208.211143695	3.21267974083335e-10\\
-13161.2903225806	6.82246263937947e-09\\
-13114.3695014663	2.69702319474117e-08\\
-13067.4486803519	4.30326947383211e-08\\
-13020.5278592375	4.06876285466308e-08\\
-12973.6070381232	2.19155935686072e-08\\
-12926.6862170088	3.46005634314609e-09\\
-12879.7653958944	2.02620121269168e-09\\
-12832.8445747801	1.92984026128656e-08\\
-12785.9237536657	4.0209500706313e-08\\
-12739.0029325513	4.62081123007468e-08\\
-12692.082111437	3.18462678628312e-08\\
-12645.1612903226	9.86006080095412e-09\\
-12598.2404692082	1.62533751602375e-11\\
-12551.3196480938	1.14866991111604e-08\\
-12504.3988269795	3.44623872226517e-08\\
-12457.4780058651	4.86670784691333e-08\\
-12410.5571847507	4.13883434904203e-08\\
-12363.6363636364	1.89963186330952e-08\\
-12316.715542522	1.51925143295691e-09\\
-12269.7947214076	4.88401874838794e-09\\
-12222.8739002933	2.65643512500772e-08\\
-12175.9530791789	4.75635021315855e-08\\
-12129.0322580645	4.92039880439729e-08\\
-12082.1114369501	2.98906820904132e-08\\
-12035.1906158358	6.80796322960748e-09\\
-11988.2697947214	7.9383723537795e-10\\
-11941.348973607	1.76990385259504e-08\\
-11894.4281524927	4.29009428777117e-08\\
-11847.5073313783	5.41017850601774e-08\\
-11800.5865102639	4.12167604091959e-08\\
-11753.6656891496	1.56190099729607e-08\\
-11706.7448680352	2.76166518129845e-10\\
-11659.8240469208	9.32373425221981e-09\\
-11612.9032258065	3.5235997885552e-08\\
-11565.9824046921	5.52298490911124e-08\\
-11519.0615835777	5.1465961154726e-08\\
-11472.1407624633	2.71419275685875e-08\\
-11425.219941349	3.96695071779788e-09\\
-11378.2991202346	2.97940630475905e-09\\
-11331.3782991202	2.56501582014599e-08\\
-11284.4574780059	5.22374317145674e-08\\
-11237.5366568915	5.9156195932314e-08\\
-11190.6158357771	4.00876947645693e-08\\
-11143.6950146628	1.19426260817229e-08\\
-11096.7741935484	7.03941659286391e-11\\
-11049.853372434	1.56420779646231e-08\\
-11002.9325513196	4.53787882947894e-08\\
-10956.0117302053	6.3057125076491e-08\\
-10909.0909090909	5.28342871030355e-08\\
-10862.1700879765	2.36536295575937e-08\\
-10815.2492668622	1.64266439144788e-09\\
-10768.3284457478	6.94828889460671e-09\\
-10721.4076246334	3.55389065891409e-08\\
-10674.4868035191	6.24011234321204e-08\\
-10627.5659824047	6.36356958232267e-08\\
-10580.6451612903	3.79430477440727e-08\\
-10533.724340176	8.19239173405136e-09\\
-10486.8035190616	1.31124055246936e-09\\
-10439.8826979472	2.41698692921186e-08\\
-10392.9618768328	5.7048339234264e-08\\
-10346.0410557185	7.08706571557452e-08\\
-10299.1202346041	5.31561017090129e-08\\
-10252.1994134897	1.9535304730583e-08\\
-10205.2785923754	2.22721883605852e-10\\
-10158.357771261	1.31381489186315e-08\\
-10111.4369501466	4.75777154005631e-08\\
-10064.5161290323	7.32995369759685e-08\\
-10017.5953079179	6.73339598903793e-08\\
-9970.67448680352	3.47615976476198e-08\\
-9923.75366568915	4.67683602643126e-09\\
-9876.83284457478	4.49563950704745e-09\\
-9829.91202346041	3.52835243535808e-08\\
-9782.99120234604	7.02936640422158e-08\\
-9736.07038123167	7.84729767506899e-08\\
-9689.1495601173	5.22916668486766e-08\\
-9642.22873900293	1.49686789655161e-08\\
-9595.30791788856	1.99105197745068e-10\\
-9548.38709677419	2.20681417110534e-08\\
-9501.46627565983	6.20017644480956e-08\\
-9454.54545454546	8.48195248875459e-08\\
-9407.62463343109	7.00354409845519e-08\\
-9360.70381231671	3.05721041137414e-08\\
-9313.78299120235	1.81232279334637e-09\\
-9266.86217008798	1.02352418255839e-08\\
-9219.94134897361	4.94233488273995e-08\\
-9173.02052785924	8.51627099322532e-08\\
-9126.09970674487	8.56440665971707e-08\\
-9079.1788856305	5.01224129420864e-08\\
-9032.25806451613	1.02310426635353e-08\\
-8985.33724340176	2.20173696786503e-09\\
-8938.41642228739	3.43686567840311e-08\\
-8891.49560117302	7.90837358538688e-08\\
-8844.57478005865	9.68289179120287e-08\\
-8797.65395894428	7.15180963156464e-08\\
-8750.73313782991	2.54719000942624e-08\\
-8703.81231671554	1.59789017423426e-10\\
-8656.89149560118	1.92981659174138e-08\\
-8609.97067448681	6.71223569857598e-08\\
-8563.04985337243	1.01711159478848e-07\\
-8516.12903225806	9.21410532873322e-08\\
-8469.2082111437	4.65620370905772e-08\\
-8422.28739002933	5.73029561687681e-09\\
-8375.36656891496	7.05027848601463e-09\\
-8328.44574780059	5.08279343091034e-08\\
-8281.52492668622	9.915711322473e-08\\
-8234.60410557185	1.09178701999831e-07\\
-8187.68328445748	7.15565591570492e-08\\
-8140.76246334311	1.96548341196358e-08\\
-8093.84164222874	4.81766970056708e-10\\
-8046.92082111437	3.26746481522892e-08\\
-8000	8.90518215609265e-08\\
-7953.07917888563	1.20016034114531e-07\\
-7906.15835777126	9.76968552571976e-08\\
-7859.23753665689	4.15730396261852e-08\\
-7812.31671554252	2.05986737664702e-09\\
-7765.39589442815	1.58362804980635e-08\\
-7718.47507331378	7.24662566046066e-08\\
-7671.55425219941	1.2265297657208e-07\\
-7624.63343108504	1.21705684156255e-07\\
-7577.71260997068	6.99263334112556e-08\\
-7530.79178885631	1.34555960434486e-08\\
-7483.87096774194	3.83397241093388e-09\\
-7436.95014662757	5.16827122206275e-08\\
-7390.0293255132	1.16094249944747e-07\\
-7343.10850439883	1.40196698415707e-07\\
-7296.18768328446	1.02017073731775e-07\\
-7249.26686217009	3.51932884357787e-08\\
-7202.34604105572	8.84004804483417e-11\\
-7155.42521994135	3.00573904343058e-08\\
-7108.50439882698	1.00656654750734e-07\\
-7061.58357771261	1.50158939668907e-07\\
-7014.66275659824	1.34235588918774e-07\\
-6967.74193548387	6.64104187979461e-08\\
-6920.8211143695	7.42366372834707e-09\\
-6873.90029325513	1.17176997981171e-08\\
-6826.97947214076	7.81431421444177e-08\\
-6780.05865102639	1.49463018242459e-07\\
-6733.13782991202	1.62447859198343e-07\\
-6686.21700879766	1.04774079261669e-07\\
-6639.29618768329	2.75817127597968e-08\\
-6592.37536656891	1.11237119001193e-09\\
-6545.45454545455	5.18441260234193e-08\\
-6498.53372434018	1.37327156090676e-07\\
-6451.61290322581	1.82516029394662e-07\\
-6404.69208211144	1.46586514037284e-07\\
-6357.77126099707	6.08116648238379e-08\\
-6310.8504398827	2.4528321989262e-09\\
-6263.92961876833	2.63434415265087e-08\\
-6217.00879765396	1.14679480196104e-07\\
-6170.08797653959	1.90904917520172e-07\\
-6123.16715542522	1.87093075392426e-07\\
-6076.24633431085	1.05596749692858e-07\\
-6029.32551319648	1.91019136282015e-08\\
-5982.40469208211	7.12755296178205e-09\\
-5935.48387096774	8.4358635696611e-08\\
-5888.56304985337	1.85312677501665e-07\\
-5841.64222873901	2.20983874565184e-07\\
-5794.72140762463	1.58572662511318e-07\\
-5747.80058651026	5.29786237383229e-08\\
-5700.87976539589	1.9845244546515e-11\\
-5653.95894428153	5.11078166359495e-08\\
-5607.03812316716	1.65254913334039e-07\\
-5560.11730205279	2.43057766744156e-07\\
-5513.19648093842	2.14675426070204e-07\\
-5466.27565982405	1.04053635326746e-07\\
-5419.35483870968	1.04844947438441e-08\\
-5372.43401759531	2.13366633531731e-08\\
-5325.51319648094	1.32530385388584e-07\\
-5278.59237536657	2.48996955101481e-07\\
-5231.6715542522	2.67536086021841e-07\\
-5184.75073313783	1.7000825906952e-07\\
-5137.82991202346	4.28638159751534e-08\\
-5090.90909090909	2.6492948793743e-09\\
-5043.98826979472	9.15037719720006e-08\\
-4997.06744868035	2.36185265996162e-07\\
-4950.14662756598	3.10116138417367e-07\\
-4903.22580645161	2.46119962278863e-07\\
-4856.30498533724	9.96266647152229e-08\\
-4809.38416422287	3.16251468115288e-09\\
-4762.46334310851	4.91566264838672e-08\\
-4715.54252199414	2.04495642344742e-07\\
-4668.62170087976	3.35554711821561e-07\\
-4621.7008797654	3.25420990386005e-07\\
-4574.78005865103	1.80711552506629e-07\\
-4527.85923753666	3.06630155110207e-08\\
-4480.93841642229	1.48806843873657e-08\\
-4434.01759530792	1.56974824384879e-07\\
-4387.09677419355	3.38180001199365e-07\\
-4340.17595307918	3.99155635973961e-07\\
-4293.25513196481	2.83045557471193e-07\\
-4246.33431085044	9.16731636359196e-08\\
-4199.41348973607	1.62550102743481e-11\\
-4152.4926686217	1.00367331998179e-07\\
-4105.57184750733	3.14627701904653e-07\\
-4058.65102639296	4.57539811020391e-07\\
-4011.73020527859	4.00308986401697e-07\\
-3964.80938416422	1.90508801534068e-07\\
-3917.88856304986	1.71703625145706e-08\\
-3870.96774193548	4.54484435873974e-08\\
-3824.04692082111	2.65013625162872e-07\\
-3777.12609970674	4.90809144346716e-07\\
-3730.20527859238	5.2299751595627e-07\\
-3683.28445747801	3.2841705344907e-07\\
-3636.36363636364	7.93824118448216e-08\\
-3589.44281524927	7.17944440942902e-09\\
-3542.5219941349	1.94133867823101e-07\\
-3495.60117302053	4.90536842629885e-07\\
-3448.68035190616	6.38811448915435e-07\\
-3401.75953079179	5.02875515370924e-07\\
-3354.83870967742	1.99238140808355e-07\\
-3307.91788856305	4.76357041172684e-09\\
-3260.99706744868	1.12726717102963e-07\\
-3214.07624633431	4.51223259052829e-07\\
-3167.15542521994	7.3338167381664e-07\\
-3120.23460410557	7.07115126919184e-07\\
-3073.3137829912	3.88066429072602e-07\\
-3026.39296187683	6.17813756141332e-08\\
-2979.47214076246	3.89517853391433e-08\\
-2932.55131964809	3.72230763671006e-07\\
-2885.63049853372	7.91334987146907e-07\\
-2838.70967741936	9.29896146639201e-07\\
-2791.78885630499	6.55394182394294e-07\\
-2744.86803519062	2.06753225904335e-07\\
-2697.94721407625	4.82361104702807e-10\\
-2651.02639296188	2.60315443678972e-07\\
-2604.10557184751	7.9773106365113e-07\\
-2557.18475073314	1.1555174620202e-06\\
-2510.26392961877	1.00889958515531e-06\\
-2463.3431085044	4.74798625732497e-07\\
-2416.42228739003	3.81113855783707e-08\\
-2369.50146627566	1.33427593616724e-07\\
-2322.58064516129	7.40027124576974e-07\\
-2275.65982404692	1.3639674297254e-06\\
-2228.73900293255	1.45523988441598e-06\\
-2181.81818181818	9.1175935365922e-07\\
-2134.89736070381	2.12926560923834e-07\\
-2087.97653958944	2.7487186195759e-08\\
-2041.05571847507	6.11106688031465e-07\\
-1994.13489736071	1.53101283605391e-06\\
-1947.21407624634	2.00246264798814e-06\\
-1900.29325513196	1.58385747506734e-06\\
-1853.37243401759	6.22048777086201e-07\\
-1806.45161290323	1.06130891101384e-08\\
-1759.53079178886	4.15066031411637e-07\\
-1712.60997067449	1.62783145664103e-06\\
-1665.68914956012	2.66561509024641e-06\\
-1618.76832844575	2.60066590691021e-06\\
-1571.84750733138	1.4363753733379e-06\\
-1524.92668621701	2.17652420253759e-07\\
-1478.00586510264	1.81295515958513e-07\\
-1431.08504398827	1.61944072741733e-06\\
-1384.1642228739	3.47992465334824e-06\\
-1337.24340175953	4.17414237440862e-06\\
-1290.32258064516	3.001956758107e-06\\
-1243.40175953079	9.47321563795512e-07\\
-1196.48093841642	7.20550319795157e-09\\
-1149.56011730205	1.46060101651337e-06\\
-1102.63929618768	4.53574320857232e-06\\
-1055.71847507331	6.79370893211522e-06\\
-1008.79765395894	6.16114925760834e-06\\
-961.876832844573	2.9921245759359e-06\\
-914.956011730206	2.20849286322483e-07\\
-868.035190615836	1.08844578015071e-06\\
-821.114369501465	6.09349793285367e-06\\
-774.193548387098	1.18818026050399e-05\\
-727.272727272728	1.35684751446976e-05\\
-680.351906158357	9.12667644840757e-06\\
-633.431085043987	2.22794843953179e-06\\
-586.51026392962	4.37908764301979e-07\\
-539.58944281525	9.12407332062155e-06\\
-492.668621700879	2.5462144718253e-05\\
-445.747800586509	3.81699582764901e-05\\
-398.826979472142	3.53187908643942e-05\\
-351.906158357771	1.64135158052956e-05\\
-304.985337243401	2.22461737867166e-07\\
-258.064516129034	2.0626513867124e-05\\
-211.143695014664	0.000109614171797184\\
-164.222873900293	0.00027556507604712\\
-117.302052785923	0.000490008604388786\\
-70.3812316715557	0.000692955513443509\\
-23.4604105571852	0.000816701813950125\\
23.4604105571852	0.000816701813950125\\
70.3812316715557	0.000692955513443509\\
117.302052785923	0.000490008604388786\\
164.222873900293	0.00027556507604712\\
211.143695014664	0.000109614171797184\\
258.064516129034	2.0626513867124e-05\\
304.985337243401	2.22461737867166e-07\\
351.906158357771	1.64135158052956e-05\\
398.826979472142	3.53187908643942e-05\\
445.747800586509	3.81699582764901e-05\\
492.668621700879	2.5462144718253e-05\\
539.58944281525	9.12407332062155e-06\\
586.51026392962	4.37908764301979e-07\\
633.431085043987	2.22794843953179e-06\\
680.351906158357	9.12667644840757e-06\\
727.272727272728	1.35684751446976e-05\\
774.193548387098	1.18818026050399e-05\\
821.114369501465	6.09349793285367e-06\\
868.035190615836	1.08844578015071e-06\\
914.956011730206	2.20849286322483e-07\\
961.876832844573	2.9921245759359e-06\\
1008.79765395894	6.16114925760834e-06\\
1055.71847507331	6.79370893211522e-06\\
1102.63929618768	4.53574320857232e-06\\
1149.56011730205	1.46060101651337e-06\\
1196.48093841642	7.20550319795157e-09\\
1243.40175953079	9.47321563795512e-07\\
1290.32258064516	3.001956758107e-06\\
1337.24340175953	4.17414237440862e-06\\
1384.1642228739	3.47992465334824e-06\\
1431.08504398827	1.61944072741733e-06\\
1478.00586510264	1.81295515958513e-07\\
1524.92668621701	2.17652420253759e-07\\
1571.84750733138	1.4363753733379e-06\\
1618.76832844575	2.60066590691021e-06\\
1665.68914956012	2.66561509024641e-06\\
1712.60997067449	1.62783145664103e-06\\
1759.53079178886	4.15066031411637e-07\\
1806.45161290323	1.06130891101384e-08\\
1853.37243401759	6.22048777086201e-07\\
1900.29325513196	1.58385747506734e-06\\
1947.21407624634	2.00246264798814e-06\\
1994.13489736071	1.53101283605391e-06\\
2041.05571847507	6.11106688031465e-07\\
2087.97653958944	2.7487186195759e-08\\
2134.89736070381	2.12926560923834e-07\\
2181.81818181818	9.1175935365922e-07\\
2228.73900293255	1.45523988441598e-06\\
2275.65982404692	1.3639674297254e-06\\
2322.58064516129	7.40027124576974e-07\\
2369.50146627566	1.33427593616724e-07\\
2416.42228739003	3.81113855783707e-08\\
2463.3431085044	4.74798625732497e-07\\
2510.26392961877	1.00889958515531e-06\\
2557.18475073314	1.1555174620202e-06\\
2604.10557184751	7.9773106365113e-07\\
2651.02639296188	2.60315443678972e-07\\
2697.94721407625	4.82361104702807e-10\\
2744.86803519062	2.06753225904335e-07\\
2791.78885630499	6.55394182394294e-07\\
2838.70967741936	9.29896146639201e-07\\
2885.63049853372	7.91334987146907e-07\\
2932.55131964809	3.72230763671006e-07\\
2979.47214076246	3.89517853391433e-08\\
3026.39296187683	6.17813756141332e-08\\
3073.3137829912	3.88066429072602e-07\\
3120.23460410557	7.07115126919184e-07\\
3167.15542521994	7.3338167381664e-07\\
3214.07624633431	4.51223259052829e-07\\
3260.99706744868	1.12726717102963e-07\\
3307.91788856305	4.76357041172684e-09\\
3354.83870967742	1.99238140808355e-07\\
3401.75953079179	5.02875515370924e-07\\
3448.68035190616	6.38811448915435e-07\\
3495.60117302053	4.90536842629885e-07\\
3542.5219941349	1.94133867823101e-07\\
3589.44281524927	7.17944440942902e-09\\
3636.36363636364	7.93824118448216e-08\\
3683.28445747801	3.2841705344907e-07\\
3730.20527859238	5.2299751595627e-07\\
3777.12609970674	4.90809144346716e-07\\
3824.04692082111	2.65013625162872e-07\\
3870.96774193548	4.54484435873974e-08\\
3917.88856304986	1.71703625145706e-08\\
3964.80938416422	1.90508801534068e-07\\
4011.73020527859	4.00308986401697e-07\\
4058.65102639296	4.57539811020391e-07\\
4105.57184750733	3.14627701904653e-07\\
4152.4926686217	1.00367331998179e-07\\
4199.41348973607	1.62550102743481e-11\\
4246.33431085044	9.16731636359196e-08\\
4293.25513196481	2.83045557471193e-07\\
4340.17595307918	3.99155635973961e-07\\
4387.09677419355	3.38180001199365e-07\\
4434.01759530792	1.56974824384879e-07\\
4480.93841642229	1.48806843873657e-08\\
4527.85923753666	3.06630155110207e-08\\
4574.78005865103	1.80711552506629e-07\\
4621.7008797654	3.25420990386005e-07\\
4668.62170087976	3.35554711821561e-07\\
4715.54252199414	2.04495642344742e-07\\
4762.46334310851	4.91566264838672e-08\\
4809.38416422287	3.16251468115288e-09\\
4856.30498533724	9.96266647152229e-08\\
4903.22580645161	2.46119962278863e-07\\
4950.14662756598	3.10116138417367e-07\\
4997.06744868035	2.36185265996162e-07\\
5043.98826979472	9.15037719720006e-08\\
5090.90909090909	2.6492948793743e-09\\
5137.82991202346	4.28638159751534e-08\\
5184.75073313783	1.7000825906952e-07\\
5231.6715542522	2.67536086021841e-07\\
5278.59237536657	2.48996955101481e-07\\
5325.51319648094	1.32530385388584e-07\\
5372.43401759531	2.13366633531731e-08\\
5419.35483870968	1.04844947438441e-08\\
5466.27565982405	1.04053635326746e-07\\
5513.19648093842	2.14675426070204e-07\\
5560.11730205279	2.43057766744156e-07\\
5607.03812316716	1.65254913334039e-07\\
5653.95894428153	5.11078166359495e-08\\
5700.87976539589	1.9845244546515e-11\\
5747.80058651026	5.29786237383229e-08\\
5794.72140762463	1.58572662511318e-07\\
5841.64222873901	2.20983874565184e-07\\
5888.56304985337	1.85312677501665e-07\\
5935.48387096774	8.4358635696611e-08\\
5982.40469208211	7.12755296178205e-09\\
6029.32551319648	1.91019136282015e-08\\
6076.24633431085	1.05596749692858e-07\\
6123.16715542522	1.87093075392426e-07\\
6170.08797653959	1.90904917520172e-07\\
6217.00879765396	1.14679480196104e-07\\
6263.92961876833	2.63434415265087e-08\\
6310.8504398827	2.4528321989262e-09\\
6357.77126099707	6.08116648238379e-08\\
6404.69208211144	1.46586514037284e-07\\
6451.61290322581	1.82516029394662e-07\\
6498.53372434018	1.37327156090676e-07\\
6545.45454545455	5.18441260234193e-08\\
6592.37536656891	1.11237119001193e-09\\
6639.29618768329	2.75817127597968e-08\\
6686.21700879766	1.04774079261669e-07\\
6733.13782991202	1.62447859198343e-07\\
6780.05865102639	1.49463018242459e-07\\
6826.97947214076	7.81431421444177e-08\\
6873.90029325513	1.17176997981171e-08\\
6920.8211143695	7.42366372834707e-09\\
6967.74193548387	6.64104187979461e-08\\
7014.66275659824	1.34235588918774e-07\\
7061.58357771261	1.50158939668907e-07\\
7108.50439882698	1.00656654750734e-07\\
7155.42521994135	3.00573904343058e-08\\
7202.34604105572	8.84004804483417e-11\\
7249.26686217009	3.51932884357787e-08\\
7296.18768328446	1.02017073731775e-07\\
7343.10850439883	1.40196698415707e-07\\
7390.0293255132	1.16094249944747e-07\\
7436.95014662757	5.16827122206275e-08\\
7483.87096774194	3.83397241093388e-09\\
7530.79178885631	1.34555960434486e-08\\
7577.71260997068	6.99263334112556e-08\\
7624.63343108504	1.21705684156255e-07\\
7671.55425219941	1.2265297657208e-07\\
7718.47507331378	7.24662566046066e-08\\
7765.39589442815	1.58362804980635e-08\\
7812.31671554252	2.05986737664702e-09\\
7859.23753665689	4.15730396261852e-08\\
7906.15835777126	9.76968552571976e-08\\
7953.07917888563	1.20016034114531e-07\\
8000	8.90518215609265e-08\\
8046.92082111437	3.26746481522892e-08\\
8093.84164222874	4.81766970056708e-10\\
8140.76246334311	1.96548341196358e-08\\
8187.68328445748	7.15565591570492e-08\\
8234.60410557185	1.09178701999831e-07\\
8281.52492668622	9.915711322473e-08\\
8328.44574780059	5.08279343091034e-08\\
8375.36656891496	7.05027848601463e-09\\
8422.28739002933	5.73029561687681e-09\\
8469.2082111437	4.65620370905772e-08\\
8516.12903225806	9.21410532873322e-08\\
8563.04985337243	1.01711159478848e-07\\
8609.97067448681	6.71223569857598e-08\\
8656.89149560118	1.92981659174138e-08\\
8703.81231671554	1.59789017423426e-10\\
8750.73313782991	2.54719000942624e-08\\
8797.65395894428	7.15180963156464e-08\\
8844.57478005865	9.68289179120287e-08\\
8891.49560117302	7.90837358538688e-08\\
8938.41642228739	3.43686567840311e-08\\
8985.33724340176	2.20173696786503e-09\\
9032.25806451613	1.02310426635353e-08\\
9079.1788856305	5.01224129420864e-08\\
9126.09970674487	8.56440665971707e-08\\
9173.02052785924	8.51627099322532e-08\\
9219.94134897361	4.94233488273995e-08\\
9266.86217008798	1.02352418255839e-08\\
9313.78299120235	1.81232279334637e-09\\
9360.70381231671	3.05721041137414e-08\\
9407.62463343109	7.00354409845519e-08\\
9454.54545454546	8.48195248875459e-08\\
9501.46627565983	6.20017644480956e-08\\
9548.38709677419	2.20681417110534e-08\\
9595.30791788856	1.99105197745068e-10\\
9642.22873900293	1.49686789655161e-08\\
9689.1495601173	5.22916668486766e-08\\
9736.07038123167	7.84729767506899e-08\\
9782.99120234604	7.02936640422158e-08\\
9829.91202346041	3.52835243535808e-08\\
9876.83284457478	4.49563950704745e-09\\
9923.75366568915	4.67683602643126e-09\\
9970.67448680352	3.47615976476198e-08\\
10017.5953079179	6.73339598903793e-08\\
10064.5161290323	7.32995369759685e-08\\
10111.4369501466	4.75777154005631e-08\\
10158.357771261	1.31381489186315e-08\\
10205.2785923754	2.22721883605852e-10\\
10252.1994134897	1.9535304730583e-08\\
10299.1202346041	5.31561017090129e-08\\
10346.0410557185	7.08706571557452e-08\\
10392.9618768328	5.7048339234264e-08\\
10439.8826979472	2.41698692921186e-08\\
10486.8035190616	1.31124055246936e-09\\
10533.724340176	8.19239173405136e-09\\
10580.6451612903	3.79430477440727e-08\\
10627.5659824047	6.36356958232267e-08\\
10674.4868035191	6.24011234321204e-08\\
10721.4076246334	3.55389065891409e-08\\
10768.3284457478	6.94828889460671e-09\\
10815.2492668622	1.64266439144788e-09\\
10862.1700879765	2.36536295575937e-08\\
10909.0909090909	5.28342871030355e-08\\
10956.0117302053	6.3057125076491e-08\\
11002.9325513196	4.53787882947894e-08\\
11049.853372434	1.56420779646231e-08\\
11096.7741935484	7.03941659286391e-11\\
11143.6950146628	1.19426260817229e-08\\
11190.6158357771	4.00876947645693e-08\\
11237.5366568915	5.9156195932314e-08\\
11284.4574780059	5.22374317145674e-08\\
11331.3782991202	2.56501582014599e-08\\
11378.2991202346	2.97940630475905e-09\\
11425.219941349	3.96695071779788e-09\\
11472.1407624633	2.71419275685875e-08\\
11519.0615835777	5.1465961154726e-08\\
11565.9824046921	5.52298490911124e-08\\
11612.9032258065	3.5235997885552e-08\\
11659.8240469208	9.32373425221981e-09\\
11706.7448680352	2.76166518129845e-10\\
11753.6656891496	1.56190099729607e-08\\
11800.5865102639	4.12167604091959e-08\\
11847.5073313783	5.41017850601774e-08\\
11894.4281524927	4.29009428777117e-08\\
11941.348973607	1.76990385259504e-08\\
11988.2697947214	7.9383723537795e-10\\
12035.1906158358	6.80796322960748e-09\\
12082.1114369501	2.98906820904132e-08\\
12129.0322580645	4.92039880439729e-08\\
12175.9530791789	4.75635021315855e-08\\
12222.8739002933	2.65643512500772e-08\\
12269.7947214076	4.88401874838794e-09\\
12316.715542522	1.51925143295691e-09\\
12363.6363636364	1.89963186330952e-08\\
12410.5571847507	4.13883434904203e-08\\
12457.4780058651	4.86670784691333e-08\\
12504.3988269795	3.44623872226517e-08\\
12551.3196480938	1.14866991111604e-08\\
12598.2404692082	1.62533751602375e-11\\
12645.1612903226	9.86006080095412e-09\\
12692.082111437	3.18462678628312e-08\\
12739.0029325513	4.62081123007468e-08\\
12785.9237536657	4.0209500706313e-08\\
12832.8445747801	1.92984026128656e-08\\
12879.7653958944	2.02620121269168e-09\\
12926.6862170088	3.46005634314609e-09\\
12973.6070381232	2.19155935686072e-08\\
13020.5278592375	4.06876285466308e-08\\
13067.4486803519	4.30326947383211e-08\\
13114.3695014663	2.69702319474117e-08\\
13161.2903225806	6.82246263937947e-09\\
13208.211143695	3.21267974083335e-10\\
13255.1319648094	1.28841713297172e-08\\
13302.0527859238	3.29989568318478e-08\\
13348.9736070381	4.26398428464821e-08\\
13395.8944281525	3.32955676675219e-08\\
13442.8152492669	1.33613436545961e-08\\
13489.7360703812	4.80580456965759e-10\\
13536.6568914956	5.81660281311295e-09\\
13583.57771261	2.42717699981792e-08\\
13630.4985337243	3.92191627790634e-08\\
13677.4193548387	3.73632002373573e-08\\
13724.3401759531	2.04507363571638e-08\\
13771.2609970674	3.52097605968305e-09\\
13818.1818181818	1.42540146204612e-09\\
13865.1026392962	1.56967799820872e-08\\
13912.0234604106	3.33736339314251e-08\\
13958.9442815249	3.86581191950147e-08\\
14005.8651026393	2.6925481990283e-08\\
14052.7859237537	8.66473671352555e-09\\
14099.706744868	3.60486056272737e-13\\
14146.6275659824	8.35612719061392e-09\\
14193.5483870968	2.60042409284694e-08\\
14240.4692082111	3.71004851666891e-08\\
14287.3900293255	3.18053414137659e-08\\
14334.3108504399	1.490855453272e-08\\
14381.2316715543	1.40113023601688e-09\\
14428.1524926686	3.08185523550224e-09\\
14475.073313783	1.81616924549671e-08\\
14521.9941348974	3.30222532590576e-08\\
14568.9149560117	3.44155870463105e-08\\
14615.8357771261	2.11791241206796e-08\\
14662.7565982405	5.10920541171401e-09\\
14709.6774193548	3.59454821828681e-10\\
14756.5982404692	1.08890216966466e-08\\
14803.5190615836	2.70892871866185e-08\\
14850.4398826979	3.44566987958346e-08\\
14897.3607038123	2.64866235683659e-08\\
14944.2815249267	1.03278506209546e-08\\
14991.2023460411	2.86187395380787e-10\\
15038.1231671554	5.07706708334125e-09\\
15085.0439882698	2.01835425967106e-08\\
15131.9648093842	3.20174519259081e-08\\
15178.8856304985	3.00554058482492e-08\\
15225.8064516129	1.61128830946149e-08\\
15272.7272727273	2.58584445984407e-09\\
15319.6480938416	1.35152708337457e-09\\
15366.568914956	1.32642275720533e-08\\
15413.4897360704	2.75337080341024e-08\\
15460.4105571848	3.14157588473143e-08\\
15507.3313782991	2.15142862506303e-08\\
15554.2521994135	6.67387855185638e-09\\
15601.1730205279	4.71078873603341e-12\\
15648.0938416422	7.22835604053321e-09\\
15695.0146627566	2.17023852228689e-08\\
15741.935483871	3.04469761158933e-08\\
15788.8563049853	2.57085612148816e-08\\
15835.7771260997	1.1760457977683e-08\\
15882.6979472141	9.78155331374201e-10\\
15929.6187683284	2.78978123823597e-09\\
15976.5395894428	1.53654950039361e-08\\
16023.4604105572	2.73692425401226e-08\\
16070.3812316716	2.81043562746152e-08\\
16117.3020527859	1.69747116042045e-08\\
16164.2228739003	3.89496637686729e-09\\
16211.1436950147	3.91992325798081e-10\\
16258.064516129	9.38218418860919e-09\\
16304.9853372434	2.2688954938905e-08\\
16351.9061583578	2.84087214914391e-08\\
16398.8269794721	2.14916791767361e-08\\
16445.7478005865	8.13405887827855e-09\\
16492.6686217009	1.64344320310107e-10\\
16539.5894428152	4.50715310333232e-09\\
16586.5102639296	1.7108125071599e-08\\
16633.431085044	2.66475360700914e-08\\
16680.3519061584	2.46443402363855e-08\\
16727.2727272727	1.29332859474327e-08\\
16774.1935483871	1.92483586787159e-09\\
16821.1143695015	1.29174398062791e-09\\
16868.0351906158	1.14127120137596e-08\\
16914.9560117302	2.31403303283351e-08\\
16961.8768328446	2.60062822483292e-08\\
17008.7976539589	1.75053477763237e-08\\
17055.7184750733	5.22599047484223e-09\\
17102.6392961877	1.98407531714791e-11\\
17149.5601173021	6.35665121761528e-09\\
17196.4809384164	1.84359451551755e-08\\
17243.4017595308	2.54355130176038e-08\\
17290.3225806452	2.11496576900453e-08\\
17337.2434017595	9.43471829020149e-09\\
17384.1642228739	6.85180768683946e-10\\
17431.0850439883	2.55786963649986e-09\\
17478.0058651026	1.32204283445573e-08\\
17524.926686217	2.30758495135726e-08\\
17571.8475073314	2.33449601308756e-08\\
17618.7683284457	1.3832950713401e-08\\
17665.6891495601	3.01058284640455e-09\\
17712.6099706745	4.1990976618359e-10\\
17759.5307917889	8.21166538979011e-09\\
17806.4516129032	1.93183639707894e-08\\
17853.3724340176	2.38110865541511e-08\\
17900.293255132	1.77238063584767e-08\\
17947.2140762463	6.50380340246039e-09\\
17994.1348973607	8.84223138747357e-11\\
18041.0557184751	4.05617305845558e-09\\
18087.9765395894	1.4730711249046e-08\\
18134.8973607038	2.25331956538859e-08\\
18181.8181818182	2.05282187311429e-08\\
18228.7390029326	1.05400114438394e-08\\
18275.6598240469	1.44641995133746e-09\\
18322.5806451613	1.24224845904081e-09\\
18369.5014662757	9.96611427772032e-09\\
18416.42228739	1.97474195142822e-08\\
18463.3431085044	2.18591810683732e-08\\
18510.2639296188	1.4457803771731e-08\\
18557.1847507331	4.1466021163547e-09\\
18604.1055718475	4.05584053743759e-11\\
18651.0263929619	5.66588926955118e-09\\
18697.9472140762	1.58922321113116e-08\\
18744.8680351906	2.15646203932857e-08\\
18791.788856305	1.76545870726849e-08\\
18838.7096774194	7.67403677455527e-09\\
18885.6304985337	4.78805070896611e-10\\
18979.4721407625	1.15343449190619e-08\\
19026.3929618768	1.97347813139601e-08\\
19073.3137829912	1.96680709240235e-08\\
19120.2346041056	1.14286742805807e-08\\
19167.1554252199	2.35199826980033e-09\\
19214.0762463343	4.44023175579727e-10\\
19260.9970674487	7.28095341514147e-09\\
19307.917888563	1.66751573579267e-08\\
19354.8387096774	2.02333065351424e-08\\
19401.7595307918	1.48150247257085e-08\\
19448.6803519062	5.26483820535347e-09\\
19495.6011730205	4.23947470129056e-11\\
19542.5219941349	3.69137613258697e-09\\
19589.4428152493	1.28507306549207e-08\\
19636.3636363636	1.93087029308977e-08\\
19683.284457478	1.73260551953019e-08\\
19730.2052785924	8.69845048952297e-09\\
19777.1260997067	1.09358041374903e-09\\
19824.0469208211	1.20047648353443e-09\\
19870.9677419355	8.81094478723086e-09\\
19917.8885630499	1.7069134654615e-08\\
19964.8093841642	1.86098927922294e-08\\
20011.7302052786	1.20907617712596e-08\\
20058.651026393	3.32529820655912e-09\\
20105.5718475073	6.39280456193345e-11\\
20152.4926686217	5.10699400944169e-09\\
20199.4134897361	1.3868941476874e-08\\
20246.3343108504	1.85109451249227e-08\\
20293.2551319648	1.49184955227334e-08\\
20340.1759530792	6.31367734625314e-09\\
20387.0967741935	3.31787369708964e-10\\
20434.0175953079	2.21344803691576e-09\\
20480.9384164223	1.01816649117118e-08\\
20527.8592375367	1.70810751984579e-08\\
20574.780058651	1.67691963336907e-08\\
20621.7008797654	9.55162314954285e-09\\
20668.6217008798	1.85253810259587e-09\\
20715.5425219941	4.64976215478828e-10\\
20762.4633431085	6.52624412416381e-09\\
20809.3841642229	1.45608837462681e-08\\
20856.3049853372	1.73937068099602e-08\\
20903.2258064516	1.25252455180207e-08\\
20950.146627566	4.30539747317656e-09\\
20997.0674486804	1.62501053267456e-11\\
21043.9882697947	3.39076808187827e-09\\
21090.9090909091	1.13353279077496e-08\\
21137.8299120235	1.67327625036502e-08\\
21184.7507331378	1.47871751053481e-08\\
21231.6715542522	7.25483990276086e-09\\
21278.5923753666	8.29434876339403e-10\\
21325.5131964809	1.16463880756861e-09\\
21372.4340175953	7.87127480657984e-09\\
21419.3548387097	1.49153252430842e-08\\
21466.275659824	1.60166114862768e-08\\
21513.1964809384	1.0218496921197e-08\\
21560.1173020528	2.68958625721717e-09\\
21607.0381231672	8.82582836955969e-11\\
21653.9589442815	4.64672567009898e-09\\
21700.8797653959	1.22303588679944e-08\\
21747.8005865103	1.60583318226674e-08\\
21794.7214076246	1.27381730534595e-08\\
21841.642228739	5.24434084630197e-09\\
21888.5630498534	2.26426456790727e-10\\
21935.4838709677	2.08211584986131e-09\\
21982.4046920821	9.07739000075142e-09\\
22029.3255131965	1.49362370380492e-08\\
22076.2463343109	1.44438023698404e-08\\
22123.1671554252	8.06106129461077e-09\\
22170.0879765396	1.46799548364856e-09\\
22217.008797654	4.83278248060095e-10\\
22263.9296187683	5.90390606041906e-09\\
22310.8504398827	1.28408117773952e-08\\
22357.7712609971	1.51016737666314e-08\\
22404.6920821114	1.06924885881024e-08\\
22451.6129032258	3.5505082131424e-09\\
22498.5337243402	3.53921904939615e-12\\
22545.4545454545	3.13909147118045e-09\\
22592.3753665689	1.00935731499371e-08\\
22639.2961876833	1.46408960085715e-08\\
22686.2170087977	1.27412008425165e-08\\
22733.137829912	6.1050974487918e-09\\
22780.0586510264	6.29355507531859e-10\\
22826.9794721408	1.1334500672449e-09\\
22873.9002932551	7.09467046867304e-09\\
22920.8211143695	1.31554323921928e-08\\
22967.7419354839	1.39138212944516e-08\\
23014.6627565982	8.71430081514548e-09\\
23061.5835777126	2.1903706296943e-09\\
23108.504398827	1.12567711644758e-10\\
23155.4252199413	4.2618721256523e-09\\
23202.3460410557	1.08826063181257e-08\\
23249.2668621701	1.40578011024185e-08\\
23296.1876832845	1.0973980823713e-08\\
23343.1085043988	4.3912848758425e-09\\
23390.0293255132	1.50878548329001e-10\\
23436.9501466276	1.96999445480361e-09\\
23483.8709677419	8.16225373893329e-09\\
23530.7917888563	1.31764034101032e-08\\
23577.7126099707	1.255038148357e-08\\
23624.633431085	6.85998003453254e-09\\
23671.5542521994	1.16819700264575e-09\\
23718.4750733138	4.99335130363089e-10\\
23765.3958944282	5.38323193650717e-09\\
23812.3167155425	1.14207989811639e-08\\
23859.2375366569	1.32244628263452e-08\\
23906.1583577713	9.20430684265351e-09\\
23953.0791788856	2.94843294621119e-09\\
24000	1.26631196965301e-36\\
};
\addlegendentry{Théorique}

\end{axis}
\end{tikzpicture}%

\subsection{Génération d'un signal modulé en fréquence}

Pour construire notre signal modulé en fréquence nous allons nous baser sur le signal NRZ et sur une simple sinusoïdale. Lorsque le signal NRZ vaut 0, la fréquence de la sinusoïdale sera $F_0=\SI{1180}{Hz}$ et lorsqu'il vaut 1 la sinusoïdale sera de fréquence $F_1=\SI{980}{Hz}$. Au final, notre signal modulé suit la formule suivante:
\[
\operatorname{modulé}(t)=\operatorname{NRZ}(t) \cos(2\pi F_1 t + \phi_1) + (1-\operatorname{NRZ}(t)) \cos(2 \pi F_0 t + \phi_1)
\]
$\phi_1$ et $\phi_2$ étant des déphasages tirés aléatoirement dans $[0, 2\pi]$

On obtient ainsi le signal suivant:

% This file was created by matlab2tikz.
%
%The latest updates can be retrieved from
%  http://www.mathworks.com/matlabcentral/fileexchange/22022-matlab2tikz-matlab2tikz
%where you can also make suggestions and rate matlab2tikz.
%
\definecolor{mycolor1}{rgb}{0.00000,0.44700,0.74100}%
%
\begin{tikzpicture}

\begin{axis}[%
width=4.521in,
height=3.559in,
at={(0.758in,0.488in)},
scale only axis,
unbounded coords=jump,
xmin=0,
xmax=0.09,
xlabel style={font=\color{white!15!black}},
xlabel={Temps [s]},
ymin=-0.1,
ymax=1.1,
ylabel style={font=\color{white!15!black}},
ylabel={Amplitude},
axis background/.style={fill=white},
title style={font=\bfseries},
title={NRZ modulé en fréquence}
]
\addplot [color=mycolor1, forget plot]
  table[row sep=crcr]{%
0	0.395553951938314\\
0.000104166666666683	0.924216648074986\\
0.00014583333333329	0.996566264262732\\
0.000166666666666648	0.99744012513674\\
0.000208333333333366	0.928482410384751\\
0.000270833333333331	0.664571662524404\\
0.000395833333333373	-0.198521729546531\\
nan	nan\\
0.000770833333333387	-0.160733485568556\\
0.000916666666666677	0.79547905570693\\
0.000979166666666642	0.982452113641172\\
0.001	0.999450520915747\\
0.00102083333333336	0.992651011529854\\
0.00106249999999997	0.908868652382631\\
0.00112500000000004	0.626596641695636\\
0.00124629878397653	-0.22\\
nan	nan\\
0.00162499999999999	-0.111459789877175\\
0.00177083333333339	0.824625361532427\\
0.00183333333333335	0.990510727639135\\
0.00185416666666671	0.999862391967019\\
0.00187499999999996	0.985406332572705\\
0.00191666666666668	0.887006585211621\\
0.00197916666666664	0.587071580637151\\
0.00208333333333333	-0.144365304293154\\
nan	nan\\
0.00245833333333334	-0.21472634210758\\
0.00262499999999999	0.851731754597843\\
0.00268749999999995	0.996119070803136\\
0.00270833333333331	0.997800858552645\\
0.00275000000000003	0.930414196256914\\
0.0028125	0.668474993378518\\
0.00293750000000004	-0.193387258505036\\
nan	nan\\
0.00331250000000005	-0.165899167293362\\
0.0033333333333333	0.46815966618673\\
0.00343749999999998	-0.153591160938219\\
nan	nan\\
0.00389583333333332	-0.164443411615344\\
0.00408333333333333	0.835664532066132\\
0.00414583333333329	0.980734504165017\\
0.00418750000000001	0.998203311673488\\
0.00422916666666662	0.950325244026688\\
0.00429166666666669	0.763963886572319\\
0.00439583333333338	0.226054642075217\\
0.00445833333333334	-0.156177561660336\\
nan	nan\\
0.00491666666666668	-0.161860497159411\\
0.00510416666666669	0.837099574111416\\
0.00516666666666665	0.981242555767045\\
0.00520833333333337	0.998043026505361\\
0.00524999999999998	0.949507115075908\\
0.00531250000000005	0.762271986322878\\
0.00541666666666663	0.223503644192221\\
0.0054791666666667	-0.158762891958937\\
nan	nan\\
0.00593750000000004	-0.159276473329757\\
0.00612500000000005	0.838528878769943\\
0.00618750000000001	0.981743882042463\\
0.00620833333333337	0.998010362832482\\
0.00622916666666662	0.997875900862081\\
0.00627083333333334	0.948682478309675\\
0.0063333333333333	0.760574861546594\\
0.00643749999999998	0.220951114440288\\
0.00649999999999995	-0.161347134114452\\
nan	nan\\
0.00695833333333329	-0.156691357836996\\
0.0071458333333333	0.839952436245432\\
0.00720833333333337	0.982238479555241\\
0.00722916666666662	0.998172007305905\\
0.00724999999999998	0.997701935889108\\
0.0072916666666667	0.947851339379957\\
0.00735416666666666	0.758872523875365\\
0.00745833333333334	0.218397070314143\\
0.00752083333333331	-0.163930270414783\\
nan	nan\\
0.00797916666666665	-0.154105168399207\\
0.00816666666666666	0.841370236780965\\
0.00822916666666662	0.982726344915459\\
0.00824999999999998	0.998326810420155\\
0.00827083333333334	0.997521132778778\\
0.00831250000000006	0.947013703983291\\
0.00837500000000002	0.757164984976833\\
0.0084791666666667	0.215841529318945\\
0.00854166666666667	-0.166512283155382\\
nan	nan\\
0.00900000000000001	-0.151517922741841\\
0.00918750000000002	0.842782270659109\\
0.00924999999999998	0.983207474779346\\
0.00927083333333334	0.998474771114228\\
0.0092916666666667	0.997333492770296\\
0.0093333333333333	0.946169577860737\\
0.00939583333333338	0.755452256554278\\
0.00949999999999995	0.213284508970052\\
0.00956250000000003	-0.169093154639458\\
nan	nan\\
0.0100847053772735	-0.22\\
0.01025	0.843396388749533\\
0.0103124999999999	0.99461368856252\\
0.0103333333333333	0.998718891704206\\
0.010375	0.936056298876726\\
0.0104375	0.680074600931073\\
0.0105625	-0.177952598303627\\
nan	nan\\
0.0109375	-0.181368357359167\\
0.0111041666666667	0.869068103924507\\
0.0111666666666667	0.998537167790511\\
0.0111875	0.994967596512962\\
0.0112291666666666	0.917404038327582\\
0.0112916666666667	0.642780562312883\\
0.0114157570481491	-0.22\\
nan	nan\\
0.0117916666666666	-0.132247271408124\\
0.0119375	0.812597815204905\\
0.012	0.987415261709168\\
0.0120208333333334	0.999990520819369\\
0.0120416666666666	0.988755005331231\\
0.0120833333333333	0.896482354242329\\
0.0121458333333333	0.60389644857433\\
0.01225	-0.12361060584914\\
nan	nan\\
0.0126270696241862	-0.22\\
0.0127916666666666	0.840571586990076\\
0.0128541666666667	0.99405733971821\\
0.012875	0.99897015242449\\
0.0128958333333333	0.980096486524596\\
0.0129375	0.873343001529704\\
0.0130208333333334	0.42971497178553\\
0.0131041666666667	-0.172797766043521\\
nan	nan\\
0.0134791666666667	-0.186514997793727\\
0.0136458333333334	0.866465999121708\\
0.0137083333333333	0.998240373479843\\
0.0137291666666667	0.995478586736962\\
0.0137708333333333	0.919475151610898\\
0.0138333333333334	0.646782763195726\\
0.013958120790708	-0.22\\
nan	nan\\
0.0143333333333333	-0.137435433683586\\
0.0144791666666667	0.809535026548075\\
0.0145416666666667	0.986573664108124\\
0.0145625	0.99995401523575\\
0.0145833333333333	0.989524460999497\\
0.014625	0.898790013396755\\
0.0146875	0.60806156241228\\
0.0147916666666666	-0.118413103244006\\
nan	nan\\
0.0151694342263575	-0.22\\
0.0153333333333333	0.837723740535949\\
0.0153958333333334	0.993473738289785\\
0.0154166666666666	0.99919402587342\\
0.0154375	0.981122504200115\\
0.0154791666666667	0.875881497995239\\
0.0155625	0.434436971270776\\
0.0156458333333334	-0.16763819644537\\
nan	nan\\
0.0160208333333334	-0.191656524825432\\
0.0161875	0.863840139715857\\
0.01625	0.99791621190508\\
0.0162708333333333	0.995962285412644\\
0.0163125	0.921521057018457\\
0.016375	0.650767232202384\\
0.0165	-0.216448595517777\\
nan	nan\\
0.016660845327791	-0.22\\
0.0166666666666667	0.0682747863999764\\
0.0167083333333333	-0.187125814984404\\
nan	nan\\
0.0171666666666667	-0.130784062965581\\
0.0173333333333333	0.780267755102342\\
0.0174166666666666	0.986813649814041\\
0.0174375	0.999412017699207\\
0.0174583333333334	0.995586408532836\\
0.0175	0.93918360633749\\
0.0175625	0.741565022402746\\
0.0176666666666667	0.192777194584215\\
0.0177291666666667	-0.189696920294464\\
nan	nan\\
0.0181875	-0.128188110131825\\
0.0183541666666667	0.781902492116328\\
0.0184375	0.987234017365504\\
0.0184583333333334	0.999498356606728\\
0.0184791666666667	0.99533729993391\\
0.0185208333333333	0.938281329356657\\
0.0185833333333333	0.739806120753668\\
0.0186875	0.190207649878225\\
0.01875	-0.192266725443059\\
nan	nan\\
0.0192083333333334	-0.125591278711086\\
0.019375	0.783531870058191\\
0.0194583333333334	0.987647618525553\\
0.0194791666666667	0.999577845064425\\
0.0195	0.995081369404577\\
0.0195416666666667	0.937372621500662\\
0.0196041666666666	0.738042148556292\\
0.0197083333333333	0.187636801510301\\
0.0197708333333333	-0.194835212817046\\
nan	nan\\
0.0202291666666666	-0.1229935865018\\
0.0203958333333333	0.785155877760349\\
0.0204791666666667	0.988054450459413\\
0.0205	0.999650482527495\\
0.0205208333333333	0.994818618698957\\
0.0205625	0.936457488997666\\
0.020625	0.736273117900649\\
0.0207291666666667	0.18506466710072\\
0.0207916666666667	-0.197402364812328\\
nan	nan\\
0.02125	-0.120395051308258\\
0.0214166666666666	0.786774504092043\\
0.0215	0.988454510378703\\
0.0215208333333333	0.999716268498087\\
0.0215416666666667	0.994549049617912\\
0.0215833333333333	0.935535938119905\\
0.0216458333333334	0.734499040911516\\
0.02175	0.182491264278661\\
0.0218125	-0.199968163833877\\
nan	nan\\
0.0222708333333334	-0.117795690940486\\
0.0224375	0.788387737959363\\
0.0225208333333333	0.988847795541452\\
0.0225416666666667	0.999775202525313\\
0.0225625	0.994272664009043\\
0.0226041666666666	0.934607975183576\\
0.0226666666666666	0.732719929748197\\
0.0227708333333333	0.179916610681881\\
0.0228333333333334	-0.202532592296048\\
nan	nan\\
0.0232916666666667	-0.115195523214266\\
0.0234583333333334	0.789995568305402\\
0.0235416666666667	0.98923430325213\\
0.0235625	0.999827284205246\\
0.0235833333333333	0.993989463766663\\
0.023625	0.933673606548822\\
0.0236875	0.730935796604503\\
0.0237916666666667	0.177340723956826\\
0.0238541666666666	-0.205095632622482\\
nan	nan\\
0.0243125	-0.112594565950829\\
0.0244791666666667	0.791597984110288\\
0.0245416666666667	0.963092588076733\\
0.0245833333333333	0.999872513180922\\
0.0246041666666666	0.993699450831795\\
0.0246458333333334	0.932732838619723\\
0.0247083333333333	0.729146653708721\\
0.0248125	0.174763621758305\\
0.024875	-0.207657267246414\\
nan	nan\\
0.0253333333333333	-0.109992836976844\\
0.0255	0.79319497439121\\
0.0255625	0.963793973084617\\
0.0256041666666667	0.999910889142348\\
0.025625	0.993402627192155\\
0.0256666666666666	0.931785677844174\\
0.0257291666666667	0.727352513323417\\
0.0258333333333334	0.17218532174946\\
0.0258958333333333	-0.210217478610716\\
nan	nan\\
0.0263541666666667	-0.107390354124269\\
0.0265208333333333	0.79478652820259\\
0.0265833333333333	0.964488752356533\\
0.026625	0.9999424118265\\
0.0266458333333334	0.993098994882136\\
0.0266666666666666	-0.195955165553164\\
nan	nan\\
0.0270416666666666	-0.163316886108793\\
0.0272083333333333	0.877987379191929\\
0.0272708333333334	0.999360319971255\\
0.0272916666666667	0.992964418788572\\
0.0273333333333333	0.909957455802272\\
0.0273958333333333	0.628634809165884\\
0.0275174814066129	-0.22\\
nan	nan\\
0.0278958333333333	-0.114061085951873\\
0.0280416666666666	0.823141592467542\\
0.0281041666666667	0.990147528501332\\
0.028125	0.999902395519715\\
0.0281458333333333	0.985848586291298\\
0.0281875	0.888212420514705\\
0.02825	0.589188924098282\\
0.0283541666666667	-0.141774243539554\\
nan	nan\\
0.0287291666666667	-0.21728253054883\\
0.0288958333333333	0.850357065947416\\
0.0289583333333333	0.995885232362518\\
0.0289791666666667	0.997970967643946\\
0.0290208333333334	0.931370527060059\\
0.0290833333333333	0.670419793006866\\
0.0292083333333333	-0.190818026002285\\
nan	nan\\
0.0295833333333333	-0.168480311423634\\
0.0297500000000001	0.875468972995344\\
0.0298125	0.99915937023624\\
0.0298333333333334	0.993570814203754\\
0.029875	0.912116344982188\\
0.0299375	0.632698204388695\\
0.0299791666666667	0.36730447297409\\
0.0299963823234987	-0.22\\
nan	nan\\
0.0304172435445186	-0.22\\
0.0306041666666667	0.80109816114657\\
0.0306666666666666	0.967201717525433\\
0.0307083333333333	0.999999966173902\\
0.0307291666666667	0.991816421259231\\
0.0307708333333333	0.92695421001165\\
0.0308333333333334	0.718307280637447\\
0.0309375	0.159276473329759\\
0.0309995066535586	-0.22\\
nan	nan\\
0.0314376565356314	-0.22\\
0.031625	0.802662369733119\\
0.0316875	0.967863397595014\\
0.0317291666666667	0.999997220170363\\
0.0317500000000001	0.99147877774607\\
0.0317916666666667	0.925968830014317\\
0.0318541666666666	0.716483414741508\\
0.0319791666666667	0.0290539439037264\\
0.0320199195571432	-0.22\\
nan	nan\\
0.0324583333333334	-0.218397070314138\\
0.0326458333333334	0.804221076961643\\
0.0327083333333333	0.968518444037231\\
0.0327500000000001	0.999987620297845\\
0.0327708333333333	0.991134338748386\\
0.0328125	0.924977103530283\\
0.032875	0.714654638148428\\
0.033	0.0264369586477952\\
0.0330208333333334	-0.101665706557803\\
nan	nan\\
0.0334791666666666	-0.21584152931894\\
0.0336666666666666	0.805774272148953\\
0.0337291666666667	0.969166852362478\\
0.0337708333333333	0.999971166622147\\
0.0337916666666667	0.990783106626921\\
0.0338333333333334	0.923979037356771\\
0.0338958333333333	0.712820963392518\\
0.0340208333333334	0.0238197921959664\\
0.0340416666666666	-0.104269784233497\\
nan	nan\\
0.0345	-0.213284508970068\\
0.0346875	0.807321944649624\\
0.0347499999999999	0.969808618126635\\
0.0347916666666667	0.999947859256038\\
0.0348125	0.99042508378898\\
0.0348541666666666	0.922974638334395\\
0.0349166666666667	0.710982403041538\\
0.0350416666666666	0.021202462485949\\
0.0350625	-0.106873147255766\\
nan	nan\\
0.0355208333333333	-0.210726026793025\\
0.0357083333333333	0.808864083856099\\
0.0357708333333333	0.97044373693112\\
0.0358125	0.999917698359266\\
0.0358333333333334	0.990060272688409\\
0.035875	0.921963913347193\\
0.0359375	0.709138969696777\\
0.0360625	0.0185849874566562\\
0.0360833333333334	-0.109475777781506\\
nan	nan\\
0.0365416666666667	-0.208166100323417\\
0.0366458333333334	0.418423515013602\\
0.0366666666666666	0.872064770244296\\
0.0367499999999999	0.427349546119357\\
0.0368333333333334	-0.175375783176547\\
nan	nan\\
0.0372083333333333	-0.183942307936414\\
0.037375	0.867770025322425\\
0.0374375	0.998392192069327\\
0.0374583333333334	0.995226502210465\\
0.0375	0.918442742421081\\
0.0375625000000001	0.644783872392502\\
0.0376869389532877	-0.22\\
nan	nan\\
0.0380625	-0.134841814641166\\
0.0382083333333333	0.811069200365234\\
0.0382708333333334	0.986997845295005\\
0.0382916666666667	0.999975694888283\\
0.0383125	0.989143122903466\\
0.0383541666666667	0.897639259979012\\
0.0384166666666667	0.605981082156511\\
0.0385208333333333	-0.121012269248855\\
nan	nan\\
0.0388982518784159	-0.22\\
0.0390625	0.839150539484941\\
0.039125	0.993768944594535\\
0.0391458333333333	0.999085512959075\\
0.0391666666666667	0.98061285586772\\
0.0392083333333333	0.874615247019963\\
0.0392916666666666	0.432077452233369\\
0.039375	-0.170218564573989\\
nan	nan\\
0.03975	-0.189086409298066\\
0.0399166666666667	0.865156034260081\\
0.0399791666666667	0.998081713062616\\
0.0400416666666666	0.604806272125851\\
0.0401666666666667	-0.119878618233503\\
nan	nan\\
0.040625	-0.197912302573948\\
0.0408125	0.816491411868903\\
0.040875	0.973519475798327\\
0.0409166666666667	0.999664099794608\\
0.0409375	0.988134482386714\\
0.0409791666666667	0.916815642912489\\
0.0410416666666666	0.699849151444347\\
0.0411666666666667	0.00549606044426498\\
0.0411875	-0.122477318808877\\
nan	nan\\
0.0416458333333334	-0.195345418016943\\
0.0418333333333334	0.818000130679827\\
0.0418958333333334	0.974114623027614\\
0.0419375	0.99959282369891\\
0.0419583333333333	0.98772899584544\\
0.042	0.915767109125814\\
0.0420625	0.697976746628182\\
0.0421875	0.00287809026286956\\
0.0422083333333333	-0.125075179938422\\
nan	nan\\
0.0426666666666666	-0.192777194584259\\
0.0428541666666666	0.81950324300948\\
0.0429166666666667	0.974703093784347\\
0.0429583333333333	0.999514696505922\\
0.0429791666666667	0.987316739520225\\
0.0430208333333333	0.914712298773888\\
0.0430833333333334	0.696099557957507\\
0.0432083333333333	0.000260100355308879\\
0.0432291666666667	-0.127672183816688\\
nan	nan\\
0.0433175975670845	-0.22\\
0.0433541666666667	0.985260561215792\\
0.0433750000000001	0.999847855670242\\
0.0433958333333333	0.99062777252626\\
0.0434375	0.902205281075837\\
0.0435	0.614277940769656\\
0.0436041666666667	-0.110610806973883\\
nan	nan\\
0.0439829818815969	-0.22\\
0.0441458333333333	0.833408933013911\\
0.0442083333333333	0.992547272755538\\
0.0442291666666667	0.999478471692362\\
0.04425	0.982611088432773\\
0.0442916666666666	0.879644197249586\\
0.0443541666666667	0.574283531536117\\
0.0444583333333334	-0.159890269000541\\
nan	nan\\
0.0448333333333333	-0.199358919437182\\
0.045	0.859856968380438\\
0.0450625	0.997378675568279\\
0.0450833333333334	0.996636632969654\\
0.045125	0.924542527856269\\
0.0451875	0.656710449623969\\
0.0453125	-0.208774203294104\\
nan	nan\\
0.0456875	-0.150389044090394\\
0.0458333333333333	0.801781142156464\\
0.0458958333333334	0.984351389959767\\
0.0459166666666667	0.999742817996091\\
0.0459375	0.991329369486071\\
0.0459791666666667	0.904451221129791\\
0.0460416666666666	0.618401165717133\\
0.0461458333333333	-0.105405455847521\\
nan	nan\\
0.0465416666666667	-0.101047144616917\\
0.0466875	0.830503817778625\\
0.04675	0.99189561149893\\
0.0467708333333333	0.999633851753752\\
0.0467916666666667	0.983569810057461\\
0.0468333333333333	0.882122532832384\\
0.0468958333333334	0.578562115097353\\
0.047	-0.15471947518526\\
nan	nan\\
0.0473750000000001	-0.204487046588479\\
0.0475416666666667	0.857172038839706\\
0.0476041666666667	0.996986136354783\\
0.047625	0.997052046395946\\
0.0476666666666666	0.926525169698283\\
0.0477291666666667	0.660650112102448\\
0.0478541666666666	-0.203650758014535\\
nan	nan\\
0.0482291666666667	-0.15556339725405\\
0.0483750000000001	0.798641046505575\\
0.0484375	0.983415232213154\\
0.0484583333333334	0.999610371867566\\
0.0484791666666666	0.992003788650451\\
0.0485208333333333	0.906672365196662\\
0.0485833333333333	0.622507436884313\\
0.0486875	-0.100197214977399\\
nan	nan\\
0.0490833333333334	-0.106254923763267\\
0.0492291666666667	0.827575933861648\\
0.0492916666666666	0.991216756923079\\
0.0493125	0.999761826348132\\
0.0493333333333333	0.984501566618969\\
0.0493750000000001	0.884576684580318\\
0.0494375	0.582824837085898\\
0.0495416666666667	-0.149544439657411\\
nan	nan\\
0.0499166666666667	-0.209609567624099\\
0.0499791666666667	0.249543949248361\\
0.0500208333333333	0.894428350281005\\
0.0500833333333334	0.996896892772626\\
0.0501041666666666	0.998776026326348\\
0.0501458333333333	0.953532570197896\\
0.0502083333333333	0.770679010918565\\
0.0503125	0.236242965392621\\
0.0503749999999999	-0.145825713535653\\
nan	nan\\
0.0508333333333333	-0.172185321749477\\
0.0510208333333333	0.831325080162164\\
0.0510833333333334	0.979170032450905\\
0.051125	0.998643113575651\\
0.0511666666666667	0.952740528135949\\
0.0512291666666667	0.769008124378175\\
0.0513333333333333	0.23369826972114\\
0.0513958333333333	-0.14841521913824\\
nan	nan\\
0.0518541666666666	-0.169605841601745\\
0.0520416666666667	0.832777275197035\\
0.0521041666666666	0.979698238955723\\
0.0521458333333333	0.998503356236869\\
0.0521875000000001	0.9519419560971\\
0.05225	0.767331967142207\\
0.0523541666666667	0.23115197230794\\
0.0524166666666667	-0.151003707519477\\
nan	nan\\
0.052875	-0.167025198994565\\
0.0530625	0.834223762469708\\
0.053125	0.980219730718508\\
0.0531666666666667	0.998356755267887\\
0.0532083333333333	0.951136859554672\\
0.0532708333333334	0.765650550698817\\
0.053375	0.228604090605005\\
0.0534375	-0.15359116093821\\
nan	nan\\
0.0538958333333334	-0.164443411615373\\
0.0540833333333334	0.83566453206612\\
0.0541458333333333	0.980734504165013\\
0.0541875000000001	0.998203311673491\\
0.0542291666666667	0.950325244026707\\
0.0542916666666666	0.763963886572318\\
0.0543958333333333	0.226054642075229\\
0.0544583333333334	-0.156177561660324\\
nan	nan\\
0.0549166666666667	-0.161860497159451\\
0.0551041666666666	0.837099574111398\\
0.0551666666666667	0.981242555767039\\
0.0552083333333333	0.998043026505364\\
0.05525	0.949507115075928\\
0.0553125	0.762271986322881\\
0.0554166666666667	0.223503644192247\\
0.0554791666666666	-0.158762891958918\\
nan	nan\\
0.0559375	-0.159276473329804\\
0.056125	0.838528878769925\\
0.0561875000000001	0.981743882042457\\
0.0562083333333333	0.998010362832481\\
0.0562291666666667	0.997875900862085\\
0.0562708333333334	0.948682478309679\\
0.0563333333333333	0.760574861546602\\
0.0564375	0.220951114440314\\
0.0565	-0.16134713411437\\
nan	nan\\
0.0569583333333333	-0.156691357837043\\
0.0571458333333333	0.839952436245406\\
0.0572083333333333	0.982238479555232\\
0.0572291666666667	0.998172007305903\\
0.05725	0.997701935889112\\
0.0572916666666666	0.947851339379965\\
0.0573541666666667	0.758872523875382\\
0.0574583333333334	0.218397070314182\\
0.0575208333333334	-0.163930270414743\\
nan	nan\\
0.0579791666666667	-0.154105168399261\\
0.0581666666666667	0.841370236780939\\
0.0582291666666667	0.98272634491545\\
0.05825	0.998326810420153\\
0.0582708333333334	0.997521132778783\\
0.0583125	0.947013703983302\\
0.058375	0.757164984976855\\
0.0584791666666666	0.215841529318984\\
0.0585416666666667	-0.166512283155342\\
nan	nan\\
0.0590000000000001	-0.151517922741909\\
0.0591874999999999	0.842782270659076\\
0.05925	0.983207474779335\\
0.0592708333333334	0.998474771114226\\
0.0592916666666666	0.997333492770298\\
0.0593333333333333	0.946169577860748\\
0.0593958333333333	0.755452256554299\\
0.0595	0.213284508970112\\
0.0595625	-0.169093154639397\\
nan	nan\\
0.0600208333333333	-0.14892963859768\\
0.0602083333333333	0.844188528201927\\
0.0602708333333334	0.983681865849277\\
0.0602916666666666	0.998615888374016\\
0.0603125	0.997139017149728\\
0.0603541666666667	0.94531896679785\\
0.0604166666666667	0.753734350346565\\
0.0605208333333334	0.210726026793096\\
0.0605833333333333	-0.171672867177959\\
nan	nan\\
0.0610416666666667	-0.146340333706384\\
0.0612083333333333	0.770347422506122\\
0.0612916666666666	0.984149514873853\\
0.0613125	0.998750161232322\\
0.0613333333333334	0.99693770724998\\
0.061375	0.9444618766246\\
0.0614375	0.752011278127989\\
0.0615416666666667	0.208166100323488\\
0.0616041666666667	-0.174251403089855\\
nan	nan\\
0.0620625	-0.14375002581477\\
0.0622291666666667	0.772014077382328\\
0.0623125	0.984610418647859\\
0.0623333333333334	0.998877588768853\\
0.0623541666666667	0.996729564450817\\
0.0623958333333333	0.943598313215399\\
0.0624583333333333	0.750283051708315\\
0.0625625	0.205604747106737\\
0.062625	-0.176828744702145\\
nan	nan\\
0.0630833333333334	-0.141158732676687\\
0.06325	0.77367544096049\\
0.0633125	0.954942781583555\\
0.0633253177010952	-0.22\\
nan	nan\\
0.0634791666666666	-0.186514997793706\\
0.0636458333333333	0.866465999121732\\
0.0637083333333334	0.99824037347985\\
0.0637291666666667	0.995478586736953\\
0.0637708333333333	0.919475151610878\\
0.0638333333333333	0.646782763195731\\
0.0639581207907081	-0.22\\
nan	nan\\
0.0643333333333334	-0.137435433683608\\
0.0644791666666666	0.809535026548112\\
0.0645416666666667	0.986573664108125\\
0.0645625	0.999954015235749\\
0.0645833333333333	0.989524460999502\\
0.064625	0.898790013396734\\
0.0646875	0.608061562412286\\
0.0647916666666667	-0.118413103244013\\
nan	nan\\
0.0651694342263576	-0.22\\
0.0653333333333334	0.83772374053593\\
0.0653958333333333	0.993473738289786\\
0.0654166666666667	0.99919402587342\\
0.0654375	0.981122504200111\\
0.0654791666666666	0.875881497995215\\
0.0655625	0.434436971270757\\
0.0656458333333333	-0.167638196445419\\
nan	nan\\
0.0660208333333333	-0.19165652482537\\
0.0661875	0.86384013971586\\
0.06625	0.997916211905084\\
0.0662708333333334	0.995962285412638\\
0.0663125	0.921521057018454\\
0.066375	0.650767232202422\\
0.0665	-0.216448595517825\\
nan	nan\\
0.066660845327791	-0.22\\
0.0666666666666667	0.0682747864000096\\
0.0667083333333334	-0.187125814984357\\
nan	nan\\
0.0671666666666667	-0.1307840629656\\
0.0673333333333334	0.780267755102321\\
0.0674166666666667	0.986813649814038\\
0.0674375	0.999412017699206\\
0.0674583333333333	0.995586408532839\\
0.0675	0.939183606337482\\
0.0675625	0.741565022402769\\
0.0676666666666667	0.19277719458422\\
0.0677291666666666	-0.189696920294404\\
nan	nan\\
0.0681875	-0.128188110131843\\
0.0683541666666667	0.781902492116307\\
0.0684375	0.987234017365501\\
0.0684583333333333	0.999498356606726\\
0.0684791666666666	0.995337299933914\\
0.0685208333333334	0.938281329356659\\
0.0685833333333333	0.739806120753709\\
0.0686875	0.19020764987823\\
0.06875	-0.192266725443055\\
nan	nan\\
0.0692083333333333	-0.125591278711133\\
0.069375	0.783531870058162\\
0.0694583333333333	0.98764761852555\\
0.0694791666666666	0.999577845064423\\
0.0695	0.995081369404582\\
0.0695416666666667	0.937372621500664\\
0.0696041666666667	0.738042148556334\\
0.0697083333333334	0.187636801510305\\
0.0697708333333333	-0.194835212817042\\
nan	nan\\
0.0702291666666667	-0.122993586501861\\
0.0703958333333333	0.785155877760311\\
0.0704791666666666	0.988054450459407\\
0.0705	0.999650482527493\\
0.0705208333333334	0.994818618698958\\
0.0705625	0.936457488997678\\
0.070625	0.73627311790071\\
0.0707291666666666	0.185064667100753\\
0.0707916666666667	-0.197402364812295\\
nan	nan\\
0.07125	-0.120395051308319\\
0.0714166666666667	0.786774504092022\\
0.0715	0.988454510378694\\
0.0715208333333334	0.999716268498088\\
0.0715416666666667	0.994549049617916\\
0.0715833333333333	0.935535938119917\\
0.0716458333333333	0.734499040911577\\
0.07175	0.182491264278693\\
0.0718125000000001	-0.199968163833844\\
nan	nan\\
0.0722708333333333	-0.117795690940575\\
0.0724375	0.788387737959342\\
0.0725208333333334	0.988847795541442\\
0.0725416666666667	0.999775202525314\\
0.0725625	0.994272664009047\\
0.0726041666666667	0.934607975183588\\
0.0726666666666667	0.732719929748259\\
0.0727708333333333	0.179916610681942\\
0.0728333333333333	-0.202532592295988\\
nan	nan\\
0.0732916666666666	-0.115195523214299\\
0.0734583333333333	0.789995568305381\\
0.0735416666666667	0.98923430325213\\
0.0735625	0.999827284205246\\
0.0735833333333333	0.993989463766667\\
0.0736250000000001	0.933673606548845\\
0.0736875	0.730935796604507\\
0.0737916666666667	0.177340723956831\\
0.0738541666666667	-0.205095632622422\\
nan	nan\\
0.0743125	-0.112594565950862\\
0.0744791666666667	0.791597984110251\\
0.0745416666666666	0.963092588076709\\
0.0745833333333333	0.999872513180922\\
0.0746041666666667	0.993699450831799\\
0.0746458333333333	0.932732838619745\\
0.0747083333333334	0.729146653708725\\
0.0748125000000001	0.17476362175831\\
0.074875	-0.207657267246354\\
nan	nan\\
0.0753333333333334	-0.109992836976877\\
0.0755	0.793194974391172\\
0.0755625	0.963793973084623\\
0.0756041666666667	0.999910889142348\\
0.0756250000000001	0.993402627192159\\
0.0756666666666667	0.931785677844207\\
0.0757291666666666	0.72735251332342\\
0.0758333333333333	0.172185321749493\\
0.0758958333333334	-0.210217478610628\\
nan	nan\\
0.0763541666666667	-0.107390354124303\\
0.0765208333333334	0.794786528202552\\
0.0765833333333333	0.964488752356531\\
0.0766250000000001	0.999942411826499\\
0.0766458333333333	0.993098994882143\\
0.0766875	0.930832130713964\\
0.07675	0.725553387745431\\
0.0768541666666667	0.169605841601761\\
0.0769166666666666	-0.212776249167841\\
nan	nan\\
0.077375	-0.104787135230266\\
0.0775416666666666	0.79637263463606\\
0.0776041666666667	0.965176921130524\\
0.0776458333333333	0.999967081017323\\
0.0776666666666667	0.992788555982813\\
0.0777083333333334	0.929872203764502\\
0.0777708333333333	0.723749289305761\\
0.077875	0.167025198994581\\
0.0779375	-0.215333561380467\\
nan	nan\\
0.0783958333333333	-0.102183198136937\\
0.0785625	0.7979532828207\\
0.0786250000000001	0.96585847468997\\
0.0786666666666667	0.999984896545739\\
0.0786875	0.992471312621882\\
0.0787291666666666	0.928905903575093\\
0.0787916666666667	0.721940230369501\\
0.0788958333333334	0.16444341161539\\
0.0789583333333334	-0.217889397720974\\
nan	nan\\
0.0793968306632252	-0.22\\
0.0795833333333333	0.799528461922885\\
0.0796458333333333	0.966533408363577\\
0.0796875	0.999995858289643\\
0.0797083333333334	0.992147266973701\\
0.07975	0.92793323676863\\
0.0798125	0.720126223335737\\
0.0799166666666666	0.161860497159467\\
0.0799790936413272	-0.22\\
nan	nan\\
0.0804172435445185	-0.22\\
0.0806041666666667	0.801098161146584\\
0.0806666666666667	0.967201717525425\\
0.0807083333333334	0.999999966173902\\
0.0807291666666666	0.991816421259243\\
0.0807708333333333	0.926954210011662\\
0.0808333333333333	0.718307280637471\\
0.0809375	0.15927647332982\\
0.0809995066535586	-0.22\\
nan	nan\\
0.0814376565356313	-0.22\\
0.0816250000000001	0.802662369733116\\
0.0816875	0.967863397594998\\
0.0817291666666666	0.999997220170363\\
0.08175	0.991478777746078\\
0.0817916666666667	0.92596883001433\\
0.0818541666666667	0.716483414741531\\
0.0819791666666667	0.0290539439038165\\
0.0820199195571433	-0.22\\
nan	nan\\
0.0824583333333333	-0.218397070314198\\
0.0826458333333333	0.80422107696164\\
0.0827083333333334	0.968518444037216\\
0.08275	0.999987620297845\\
0.0827708333333333	0.991134338748387\\
0.0828125	0.924977103530317\\
0.082875	0.714654638148451\\
0.083	0.0264369586478285\\
0.0830208333333333	-0.101665706557826\\
};
\end{axis}
\end{tikzpicture}%

On calcule ensuite la densité spectrale de puissance de ce signal NRZ en utilisant la fonction \verb|pwelch| de Matlab utilisant un périodogramme de Welch. 
On obtient ceci:

% This file was created by matlab2tikz.
%
%The latest updates can be retrieved from
%  http://www.mathworks.com/matlabcentral/fileexchange/22022-matlab2tikz-matlab2tikz
%where you can also make suggestions and rate matlab2tikz.
%
\definecolor{mycolor1}{rgb}{0.00000,0.44700,0.74100}%
\definecolor{mycolor2}{rgb}{0.85000,0.32500,0.09800}%
%
\begin{tikzpicture}

\begin{axis}[%
width=4.521in,
height=3.548in,
at={(0.758in,0.499in)},
scale only axis,
xmin=-25000,
xmax=25000,
xlabel style={font=\color{white!15!black}},
xlabel={Fréquence [Hz]},
ymode=log,
ymin=1e-10,
ymax=1134648.55313706,
yminorticks=true,
ylabel style={font=\color{white!15!black}},
ylabel={Amplitude},
axis background/.style={fill=white},
title style={font=\bfseries},
title={Densité spectrale de puissance},
legend style={legend cell align=left, align=left, draw=white!15!black}
]
\addplot [color=mycolor1]
  table[row sep=crcr]{%
-24000	0.204533895934363\\
-23987.9969992498	1.80246216990471\\
-23975.9939984996	4.55454749635279\\
-23963.9909977494	0.468876972909134\\
-23951.9879969993	2.28874008361329\\
-23939.9849962491	8.86539502362064\\
-23927.9819954989	0.697667228854273\\
-23915.9789947487	5.22407883693436\\
-23903.9759939985	7.10566788994202\\
-23891.9729932483	0.538085516395813\\
-23879.9699924981	4.5600416984223\\
-23867.9669917479	0.909819571553502\\
-23855.9639909978	1.52142742968465\\
-23843.9609902476	1.52029313424053\\
-23831.9579894974	0.910518365361075\\
-23819.9549887472	4.56706319039352\\
-23807.951987997	0.539475521527492\\
-23795.9489872468	7.10085270872724\\
-23783.9459864966	5.23938130109819\\
-23771.9429857464	0.692309595258901\\
-23759.9399849963	8.85812786993753\\
-23747.9369842461	2.28410500618647\\
-23735.9339834959	0.470359000629063\\
-23723.9309827457	4.54911704000567\\
-23711.9279819955	1.79921179298907\\
-23699.9249812453	0.204613672915126\\
-23687.9219804951	1.8071230601318\\
-23675.9189797449	4.56354100071574\\
-23663.9159789948	0.467767901362945\\
-23651.9129782446	2.29516503302075\\
-23639.9099774944	8.87957487109989\\
-23627.9069767442	0.703591570138942\\
-23615.903975994	5.21287143434495\\
-23603.9009752438	7.11603695737465\\
-23591.8979744936	0.537116609921279\\
-23579.8949737434	4.55658628719235\\
-23567.8919729932	0.909836436461999\\
-23555.8889722431	1.52376689867349\\
-23543.8859714929	1.52036048865927\\
-23531.8829707427	0.911935013161448\\
-23519.8799699925	4.57767263323894\\
-23507.8769692423	0.541290957608532\\
-23495.8739684921	7.10157648584391\\
-23483.8709677419	5.25882655066865\\
-23471.8679669918	0.687502056513389\\
-23459.8649662416	8.85775089843864\\
-23447.8619654914	2.28124541309189\\
-23435.8589647412	0.472218617960026\\
-23423.855963991	4.54723283848289\\
-23411.8529632408	1.79736185948528\\
-23399.8499624906	0.204853254319888\\
-23387.8469617404	1.8132089825977\\
-23375.8439609902	4.57612557966219\\
-23363.8409602401	0.467028356796072\\
-23351.8379594899	2.30339988230193\\
-23339.8349587397	8.90071161335376\\
-23327.8319579895	0.710101115292892\\
-23315.8289572393	5.20572431842944\\
-23303.8259564891	7.13199232677951\\
-23291.8229557389	0.5365657981727\\
-23279.8199549888	4.55668632413186\\
-23267.8169542386	0.910569039128882\\
-23255.8139534884	1.52731893376704\\
-23243.8109527382	1.52162981488226\\
-23231.807951988	0.914073999305256\\
-23219.8049512378	4.59190328007797\\
-23207.8019504876	0.543537514100889\\
-23195.7989497374	7.10784168265674\\
-23183.7959489872	5.28247557602143\\
-23171.7929482371	0.683229773369541\\
-23159.7899474869	8.86426313831166\\
-23147.7869467367	2.28015246172863\\
-23135.7839459865	0.474461693539366\\
-23123.7809452363	4.5488892215142\\
-23111.7779444861	1.79690668201585\\
-23099.7749437359	0.205253393193517\\
-23087.7719429857	1.82073900174782\\
-23075.7689422356	4.59234067956093\\
-23063.7659414854	0.466656091698789\\
-23051.7629407352	2.31347043285197\\
-23039.759939985	8.92887143466661\\
-23027.7569392348	0.717216367122924\\
-23015.7539384846	5.20261543337145\\
-23003.7509377344	7.15358411826059\\
-22991.7479369842	0.536431385733559\\
-22979.7449362341	4.56034234505274\\
-22967.7419354839	0.912019733656818\\
-22955.7389347337	1.53209484653869\\
-22943.7359339835	1.52410528352873\\
-22931.7329332333	0.916942127650869\\
-22919.7299324831	4.60979999002062\\
-22907.7269317329	0.54622227577468\\
-22895.7239309827	7.11966816797817\\
-22883.7209302326	5.31040304342957\\
-22871.7179294824	0.679479582750625\\
-22859.7149287322	8.87768515506912\\
-22847.711927982	2.28082279641397\\
-22835.7089272318	0.477095370885502\\
-22823.7059264816	4.55409160574676\\
-22811.7029257314	1.79784491863389\\
-22799.6999249812	0.205815350241107\\
-22787.6969242311	1.82973685511217\\
-22775.6939234809	4.61223743089765\\
-22763.6909227307	0.466650025797165\\
-22751.6879219805	2.32540843242592\\
-22739.6849212303	8.96414294575418\\
-22727.6819204801	0.724959973353701\\
-22715.6789197299	5.2035352974747\\
-22703.6759189797	7.18088049306083\\
-22691.6729182296	0.536712974616118\\
-22679.6699174794	4.56756605930105\\
-22667.6669167292	0.914193165509036\\
-22655.663915979	1.5381099449351\\
-22643.6609152288	1.52779494364335\\
-22631.6579144786	0.920548568070621\\
-22619.6549137284	4.63141953088617\\
-22607.6519129782	0.549353770556082\\
-22595.6489122281	7.13709335334251\\
-22583.6459114779	5.3426978000989\\
-22571.6429107277	0.676239911580905\\
-22559.6399099775	8.89805918953195\\
-22547.6369092273	2.28325852647781\\
-22535.6339084771	0.480128117270493\\
-22523.6309077269	4.56285653316172\\
-22511.6279069767	1.80017956457287\\
-22499.6249062266	0.206540902337946\\
-22487.6219054764	1.84023111379881\\
-22475.6189047262	4.63587899383571\\
-22463.615903976	0.467010239611327\\
-22451.6129032258	2.33925178945991\\
-22439.6099024756	9.00663778084349\\
-22427.6069017254	0.733356881573263\\
-22415.6039009752	5.20848694229355\\
-22403.6009002251	7.21396811495811\\
-22391.5978994749	0.537411461809897\\
-22379.5948987247	4.57838043087074\\
-22367.5918979745	0.917096303563125\\
-22355.5888972243	1.54538363897074\\
-22343.5858964741	1.53271077921393\\
-22331.5828957239	0.924904919524819\\
-22319.5798949737	4.65683096597142\\
-22307.5768942236	0.552942027725126\\
-22295.5738934734	7.16017244688154\\
-22283.5708927232	5.37946348168915\\
-22271.567891973	0.673500700850453\\
-22259.5648912228	8.92544944273526\\
-22247.5618904726	2.28746724050437\\
-22235.5588897224	0.483569782432542\\
-22223.5558889722	4.57521178406601\\
-22211.5528882221	1.80391797285953\\
-22199.5498874719	0.207432354558275\\
-22187.5468867217	1.85225537712819\\
-22175.5438859715	4.66334098855466\\
-22163.5408852213	0.46773797587298\\
-22151.5378844711	2.35504483175573\\
-22139.5348837209	9.05649135547653\\
-22127.5318829707	0.742434512517674\\
-22115.5288822206	5.21748593395345\\
-22103.5258814704	7.25295274111143\\
-22091.5228807202	0.538529045430468\\
-22079.51987997	4.5928198351394\\
-22067.5168792198	0.920738488178679\\
-22055.5138784696	1.55393957585676\\
-22043.5108777194	1.53886879331985\\
-22031.5078769692	0.930025290517272\\
-22019.5048762191	4.68611612869519\\
-22007.5018754689	0.556998647106487\\
-21995.4988747187	7.18897882867053\\
-21983.4958739685	5.4208192291446\\
-21971.4928732183	0.671253339100615\\
-21959.4898724681	8.95994250915033\\
-21947.4868717179	2.29346205671972\\
-21935.4838709677	0.487431667719346\\
-21923.4808702176	4.59119656599723\\
-21911.4778694674	1.80907190384604\\
-21899.4748687172	0.208492555938663\\
-21887.471867967	1.86584850441439\\
-21875.4688672168	4.69471201615124\\
-21863.4658664666	0.46883564891242\\
-21851.4628657164	2.372838612052\\
-21839.4598649662	9.11386379438407\\
-21827.4568642161	0.752222953992032\\
-21815.4538634659	5.23056047820717\\
-21803.4508627157	7.29795994950883\\
-21791.4478619655	0.540069239516798\\
-21779.4448612153	4.61093029357894\\
-21767.4418604651	0.925131496039956\\
-21755.4388597149	1.56380580635476\\
-21743.4358589647	1.54628912095651\\
-21731.4328582146	0.935926397870953\\
-21719.4298574644	4.71937019140586\\
-21707.4268567142	0.561536880045163\\
-21695.423855964	7.22360455228986\\
-21683.4208552138	5.46690052310475\\
-21671.4178544636	0.669490604535826\\
-21659.4148537134	9.0016479641731\\
-21647.4118529632	2.30126171030915\\
-21635.4088522131	0.491726606607471\\
-21623.4058514629	4.61086178124345\\
-21611.4028507127	1.8156576045868\\
-21599.3998499625	0.209724919048419\\
-21587.3968492123	1.88105488621274\\
-21575.3938484621	4.73009427576077\\
-21563.3908477119	0.470306862106646\\
-21551.3878469617	2.3926912646057\\
-21539.3848462116	9.17894104100378\\
-21527.3818454614	0.762755177694261\\
-21515.3788447112	5.24775160930034\\
-21503.375843961	7.34913601231912\\
-21491.3728432108	0.542036897721715\\
-21479.3698424606	4.6327697896592\\
-21467.3668417104	0.930289622457675\\
-21455.3638409602	1.57501498430189\\
-21443.3608402101	1.5549961717302\\
-21431.3578394599	0.942627685010802\\
-21419.3548387097	4.75670233460506\\
-21407.3518379595	0.566571723264196\\
-21395.3488372093	7.26416097834436\\
-21383.3458364591	5.51786014677483\\
-21371.3428357089	0.668206615118309\\
-21359.3398349587	9.05069911219917\\
-21347.3368342086	2.31089067857565\\
-21335.3338334584	0.496469057557959\\
-21323.3308327082	4.6342703754727\\
-21311.327831958	1.82369591850424\\
-21299.3248312078	0.211133443710906\\
-21287.3218304576	1.89792475840672\\
-21275.3188297074	4.76960428546071\\
-21263.3158289572	0.472156433568832\\
-21251.3128282071	2.41466841680092\\
-21239.3098274569	9.25193615986741\\
-21227.3068267067	0.774067281694585\\
-21215.3038259565	5.2691134659376\\
-21203.3008252063	7.40664892492858\\
-21191.2978244561	0.544438246087061\\
-21179.2948237059	4.65840866891492\\
-21167.2918229557	0.936229782108346\\
-21155.2888222056	1.58760460179161\\
-21143.2858214554	1.56501880451338\\
-21131.2828207052	0.950151461266578\\
-21119.279819955	4.79823652558167\\
-21107.2768192048	0.572120026653276\\
-21095.2738184546	7.31077954737495\\
-21083.2708177044	5.57386928891357\\
-21071.2678169542	0.667396786051688\\
-21059.2648162041	9.1072539042477\\
-21047.2618154539	2.32237934522041\\
-21035.2588147037	0.50167521038886\\
-21023.2558139535	4.66149777214939\\
-21011.2528132033	1.83321242719892\\
-20999.2498124531	0.212722745091947\\
-20987.2468117029	1.91651456322037\\
-20975.2438109527	4.81337371631222\\
-20963.2408102026	0.474390430501859\\
-20951.2378094524	2.43884366122617\\
-20939.2348087022	9.33309085098236\\
-20927.231807952	0.786198761734028\\
-20915.2288072018	5.29471365652673\\
-20903.2258064516	7.47068960339705\\
-20891.2228057014	0.547280925353212\\
-20879.2198049512	4.68793012842136\\
-20867.2168042011	0.942971629495915\\
-20855.2138034509	1.6016172630836\\
-20843.2108027007	1.57639053585198\\
-20831.2078019505	0.958523063945631\\
-20819.2048012003	4.84411241544257\\
-20807.2018004501	0.578200616388963\\
-20795.1987996999	7.36361270185087\\
-20783.1957989497	5.6351188005537\\
-20771.1927981996	0.667057794061394\\
-20759.1897974494	9.17149603653661\\
-20747.1867966992	2.33576420594253\\
-20735.183795949	0.507363107570372\\
-20723.1807951988	4.69263239756124\\
-20711.1777944486	1.84423762547936\\
-20699.1747936984	0.214498086533157\\
-20687.1717929482	1.93688736145785\\
-20675.1687921981	4.86155034950508\\
-20663.1657914479	0.477016212521047\\
-20651.1627906977	2.46529909377103\\
-20639.1597899475	9.42267719192665\\
-20627.1567891973	0.799192814919013\\
-20615.1537884471	5.324633718755\\
-20603.1507876969	7.54147326598871\\
-20591.1477869467	0.550574043264813\\
-20579.1447861965	4.72143080167763\\
-20567.1417854464	0.950537700681892\\
-20555.1387846962	1.61710100063043\\
-20543.135783946	1.58914978526584\\
-20531.1327831958	0.967771045127333\\
-20519.1297824456	4.89448636632487\\
-20507.1267816954	0.584834435035487\\
-20495.1237809452	7.42283496845099\\
-20483.1207801951	5.70182062271232\\
-20471.1177794449	0.667187548096823\\
-20459.1147786947	9.24363624244331\\
-20447.1117779445	2.35108811758047\\
-20435.1087771943	0.513552782003965\\
-20423.1057764441	4.72777630329213\\
-20411.1027756939	1.85680713224469\\
-20399.0997749437	0.216465417524755\\
-20387.0967741935	1.95911330151596\\
-20375.0937734434	4.91429916894131\\
-20363.0907726932	0.480042484626108\\
-20351.087771943	2.4941259249901\\
-20339.0847711928	9.52099963209569\\
-20327.0817704426	0.813096680017778\\
-20315.0787696924	5.35896967839115\\
-20303.0757689422	7.61924101590352\\
-20291.0727681921	0.554328237499473\\
-20279.0697674419	4.75902144610151\\
-20267.0667666917	0.95895357799182\\
-20255.0637659415	1.63410963749165\\
-20243.0607651913	1.60334016004164\\
-20231.0577644411	0.977927385868648\\
-20219.0547636909	4.94953262214849\\
-20207.0517629407	0.592044700304613\\
-20195.0487621905	7.48864421404258\\
-20183.0457614404	5.7742094026297\\
-20171.0427606902	0.667785165920205\\
-20159.03975994	9.32391379498035\\
-20147.0367591898	2.36840059375576\\
-20135.0337584396	0.520266413183663\\
-20123.0307576894	4.76704589331893\\
-20111.0277569392	1.87096193934684\\
-20099.0247561891	0.218631417310282\\
-20087.0217554389	1.98327015103004\\
-20075.0187546887	4.97180360378746\\
-20063.0157539385	0.483479360353565\\
-20051.0127531883	2.52542517267864\\
-20039.0097524381	9.62839726354941\\
-20027.0067516879	0.82796201893958\\
-20015.0037509377	5.39783271433694\\
-20003.0007501875	7.70426164507499\\
-19990.9977494374	0.558555750009746\\
-19978.9947486872	4.80082774126993\\
-19966.991747937	0.968248079850643\\
-19954.9887471868	1.65270320082466\\
-19942.9857464366	1.61901078380793\\
-19930.9827456864	0.989027740361141\\
-19918.9797449362	5.00944463853419\\
-19906.976744186	0.599857084856885\\
-19894.9737434359	7.56126309096223\\
-19882.9707426857	5.85254432171867\\
-19870.9677419355	0.668850956159725\\
-19858.9647411853	9.41259823748342\\
-19846.9617404351	2.3877581506376\\
-19834.9587396849	0.527528503919264\\
-19822.9557389347	4.81057276538142\\
-19810.9527381845	1.8867487017145\\
-19798.9497374344	0.22100354468386\\
-19786.9467366842	2.00944389868789\\
-19774.943735934	5.03426693730449\\
-19762.9407351838	0.487338436051973\\
-19750.9377344336	2.55930844495378\\
-19738.9347336834	9.745246398792\\
-19726.9317329332	0.843845345076874\\
-19714.928732183	5.44134993788144\\
-19702.9257314329	7.79683368468031\\
-19690.9227306827	0.563270513625248\\
-19678.9197299325	4.84699120867225\\
-19666.9167291823	0.978453478319791\\
-19654.9137284321	1.67294839234469\\
-19642.9107276819	1.63621667293952\\
-19630.9077269317	1.0011117136231\\
-19618.9047261815	5.07443658928982\\
-19606.9017254314	0.608299919342938\\
-19594.8987246812	7.64094069095437\\
-19582.895723931	5.93711115968295\\
-19570.8927231808	0.670386405565303\\
-19558.8897224306	9.5099913653246\\
-19546.8867216804	2.40922470705287\\
-19534.8837209302	0.535366080059402\\
-19522.88072018	4.85850467667542\\
-19510.8777194299	1.9042200721909\\
-19498.8747186797	0.223590094622424\\
-19486.8717179295	2.03772943407388\\
-19474.8687171793	5.10191390117173\\
-19462.8657164291	0.491632877163824\\
-19450.8627156789	2.59589882465737\\
-19438.8597149287	9.87196349022195\\
-19426.8567141785	0.860808504730991\\
-19414.8537134284	5.48966529543469\\
-19402.8507126782	7.89728772830713\\
-19390.847711928	0.568488252056471\\
-19378.8447111778	4.89767026464689\\
-19366.8417104276	0.989605747135098\\
-19354.8387096774	1.69491912214144\\
-19342.8357089272	1.65501916600572\\
-19330.832708177	1.01422317637611\\
-19318.8297074269	5.1447450713367\\
-19306.8267066767	0.61740442175772\\
-19294.8237059265	7.72795442906242\\
-19282.8207051763	6.02822462458745\\
-19270.8177044261	0.672394170939991\\
-19258.8147036759	9.61642948377279\\
-19246.8117029257	2.43287204374209\\
-19234.8087021755	0.543808916252823\\
-19222.8057014254	4.91100664709302\\
-19210.8027006752	1.92343508553313\\
-19198.799699925	0.226400262547404\\
-19186.7966991748	2.06823131545183\\
-19174.7936984246	5.17499247802354\\
-19162.7906976744	0.496377517704824\\
-19150.7876969242	2.6353318676504\\
-19138.784696174	10.0090084323169\\
-19126.7816954239	0.878919218978203\\
-19114.7786946737	5.5429406063306\\
-19102.7756939235	8.00598906109515\\
-19090.7726931733	0.574226594530475\\
-19078.7696924231	4.9530414200228\\
-19066.7666916729	1.00174484373787\\
-19054.7636909227	1.71869711348751\\
-19042.7606901725	1.67548641242161\\
-19030.7576894224	1.02841062178801\\
-19018.7546886722	5.22063103146351\\
-19006.751687922	0.627204956373845\\
-18994.7486871718	7.8226121829896\\
-18982.7456864216	6.12623098263453\\
-18970.7426856714	0.674878075477864\\
-18958.7396849212	9.7322859724916\\
-18946.736684171	2.4587803278316\\
-18934.7336834209	0.552889791002036\\
-18922.7306826707	4.96826221431961\\
-18910.7276819205	1.94445959616804\\
-18898.7246811703	0.229444217088839\\
-18886.7216804201	2.10106463654419\\
-18874.7186796699	5.25377593758476\\
-18862.7156789197	0.50158897425213\\
-18850.7126781695	2.6777567294394\\
-18838.7096774194	10.1568882918226\\
-18826.7066766692	0.8982516945072\\
-18814.703675919	5.60135674896967\\
-18802.7006751688	8.12334063226933\\
-18790.6976744186	0.580505206519567\\
-18778.6946736684	5.01330064343355\\
-18766.6916729182	1.0149150292276\\
-18754.688672168	1.74437258785146\\
-18742.6856714179	1.69769392710346\\
-18730.6826706677	1.04372756915495\\
-18718.6796699175	5.30238194400624\\
-18706.6766691673	0.637739326061943\\
-18694.6736684171	7.92525471658445\\
-18682.6706676669	6.2315110278539\\
-18670.6676669167	0.677843109041144\\
-18658.6646661665	9.85797419167816\\
-18646.6616654164	2.48703870905329\\
-18634.6586646662	0.562644775079932\\
-18622.655663916	5.03047485755866\\
-18610.6526631658	1.96736677588799\\
-18598.6496624156	0.232733182349985\\
-18586.6466616654	2.13635600508435\\
-18574.6436609152	5.33856513664459\\
-18562.640660165	0.507285776068312\\
-18550.6376594149	2.72333743697707\\
-18538.6346586647	10.3161615221758\\
-18526.6316579145	0.918887313253485\\
-18514.6286571643	5.66511501005927\\
-18502.6256564141	8.24978641381491\\
-18490.6226556639	0.587345938299072\\
-18478.6196549137	5.07866490632169\\
-18466.6166541635	1.02916523083546\\
-18454.6136534134	1.77204504006764\\
-18442.6106526632	1.72172521948913\\
-18430.607651913	1.06023302065373\\
-18418.6046511628	5.39031427111923\\
-18406.6016504126	0.649049102565254\\
-18394.5986496624	8.0362584236408\\
-18382.5956489122	6.34448343660934\\
-18370.592648162	0.681295432032719\\
-18358.5896474119	9.99395076878678\\
-18346.5866466617	2.51774599560863\\
-18334.5836459115	0.573113557758948\\
-18322.5806451613	5.09786961006129\\
-18310.5776444111	1.99223767776471\\
-18298.5746436609	0.236279530949383\\
-18286.5716429107	2.17424464831934\\
-18274.5686421605	5.42969111777556\\
-18262.5656414104	0.513488513176356\\
-18250.5626406602	2.77225432525164\\
-18238.55963991	10.4874427232084\\
-18226.5566391598	0.940915412492521\\
-18214.5536384096	5.73443861553345\\
-18202.5506376594	8.38581519676832\\
-18190.5476369092	0.59477299337748\\
-18178.544636159	5.1493739320627\\
-18166.5416354089	1.04454945229982\\
-18154.5386346587	1.80182411606074\\
-18142.5356339085	1.74767250618221\\
-18130.5326331583	1.07799197821316\\
-18118.5296324081	5.48477624478243\\
-18106.5266316579	0.661179999899031\\
-18094.5236309077	8.15603842923546\\
-18082.5206301575	6.46560856133365\\
-18070.5176294074	0.685242382294609\\
-18058.5146286572	10.1407193153864\\
-18046.511627907	2.55101141880086\\
-18034.5086271568	0.584339816330935\\
-18022.5056264066	5.17069488358244\\
-18010.5026256564	2.01916187432361\\
-17998.4996249062	0.240096889142646\\
-17986.496624156	2.21488366281\\
-17974.4936234059	5.52751804766043\\
-17962.4906226557	0.520220004516776\\
-17950.4876219055	2.82470566135351\\
-17938.4846211553	10.6714080217142\\
-17926.4816204051	0.964434168736622\\
-17914.4786196549	5.80957446327964\\
-17902.4756189047	8.53196488343717\\
-17890.4726181545	0.602813118991802\\
-17878.4696174044	5.22569217448887\\
-17866.4666166542	1.06112723839552\\
-17854.463615904	1.83383060710358\\
-17842.4606151538	1.77563751866485\\
-17830.4576144036	1.09707602879144\\
-17818.4546136534	5.58615101423469\\
-17806.4516129032	0.67418229697366\\
-17794.448612153	8.28505209665051\\
-17782.4456114029	6.59539272556684\\
-17770.4426106527	0.689692484620319\\
-17758.4396099025	10.2988346254222\\
-17746.4366091523	2.58695549703342\\
-17734.4336084021	0.59637163512018\\
-17722.4306076519	5.24922453046595\\
-17710.4276069017	2.04823817894263\\
-17698.4246061515	0.244200255694661\\
-17686.4216054014	2.25844142894988\\
-17674.4186046512	5.63244654182045\\
-17662.415603901	0.527505488747392\\
-17650.4126031508	2.88090948260056\\
-17638.4096024006	10.8688011559974\\
-17626.4066016504	0.989551601268463\\
-17614.4036009002	5.89079508238344\\
-17602.40060015	8.68882734318941\\
-17590.3975993999	0.611495821524138\\
-17578.3945986497	5.30791105517785\\
-17566.3915978995	1.07896420081053\\
-17554.3885971493	1.86819757699569\\
-17542.3855963991	1.8057324185626\\
-17530.3825956489	1.11756400760929\\
-17518.3795948987	5.69486021111478\\
-17506.3765941485	0.688111316556273\\
-17494.3735933984	8.42380299291473\\
-17482.3705926482	6.73439309308765\\
-17470.367591898	0.694655462122788\\
-17458.3645911478	10.4689074213489\\
-17446.3615903976	2.6257110113171\\
-17434.3585896474	0.609261981306872\\
-17422.3555888972	5.33376017578137\\
-17410.352588147	2.07957546081392\\
-17398.3495873968	0.248606136377257\\
-17386.3465866467	2.30510321396659\\
-17374.3435858965	5.74491743137696\\
-17362.3405851463	0.535372840580023\\
-17350.3375843961	2.94110567973436\\
-17338.3345836459	11.0804403658644\\
-17326.3315828957	1.01638671391135\\
-17314.3285821455	5.97840084588993\\
-17302.3255813954	8.85705391341536\\
-17290.3225806452	0.620853609787503\\
-17278.319579895	5.39635149424624\\
-17266.3165791448	1.09813261400496\\
-17254.3135783946	1.90507164168305\\
-17242.3105776444	1.83808083605403\\
-17230.3075768942	1.13954275066881\\
-17218.304576144	5.81136799201278\\
-17206.3015753938	0.703027968907326\\
-17194.2985746437	8.57284537532688\\
-17182.2955738935	6.88322319680359\\
-17170.2925731433	0.700142248759494\\
-17158.2895723931	10.651609719063\\
-17146.2865716429	2.66742410650764\\
-17134.2835708927	0.623069246171345\\
-17122.2805701425	5.4246338542093\\
-17110.2775693924	2.11329356568162\\
-17098.2745686422	0.253332696280234\\
-17086.271567892	2.35507299122974\\
-17074.2685671418	5.86541603650894\\
-17062.2655663916	0.543852816098958\\
-17050.2625656414	3.00555836152661\\
-17038.2595648912	11.3072262036791\\
-17026.256564141	1.04507079677741\\
-17014.2535633908	6.07272247017327\\
-17002.2505626407	9.03736163763192\\
-16990.2475618905	0.630922269962847\\
-16978.2445611403	5.49136677436428\\
-16966.2415603901	1.11871209082222\\
-16954.2385596399	1.944614423576\\
-16942.2355588897	1.87281904875788\\
-16930.2325581395	1.16310794970484\\
-16918.2295573893	5.9361856310122\\
-16906.2265566392	0.718999369850247\\
-16894.223555889	8.73278927236999\\
-16882.2205551388	7.04255922778816\\
-16870.2175543886	0.706165002115773\\
-16858.2145536384	10.8476808991286\\
-16846.2115528882	2.71225553456398\\
-16834.208552138	0.637857861958614\\
-16822.2055513878	5.52221099490639\\
-16810.2025506377	2.14952435658346\\
-16798.1995498875	0.258399932512669\\
-16786.1965491373	2.40857550851267\\
-16774.1935483871	5.99447702244245\\
-16762.1905476369	0.552979330994214\\
-16750.1875468867	3.07455854336453\\
-16738.1845461365	11.550150402195\\
-16726.1815453863	1.07574891392051\\
-16714.1785446362	6.17412383850541\\
-16702.175543886	9.23054035065022\\
-16690.1725431358	0.641741176292446\\
-16678.1695423856	5.5933457850886\\
-16666.1665416354	1.14079034952358\\
-16654.1635408852	1.98700420745802\\
-16642.160540135	1.91009732222008\\
-16630.1575393848	1.18836512507377\\
-16618.1545386347	6.0698767436708\\
-16606.1515378845	0.736099544787048\\
-16594.1485371343	8.90430624390315\\
-16582.1455363841	7.21314720295137\\
-16570.1425356339	0.712737115238768\\
-16558.1395348837	11.0579345841283\\
-16546.1365341335	2.76038205926808\\
-16534.1335333833	0.653699006255712\\
-16522.1305326332	5.62689380273833\\
-16510.127531883	2.18841289084702\\
-16498.1245311328	0.263829870303492\\
-16486.1215303826	2.46585864386858\\
-16474.1185296324	6.13268992684658\\
-16462.1155288822	0.562789776569888\\
-16450.112528132	3.14842721006839\\
-16438.1095273818	11.8103059617073\\
-16426.1065266317	1.10858160731809\\
-16414.1035258815	6.28300519167816\\
-16402.1005251313	9.43746074008476\\
-16390.0975243811	0.653353642482057\\
-16378.0945236309	5.70271670222109\\
-16366.0915228807	1.1644640860042\\
-16354.0885221305	2.03243782936844\\
-16342.0855213803	1.95008143631105\\
-16330.0825206302	1.21543073505754\\
-16318.07951988	6.21306324269601\\
-16306.0765191298	0.754410232276114\\
-16294.0735183796	9.08813592140451\\
-16282.0705176294	7.3958111501778\\
-16270.0675168792	0.719873226106161\\
-16258.064516129	11.283266438774\\
-16246.0615153788	2.81199804434637\\
-16234.0585146287	0.670671408120499\\
-16222.0555138785	5.73912509381341\\
-16210.0525131283	2.23011875272844\\
-16198.0495123781	0.269646786046352\\
-16186.0465116279	2.52719609412522\\
-16174.0435108777	6.28070546421958\\
-16162.0405101275	0.573325378807729\\
-16150.0375093773	3.22751881208381\\
-16138.0345086272	12.0888986426857\\
-16126.031507877	1.14374685367643\\
-16114.0285071268	6.3998067379012\\
-16102.0255063766	9.65908353527592\\
-16090.0225056264	0.665807319558815\\
-16078.0195048762	5.81995116617072\\
-16066.016504126	1.18983996704404\\
-16054.0135033758	2.08113283516049\\
-16042.0105026257	1.99295442634847\\
-16030.0075018755	1.24443344284654\\
-16018.0045011253	6.36643214001717\\
-16006.0015003751	0.774021803160952\\
-15993.9984996249	9.28509344487219\\
-15981.9954988747	7.59146247763186\\
-15969.9924981245	0.727589223046481\\
-15957.9894973743	11.5246630311637\\
-15945.9864966242	2.86731725118745\\
-15933.983495874	0.68886227270349\\
-15921.9804951238	5.85939265224556\\
-15909.9774943736	2.27481756392716\\
-15897.9744936234	0.275877461456349\\
-15885.9714928732	2.59289045025834\\
-15873.968492123	6.43924273162297\\
-15861.9654913728	0.58463160715331\\
-15849.9624906227	3.31222526556372\\
-15837.9594898725	12.3872600883572\\
-15825.9564891223	1.18144231733753\\
-15813.9534883721	6.52501274025295\\
-15801.9504876219	9.89647000247771\\
-15789.9474868717	0.679154646849209\\
-15777.9444861215	5.94556903520025\\
-15765.9414853713	1.21703576374009\\
-15753.9384846212	2.13332995343303\\
-15741.935483871	2.03891857355363\\
-15729.9324831208	1.27551556726909\\
-15717.9294823706	6.5307433336097\\
-15705.9264816204	0.795034314452611\\
-15693.9234808702	9.4960779351813\\
-15681.92048012	7.80111072290146\\
-15669.9174793698	0.735902243790814\\
-15657.9144786197	11.7832119162913\\
-15645.9114778695	2.92657487617458\\
-15633.9084771193	0.708368344592474\\
-15621.9054763691	5.98823418863049\\
-15609.9024756189	2.32270269847948\\
-15597.8994748687	0.282551473752425\\
-15585.8964741185	2.66327672280962\\
-15573.8934733683	6.60909746293096\\
-15561.8904726182	0.596758640621718\\
-15549.887471868	3.40298053895231\\
-15537.8844711178	12.7068628409146\\
-15525.8814703676	1.22188795105485\\
-15513.8784696174	6.65915615205613\\
-15501.8754688672	10.1507939606869\\
-15489.872468117	0.693453363921622\\
-15477.8694673668	6.08014380199663\\
-15465.8664666167	1.24618164754267\\
-15453.8634658665	2.18929593447276\\
-15441.8604651163	2.08819768491816\\
-15429.8574643661	1.30883474741188\\
-15417.8544636159	6.70683854302333\\
-15405.8514628657	0.817558720615267\\
-15393.8484621155	9.72208216566092\\
-15381.8454613653	8.02587591697315\\
-15369.8424606152	0.744830665418755\\
-15357.839459865	12.0601131326271\\
-15345.8364591148	2.99002986361215\\
-15333.8334583646	0.729297133705165\\
-15321.8304576144	6.12624299222475\\
-15309.8274568642	2.37398723291028\\
-15297.824456114	0.289701527702511\\
-15285.8214553638	2.73872639386883\\
-15273.8184546137	6.7911515073607\\
-15261.8154538635	0.609761900275317\\
-15249.8124531133	3.50026592604704\\
-15237.8094523631	13.0493375651879\\
-15225.8064516129	1.26532900709205\\
-15213.8034508627	6.80282388140933\\
-15201.8004501125	10.423355567722\\
-15189.7974493623	0.708767092787131\\
-15177.7944486122	6.22430877929453\\
-15165.791447862	1.27742167544271\\
-15153.7884471118	2.24932681834899\\
-15141.7854463616	2.14103971118676\\
-15129.7824456114	1.34456585770686\\
-15117.7794448612	6.89565158817691\\
-15105.776444111	0.841718269440613\\
-15093.7734433608	9.96420362598962\\
-15081.7704426107	8.26700284270346\\
-15069.7674418605	0.75439408165765\\
-15057.7644411103	12.3566923370948\\
-15033.7584396099	0.751768332624693\\
-15021.7554388597	6.2740743889008\\
-15009.7524381095	2.42890616781825\\
-14997.7494373593	0.297363836484063\\
-14985.7464366092	2.81965208604261\\
-14973.743435859	6.98638374217131\\
-14961.7404351088	0.623702658863533\\
-14949.7374343586	3.60461612383042\\
-14937.7344336084	13.416492853386\\
-14925.7314328582	1.3120395334146\\
-14913.728432108	6.95666278100186\\
-14901.7254313578	10.7155971811679\\
-14889.7224306077	0.725166001118802\\
-14877.7194298575	6.37876417872761\\
-14865.7164291073	1.31091549622841\\
-14853.7134283571	2.31375170681758\\
-14841.7104276069	2.19771976079337\\
-14829.7074268567	1.382903217024\\
-14817.7044261065	7.09822024582336\\
-14805.7014253563	0.867650115212294\\
-14793.6984246062	10.223657208546\\
-14781.695423856	8.52587752605412\\
-14769.6924231058	0.764613263230436\\
-14757.6894223556	12.6744158462438\\
-14733.6834208552	0.775915460013202\\
-14721.680420105	6.43245313549281\\
-14709.6774193548	2.48771896424748\\
-14697.6744186047	0.305578559587764\\
-14685.6714178545	2.90651295742426\\
-14673.6684171043	7.19588266955787\\
-14661.6654163541	0.638648740503103\\
-14649.6624156039	3.71662625733672\\
-14637.6594148537	13.8103380592998\\
-14625.6564141035	1.36232644599567\\
-14613.6534133533	7.12138647603517\\
-14601.6504126032	11.0291216521535\\
-14589.647411853	0.742727559435058\\
-14577.6444111028	6.54428523149197\\
-14565.6414103526	1.3468403157694\\
-14553.6384096024	2.38293712908672\\
-14541.6354088522	2.25854357872977\\
-14529.632408102	1.42406314405687\\
-14517.6294073518	7.31569996392207\\
-14505.6264066017	0.895507188257302\\
-14493.6234058515	10.5017897924765\\
-14481.6204051013	8.80404636409461\\
-14469.6174043511	0.775510095520686\\
-14457.6144036009	13.0149079020149\\
-14433.6084021005	0.801887772063775\\
-14421.6054013503	6.60218190718264\\
-14409.6024006002	2.55071244559659\\
-14397.59939985	0.314390307702318\\
-14385.5963990998	2.99982095354973\\
-14373.5933983496	7.42086100159603\\
-14361.5903975994	0.654675325671615\\
-14349.5873968492	3.83696002323977\\
-14337.584396099	14.2331096984868\\
-14325.5813953488	1.41653428722733\\
-14313.5783945987	7.29778316395229\\
-14301.5753938485	11.3657134880368\\
-14289.5723930983	0.761537407363424\\
-14277.5693923481	6.72173152798558\\
-14265.5663915979	1.38539316673834\\
-14253.5633908477	2.45729211060549\\
-14241.5603900975	2.32385157427104\\
-14229.5573893473	1.4682869223965\\
-14217.5543885972	7.54937977334386\\
-14205.551387847	0.925460368469005\\
-14193.5483870968	10.8000970526022\\
-14181.5453863466	9.10323838234682\\
-14169.5423855964	0.787107486595673\\
-14157.5393848462	13.3799705405067\\
-14145.536384096	3.29199150723413\\
-14133.5333833458	0.82985249287112\\
-14121.5303825957	6.78415106446271\\
-14109.5273818455	2.61820412573044\\
-14097.5243810953	0.32384872640304\\
-14085.5213803451	3.10014807509361\\
-14073.5183795949	7.66267259553092\\
-14061.5153788447	0.67186588007285\\
-14049.5123780945	3.96635915961286\\
-14037.5093773443	14.6873020621052\\
-14025.5063765942	1.47505080503109\\
-14013.503375844	7.48672454554944\\
-14001.5003750938	11.7273634044787\\
-13989.4973743436	0.781690347197262\\
-13977.4943735934	6.91205778791685\\
-13965.4913728432	1.42679353807603\\
-13953.488372093	2.53727407653504\\
-13941.4853713428	2.39402349819954\\
-13929.4823705927	1.51584425165573\\
-13917.4793698425	7.80070080665241\\
-13905.4763690923	0.957701020341982\\
-13893.4733683421	11.1202428887058\\
-13881.4703675919	9.42539121689932\\
-13869.4673668417	0.799429236776484\\
-13857.4643660915	13.771606521989\\
-13845.4613653413	3.38135496677542\\
-13833.4583645911	0.859997425757213\\
-13821.455363841	6.97934992354165\\
-13809.4523630908	2.69054603575131\\
-13797.4493623406	0.334009173038392\\
-13785.4463615904	3.20813485278604\\
-13773.4433608402	7.92283218291037\\
-13761.44036009	0.690313229551113\\
-13749.4373593398	4.10565449282423\\
-13737.4343585896	15.1757028287108\\
-13725.4313578395	1.53831351714552\\
-13713.4283570893	7.68917607468757\\
-13701.4253563391	12.1162969029475\\
-13689.4223555889	0.803291486184128\\
-13677.4193548387	7.11632631712989\\
-13665.4163540885	1.47128643012778\\
-13653.4133533383	2.62339575070541\\
-13641.4103525881	2.46948389216469\\
-13629.407351838	1.56703727790879\\
-13617.4043510878	8.07127792212821\\
-13605.4013503376	0.9924439596904\\
-13593.3983495874	11.4640819509697\\
-13581.3953488372	9.77268155039703\\
-13569.392348087	0.81249985831455\\
-13557.3893473368	14.19204586724\\
-13545.3863465866	3.47714464316921\\
-13533.3833458365	0.892534021625443\\
-13521.3803450863	7.18887980052933\\
-13509.3773443361	2.76812913582634\\
-13497.3743435859	0.344933504099161\\
-13485.3713428357	3.32450026469682\\
-13473.3683420855	8.20303843005952\\
-13461.3653413353	0.710120808021456\\
-13449.3623405851	4.2557788692154\\
-13437.359339835	15.7014346249984\\
-13425.3563390848	1.60681746327307\\
-13413.3533383346	7.90620875367867\\
-13401.3503375844	12.5350076416126\\
-13389.3473368342	0.82645755355926\\
-13377.344336084	7.33572145777247\\
-13365.3413353338	1.51914591661957\\
-13353.3383345836	2.71623324637121\\
-13341.3353338335	2.55070845966408\\
-13329.3323330833	1.62220531740538\\
-13317.3293323331	8.36292504360456\\
-13305.3263315829	1.02993093801276\\
-13293.3233308327	11.8336858364779\\
-13281.3203300825	10.1475608927696\\
-13269.3173293323	0.826344330928038\\
-13257.3143285821	14.6437766647252\\
-13245.311327832	3.57988459389685\\
-13233.3083270818	0.927700997801595\\
-13221.3053263316	7.41396915379732\\
-13209.3023255814	2.85138841640437\\
-13197.2993248312	0.356690994164336\\
-13185.296324081	3.45005338167329\\
-13173.2933233308	8.50520098653555\\
-13161.2903225806	0.731404111268398\\
-13149.2873218305	4.41778234644079\\
-13137.2843210803	16.2680036982382\\
-13125.2813203301	1.68112439520474\\
-13113.2783195799	8.13901274321493\\
-13101.2753188297	12.9862965431461\\
-13089.2723180795	0.851318423273559\\
-13077.2693173293	7.57156640654185\\
-13065.2663165791	1.57067931153421\\
-13053.263315829	2.81643559038131\\
-13041.2603150788	2.63823154168507\\
-13029.2573143286	1.68173041318302\\
-13017.2543135784	8.67768496183325\\
-13005.2513128282	1.07043475025498\\
-12993.248312078	12.2313736580303\\
-12981.2453113278	10.5527978067243\\
-12969.2423105776	0.840987775128825\\
-12957.2393098275	15.1295809488297\\
-12945.2363090773	3.69015805987685\\
-12933.2333083271	0.965768622876104\\
-12921.2303075769	7.65599122124136\\
-12909.2273068267	2.94080881428977\\
-12897.2243060765	0.369359412165249\\
-12885.2213053263	3.58570709417693\\
-12873.2183045761	8.83147233076237\\
-12861.215303826	0.754292396598404\\
-12849.2123030758	4.59285010751028\\
-12837.2093023256	16.8793571254539\\
-12825.2063015754	1.76187371630276\\
-12813.2033008252	8.38891311339501\\
-12801.200300075	13.473317791531\\
-12789.1972993248	0.878018879957263\\
-12777.1942985746	7.8253428552318\\
-12765.1912978245	1.62623206185633\\
-12753.1882970743	2.92473597878889\\
-12741.1852963241	2.73265492270181\\
-12729.1822955739	1.74604389738133\\
-12717.1792948237	9.01786451945496\\
-12705.1762940735	1.11426409657209\\
-12693.1732933233	12.6597478422185\\
-12681.1702925731	10.9915279367732\\
-12669.167291823	0.856455020019052\\
-12657.1642910728	15.6525766279965\\
-12645.1612903226	3.80861544035959\\
-12633.1582895724	1.0070438114075\\
-12621.1552888222	7.91648463332535\\
-12609.152288072	3.03693209594802\\
-12597.1492873218	0.383026286531872\\
-12585.1462865716	3.73249435589822\\
-12573.1432858215	9.18428540768037\\
-12561.1402850713	0.778930677755735\\
-12549.1372843211	4.7823236682443\\
-12537.1342835709	17.5399503128911\\
-12525.1312828207	1.84979555809804\\
-12513.1282820705	8.65738812749883\\
-12501.1252813203	13.9996331466891\\
-12489.1222805701	0.906720673638814\\
-12477.11927982	8.09871401081912\\
-12465.1162790698	1.68619351453044\\
-12453.1132783196	3.04196513380073\\
-12441.1102775694	2.83465824571001\\
-12429.1072768192	1.81563417303098\\
-12417.104276069	9.38607632228132\\
-12405.1012753188	1.16176936050176\\
-12393.0982745686	13.1217362037133\\
-12381.0952738185	11.4673135335081\\
-12369.0922730683	0.872770036182498\\
-12357.0892723181	16.2162666609181\\
-12345.0862715679	3.93598356903332\\
-12333.0832708177	1.05187620761744\\
-12321.0802700675	8.19717759349041\\
-12309.0772693173	3.14036489344743\\
-12297.0742685671	0.397790398138639\\
-12285.071267817	3.89158748522373\\
-12273.0682670668	9.56639829768525\\
-12261.0652663166	0.805482075808114\\
-12249.0622655664	4.98772608892279\\
-12237.0592648162	18.2548269603574\\
-12225.056264066	1.94572648191427\\
-12213.0532633158	8.94609053658704\\
-12201.0502625656	14.5692763387216\\
-12189.0472618155	0.93760491844247\\
-12177.0442610653	8.39355167919082\\
-12165.0412603151	1.75100374207308\\
-12153.0382595649	3.16906722394127\\
-12141.0352588147	2.94501138424857\\
-12129.0322580645	1.89105598259247\\
-12117.0292573143	9.78528839804437\\
-12105.0262565641	1.21334950641033\\
-12093.023255814	13.6206415955683\\
-12081.0202550638	11.9842145899912\\
-12069.0172543136	0.889955195771654\\
-12057.0142535634	16.824596948275\\
-12045.0112528132	4.0730765398182\\
-12033.008252063	1.10066548386559\\
-12021.0052513128	8.50001635161782\\
-12009.0022505626	3.25178811958717\\
-11996.9992498125	0.413763549266981\\
-11984.9962490623	4.06432120045806\\
-11972.9932483121	9.98094745653082\\
-11960.9902475619	0.834130601893021\\
-11948.9872468117	5.21079207775793\\
-11936.9842460615	19.0297141925291\\
-11924.9812453113	2.05062842070116\\
-11912.9782445611	9.25687246413499\\
-11900.975243811	15.1868297462719\\
-11888.9722430608	0.970874903037497\\
-11876.9692423106	8.7119682600255\\
-11864.9662415604	1.82116165564924\\
-11852.9632408102	3.30711892713847\\
-11840.96024006	3.06458920541284\\
-11828.9572393098	1.97294149772935\\
-11816.9542385596	10.2188835852279\\
-11804.9512378095	1.26946035104635\\
-11792.9482370593	14.1602007384502\\
-11780.9452363091	12.546874250302\\
-11768.9422355589	0.90803031157382\\
-11756.9392348087	17.4820247640477\\
-11744.9362340585	4.22080838461371\\
-11732.9332333083	1.15387013847017\\
-11720.9302325581	8.82719887108341\\
-11708.927231808	3.37196804091296\\
-11696.9242310578	0.431072668638229\\
-11684.9212303076	4.25222023762138\\
-11672.9182295574	10.4315114630441\\
-11660.9152288072	0.865084466154094\\
-11648.912228057	5.45350410107408\\
-11636.9092273068	19.8711362455314\\
-11624.9062265566	2.16561164265262\\
-11612.9032258065	9.5918145938358\\
-11600.9002250563	15.8575161166441\\
-11588.8972243061	1.00675939601974\\
-11576.8942235559	9.05635470402064\\
-11564.8912228057	1.89723469268241\\
-11552.8882220555	3.45735236822139\\
-11540.8852213053	3.19438927195615\\
-11528.8822205551	2.06201365323912\\
-11516.879219805	10.6907308991366\\
-11504.8762190548	1.33062453123909\\
-11492.8732183046	14.7446542356764\\
-11480.8702175544	13.1606218603003\\
-11468.8672168042	0.927011392677102\\
-11456.864216054	18.1935999824012\\
-11444.8612153038	4.38020797480077\\
-11432.8582145536	1.21201815546858\\
-11420.8552138035	9.18121480847202\\
-11408.8522130533	3.50176935302562\\
-11396.8492123031	0.449862327785018\\
-11384.8462115529	4.45703262301913\\
-11372.8432108027	10.9221877144058\\
-11360.8402100525	0.898580032096649\\
-11348.8372093023	5.71813591068831\\
-11336.8342085521	20.786550968264\\
-11324.831207802	2.29196273557642\\
-11312.8282070518	9.9532605350209\\
-11300.8252063016	16.5873088076746\\
-11288.8222055514	1.04551654895449\\
-11276.8192048012	9.42942574869832\\
-11264.816204051	1.97987044050849\\
-11252.8132033008	3.62118286128944\\
-11240.8102025506	3.33555317690058\\
-11228.8072018005	2.15910226158809\\
-11216.8042010503	11.2052717225795\\
-11204.8012003001	1.39744357766142\\
-11192.7981995499	15.3788302841789\\
-11180.7951987997	13.8315979517692\\
-11168.7921980495	0.946909036829519\\
-11156.7891972993	18.9650619265155\\
-11144.7861965491	4.55243660421735\\
-11132.783195799	1.27571998988174\\
-11120.7801950488	9.56489320716965\\
-11108.7771942986	3.64217068558818\\
-11096.7741935484	0.470297763578925\\
-11084.7711927982	4.680769963866\\
-11072.768192048	11.4576851697049\\
-11060.7651912978	0.934886566927098\\
-11048.7621905476	6.00730528251615\\
-11036.7591897975	21.7845145371758\\
-11024.7561890473	2.43117889710576\\
-11012.7531882971	10.3438574489363\\
-11000.7501875469	17.3830649664124\\
-10988.7471867967	1.08743852487078\\
-10976.7441860465	9.83427408788618\\
-10964.7411852963	2.06981065414727\\
-10952.7381845461	3.80024264652831\\
-10940.735183796	3.48939239606928\\
-10928.7321830458	2.2651635944785\\
-10916.7291822956	11.7676244577235\\
-10904.7261815454	1.47061261912259\\
-10892.7231807952	16.0682452565344\\
-10880.720180045	14.5669066620496\\
-10868.7171792948	0.967726355774079\\
-10856.7141785446	19.802955391193\\
-10844.7111777945	4.73880882079277\\
-10832.7081770443	1.34568447622693\\
-10820.7051762941	9.9814596708434\\
-10808.7021755439	3.79428307145671\\
-10796.6991747937	0.492568527177528\\
-10784.6961740435	4.92575650189662\\
-10772.6931732933	12.0434371020069\\
-10760.6901725431	0.97431198044128\\
-10748.6871717929	6.32403826423473\\
-10736.6841710428	22.8748812709642\\
-10724.6811702926	2.58501019560243\\
-10712.6781695424	10.7666042833148\\
-10700.6751687922	18.2526872838889\\
-10688.672168042	1.13285701193649\\
-10676.6691672918	10.2744355710393\\
-10664.6661665416	2.1679082531834\\
-10652.6631657915	3.99642215460033\\
-10640.6601650413	3.65741979631255\\
-10628.6571642911	2.38130431484689\\
-10616.6541635409	12.3837123142193\\
-10604.6511627907	1.55093839401802\\
-10592.6481620405	16.8192251814459\\
-10580.6451612903	15.3748026957892\\
-10568.6421605401	0.989456299707413\\
-10556.6391597899	20.7147703324351\\
-10544.6361590398	4.94081721269131\\
-10532.6331582896	1.42273843659026\\
-10520.6301575394	10.4346052533311\\
-10508.6271567892	3.95937205149428\\
-10496.624156039	0.51689291298316\\
-10484.6211552888	5.19468917880472\\
-10472.6181545386	12.6857389582511\\
-10460.6151537884	1.01720979955884\\
-10448.6121530383	6.67184789599781\\
-10436.6091522881	24.0690473818048\\
-10424.6061515379	2.7555119745507\\
-10412.6031507877	11.2249093003899\\
-10400.6001500375	19.2053215840897\\
-10388.5971492873	1.18214982273183\\
-10376.5941485371	10.7539681044872\\
-10364.5911477869	2.27514804953577\\
-10352.5881470368	4.21192078857792\\
-10340.5851462866	3.84138827247486\\
-10328.5821455364	2.50881090646106\\
-10316.5791447862	13.0604202872839\\
-10304.576144036	1.63936144817076\\
-10292.5731432858	17.6390532802812\\
-10280.5701425356	16.2649220771046\\
-10268.5671417854	1.01207820639276\\
-10256.5641410353	21.7091109462574\\
-10244.5611402851	5.1601620347808\\
-10232.5581395349	1.5078510036023\\
-10220.5551387847	10.9285699246028\\
-10208.5521380345	4.13888426579406\\
-10196.5491372843	0.543523364993922\\
-10184.5461365341	5.49071163361071\\
-10172.5431357839	13.3919179382363\\
-10160.5401350338	1.06398769935902\\
-10148.5371342836	7.05483125621973\\
-10136.5341335334	25.3802500986649\\
-10124.5311327832	2.94511025825276\\
-10112.528132033	11.7226590231938\\
-10100.5251312828	20.2515996489629\\
-10088.5221305326	1.2357488337372\\
-10076.5191297824	11.2775476880775\\
-10064.5161290323	2.39267218010717\\
-10052.5131282821	4.44930982824631\\
-10040.5101275319	4.04333843683615\\
-10028.5071267817	2.64918610047156\\
-10016.5041260315	13.8057892253016\\
-10004.5011252813	1.73698367001442\\
-9992.49812453114	18.5361502050229\\
-9980.49512378095	17.248568822742\\
-9968.49212303076	1.03555334806929\\
-9956.48912228057	22.7959014807759\\
-9944.48612153038	5.39878678983418\\
-9932.4831207802	1.6021639982652\\
-9920.48012003001	11.4682442873506\\
-9908.47711927982	4.33447961389115\\
-9896.47411852963	0.572753116416909\\
-9884.47111777945	5.81750595309967\\
-9872.46811702926	14.1705429322993\\
-9860.46511627907	1.11511800968398\\
-9848.46211552888	7.47778987853249\\
-9836.4591147787	26.8239370627903\\
-9824.45611402851	3.15668394904821\\
-9812.45311327832	12.2643012885626\\
-9800.45011252813	21.4039395714591\\
-9788.44711177795	1.29414958866571\\
-9776.44411102776	11.8505860331685\\
-9764.44111027757	2.52181151591619\\
-9752.43810952738	4.71161089289855\\
-9740.4351087772	4.26565789320827\\
-9728.43210802701	2.80419427782325\\
-9716.42910727682	14.6292573895648\\
-9704.42610652663	1.8451026838031\\
-9692.42310577644	19.5202956107281\\
-9680.42010502626	18.3390735832179\\
-9668.41710427607	1.05981917942691\\
-9656.41410352588	23.9866382739983\\
-9644.4111027757	5.65892117970216\\
-9632.40810202551	1.70703014594346\\
-9620.40510127532	12.0592943127242\\
-9608.40210052513	4.54807037557007\\
-9596.39909977495	0.604924396453939\\
-9584.39609902476	6.1794072191478\\
-9572.39309827457	15.0316862032875\\
-9560.39009752438	1.17115075077075\\
-9548.3870967742	7.94638020552779\\
-9536.38409602401	28.4182255826062\\
-9524.38109527382	3.39366888695897\\
-9512.37809452363	12.854945844889\\
-9500.37509377344	22.676919821342\\
-9488.37209302326	1.35792298210178\\
-9476.36909227307	12.4793755645662\\
-9464.36609152288	2.66412472270509\\
-9452.36309077269	5.00239453990566\\
-9440.36009002251	4.51115546128421\\
-9428.35708927232	2.97591848170653\\
-9416.35408852213	15.5419633126653\\
-9404.35108777195	1.9652551268956\\
-9392.34808702176	20.6029023705505\\
-9380.34508627157	19.5522456760874\\
-9368.34208552138	1.08478189819522\\
-9356.33908477119	25.2947003855434\\
-9344.33608402101	5.9431332333548\\
-9332.33308327082	1.82406152598221\\
-9320.33008252063	12.7083153280726\\
-9308.32708177044	4.78186909236582\\
-9296.32408102026	0.640438644101288\\
-9284.32108027007	6.58154756818216\\
-9272.31807951988	15.9872519502628\\
-9260.31507876969	1.23272993555879\\
-9248.31207801951	8.46730296521593\\
-9236.30907726932	30.1844777205765\\
-9224.30607651913	3.66019061979303\\
-9212.30307576894	13.5004869134109\\
-9200.30007501876	24.0877485561938\\
-9188.29707426857	1.42772956267356\\
-9176.29407351838	13.1712694556795\\
-9164.29107276819	2.82144719955701\\
-9152.28807201801	5.32590515254279\\
-9140.28507126782	4.78315486697574\\
-9128.28207051763	3.16683258119639\\
-9116.27906976744	16.5571284690735\\
-9104.27606901725	2.09927153760633\\
-9092.27306826707	21.7973583891906\\
-9080.27006751688	20.9069473880501\\
-9068.26706676669	1.11030680851302\\
-9056.26406601651	26.735735084318\\
-9044.26106526632	6.25439293553792\\
-9032.25806451613	1.95519150517634\\
-9020.25506376594	13.4230234868014\\
-9008.25206301576	5.03844755937238\\
-8996.24906226557	0.679769313425613\\
-8984.24606151538	7.03003878963905\\
-8972.24306076519	17.0513920729935\\
-8960.24006001501	1.30061413239159\\
-8948.23705926482	9.04854342871795\\
-8936.23405851463	32.1480259731849\\
-8924.23105776444	3.96123522412978\\
-8912.22805701425	14.207753445958\\
-8900.22505626407	25.6568570698043\\
-8888.22205551388	1.50433715994491\\
-8876.21905476369	13.9349068709624\\
-8864.2160540135	2.99595288326684\\
-8852.21305326332	5.68722047642389\\
-8840.21005251313	5.08561402367667\\
-8828.20705176294	3.37989339599282\\
-8816.20405101276	17.6905448265874\\
-8804.20105026257	2.24934655931987\\
-8792.19804951238	23.1194559643317\\
-8780.19504876219	22.4258298740057\\
-8768.19204801201	1.13620581714166\\
-8756.18904726182	28.3281397673203\\
-8744.18604651163	6.59615036963005\\
-8732.18304576144	2.10275461203443\\
-8720.18004501125	14.2124956849652\\
-8708.17704426107	5.32081002369729\\
-8696.17404351088	0.723478054314633\\
-8684.17104276069	7.53220572248343\\
-8672.1680420105	18.2410366857278\\
-8660.16504126032	1.37570264239481\\
-8648.16204051013	9.69967881000783\\
-8636.15903975994	34.3390965386084\\
-8624.15603900975	4.30287105450845\\
-8612.15303825957	14.984694580119\\
-8600.15003750938	27.408656576927\\
-8588.14703675919	1.58864276438881\\
-8576.14403600901	14.7804970956976\\
-8564.14103525882	3.19023297276568\\
-8552.13803450863	6.09245728643897\\
-8540.13503375844	5.42327830235321\\
-8528.13203300825	3.61865938952978\\
-8516.12903225806	18.9612016011519\\
-8504.12603150788	2.41812955455879\\
-8492.12303075769	24.5879356128718\\
-8480.1200300075	24.1362848135049\\
-8468.11702925732	1.1622211819647\\
-8456.11402850713	30.0936692211558\\
-8444.11102775694	6.97243231319121\\
-8432.10802700675	2.26959053306984\\
-8420.10502625657	15.0874726996203\\
-8408.10202550638	5.63248470919833\\
-8396.09902475619	0.772235331753052\\
-8384.096024006	8.09688727814807\\
-8372.09302325582	19.5765771438512\\
-8360.09002250563	1.45906916015822\\
-8348.08702175544	10.4322751623511\\
-8336.08402100525	36.7939944094915\\
-8324.08102025506	4.6925393749086\\
-8312.07801950488	15.8406101991226\\
-8300.07501875469	29.3725120891351\\
-8288.0720180045	1.6816998985045\\
-8276.06901725431	15.720181152945\\
-8264.06601650413	3.40739713292501\\
-8252.06301575394	6.54903914161574\\
-8240.06001500375	5.80187944859705\\
-8228.05701425357	3.88744511532192\\
-8216.05401350338	20.3920987515462\\
-8204.05101275319	2.60884271303994\\
-8192.048012003	26.2251810646473\\
-8180.04501125282	26.0716873024843\\
-8168.04201050263	1.18800435719496\\
-8156.03900975244	32.0582073933804\\
-8144.03600900225	7.38796248029374\\
-8132.03300825206	2.45918090981356\\
-8120.03000750187	16.0607456720818\\
-8108.02700675169	5.97763920757433\\
-8096.0240060015	0.826846940072763\\
-8084.02100525131	8.73482846512477\\
-8072.01800450113	21.0827539513545\\
-8060.01500375094	1.5520055290425\\
-8048.01200300075	11.260404871086\\
-8036.00900225056	39.5566390929346\\
-8024.00600150038	5.13943921811965\\
-8012.00300075019	16.7864398205003\\
-8000	31.584007998332\\
-7987.99699924981	1.78475314728964\\
-7975.99399849963	16.7684965138504\\
-7963.99099774944	3.65120489898941\\
-7951.98799699925	7.06604869104166\\
-7939.98499624906	6.22839654169145\\
-7927.98199549887	4.19152435137344\\
-7915.97899474869	22.0113139039779\\
-7903.9759939985	2.8254366322111\\
-7891.97299324831	28.0581160829486\\
-7879.96999249813	28.2730365265706\\
-7867.96699174794	1.21308842490749\\
-7855.96399099775	34.252757345919\\
-7843.96099024756	7.84831232473033\\
-7831.95798949738	2.67583128251868\\
-7819.95498874719	17.1476537049733\\
-7807.951987997	6.36122727769481\\
-7795.94898724681	0.888288432943393\\
-7783.94598649663	9.45919629447507\\
-7771.94298574644	22.7898230936409\\
-7759.93998499625	1.65607929503982\\
-7747.93698424606	12.2013285614241\\
-7735.93398349587	42.6805752993445\\
-7723.93098274569	5.65504272232776\\
-7711.9279819955	17.8351276570317\\
-7699.92498124531	34.0866102702207\\
-7687.92198049512	1.89928212398643\\
-7675.91897974494	17.9429806177207\\
-7663.91597899475	3.92623814233735\\
-7651.91297824456	7.65469658996569\\
-7639.90997749438	6.71140237600855\\
-7627.90697674419	4.5374003950473\\
-7615.903975994	23.8534175291787\\
-7603.90097524381	3.07279762140173\\
-7591.89797449363	30.1193739665561\\
-7579.89497374344	30.7911467247662\\
-7567.89197299325	1.23685215540934\\
-7555.88897224306	36.7147238804772\\
-7543.88597149287	8.36009170562334\\
-7531.88297074269	2.9249159953522\\
-7519.8799699925	18.3667313875806\\
-7507.87696924231	6.78917744664109\\
-7495.87396849212	0.957750307313603\\
-7483.87096774194	10.2862664450611\\
-7471.86796699175	24.7351054904582\\
-7459.86496624156	1.77321039198088\\
-7447.86196549137	13.2764040230814\\
-7435.85896474119	46.2316350482036\\
-7423.855963991	6.25379353645891\\
-7411.85296324081	19.0020882299273\\
-7399.84996249063	36.9338748237518\\
-7387.84696174044	2.0270580169693\\
-7375.84396099025	19.2649637608024\\
-7363.84096024006	4.23813009400237\\
-7351.83795948987	8.32895346462512\\
-7339.83495873968	7.26152912580151\\
-7327.8319579895	4.93317030138183\\
-7315.82895723931	25.9613732342507\\
-7303.82595648912	3.35702738921659\\
-7291.82295573893	32.4488402362112\\
-7279.81995498875	33.6896100505343\\
-7267.81695423856	1.25847319873764\\
-7255.81395348837	39.4895936416092\\
-7243.81095273819	8.93119206405166\\
-7231.807951988	3.21321216845607\\
-7219.80495123781	19.7405613314059\\
-7207.80195048762	7.26863794270656\\
-7195.79894973744	1.03669798481464\\
-7183.79594898725	11.2363485256034\\
-7171.79294823706	26.9650708172305\\
-7159.78994748687	1.90577476715338\\
-7147.78694673668	14.512312931803\\
-7135.7839459865	50.2915053120012\\
-7123.78094523631	6.9540657803599\\
-7111.77794448612	20.3058062890609\\
-7099.77494373593	40.19241860302\\
-7087.77194298575	2.17021713183345\\
-7075.76894223556	20.7606240663824\\
-7063.76594148537	4.59387337883016\\
-7051.76294073519	9.10641329518312\\
-7039.759939985	7.89210314544679\\
-7027.75693923481	5.38902254502026\\
-7015.75393848462	28.389123890467\\
-7003.75093773444	3.68582555685313\\
-6991.74793698425	35.0957133068565\\
-6979.74493623406	37.0488586601582\\
-6967.74193548387	1.27686731037289\\
-6955.73893473368	42.6331623855054\\
-6943.7359339835	9.57109952421623\\
-6931.73293323331	3.54936163652332\\
-6919.72993248312	21.2969111326189\\
-6907.72693173293	7.80829858638301\\
-6895.72393098275	1.12695243691875\\
-6883.72093023256	12.3350496785895\\
-6871.71792948237	29.5381777321923\\
-6859.71492873218	2.05674659118262\\
-6847.711927982	15.9427391765994\\
-6835.70892723181	54.9625730733317\\
-6823.70592648162	7.77949965451671\\
-6811.70292573144	21.7686184713161\\
-6799.69992498125	43.9459719990825\\
-6787.69692423106	2.33135772337033\\
-6775.69392348087	22.4624109609567\\
-6763.69092273068	5.00224014064382\\
-6751.6879219805	10.0094906595119\\
-6739.68492123031	8.62002364833599\\
-6727.68192048012	5.91792735994056\\
-6715.67891972993	31.2051632332807\\
-6703.67591897974	4.06902065802805\\
-6691.67291822956	38.1212953430676\\
-6679.66991747937	40.9718182701865\\
-6667.66691672918	1.29060997731466\\
-6655.663915979	46.2145267579827\\
-6643.66091522881	10.2913022069657\\
-6631.65791447862	3.94451988957803\\
-6619.65491372843	23.0702713152696\\
-6607.65191297825	8.41881942197229\\
-6595.64891222806	1.23080004725877\\
-6583.64591147787	13.6150257818357\\
-6571.64291072768	32.5288022049794\\
-6559.63990997749	2.22989676107426\\
-6547.63690922731	17.6106995703221\\
-6535.63390847712	60.374601610078\\
-6523.63090772693	8.76089085790107\\
-6511.62790697674	23.4177444437896\\
-6499.62490622656	48.3009901510565\\
-6487.62190547637	2.51366926114736\\
-6475.61890472618	24.4109959790258\\
-6463.615903976	5.47436389511204\\
-6451.61290322581	11.0671083359634\\
-6439.60990247562	9.46699956392599\\
-6427.60690172543	6.53661046618703\\
-6415.60390097525	34.4975489440523\\
-6403.60090022506	4.5193194664925\\
-6391.59789947487	41.6028301417696\\
-6379.59489872468	45.5919080921512\\
-6367.59189797449	1.29783666130236\\
-6355.58889722431	50.3201619375366\\
-6343.58589647412	11.1058261274821\\
-6331.58289572393	4.41328342368152\\
-6319.57989497374	25.1039686900756\\
-6307.57689422356	9.11340991513856\\
-6295.57389347337	1.35114456867576\\
-6283.57089272318	15.1184479089334\\
-6271.56789197299	36.0327590612566\\
-6259.56489122281	2.43007521860051\\
-6247.56189047262	19.571834158611\\
-6235.55888972243	66.6940787930981\\
-6223.55588897225	9.93890963426721\\
-6211.55288822206	25.2866660843155\\
-6199.54988747187	53.3945532086613\\
-6187.54688672168	2.72110768452207\\
-6175.54388597149	26.6579930651469\\
-6163.54088522131	6.0245590988615\\
-6151.53788447112	12.3171194631059\\
-6139.53488372093	10.4613232203492\\
-6127.53188297074	7.26695210060846\\
-6115.52888222055	38.3810661910616\\
-6103.52588147037	5.05338379768906\\
-6091.52288072018	45.638871032569\\
-6079.51987996999	51.0845702382737\\
-6067.51687921981	1.29611891195771\\
-6055.51387846962	55.0595699303192\\
-6043.51087771943	12.031949104348\\
-6031.50787696924	4.97504019202686\\
-6019.50487621906	27.4531218958989\\
-6007.50187546887	9.90862457746423\\
-5995.49887471868	1.49172079112891\\
-5983.49587396849	16.900538713925\\
-5971.49287321831	40.1752023428816\\
-5959.48987246812	2.66362088427602\\
-5947.48687171793	21.8991364476025\\
-5935.48387096774	74.1375411993724\\
-5923.48087021755	11.368088479409\\
-5911.47786946737	27.4170009961406\\
-5899.47486871718	59.4056965767703\\
-5887.47186796699	2.95863721689365\\
-5875.4688672168	29.2698266131986\\
-5863.46586646662	6.67149736441343\\
-5851.46286571643	13.8098543538531\\
-5839.45986496624	11.6404668767656\\
-5827.45686421606	8.13803880885359\\
-5815.45386346587	43.0076702560934\\
-5803.45086271568	5.69340948044277\\
-5791.44786196549	50.3569321950084\\
-5779.44486121531	57.6842278233911\\
-5767.44186046512	1.28231770768923\\
-5755.43885971493	60.5732464744981\\
-5743.43585896474	13.0911650252262\\
-5731.43285821455	5.65597785515388\\
-5719.42985746437	30.1888570800971\\
-5707.42685671418	10.8254763608973\\
-5695.42385596399	1.65740049816194\\
-5683.4208552138	19.0347443218354\\
-5671.41785446362	45.1221552055094\\
-5659.41485371343	2.9389707618715\\
-5647.41185296324	24.6898927119167\\
-5635.40885221306	82.9909440231142\\
-5623.40585146287	13.1227946569525\\
-5611.40285071268	29.8610926117728\\
-5599.39984996249	66.5719986459498\\
-5587.3968492123	3.23257078276362\\
-5575.39384846212	32.3333541497296\\
-5563.39084771193	7.43993106528496\\
-5551.38784696174	15.613431567334\\
-5539.38484621155	13.0549724697705\\
-5527.38184546136	9.18924246438816\\
-5515.37884471118	48.5820529400703\\
-5503.37584396099	6.46949464723316\\
-5491.3728432108	55.9246292160526\\
-5479.36984246062	65.7098032214698\\
-5467.36684171043	1.25242695069235\\
-5455.36384096024	67.0441498900054\\
-5443.36084021005	14.3105064900464\\
-5431.35783945987	6.49214102503255\\
-5419.35483870968	33.4044564937639\\
-5407.35183795949	11.891027978138\\
-5395.3488372093	1.85463951413661\\
-5383.34583645912	21.6204674039623\\
-5371.34283570893	51.0977130185007\\
-5359.33983495874	3.26758854507027\\
-5347.33683420855	28.0760953692369\\
-5335.33383345836	93.6384506419738\\
-5323.33083270818	15.3063880082426\\
-5311.32783195799	32.6856633934942\\
-5299.3248312078	75.2144342689316\\
-5287.32183045761	3.55106047570719\\
-5275.31882970743	35.9642476362251\\
-5263.31582895724	8.36327870637301\\
-5251.31282820705	17.8219129301367\\
-5239.30982745687	14.7744315556896\\
-5227.30682670668	10.474961824699\\
-5215.30382595649	55.3854383517058\\
-5203.3008252063	7.42328363358883\\
-5191.29782445612	62.5662954603334\\
-5179.29482370593	75.604117077434\\
-5167.29182295574	1.20144941249923\\
-5155.28882220555	74.7145980288923\\
-5143.28582145536	15.7243907636002\\
-5131.28282070518	7.53420973305678\\
-5119.27981995499	37.2245564996617\\
-5107.2768192048	13.1407219924149\\
-5095.27381845461	2.09214580119702\\
-5083.27081770443	24.7949212424031\\
-5071.26781695424	58.4103576754894\\
-5059.26481620405	3.66542172399143\\
-5047.26181545386	32.2404741361093\\
-5035.25881470368	106.606305504882\\
-5023.25581395349	18.0656389520635\\
-5011.2528132033	35.9770876478846\\
-4999.24981245312	85.7756008858855\\
-4987.24681170293	3.92482359842214\\
-4975.24381095274	40.319852778033\\
-4963.24081020255	9.48760623575733\\
-4951.23780945236	20.5681857839154\\
-4939.23480870217	16.89695204351\\
-4927.23180795199	12.0721416583996\\
-4915.2288072018	63.8130143121655\\
-4903.22580645161	8.61373829927692\\
-4891.22280570143	70.5884561547286\\
-4879.21980495124	87.9965201956708\\
-4867.21680420105	1.12341934523446\\
-4855.21380345086	83.9118315710836\\
-4843.21080270068	17.377247364584\\
-4831.20780195049	8.85519472942114\\
-4819.2048012003	41.8193130839395\\
-4807.20180045012	14.6218894147435\\
-4795.19879969993	2.3819034830874\\
-4783.19579894974	28.7518208629387\\
-4771.19279819955	67.4943599066021\\
-4759.18979744936	4.15526356816251\\
-4747.18679669917	37.4419030991175\\
-4735.18379594899	122.631620759624\\
-4723.1807951988	21.6141190513291\\
-4711.17779444861	39.8492086329612\\
-4699.17479369842	98.8802954998641\\
-4687.17179294824	4.3682522938738\\
-4675.16879219805	45.61959037934\\
-4663.16579144786	10.87794371879\\
-4651.16279069768	24.0449818229956\\
-4639.15978994749	19.564654208601\\
-4627.1567891973	14.0926034093404\\
-4615.15378844711	74.4347763558764\\
-4603.15078769693	10.1265873894246\\
-4591.14778694674	80.4201370708353\\
-4579.14478619655	103.805845170084\\
-4567.14178544636	1.01186142796849\\
-4555.13878469617	95.0878938498756\\
-4543.13578394599	19.3273438742984\\
-4531.1327831958	10.5632610928338\\
-4519.12978244561	47.426962994381\\
-4507.12678169542	16.3992079854313\\
-4495.12378094524	2.74078783761391\\
-4483.12078019505	33.7718207201671\\
-4471.11777944486	78.977055425961\\
-4459.11477869467	4.77072610888334\\
-4447.11177794449	44.0570319329049\\
-4435.1087771943	142.773802258946\\
-4423.10577644411	26.2714733054037\\
-4411.10277569393	44.4552977301944\\
-4399.09977494374	115.434871451847\\
-4387.09677419355	4.90117535051044\\
-4375.09377344336	52.1786170835031\\
-4363.09077269317	12.6286553195634\\
-4351.08777194299	28.5404729167448\\
-4339.0847711928	22.9900310764952\\
-4327.08177044261	16.7040656550392\\
-4315.07876969242	88.0982591852908\\
-4303.07576894224	12.0904045424999\\
-4291.07276819205	92.6790367030454\\
-4279.06976744186	124.416318467805\\
-4267.06676669167	0.86141088604732\\
-4255.06376594149	108.884119610359\\
-4243.0607651913	21.6525048667069\\
-4231.05776444111	12.8239592860366\\
-4219.05476369093	54.3911933502222\\
-4207.05176294074	18.5635258179593\\
-4195.04876219055	3.19319606054724\\
-4183.04576144036	40.2739619943976\\
-4171.04276069017	93.7923121746107\\
-4159.03975993999	5.56321225901281\\
-4147.0367591898	52.6511897799494\\
-4135.03375843961	168.60201320733\\
-4123.03075768942	32.5320301919657\\
-4111.02775693923	50.007044928809\\
-4099.02475618905	136.796863089881\\
-4087.02175543886	5.55178749962865\\
-4075.01875468867	60.466003499966\\
-4063.01575393849	14.8811482874863\\
-4051.0127531883	34.5012243449398\\
-4039.00975243811	27.5028444780095\\
-4027.00675168792	20.1676395200699\\
-4015.00375093774	106.109897290061\\
-4003.00075018755	14.7052269939378\\
-3990.99774943736	108.284962979938\\
-3978.99474868717	151.992106870312\\
-3966.99174793698	0.672434538272767\\
-3954.9887471868	126.239953556781\\
-3942.98574643661	24.4589299024101\\
-3930.98274568642	15.9005897440392\\
-3918.97974493623	63.225970513619\\
-3906.97674418605	21.2467913249752\\
-3894.97374343586	3.77549367287403\\
-3882.97074268567	48.9064991539893\\
-3870.96774193548	113.380413993999\\
-3858.9647411853	6.61476731284855\\
-3846.96174043511	64.1047528208648\\
-3834.95873968492	202.524916155066\\
-3822.95573893474	41.1903806806291\\
-3810.95273818455	56.8060900353443\\
-3798.94973743436	165.078594226375\\
-3786.94673668417	6.36180295291582\\
-3774.94373593398	71.211056142505\\
-3762.9407351838	17.8555403342155\\
-3750.93773443361	42.6493753509518\\
-3738.93473368342	33.638081529685\\
-3726.93173293323	24.9084021496998\\
-3714.92873218305	130.572630031267\\
-3702.92573143286	18.2963055936871\\
-3690.92273068267	128.664603806692\\
-3678.91972993248	190.070988235095\\
-3666.9167291823	0.463487955367887\\
-3654.91372843211	148.586227316712\\
-3642.91072768192	27.8953035607257\\
-3630.90772693173	20.2316327072501\\
-3618.90472618155	74.7343991840645\\
-3606.90172543136	24.6487395418496\\
-3594.89872468117	4.54385352131068\\
-3582.89572393098	60.715733501414\\
-3570.8927231808	140.058805154834\\
-3558.88972243061	8.0631762989206\\
-3546.88672168042	79.8508883536629\\
-3534.88372093023	248.402296893968\\
-3522.88072018004	53.5843333815989\\
-3510.87771942986	65.2992773851818\\
-3498.87471867967	203.722292013415\\
-3486.87171792948	7.39618142878185\\
-3474.8687171793	85.6113145797493\\
-3462.86571642911	21.910460100068\\
-3450.86271567892	54.2145646160122\\
-3438.85971492873	42.3117132044151\\
-3426.85671417855	31.6571167804464\\
-3414.85371342836	165.056027359511\\
-3402.85071267817	23.4218376523474\\
-3390.84771192798	156.145084113401\\
-3378.84471117779	244.760652120884\\
-3366.84171042761	0.305023019524046\\
-3354.83870967742	178.2107144812\\
-3342.83570892723	32.1764140397529\\
-3330.83270817704	26.5894908231835\\
-3318.82970742686	90.2417729919093\\
-3306.82670667667	29.0878859855085\\
-3294.82370592648	5.5887594453116\\
-3282.82070517629	77.4799784555171\\
-3270.81770442611	177.75382560156\\
-3258.81470367592	10.1544827844404\\
-3246.81170292573	102.353241402555\\
-3234.80870217555	312.755111193064\\
-3222.80570142536	72.0976605199214\\
-3210.80270067517	76.1820786977962\\
-3198.79969992498	258.667660380721\\
-3186.79669917479	8.76318450764026\\
-3174.79369842461	105.775453859703\\
-3162.79069767442	27.6636263841022\\
-3150.78769692423	71.4287956753169\\
-3138.78469617404	55.2002311137862\\
-3126.78169542386	41.7584584388796\\
-3114.77869467367	216.036446730199\\
-3102.77569392348	31.107012805727\\
-3090.77269317329	194.771464472556\\
-3078.76969242311	327.353036880724\\
-3066.76669167292	0.414482691627561\\
-3054.76369092273	219.006242536562\\
-3042.76069017254	37.6249483626593\\
-3030.75768942236	36.4338617370424\\
-3018.75468867217	112.092136824184\\
-3006.75168792198	35.1073548626012\\
-2994.74868717179	7.06334913462148\\
-2982.74568642161	102.422723179404\\
-2970.74268567142	233.559684879217\\
-2958.73968492123	13.3625930666803\\
-2946.73668417104	136.147524552395\\
-2934.73368342085	407.359319290365\\
-2922.73068267067	101.291127611182\\
-2910.72768192048	90.6085888682143\\
-2898.72468117029	340.937253680043\\
-2886.72168042011	10.6607311474041\\
-2874.71867966992	135.774205262442\\
-2862.71567891973	36.2553507078789\\
-2850.71267816954	98.6870480934501\\
-2826.70667666917	57.912645403245\\
-2814.70367591898	296.321448748199\\
-2802.70067516879	43.4074551602859\\
-2790.69767441861	252.181404474322\\
-2778.69467366842	460.660248566065\\
-2766.69167291823	1.44588522766558\\
-2754.68867216804	278.165822855253\\
-2742.68567141785	44.750768949342\\
-2730.68267066767	52.7855001663123\\
-2718.67966991748	144.824313995113\\
-2706.67666916729	43.7186350859566\\
-2694.6736684171	9.2421944061489\\
-2682.67066766692	141.890506141043\\
-2670.66766691673	321.397005861348\\
-2658.66466616654	18.6923641310542\\
-2646.66166541636	190.363885325758\\
-2634.65866466617	555.407644781584\\
-2622.65566391598	150.734916303298\\
-2610.65266316579	110.663596768657\\
-2598.6496624156	473.042648456221\\
-2586.64666166542	13.5009686467975\\
-2574.64366091523	184.51732997463\\
-2562.64066016504	49.9857530950483\\
-2550.63765941485	145.626280380616\\
-2538.63465866467	111.091465411\\
-2526.63165791448	86.213454191629\\
-2514.62865716429	434.357106004858\\
-2502.6256564141	64.9633342586475\\
-2490.62265566392	344.504105217851\\
-2478.61965491373	696.628114094702\\
-2466.61665416354	5.48240656888319\\
-2454.61365341335	370.586667598864\\
-2442.61065266317	54.4148756674257\\
-2430.60765191298	82.7105389262932\\
-2418.60465116279	198.374483546832\\
-2406.6016504126	57.0446397864269\\
-2394.59864966242	12.6500309921666\\
-2382.59564891223	209.836343967618\\
-2370.59264816204	471.758693270304\\
-2358.58964741185	28.563730569549\\
-2346.58664666166	285.755743919847\\
-2334.58364591148	808.300492464135\\
-2322.58064516129	243.197431831514\\
-2310.5776444111	140.570120101183\\
-2298.57464366092	707.473016326323\\
-2286.57164291073	18.3222353259713\\
-2274.56864216054	275.356974331012\\
-2262.56564141035	74.0727835052562\\
-2250.56264066017	236.797850703818\\
-2238.55963990998	180.965281178957\\
-2226.55663915979	142.882352002456\\
-2214.5536384096	704.711423005425\\
-2202.55063765942	108.185487530434\\
-2190.54763690923	511.718933757036\\
-2178.54463615904	1175.69948205835\\
-2166.54163540885	20.1116137805067\\
-2154.53863465866	532.770897482474\\
-2142.53563390848	68.2050408716647\\
-2130.53263315829	145.923745600745\\
-2118.5296324081	298.67748554315\\
-2106.52663165791	80.3621125798468\\
-2094.52363090773	18.3431425560257\\
-2082.52063015754	342.110936346619\\
-2070.51762940735	762.635918967568\\
-2058.51462865717	49.9540842440765\\
-2046.51162790698	478.625388179377\\
-2034.50862715679	1301.25760046293\\
-2022.5056264066	443.720258905899\\
-2010.50262565642	190.336043654398\\
-1998.49962490623	1193.30895059209\\
-1986.49662415604	28.6392546890262\\
-1974.49362340585	489.211504995666\\
-1962.49062265566	122.453309764114\\
-1950.48762190547	450.713329697232\\
-1938.48462115529	349.554801989601\\
-1926.4816204051	283.658282599233\\
-1914.47861965491	1360.60330040557\\
-1902.47561890473	216.408280222077\\
-1890.47261815454	881.142132450231\\
-1878.46961740435	2395.7509945473\\
-1866.46661665416	80.847607869558\\
-1854.46361590398	882.310130140934\\
-1842.46061515379	89.3657381049794\\
-1830.4576144036	315.29393566703\\
-1818.45461365341	535.062252201764\\
-1806.45161290323	130.656207124266\\
-1794.44861215304	28.2531211116094\\
-1782.44561140285	655.923418775701\\
-1770.44261065266	1447.05646160751\\
-1758.43960990247	109.598728776581\\
-1746.43660915229	972.714731548502\\
-1734.4336084021	2504.03669770773\\
-1722.43060765191	1003.07880408456\\
-1710.42760690173	289.268108679016\\
-1698.42460615154	2522.08128828623\\
-1686.42160540135	64.3397208801361\\
-1674.41860465116	1286.51501847474\\
-1662.41560390098	242.4968656904\\
-1650.41260315079	1161.98637411208\\
-1638.4096024006	941.893505538606\\
-1626.40660165041	813.333214727217\\
-1614.40360090023	3787.50423327063\\
-1602.40060015004	638.471679118419\\
-1590.39759939985	2098.10977217129\\
-1578.39459864966	7303.52254812055\\
-1566.39159789947	467.055136895017\\
-1554.38859714929	2069.75782020651\\
-1542.3855963991	123.857831444997\\
-1530.38259564891	1058.46713266393\\
-1518.37959489872	1433.12694335601\\
-1506.37659414854	295.593595026244\\
-1494.37359339835	38.8326268375712\\
-1482.37059264816	1749.39883029627\\
-1470.36759189798	3785.63698570825\\
-1458.36459114779	388.561582052675\\
-1446.3615903976	3113.46116854022\\
-1434.35858964741	7330.86335053547\\
-1422.35558889723	3762.76543810872\\
-1410.35258814704	532.786804532551\\
-1398.34958739685	9773.20872892199\\
-1386.34658664666	505.768893850748\\
-1374.34358589647	11663.3495410698\\
-1362.34058514629	664.971926538957\\
-1350.3375843961	7418.8693636665\\
-1338.33458364591	7082.97173923117\\
-1326.33158289572	7920.12111018756\\
-1314.32858214554	38456.026583865\\
-1302.32558139535	7887.78843395165\\
-1290.32258064516	18133.4805388462\\
-1278.31957989497	111029.728891917\\
-1266.31657914479	15855.7174443918\\
-1254.3135783946	32539.3786528162\\
-1242.31057764441	24.5466295891042\\
-1230.30757689423	35146.600807661\\
-1218.30457614404	41137.4603830745\\
-1206.30157539385	6537.58256275289\\
-1194.29857464366	31837.1056379756\\
-1182.29557389347	158193.871988007\\
-1170.29257314328	359485.989674466\\
-1158.2895723931	13432.9128978492\\
-1146.28657164291	32592.1767655665\\
-1134.28357089272	63276.5436016435\\
-1122.28057014253	11325.9303950939\\
-1110.27756939235	19107.848673316\\
-1098.27456864216	14954.4796440505\\
-1086.27156789197	24340.8308511086\\
-1074.26856714179	390522.984143195\\
-1062.2655663916	34426.0206578952\\
-1050.26256564141	4470.97733217436\\
-1038.25956489122	5353.20158081166\\
-1026.25656414104	40066.9795611274\\
-1014.25356339085	99198.681545248\\
-1002.25056264066	58708.8731191483\\
-990.247561890472	180810.300521279\\
-978.244561140284	1134648.55313706\\
-966.241560390099	203951.689943959\\
-954.238559639911	180558.384097101\\
-942.235558889723	23868.9863123291\\
-930.232558139534	26483.3244399311\\
-918.229557389346	39827.4965618752\\
-906.226556639162	14721.9181249682\\
-894.223555888973	12900.4019154985\\
-882.220555138785	46995.1148880861\\
-870.217554388597	83052.9551527789\\
-858.214553638409	5561.37951222576\\
-846.21155288822	19649.200820421\\
-834.208552138036	43718.789422096\\
-822.205551387848	21207.3076459427\\
-810.202550637659	6718.25604702356\\
-798.199549887471	16794.785225199\\
-786.196549137283	376.366609973371\\
-774.193548387098	4106.91254098212\\
-762.19054763691	4251.66798711434\\
-750.187546886722	4553.02080702347\\
-738.184546136534	3011.64910606898\\
-726.181545386346	1337.11718584822\\
-714.178544636161	2140.84247990093\\
-702.175543885973	453.02248081492\\
-690.172543135785	2004.54195317837\\
-678.169542385596	5257.56186233556\\
-666.166541635408	1395.27775093434\\
-654.16354088522	122.988967704739\\
-642.160540135035	226.244231947508\\
-630.157539384847	1251.40307204793\\
-618.154538634659	2022.78657196599\\
-606.151537884471	800.429499916553\\
-594.148537134282	445.42075647898\\
-582.145536384098	2821.87714568807\\
-570.14253563391	5807.22868586146\\
-558.139534883721	755.843328686748\\
-546.136534133533	2048.03372534423\\
-534.133533383345	4050.82670635662\\
-522.130532633157	3265.09719622167\\
-510.127531882972	1277.19044471412\\
-498.124531132784	2657.06188049997\\
-486.121530382596	43.2502231856002\\
-474.118529632407	234.467802520072\\
-462.115528882219	839.141487690402\\
-450.112528132035	1414.57130464108\\
-438.109527381846	1136.54259553406\\
-426.106526631658	540.154137237685\\
-414.10352588147	758.830916066797\\
-402.100525131282	157.588903587999\\
-390.097524381094	720.123528062079\\
-378.094523630909	2133.87509503543\\
-366.091522880721	1026.61704409437\\
-354.088522130533	66.2315646034946\\
-342.085521380344	222.04054730521\\
-330.082520630156	566.812880336863\\
-318.079519879971	1156.67049265391\\
-306.076519129783	497.700404152089\\
-294.073518379595	161.323658397038\\
-282.070517629407	1000.69238108483\\
-270.067516879219	2104.05526093799\\
-258.064516129034	439.774736006491\\
-246.061515378846	730.101117970777\\
-234.058514628658	1142.69907941155\\
-222.055513878469	1544.47109572113\\
-210.052513128281	995.524292930222\\
-198.049512378093	1050.69104023601\\
-186.046511627908	47.9391614340489\\
-174.04351087772	184.250185736912\\
-162.040510127532	456.260569404719\\
-150.037509377344	891.587942072479\\
-138.034508627155	829.554470720175\\
-126.031507876971	403.364028732361\\
-114.028507126783	342.952508891427\\
-102.025506376594	79.2258728098134\\
-90.0225056264062	648.665092725201\\
-78.019504876218	1256.32401167675\\
-66.0165041260298	1088.72827958876\\
-54.0135033758452	366.608954621014\\
-42.010502625657	364.98216782495\\
-30.0075018754687	416.53560014004\\
-18.0045011252805	1263.82838833665\\
-6.00150037509229	581.627206522729\\
6.00150037509229	121.313375889645\\
18.0045011252805	581.627206522729\\
30.0075018754687	1263.82838833665\\
42.010502625657	416.53560014004\\
54.0135033758452	364.98216782495\\
66.0165041260298	366.608954621014\\
78.019504876218	1088.72827958876\\
90.0225056264062	1256.32401167675\\
102.025506376594	648.665092725201\\
114.028507126783	79.2258728098134\\
126.031507876971	342.952508891427\\
138.034508627155	403.364028732361\\
150.037509377344	829.554470720175\\
162.040510127532	891.587942072479\\
174.04351087772	456.260569404719\\
186.046511627908	184.250185736912\\
198.049512378093	47.9391614340489\\
210.052513128281	1050.69104023601\\
222.055513878469	995.524292930222\\
234.058514628658	1544.47109572113\\
246.061515378846	1142.69907941155\\
258.064516129034	730.101117970777\\
270.067516879219	439.774736006491\\
282.070517629407	2104.05526093799\\
294.073518379595	1000.69238108483\\
306.076519129783	161.323658397038\\
318.079519879971	497.700404152089\\
330.082520630156	1156.67049265391\\
342.085521380344	566.812880336863\\
354.088522130533	222.04054730521\\
366.091522880721	66.2315646034946\\
378.094523630909	1026.61704409437\\
390.097524381094	2133.87509503543\\
402.100525131282	720.123528062079\\
414.10352588147	157.588903587999\\
426.106526631658	758.830916066797\\
438.109527381846	540.154137237685\\
450.112528132035	1136.54259553406\\
462.115528882219	1414.57130464108\\
474.118529632407	839.141487690402\\
486.121530382596	234.467802520072\\
498.124531132784	43.2502231856002\\
510.127531882972	2657.06188049997\\
522.130532633157	1277.19044471412\\
534.133533383345	3265.09719622167\\
546.136534133533	4050.82670635662\\
558.139534883721	2048.03372534423\\
570.14253563391	755.843328686748\\
582.145536384098	5807.22868586146\\
594.148537134282	2821.87714568807\\
606.151537884471	445.42075647898\\
618.154538634659	800.429499916553\\
630.157539384847	2022.78657196599\\
642.160540135035	1251.40307204793\\
654.16354088522	226.244231947508\\
666.166541635408	122.988967704739\\
678.169542385596	1395.27775093434\\
690.172543135785	5257.56186233556\\
702.175543885973	2004.54195317837\\
714.178544636161	453.02248081492\\
726.181545386346	2140.84247990093\\
738.184546136534	1337.11718584822\\
750.187546886722	3011.64910606898\\
762.19054763691	4553.02080702347\\
774.193548387098	4251.66798711434\\
786.196549137283	4106.91254098212\\
798.199549887471	376.366609973371\\
810.202550637659	16794.785225199\\
822.205551387848	6718.25604702356\\
834.208552138036	21207.3076459427\\
846.21155288822	43718.789422096\\
858.214553638409	19649.200820421\\
870.217554388597	5561.37951222576\\
882.220555138785	83052.9551527789\\
894.223555888973	46995.1148880861\\
906.226556639162	12900.4019154985\\
918.229557389346	14721.9181249682\\
930.232558139534	39827.4965618752\\
942.235558889723	26483.3244399311\\
954.238559639911	23868.9863123291\\
966.241560390099	180558.384097101\\
978.244561140284	203951.689943959\\
990.247561890472	1134648.55313706\\
1002.25056264066	180810.300521279\\
1014.25356339085	58708.8731191483\\
1026.25656414104	99198.681545248\\
1038.25956489122	40066.9795611274\\
1050.26256564141	5353.20158081166\\
1062.2655663916	4470.97733217436\\
1074.26856714179	34426.0206578952\\
1086.27156789197	390522.984143195\\
1098.27456864216	24340.8308511086\\
1110.27756939235	14954.4796440505\\
1122.28057014253	19107.848673316\\
1134.28357089272	11325.9303950939\\
1146.28657164291	63276.5436016435\\
1158.2895723931	32592.1767655665\\
1170.29257314328	13432.9128978492\\
1182.29557389347	359485.989674466\\
1194.29857464366	158193.871988007\\
1206.30157539385	31837.1056379756\\
1218.30457614404	6537.58256275289\\
1230.30757689423	41137.4603830745\\
1242.31057764441	35146.600807661\\
1254.3135783946	24.5466295891042\\
1266.31657914479	32539.3786528162\\
1278.31957989497	15855.7174443918\\
1290.32258064516	111029.728891917\\
1302.32558139535	18133.4805388462\\
1314.32858214554	7887.78843395165\\
1326.33158289572	38456.026583865\\
1338.33458364591	7920.12111018756\\
1350.3375843961	7082.97173923117\\
1362.34058514629	7418.8693636665\\
1374.34358589647	664.971926538957\\
1386.34658664666	11663.3495410698\\
1398.34958739685	505.768893850748\\
1410.35258814704	9773.20872892199\\
1422.35558889723	532.786804532551\\
1434.35858964741	3762.76543810872\\
1446.3615903976	7330.86335053547\\
1458.36459114779	3113.46116854022\\
1470.36759189798	388.561582052675\\
1482.37059264816	3785.63698570825\\
1494.37359339835	1749.39883029627\\
1506.37659414854	38.8326268375712\\
1518.37959489872	295.593595026244\\
1530.38259564891	1433.12694335601\\
1542.3855963991	1058.46713266393\\
1554.38859714929	123.857831444997\\
1566.39159789947	2069.75782020651\\
1578.39459864966	467.055136895017\\
1590.39759939985	7303.52254812055\\
1602.40060015004	2098.10977217129\\
1614.40360090023	638.471679118419\\
1626.40660165041	3787.50423327063\\
1638.4096024006	813.333214727217\\
1650.41260315079	941.893505538606\\
1662.41560390098	1161.98637411208\\
1674.41860465116	242.4968656904\\
1686.42160540135	1286.51501847474\\
1698.42460615154	64.3397208801361\\
1710.42760690173	2522.08128828623\\
1722.43060765191	289.268108679016\\
1734.4336084021	1003.07880408456\\
1746.43660915229	2504.03669770773\\
1758.43960990247	972.714731548502\\
1770.44261065266	109.598728776581\\
1782.44561140285	1447.05646160751\\
1794.44861215304	655.923418775701\\
1806.45161290323	28.2531211116094\\
1818.45461365341	130.656207124266\\
1830.4576144036	535.062252201764\\
1842.46061515379	315.29393566703\\
1854.46361590398	89.3657381049794\\
1866.46661665416	882.310130140934\\
1878.46961740435	80.847607869558\\
1890.47261815454	2395.7509945473\\
1902.47561890473	881.142132450231\\
1914.47861965491	216.408280222077\\
1926.4816204051	1360.60330040557\\
1938.48462115529	283.658282599233\\
1950.48762190547	349.554801989601\\
1962.49062265566	450.713329697232\\
1974.49362340585	122.453309764114\\
1986.49662415604	489.211504995666\\
1998.49962490623	28.6392546890262\\
2010.50262565642	1193.30895059209\\
2022.5056264066	190.336043654398\\
2034.50862715679	443.720258905899\\
2046.51162790698	1301.25760046293\\
2058.51462865717	478.625388179377\\
2070.51762940735	49.9540842440765\\
2082.52063015754	762.635918967568\\
2094.52363090773	342.110936346619\\
2106.52663165791	18.3431425560257\\
2118.5296324081	80.3621125798468\\
2130.53263315829	298.67748554315\\
2142.53563390848	145.923745600745\\
2154.53863465866	68.2050408716647\\
2166.54163540885	532.770897482474\\
2178.54463615904	20.1116137805067\\
2190.54763690923	1175.69948205835\\
2202.55063765942	511.718933757036\\
2214.5536384096	108.185487530434\\
2226.55663915979	704.711423005425\\
2238.55963990998	142.882352002456\\
2250.56264066017	180.965281178957\\
2262.56564141035	236.797850703818\\
2274.56864216054	74.0727835052562\\
2286.57164291073	275.356974331012\\
2298.57464366092	18.3222353259713\\
2310.5776444111	707.473016326323\\
2322.58064516129	140.570120101183\\
2334.58364591148	243.197431831514\\
2346.58664666166	808.300492464135\\
2358.58964741185	285.755743919847\\
2370.59264816204	28.563730569549\\
2382.59564891223	471.758693270304\\
2394.59864966242	209.836343967618\\
2406.6016504126	12.6500309921666\\
2418.60465116279	57.0446397864269\\
2430.60765191298	198.374483546832\\
2442.61065266317	82.7105389262932\\
2454.61365341335	54.4148756674257\\
2466.61665416354	370.586667598864\\
2478.61965491373	5.48240656888319\\
2490.62265566392	696.628114094702\\
2502.6256564141	344.504105217851\\
2514.62865716429	64.9633342586475\\
2526.63165791448	434.357106004858\\
2538.63465866467	86.213454191629\\
2550.63765941485	111.091465411\\
2562.64066016504	145.626280380616\\
2574.64366091523	49.9857530950483\\
2586.64666166542	184.51732997463\\
2598.6496624156	13.5009686467975\\
2610.65266316579	473.042648456221\\
2622.65566391598	110.663596768657\\
2634.65866466617	150.734916303298\\
2646.66166541636	555.407644781584\\
2658.66466616654	190.363885325758\\
2670.66766691673	18.6923641310542\\
2682.67066766692	321.397005861348\\
2694.6736684171	141.890506141043\\
2706.67666916729	9.2421944061489\\
2718.67966991748	43.7186350859566\\
2730.68267066767	144.824313995113\\
2742.68567141785	52.7855001663123\\
2754.68867216804	44.750768949342\\
2766.69167291823	278.165822855253\\
2778.69467366842	1.44588522766558\\
2790.69767441861	460.660248566065\\
2802.70067516879	252.181404474322\\
2814.70367591898	43.4074551602859\\
2826.70667666917	296.321448748199\\
2838.70967741936	57.912645403245\\
2862.71567891973	98.6870480934501\\
2874.71867966992	36.2553507078789\\
2886.72168042011	135.774205262442\\
2898.72468117029	10.6607311474041\\
2910.72768192048	340.937253680043\\
2922.73068267067	90.6085888682143\\
2934.73368342085	101.291127611182\\
2946.73668417104	407.359319290365\\
2958.73968492123	136.147524552395\\
2970.74268567142	13.3625930666803\\
2982.74568642161	233.559684879217\\
2994.74868717179	102.422723179404\\
3006.75168792198	7.06334913462148\\
3018.75468867217	35.1073548626012\\
3030.75768942236	112.092136824184\\
3042.76069017254	36.4338617370424\\
3054.76369092273	37.6249483626593\\
3066.76669167292	219.006242536562\\
3078.76969242311	0.414482691627561\\
3090.77269317329	327.353036880724\\
3102.77569392348	194.771464472556\\
3114.77869467367	31.107012805727\\
3126.78169542386	216.036446730199\\
3138.78469617404	41.7584584388796\\
3150.78769692423	55.2002311137862\\
3162.79069767442	71.4287956753169\\
3174.79369842461	27.6636263841022\\
3186.79669917479	105.775453859703\\
3198.79969992498	8.76318450764026\\
3210.80270067517	258.667660380721\\
3222.80570142536	76.1820786977962\\
3234.80870217555	72.0976605199214\\
3246.81170292573	312.755111193064\\
3258.81470367592	102.353241402555\\
3270.81770442611	10.1544827844404\\
3282.82070517629	177.75382560156\\
3294.82370592648	77.4799784555171\\
3306.82670667667	5.5887594453116\\
3318.82970742686	29.0878859855085\\
3330.83270817704	90.2417729919093\\
3342.83570892723	26.5894908231835\\
3354.83870967742	32.1764140397529\\
3366.84171042761	178.2107144812\\
3378.84471117779	0.305023019524046\\
3390.84771192798	244.760652120884\\
3402.85071267817	156.145084113401\\
3414.85371342836	23.4218376523474\\
3426.85671417855	165.056027359511\\
3438.85971492873	31.6571167804464\\
3450.86271567892	42.3117132044151\\
3462.86571642911	54.2145646160122\\
3474.8687171793	21.910460100068\\
3486.87171792948	85.6113145797493\\
3498.87471867967	7.39618142878185\\
3510.87771942986	203.722292013415\\
3522.88072018004	65.2992773851818\\
3534.88372093023	53.5843333815989\\
3546.88672168042	248.402296893968\\
3558.88972243061	79.8508883536629\\
3570.8927231808	8.0631762989206\\
3582.89572393098	140.058805154834\\
3594.89872468117	60.715733501414\\
3606.90172543136	4.54385352131068\\
3618.90472618155	24.6487395418496\\
3630.90772693173	74.7343991840645\\
3642.91072768192	20.2316327072501\\
3654.91372843211	27.8953035607257\\
3666.9167291823	148.586227316712\\
3678.91972993248	0.463487955367887\\
3690.92273068267	190.070988235095\\
3702.92573143286	128.664603806692\\
3714.92873218305	18.2963055936871\\
3726.93173293323	130.572630031267\\
3738.93473368342	24.9084021496998\\
3750.93773443361	33.638081529685\\
3762.9407351838	42.6493753509518\\
3774.94373593398	17.8555403342155\\
3786.94673668417	71.211056142505\\
3798.94973743436	6.36180295291582\\
3810.95273818455	165.078594226375\\
3822.95573893474	56.8060900353443\\
3834.95873968492	41.1903806806291\\
3846.96174043511	202.524916155066\\
3858.9647411853	64.1047528208648\\
3870.96774193548	6.61476731284855\\
3882.97074268567	113.380413993999\\
3894.97374343586	48.9064991539893\\
3906.97674418605	3.77549367287403\\
3918.97974493623	21.2467913249752\\
3930.98274568642	63.225970513619\\
3942.98574643661	15.9005897440392\\
3954.9887471868	24.4589299024101\\
3966.99174793698	126.239953556781\\
3978.99474868717	0.672434538272767\\
3990.99774943736	151.992106870312\\
4003.00075018755	108.284962979938\\
4015.00375093774	14.7052269939378\\
4027.00675168792	106.109897290061\\
4039.00975243811	20.1676395200699\\
4051.0127531883	27.5028444780095\\
4063.01575393849	34.5012243449398\\
4075.01875468867	14.8811482874863\\
4087.02175543886	60.466003499966\\
4099.02475618905	5.55178749962865\\
4111.02775693923	136.796863089881\\
4123.03075768942	50.007044928809\\
4135.03375843961	32.5320301919657\\
4147.0367591898	168.60201320733\\
4159.03975993999	52.6511897799494\\
4171.04276069017	5.56321225901281\\
4183.04576144036	93.7923121746107\\
4195.04876219055	40.2739619943976\\
4207.05176294074	3.19319606054724\\
4219.05476369093	18.5635258179593\\
4231.05776444111	54.3911933502222\\
4243.0607651913	12.8239592860366\\
4255.06376594149	21.6525048667069\\
4267.06676669167	108.884119610359\\
4279.06976744186	0.86141088604732\\
4291.07276819205	124.416318467805\\
4303.07576894224	92.6790367030454\\
4315.07876969242	12.0904045424999\\
4327.08177044261	88.0982591852908\\
4339.0847711928	16.7040656550392\\
4351.08777194299	22.9900310764952\\
4363.09077269317	28.5404729167448\\
4375.09377344336	12.6286553195634\\
4387.09677419355	52.1786170835031\\
4399.09977494374	4.90117535051044\\
4411.10277569393	115.434871451847\\
4423.10577644411	44.4552977301944\\
4435.1087771943	26.2714733054037\\
4447.11177794449	142.773802258946\\
4459.11477869467	44.0570319329049\\
4471.11777944486	4.77072610888334\\
4483.12078019505	78.977055425961\\
4495.12378094524	33.7718207201671\\
4507.12678169542	2.74078783761391\\
4519.12978244561	16.3992079854313\\
4531.1327831958	47.426962994381\\
4543.13578394599	10.5632610928338\\
4555.13878469617	19.3273438742984\\
4567.14178544636	95.0878938498756\\
4579.14478619655	1.01186142796849\\
4591.14778694674	103.805845170084\\
4603.15078769693	80.4201370708353\\
4615.15378844711	10.1265873894246\\
4627.1567891973	74.4347763558764\\
4639.15978994749	14.0926034093404\\
4651.16279069768	19.564654208601\\
4663.16579144786	24.0449818229956\\
4675.16879219805	10.87794371879\\
4687.17179294824	45.61959037934\\
4699.17479369842	4.3682522938738\\
4711.17779444861	98.8802954998641\\
4723.1807951988	39.8492086329612\\
4735.18379594899	21.6141190513291\\
4747.18679669917	122.631620759624\\
4759.18979744936	37.4419030991175\\
4771.19279819955	4.15526356816251\\
4783.19579894974	67.4943599066021\\
4795.19879969993	28.7518208629387\\
4807.20180045012	2.3819034830874\\
4819.2048012003	14.6218894147435\\
4831.20780195049	41.8193130839395\\
4843.21080270068	8.85519472942114\\
4855.21380345086	17.377247364584\\
4867.21680420105	83.9118315710836\\
4879.21980495124	1.12341934523446\\
4891.22280570143	87.9965201956708\\
4903.22580645161	70.5884561547286\\
4915.2288072018	8.61373829927692\\
4927.23180795199	63.8130143121655\\
4939.23480870217	12.0721416583996\\
4951.23780945236	16.89695204351\\
4963.24081020255	20.5681857839154\\
4975.24381095274	9.48760623575733\\
4987.24681170293	40.319852778033\\
4999.24981245312	3.92482359842214\\
5011.2528132033	85.7756008858855\\
5023.25581395349	35.9770876478846\\
5035.25881470368	18.0656389520635\\
5047.26181545386	106.606305504882\\
5059.26481620405	32.2404741361093\\
5071.26781695424	3.66542172399143\\
5083.27081770443	58.4103576754894\\
5095.27381845461	24.7949212424031\\
5107.2768192048	2.09214580119702\\
5119.27981995499	13.1407219924149\\
5131.28282070518	37.2245564996617\\
5143.28582145536	7.53420973305678\\
5155.28882220555	15.7243907636002\\
5167.29182295574	74.7145980288923\\
5179.29482370593	1.20144941249923\\
5191.29782445612	75.604117077434\\
5203.3008252063	62.5662954603334\\
5215.30382595649	7.42328363358883\\
5227.30682670668	55.3854383517058\\
5239.30982745687	10.474961824699\\
5251.31282820705	14.7744315556896\\
5263.31582895724	17.8219129301367\\
5275.31882970743	8.36327870637301\\
5287.32183045761	35.9642476362251\\
5299.3248312078	3.55106047570719\\
5311.32783195799	75.2144342689316\\
5323.33083270818	32.6856633934942\\
5335.33383345836	15.3063880082426\\
5347.33683420855	93.6384506419738\\
5359.33983495874	28.0760953692369\\
5371.34283570893	3.26758854507027\\
5383.34583645912	51.0977130185007\\
5395.3488372093	21.6204674039623\\
5407.35183795949	1.85463951413661\\
5419.35483870968	11.891027978138\\
5431.35783945987	33.4044564937639\\
5443.36084021005	6.49214102503255\\
5455.36384096024	14.3105064900464\\
5467.36684171043	67.0441498900054\\
5479.36984246062	1.25242695069235\\
5491.3728432108	65.7098032214698\\
5503.37584396099	55.9246292160526\\
5515.37884471118	6.46949464723316\\
5527.38184546136	48.5820529400703\\
5539.38484621155	9.18924246438816\\
5551.38784696174	13.0549724697705\\
5563.39084771193	15.613431567334\\
5575.39384846212	7.43993106528496\\
5587.3968492123	32.3333541497296\\
5599.39984996249	3.23257078276362\\
5611.40285071268	66.5719986459498\\
5623.40585146287	29.8610926117728\\
5635.40885221306	13.1227946569525\\
5647.41185296324	82.9909440231142\\
5659.41485371343	24.6898927119167\\
5671.41785446362	2.9389707618715\\
5683.4208552138	45.1221552055094\\
5695.42385596399	19.0347443218354\\
5707.42685671418	1.65740049816194\\
5719.42985746437	10.8254763608973\\
5731.43285821455	30.1888570800971\\
5743.43585896474	5.65597785515388\\
5755.43885971493	13.0911650252262\\
5767.44186046512	60.5732464744981\\
5779.44486121531	1.28231770768923\\
5791.44786196549	57.6842278233911\\
5803.45086271568	50.3569321950084\\
5815.45386346587	5.69340948044277\\
5827.45686421606	43.0076702560934\\
5839.45986496624	8.13803880885359\\
5851.46286571643	11.6404668767656\\
5863.46586646662	13.8098543538531\\
5875.4688672168	6.67149736441343\\
5887.47186796699	29.2698266131986\\
5899.47486871718	2.95863721689365\\
5911.47786946737	59.4056965767703\\
5923.48087021755	27.4170009961406\\
5935.48387096774	11.368088479409\\
5947.48687171793	74.1375411993724\\
5959.48987246812	21.8991364476025\\
5971.49287321831	2.66362088427602\\
5983.49587396849	40.1752023428816\\
5995.49887471868	16.900538713925\\
6007.50187546887	1.49172079112891\\
6019.50487621906	9.90862457746423\\
6031.50787696924	27.4531218958989\\
6043.51087771943	4.97504019202686\\
6055.51387846962	12.031949104348\\
6067.51687921981	55.0595699303192\\
6079.51987996999	1.29611891195771\\
6091.52288072018	51.0845702382737\\
6103.52588147037	45.638871032569\\
6115.52888222055	5.05338379768906\\
6127.53188297074	38.3810661910616\\
6139.53488372093	7.26695210060846\\
6151.53788447112	10.4613232203492\\
6163.54088522131	12.3171194631059\\
6175.54388597149	6.0245590988615\\
6187.54688672168	26.6579930651469\\
6199.54988747187	2.72110768452207\\
6211.55288822206	53.3945532086613\\
6223.55588897225	25.2866660843155\\
6235.55888972243	9.93890963426721\\
6247.56189047262	66.6940787930981\\
6259.56489122281	19.571834158611\\
6271.56789197299	2.43007521860051\\
6283.57089272318	36.0327590612566\\
6295.57389347337	15.1184479089334\\
6307.57689422356	1.35114456867576\\
6319.57989497374	9.11340991513856\\
6331.58289572393	25.1039686900756\\
6343.58589647412	4.41328342368152\\
6355.58889722431	11.1058261274821\\
6367.59189797449	50.3201619375366\\
6379.59489872468	1.29783666130236\\
6391.59789947487	45.5919080921512\\
6403.60090022506	41.6028301417696\\
6415.60390097525	4.5193194664925\\
6427.60690172543	34.4975489440523\\
6439.60990247562	6.53661046618703\\
6451.61290322581	9.46699956392599\\
6463.615903976	11.0671083359634\\
6475.61890472618	5.47436389511204\\
6487.62190547637	24.4109959790258\\
6499.62490622656	2.51366926114736\\
6511.62790697674	48.3009901510565\\
6523.63090772693	23.4177444437896\\
6535.63390847712	8.76089085790107\\
6547.63690922731	60.374601610078\\
6559.63990997749	17.6106995703221\\
6571.64291072768	2.22989676107426\\
6583.64591147787	32.5288022049794\\
6595.64891222806	13.6150257818357\\
6607.65191297825	1.23080004725877\\
6619.65491372843	8.41881942197229\\
6631.65791447862	23.0702713152696\\
6643.66091522881	3.94451988957803\\
6655.663915979	10.2913022069657\\
6667.66691672918	46.2145267579827\\
6679.66991747937	1.29060997731466\\
6691.67291822956	40.9718182701865\\
6703.67591897974	38.1212953430676\\
6715.67891972993	4.06902065802805\\
6727.68192048012	31.2051632332807\\
6739.68492123031	5.91792735994056\\
6751.6879219805	8.62002364833599\\
6763.69092273068	10.0094906595119\\
6775.69392348087	5.00224014064382\\
6787.69692423106	22.4624109609567\\
6799.69992498125	2.33135772337033\\
6811.70292573144	43.9459719990825\\
6823.70592648162	21.7686184713161\\
6835.70892723181	7.77949965451671\\
6847.711927982	54.9625730733317\\
6859.71492873218	15.9427391765994\\
6871.71792948237	2.05674659118262\\
6883.72093023256	29.5381777321923\\
6895.72393098275	12.3350496785895\\
6907.72693173293	1.12695243691875\\
6919.72993248312	7.80829858638301\\
6931.73293323331	21.2969111326189\\
6943.7359339835	3.54936163652332\\
6955.73893473368	9.57109952421623\\
6967.74193548387	42.6331623855054\\
6979.74493623406	1.27686731037289\\
6991.74793698425	37.0488586601582\\
7003.75093773444	35.0957133068565\\
7015.75393848462	3.68582555685313\\
7027.75693923481	28.389123890467\\
7039.759939985	5.38902254502026\\
7051.76294073519	7.89210314544679\\
7063.76594148537	9.10641329518312\\
7075.76894223556	4.59387337883016\\
7087.77194298575	20.7606240663824\\
7099.77494373593	2.17021713183345\\
7111.77794448612	40.19241860302\\
7123.78094523631	20.3058062890609\\
7135.7839459865	6.9540657803599\\
7147.78694673668	50.2915053120012\\
7159.78994748687	14.512312931803\\
7171.79294823706	1.90577476715338\\
7183.79594898725	26.9650708172305\\
7195.79894973744	11.2363485256034\\
7207.80195048762	1.03669798481464\\
7219.80495123781	7.26863794270656\\
7231.807951988	19.7405613314059\\
7243.81095273819	3.21321216845607\\
7255.81395348837	8.93119206405166\\
7267.81695423856	39.4895936416092\\
7279.81995498875	1.25847319873764\\
7291.82295573893	33.6896100505343\\
7303.82595648912	32.4488402362112\\
7315.82895723931	3.35702738921659\\
7327.8319579895	25.9613732342507\\
7339.83495873968	4.93317030138183\\
7351.83795948987	7.26152912580151\\
7363.84096024006	8.32895346462512\\
7375.84396099025	4.23813009400237\\
7387.84696174044	19.2649637608024\\
7399.84996249063	2.0270580169693\\
7411.85296324081	36.9338748237518\\
7423.855963991	19.0020882299273\\
7435.85896474119	6.25379353645891\\
7447.86196549137	46.2316350482036\\
7459.86496624156	13.2764040230814\\
7471.86796699175	1.77321039198088\\
7483.87096774194	24.7351054904582\\
7495.87396849212	10.2862664450611\\
7507.87696924231	0.957750307313603\\
7519.8799699925	6.78917744664109\\
7531.88297074269	18.3667313875806\\
7543.88597149287	2.9249159953522\\
7555.88897224306	8.36009170562334\\
7567.89197299325	36.7147238804772\\
7579.89497374344	1.23685215540934\\
7591.89797449363	30.7911467247662\\
7603.90097524381	30.1193739665561\\
7615.903975994	3.07279762140173\\
7627.90697674419	23.8534175291787\\
7639.90997749438	4.5374003950473\\
7651.91297824456	6.71140237600855\\
7663.91597899475	7.65469658996569\\
7675.91897974494	3.92623814233735\\
7687.92198049512	17.9429806177207\\
7699.92498124531	1.89928212398643\\
7711.9279819955	34.0866102702207\\
7723.93098274569	17.8351276570317\\
7735.93398349587	5.65504272232776\\
7747.93698424606	42.6805752993445\\
7759.93998499625	12.2013285614241\\
7771.94298574644	1.65607929503982\\
7783.94598649663	22.7898230936409\\
7795.94898724681	9.45919629447507\\
7807.951987997	0.888288432943393\\
7819.95498874719	6.36122727769481\\
7831.95798949738	17.1476537049733\\
7843.96099024756	2.67583128251868\\
7855.96399099775	7.84831232473033\\
7867.96699174794	34.252757345919\\
7879.96999249813	1.21308842490749\\
7891.97299324831	28.2730365265706\\
7903.9759939985	28.0581160829486\\
7915.97899474869	2.8254366322111\\
7927.98199549887	22.0113139039779\\
7939.98499624906	4.19152435137344\\
7951.98799699925	6.22839654169145\\
7963.99099774944	7.06604869104166\\
7975.99399849963	3.65120489898941\\
7987.99699924981	16.7684965138504\\
8000	1.78475314728964\\
8012.00300075019	31.584007998332\\
8024.00600150038	16.7864398205003\\
8036.00900225056	5.13943921811965\\
8048.01200300075	39.5566390929346\\
8060.01500375094	11.260404871086\\
8072.01800450113	1.5520055290425\\
8084.02100525131	21.0827539513545\\
8096.0240060015	8.73482846512477\\
8108.02700675169	0.826846940072763\\
8120.03000750187	5.97763920757433\\
8132.03300825206	16.0607456720818\\
8144.03600900225	2.45918090981356\\
8156.03900975244	7.38796248029374\\
8168.04201050263	32.0582073933804\\
8180.04501125282	1.18800435719496\\
8192.048012003	26.0716873024843\\
8204.05101275319	26.2251810646473\\
8216.05401350338	2.60884271303994\\
8228.05701425357	20.3920987515462\\
8240.06001500375	3.88744511532192\\
8252.06301575394	5.80187944859705\\
8264.06601650413	6.54903914161574\\
8276.06901725431	3.40739713292501\\
8288.0720180045	15.720181152945\\
8300.07501875469	1.6816998985045\\
8312.07801950488	29.3725120891351\\
8324.08102025506	15.8406101991226\\
8336.08402100525	4.6925393749086\\
8348.08702175544	36.7939944094915\\
8360.09002250563	10.4322751623511\\
8372.09302325582	1.45906916015822\\
8384.096024006	19.5765771438512\\
8396.09902475619	8.09688727814807\\
8408.10202550638	0.772235331753052\\
8420.10502625657	5.63248470919833\\
8432.10802700675	15.0874726996203\\
8444.11102775694	2.26959053306984\\
8456.11402850713	6.97243231319121\\
8468.11702925732	30.0936692211558\\
8480.1200300075	1.1622211819647\\
8492.12303075769	24.1362848135049\\
8504.12603150788	24.5879356128718\\
8516.12903225806	2.41812955455879\\
8528.13203300825	18.9612016011519\\
8540.13503375844	3.61865938952978\\
8552.13803450863	5.42327830235321\\
8564.14103525882	6.09245728643897\\
8576.14403600901	3.19023297276568\\
8588.14703675919	14.7804970956976\\
8600.15003750938	1.58864276438881\\
8612.15303825957	27.408656576927\\
8624.15603900975	14.984694580119\\
8636.15903975994	4.30287105450845\\
8648.16204051013	34.3390965386084\\
8660.16504126032	9.69967881000783\\
8672.1680420105	1.37570264239481\\
8684.17104276069	18.2410366857278\\
8696.17404351088	7.53220572248343\\
8708.17704426107	0.723478054314633\\
8720.18004501125	5.32081002369729\\
8732.18304576144	14.2124956849652\\
8744.18604651163	2.10275461203443\\
8756.18904726182	6.59615036963005\\
8768.19204801201	28.3281397673203\\
8780.19504876219	1.13620581714166\\
8792.19804951238	22.4258298740057\\
8804.20105026257	23.1194559643317\\
8816.20405101276	2.24934655931987\\
8828.20705176294	17.6905448265874\\
8840.21005251313	3.37989339599282\\
8852.21305326332	5.08561402367667\\
8864.2160540135	5.68722047642389\\
8876.21905476369	2.99595288326684\\
8888.22205551388	13.9349068709624\\
8900.22505626407	1.50433715994491\\
8912.22805701425	25.6568570698043\\
8924.23105776444	14.207753445958\\
8936.23405851463	3.96123522412978\\
8948.23705926482	32.1480259731849\\
8960.24006001501	9.04854342871795\\
8972.24306076519	1.30061413239159\\
8984.24606151538	17.0513920729935\\
8996.24906226557	7.03003878963905\\
9008.25206301576	0.679769313425613\\
9020.25506376594	5.03844755937238\\
9032.25806451613	13.4230234868014\\
9044.26106526632	1.95519150517634\\
9056.26406601651	6.25439293553792\\
9068.26706676669	26.735735084318\\
9080.27006751688	1.11030680851302\\
9092.27306826707	20.9069473880501\\
9104.27606901725	21.7973583891906\\
9116.27906976744	2.09927153760633\\
9128.28207051763	16.5571284690735\\
9140.28507126782	3.16683258119639\\
9152.28807201801	4.78315486697574\\
9164.29107276819	5.32590515254279\\
9176.29407351838	2.82144719955701\\
9188.29707426857	13.1712694556795\\
9200.30007501876	1.42772956267356\\
9212.30307576894	24.0877485561938\\
9224.30607651913	13.5004869134109\\
9236.30907726932	3.66019061979303\\
9248.31207801951	30.1844777205765\\
9260.31507876969	8.46730296521593\\
9272.31807951988	1.23272993555879\\
9284.32108027007	15.9872519502628\\
9296.32408102026	6.58154756818216\\
9308.32708177044	0.640438644101288\\
9320.33008252063	4.78186909236582\\
9332.33308327082	12.7083153280726\\
9344.33608402101	1.82406152598221\\
9356.33908477119	5.9431332333548\\
9368.34208552138	25.2947003855434\\
9380.34508627157	1.08478189819522\\
9392.34808702176	19.5522456760874\\
9404.35108777195	20.6029023705505\\
9416.35408852213	1.9652551268956\\
9428.35708927232	15.5419633126653\\
9440.36009002251	2.97591848170653\\
9452.36309077269	4.51115546128421\\
9464.36609152288	5.00239453990566\\
9476.36909227307	2.66412472270509\\
9488.37209302326	12.4793755645662\\
9500.37509377344	1.35792298210178\\
9512.37809452363	22.676919821342\\
9524.38109527382	12.854945844889\\
9536.38409602401	3.39366888695897\\
9548.3870967742	28.4182255826062\\
9560.39009752438	7.94638020552779\\
9572.39309827457	1.17115075077075\\
9584.39609902476	15.0316862032875\\
9596.39909977495	6.1794072191478\\
9608.40210052513	0.604924396453939\\
9620.40510127532	4.54807037557007\\
9632.40810202551	12.0592943127242\\
9644.4111027757	1.70703014594346\\
9656.41410352588	5.65892117970216\\
9668.41710427607	23.9866382739983\\
9680.42010502626	1.05981917942691\\
9692.42310577644	18.3390735832179\\
9704.42610652663	19.5202956107281\\
9716.42910727682	1.8451026838031\\
9728.43210802701	14.6292573895648\\
9740.4351087772	2.80419427782325\\
9752.43810952738	4.26565789320827\\
9764.44111027757	4.71161089289855\\
9776.44411102776	2.52181151591619\\
9788.44711177795	11.8505860331685\\
9800.45011252813	1.29414958866571\\
9812.45311327832	21.4039395714591\\
9824.45611402851	12.2643012885626\\
9836.4591147787	3.15668394904821\\
9848.46211552888	26.8239370627903\\
9860.46511627907	7.47778987853249\\
9872.46811702926	1.11511800968398\\
9884.47111777945	14.1705429322993\\
9896.47411852963	5.81750595309967\\
9908.47711927982	0.572753116416909\\
9920.48012003001	4.33447961389115\\
9932.4831207802	11.4682442873506\\
9944.48612153038	1.6021639982652\\
9956.48912228057	5.39878678983418\\
9968.49212303076	22.7959014807759\\
9980.49512378095	1.03555334806929\\
9992.49812453114	17.248568822742\\
10004.5011252813	18.5361502050229\\
10016.5041260315	1.73698367001442\\
10028.5071267817	13.8057892253016\\
10040.5101275319	2.64918610047156\\
10052.5131282821	4.04333843683615\\
10064.5161290323	4.44930982824631\\
10076.5191297824	2.39267218010717\\
10088.5221305326	11.2775476880775\\
10100.5251312828	1.2357488337372\\
10112.528132033	20.2515996489629\\
10124.5311327832	11.7226590231938\\
10136.5341335334	2.94511025825276\\
10148.5371342836	25.3802500986649\\
10160.5401350338	7.05483125621973\\
10172.5431357839	1.06398769935902\\
10184.5461365341	13.3919179382363\\
10196.5491372843	5.49071163361071\\
10208.5521380345	0.543523364993922\\
10220.5551387847	4.13888426579406\\
10232.5581395349	10.9285699246028\\
10244.5611402851	1.5078510036023\\
10256.5641410353	5.1601620347808\\
10268.5671417854	21.7091109462574\\
10280.5701425356	1.01207820639276\\
10292.5731432858	16.2649220771046\\
10304.576144036	17.6390532802812\\
10316.5791447862	1.63936144817076\\
10328.5821455364	13.0604202872839\\
10340.5851462866	2.50881090646106\\
10352.5881470368	3.84138827247486\\
10364.5911477869	4.21192078857792\\
10376.5941485371	2.27514804953577\\
10388.5971492873	10.7539681044872\\
10400.6001500375	1.18214982273183\\
10412.6031507877	19.2053215840897\\
10424.6061515379	11.2249093003899\\
10436.6091522881	2.7555119745507\\
10448.6121530383	24.0690473818048\\
10460.6151537884	6.67184789599781\\
10472.6181545386	1.01720979955884\\
10484.6211552888	12.6857389582511\\
10496.624156039	5.19468917880472\\
10508.6271567892	0.51689291298316\\
10520.6301575394	3.95937205149428\\
10532.6331582896	10.4346052533311\\
10544.6361590398	1.42273843659026\\
10556.6391597899	4.94081721269131\\
10568.6421605401	20.7147703324351\\
10580.6451612903	0.989456299707413\\
10592.6481620405	15.3748026957892\\
10604.6511627907	16.8192251814459\\
10616.6541635409	1.55093839401802\\
10628.6571642911	12.3837123142193\\
10640.6601650413	2.38130431484689\\
10652.6631657915	3.65741979631255\\
10664.6661665416	3.99642215460033\\
10676.6691672918	2.1679082531834\\
10688.672168042	10.2744355710393\\
10700.6751687922	1.13285701193649\\
10712.6781695424	18.2526872838889\\
10724.6811702926	10.7666042833148\\
10736.6841710428	2.58501019560243\\
10748.6871717929	22.8748812709642\\
10760.6901725431	6.32403826423473\\
10772.6931732933	0.97431198044128\\
10784.6961740435	12.0434371020069\\
10796.6991747937	4.92575650189662\\
10808.7021755439	0.492568527177528\\
10820.7051762941	3.79428307145671\\
10832.7081770443	9.9814596708434\\
10844.7111777945	1.34568447622693\\
10856.7141785446	4.73880882079277\\
10868.7171792948	19.802955391193\\
10880.720180045	0.967726355774079\\
10892.7231807952	14.5669066620496\\
10904.7261815454	16.0682452565344\\
10916.7291822956	1.47061261912259\\
10928.7321830458	11.7676244577235\\
10940.735183796	2.2651635944785\\
10952.7381845461	3.48939239606928\\
10964.7411852963	3.80024264652831\\
10976.7441860465	2.06981065414727\\
10988.7471867967	9.83427408788618\\
11000.7501875469	1.08743852487078\\
11012.7531882971	17.3830649664124\\
11024.7561890473	10.3438574489363\\
11036.7591897975	2.43117889710576\\
11048.7621905476	21.7845145371758\\
11060.7651912978	6.00730528251615\\
11072.768192048	0.934886566927098\\
11084.7711927982	11.4576851697049\\
11096.7741935484	4.680769963866\\
11108.7771942986	0.470297763578925\\
11120.7801950488	3.64217068558818\\
11132.783195799	9.56489320716965\\
11144.7861965491	1.27571998988174\\
11156.7891972993	4.55243660421735\\
11168.7921980495	18.9650619265155\\
11180.7951987997	0.946909036829519\\
11192.7981995499	13.8315979517692\\
11204.8012003001	15.3788302841789\\
11216.8042010503	1.39744357766142\\
11228.8072018005	11.2052717225795\\
11240.8102025506	2.15910226158809\\
11252.8132033008	3.33555317690058\\
11264.816204051	3.62118286128944\\
11276.8192048012	1.97987044050849\\
11288.8222055514	9.42942574869832\\
11300.8252063016	1.04551654895449\\
11312.8282070518	16.5873088076746\\
11324.831207802	9.9532605350209\\
11336.8342085521	2.29196273557642\\
11348.8372093023	20.786550968264\\
11360.8402100525	5.71813591068831\\
11372.8432108027	0.898580032096649\\
11384.8462115529	10.9221877144058\\
11396.8492123031	4.45703262301913\\
11408.8522130533	0.449862327785018\\
11420.8552138035	3.50176935302562\\
11432.8582145536	9.18121480847202\\
11444.8612153038	1.21201815546858\\
11456.864216054	4.38020797480077\\
11468.8672168042	18.1935999824012\\
11480.8702175544	0.927011392677102\\
11492.8732183046	13.1606218603003\\
11504.8762190548	14.7446542356764\\
11516.879219805	1.33062453123909\\
11528.8822205551	10.6907308991366\\
11540.8852213053	2.06201365323912\\
11552.8882220555	3.19438927195615\\
11564.8912228057	3.45735236822139\\
11576.8942235559	1.89723469268241\\
11588.8972243061	9.05635470402064\\
11600.9002250563	1.00675939601974\\
11612.9032258065	15.8575161166441\\
11624.9062265566	9.5918145938358\\
11636.9092273068	2.16561164265262\\
11648.912228057	19.8711362455314\\
11660.9152288072	5.45350410107408\\
11672.9182295574	0.865084466154094\\
11684.9212303076	10.4315114630441\\
11696.9242310578	4.25222023762138\\
11708.927231808	0.431072668638229\\
11720.9302325581	3.37196804091296\\
11732.9332333083	8.82719887108341\\
11744.9362340585	1.15387013847017\\
11756.9392348087	4.22080838461371\\
11768.9422355589	17.4820247640477\\
11780.9452363091	0.90803031157382\\
11792.9482370593	12.546874250302\\
11804.9512378095	14.1602007384502\\
11816.9542385596	1.26946035104635\\
11828.9572393098	10.2188835852279\\
11840.96024006	1.97294149772935\\
11852.9632408102	3.06458920541284\\
11864.9662415604	3.30711892713847\\
11876.9692423106	1.82116165564924\\
11888.9722430608	8.7119682600255\\
11900.975243811	0.970874903037497\\
11912.9782445611	15.1868297462719\\
11924.9812453113	9.25687246413499\\
11936.9842460615	2.05062842070116\\
11948.9872468117	19.0297141925291\\
11960.9902475619	5.21079207775793\\
11972.9932483121	0.834130601893021\\
11984.9962490623	9.98094745653082\\
11996.9992498125	4.06432120045806\\
12009.0022505626	0.413763549266981\\
12021.0052513128	3.25178811958717\\
12033.008252063	8.50001635161782\\
12045.0112528132	1.10066548386559\\
12057.0142535634	4.0730765398182\\
12069.0172543136	16.824596948275\\
12081.0202550638	0.889955195771654\\
12093.023255814	11.9842145899912\\
12105.0262565641	13.6206415955683\\
12117.0292573143	1.21334950641033\\
12129.0322580645	9.78528839804437\\
12141.0352588147	1.89105598259247\\
12153.0382595649	2.94501138424857\\
12165.0412603151	3.16906722394127\\
12177.0442610653	1.75100374207308\\
12189.0472618155	8.39355167919082\\
12201.0502625656	0.93760491844247\\
12213.0532633158	14.5692763387216\\
12225.056264066	8.94609053658704\\
12237.0592648162	1.94572648191427\\
12249.0622655664	18.2548269603574\\
12261.0652663166	4.98772608892279\\
12273.0682670668	0.805482075808114\\
12285.071267817	9.56639829768525\\
12297.0742685671	3.89158748522373\\
12309.0772693173	0.397790398138639\\
12321.0802700675	3.14036489344743\\
12333.0832708177	8.19717759349041\\
12345.0862715679	1.05187620761744\\
12357.0892723181	3.93598356903332\\
12369.0922730683	16.2162666609181\\
12381.0952738185	0.872770036182498\\
12393.0982745686	11.4673135335081\\
12405.1012753188	13.1217362037133\\
12417.104276069	1.16176936050176\\
12429.1072768192	9.38607632228132\\
12441.1102775694	1.81563417303098\\
12453.1132783196	2.83465824571001\\
12465.1162790698	3.04196513380073\\
12477.11927982	1.68619351453044\\
12489.1222805701	8.09871401081912\\
12501.1252813203	0.906720673638814\\
12513.1282820705	13.9996331466891\\
12525.1312828207	8.65738812749883\\
12537.1342835709	1.84979555809804\\
12549.1372843211	17.5399503128911\\
12561.1402850713	4.7823236682443\\
12573.1432858215	0.778930677755735\\
12585.1462865716	9.18428540768037\\
12597.1492873218	3.73249435589822\\
12609.152288072	0.383026286531872\\
12621.1552888222	3.03693209594802\\
12633.1582895724	7.91648463332535\\
12645.1612903226	1.0070438114075\\
12657.1642910728	3.80861544035959\\
12669.167291823	15.6525766279965\\
12681.1702925731	0.856455020019052\\
12693.1732933233	10.9915279367732\\
12705.1762940735	12.6597478422185\\
12717.1792948237	1.11426409657209\\
12729.1822955739	9.01786451945496\\
12741.1852963241	1.74604389738133\\
12753.1882970743	2.73265492270181\\
12765.1912978245	2.92473597878889\\
12777.1942985746	1.62623206185633\\
12789.1972993248	7.8253428552318\\
12801.200300075	0.878018879957263\\
12813.2033008252	13.473317791531\\
12825.2063015754	8.38891311339501\\
12837.2093023256	1.76187371630276\\
12849.2123030758	16.8793571254539\\
12861.215303826	4.59285010751028\\
12873.2183045761	0.754292396598404\\
12885.2213053263	8.83147233076237\\
12897.2243060765	3.58570709417693\\
12909.2273068267	0.369359412165249\\
12921.2303075769	2.94080881428977\\
12933.2333083271	7.65599122124136\\
12945.2363090773	0.965768622876104\\
12957.2393098275	3.69015805987685\\
12969.2423105776	15.1295809488297\\
12981.2453113278	0.840987775128825\\
12993.248312078	10.5527978067243\\
13005.2513128282	12.2313736580303\\
13017.2543135784	1.07043475025498\\
13029.2573143286	8.67768496183325\\
13041.2603150788	1.68173041318302\\
13053.263315829	2.63823154168507\\
13065.2663165791	2.81643559038131\\
13077.2693173293	1.57067931153421\\
13089.2723180795	7.57156640654185\\
13101.2753188297	0.851318423273559\\
13113.2783195799	12.9862965431461\\
13125.2813203301	8.13901274321493\\
13137.2843210803	1.68112439520474\\
13149.2873218305	16.2680036982382\\
13161.2903225806	4.41778234644079\\
13173.2933233308	0.731404111268398\\
13185.296324081	8.50520098653555\\
13197.2993248312	3.45005338167329\\
13209.3023255814	0.356690994164336\\
13221.3053263316	2.85138841640437\\
13233.3083270818	7.41396915379732\\
13245.311327832	0.927700997801595\\
13257.3143285821	3.57988459389685\\
13269.3173293323	14.6437766647252\\
13281.3203300825	0.826344330928038\\
13293.3233308327	10.1475608927696\\
13305.3263315829	11.8336858364779\\
13317.3293323331	1.02993093801276\\
13329.3323330833	8.36292504360456\\
13341.3353338335	1.62220531740538\\
13353.3383345836	2.55070845966408\\
13365.3413353338	2.71623324637121\\
13377.344336084	1.51914591661957\\
13389.3473368342	7.33572145777247\\
13401.3503375844	0.82645755355926\\
13413.3533383346	12.5350076416126\\
13425.3563390848	7.90620875367867\\
13437.359339835	1.60681746327307\\
13449.3623405851	15.7014346249984\\
13461.3653413353	4.2557788692154\\
13473.3683420855	0.710120808021456\\
13485.3713428357	8.20303843005952\\
13497.3743435859	3.32450026469682\\
13509.3773443361	0.344933504099161\\
13521.3803450863	2.76812913582634\\
13533.3833458365	7.18887980052933\\
13545.3863465866	0.892534021625443\\
13557.3893473368	3.47714464316921\\
13569.392348087	14.19204586724\\
13581.3953488372	0.81249985831455\\
13593.3983495874	9.77268155039703\\
13605.4013503376	11.4640819509697\\
13617.4043510878	0.9924439596904\\
13629.407351838	8.07127792212821\\
13641.4103525881	1.56703727790879\\
13653.4133533383	2.46948389216469\\
13665.4163540885	2.62339575070541\\
13677.4193548387	1.47128643012778\\
13689.4223555889	7.11632631712989\\
13701.4253563391	0.803291486184128\\
13713.4283570893	12.1162969029475\\
13725.4313578395	7.68917607468757\\
13737.4343585896	1.53831351714552\\
13749.4373593398	15.1757028287108\\
13761.44036009	4.10565449282423\\
13773.4433608402	0.690313229551113\\
13785.4463615904	7.92283218291037\\
13797.4493623406	3.20813485278604\\
13809.4523630908	0.334009173038392\\
13821.455363841	2.69054603575131\\
13833.4583645911	6.97934992354165\\
13845.4613653413	0.859997425757213\\
13857.4643660915	3.38135496677542\\
13869.4673668417	13.771606521989\\
13881.4703675919	0.799429236776484\\
13893.4733683421	9.42539121689932\\
13905.4763690923	11.1202428887058\\
13917.4793698425	0.957701020341982\\
13929.4823705927	7.80070080665241\\
13941.4853713428	1.51584425165573\\
13953.488372093	2.39402349819954\\
13965.4913728432	2.53727407653504\\
13977.4943735934	1.42679353807603\\
13989.4973743436	6.91205778791685\\
14001.5003750938	0.781690347197262\\
14013.503375844	11.7273634044787\\
14025.5063765942	7.48672454554944\\
14037.5093773443	1.47505080503109\\
14049.5123780945	14.6873020621052\\
14061.5153788447	3.96635915961286\\
14073.5183795949	0.67186588007285\\
14085.5213803451	7.66267259553092\\
14097.5243810953	3.10014807509361\\
14109.5273818455	0.32384872640304\\
14121.5303825957	2.61820412573044\\
14133.5333833458	6.78415106446271\\
14145.536384096	0.82985249287112\\
14157.5393848462	3.29199150723413\\
14169.5423855964	13.3799705405067\\
14181.5453863466	0.787107486595673\\
14193.5483870968	9.10323838234682\\
14205.551387847	10.8000970526022\\
14217.5543885972	0.925460368469005\\
14229.5573893473	7.54937977334386\\
14241.5603900975	1.4682869223965\\
14253.5633908477	2.32385157427104\\
14265.5663915979	2.45729211060549\\
14277.5693923481	1.38539316673834\\
14289.5723930983	6.72173152798558\\
14301.5753938485	0.761537407363424\\
14313.5783945987	11.3657134880368\\
14325.5813953488	7.29778316395229\\
14337.584396099	1.41653428722733\\
14349.5873968492	14.2331096984868\\
14361.5903975994	3.83696002323977\\
14373.5933983496	0.654675325671615\\
14385.5963990998	7.42086100159603\\
14397.59939985	2.99982095354973\\
14409.6024006002	0.314390307702318\\
14421.6054013503	2.55071244559659\\
14433.6084021005	6.60218190718264\\
14445.6114028507	0.801887772063775\\
14469.6174043511	13.0149079020149\\
14481.6204051013	0.775510095520686\\
14493.6234058515	8.80404636409461\\
14505.6264066017	10.5017897924765\\
14517.6294073518	0.895507188257302\\
14529.632408102	7.31569996392207\\
14541.6354088522	1.42406314405687\\
14553.6384096024	2.25854357872977\\
14565.6414103526	2.38293712908672\\
14577.6444111028	1.3468403157694\\
14589.647411853	6.54428523149197\\
14601.6504126032	0.742727559435058\\
14613.6534133533	11.0291216521535\\
14625.6564141035	7.12138647603517\\
14637.6594148537	1.36232644599567\\
14649.6624156039	13.8103380592998\\
14661.6654163541	3.71662625733672\\
14673.6684171043	0.638648740503103\\
14685.6714178545	7.19588266955787\\
14697.6744186047	2.90651295742426\\
14709.6774193548	0.305578559587764\\
14721.680420105	2.48771896424748\\
14733.6834208552	6.43245313549281\\
14745.6864216054	0.775915460013202\\
14769.6924231058	12.6744158462438\\
14781.695423856	0.764613263230436\\
14793.6984246062	8.52587752605412\\
14805.7014253563	10.223657208546\\
14817.7044261065	0.867650115212294\\
14829.7074268567	7.09822024582336\\
14841.7104276069	1.382903217024\\
14853.7134283571	2.19771976079337\\
14865.7164291073	2.31375170681758\\
14877.7194298575	1.31091549622841\\
14889.7224306077	6.37876417872761\\
14901.7254313578	0.725166001118802\\
14913.728432108	10.7155971811679\\
14925.7314328582	6.95666278100186\\
14937.7344336084	1.3120395334146\\
14949.7374343586	13.416492853386\\
14961.7404351088	3.60461612383042\\
14973.743435859	0.623702658863533\\
14985.7464366092	6.98638374217131\\
14997.7494373593	2.81965208604261\\
15009.7524381095	0.297363836484063\\
15021.7554388597	2.42890616781825\\
15033.7584396099	6.2740743889008\\
15045.7614403601	0.751768332624693\\
15069.7674418605	12.3566923370948\\
15081.7704426107	0.75439408165765\\
15093.7734433608	8.26700284270346\\
15105.776444111	9.96420362598962\\
15117.7794448612	0.841718269440613\\
15129.7824456114	6.89565158817691\\
15141.7854463616	1.34456585770686\\
15153.7884471118	2.14103971118676\\
15165.791447862	2.24932681834899\\
15177.7944486122	1.27742167544271\\
15189.7974493623	6.22430877929453\\
15201.8004501125	0.708767092787131\\
15213.8034508627	10.423355567722\\
15225.8064516129	6.80282388140933\\
15237.8094523631	1.26532900709205\\
15249.8124531133	13.0493375651879\\
15261.8154538635	3.50026592604704\\
15273.8184546137	0.609761900275317\\
15285.8214553638	6.7911515073607\\
15297.824456114	2.73872639386883\\
15309.8274568642	0.289701527702511\\
15321.8304576144	2.37398723291028\\
15333.8334583646	6.12624299222475\\
15345.8364591148	0.729297133705165\\
15357.839459865	2.99002986361215\\
15369.8424606152	12.0601131326271\\
15381.8454613653	0.744830665418755\\
15393.8484621155	8.02587591697315\\
15405.8514628657	9.72208216566092\\
15417.8544636159	0.817558720615267\\
15429.8574643661	6.70683854302333\\
15441.8604651163	1.30883474741188\\
15453.8634658665	2.08819768491816\\
15465.8664666167	2.18929593447276\\
15477.8694673668	1.24618164754267\\
15489.872468117	6.08014380199663\\
15501.8754688672	0.693453363921622\\
15513.8784696174	10.1507939606869\\
15525.8814703676	6.65915615205613\\
15537.8844711178	1.22188795105485\\
15549.887471868	12.7068628409146\\
15561.8904726182	3.40298053895231\\
15573.8934733683	0.596758640621718\\
15585.8964741185	6.60909746293096\\
15597.8994748687	2.66327672280962\\
15609.9024756189	0.282551473752425\\
15621.9054763691	2.32270269847948\\
15633.9084771193	5.98823418863049\\
15645.9114778695	0.708368344592474\\
15657.9144786197	2.92657487617458\\
15669.9174793698	11.7832119162913\\
15681.92048012	0.735902243790814\\
15693.9234808702	7.80111072290146\\
15705.9264816204	9.4960779351813\\
15717.9294823706	0.795034314452611\\
15729.9324831208	6.5307433336097\\
15741.935483871	1.27551556726909\\
15753.9384846212	2.03891857355363\\
15765.9414853713	2.13332995343303\\
15777.9444861215	1.21703576374009\\
15789.9474868717	5.94556903520025\\
15801.9504876219	0.679154646849209\\
15813.9534883721	9.89647000247771\\
15825.9564891223	6.52501274025295\\
15837.9594898725	1.18144231733753\\
15849.9624906227	12.3872600883572\\
15861.9654913728	3.31222526556372\\
15873.968492123	0.58463160715331\\
15885.9714928732	6.43924273162297\\
15897.9744936234	2.59289045025834\\
15909.9774943736	0.275877461456349\\
15921.9804951238	2.27481756392716\\
15933.983495874	5.85939265224556\\
15945.9864966242	0.68886227270349\\
15957.9894973743	2.86731725118745\\
15969.9924981245	11.5246630311637\\
15981.9954988747	0.727589223046481\\
15993.9984996249	7.59146247763186\\
16006.0015003751	9.28509344487219\\
16018.0045011253	0.774021803160952\\
16030.0075018755	6.36643214001717\\
16042.0105026257	1.24443344284654\\
16054.0135033758	1.99295442634847\\
16066.016504126	2.08113283516049\\
16078.0195048762	1.18983996704404\\
16090.0225056264	5.81995116617072\\
16102.0255063766	0.665807319558815\\
16114.0285071268	9.65908353527592\\
16126.031507877	6.3998067379012\\
16138.0345086272	1.14374685367643\\
16150.0375093773	12.0888986426857\\
16162.0405101275	3.22751881208381\\
16174.0435108777	0.573325378807729\\
16186.0465116279	6.28070546421958\\
16198.0495123781	2.52719609412522\\
16210.0525131283	0.269646786046352\\
16222.0555138785	2.23011875272844\\
16234.0585146287	5.73912509381341\\
16246.0615153788	0.670671408120499\\
16258.064516129	2.81199804434637\\
16270.0675168792	11.283266438774\\
16282.0705176294	0.719873226106161\\
16294.0735183796	7.3958111501778\\
16306.0765191298	9.08813592140451\\
16318.07951988	0.754410232276114\\
16330.0825206302	6.21306324269601\\
16342.0855213803	1.21543073505754\\
16354.0885221305	1.95008143631105\\
16366.0915228807	2.03243782936844\\
16378.0945236309	1.1644640860042\\
16390.0975243811	5.70271670222109\\
16402.1005251313	0.653353642482057\\
16414.1035258815	9.43746074008476\\
16426.1065266317	6.28300519167816\\
16438.1095273818	1.10858160731809\\
16450.112528132	11.8103059617073\\
16462.1155288822	3.14842721006839\\
16474.1185296324	0.562789776569888\\
16486.1215303826	6.13268992684658\\
16498.1245311328	2.46585864386858\\
16510.127531883	0.263829870303492\\
16522.1305326332	2.18841289084702\\
16534.1335333833	5.62689380273833\\
16546.1365341335	0.653699006255712\\
16558.1395348837	2.76038205926808\\
16570.1425356339	11.0579345841283\\
16582.1455363841	0.712737115238768\\
16594.1485371343	7.21314720295137\\
16606.1515378845	8.90430624390315\\
16618.1545386347	0.736099544787048\\
16630.1575393848	6.0698767436708\\
16642.160540135	1.18836512507377\\
16654.1635408852	1.91009732222008\\
16666.1665416354	1.98700420745802\\
16678.1695423856	1.14079034952358\\
16690.1725431358	5.5933457850886\\
16702.175543886	0.641741176292446\\
16714.1785446362	9.23054035065022\\
16726.1815453863	6.17412383850541\\
16738.1845461365	1.07574891392051\\
16750.1875468867	11.550150402195\\
16762.1905476369	3.07455854336453\\
16774.1935483871	0.552979330994214\\
16786.1965491373	5.99447702244245\\
16798.1995498875	2.40857550851267\\
16810.2025506377	0.258399932512669\\
16822.2055513878	2.14952435658346\\
16834.208552138	5.52221099490639\\
16846.2115528882	0.637857861958614\\
16858.2145536384	2.71225553456398\\
16870.2175543886	10.8476808991286\\
16882.2205551388	0.706165002115773\\
16894.223555889	7.04255922778816\\
16906.2265566392	8.73278927236999\\
16918.2295573893	0.718999369850247\\
16930.2325581395	5.9361856310122\\
16942.2355588897	1.16310794970484\\
16954.2385596399	1.87281904875788\\
16966.2415603901	1.944614423576\\
16978.2445611403	1.11871209082222\\
16990.2475618905	5.49136677436428\\
17002.2505626407	0.630922269962847\\
17014.2535633908	9.03736163763192\\
17026.256564141	6.07272247017327\\
17038.2595648912	1.04507079677741\\
17050.2625656414	11.3072262036791\\
17062.2655663916	3.00555836152661\\
17074.2685671418	0.543852816098958\\
17086.271567892	5.86541603650894\\
17098.2745686422	2.35507299122974\\
17110.2775693924	0.253332696280234\\
17122.2805701425	2.11329356568162\\
17134.2835708927	5.4246338542093\\
17146.2865716429	0.623069246171345\\
17158.2895723931	2.66742410650764\\
17170.2925731433	10.651609719063\\
17182.2955738935	0.700142248759494\\
17194.2985746437	6.88322319680359\\
17206.3015753938	8.57284537532688\\
17218.304576144	0.703027968907326\\
17230.3075768942	5.81136799201278\\
17242.3105776444	1.13954275066881\\
17254.3135783946	1.83808083605403\\
17266.3165791448	1.90507164168305\\
17278.319579895	1.09813261400496\\
17290.3225806452	5.39635149424624\\
17302.3255813954	0.620853609787503\\
17314.3285821455	8.85705391341536\\
17326.3315828957	5.97840084588993\\
17338.3345836459	1.01638671391135\\
17350.3375843961	11.0804403658644\\
17362.3405851463	2.94110567973436\\
17374.3435858965	0.535372840580023\\
17386.3465866467	5.74491743137696\\
17398.3495873968	2.30510321396659\\
17410.352588147	0.248606136377257\\
17422.3555888972	2.07957546081392\\
17434.3585896474	5.33376017578137\\
17446.3615903976	0.609261981306872\\
17458.3645911478	2.6257110113171\\
17470.367591898	10.4689074213489\\
17482.3705926482	0.694655462122788\\
17494.3735933984	6.73439309308765\\
17506.3765941485	8.42380299291473\\
17518.3795948987	0.688111316556273\\
17530.3825956489	5.69486021111478\\
17542.3855963991	1.11756400760929\\
17554.3885971493	1.8057324185626\\
17566.3915978995	1.86819757699569\\
17578.3945986497	1.07896420081053\\
17590.3975993999	5.30791105517785\\
17602.40060015	0.611495821524138\\
17614.4036009002	8.68882734318941\\
17626.4066016504	5.89079508238344\\
17638.4096024006	0.989551601268463\\
17650.4126031508	10.8688011559974\\
17662.415603901	2.88090948260056\\
17674.4186046512	0.527505488747392\\
17686.4216054014	5.63244654182045\\
17698.4246061515	2.25844142894988\\
17710.4276069017	0.244200255694661\\
17722.4306076519	2.04823817894263\\
17734.4336084021	5.24922453046595\\
17746.4366091523	0.59637163512018\\
17758.4396099025	2.58695549703342\\
17770.4426106527	10.2988346254222\\
17782.4456114029	0.689692484620319\\
17794.448612153	6.59539272556684\\
17806.4516129032	8.28505209665051\\
17818.4546136534	0.67418229697366\\
17830.4576144036	5.58615101423469\\
17842.4606151538	1.09707602879144\\
17854.463615904	1.77563751866485\\
17866.4666166542	1.83383060710358\\
17878.4696174044	1.06112723839552\\
17890.4726181545	5.22569217448887\\
17902.4756189047	0.602813118991802\\
17914.4786196549	8.53196488343717\\
17926.4816204051	5.80957446327964\\
17938.4846211553	0.964434168736622\\
17950.4876219055	10.6714080217142\\
17962.4906226557	2.82470566135351\\
17974.4936234059	0.520220004516776\\
17986.496624156	5.52751804766043\\
17998.4996249062	2.21488366281\\
18010.5026256564	0.240096889142646\\
18022.5056264066	2.01916187432361\\
18034.5086271568	5.17069488358244\\
18046.511627907	0.584339816330935\\
18058.5146286572	2.55101141880086\\
18070.5176294074	10.1407193153864\\
18082.5206301575	0.685242382294609\\
18094.5236309077	6.46560856133365\\
18106.5266316579	8.15603842923546\\
18118.5296324081	0.661179999899031\\
18130.5326331583	5.48477624478243\\
18142.5356339085	1.07799197821316\\
18154.5386346587	1.74767250618221\\
18166.5416354089	1.80182411606074\\
18178.544636159	1.04454945229982\\
18190.5476369092	5.1493739320627\\
18202.5506376594	0.59477299337748\\
18214.5536384096	8.38581519676832\\
18226.5566391598	5.73443861553345\\
18238.55963991	0.940915412492521\\
18250.5626406602	10.4874427232084\\
18262.5656414104	2.77225432525164\\
18274.5686421605	0.513488513176356\\
18286.5716429107	5.42969111777556\\
18298.5746436609	2.17424464831934\\
18310.5776444111	0.236279530949383\\
18322.5806451613	1.99223767776471\\
18334.5836459115	5.09786961006129\\
18346.5866466617	0.573113557758948\\
18358.5896474119	2.51774599560863\\
18370.592648162	9.99395076878678\\
18382.5956489122	0.681295432032719\\
18394.5986496624	6.34448343660934\\
18406.6016504126	8.0362584236408\\
18418.6046511628	0.649049102565254\\
18430.607651913	5.39031427111923\\
18442.6106526632	1.06023302065373\\
18454.6136534134	1.72172521948913\\
18466.6166541635	1.77204504006764\\
18478.6196549137	1.02916523083546\\
18490.6226556639	5.07866490632169\\
18502.6256564141	0.587345938299072\\
18514.6286571643	8.24978641381491\\
18526.6316579145	5.66511501005927\\
18538.6346586647	0.918887313253485\\
18550.6376594149	10.3161615221758\\
18562.640660165	2.72333743697707\\
18574.6436609152	0.507285776068312\\
18586.6466616654	5.33856513664459\\
18598.6496624156	2.13635600508435\\
18610.6526631658	0.232733182349985\\
18622.655663916	1.96736677588799\\
18634.6586646662	5.03047485755866\\
18646.6616654164	0.562644775079932\\
18658.6646661665	2.48703870905329\\
18670.6676669167	9.85797419167816\\
18682.6706676669	0.677843109041144\\
18694.6736684171	6.2315110278539\\
18706.6766691673	7.92525471658445\\
18718.6796699175	0.637739326061943\\
18730.6826706677	5.30238194400624\\
18742.6856714179	1.04372756915495\\
18754.688672168	1.69769392710346\\
18766.6916729182	1.74437258785146\\
18778.6946736684	1.0149150292276\\
18790.6976744186	5.01330064343355\\
18802.7006751688	0.580505206519567\\
18814.703675919	8.12334063226933\\
18826.7066766692	5.60135674896967\\
18838.7096774194	0.8982516945072\\
18850.7126781695	10.1568882918226\\
18862.7156789197	2.6777567294394\\
18874.7186796699	0.50158897425213\\
18886.7216804201	5.25377593758476\\
18898.7246811703	2.10106463654419\\
18910.7276819205	0.229444217088839\\
18922.7306826707	1.94445959616804\\
18934.7336834209	4.96826221431961\\
18946.736684171	0.552889791002036\\
18958.7396849212	2.4587803278316\\
18970.7426856714	9.7322859724916\\
18982.7456864216	0.674878075477864\\
18994.7486871718	6.12623098263453\\
19006.751687922	7.8226121829896\\
19018.7546886722	0.627204956373845\\
19030.7576894224	5.22063103146351\\
19042.7606901725	1.02841062178801\\
19054.7636909227	1.67548641242161\\
19066.7666916729	1.71869711348751\\
19078.7696924231	1.00174484373787\\
19090.7726931733	4.9530414200228\\
19102.7756939235	0.574226594530475\\
19114.7786946737	8.00598906109515\\
19126.7816954239	5.5429406063306\\
19138.784696174	0.878919218978203\\
19150.7876969242	10.0090084323169\\
19162.7906976744	2.6353318676504\\
19174.7936984246	0.496377517704824\\
19186.7966991748	5.17499247802354\\
19198.799699925	2.06823131545183\\
19210.8027006752	0.226400262547404\\
19222.8057014254	1.92343508553313\\
19234.8087021755	4.91100664709302\\
19246.8117029257	0.543808916252823\\
19258.8147036759	2.43287204374209\\
19270.8177044261	9.61642948377279\\
19282.8207051763	0.672394170939991\\
19294.8237059265	6.02822462458745\\
19306.8267066767	7.72795442906242\\
19318.8297074269	0.61740442175772\\
19330.832708177	5.1447450713367\\
19342.8357089272	1.01422317637611\\
19354.8387096774	1.65501916600572\\
19366.8417104276	1.69491912214144\\
19378.8447111778	0.989605747135098\\
19390.847711928	4.89767026464689\\
19402.8507126782	0.568488252056471\\
19414.8537134284	7.89728772830713\\
19426.8567141785	5.48966529543469\\
19438.8597149287	0.860808504730991\\
19450.8627156789	9.87196349022195\\
19462.8657164291	2.59589882465737\\
19474.8687171793	0.491632877163824\\
19486.8717179295	5.10191390117173\\
19498.8747186797	2.03772943407388\\
19510.8777194299	0.223590094622424\\
19522.88072018	1.9042200721909\\
19534.8837209302	4.85850467667542\\
19546.8867216804	0.535366080059402\\
19558.8897224306	2.40922470705287\\
19570.8927231808	9.5099913653246\\
19582.895723931	0.670386405565303\\
19594.8987246812	5.93711115968295\\
19606.9017254314	7.64094069095437\\
19618.9047261815	0.608299919342938\\
19630.9077269317	5.07443658928982\\
19642.9107276819	1.0011117136231\\
19654.9137284321	1.63621667293952\\
19666.9167291823	1.67294839234469\\
19678.9197299325	0.978453478319791\\
19690.9227306827	4.84699120867225\\
19702.9257314329	0.563270513625248\\
19714.928732183	7.79683368468031\\
19726.9317329332	5.44134993788144\\
19738.9347336834	0.843845345076874\\
19750.9377344336	9.745246398792\\
19762.9407351838	2.55930844495378\\
19774.943735934	0.487338436051973\\
19786.9467366842	5.03426693730449\\
19798.9497374344	2.00944389868789\\
19810.9527381845	0.22100354468386\\
19822.9557389347	1.8867487017145\\
19834.9587396849	4.81057276538142\\
19846.9617404351	0.527528503919264\\
19858.9647411853	2.3877581506376\\
19870.9677419355	9.41259823748342\\
19882.9707426857	0.668850956159725\\
19894.9737434359	5.85254432171867\\
19906.976744186	7.56126309096223\\
19918.9797449362	0.599857084856885\\
19930.9827456864	5.00944463853419\\
19942.9857464366	0.989027740361141\\
19954.9887471868	1.61901078380793\\
19966.991747937	1.65270320082466\\
19978.9947486872	0.968248079850643\\
19990.9977494374	4.80082774126993\\
20003.0007501875	0.558555750009746\\
20015.0037509377	7.70426164507499\\
20027.0067516879	5.39783271433694\\
20039.0097524381	0.82796201893958\\
20051.0127531883	9.62839726354941\\
20063.0157539385	2.52542517267864\\
20075.0187546887	0.483479360353565\\
20087.0217554389	4.97180360378746\\
20099.0247561891	1.98327015103004\\
20111.0277569392	0.218631417310282\\
20123.0307576894	1.87096193934684\\
20135.0337584396	4.76704589331893\\
20147.0367591898	0.520266413183663\\
20159.03975994	2.36840059375576\\
20171.0427606902	9.32391379498035\\
20183.0457614404	0.667785165920205\\
20195.0487621905	5.7742094026297\\
20207.0517629407	7.48864421404258\\
20219.0547636909	0.592044700304613\\
20231.0577644411	4.94953262214849\\
20243.0607651913	0.977927385868648\\
20255.0637659415	1.60334016004164\\
20267.0667666917	1.63410963749165\\
20279.0697674419	0.95895357799182\\
20291.0727681921	4.75902144610151\\
20303.0757689422	0.554328237499473\\
20315.0787696924	7.61924101590352\\
20327.0817704426	5.35896967839115\\
20339.0847711928	0.813096680017778\\
20351.087771943	9.52099963209569\\
20363.0907726932	2.4941259249901\\
20375.0937734434	0.480042484626108\\
20387.0967741935	4.91429916894131\\
20399.0997749437	1.95911330151596\\
20411.1027756939	0.216465417524755\\
20423.1057764441	1.85680713224469\\
20435.1087771943	4.72777630329213\\
20447.1117779445	0.513552782003965\\
20459.1147786947	2.35108811758047\\
20471.1177794449	9.24363624244331\\
20483.1207801951	0.667187548096823\\
20495.1237809452	5.70182062271232\\
20507.1267816954	7.42283496845099\\
20519.1297824456	0.584834435035487\\
20531.1327831958	4.89448636632487\\
20543.135783946	0.967771045127333\\
20555.1387846962	1.58914978526584\\
20567.1417854464	1.61710100063043\\
20579.1447861965	0.950537700681892\\
20591.1477869467	4.72143080167763\\
20603.1507876969	0.550574043264813\\
20615.1537884471	7.54147326598871\\
20627.1567891973	5.324633718755\\
20639.1597899475	0.799192814919013\\
20651.1627906977	9.42267719192665\\
20663.1657914479	2.46529909377103\\
20675.1687921981	0.477016212521047\\
20687.1717929482	4.86155034950508\\
20699.1747936984	1.93688736145785\\
20711.1777944486	0.214498086533157\\
20723.1807951988	1.84423762547936\\
20735.183795949	4.69263239756124\\
20747.1867966992	0.507363107570372\\
20759.1897974494	2.33576420594253\\
20771.1927981996	9.17149603653661\\
20783.1957989497	0.667057794061394\\
20795.1987996999	5.6351188005537\\
20807.2018004501	7.36361270185087\\
20819.2048012003	0.578200616388963\\
20831.2078019505	4.84411241544257\\
20843.2108027007	0.958523063945631\\
20855.2138034509	1.57639053585198\\
20867.2168042011	1.6016172630836\\
20879.2198049512	0.942971629495915\\
20891.2228057014	4.68793012842136\\
20903.2258064516	0.547280925353212\\
20915.2288072018	7.47068960339705\\
20927.231807952	5.29471365652673\\
20939.2348087022	0.786198761734028\\
20951.2378094524	9.33309085098236\\
20963.2408102026	2.43884366122617\\
20975.2438109527	0.474390430501859\\
20987.2468117029	4.81337371631222\\
20999.2498124531	1.91651456322037\\
21011.2528132033	0.212722745091947\\
21023.2558139535	1.83321242719892\\
21035.2588147037	4.66149777214939\\
21047.2618154539	0.50167521038886\\
21059.2648162041	2.32237934522041\\
21071.2678169542	9.1072539042477\\
21083.2708177044	0.667396786051688\\
21095.2738184546	5.57386928891357\\
21107.2768192048	7.31077954737495\\
21119.279819955	0.572120026653276\\
21131.2828207052	4.79823652558167\\
21143.2858214554	0.950151461266578\\
21155.2888222056	1.56501880451338\\
21167.2918229557	1.58760460179161\\
21179.2948237059	0.936229782108346\\
21191.2978244561	4.65840866891492\\
21203.3008252063	0.544438246087061\\
21215.3038259565	7.40664892492858\\
21227.3068267067	5.2691134659376\\
21239.3098274569	0.774067281694585\\
21251.3128282071	9.25193615986741\\
21263.3158289572	2.41466841680092\\
21275.3188297074	0.472156433568832\\
21287.3218304576	4.76960428546071\\
21299.3248312078	1.89792475840672\\
21311.327831958	0.211133443710906\\
21323.3308327082	1.82369591850424\\
21335.3338334584	4.6342703754727\\
21347.3368342086	0.496469057557959\\
21359.3398349587	2.31089067857565\\
21371.3428357089	9.05069911219917\\
21383.3458364591	0.668206615118309\\
21395.3488372093	5.51786014677483\\
21407.3518379595	7.26416097834436\\
21419.3548387097	0.566571723264196\\
21431.3578394599	4.75670233460506\\
21443.3608402101	0.942627685010802\\
21455.3638409602	1.5549961717302\\
21467.3668417104	1.57501498430189\\
21479.3698424606	0.930289622457675\\
21491.3728432108	4.6327697896592\\
21503.375843961	0.542036897721715\\
21515.3788447112	7.34913601231912\\
21527.3818454614	5.24775160930034\\
21539.3848462116	0.762755177694261\\
21551.3878469617	9.17894104100378\\
21563.3908477119	2.3926912646057\\
21575.3938484621	0.470306862106646\\
21587.3968492123	4.73009427576077\\
21599.3998499625	1.88105488621274\\
21611.4028507127	0.209724919048419\\
21623.4058514629	1.8156576045868\\
21635.4088522131	4.61086178124345\\
21647.4118529632	0.491726606607471\\
21659.4148537134	2.30126171030915\\
21671.4178544636	9.0016479641731\\
21683.4208552138	0.669490604535826\\
21695.423855964	5.46690052310475\\
21707.4268567142	7.22360455228986\\
21719.4298574644	0.561536880045163\\
21731.4328582146	4.71937019140586\\
21743.4358589647	0.935926397870953\\
21755.4388597149	1.54628912095651\\
21767.4418604651	1.56380580635476\\
21779.4448612153	0.925131496039956\\
21791.4478619655	4.61093029357894\\
21803.4508627157	0.540069239516798\\
21815.4538634659	7.29795994950883\\
21827.4568642161	5.23056047820717\\
21839.4598649662	0.752222953992032\\
21851.4628657164	9.11386379438407\\
21863.4658664666	2.372838612052\\
21875.4688672168	0.46883564891242\\
21887.471867967	4.69471201615124\\
21899.4748687172	1.86584850441439\\
21911.4778694674	0.208492555938663\\
21923.4808702176	1.80907190384604\\
21935.4838709677	4.59119656599723\\
21947.4868717179	0.487431667719346\\
21959.4898724681	2.29346205671972\\
21971.4928732183	8.95994250915033\\
21983.4958739685	0.671253339100615\\
21995.4988747187	5.4208192291446\\
22007.5018754689	7.18897882867053\\
22019.5048762191	0.556998647106487\\
22031.5078769692	4.68611612869519\\
22043.5108777194	0.930025290517272\\
22055.5138784696	1.53886879331985\\
22067.5168792198	1.55393957585676\\
22079.51987997	0.920738488178679\\
22091.5228807202	4.5928198351394\\
22103.5258814704	0.538529045430468\\
22115.5288822206	7.25295274111143\\
22127.5318829707	5.21748593395345\\
22139.5348837209	0.742434512517674\\
22151.5378844711	9.05649135547653\\
22163.5408852213	2.35504483175573\\
22175.5438859715	0.46773797587298\\
22187.5468867217	4.66334098855466\\
22199.5498874719	1.85225537712819\\
22211.5528882221	0.207432354558275\\
22223.5558889722	1.80391797285953\\
22235.5588897224	4.57521178406601\\
22247.5618904726	0.483569782432542\\
22259.5648912228	2.28746724050437\\
22271.567891973	8.92544944273526\\
22283.5708927232	0.673500700850453\\
22295.5738934734	5.37946348168915\\
22307.5768942236	7.16017244688154\\
22319.5798949737	0.552942027725126\\
22331.5828957239	4.65683096597142\\
22343.5858964741	0.924904919524819\\
22355.5888972243	1.53271077921393\\
22367.5918979745	1.54538363897074\\
22379.5948987247	0.917096303563125\\
22391.5978994749	4.57838043087074\\
22403.6009002251	0.537411461809897\\
22415.6039009752	7.21396811495811\\
22427.6069017254	5.20848694229355\\
22439.6099024756	0.733356881573263\\
22451.6129032258	9.00663778084349\\
22463.615903976	2.33925178945991\\
22475.6189047262	0.467010239611327\\
22487.6219054764	4.63587899383571\\
22499.6249062266	1.84023111379881\\
22511.6279069767	0.206540902337946\\
22523.6309077269	1.80017956457287\\
22535.6339084771	4.56285653316172\\
22547.6369092273	0.480128117270493\\
22559.6399099775	2.28325852647781\\
22571.6429107277	8.89805918953195\\
22583.6459114779	0.676239911580905\\
22595.6489122281	5.3426978000989\\
22607.6519129782	7.13709335334251\\
22619.6549137284	0.549353770556082\\
22631.6579144786	4.63141953088617\\
22643.6609152288	0.920548568070621\\
22655.663915979	1.52779494364335\\
22667.6669167292	1.5381099449351\\
22679.6699174794	0.914193165509036\\
22691.6729182296	4.56756605930105\\
22703.6759189797	0.536712974616118\\
22715.6789197299	7.18088049306083\\
22727.6819204801	5.2035352974747\\
22739.6849212303	0.724959973353701\\
22751.6879219805	8.96414294575418\\
22763.6909227307	2.32540843242592\\
22775.6939234809	0.466650025797165\\
22787.6969242311	4.61223743089765\\
22799.6999249812	1.82973685511217\\
22811.7029257314	0.205815350241107\\
22823.7059264816	1.79784491863389\\
22835.7089272318	4.55409160574676\\
22847.711927982	0.477095370885502\\
22859.7149287322	2.28082279641397\\
22871.7179294824	8.87768515506912\\
22883.7209302326	0.679479582750625\\
22895.7239309827	5.31040304342957\\
22907.7269317329	7.11966816797817\\
22919.7299324831	0.54622227577468\\
22931.7329332333	4.60979999002062\\
22943.7359339835	0.916942127650869\\
22955.7389347337	1.52410528352873\\
22967.7419354839	1.53209484653869\\
22979.7449362341	0.912019733656818\\
22991.7479369842	4.56034234505274\\
23003.7509377344	0.536431385733559\\
23015.7539384846	7.15358411826059\\
23027.7569392348	5.20261543337145\\
23039.759939985	0.717216367122924\\
23051.7629407352	8.92887143466661\\
23063.7659414854	2.31347043285197\\
23075.7689422356	0.466656091698789\\
23087.7719429857	4.59234067956093\\
23099.7749437359	1.82073900174782\\
23111.7779444861	0.205253393193517\\
23123.7809452363	1.79690668201585\\
23135.7839459865	4.5488892215142\\
23147.7869467367	0.474461693539366\\
23159.7899474869	2.28015246172863\\
23171.7929482371	8.86426313831166\\
23183.7959489872	0.683229773369541\\
23195.7989497374	5.28247557602143\\
23207.8019504876	7.10784168265674\\
23219.8049512378	0.543537514100889\\
23231.807951988	4.59190328007797\\
23243.8109527382	0.914073999305256\\
23255.8139534884	1.52162981488226\\
23267.8169542386	1.52731893376704\\
23279.8199549888	0.910569039128882\\
23291.8229557389	4.55668632413186\\
23303.8259564891	0.5365657981727\\
23315.8289572393	7.13199232677951\\
23327.8319579895	5.20572431842944\\
23339.8349587397	0.710101115292892\\
23351.8379594899	8.90071161335376\\
23363.8409602401	2.30339988230193\\
23375.8439609902	0.467028356796072\\
23387.8469617404	4.57612557966219\\
23399.8499624906	1.8132089825977\\
23411.8529632408	0.204853254319888\\
23423.855963991	1.79736185948528\\
23435.8589647412	4.54723283848289\\
23447.8619654914	0.472218617960026\\
23459.8649662416	2.28124541309189\\
23471.8679669918	8.85775089843864\\
23483.8709677419	0.687502056513389\\
23495.8739684921	5.25882655066865\\
23507.8769692423	7.10157648584391\\
23519.8799699925	0.541290957608532\\
23531.8829707427	4.57767263323894\\
23543.8859714929	0.911935013161448\\
23555.8889722431	1.52036048865927\\
23567.8919729932	1.52376689867349\\
23579.8949737434	0.909836436461999\\
23591.8979744936	4.55658628719235\\
23603.9009752438	0.537116609921279\\
23615.903975994	7.11603695737465\\
23627.9069767442	5.21287143434495\\
23639.9099774944	0.703591570138942\\
23651.9129782446	8.87957487109989\\
23663.9159789948	2.29516503302075\\
23675.9189797449	0.467767901362945\\
23687.9219804951	4.56354100071574\\
23699.9249812453	1.8071230601318\\
23711.9279819955	0.204613672915126\\
23723.9309827457	1.79921179298907\\
23735.9339834959	4.54911704000567\\
23747.9369842461	0.470359000629063\\
23759.9399849963	2.28410500618647\\
23771.9429857464	8.85812786993753\\
23783.9459864966	0.692309595258901\\
23795.9489872468	5.23938130109819\\
23807.951987997	7.10085270872724\\
23819.9549887472	0.539475521527492\\
23831.9579894974	4.56706319039352\\
23843.9609902476	0.910518365361075\\
23855.9639909978	1.52029313424053\\
23867.9669917479	1.52142742968465\\
23879.9699924981	0.909819571553502\\
23891.9729932483	4.5600416984223\\
23903.9759939985	0.538085516395813\\
23915.9789947487	7.10566788994202\\
23927.9819954989	5.22407883693436\\
23939.9849962491	0.697667228854273\\
23951.9879969993	8.86539502362064\\
23963.9909977494	2.28874008361329\\
23975.9939984996	0.468876972909134\\
23987.9969992498	4.55454749635279\\
24000	1.80246216990471\\
};
\addlegendentry{Théorique}

\addplot [color=mycolor2]
  table[row sep=crcr]{%
-24000	6.72459160945331e-09\\
-23953.0791788856	8.92836719464468e-09\\
-23906.1583577713	1.3049156754818e-08\\
-23859.2375366569	1.19509994658612e-08\\
-23812.3167155425	1.35905497002912e-08\\
-23765.3958944282	1.01517810066037e-08\\
-23718.4750733138	7.43105876110364e-09\\
-23671.5542521994	7.62272750704966e-09\\
-23624.633431085	1.10422237183575e-08\\
-23577.7126099707	1.29777028874937e-08\\
-23530.7917888563	1.26913339495327e-08\\
-23483.8709677419	1.17334277267771e-08\\
-23390.0293255132	6.53725601833143e-09\\
-23343.1085043988	9.96740803528657e-09\\
-23296.1876832845	1.28389178265501e-08\\
-23249.2668621701	1.22429945515344e-08\\
-23202.3460410557	1.31739030866858e-08\\
-23155.4252199413	1.00764348561627e-08\\
-23108.504398827	6.58244301516118e-09\\
-23061.5835777126	8.92344156059278e-09\\
-23014.6627565982	1.15420855904474e-08\\
-22967.7419354839	1.28793919011637e-08\\
-22920.8211143695	1.31722920525289e-08\\
-22873.9002932551	1.0950072622494e-08\\
-22826.9794721408	8.20975002342248e-09\\
-22780.0586510264	6.8105043280485e-09\\
-22733.137829912	1.06755839048111e-08\\
-22686.2170087977	1.31756149123542e-08\\
-22639.2961876833	1.23768219559701e-08\\
-22592.3753665689	1.2893259610642e-08\\
-22545.4545454545	9.03841594477153e-09\\
-22498.5337243402	6.24534179658844e-09\\
-22451.6129032258	9.88129097747078e-09\\
-22404.6920821114	1.22098897745271e-08\\
-22357.7712609971	1.2993891475796e-08\\
-22310.8504398827	1.34733799070114e-08\\
-22263.9296187683	1.03053463067894e-08\\
-22217.008797654	7.3647085114522e-09\\
-22170.0879765396	7.73465804140546e-09\\
-22123.1671554252	1.15068960299761e-08\\
-22076.2463343109	1.36873922403323e-08\\
-22029.3255131965	1.22167099213214e-08\\
-21982.4046920821	1.27767313434303e-08\\
-21935.4838709677	8.05095575662957e-09\\
-21888.5630498534	6.8246436517229e-09\\
-21841.642228739	1.02249518923728e-08\\
-21794.7214076246	1.31151543904637e-08\\
-21747.8005865103	1.2342741854579e-08\\
-21700.8797653959	1.37833185242507e-08\\
-21653.9589442815	9.35235984173321e-09\\
-21607.0381231672	7.50954600398191e-09\\
-21560.1173020528	8.48606173172853e-09\\
-21513.1964809384	1.21999717549949e-08\\
-21466.275659824	1.3288677235466e-08\\
-21419.3548387097	1.30173314472507e-08\\
-21372.4340175953	1.19137943054648e-08\\
-21325.5131964809	8.22493070885173e-09\\
-21278.5923753666	7.01002042911499e-09\\
-21231.6715542522	1.14870218532946e-08\\
-21184.7507331378	1.30595523613011e-08\\
-21137.8299120235	1.27410817430401e-08\\
-21090.9090909091	1.34092023921666e-08\\
-21043.9882697947	8.96830233341534e-09\\
-20997.0674486804	7.18701520350616e-09\\
-20950.146627566	9.59908043506664e-09\\
-20903.2258064516	1.26128892891273e-08\\
-20856.3049853372	1.3278766101814e-08\\
-20809.3841642229	1.35652056356108e-08\\
-20762.4633431085	1.06482827473578e-08\\
-20715.5425219941	8.06122428603486e-09\\
-20668.6217008798	7.67025592245308e-09\\
-20621.7008797654	1.25251214769233e-08\\
-20574.780058651	1.33245711578557e-08\\
-20527.8592375367	1.33101750108813e-08\\
-20480.9384164223	1.29449271367019e-08\\
-20434.0175953079	8.73881580663949e-09\\
-20387.0967741935	6.84009101182613e-09\\
-20340.1759530792	1.13237892791318e-08\\
-20293.2551319648	1.28661537181289e-08\\
-20199.4134897361	1.38964054438722e-08\\
-20152.4926686217	1.00058130978622e-08\\
-20105.5718475073	7.56420023056494e-09\\
-20058.651026393	8.77239660578129e-09\\
-20011.7302052786	1.24193096046869e-08\\
-19964.8093841642	1.43663210678154e-08\\
-19917.8885630499	1.30795848535725e-08\\
-19870.9677419355	1.31434278839209e-08\\
-19824.0469208211	7.93178115729517e-09\\
-19777.1260997067	7.55057840817098e-09\\
-19730.2052785924	1.18400248150701e-08\\
-19683.284457478	1.38585251912532e-08\\
-19636.3636363636	1.35284962740432e-08\\
-19589.4428152493	1.47607201712728e-08\\
-19542.5219941349	8.76835437262436e-09\\
-19495.6011730205	7.93966121281652e-09\\
-19448.6803519062	9.68305888305488e-09\\
-19401.7595307918	1.34378421193683e-08\\
-19354.8387096774	1.43418612024481e-08\\
-19307.917888563	1.38311658549279e-08\\
-19260.9970674487	1.16565540506434e-08\\
-19214.0762463343	8.1717217835041e-09\\
-19167.1554252199	7.88552328371312e-09\\
-19120.2346041056	1.33734386753805e-08\\
-19073.3137829912	1.39631814261642e-08\\
-19026.3929618768	1.4030815982392e-08\\
-18979.4721407625	1.38977234430283e-08\\
-18932.5513196481	8.65267152026991e-09\\
-18885.6304985337	8.13682682418547e-09\\
-18838.7096774194	1.16759249407779e-08\\
-18791.788856305	1.3755463248541e-08\\
-18744.8680351906	1.46667012672409e-08\\
-18697.9472140762	1.44232037069632e-08\\
-18651.0263929619	1.05865362281966e-08\\
-18604.1055718475	8.50162413565557e-09\\
-18557.1847507331	8.89528832953516e-09\\
-18510.2639296188	1.44629036800832e-08\\
-18463.3431085044	1.39439705751788e-08\\
-18416.42228739	1.47771395374907e-08\\
-18369.5014662757	1.32574605698092e-08\\
-18322.5806451613	8.50890630782731e-09\\
-18275.6598240469	8.13478040062613e-09\\
-18228.7390029326	1.30786811039899e-08\\
-18134.8973607038	1.52955974822771e-08\\
-18087.9765395894	1.47525805405675e-08\\
-18041.0557184751	1.01178866726466e-08\\
-17994.1348973607	8.29117137557902e-09\\
-17947.2140762463	1.06524768816904e-08\\
-17900.293255132	1.46406864932785e-08\\
-17853.3724340176	1.52922293555899e-08\\
-17806.4516129032	1.49902446650466e-08\\
-17759.5307917889	1.30277928127461e-08\\
-17712.6099706745	8.15502559430557e-09\\
-17665.6891495601	9.2588846644044e-09\\
-17618.7683284457	1.390864171079e-08\\
-17571.8475073314	1.49902004163874e-08\\
-17478.0058651026	1.53566530987637e-08\\
-17431.0850439883	9.05412152093414e-09\\
-17384.1642228739	9.16522545637339e-09\\
-17337.2434017595	1.19406447503694e-08\\
-17290.3225806452	1.50561101779554e-08\\
-17243.4017595308	1.59598506972243e-08\\
-17196.4809384164	1.55269531852034e-08\\
-17149.5601173021	1.22886759348377e-08\\
-17102.6392961877	8.83909795249362e-09\\
-17055.7184750733	1.01360639530724e-08\\
-17008.7976539589	1.53767082534307e-08\\
-16961.8768328446	1.53390932911658e-08\\
-16914.9560117302	1.62983478419797e-08\\
-16868.0351906158	1.49840554200729e-08\\
-16821.1143695015	8.52610079359033e-09\\
-16774.1935483871	9.72002419711035e-09\\
-16727.2727272727	1.37235211261782e-08\\
-16680.3519061584	1.55198600816918e-08\\
-16633.431085044	1.68272748785585e-08\\
-16586.5102639296	1.60956781048847e-08\\
-16539.5894428152	1.07690040156423e-08\\
-16492.6686217009	9.38500346840294e-09\\
-16445.7478005865	1.1004591442907e-08\\
-16398.8269794721	1.72805891251989e-08\\
-16351.9061583578	1.59511289467411e-08\\
-16304.9853372434	1.73393728780846e-08\\
-16258.064516129	1.39816847646556e-08\\
-16211.1436950147	8.90148832366228e-09\\
-16164.2228739003	1.01898655702551e-08\\
-16117.3020527859	1.63261697812564e-08\\
-16070.3812316716	1.56617941915495e-08\\
-16023.4604105572	1.8332832866193e-08\\
-15976.5395894428	1.56679300397253e-08\\
-15929.6187683284	1.04644181585086e-08\\
-15882.6979472141	9.79671187944399e-09\\
-15835.7771260997	1.32177165090101e-08\\
-15788.8563049853	1.72230981874331e-08\\
-15741.935483871	1.73773366140364e-08\\
-15695.0146627566	1.70709615221768e-08\\
-15648.0938416422	1.4257582752522e-08\\
-15601.1730205279	8.66762910231275e-09\\
-15507.3313782991	1.70347484110997e-08\\
-15460.4105571848	1.707319308849e-08\\
-15413.4897360704	1.85575804255382e-08\\
-15366.568914956	1.65113486217119e-08\\
-15319.6480938416	9.93645052377353e-09\\
-15272.7272727273	1.12849489937022e-08\\
-15225.8064516129	1.4984683703795e-08\\
-15178.8856304985	1.78881480817967e-08\\
-15131.9648093842	1.86614615436661e-08\\
-15085.0439882698	1.77743567457713e-08\\
-15038.1231671554	1.2781095454956e-08\\
-14991.2023460411	1.00253318626072e-08\\
-14944.2815249267	1.33833516623024e-08\\
-14897.3607038123	1.85765532670647e-08\\
-14850.4398826979	1.78269904989518e-08\\
-14803.5190615836	1.9458133098522e-08\\
-14756.5982404692	1.61089638395266e-08\\
-14709.6774193548	9.75890067375817e-09\\
-14662.7565982405	1.25200708970751e-08\\
-14615.8357771261	1.77090017587421e-08\\
-14568.9149560117	1.84157202794357e-08\\
-14521.9941348974	2.0241620646665e-08\\
-14475.073313783	1.83600831306051e-08\\
-14428.1524926686	1.19363893036419e-08\\
-14381.2316715543	1.1585834441759e-08\\
-14334.3108504399	1.47250689856245e-08\\
-14287.3900293255	2.04452271751914e-08\\
-14240.4692082111	1.88269497298503e-08\\
-14193.5483870968	2.08864393832378e-08\\
-14146.6275659824	1.51363217230634e-08\\
-14099.706744868	1.04327112779137e-08\\
-14052.7859237537	1.33262446363618e-08\\
-14005.8651026393	1.99925854252358e-08\\
-13958.9442815249	1.88335076916587e-08\\
-13912.0234604106	2.24484183730574e-08\\
-13865.1026392962	1.83294915175785e-08\\
-13818.1818181818	1.1766154785629e-08\\
-13771.2609970674	1.24235010611659e-08\\
-13724.3401759531	1.73689417176735e-08\\
-13677.4193548387	2.11642268043844e-08\\
-13630.4985337243	2.17679491075695e-08\\
-13583.57771261	2.07298211827104e-08\\
-13536.6568914956	1.56049557373779e-08\\
-13489.7360703812	1.06002994683745e-08\\
-13442.8152492669	1.59862106850929e-08\\
-13395.8944281525	2.14341720180757e-08\\
-13348.9736070381	2.08116629271361e-08\\
-13302.0527859238	2.34060452649697e-08\\
-13255.1319648094	1.79780867257279e-08\\
-13208.211143695	1.13553650831213e-08\\
-13161.2903225806	1.50167368038937e-08\\
-13114.3695014663	1.95545702653738e-08\\
-13067.4486803519	2.23036732924281e-08\\
-13020.5278592375	2.3762947263123e-08\\
-12973.6070381232	2.06710659277273e-08\\
-12926.6862170088	1.52893237472395e-08\\
-12879.7653958944	1.20428122846699e-08\\
-12832.8445747801	1.93315227728374e-08\\
-12785.9237536657	2.32806463824869e-08\\
-12739.0029325513	2.25958066433214e-08\\
-12692.082111437	2.46566817734511e-08\\
-12645.1612903226	1.77431854760689e-08\\
-12598.2404692082	1.20339130035009e-08\\
-12551.3196480938	1.7588467653766e-08\\
-12504.3988269795	2.21488794116757e-08\\
-12457.4780058651	2.36671096451824e-08\\
-12410.5571847507	2.57889479178976e-08\\
-12363.6363636364	2.11747616712365e-08\\
-12316.715542522	1.41331476617247e-08\\
-12269.7947214076	1.50761718554769e-08\\
-12222.8739002933	2.07425780026891e-08\\
-12175.9530791789	2.58888868101423e-08\\
-12129.0322580645	2.42676843728448e-08\\
-12082.1114369501	2.68798529532165e-08\\
-12035.1906158358	1.70430997081101e-08\\
-11988.2697947214	1.36903089546154e-08\\
-11941.348973607	1.94007632470975e-08\\
-11894.4281524927	2.62258106982109e-08\\
-11847.5073313783	2.44363880599171e-08\\
-11800.5865102639	2.95456602803354e-08\\
-11753.6656891496	2.05998745153835e-08\\
-11706.7448680352	1.52048025553725e-08\\
-11659.8240469208	1.70861622359914e-08\\
-11612.9032258065	2.39480506013422e-08\\
-11565.9824046921	2.70272620346509e-08\\
-11519.0615835777	2.83430762094155e-08\\
-11472.1407624633	2.58682991474471e-08\\
-11425.219941349	1.82472107826131e-08\\
-11378.2991202346	1.48104835921893e-08\\
-11331.3782991202	2.34149717930602e-08\\
-11284.4574780059	2.76841071983939e-08\\
-11237.5366568915	2.77195022795462e-08\\
-11190.6158357771	3.11438633804255e-08\\
-11096.7741935484	1.50354360233953e-08\\
-11049.853372434	2.18738298966408e-08\\
-11002.9325513196	2.70060114999204e-08\\
-10956.0117302053	2.90499715550711e-08\\
-10909.0909090909	3.27659962777514e-08\\
-10862.1700879765	2.57086002860029e-08\\
-10815.2492668622	1.86887439851619e-08\\
-10768.3284457478	1.73262483955033e-08\\
-10721.4076246334	2.68244372395248e-08\\
-10674.4868035191	3.100820401171e-08\\
-10627.5659824047	3.12073630348454e-08\\
-10580.6451612903	3.28801327243696e-08\\
-10533.724340176	2.14181411092523e-08\\
-10486.8035190616	1.56368141164547e-08\\
-10439.8826979472	2.67576240503435e-08\\
-10392.9618768328	3.03820349707035e-08\\
-10346.0410557185	3.29362130414526e-08\\
-10299.1202346041	3.68360390147591e-08\\
-10205.2785923754	1.84651518929345e-08\\
-10158.357771261	2.20762989564047e-08\\
-10111.4369501466	3.10798882332002e-08\\
-10064.5161290323	3.55042291720579e-08\\
-10017.5953079179	3.44605761322739e-08\\
-9970.67448680352	3.52888590806198e-08\\
-9923.75366568915	2.06784504503011e-08\\
-9876.83284457478	1.95063736594717e-08\\
-9829.91202346041	3.03251214600934e-08\\
-9782.99120234604	3.53777552188027e-08\\
-9736.07038123167	3.54514987437951e-08\\
-9689.1495601173	4.08448489143694e-08\\
-9642.22873900293	2.57168098671669e-08\\
-9595.30791788856	2.1162341136828e-08\\
-9548.38709677419	2.71261829405575e-08\\
-9501.46627565983	3.55649598824618e-08\\
-9407.62463343109	4.13359877315872e-08\\
-9360.70381231671	3.5449099853054e-08\\
-9313.78299120235	2.36162185655628e-08\\
-9266.86217008798	2.28862530867999e-08\\
-9219.94134897361	3.71053089400181e-08\\
-9173.02052785924	3.85605125407805e-08\\
-9079.1788856305	4.41044480333015e-08\\
-9032.25806451613	2.65019521480667e-08\\
-8985.33724340176	2.3529621482552e-08\\
-8938.41642228739	3.35080157688485e-08\\
-8891.49560117302	4.05422205334293e-08\\
-8844.57478005865	4.29967562604897e-08\\
-8797.65395894428	4.81387425834155e-08\\
-8750.73313782991	3.4453595149644e-08\\
-8703.81231671554	2.62274017285025e-08\\
-8656.89149560118	2.81161577861827e-08\\
-8609.97067448681	4.4752625182744e-08\\
-8563.04985337243	4.3518614852857e-08\\
-8516.12903225806	4.97199245947075e-08\\
-8469.2082111437	4.64658975369796e-08\\
-8422.28739002933	2.88235795635722e-08\\
-8375.36656891496	2.61273675032865e-08\\
-8328.44574780059	4.37016641981319e-08\\
-8281.52492668622	4.55419229436686e-08\\
-8187.68328445748	5.50862019546456e-08\\
-8140.76246334311	3.59066278171357e-08\\
-8093.84164222874	2.80091486426542e-08\\
-8046.92082111437	3.71580606777431e-08\\
-8000	4.80956951066521e-08\\
-7953.07917888563	5.44687288648422e-08\\
-7906.15835777126	5.53746118313853e-08\\
-7859.23753665689	5.24259401895369e-08\\
-7812.31671554252	2.9824188590169e-08\\
-7765.39589442815	3.30594999822832e-08\\
-7718.47507331378	5.15164531766192e-08\\
-7671.55425219941	5.43346774856836e-08\\
-7577.71260997068	6.55629111318339e-08\\
-7530.79178885631	3.43782589536742e-08\\
-7483.87096774194	3.47441883955817e-08\\
-7436.95014662757	4.72621312938441e-08\\
-7390.0293255132	5.74165190516476e-08\\
-7343.10850439883	6.37760358276781e-08\\
-7296.18768328446	6.69919005210527e-08\\
-7249.26686217009	5.17581066812041e-08\\
-7202.34604105572	3.54734991111502e-08\\
-7155.42521994135	4.1104968628892e-08\\
-7108.50439882698	6.50910095146688e-08\\
-7061.58357771261	6.29561482131516e-08\\
-7014.66275659824	7.20580107690728e-08\\
-6967.74193548387	6.98313985062328e-08\\
-6920.8211143695	3.88905310897258e-08\\
-6873.90029325513	4.24844962909054e-08\\
-6826.97947214076	6.49449824674205e-08\\
-6780.05865102639	6.62300711429117e-08\\
-6733.13782991202	7.68828248208949e-08\\
-6686.21700879766	8.2090740445558e-08\\
-6639.29618768329	5.15628152200105e-08\\
-6592.37536656891	4.54342085026546e-08\\
-6545.45454545455	5.38939596406419e-08\\
-6498.53372434018	7.98421651433195e-08\\
-6451.61290322581	7.41789708613946e-08\\
-6404.69208211144	9.02904678247699e-08\\
-6357.77126099707	7.60879296458272e-08\\
-6310.8504398827	4.4985131890539e-08\\
-6263.92961876833	5.16368379490816e-08\\
-6217.00879765396	8.37508220763634e-08\\
-6170.08797653959	7.87997528778623e-08\\
-6123.16715542522	9.82957273057877e-08\\
-6076.24633431085	9.6381955037312e-08\\
-6029.32551319648	5.82165429902528e-08\\
-5982.40469208211	5.34110380057105e-08\\
-5935.48387096774	7.73861365523169e-08\\
-5888.56304985337	9.26497727247709e-08\\
-5841.64222873901	9.94169437996452e-08\\
-5794.72140762463	1.11754926493343e-07\\
-5747.80058651026	8.41829357737322e-08\\
-5700.87976539589	5.39988331430337e-08\\
-5607.03812316716	1.02375576767333e-07\\
-5560.11730205279	1.01228274269448e-07\\
-5513.19648093842	1.21568444075382e-07\\
-5466.27565982405	1.17746073736911e-07\\
-5419.35483870968	6.33417549412273e-08\\
-5372.43401759531	7.30374448375361e-08\\
-5325.51319648094	1.06908550910572e-07\\
-5278.59237536657	1.10643798778061e-07\\
-5231.6715542522	1.32769016067971e-07\\
-5184.75073313783	1.41531751872368e-07\\
-5137.82991202346	9.44001813356721e-08\\
-5090.90909090909	7.36574441390556e-08\\
-5043.98826979472	1.03042863173696e-07\\
-4997.06744868035	1.34421036579449e-07\\
-4950.14662756598	1.31169798838916e-07\\
-4903.22580645161	1.67963337072995e-07\\
-4856.30498533724	1.38717728946187e-07\\
-4809.38416422287	7.20526562087229e-08\\
-4762.46334310851	1.06241170118152e-07\\
-4715.54252199414	1.46920854607024e-07\\
-4668.62170087976	1.4555696732084e-07\\
-4621.7008797654	1.81268823511103e-07\\
-4574.78005865103	1.86175734181491e-07\\
-4527.85923753666	1.00492857309306e-07\\
-4480.93841642229	1.04640348041572e-07\\
-4434.01759530792	1.4639865254799e-07\\
-4387.09677419355	1.87409468522682e-07\\
-4340.17595307918	1.81688608129274e-07\\
-4293.25513196481	2.35490672032221e-07\\
-4246.33431085044	1.61276029703081e-07\\
-4199.41348973607	1.03651212314322e-07\\
-4152.4926686217	1.52980808836841e-07\\
-4105.57184750733	2.22303142913495e-07\\
-4058.65102639296	1.90921285187444e-07\\
-4011.73020527859	2.79221193691151e-07\\
-3964.80938416422	2.35405031830063e-07\\
-3917.88856304986	1.30080471756143e-07\\
-3870.96774193548	1.60124068939395e-07\\
-3824.04692082111	2.3267476019259e-07\\
-3777.12609970674	2.42571385672e-07\\
-3730.20527859238	2.94426458674383e-07\\
-3683.28445747801	3.18529752453022e-07\\
-3636.36363636364	2.26734387722968e-07\\
-3589.44281524927	1.40573951362283e-07\\
-3542.5219941349	2.68926681891299e-07\\
-3495.60117302053	3.15752386716214e-07\\
-3448.68035190616	2.97471210213324e-07\\
-3401.75953079179	4.41035524614696e-07\\
-3354.83870967742	3.33653305152485e-07\\
-3307.91788856305	1.88939265451284e-07\\
-3260.99706744868	2.82947281468067e-07\\
-3214.07624633431	3.91186686765219e-07\\
-3167.15542521994	3.61150640006752e-07\\
-3120.23460410557	5.18809422432396e-07\\
-3073.3137829912	5.07357558769006e-07\\
-3026.39296187683	2.95177290437825e-07\\
-2979.47214076246	2.81915656881698e-07\\
-2932.55131964809	4.88117656888039e-07\\
-2885.63049853372	5.0398158428578e-07\\
-2838.70967741936	5.48631773401104e-07\\
-2791.78885630499	7.84545787676794e-07\\
-2744.86803519062	5.34628092721168e-07\\
-2697.94721407625	3.15202453980679e-07\\
-2651.02639296188	5.75123361486863e-07\\
-2604.10557184751	7.60032567458687e-07\\
-2557.18475073314	6.60660719702855e-07\\
-2510.26392961877	1.0959054675064e-06\\
-2463.3431085044	9.85726846392606e-07\\
-2416.42228739003	4.97985507342328e-07\\
-2369.50146627566	6.90868732312375e-07\\
-2322.58064516129	1.01502031262659e-06\\
-2275.65982404692	1.07227840015822e-06\\
-2228.73900293255	1.37384440758824e-06\\
-2181.81818181818	1.9087569359186e-06\\
-2134.89736070381	1.02462400054713e-06\\
-2087.97653958944	8.45618444044269e-07\\
-2041.05571847507	1.538834005776e-06\\
-1994.13489736071	1.85410554516359e-06\\
-1947.21407624634	2.06767909089684e-06\\
-1900.29325513196	3.51547533327655e-06\\
-1853.37243401759	2.93380082834933e-06\\
-1806.45161290323	1.4902130974484e-06\\
-1712.60997067449	3.95053304637126e-06\\
-1665.68914956012	4.01833931532816e-06\\
-1618.76832844575	8.23546362675232e-06\\
-1571.84750733138	1.00199470018175e-05\\
-1524.92668621701	5.55454114232047e-06\\
-1478.00586510264	5.33973947988479e-06\\
-1431.08504398827	1.11792245420725e-05\\
-1384.1642228739	1.89636756868754e-05\\
-1337.24340175953	4.89523723591833e-05\\
-1290.32258064516	0.000116020681843558\\
-1243.40175953079	0.000126807657383492\\
-1196.48093841642	0.000288249037008455\\
-1149.56011730205	0.000542807253474537\\
-1102.63929618768	0.000202862961523694\\
-1055.71847507331	0.000446044630113422\\
-1008.79765395894	0.000683265444009037\\
-961.876832844573	0.0015640526762047\\
-914.956011730206	0.000714604061124842\\
-868.035190615836	0.000170929441090224\\
-821.114369501465	0.000116708438030728\\
-774.193548387098	4.67480473192138e-05\\
-727.272727272728	1.91800949706186e-05\\
-680.351906158357	1.27404836532775e-05\\
-633.431085043987	8.09844238089178e-06\\
-586.51026392962	8.2193365896457e-06\\
-539.58944281525	1.26531473575974e-05\\
-492.668621700879	9.50803347374055e-06\\
-445.747800586509	5.47049797668588e-06\\
-398.826979472142	4.86403356449701e-06\\
-351.906158357771	4.48723415594088e-06\\
-304.985337243401	3.85128280991713e-06\\
-258.064516129034	5.06903456206398e-06\\
-211.143695014664	5.14448420991805e-06\\
-164.222873900293	3.54208115399247e-06\\
-117.302052785923	3.01226506775526e-06\\
-70.3812316715557	3.76447864723285e-06\\
-23.4604105571852	4.02636412860136e-06\\
23.4604105571852	3.06711535422603e-06\\
70.3812316715557	4.02636412860136e-06\\
117.302052785923	3.76447864723285e-06\\
164.222873900293	3.01226506775526e-06\\
211.143695014664	3.54208115399247e-06\\
258.064516129034	5.14448420991805e-06\\
304.985337243401	5.06903456206398e-06\\
351.906158357771	3.85128280991713e-06\\
398.826979472142	4.48723415594088e-06\\
445.747800586509	4.86403356449701e-06\\
492.668621700879	5.47049797668588e-06\\
539.58944281525	9.50803347374055e-06\\
586.51026392962	1.26531473575974e-05\\
633.431085043987	8.2193365896457e-06\\
680.351906158357	8.09844238089178e-06\\
727.272727272728	1.27404836532775e-05\\
774.193548387098	1.91800949706186e-05\\
821.114369501465	4.67480473192138e-05\\
868.035190615836	0.000116708438030728\\
914.956011730206	0.000170929441090224\\
961.876832844573	0.000714604061124842\\
1008.79765395894	0.0015640526762047\\
1055.71847507331	0.000683265444009037\\
1102.63929618768	0.000446044630113422\\
1149.56011730205	0.000202862961523694\\
1196.48093841642	0.000542807253474537\\
1243.40175953079	0.000288249037008455\\
1290.32258064516	0.000126807657383492\\
1337.24340175953	0.000116020681843558\\
1384.1642228739	4.89523723591833e-05\\
1431.08504398827	1.89636756868754e-05\\
1478.00586510264	1.11792245420725e-05\\
1524.92668621701	5.33973947988479e-06\\
1571.84750733138	5.55454114232047e-06\\
1618.76832844575	1.00199470018175e-05\\
1665.68914956012	8.23546362675232e-06\\
1712.60997067449	4.01833931532816e-06\\
1759.53079178886	3.95053304637126e-06\\
1853.37243401759	1.4902130974484e-06\\
1900.29325513196	2.93380082834933e-06\\
1947.21407624634	3.51547533327655e-06\\
1994.13489736071	2.06767909089684e-06\\
2041.05571847507	1.85410554516359e-06\\
2087.97653958944	1.538834005776e-06\\
2134.89736070381	8.45618444044269e-07\\
2181.81818181818	1.02462400054713e-06\\
2228.73900293255	1.9087569359186e-06\\
2275.65982404692	1.37384440758824e-06\\
2322.58064516129	1.07227840015822e-06\\
2369.50146627566	1.01502031262659e-06\\
2416.42228739003	6.90868732312375e-07\\
2463.3431085044	4.97985507342328e-07\\
2510.26392961877	9.85726846392606e-07\\
2557.18475073314	1.0959054675064e-06\\
2604.10557184751	6.60660719702855e-07\\
2651.02639296188	7.60032567458687e-07\\
2697.94721407625	5.75123361486863e-07\\
2744.86803519062	3.15202453980679e-07\\
2791.78885630499	5.34628092721168e-07\\
2838.70967741936	7.84545787676794e-07\\
2885.63049853372	5.48631773401104e-07\\
2932.55131964809	5.0398158428578e-07\\
2979.47214076246	4.88117656888039e-07\\
3026.39296187683	2.81915656881698e-07\\
3073.3137829912	2.95177290437825e-07\\
3120.23460410557	5.07357558769006e-07\\
3167.15542521994	5.18809422432396e-07\\
3214.07624633431	3.61150640006752e-07\\
3260.99706744868	3.91186686765219e-07\\
3307.91788856305	2.82947281468067e-07\\
3354.83870967742	1.88939265451284e-07\\
3401.75953079179	3.33653305152485e-07\\
3448.68035190616	4.41035524614696e-07\\
3495.60117302053	2.97471210213324e-07\\
3542.5219941349	3.15752386716214e-07\\
3589.44281524927	2.68926681891299e-07\\
3636.36363636364	1.40573951362283e-07\\
3683.28445747801	2.26734387722968e-07\\
3730.20527859238	3.18529752453022e-07\\
3777.12609970674	2.94426458674383e-07\\
3824.04692082111	2.42571385672e-07\\
3870.96774193548	2.3267476019259e-07\\
3917.88856304986	1.60124068939395e-07\\
3964.80938416422	1.30080471756143e-07\\
4011.73020527859	2.35405031830063e-07\\
4058.65102639296	2.79221193691151e-07\\
4105.57184750733	1.90921285187444e-07\\
4152.4926686217	2.22303142913495e-07\\
4199.41348973607	1.52980808836841e-07\\
4246.33431085044	1.03651212314322e-07\\
4293.25513196481	1.61276029703081e-07\\
4340.17595307918	2.35490672032221e-07\\
4387.09677419355	1.81688608129274e-07\\
4434.01759530792	1.87409468522682e-07\\
4480.93841642229	1.4639865254799e-07\\
4527.85923753666	1.04640348041572e-07\\
4574.78005865103	1.00492857309306e-07\\
4621.7008797654	1.86175734181491e-07\\
4668.62170087976	1.81268823511103e-07\\
4715.54252199414	1.4555696732084e-07\\
4762.46334310851	1.46920854607024e-07\\
4809.38416422287	1.06241170118152e-07\\
4856.30498533724	7.20526562087229e-08\\
4903.22580645161	1.38717728946187e-07\\
4950.14662756598	1.67963337072995e-07\\
4997.06744868035	1.31169798838916e-07\\
5043.98826979472	1.34421036579449e-07\\
5090.90909090909	1.03042863173696e-07\\
5137.82991202346	7.36574441390556e-08\\
5184.75073313783	9.44001813356721e-08\\
5231.6715542522	1.41531751872368e-07\\
5278.59237536657	1.32769016067971e-07\\
5325.51319648094	1.10643798778061e-07\\
5372.43401759531	1.06908550910572e-07\\
5419.35483870968	7.30374448375361e-08\\
5466.27565982405	6.33417549412273e-08\\
5513.19648093842	1.17746073736911e-07\\
5560.11730205279	1.21568444075382e-07\\
5607.03812316716	1.01228274269448e-07\\
5653.95894428153	1.02375576767333e-07\\
5747.80058651026	5.39988331430337e-08\\
5794.72140762463	8.41829357737322e-08\\
5841.64222873901	1.11754926493343e-07\\
5888.56304985337	9.94169437996452e-08\\
5935.48387096774	9.26497727247709e-08\\
5982.40469208211	7.73861365523169e-08\\
6029.32551319648	5.34110380057105e-08\\
6076.24633431085	5.82165429902528e-08\\
6123.16715542522	9.6381955037312e-08\\
6170.08797653959	9.82957273057877e-08\\
6217.00879765396	7.87997528778623e-08\\
6263.92961876833	8.37508220763634e-08\\
6310.8504398827	5.16368379490816e-08\\
6357.77126099707	4.4985131890539e-08\\
6404.69208211144	7.60879296458272e-08\\
6451.61290322581	9.02904678247699e-08\\
6498.53372434018	7.41789708613946e-08\\
6545.45454545455	7.98421651433195e-08\\
6592.37536656891	5.38939596406419e-08\\
6639.29618768329	4.54342085026546e-08\\
6686.21700879766	5.15628152200105e-08\\
6733.13782991202	8.2090740445558e-08\\
6780.05865102639	7.68828248208949e-08\\
6826.97947214076	6.62300711429117e-08\\
6873.90029325513	6.49449824674205e-08\\
6920.8211143695	4.24844962909054e-08\\
6967.74193548387	3.88905310897258e-08\\
7014.66275659824	6.98313985062328e-08\\
7061.58357771261	7.20580107690728e-08\\
7108.50439882698	6.29561482131516e-08\\
7155.42521994135	6.50910095146688e-08\\
7202.34604105572	4.1104968628892e-08\\
7249.26686217009	3.54734991111502e-08\\
7296.18768328446	5.17581066812041e-08\\
7343.10850439883	6.69919005210527e-08\\
7390.0293255132	6.37760358276781e-08\\
7436.95014662757	5.74165190516476e-08\\
7483.87096774194	4.72621312938441e-08\\
7530.79178885631	3.47441883955817e-08\\
7577.71260997068	3.43782589536742e-08\\
7624.63343108504	6.55629111318339e-08\\
7718.47507331378	5.43346774856836e-08\\
7765.39589442815	5.15164531766192e-08\\
7812.31671554252	3.30594999822832e-08\\
7859.23753665689	2.9824188590169e-08\\
7906.15835777126	5.24259401895369e-08\\
7953.07917888563	5.53746118313853e-08\\
8000	5.44687288648422e-08\\
8046.92082111437	4.80956951066521e-08\\
8093.84164222874	3.71580606777431e-08\\
8140.76246334311	2.80091486426542e-08\\
8187.68328445748	3.59066278171357e-08\\
8234.60410557185	5.50862019546456e-08\\
8328.44574780059	4.55419229436686e-08\\
8375.36656891496	4.37016641981319e-08\\
8422.28739002933	2.61273675032865e-08\\
8469.2082111437	2.88235795635722e-08\\
8516.12903225806	4.64658975369796e-08\\
8563.04985337243	4.97199245947075e-08\\
8609.97067448681	4.3518614852857e-08\\
8656.89149560118	4.4752625182744e-08\\
8703.81231671554	2.81161577861827e-08\\
8750.73313782991	2.62274017285025e-08\\
8797.65395894428	3.4453595149644e-08\\
8844.57478005865	4.81387425834155e-08\\
8891.49560117302	4.29967562604897e-08\\
8938.41642228739	4.05422205334293e-08\\
8985.33724340176	3.35080157688485e-08\\
9032.25806451613	2.3529621482552e-08\\
9079.1788856305	2.65019521480667e-08\\
9126.09970674487	4.41044480333015e-08\\
9219.94134897361	3.85605125407805e-08\\
9266.86217008798	3.71053089400181e-08\\
9313.78299120235	2.28862530867999e-08\\
9360.70381231671	2.36162185655628e-08\\
9407.62463343109	3.5449099853054e-08\\
9454.54545454546	4.13359877315872e-08\\
9548.38709677419	3.55649598824618e-08\\
9595.30791788856	2.71261829405575e-08\\
9642.22873900293	2.1162341136828e-08\\
9689.1495601173	2.57168098671669e-08\\
9736.07038123167	4.08448489143694e-08\\
9782.99120234604	3.54514987437951e-08\\
9829.91202346041	3.53777552188027e-08\\
9876.83284457478	3.03251214600934e-08\\
9923.75366568915	1.95063736594717e-08\\
9970.67448680352	2.06784504503011e-08\\
10017.5953079179	3.52888590806198e-08\\
10064.5161290323	3.44605761322739e-08\\
10111.4369501466	3.55042291720579e-08\\
10158.357771261	3.10798882332002e-08\\
10205.2785923754	2.20762989564047e-08\\
10252.1994134897	1.84651518929345e-08\\
10346.0410557185	3.68360390147591e-08\\
10392.9618768328	3.29362130414526e-08\\
10439.8826979472	3.03820349707035e-08\\
10486.8035190616	2.67576240503435e-08\\
10533.724340176	1.56368141164547e-08\\
10580.6451612903	2.14181411092523e-08\\
10627.5659824047	3.28801327243696e-08\\
10674.4868035191	3.12073630348454e-08\\
10721.4076246334	3.100820401171e-08\\
10768.3284457478	2.68244372395248e-08\\
10815.2492668622	1.73262483955033e-08\\
10862.1700879765	1.86887439851619e-08\\
10909.0909090909	2.57086002860029e-08\\
10956.0117302053	3.27659962777514e-08\\
11002.9325513196	2.90499715550711e-08\\
11049.853372434	2.70060114999204e-08\\
11096.7741935484	2.18738298966408e-08\\
11143.6950146628	1.50354360233953e-08\\
11237.5366568915	3.11438633804255e-08\\
11284.4574780059	2.77195022795462e-08\\
11331.3782991202	2.76841071983939e-08\\
11378.2991202346	2.34149717930602e-08\\
11425.219941349	1.48104835921893e-08\\
11472.1407624633	1.82472107826131e-08\\
11519.0615835777	2.58682991474471e-08\\
11565.9824046921	2.83430762094155e-08\\
11612.9032258065	2.70272620346509e-08\\
11659.8240469208	2.39480506013422e-08\\
11706.7448680352	1.70861622359914e-08\\
11753.6656891496	1.52048025553725e-08\\
11800.5865102639	2.05998745153835e-08\\
11847.5073313783	2.95456602803354e-08\\
11894.4281524927	2.44363880599171e-08\\
11941.348973607	2.62258106982109e-08\\
11988.2697947214	1.94007632470975e-08\\
12035.1906158358	1.36903089546154e-08\\
12082.1114369501	1.70430997081101e-08\\
12129.0322580645	2.68798529532165e-08\\
12175.9530791789	2.42676843728448e-08\\
12222.8739002933	2.58888868101423e-08\\
12269.7947214076	2.07425780026891e-08\\
12316.715542522	1.50761718554769e-08\\
12363.6363636364	1.41331476617247e-08\\
12410.5571847507	2.11747616712365e-08\\
12457.4780058651	2.57889479178976e-08\\
12504.3988269795	2.36671096451824e-08\\
12551.3196480938	2.21488794116757e-08\\
12598.2404692082	1.7588467653766e-08\\
12645.1612903226	1.20339130035009e-08\\
12692.082111437	1.77431854760689e-08\\
12739.0029325513	2.46566817734511e-08\\
12785.9237536657	2.25958066433214e-08\\
12832.8445747801	2.32806463824869e-08\\
12879.7653958944	1.93315227728374e-08\\
12926.6862170088	1.20428122846699e-08\\
12973.6070381232	1.52893237472395e-08\\
13020.5278592375	2.06710659277273e-08\\
13067.4486803519	2.3762947263123e-08\\
13114.3695014663	2.23036732924281e-08\\
13161.2903225806	1.95545702653738e-08\\
13208.211143695	1.50167368038937e-08\\
13255.1319648094	1.13553650831213e-08\\
13302.0527859238	1.79780867257279e-08\\
13348.9736070381	2.34060452649697e-08\\
13395.8944281525	2.08116629271361e-08\\
13442.8152492669	2.14341720180757e-08\\
13489.7360703812	1.59862106850929e-08\\
13536.6568914956	1.06002994683745e-08\\
13583.57771261	1.56049557373779e-08\\
13630.4985337243	2.07298211827104e-08\\
13677.4193548387	2.17679491075695e-08\\
13724.3401759531	2.11642268043844e-08\\
13771.2609970674	1.73689417176735e-08\\
13818.1818181818	1.24235010611659e-08\\
13865.1026392962	1.1766154785629e-08\\
13912.0234604106	1.83294915175785e-08\\
13958.9442815249	2.24484183730574e-08\\
14005.8651026393	1.88335076916587e-08\\
14052.7859237537	1.99925854252358e-08\\
14099.706744868	1.33262446363618e-08\\
14146.6275659824	1.04327112779137e-08\\
14193.5483870968	1.51363217230634e-08\\
14240.4692082111	2.08864393832378e-08\\
14287.3900293255	1.88269497298503e-08\\
14334.3108504399	2.04452271751914e-08\\
14381.2316715543	1.47250689856245e-08\\
14428.1524926686	1.1585834441759e-08\\
14475.073313783	1.19363893036419e-08\\
14521.9941348974	1.83600831306051e-08\\
14568.9149560117	2.0241620646665e-08\\
14615.8357771261	1.84157202794357e-08\\
14662.7565982405	1.77090017587421e-08\\
14709.6774193548	1.25200708970751e-08\\
14756.5982404692	9.75890067375817e-09\\
14803.5190615836	1.61089638395266e-08\\
14850.4398826979	1.9458133098522e-08\\
14897.3607038123	1.78269904989518e-08\\
14944.2815249267	1.85765532670647e-08\\
14991.2023460411	1.33833516623024e-08\\
15038.1231671554	1.00253318626072e-08\\
15085.0439882698	1.2781095454956e-08\\
15131.9648093842	1.77743567457713e-08\\
15178.8856304985	1.86614615436661e-08\\
15225.8064516129	1.78881480817967e-08\\
15272.7272727273	1.4984683703795e-08\\
15319.6480938416	1.12849489937022e-08\\
15366.568914956	9.93645052377353e-09\\
15413.4897360704	1.65113486217119e-08\\
15460.4105571848	1.85575804255382e-08\\
15507.3313782991	1.707319308849e-08\\
15554.2521994135	1.70347484110997e-08\\
15648.0938416422	8.66762910231275e-09\\
15695.0146627566	1.4257582752522e-08\\
15741.935483871	1.70709615221768e-08\\
15788.8563049853	1.73773366140364e-08\\
15835.7771260997	1.72230981874331e-08\\
15882.6979472141	1.32177165090101e-08\\
15929.6187683284	9.79671187944399e-09\\
15976.5395894428	1.04644181585086e-08\\
16023.4604105572	1.56679300397253e-08\\
16070.3812316716	1.8332832866193e-08\\
16117.3020527859	1.56617941915495e-08\\
16164.2228739003	1.63261697812564e-08\\
16211.1436950147	1.01898655702551e-08\\
16258.064516129	8.90148832366228e-09\\
16304.9853372434	1.39816847646556e-08\\
16351.9061583578	1.73393728780846e-08\\
16398.8269794721	1.59511289467411e-08\\
16445.7478005865	1.72805891251989e-08\\
16492.6686217009	1.1004591442907e-08\\
16539.5894428152	9.38500346840294e-09\\
16586.5102639296	1.07690040156423e-08\\
16633.431085044	1.60956781048847e-08\\
16680.3519061584	1.68272748785585e-08\\
16727.2727272727	1.55198600816918e-08\\
16774.1935483871	1.37235211261782e-08\\
16821.1143695015	9.72002419711035e-09\\
16868.0351906158	8.52610079359033e-09\\
16914.9560117302	1.49840554200729e-08\\
16961.8768328446	1.62983478419797e-08\\
17008.7976539589	1.53390932911658e-08\\
17055.7184750733	1.53767082534307e-08\\
17102.6392961877	1.01360639530724e-08\\
17149.5601173021	8.83909795249362e-09\\
17196.4809384164	1.22886759348377e-08\\
17243.4017595308	1.55269531852034e-08\\
17290.3225806452	1.59598506972243e-08\\
17337.2434017595	1.50561101779554e-08\\
17384.1642228739	1.19406447503694e-08\\
17431.0850439883	9.16522545637339e-09\\
17478.0058651026	9.05412152093414e-09\\
17524.926686217	1.53566530987637e-08\\
17618.7683284457	1.49902004163874e-08\\
17665.6891495601	1.390864171079e-08\\
17712.6099706745	9.2588846644044e-09\\
17759.5307917889	8.15502559430557e-09\\
17806.4516129032	1.30277928127461e-08\\
17853.3724340176	1.49902446650466e-08\\
17900.293255132	1.52922293555899e-08\\
17947.2140762463	1.46406864932785e-08\\
17994.1348973607	1.06524768816904e-08\\
18041.0557184751	8.29117137557902e-09\\
18087.9765395894	1.01178866726466e-08\\
18134.8973607038	1.47525805405675e-08\\
18181.8181818182	1.52955974822771e-08\\
18275.6598240469	1.30786811039899e-08\\
18322.5806451613	8.13478040062613e-09\\
18369.5014662757	8.50890630782731e-09\\
18416.42228739	1.32574605698092e-08\\
18463.3431085044	1.47771395374907e-08\\
18510.2639296188	1.39439705751788e-08\\
18557.1847507331	1.44629036800832e-08\\
18604.1055718475	8.89528832953516e-09\\
18651.0263929619	8.50162413565557e-09\\
18697.9472140762	1.05865362281966e-08\\
18744.8680351906	1.44232037069632e-08\\
18791.788856305	1.46667012672409e-08\\
18838.7096774194	1.3755463248541e-08\\
18885.6304985337	1.16759249407779e-08\\
18932.5513196481	8.13682682418547e-09\\
18979.4721407625	8.65267152026991e-09\\
19026.3929618768	1.38977234430283e-08\\
19073.3137829912	1.4030815982392e-08\\
19120.2346041056	1.39631814261642e-08\\
19167.1554252199	1.33734386753805e-08\\
19214.0762463343	7.88552328371312e-09\\
19260.9970674487	8.1717217835041e-09\\
19307.917888563	1.16565540506434e-08\\
19354.8387096774	1.38311658549279e-08\\
19401.7595307918	1.43418612024481e-08\\
19448.6803519062	1.34378421193683e-08\\
19495.6011730205	9.68305888305488e-09\\
19542.5219941349	7.93966121281652e-09\\
19589.4428152493	8.76835437262436e-09\\
19636.3636363636	1.47607201712728e-08\\
19683.284457478	1.35284962740432e-08\\
19730.2052785924	1.38585251912532e-08\\
19777.1260997067	1.18400248150701e-08\\
19824.0469208211	7.55057840817098e-09\\
19870.9677419355	7.93178115729517e-09\\
19917.8885630499	1.31434278839209e-08\\
19964.8093841642	1.30795848535725e-08\\
20011.7302052786	1.43663210678154e-08\\
20058.651026393	1.24193096046869e-08\\
20105.5718475073	8.77239660578129e-09\\
20152.4926686217	7.56420023056494e-09\\
20199.4134897361	1.00058130978622e-08\\
20246.3343108504	1.38964054438722e-08\\
20340.1759530792	1.28661537181289e-08\\
20387.0967741935	1.13237892791318e-08\\
20434.0175953079	6.84009101182613e-09\\
20480.9384164223	8.73881580663949e-09\\
20527.8592375367	1.29449271367019e-08\\
20574.780058651	1.33101750108813e-08\\
20621.7008797654	1.33245711578557e-08\\
20668.6217008798	1.25251214769233e-08\\
20715.5425219941	7.67025592245308e-09\\
20762.4633431085	8.06122428603486e-09\\
20809.3841642229	1.06482827473578e-08\\
20856.3049853372	1.35652056356108e-08\\
20903.2258064516	1.3278766101814e-08\\
20950.146627566	1.26128892891273e-08\\
20997.0674486804	9.59908043506664e-09\\
21043.9882697947	7.18701520350616e-09\\
21090.9090909091	8.96830233341534e-09\\
21137.8299120235	1.34092023921666e-08\\
21184.7507331378	1.27410817430401e-08\\
21231.6715542522	1.30595523613011e-08\\
21278.5923753666	1.14870218532946e-08\\
21325.5131964809	7.01002042911499e-09\\
21372.4340175953	8.22493070885173e-09\\
21419.3548387097	1.19137943054648e-08\\
21466.275659824	1.30173314472507e-08\\
21513.1964809384	1.3288677235466e-08\\
21560.1173020528	1.21999717549949e-08\\
21607.0381231672	8.48606173172853e-09\\
21653.9589442815	7.50954600398191e-09\\
21700.8797653959	9.35235984173321e-09\\
21747.8005865103	1.37833185242507e-08\\
21794.7214076246	1.2342741854579e-08\\
21841.642228739	1.31151543904637e-08\\
21888.5630498534	1.02249518923728e-08\\
21935.4838709677	6.8246436517229e-09\\
21982.4046920821	8.05095575662957e-09\\
22029.3255131965	1.27767313434303e-08\\
22076.2463343109	1.22167099213214e-08\\
22123.1671554252	1.36873922403323e-08\\
22170.0879765396	1.15068960299761e-08\\
22217.008797654	7.73465804140546e-09\\
22263.9296187683	7.3647085114522e-09\\
22310.8504398827	1.03053463067894e-08\\
22357.7712609971	1.34733799070114e-08\\
22404.6920821114	1.2993891475796e-08\\
22451.6129032258	1.22098897745271e-08\\
22498.5337243402	9.88129097747078e-09\\
22545.4545454545	6.24534179658844e-09\\
22592.3753665689	9.03841594477153e-09\\
22639.2961876833	1.2893259610642e-08\\
22686.2170087977	1.23768219559701e-08\\
22733.137829912	1.31756149123542e-08\\
22780.0586510264	1.06755839048111e-08\\
22826.9794721408	6.8105043280485e-09\\
22873.9002932551	8.20975002342248e-09\\
22920.8211143695	1.0950072622494e-08\\
22967.7419354839	1.31722920525289e-08\\
23014.6627565982	1.28793919011637e-08\\
23061.5835777126	1.15420855904474e-08\\
23108.504398827	8.92344156059278e-09\\
23155.4252199413	6.58244301516118e-09\\
23202.3460410557	1.00764348561627e-08\\
23249.2668621701	1.31739030866858e-08\\
23296.1876832845	1.22429945515344e-08\\
23343.1085043988	1.28389178265501e-08\\
23390.0293255132	9.96740803528657e-09\\
23436.9501466276	6.53725601833143e-09\\
23530.7917888563	1.17334277267771e-08\\
23577.7126099707	1.26913339495327e-08\\
23624.633431085	1.29777028874937e-08\\
23671.5542521994	1.10422237183575e-08\\
23718.4750733138	7.62272750704966e-09\\
23765.3958944282	7.43105876110364e-09\\
23812.3167155425	1.01517810066037e-08\\
23859.2375366569	1.35905497002912e-08\\
23906.1583577713	1.19509994658612e-08\\
23953.0791788856	1.3049156754818e-08\\
24000	8.92836719464468e-09\\
};
\addlegendentry{Expérimentale}

\end{axis}
\end{tikzpicture}%


\section{Canal de transmission à bruit additif, blanc et Gaussien}

Dans cette section, nous allons tenter de simuler un bruit blanc Gaussion que nous additionnerons à notre signal modulé en fréquence afin de modéliser le signal reçu par le modem.
Le bruit simulé sera généré aléatoirement grâce au module \verb|rand| de Matlab et sera de puissance $\sigma^2$ avec:

\[
\sigma=\sqrt{\frac{S_\text{module}}{10^{\operatorname{SNR}/10}}} \huge
\]

avec $S_\text{module}$ représentant la densité spectrale de puissance du signal modulé en fréquence et $\operatorname{SNR}$ le rapport signal sur bruit (signal to noise ratio) que nous fixerons à 10 par la suite.

% This file was created by matlab2tikz.
%
%The latest updates can be retrieved from
%  http://www.mathworks.com/matlabcentral/fileexchange/22022-matlab2tikz-matlab2tikz
%where you can also make suggestions and rate matlab2tikz.
%
\definecolor{mycolor1}{rgb}{0.00000,0.44700,0.74100}%
%
\begin{tikzpicture}

\begin{axis}[%
width=4.521in,
height=3.559in,
at={(0.758in,0.488in)},
scale only axis,
xmin=0,
xmax=4000,
xlabel style={font=\color{white!15!black}},
xlabel={Temps [s]},
ymin=-2,
ymax=2,
ylabel style={font=\color{white!15!black}},
ylabel={Amplitude},
axis background/.style={fill=white},
title style={font=\bfseries},
title={Signal bruité}
]
\addplot [color=mycolor1, forget plot]
  table[row sep=crcr]{%
1	0.327650206150346\\
2	0.994898273921492\\
3	0.742265282484324\\
4	0.568542610052729\\
5	-0.652244301137382\\
6	-1.16832687295688\\
7	-1.09959165501646\\
8	-1.0286075542394\\
9	0.717433440733757\\
10	1.00190682864863\\
11	0.749505540858098\\
12	0.676382423042075\\
13	-0.77855560505986\\
14	-0.952015698548656\\
15	-0.972473551041009\\
16	-0.298306403562325\\
17	0.465565176128426\\
18	0.735593898631065\\
19	0.911717889539272\\
20	0.332841726282392\\
21	-0.255086161999748\\
22	-0.684485934084136\\
23	-0.670210530290491\\
24	-0.563005685638814\\
25	0.412865301955499\\
26	0.657441918768412\\
27	0.669264516583724\\
28	0.368205472881039\\
29	-0.0525832767925879\\
30	-1.10137114067771\\
31	-0.835335915250436\\
32	-0.420219290565774\\
33	0.645594935230822\\
34	0.685426113894724\\
35	0.925728423387559\\
36	0.493382224974462\\
37	-0.149260356243119\\
38	-0.583573580394599\\
39	-0.899213148303127\\
40	-0.703524989783497\\
41	0.229442293496784\\
42	0.691576137616095\\
43	1.44442601226737\\
44	0.231979356840414\\
45	-0.228149981012709\\
46	-0.972195219700834\\
47	-0.71959027558638\\
48	-0.540895505371972\\
49	0.0817554236498745\\
50	0.61083792770883\\
51	1.02769035025033\\
52	0.330045351539592\\
53	-0.439426630636155\\
54	-0.611523984044084\\
55	-0.853192245242021\\
56	-0.325472224347438\\
57	0.750847871074014\\
58	0.749113282184807\\
59	1.07433274729404\\
60	0.556613511599946\\
61	-0.450092313725125\\
62	-0.880873508996663\\
63	-1.17933428171372\\
64	-0.626626082311257\\
65	0.419022724520626\\
66	1.09076135113491\\
67	1.49702172242676\\
68	0.220501996268806\\
69	-0.353633178462622\\
70	-0.947596503677823\\
71	-1.35101292690967\\
72	-0.467969673052807\\
73	-0.0060577156610786\\
74	1.11719447694334\\
75	0.719719696699584\\
76	0.392136896554895\\
77	-0.517408161492757\\
78	-0.861214354041843\\
79	-1.05278329748512\\
80	-0.260094196846996\\
81	0.561007987598265\\
82	1.31222072970078\\
83	0.875001946711404\\
84	-0.10878106968281\\
85	-0.583436390359521\\
86	-0.626006034452946\\
87	-1.15836871878375\\
88	-0.15469678977851\\
89	0.423313708109638\\
90	1.25063869160639\\
91	0.479634248918094\\
92	0.325497472784912\\
93	-0.665844562343159\\
94	-0.278384469406136\\
95	-0.733775977209273\\
96	-0.0611530617122698\\
97	0.158755295702694\\
98	0.824269586561937\\
99	0.857469775397058\\
100	0.615542713589853\\
101	-0.457735890972832\\
102	-0.772145636720111\\
103	-1.37759640645316\\
104	-0.448922453781266\\
105	0.211252476855186\\
106	0.576223526423451\\
107	1.03211686724844\\
108	0.432840357657789\\
109	-0.388061853057634\\
110	-1.22758526130556\\
111	-0.666133502655912\\
112	-0.29137536613168\\
113	0.328629216219575\\
114	0.934258230918394\\
115	0.859813632237327\\
116	-0.0219227894523835\\
117	-0.459476677500043\\
118	-1.11517844670236\\
119	-1.1375686762975\\
120	-0.628516764593085\\
121	0.276155006855656\\
122	0.480987892777805\\
123	1.13421716257168\\
124	0.486116805979165\\
125	-0.400035767130413\\
126	-0.93691701637679\\
127	-1.09705509400709\\
128	-0.141777704610477\\
129	0.365742660434412\\
130	0.769239035491489\\
131	1.22085467233936\\
132	0.319439138456077\\
133	-0.52736635526525\\
134	-0.994871897218514\\
135	-1.10819093850091\\
136	-0.620399544942612\\
137	0.960819962365353\\
138	1.29960178208918\\
139	0.987262692427851\\
140	0.0884213583146095\\
141	-0.589227633811786\\
142	-0.968640623322739\\
143	-0.741340345639117\\
144	-0.667812998503713\\
145	-0.125821690385122\\
146	0.60485824331426\\
147	0.993075506086728\\
148	0.4573149782058\\
149	-0.294477525077993\\
150	-0.958290959213946\\
151	-0.877336956580824\\
152	-0.476291316786757\\
153	0.588456397171003\\
154	0.624417170668076\\
155	1.02026886013609\\
156	0.179814763919022\\
157	-0.47049465938296\\
158	-0.805434597472931\\
159	-0.68591594124551\\
160	-0.619842420400352\\
161	-0.61603286608914\\
162	-0.606008093366085\\
163	-0.573147871050279\\
164	-0.367849913966483\\
165	-0.11697054585428\\
166	0.124437566007243\\
167	0.444862707439232\\
168	0.66865693259875\\
169	1.01472279883799\\
170	1.28770658843005\\
171	1.10215225992375\\
172	0.934413237670486\\
173	1.03804652715547\\
174	0.794736787631649\\
175	0.327520492847653\\
176	0.536579518778607\\
177	0.136988835151332\\
178	-0.162017394665753\\
179	-0.324404201606751\\
180	-0.598681941994149\\
181	-1.04055205291123\\
182	-0.982328858779502\\
183	-1.03035053311308\\
184	-1.1003957482397\\
185	-0.521722296769318\\
186	-0.949704300486416\\
187	-0.666228687012945\\
188	-0.48349524205544\\
189	-0.331038998948329\\
190	0.14924753368904\\
191	0.378374380401242\\
192	0.999577515715522\\
193	0.774140647391298\\
194	0.707852325406168\\
195	1.35648683151248\\
196	1.25763812965774\\
197	0.846756366574474\\
198	0.41668724397075\\
199	0.458462551203131\\
200	0.289267135262124\\
201	0.130053099751461\\
202	-0.250709698566654\\
203	-0.340477008326697\\
204	-0.569481476686768\\
205	-1.10974302348441\\
206	-1.15813494511627\\
207	-1.16351063032981\\
208	-1.09498167481657\\
209	-0.969880081296787\\
210	-0.750944039133782\\
211	-1.23582764301467\\
212	-0.426433822557982\\
213	0.209759235549169\\
214	-0.0464387843012501\\
215	0.648654452452581\\
216	0.735573749932017\\
217	0.823376302785628\\
218	0.986830062880011\\
219	0.647439682931745\\
220	0.96242453468884\\
221	1.2570716906827\\
222	0.775742529673981\\
223	0.567219527785239\\
224	0.159871759955182\\
225	0.0613793167430422\\
226	-0.140272313380911\\
227	-0.344288941558919\\
228	-0.740606036879889\\
229	-0.882780788069948\\
230	-0.493141292244501\\
231	-1.50303903830455\\
232	-0.482439293772678\\
233	-0.822602262883777\\
234	-0.529941611401468\\
235	-0.930366252980151\\
236	-0.456200905113357\\
237	-0.13049974271521\\
238	0.286861674550422\\
239	0.0659495206128534\\
240	0.762721140124648\\
241	0.558457297619243\\
242	0.960813027160656\\
243	1.1436501867125\\
244	1.05453190034795\\
245	1.14041275609565\\
246	0.978873409701584\\
247	0.412301410384846\\
248	0.381687287340067\\
249	-0.14305725636615\\
250	-0.48805540920544\\
251	-0.23274883600295\\
252	-0.657168066268933\\
253	-0.842156894183672\\
254	-0.742109810064403\\
255	-0.864611132487397\\
256	-0.902974992239826\\
257	-0.618361843079025\\
258	-0.545666875351988\\
259	-0.504306963823113\\
260	-0.478651890584292\\
261	-0.214078892711062\\
262	0.459034243459959\\
263	0.0790115775467308\\
264	0.651705150582654\\
265	0.393755863926199\\
266	1.17436723239615\\
267	1.19050067770065\\
268	0.981602755860367\\
269	0.882290083871251\\
270	0.197355900645979\\
271	0.68801521264914\\
272	-0.166457691940342\\
273	-0.450731740041388\\
274	-0.174379229391499\\
275	-0.651955825860616\\
276	-0.565096208684015\\
277	-0.678373683751859\\
278	-0.754058434088378\\
279	-1.15233354122031\\
280	-0.880781372820665\\
281	-0.875622330257481\\
282	-0.56887712576905\\
283	-0.437980212362998\\
284	-0.123234794661365\\
285	-0.0977997275386044\\
286	0.159326111828863\\
287	0.665224232340323\\
288	0.181949452221189\\
289	0.716951281092865\\
290	0.661668556765502\\
291	0.912052060448205\\
292	1.12650358595648\\
293	1.08292247579415\\
294	0.526611146216218\\
295	0.452545361944025\\
296	0.354826791334723\\
297	0.00296185444801557\\
298	-0.124726032093976\\
299	-0.350209379691967\\
300	-0.865285208919747\\
301	-0.869438110171682\\
302	-1.42311934768869\\
303	-0.741358508556863\\
304	-1.12212008825621\\
305	-1.16753847209481\\
306	-0.810561870363843\\
307	-0.877662327320139\\
308	-0.32883098389719\\
309	-0.193739910866322\\
310	0.679608147682576\\
311	0.694470370217585\\
312	0.098428135080418\\
313	0.928626943612011\\
314	0.633127000011741\\
315	0.940590528339364\\
316	1.0181329061277\\
317	1.06546924325186\\
318	0.69263214189028\\
319	0.91070405010989\\
320	0.216539787551273\\
321	0.141565163467546\\
322	-0.0435131030517717\\
323	-0.420642148620925\\
324	-0.46004037463351\\
325	-0.757538832917121\\
326	-1.12147662561427\\
327	-1.40167143000115\\
328	-0.565428395027067\\
329	-1.03342319031178\\
330	-0.730604601735843\\
331	-0.431935966017516\\
332	-0.298742847416949\\
333	-0.270733630093761\\
334	0.0876019636594287\\
335	0.41175781409867\\
336	0.988139264197496\\
337	0.637234602159405\\
338	1.12159350178857\\
339	1.06672997166881\\
340	0.929009709344298\\
341	0.661613533472092\\
342	0.690149535066619\\
343	0.538561500720298\\
344	-0.00465646523817548\\
345	0.111281161726144\\
346	-0.37627879842148\\
347	-0.460794654353262\\
348	-0.581941692925113\\
349	-1.03230988455322\\
350	-1.0105068331685\\
351	-0.919329815383606\\
352	-1.39216966532235\\
353	-0.666312187523228\\
354	-0.211190547545835\\
355	-0.343267166447249\\
356	-0.394826666318896\\
357	0.0276420234787556\\
358	-0.039565045732779\\
359	0.859932674418289\\
360	0.867689233282988\\
361	1.00605874000897\\
362	0.749998696519005\\
363	1.06926449283589\\
364	0.85640809715825\\
365	0.828450569756119\\
366	0.626178027798463\\
367	0.328423201727208\\
368	0.120805188656423\\
369	0.0213041438495955\\
370	-0.572437050132515\\
371	-0.303737250167551\\
372	-0.683540019996941\\
373	-0.673409696970571\\
374	-0.885659337666745\\
375	-0.887055427661392\\
376	-1.31323423614018\\
377	-1.12645584908226\\
378	-0.853762504424613\\
379	-0.533421672571953\\
380	-0.071575824510434\\
381	-0.13316246305164\\
382	0.474575446073586\\
383	0.546095727852129\\
384	0.919922045199027\\
385	0.858275291607572\\
386	0.799019379433942\\
387	0.666160360657761\\
388	1.01613789787071\\
389	1.08131668583607\\
390	0.688259233334012\\
391	0.436944263460138\\
392	0.255095887847616\\
393	-0.177119907252768\\
394	-0.302592032399691\\
395	-0.480151273399634\\
396	-0.64692093875215\\
397	-0.844140518348403\\
398	-0.809197018662774\\
399	-0.973203536012252\\
400	-0.575403908754775\\
401	-0.828317553252875\\
402	-0.3499263062024\\
403	-0.719780471311514\\
404	-0.206333247004522\\
405	-0.126513497499533\\
406	0.326566093214344\\
407	0.572614893695009\\
408	0.167993209433768\\
409	0.532902627451338\\
410	0.623532308334842\\
411	1.08759096574336\\
412	1.31034306760566\\
413	0.825007770324228\\
414	0.935515284621458\\
415	0.68004146569094\\
416	0.0888301223468688\\
417	0.157219604441996\\
418	-0.360525075007372\\
419	-0.10039669253779\\
420	-0.664466926506842\\
421	-0.46376460931113\\
422	-1.04112029450328\\
423	-0.865763669002373\\
424	-0.995394071617775\\
425	-1.3506141815352\\
426	-0.973514887542959\\
427	-0.420893731925561\\
428	-0.33644585473082\\
429	-0.318623659258791\\
430	0.0520962817223639\\
431	0.49554244028384\\
432	0.434962293293581\\
433	1.04804069924054\\
434	0.802623660566717\\
435	1.40245225560789\\
436	0.739691177030355\\
437	0.942716413977648\\
438	0.41336035148683\\
439	0.39602743382083\\
440	0.191406445731553\\
441	0.158085482876556\\
442	0.0185630479401916\\
443	-0.372463089640613\\
444	-0.740664648417269\\
445	-0.647377879438741\\
446	-0.767226840682784\\
447	-0.970767121014372\\
448	-0.853509408929759\\
449	-0.805766842162616\\
450	-0.974595820752279\\
451	-0.38798543837364\\
452	-0.471231418531446\\
453	-0.203419138022754\\
454	0.231863731186168\\
455	0.370892719097018\\
456	0.627680577719203\\
457	0.963095808898729\\
458	1.18026040425065\\
459	0.953367543944153\\
460	1.05467023987364\\
461	0.844815070631719\\
462	0.805113343084365\\
463	0.656423324414439\\
464	0.186121764231982\\
465	0.129777523134616\\
466	-0.0577526363031583\\
467	-0.41904898477425\\
468	-0.270076821899908\\
469	-0.96604964212072\\
470	-1.11093966936659\\
471	-1.38925302095834\\
472	-0.777595707378577\\
473	-0.704106995430334\\
474	-0.771612700267547\\
475	-0.356900802461717\\
476	-0.283054492404104\\
477	-0.00320202335470898\\
478	0.217541395819505\\
479	0.538157695218252\\
480	0.679929154533514\\
481	1.01930140086523\\
482	0.668060688991065\\
483	0.503489309179967\\
484	0.114477576774999\\
485	-0.640221398044269\\
486	-1.0261686836388\\
487	-0.956142785008992\\
488	-0.418641459231029\\
489	0.516693300347801\\
490	1.01624562500185\\
491	1.08655211013674\\
492	0.767674765466557\\
493	-0.121858079655163\\
494	-1.21630190614165\\
495	-1.43961415506745\\
496	-0.167904771653681\\
497	-0.01522367348191\\
498	0.944074776520311\\
499	0.926382310851786\\
500	0.868131941758989\\
501	-0.41104265740323\\
502	-1.04266428645294\\
503	-0.865673459331643\\
504	-0.314732213947546\\
505	0.411228607877523\\
506	0.792948356693736\\
507	0.644851860938034\\
508	0.440564384558026\\
509	-0.696060059789016\\
510	-1.1601172863058\\
511	-0.620544411429331\\
512	-0.463480001588118\\
513	0.364152879226179\\
514	1.13049742986501\\
515	0.851284163044007\\
516	0.600088771817975\\
517	-0.472772514430859\\
518	-0.702492054216191\\
519	-0.777610797287478\\
520	-0.417406539169474\\
521	0.201828962477241\\
522	0.695710106130881\\
523	0.858006978813253\\
524	0.271691318713108\\
525	-0.487006734062427\\
526	-0.709039077177997\\
527	-0.985061149950892\\
528	-0.113806739288532\\
529	0.276588465195099\\
530	1.1467758843434\\
531	0.801573983878868\\
532	0.409252634605608\\
533	-0.178323087499314\\
534	-1.02177432807535\\
535	-1.0165185371595\\
536	0.0785787959687947\\
537	0.608366448752742\\
538	0.832462529254727\\
539	1.06366213526494\\
540	0.289160648586482\\
541	-0.237591617282097\\
542	-0.612298502004791\\
543	-1.2774998608703\\
544	-0.139502366122195\\
545	0.721816711699306\\
546	0.939759064403215\\
547	1.3092185476558\\
548	0.404510692109707\\
549	-0.672395517181594\\
550	-1.41999421176258\\
551	-0.993052285465951\\
552	-0.210062110954808\\
553	0.466583177952878\\
554	1.02169334583669\\
555	0.789303033277094\\
556	0.401962807219367\\
557	-0.762253141361555\\
558	-1.09922825073515\\
559	-1.10167154815957\\
560	-0.253433731763093\\
561	0.392385226249861\\
562	0.670552617711212\\
563	0.916311264994851\\
564	0.215373008415186\\
565	-0.544747307696462\\
566	-0.735757368425187\\
567	-0.893061839794358\\
568	-0.280592918899315\\
569	0.593367963851947\\
570	0.969473882756411\\
571	1.04171248745736\\
572	0.522571360260111\\
573	-0.133596165841012\\
574	-0.822648305391404\\
575	-0.602414542199213\\
576	-0.364678902634752\\
577	0.384841777023531\\
578	1.3098591597528\\
579	0.80437989940376\\
580	0.369099329062955\\
581	-0.189706898216264\\
582	-0.895611905364479\\
583	-0.60404795993901\\
584	-0.138323392368495\\
585	0.460801297140596\\
586	0.755103271673311\\
587	1.04525750141737\\
588	0.060336451041169\\
589	-0.340845622023469\\
590	-0.748224241985025\\
591	-0.870768456753324\\
592	-0.172884820127885\\
593	0.851811903191203\\
594	1.13589277146703\\
595	0.978173296079285\\
596	0.513328662442224\\
597	-0.300339209743367\\
598	-1.22334365324879\\
599	-1.01162688238398\\
600	-0.0956786350861595\\
601	0.38580071639208\\
602	1.05947014665281\\
603	0.69320903259531\\
604	0.384175709943616\\
605	-0.261221142529616\\
606	-1.23381460347292\\
607	-0.840658766067117\\
608	-0.410430926560072\\
609	0.185305680547613\\
610	0.920736828743545\\
611	0.494089374916931\\
612	-0.10645857771332\\
613	-0.658924829538771\\
614	-1.15079641037701\\
615	-1.18094289465996\\
616	-0.755852937088296\\
617	0.460053382736482\\
618	0.572390928985093\\
619	0.943107420476604\\
620	0.545867304692174\\
621	-0.396052260239627\\
622	-0.908300175829521\\
623	-1.00306639909072\\
624	-0.701530177967909\\
625	0.385748267081998\\
626	1.14414860182831\\
627	1.30742567906564\\
628	0.273467031426545\\
629	-0.759714816166312\\
630	-0.891910876099066\\
631	-0.834242284786917\\
632	-0.420524986827397\\
633	0.13845136071086\\
634	1.38213935854618\\
635	0.390430626778352\\
636	0.255617109440249\\
637	-0.691306175236857\\
638	-1.07148819854613\\
639	-0.84731422877912\\
640	-0.338845954103233\\
641	0.236506276096229\\
642	1.10301314531468\\
643	1.05772151253675\\
644	0.514608539889397\\
645	-0.490801438845854\\
646	-0.694485483979011\\
647	-0.77059028624821\\
648	0.191672708825854\\
649	0.633534023466599\\
650	1.18803089269856\\
651	0.930298298244412\\
652	0.0814243122731326\\
653	-0.478625519291741\\
654	-1.09871398791672\\
655	-1.04464721533837\\
656	-0.245509786241697\\
657	0.270958641857807\\
658	0.728828335411137\\
659	0.826843741484081\\
660	0.333735109062244\\
661	-0.303953449706979\\
662	-1.14231602652812\\
663	-0.847433714914344\\
664	-0.352278943807456\\
665	0.691923773933963\\
666	0.881432867302942\\
667	0.888349228501594\\
668	0.107613225119127\\
669	-0.705546828329479\\
670	-0.859650692019637\\
671	-0.974273187911836\\
672	-0.257010803630665\\
673	0.195794960763383\\
674	1.35622119961152\\
675	0.945795422091554\\
676	0.604042401837798\\
677	-0.446333956447935\\
678	-0.965500515944691\\
679	-0.764023529898352\\
680	-0.245371443430855\\
681	0.144864263682106\\
682	0.586151265677938\\
683	0.672762849426268\\
684	0.0529176890917725\\
685	-0.38222329752742\\
686	-1.02116534686806\\
687	-1.00079586364656\\
688	-0.674293352968525\\
689	0.570004871973935\\
690	1.02746700876252\\
691	0.898387101089507\\
692	0.598257006704144\\
693	-0.591132315203109\\
694	-0.836334702750883\\
695	-0.840468874261103\\
696	-0.291582360469656\\
697	0.232363633715321\\
698	1.00227599916006\\
699	0.803222979174645\\
700	0.169132270989917\\
701	-0.664820258395378\\
702	-0.696894493389138\\
703	-1.10774740126297\\
704	-0.408432718980997\\
705	0.125082848117722\\
706	0.862645196594941\\
707	0.195180140228019\\
708	0.12649942015735\\
709	-0.714760592981698\\
710	-1.15614888203604\\
711	-0.966167488020122\\
712	-0.442544252399863\\
713	0.830669595846384\\
714	0.801185060370302\\
715	0.862490737072966\\
716	0.0185581608067733\\
717	-0.502382228519385\\
718	-1.22854722291428\\
719	-0.911662453274613\\
720	-0.17883519593889\\
721	0.486017440170949\\
722	0.772351792066548\\
723	0.553561274426778\\
724	0.69645868815065\\
725	0.0632028248216609\\
726	-0.902170453818765\\
727	-1.13996206100198\\
728	-0.101701952544765\\
729	0.262929859764127\\
730	0.82400244303827\\
731	1.11679549964199\\
732	0.0597549292810514\\
733	-0.833434647073436\\
734	-0.834995741110849\\
735	-0.828765942322568\\
736	-0.348447380358774\\
737	0.506701552382449\\
738	1.17131883078869\\
739	1.13560870303566\\
740	0.242504206438348\\
741	-0.21429910702645\\
742	-0.890366827104387\\
743	-1.03157261796995\\
744	-0.636775139822825\\
745	0.540332544993747\\
746	0.850002615846391\\
747	0.928833820349157\\
748	0.192293108337642\\
749	-0.742526730281282\\
750	-0.890738527943857\\
751	-0.932348181633639\\
752	-0.101420795780991\\
753	0.574958590343753\\
754	1.16484355411125\\
755	0.750859755165678\\
756	0.160207881768916\\
757	-0.679549084330222\\
758	-0.817698303058284\\
759	-0.294304597551226\\
760	-0.206922773003553\\
761	0.222558340659416\\
762	1.11635714573154\\
763	0.665945968642823\\
764	0.0509715335117867\\
765	-0.235005335616667\\
766	-1.10321499262655\\
767	-0.847731697326099\\
768	-0.0549850017450266\\
769	0.485317273458705\\
770	1.13717454155668\\
771	0.559097724061407\\
772	0.517776213930051\\
773	0.0829982193399048\\
774	-0.808040266549575\\
775	-1.26325889575027\\
776	-0.415197327202233\\
777	0.283672170182961\\
778	1.01484873054623\\
779	1.01064766036582\\
780	0.460478979242427\\
781	-0.476960491441103\\
782	-1.06324054806723\\
783	-1.05038045326505\\
784	-0.17873358532289\\
785	-0.0191105799256349\\
786	0.882745852576754\\
787	0.978947741062978\\
788	0.223661681178706\\
789	-0.288760445864187\\
790	-0.945095822187015\\
791	-1.12841498725287\\
792	-0.333628418354232\\
793	0.33554022026723\\
794	0.837415029008171\\
795	0.759263565970758\\
796	0.383488426341349\\
797	-0.808679144026823\\
798	-1.01827461148341\\
799	-1.04007746753121\\
800	-0.573802062391896\\
801	0.21433510845561\\
802	-0.356581307986382\\
803	-0.318723175849003\\
804	-0.438806398461789\\
805	-0.864971512058881\\
806	-0.883835385595778\\
807	-0.854555929350032\\
808	-0.999463895492281\\
809	-0.777106524158019\\
810	-1.03627001131877\\
811	-0.30947129695036\\
812	-0.545606958584707\\
813	-0.477530098456698\\
814	0.502088596125529\\
815	0.425668630360565\\
816	0.757638398585057\\
817	0.748577522823704\\
818	0.717615813216785\\
819	0.309995856182025\\
820	1.12149087431569\\
821	0.833987958638565\\
822	0.709573126578465\\
823	0.648726993525858\\
824	0.00654292243083382\\
825	-0.0949598334561264\\
826	0.0644812993838626\\
827	-0.305915951002137\\
828	-0.943903225346762\\
829	-1.32291173992607\\
830	-1.07383407955717\\
831	-0.949778608715188\\
832	-0.77045841670427\\
833	-0.877168165071352\\
834	-1.00488428962134\\
835	-0.48944924669207\\
836	-0.586507559690972\\
837	-0.283319046785142\\
838	0.0459744730226675\\
839	0.164592937969456\\
840	0.596542880735794\\
841	0.628477145293418\\
842	0.882070480809565\\
843	0.894186263957177\\
844	0.889641327485107\\
845	0.785460458537219\\
846	1.02972050375374\\
847	0.694534546488579\\
848	0.337382599772753\\
849	-0.259997274917199\\
850	-0.556087563981771\\
851	-0.879376161487963\\
852	-0.0741907859925575\\
853	-0.612269999209383\\
854	-0.888494098273472\\
855	-1.21568123551159\\
856	-1.23787512938107\\
857	-0.7755916825144\\
858	-0.403501219680978\\
859	-0.936895945876748\\
860	-0.424729234049514\\
861	-0.0871375930046681\\
862	-0.253500667513944\\
863	0.627958799998025\\
864	0.564387210158208\\
865	1.25777334941702\\
866	0.858526098045054\\
867	1.08923296423566\\
868	0.725691125663017\\
869	0.758310502393582\\
870	0.492198569195226\\
871	0.645811131037048\\
872	0.615487994487575\\
873	-0.064569849878289\\
874	-0.0946150204936975\\
875	-0.552571374169566\\
876	-0.634329646325718\\
877	-0.562171283477979\\
878	-0.91908429193574\\
879	-1.22969046637648\\
880	-1.17314452191258\\
881	-0.936156297481096\\
882	-0.796625711072081\\
883	-0.751712349485147\\
884	-0.283734621961686\\
885	0.215147852451861\\
886	0.136060281121193\\
887	0.393927069968946\\
888	0.164524567305772\\
889	0.656547319498154\\
890	0.634215750762191\\
891	0.911235300580784\\
892	1.09895796136435\\
893	1.23901784507834\\
894	1.15619987192895\\
895	0.53180471717183\\
896	0.252510260636011\\
897	0.251217868034793\\
898	-0.082579329199408\\
899	-0.268458339788799\\
900	-0.483016614735951\\
901	-1.1611285893698\\
902	-0.825078899678403\\
903	-1.01815107685057\\
904	-1.15147132040747\\
905	-1.05335415771788\\
906	-0.466970850873849\\
907	-0.739163353689599\\
908	-0.600937617845943\\
909	-0.0202324400739786\\
910	0.64223547977807\\
911	0.445424126528695\\
912	0.726170200701449\\
913	0.619907043933959\\
914	1.32065566288573\\
915	1.02563629893871\\
916	1.09996830319246\\
917	0.685088976136434\\
918	0.944851308157625\\
919	0.645043670046513\\
920	0.0654745687080291\\
921	0.0771677973956698\\
922	-0.293101288414909\\
923	-0.41525816847486\\
924	-0.713247718250106\\
925	-0.872363214151397\\
926	-1.17714556213567\\
927	-1.07001253880678\\
928	-0.809809994224683\\
929	-0.507720529547919\\
930	-1.01343448836873\\
931	-0.0259458235011355\\
932	0.0173409125367999\\
933	-0.0305663248066962\\
934	0.124862990725034\\
935	0.283361333658808\\
936	0.843535895381426\\
937	0.674429028675057\\
938	0.842641301982931\\
939	1.19540162732622\\
940	1.07889791511669\\
941	1.09881531736768\\
942	0.866682843832663\\
943	0.46824856488332\\
944	0.209175023901058\\
945	0.246485233282117\\
946	-0.342465075597302\\
947	-0.174156050899039\\
948	-0.480236623314651\\
949	-0.765481121905919\\
950	-0.945279138683288\\
951	-0.915849058603082\\
952	-0.192145816118548\\
953	-0.923302952099299\\
954	-1.10206758498746\\
955	-0.129401053381819\\
956	-0.187692665889376\\
957	-0.213263884948622\\
958	0.777971651814489\\
959	0.562996782844031\\
960	0.723015865643952\\
961	0.655801979810099\\
962	0.707691946663875\\
963	0.601932402795283\\
964	0.886702140539051\\
965	0.66247962958703\\
966	0.898688454735743\\
967	0.486882417298813\\
968	0.720026876569427\\
969	0.406311598762335\\
970	-0.155564639612883\\
971	-0.502982195368421\\
972	-0.399348560726759\\
973	-1.08643166638487\\
974	-0.795241368410249\\
975	-1.14740007745342\\
976	-1.07092657101389\\
977	-1.04847733560062\\
978	-0.624920590509113\\
979	-0.732082065601978\\
980	-0.562166283776104\\
981	0.0554702161020068\\
982	0.0975121890072768\\
983	0.520621238414387\\
984	0.578433753458114\\
985	0.890184772122006\\
986	0.372130833049713\\
987	1.10191722300275\\
988	1.39613095870203\\
989	1.13071323675043\\
990	0.95757399331595\\
991	0.504314118144211\\
992	0.364667076988001\\
993	0.122932737193345\\
994	-0.170696009719155\\
995	-0.625547094568\\
996	-0.736005908673536\\
997	-0.868978193401726\\
998	-1.05356127925118\\
999	-0.810399964065696\\
1000	-0.413313101649137\\
1001	-1.19427655961306\\
1002	-0.725014478026846\\
1003	-0.880735326928257\\
1004	-0.0326893319910943\\
1005	0.247233980745809\\
1006	-0.179775692040355\\
1007	0.874661596900948\\
1008	0.414446346809526\\
1009	0.880624539941659\\
1010	1.19191849532395\\
1011	1.03060432508255\\
1012	1.49506480146077\\
1013	1.51410697751173\\
1014	0.784686970302043\\
1015	0.131199361948327\\
1016	0.242485670307616\\
1017	-0.121514533108604\\
1018	-0.363402567095093\\
1019	-0.692060140821447\\
1020	-0.639682178628712\\
1021	-0.358447438280369\\
1022	-1.10619278177195\\
1023	-1.06044010468314\\
1024	-0.72030994556448\\
1025	-0.626737853248858\\
1026	-0.645080411270693\\
1027	-0.32835990119816\\
1028	-0.129313173966106\\
1029	-0.153704231861574\\
1030	0.288244033392401\\
1031	0.372766410517565\\
1032	0.455807428020136\\
1033	0.97190997364199\\
1034	0.961095781578997\\
1035	0.955792881783179\\
1036	1.04662522155553\\
1037	1.11916743138517\\
1038	0.841664906065393\\
1039	0.601497513588171\\
1040	0.386773518189579\\
1041	0.0797368231609138\\
1042	-0.365576106789341\\
1043	-0.26364025198072\\
1044	-0.341895251411467\\
1045	-0.94938730308738\\
1046	-0.514608973847983\\
1047	-1.03710707686332\\
1048	-1.03588894959042\\
1049	-1.09900626712607\\
1050	-0.931042803728742\\
1051	-0.771216822704503\\
1052	-0.244966830156112\\
1053	0.289106373020173\\
1054	0.310303558706387\\
1055	0.630858224445407\\
1056	0.957456467125394\\
1057	0.270523771300742\\
1058	0.908504796653744\\
1059	1.07666401382112\\
1060	1.14184911964271\\
1061	0.606143410802164\\
1062	0.528641872647974\\
1063	0.734899434612691\\
1064	0.298077254010102\\
1065	0.192044759652252\\
1066	-0.407238982920115\\
1067	-0.805316454429111\\
1068	-0.486838485540347\\
1069	-0.562829952530769\\
1070	-0.580780733874726\\
1071	-1.34053860877995\\
1072	-1.28050237228049\\
1073	-1.22795805901338\\
1074	-0.763057676815373\\
1075	-0.69569867182281\\
1076	-0.0300830661757573\\
1077	-0.393888318559923\\
1078	-0.197634665190938\\
1079	0.485648480846639\\
1080	0.924132146741177\\
1081	0.650212133806517\\
1082	0.662778059855771\\
1083	0.964249314170058\\
1084	0.615140093859251\\
1085	0.902023276365349\\
1086	0.939110131031139\\
1087	0.60668627450875\\
1088	-0.103085676272799\\
1089	-0.0518547330833591\\
1090	-0.259855250055908\\
1091	-0.0338637484393043\\
1092	-0.452454493228395\\
1093	-0.84264066019633\\
1094	-0.652993170741742\\
1095	-1.23145663225977\\
1096	-1.05927754923429\\
1097	-0.582040267807107\\
1098	-0.417532407478733\\
1099	-0.394518072856668\\
1100	-0.214343368520852\\
1101	-0.199437583275732\\
1102	0.358959720251226\\
1103	0.254381780036354\\
1104	0.785744970252065\\
1105	0.892937406383688\\
1106	1.20095404009868\\
1107	0.902366218824683\\
1108	1.12369753651783\\
1109	1.07563867374724\\
1110	0.552696302427869\\
1111	0.592925311734075\\
1112	0.681595573102062\\
1113	0.0934364468781667\\
1114	-0.261330544418981\\
1115	-0.337514688201392\\
1116	-0.718741045645473\\
1117	-0.730700896615574\\
1118	-0.97615869843894\\
1119	-1.00176802956976\\
1120	-0.878227649746903\\
1121	-0.593284205970064\\
1122	-0.652614253335581\\
1123	-0.189087117917536\\
1124	-0.778068556850537\\
1125	-0.168809084268304\\
1126	0.245077078522599\\
1127	0.252811299458165\\
1128	0.371647086608412\\
1129	0.967946760240321\\
1130	1.08311088449777\\
1131	1.06242401710682\\
1132	1.06980620329126\\
1133	0.703328179365466\\
1134	0.642362078662933\\
1135	0.534090050316356\\
1136	0.170241679696628\\
1137	0.142542799913116\\
1138	0.337021206849252\\
1139	-0.547618747738006\\
1140	-0.512293825819104\\
1141	-1.06134984934018\\
1142	-0.646236714903426\\
1143	-1.21453993878956\\
1144	-0.934636510413194\\
1145	-1.0365702558564\\
1146	-0.639155935664152\\
1147	-0.555420150086554\\
1148	-0.334007220841987\\
1149	0.591670573217839\\
1150	0.051275537157763\\
1151	0.429214946496042\\
1152	1.25758433277446\\
1153	0.573261744573045\\
1154	1.06975065327221\\
1155	0.756777629320746\\
1156	1.2119783340547\\
1157	0.971436309807072\\
1158	0.899666272475148\\
1159	0.495557678924856\\
1160	0.37903224560964\\
1161	0.397792569822855\\
1162	-0.701386966499208\\
1163	-0.805201953167223\\
1164	-0.564205944694955\\
1165	-0.976390790546128\\
1166	-1.01231764362977\\
1167	-1.3326457563279\\
1168	-1.18382577015338\\
1169	-0.988589367738508\\
1170	-0.916152831622672\\
1171	-0.751862240688552\\
1172	-0.418563989389698\\
1173	-0.27922383518756\\
1174	0.492552294629063\\
1175	0.218514518899362\\
1176	1.06399685228326\\
1177	0.746076089726628\\
1178	0.621131802256932\\
1179	0.859218478390651\\
1180	1.19046451702715\\
1181	1.13443690280051\\
1182	0.789589870745844\\
1183	0.622275169881755\\
1184	0.465794543016189\\
1185	-0.258228331932242\\
1186	-0.322441221016937\\
1187	-0.849254798539538\\
1188	-0.757679117397308\\
1189	-0.617439321163572\\
1190	-0.785452662225877\\
1191	-0.485698262087811\\
1192	-0.944032534581204\\
1193	-1.3807195953782\\
1194	-0.375682045151536\\
1195	-0.27101330815459\\
1196	-0.454544133987264\\
1197	-0.0184583118677024\\
1198	0.366691696974521\\
1199	0.52579501024684\\
1200	0.813011604772833\\
1201	0.804643797685199\\
1202	0.937313141819486\\
1203	1.0173789509658\\
1204	0.804634180100371\\
1205	1.21657086129618\\
1206	0.755151249104851\\
1207	0.711581062663251\\
1208	0.132847180667891\\
1209	-0.172340436189887\\
1210	-0.212623297663592\\
1211	-0.496270638240758\\
1212	-0.38980258825958\\
1213	-0.694242778445215\\
1214	-0.825045227733048\\
1215	-1.32101383867338\\
1216	-1.19789238305545\\
1217	-0.852927101281013\\
1218	-0.831582180088912\\
1219	-0.269266210103759\\
1220	-0.0240412512164787\\
1221	-0.19824518733749\\
1222	0.388103808030175\\
1223	0.751978137651387\\
1224	0.729009014774794\\
1225	1.19314629955703\\
1226	1.18375147609985\\
1227	1.04557883792341\\
1228	1.17755296248744\\
1229	0.941646205335729\\
1230	0.660890144291786\\
1231	0.638176559893911\\
1232	0.33226936147668\\
1233	-0.0133220587248164\\
1234	0.204050490335098\\
1235	-0.390189283364743\\
1236	-0.046177704240923\\
1237	-0.89614281878155\\
1238	-0.819723508797282\\
1239	-0.643509081104171\\
1240	-0.370597966590321\\
1241	-0.830210623419778\\
1242	-0.930577704206432\\
1243	-0.378181248085896\\
1244	-0.619530161201234\\
1245	-0.129555751912236\\
1246	0.253104868306146\\
1247	0.773036563941944\\
1248	0.978781499817495\\
1249	0.823699522007686\\
1250	1.15275762055077\\
1251	0.925770701809205\\
1252	1.12928865687525\\
1253	1.32674478266208\\
1254	0.788813879260986\\
1255	0.490707189875201\\
1256	0.212266137512219\\
1257	0.228606127747604\\
1258	0.106990580835703\\
1259	0.0359772289119383\\
1260	-0.645084902895272\\
1261	-0.793383151619454\\
1262	-1.08758763895177\\
1263	-0.636936502429192\\
1264	-0.998226555597783\\
1265	-1.00404256936421\\
1266	-0.26494684999522\\
1267	-0.376726488218107\\
1268	-0.163860915722894\\
1269	-0.293300618807066\\
1270	0.289383871658832\\
1271	0.556102857686437\\
1272	0.412757694934701\\
1273	0.779339101608149\\
1274	0.855388033405144\\
1275	1.11579042963723\\
1276	0.756005433374871\\
1277	1.14184343567352\\
1278	1.15315205831324\\
1279	0.489987054725171\\
1280	0.322217359760209\\
1281	0.508928296790883\\
1282	1.19839865478151\\
1283	1.03525941518623\\
1284	0.458588838180833\\
1285	-0.503596941155395\\
1286	-0.980940213537728\\
1287	-0.781179933992255\\
1288	0.0068486173479792\\
1289	0.522748570702096\\
1290	0.659251797226644\\
1291	1.01535261897698\\
1292	0.349123508714081\\
1293	-0.45016835333311\\
1294	-0.978185821095505\\
1295	-1.11531617175336\\
1296	-0.441527521937618\\
1297	0.220018180891706\\
1298	0.847464976494844\\
1299	0.9446854489441\\
1300	0.408751793223661\\
1301	-0.443813356167062\\
1302	-0.96328723699204\\
1303	-0.910904021929518\\
1304	-0.267184216870018\\
1305	0.682356269977574\\
1306	1.06789918454757\\
1307	0.854291115232747\\
1308	0.503561645094917\\
1309	-0.450499040118482\\
1310	-1.32762778669283\\
1311	-1.44370589213704\\
1312	-0.753205072626511\\
1313	0.3424898200666\\
1314	0.79047664801913\\
1315	0.757285880071975\\
1316	0.378836436451319\\
1317	-0.54302024191585\\
1318	-1.0702320777739\\
1319	-0.782021370627786\\
1320	-0.194667971436237\\
1321	0.940810676812772\\
1322	0.996808441250815\\
1323	0.931493508515439\\
1324	0.241259011643405\\
1325	-0.439238412974646\\
1326	-0.940443818299541\\
1327	-1.31134855138728\\
1328	-0.427329474349492\\
1329	0.563285782038497\\
1330	0.801410833225624\\
1331	1.02904185671108\\
1332	0.591601887204756\\
1333	-0.154512475768834\\
1334	-0.755294992227468\\
1335	-1.42414784749211\\
1336	-0.49603393380755\\
1337	0.597288919895315\\
1338	1.01745611964748\\
1339	0.919529061131671\\
1340	0.46751147948933\\
1341	-0.142667263187655\\
1342	-0.89472529056353\\
1343	-1.08820766663388\\
1344	-0.410054950471221\\
1345	0.34905497938744\\
1346	1.12980885231896\\
1347	1.01070860077728\\
1348	0.4922616881605\\
1349	-0.36247155468821\\
1350	-1.0102038014549\\
1351	-0.904760596688214\\
1352	-0.321244030257319\\
1353	0.0826833363885593\\
1354	0.969163429271583\\
1355	1.12601673065087\\
1356	0.345082687091821\\
1357	-0.0436841245145719\\
1358	-0.803709871743005\\
1359	-1.01250709206011\\
1360	-0.404187974737545\\
1361	0.333970250458142\\
1362	0.98309369834222\\
1363	1.08732580612324\\
1364	0.304414866497576\\
1365	-0.292963444208966\\
1366	-0.536339012250785\\
1367	-0.709994589369523\\
1368	-0.185061251831987\\
1369	0.213217121371419\\
1370	0.809592227268819\\
1371	0.972720741372739\\
1372	0.347215299199084\\
1373	-0.759205878204656\\
1374	-1.26803226668042\\
1375	-0.688802974181877\\
1376	-0.539391315639024\\
1377	0.860644138172064\\
1378	0.431901746340296\\
1379	1.01886667848109\\
1380	0.369878321117103\\
1381	-0.564784525688945\\
1382	-0.838656515697485\\
1383	-1.09609044723939\\
1384	-0.177336675067146\\
1385	0.410517885633083\\
1386	0.5622752903266\\
1387	0.375834319272094\\
1388	0.306222995924114\\
1389	-0.139146416592641\\
1390	-0.888594067260069\\
1391	-0.906302673540689\\
1392	-0.215833295980682\\
1393	0.0837384798500017\\
1394	1.24810307568988\\
1395	0.718417509047095\\
1396	0.378189161211598\\
1397	-0.476913170504293\\
1398	-0.895668353688236\\
1399	-1.35358531013527\\
1400	-0.0287223376892019\\
1401	0.517694875832448\\
1402	1.37891579669419\\
1403	1.23546406869088\\
1404	0.372307342892568\\
1405	-0.605693590900976\\
1406	-1.3182673771544\\
1407	-0.914644993188649\\
1408	-0.320694978528123\\
1409	0.62958677694554\\
1410	0.716325903238078\\
1411	1.09630181727399\\
1412	0.385722710183127\\
1413	-0.568688932885397\\
1414	-0.755894710135493\\
1415	-0.858994761072746\\
1416	-0.422349920440574\\
1417	0.815652770524586\\
1418	1.06547103574132\\
1419	0.893765749849494\\
1420	0.431827400823\\
1421	-0.376074893032931\\
1422	-0.890289471223459\\
1423	-0.816938509292703\\
1424	-0.504792948740641\\
1425	0.217196631706191\\
1426	1.00984166911859\\
1427	1.34028691566659\\
1428	0.30107788759954\\
1429	-0.622180437621806\\
1430	-0.982517478991135\\
1431	-1.02675515521132\\
1432	-0.442991081434523\\
1433	0.501964952118524\\
1434	0.900043591759897\\
1435	0.785438882858016\\
1436	0.270432437027398\\
1437	-0.691508484309642\\
1438	-1.12147887079234\\
1439	-1.23464291377755\\
1440	-0.485205084807943\\
1441	0.587044880581732\\
1442	0.787753742432463\\
1443	0.930162562841888\\
1444	0.712709175725688\\
1445	0.893809102647043\\
1446	0.829902178666773\\
1447	0.342409986583245\\
1448	0.573436110210006\\
1449	-0.286662504005328\\
1450	-0.279942192904477\\
1451	-0.771947367188524\\
1452	-0.734299588902517\\
1453	-1.12613193529424\\
1454	-1.07463714028571\\
1455	-1.16342126048015\\
1456	-1.2465213929189\\
1457	-1.36021455375602\\
1458	-0.627285757846185\\
1459	-0.110370860350724\\
1460	0.173370855870148\\
1461	-0.178413857779203\\
1462	0.181972228839843\\
1463	0.335481118480415\\
1464	0.674174671646452\\
1465	0.624265716789121\\
1466	0.516464528003403\\
1467	0.989511339493335\\
1468	0.707185140124436\\
1469	0.4724168795013\\
1470	1.28504162258432\\
1471	0.505743076071982\\
1472	0.414543564353358\\
1473	0.335124370669837\\
1474	-0.569280592791439\\
1475	-0.347251559589679\\
1476	-0.544899113404251\\
1477	-0.811279120523856\\
1478	-0.910689165194156\\
1479	-1.11580907018792\\
1480	-0.819536812360588\\
1481	-1.08834117614208\\
1482	-0.931948479424589\\
1483	-0.395645566857681\\
1484	0.0532499693458477\\
1485	-0.154741737019832\\
1486	0.0792407180650159\\
1487	0.531023998304239\\
1488	0.897150144630307\\
1489	1.04648155538331\\
1490	1.00645400775559\\
1491	0.855699612453566\\
1492	1.12227056166901\\
1493	0.880411113288191\\
1494	1.06186683501596\\
1495	0.242832412454635\\
1496	0.355768187427578\\
1497	0.356887068983896\\
1498	-0.303023139163717\\
1499	-0.579528361165523\\
1500	-0.866107090378417\\
1501	-0.892240005701969\\
1502	-0.990874376503198\\
1503	-0.967066735898816\\
1504	-1.17901774382973\\
1505	-0.879671709278872\\
1506	-0.888868948931552\\
1507	-0.640474666317577\\
1508	-0.511728531320685\\
1509	-0.131486805071437\\
1510	0.991128626448898\\
1511	1.20224020195793\\
1512	0.913835500917293\\
1513	1.00603899218981\\
1514	0.659980608837592\\
1515	0.86684681938497\\
1516	0.842732645920857\\
1517	1.01793117539795\\
1518	0.861816772260108\\
1519	0.58223543628038\\
1520	0.248014864928753\\
1521	-0.138910766395274\\
1522	-0.191075842337043\\
1523	-0.184787697035878\\
1524	-0.561544693797794\\
1525	-0.789355970662545\\
1526	-0.798870832664095\\
1527	-0.866924581592909\\
1528	-1.3430707684654\\
1529	-0.893940665570475\\
1530	-0.84905547773701\\
1531	-1.01313382593804\\
1532	-0.621465816981552\\
1533	-0.13985851150142\\
1534	0.376980835678167\\
1535	0.388394632586743\\
1536	0.715865479975212\\
1537	0.628918107128587\\
1538	0.463241755690333\\
1539	1.01868316423248\\
1540	0.768488146304795\\
1541	1.16054312678038\\
1542	1.14201259169066\\
1543	0.474783159400196\\
1544	0.59110157986157\\
1545	0.281681310878935\\
1546	-0.508924158685093\\
1547	-0.447170025475258\\
1548	-0.598746515814911\\
1549	-0.712831926062189\\
1550	-0.794550305739217\\
1551	-0.846750881570185\\
1552	-0.829531177724379\\
1553	-0.968090791017464\\
1554	-0.66886751087331\\
1555	-1.0621125808595\\
1556	-0.585615985395838\\
1557	0.200700033257801\\
1558	0.0444072191581059\\
1559	0.513006454291452\\
1560	0.994666585327886\\
1561	0.630947915154078\\
1562	0.731333196913475\\
1563	0.887406257642231\\
1564	1.08004196286581\\
1565	0.852320008625467\\
1566	0.53492156771114\\
1567	0.205968359628392\\
1568	0.241306673823104\\
1569	0.018787046634366\\
1570	-0.312036184498294\\
1571	-0.498329482336555\\
1572	-0.902310217613226\\
1573	-0.706960195462122\\
1574	-1.43263577156508\\
1575	-0.96530481357339\\
1576	-0.831626986547251\\
1577	-1.08809222358079\\
1578	-0.766509927482119\\
1579	-0.492665300573527\\
1580	-0.461702285828643\\
1581	-0.071327847251537\\
1582	0.284489124573929\\
1583	0.229991057686578\\
1584	0.844069935461936\\
1585	1.07480205052821\\
1586	1.01787462039762\\
1587	1.0010484921295\\
1588	0.720753579474489\\
1589	1.00887700506642\\
1590	0.481652723781927\\
1591	0.913946124132913\\
1592	0.441977612083935\\
1593	0.127535660152448\\
1594	-0.28034571849062\\
1595	-0.926718342039604\\
1596	-0.924741728584566\\
1597	-0.748114317937461\\
1598	-0.406146670199489\\
1599	-0.832014676175793\\
1600	-0.912027928842323\\
1601	-1.01942739337092\\
1602	-0.726369925934368\\
1603	-0.342622945863351\\
1604	-0.380997753552908\\
1605	0.355060720761396\\
1606	-0.0808006822886164\\
1607	0.355511557382532\\
1608	0.66891177068808\\
1609	0.392617606081871\\
1610	1.07253039860398\\
1611	1.01507533130722\\
1612	0.999401947290213\\
1613	1.07598099495711\\
1614	0.987290151279991\\
1615	0.737473158268287\\
1616	0.219250376460677\\
1617	0.0578483854649711\\
1618	-0.11317131325112\\
1619	-0.720951312775561\\
1620	-0.889021434019631\\
1621	-0.925412525611976\\
1622	-0.928620446371991\\
1623	-0.596893741211995\\
1624	-0.802931855972786\\
1625	-0.997164403989132\\
1626	-0.894511806735561\\
1627	-0.214729383264339\\
1628	0.28754580066434\\
1629	-0.0308054155149913\\
1630	0.0851066958762942\\
1631	0.123583122422086\\
1632	0.875361198995899\\
1633	0.89470024920067\\
1634	1.20679511890413\\
1635	0.794317203139781\\
1636	1.03517837807078\\
1637	0.751780054640542\\
1638	1.03904256617622\\
1639	0.434616021891023\\
1640	0.304603203986673\\
1641	-0.00956014139569045\\
1642	0.0301613445006604\\
1643	-0.343830504141366\\
1644	-0.8818949285903\\
1645	-0.971449529154113\\
1646	-1.18000914659332\\
1647	-0.884499880286685\\
1648	-0.907058391257964\\
1649	-1.26411316325831\\
1650	-1.18044539023475\\
1651	-0.567001367202862\\
1652	-0.168146168472583\\
1653	-0.0305002931195979\\
1654	-0.153202671358988\\
1655	0.09244458986703\\
1656	0.851213136938941\\
1657	0.797335746171592\\
1658	0.859568252758055\\
1659	1.29220556386146\\
1660	0.803110635090964\\
1661	0.928450925920997\\
1662	0.847230709301702\\
1663	0.741432929490426\\
1664	0.132977029520736\\
1665	0.14823522876472\\
1666	0.42278272594035\\
1667	-0.778278251174567\\
1668	-0.560063771584117\\
1669	-0.881284410536944\\
1670	-1.13108574451222\\
1671	-1.18384689912807\\
1672	-0.869923189365158\\
1673	-0.379953181707526\\
1674	-0.931375974273798\\
1675	-0.436905515478945\\
1676	-0.449269747814116\\
1677	0.374039444128968\\
1678	0.314150349711868\\
1679	0.408845192107997\\
1680	0.795894451178207\\
1681	0.828617876902688\\
1682	1.19376915832801\\
1683	0.956135361508385\\
1684	0.730392804736907\\
1685	0.95329223178888\\
1686	1.10306190172679\\
1687	0.290182011827185\\
1688	0.26993166920375\\
1689	0.29313435739928\\
1690	-0.621950805282363\\
1691	-0.299743866049569\\
1692	-0.488682395146475\\
1693	-0.782093994608532\\
1694	-1.1183633855808\\
1695	-0.999263140613668\\
1696	-0.960489506501159\\
1697	-0.688852273328186\\
1698	-0.605252887862286\\
1699	-0.636335737639657\\
1700	0.0383312777467554\\
1701	-0.0796501724443882\\
1702	0.0104020829763199\\
1703	0.341596346652108\\
1704	0.849318218021773\\
1705	0.873552439231841\\
1706	1.14490987420022\\
1707	1.01315695309202\\
1708	1.53784141434439\\
1709	0.525405807923009\\
1710	0.660658357208019\\
1711	0.539113267544905\\
1712	0.344136207435406\\
1713	0.394130138312979\\
1714	-0.143157608232152\\
1715	-0.465412709035261\\
1716	-0.641827420597706\\
1717	-0.661698322934993\\
1718	-1.10028046217618\\
1719	-0.896783883094815\\
1720	-1.33156256030154\\
1721	-0.915774856867674\\
1722	-0.964487736684867\\
1723	-0.704057755596026\\
1724	-0.260939337212853\\
1725	-0.320043994459639\\
1726	-0.0290450341742239\\
1727	0.100488024766077\\
1728	0.158485088436585\\
1729	0.79595979815465\\
1730	1.20619899693093\\
1731	0.95695301067536\\
1732	0.928988786398174\\
1733	0.663904735484536\\
1734	1.10804806941514\\
1735	0.502147458426288\\
1736	0.613737930707213\\
1737	0.149278386303282\\
1738	-0.251132386666837\\
1739	-0.197544235007649\\
1740	-0.437330978891967\\
1741	-0.854806020555725\\
1742	-0.655994338057533\\
1743	-0.599049906255817\\
1744	-0.996025945559429\\
1745	-0.955358015019094\\
1746	-0.663184776088925\\
1747	-0.479245156798025\\
1748	-0.143459362837342\\
1749	-0.125470604095802\\
1750	0.352294151956101\\
1751	0.674359770807411\\
1752	0.69691088241112\\
1753	0.864172009674602\\
1754	0.66530943681146\\
1755	0.735218199969582\\
1756	0.654302529674888\\
1757	0.619696291999269\\
1758	0.719026347258228\\
1759	0.19657697034674\\
1760	0.826785986804703\\
1761	0.560989639657866\\
1762	0.466440095140645\\
1763	0.660078419170061\\
1764	0.34575090445614\\
1765	-0.778860541645027\\
1766	-1.03205077301069\\
1767	-0.674299689778095\\
1768	-0.406411190639767\\
1769	0.380506938087454\\
1770	0.705000485834946\\
1771	0.794097307878881\\
1772	0.0672881087162032\\
1773	-0.314051059441087\\
1774	-1.03033059691267\\
1775	-0.743141788155984\\
1776	-0.0928704703867329\\
1777	0.636608422513951\\
1778	0.921339256752094\\
1779	0.805660895905524\\
1780	0.0958283097513059\\
1781	-0.369034092638327\\
1782	-0.714425828946218\\
1783	-0.994735858947852\\
1784	-0.408523234667881\\
1785	0.533469142618914\\
1786	1.1224630969097\\
1787	0.60136692456447\\
1788	0.414483284655551\\
1789	-0.351731989420461\\
1790	-0.96295625049967\\
1791	-0.718656280219352\\
1792	-0.451236372560225\\
1793	0.210670166995608\\
1794	0.990901542467633\\
1795	1.02133578185295\\
1796	0.321064756352064\\
1797	-0.217398247876604\\
1798	-1.26883774076389\\
1799	-1.15898673945045\\
1800	-1.05722478100465\\
1801	0.512233998354032\\
1802	0.70736556929479\\
1803	0.861798485408195\\
1804	0.595590710685888\\
1805	-0.3841385191383\\
1806	-1.02769563930135\\
1807	-1.10831248808807\\
1808	-0.423528186551738\\
1809	0.530476975884613\\
1810	0.58983005198633\\
1811	0.903147774126461\\
1812	0.544821403305263\\
1813	-0.71348276125381\\
1814	-0.779346027219975\\
1815	-0.765548642873433\\
1816	-0.567141885065099\\
1817	0.0588762933794552\\
1818	1.02606353035681\\
1819	1.09931350181405\\
1820	0.499292447647723\\
1821	-0.224844239166723\\
1822	-1.1936713308141\\
1823	-0.787789814983376\\
1824	-0.49953227085872\\
1825	0.269931691409279\\
1826	0.580958813839763\\
1827	1.16464643225863\\
1828	0.408912603843817\\
1829	-0.170966687267617\\
1830	-0.591003880202971\\
1831	-1.17293310185161\\
1832	-0.225844113739105\\
1833	0.392698520396261\\
1834	1.13372821778899\\
1835	1.1663216696119\\
1836	0.553342598400966\\
1837	-0.578518415449067\\
1838	-0.957440725621985\\
1839	-0.859333073087562\\
1840	0.337074450863608\\
1841	0.670045286693863\\
1842	1.44844572349095\\
1843	1.01119871256036\\
1844	0.417141693021282\\
1845	-0.258339999110008\\
1846	-1.04724144152537\\
1847	-0.640598296088381\\
1848	-0.405012666427211\\
1849	0.0881705412769035\\
1850	1.12401137661844\\
1851	0.567437726589196\\
1852	-0.0429670967602522\\
1853	-0.331013855988197\\
1854	-1.14193003671027\\
1855	-1.12224557699312\\
1856	-0.40597905379545\\
1857	0.286147940741714\\
1858	0.879282009392236\\
1859	0.979325545967142\\
1860	0.108245384170669\\
1861	-0.669512038578214\\
1862	-1.39895548735441\\
1863	-1.00577089819265\\
1864	-0.221085816534156\\
1865	0.238401924230523\\
1866	1.04132261190449\\
1867	1.03935368170682\\
1868	0.591447602171299\\
1869	-0.174151778938396\\
1870	-1.08328425282663\\
1871	-1.11018511554107\\
1872	-0.35890953356301\\
1873	0.246773636671307\\
1874	1.25422930342577\\
1875	1.22722342270121\\
1876	0.39102775451003\\
1877	-0.491136074199362\\
1878	-0.814827846372469\\
1879	-1.06530308233304\\
1880	-0.397714354762842\\
1881	0.276846575891042\\
1882	0.952764005478354\\
1883	1.17094401338649\\
1884	0.535889298739368\\
1885	-0.13963943433823\\
1886	-1.13382992988891\\
1887	-0.878206880922856\\
1888	-0.589790928059374\\
1889	0.481670554234841\\
1890	1.00202030542988\\
1891	1.20852994733937\\
1892	0.615722825713356\\
1893	-0.249376267183295\\
1894	-1.04216085561704\\
1895	-1.02496022185745\\
1896	-0.828834837194301\\
1897	0.295232753005417\\
1898	0.582010563884307\\
1899	1.12651834505388\\
1900	0.571573550680655\\
1901	-0.364608107066285\\
1902	-1.01381573156323\\
1903	-0.886427947542199\\
1904	-0.0105640551105204\\
1905	0.697457967204728\\
1906	0.828783988047214\\
1907	0.95612968109543\\
1908	0.118994435334572\\
1909	-0.305878738564509\\
1910	-0.763749178285048\\
1911	-0.717031522546757\\
1912	-0.712392900488438\\
1913	0.508477551992755\\
1914	0.735742772482043\\
1915	0.834177603323674\\
1916	0.546078491475989\\
1917	-0.328821753283564\\
1918	-0.965777978625018\\
1919	-0.782668661972622\\
1920	-0.00397151918144761\\
1921	0.690342304884384\\
1922	0.643876836875029\\
1923	0.737581280551357\\
1924	0.770032728799607\\
1925	0.74793897435869\\
1926	0.882783560128108\\
1927	0.0912091635071816\\
1928	0.376967276672449\\
1929	-0.105919665134803\\
1930	0.0537922887486532\\
1931	-0.631163010384505\\
1932	-0.655487983034798\\
1933	-1.03967931382706\\
1934	-1.09851990776014\\
1935	-1.05588032938351\\
1936	-1.03254233312716\\
1937	-1.01558416829119\\
1938	-0.501240216116543\\
1939	-0.434851294632139\\
1940	0.0909797757757713\\
1941	-0.130328469054751\\
1942	0.430956430089935\\
1943	-0.0300602635146957\\
1944	0.800058563118728\\
1945	0.912993403584695\\
1946	0.862267558043387\\
1947	1.15327099691856\\
1948	1.17773770086443\\
1949	1.12944134527181\\
1950	0.84767488629267\\
1951	0.692467724761605\\
1952	0.173332253346938\\
1953	-0.176803294555556\\
1954	-0.252153763166754\\
1955	-0.39795949798328\\
1956	-0.44437765126146\\
1957	-1.00676878155344\\
1958	-1.0555360321485\\
1959	-0.332041382913446\\
1960	-1.12066431458923\\
1961	-0.46841824357725\\
1962	-0.538649280880347\\
1963	-0.682785682819739\\
1964	-0.348008303364821\\
1965	-0.116423382691955\\
1966	0.298223836232053\\
1967	0.745307521592426\\
1968	0.290672932739831\\
1969	1.28273546962853\\
1970	1.12005408450104\\
1971	0.874833760114823\\
1972	0.953143993446292\\
1973	0.965181816266512\\
1974	0.820011279876998\\
1975	0.826671131759015\\
1976	0.568121409777018\\
1977	-0.011995508862137\\
1978	-0.221341817399929\\
1979	-0.275510433851484\\
1980	-0.630251527479969\\
1981	-0.575579962413411\\
1982	-1.09968656597325\\
1983	-0.892113774039981\\
1984	-0.916817680117838\\
1985	-0.586673911328785\\
1986	-1.05482961796618\\
1987	-0.557793722697207\\
1988	-0.312192462945685\\
1989	-0.592344290403371\\
1990	0.471564369538753\\
1991	1.06835303444389\\
1992	0.605221095365062\\
1993	0.894098991084858\\
1994	0.842032287809149\\
1995	1.0836688515379\\
1996	0.896345233564247\\
1997	0.875366130628308\\
1998	1.11584280308812\\
1999	0.36836861157917\\
2000	0.19529284880952\\
2001	0.0512870126700457\\
2002	-0.221911102930005\\
2003	-0.117618114535244\\
2004	-0.202624975983039\\
2005	-1.04876829139668\\
2006	-1.08944879374695\\
2007	-0.881194962405665\\
2008	-1.0665354193007\\
2009	-0.452183302498574\\
2010	-0.387660731401595\\
2011	-0.432864216124572\\
2012	-0.198033156787244\\
2013	-0.144000887505572\\
2014	0.363155068959434\\
2015	0.692003643640565\\
2016	0.688859893923481\\
2017	1.26134335707193\\
2018	1.09607584292325\\
2019	0.973025957600804\\
2020	1.0643563167429\\
2021	1.14353697883195\\
2022	0.664633757110978\\
2023	0.538836241332126\\
2024	0.85525567361643\\
2025	-0.0377788465565485\\
2026	0.0195046073926211\\
2027	-0.256618093025629\\
2028	-0.301600701258473\\
2029	-0.712156680848786\\
2030	-1.49766083642053\\
2031	-1.18844451454312\\
2032	-0.866827000992137\\
2033	-0.840457328626897\\
2034	-0.314579581996165\\
2035	-0.243257051479441\\
2036	-0.21298246469552\\
2037	-0.0497390353367962\\
2038	-0.154250302132686\\
2039	0.856641817182241\\
2040	0.687171582116854\\
2041	0.483904078239647\\
2042	1.04296063292956\\
2043	1.02070555994065\\
2044	0.853711152491478\\
2045	1.00847899358273\\
2046	0.595407659820049\\
2047	0.256301108466485\\
2048	-0.0684125838492715\\
2049	-0.0131380645170927\\
2050	-0.332597275691648\\
2051	-0.34120913759368\\
2052	-0.993438944163403\\
2053	-0.876465933638105\\
2054	-1.28281673985536\\
2055	-0.59268432965196\\
2056	-1.00751872513475\\
2057	-0.623741995301925\\
2058	-0.642976404677614\\
2059	-0.35463641167209\\
2060	-0.582453569657291\\
2061	0.213139891696294\\
2062	0.517313164200666\\
2063	-0.0250414871288009\\
2064	0.617684889659512\\
2065	0.683944802610003\\
2066	0.981030074006159\\
2067	0.807403766761344\\
2068	1.04378735841189\\
2069	1.03175908853711\\
2070	0.808496842019934\\
2071	0.531477497459943\\
2072	0.035620911902154\\
2073	0.0811563279961456\\
2074	-0.308840633115031\\
2075	-0.725925101224828\\
2076	-0.669761702330863\\
2077	-0.876789258411037\\
2078	-0.903012756384821\\
2079	-0.775603299185584\\
2080	-0.852391282176536\\
2081	0.687761106726386\\
2082	0.76593254936163\\
2083	0.724968972060655\\
2084	0.348496872795967\\
2085	-0.0860195469817563\\
2086	-0.637399291049513\\
2087	-0.946596882577157\\
2088	-0.503306105152043\\
2089	0.0552650419942619\\
2090	1.07948010111987\\
2091	0.923554817747659\\
2092	0.106173705825685\\
2093	-0.470425562012858\\
2094	-1.33852986578153\\
2095	-0.822220045026215\\
2096	-0.408674314968553\\
2097	0.760591959372755\\
2098	0.624924354749256\\
2099	1.13562188219741\\
2100	0.401881801207357\\
2101	-0.377068992273948\\
2102	-0.768761524225062\\
2103	-0.651365963556197\\
2104	-0.609419512614578\\
2105	0.690696739298095\\
2106	0.658362524268043\\
2107	0.678078245481551\\
2108	0.219233300382687\\
2109	-0.230749086558253\\
2110	-1.168570844713\\
2111	-0.843540484580508\\
2112	-0.27756748122669\\
2113	0.430042833233631\\
2114	1.05323732061565\\
2115	0.655984356052796\\
2116	0.595215080322359\\
2117	-0.37015048255537\\
2118	-0.765765825948895\\
2119	-1.13853215182074\\
2120	-0.358094633124862\\
2121	0.59194237496461\\
2122	1.15607957165083\\
2123	0.899567328540376\\
2124	-0.0450434717835119\\
2125	-0.640999534829835\\
2126	-0.880211673021154\\
2127	-0.740707097901475\\
2128	-0.266100680693331\\
2129	0.258460138847457\\
2130	1.43138580732041\\
2131	0.934632627189835\\
2132	0.56342226711694\\
2133	-0.488578145074259\\
2134	-1.17863585103763\\
2135	-0.76509755109682\\
2136	-0.1375286014725\\
2137	0.803326860242447\\
2138	0.810759388917332\\
2139	0.55413776521161\\
2140	0.731656738715162\\
2141	-0.336445661950877\\
2142	-1.18136776448446\\
2143	-1.04357462899669\\
2144	-0.550740874013162\\
2145	0.655362479491019\\
2146	1.06163677432\\
2147	0.972764603295378\\
2148	0.423552051572057\\
2149	-0.579388595493139\\
2150	-0.706898206643054\\
2151	-0.840926832750832\\
2152	-0.428174442808967\\
2153	0.354219058961141\\
2154	0.906369803465037\\
2155	0.719941000495797\\
2156	0.535642930497179\\
2157	-0.0840068874959378\\
2158	-0.375528721567277\\
2159	-0.805676133311103\\
2160	-0.553792191339976\\
2161	0.440529499854165\\
2162	0.703778635786019\\
2163	0.781227744872821\\
2164	0.222275956047767\\
2165	-0.582013284931384\\
2166	-0.844507339454322\\
2167	-1.16803313685863\\
2168	-0.220423521170169\\
2169	0.573532833841068\\
2170	1.1613049963313\\
2171	0.913870667940375\\
2172	0.508246991874806\\
2173	0.00793402964355605\\
2174	-0.917277176005433\\
2175	-0.958250831073312\\
2176	0.0269063939537935\\
2177	-0.165863934903227\\
2178	0.826960842923431\\
2179	1.46231051026876\\
2180	0.264236624949792\\
2181	-0.520943773523928\\
2182	-1.20359377342887\\
2183	-0.740997000881079\\
2184	-0.841898841194537\\
2185	0.216666007081298\\
2186	1.05601316537308\\
2187	0.918119851770682\\
2188	0.509359527802229\\
2189	-0.367259132925614\\
2190	-0.77677197123354\\
2191	-1.20653955401578\\
2192	-0.32093537125665\\
2193	0.0449889369950179\\
2194	1.10441812111816\\
2195	0.848943678818949\\
2196	0.516386135453565\\
2197	-0.515827929438673\\
2198	-0.855592438575462\\
2199	-0.68256343497229\\
2200	-0.812753598464096\\
2201	-0.0223285491222457\\
2202	0.519093017342097\\
2203	1.10833994664556\\
2204	0.460410661610513\\
2205	-0.552748951813213\\
2206	-0.59368076037846\\
2207	-0.887607127416213\\
2208	-0.724835719973203\\
2209	0.16749045651734\\
2210	0.619149573300358\\
2211	1.13212639588901\\
2212	0.235220028189709\\
2213	-0.657798655127349\\
2214	-1.05828121477631\\
2215	-1.10561843647696\\
2216	-0.178861431229201\\
2217	0.502370707247071\\
2218	0.996788896025664\\
2219	1.01148477612938\\
2220	0.37935507941849\\
2221	-0.607887501110369\\
2222	-0.807935324691999\\
2223	-1.10219410878236\\
2224	-0.609608187444855\\
2225	0.155194661308321\\
2226	1.12372374269608\\
2227	1.13798157326217\\
2228	-0.0238501992433811\\
2229	-0.428694736300138\\
2230	-0.872756668573168\\
2231	-1.05503408223052\\
2232	-0.19944681059894\\
2233	0.45248996429046\\
2234	0.894934985963302\\
2235	0.793705822388434\\
2236	0.521240630981849\\
2237	-0.205498910636613\\
2238	-0.824114820583249\\
2239	-0.972100883351572\\
2240	-0.163640887074714\\
2241	0.192612827162097\\
2242	0.961458766072541\\
2243	1.23092993818271\\
2244	0.327287586399829\\
2245	-0.584386812429481\\
2246	-0.494235099385267\\
2247	-0.973698629650527\\
2248	-0.270913059599909\\
2249	0.21572247411346\\
2250	0.99349696132076\\
2251	1.02295164457075\\
2252	0.370759002570497\\
2253	-0.322581735583159\\
2254	-0.659574607059852\\
2255	-1.23265456004273\\
2256	-0.323737376206338\\
2257	-0.0500341070697515\\
2258	0.935493299520338\\
2259	1.35045179870819\\
2260	0.580344980121204\\
2261	-0.397103499618012\\
2262	-1.00482739418542\\
2263	-0.80022788557912\\
2264	-0.133226139116616\\
2265	0.398852594279341\\
2266	1.29048584255235\\
2267	0.735871429837663\\
2268	-0.0773185868269239\\
2269	-0.562983573459445\\
2270	-0.97373605314109\\
2271	-1.00185866791719\\
2272	-0.559362189540582\\
2273	0.694799916214794\\
2274	0.567437905205684\\
2275	0.808384229503505\\
2276	0.152828749356817\\
2277	-0.501649409346217\\
2278	-1.07820953347188\\
2279	-0.811392472598272\\
2280	-0.602705291582645\\
2281	0.211907575597109\\
2282	0.951766668166749\\
2283	1.16101541014474\\
2284	-0.0452662799429108\\
2285	0.00719315850348734\\
2286	-1.12569780989241\\
2287	-0.773493677210795\\
2288	-0.134867461030504\\
2289	0.691634635117035\\
2290	0.813361834435843\\
2291	0.923883920684021\\
2292	0.333403892048717\\
2293	-0.531088262120655\\
2294	-0.868332998764467\\
2295	-0.838185832097301\\
2296	-0.448578980792399\\
2297	0.425173341357636\\
2298	1.52742777388978\\
2299	0.916749612649766\\
2300	0.528689181900319\\
2301	-0.499060607579765\\
2302	-1.04726402974224\\
2303	-0.995822671000033\\
2304	-0.24340196279875\\
2305	0.216186220602135\\
2306	0.514485155758277\\
2307	0.930351376444485\\
2308	0.272077592493362\\
2309	-0.332894123739684\\
2310	-0.910202941314258\\
2311	-0.755281114486441\\
2312	-0.168168466558015\\
2313	0.098903769296563\\
2314	0.587967051027607\\
2315	1.03087683596919\\
2316	0.438302215857704\\
2317	-0.605718462661794\\
2318	-0.652231203222299\\
2319	-0.911055364942346\\
2320	-0.342086198959061\\
2321	0.456855867635536\\
2322	1.09790254057913\\
2323	1.25959346609729\\
2324	0.311113501715036\\
2325	-0.782844799519512\\
2326	-0.702343814926264\\
2327	-1.12605071827881\\
2328	-0.0914285186120258\\
2329	0.263638684760705\\
2330	1.26289845625558\\
2331	0.738081666561836\\
2332	0.120761762838703\\
2333	-0.181410849084018\\
2334	-1.23415847010697\\
2335	-0.766142032394844\\
2336	-0.271762420735339\\
2337	0.363493665160036\\
2338	0.852708126705132\\
2339	1.37893781332765\\
2340	0.285925085627793\\
2341	-0.438117457757483\\
2342	-1.1392046940522\\
2343	-1.10047864162655\\
2344	-0.500339905319574\\
2345	0.778110049999141\\
2346	0.84176816635493\\
2347	0.764020516617885\\
2348	0.269435771373365\\
2349	-0.235691434684352\\
2350	-1.22356916550138\\
2351	-0.778105906297919\\
2352	-0.667654209745022\\
2353	0.366272866533801\\
2354	0.520314739564447\\
2355	1.03837100476272\\
2356	0.524786871946707\\
2357	-0.54941323713032\\
2358	-0.858069146936995\\
2359	-0.882002025417837\\
2360	-0.110352261982927\\
2361	0.424284519853016\\
2362	1.16174500852266\\
2363	0.892311798152097\\
2364	0.224662891816282\\
2365	-0.404053352878444\\
2366	-0.97374070732939\\
2367	-0.72114342658072\\
2368	-0.382302080831123\\
2369	0.20730225064224\\
2370	0.894485153852759\\
2371	1.1844254222625\\
2372	0.276890620080534\\
2373	-0.542938637259813\\
2374	-0.708939398593169\\
2375	-0.895790141879193\\
2376	-0.450904202885652\\
2377	0.230219703406219\\
2378	0.597978127729865\\
2379	0.952764138899095\\
2380	0.298796315892573\\
2381	-0.324186703260128\\
2382	-1.25508069953163\\
2383	-0.622643769184282\\
2384	-0.260712136868427\\
2385	0.0449756947639322\\
2386	1.16568385588403\\
2387	0.786955095006013\\
2388	0.437369099919514\\
2389	-0.346442849855111\\
2390	-1.12565905043744\\
2391	-1.55876428767185\\
2392	-0.249892111867952\\
2393	0.319013140714628\\
2394	0.795347639385734\\
2395	1.01218666334849\\
2396	0.215402058718512\\
2397	-0.245355405466191\\
2398	-0.895368270786516\\
2399	-1.14027921882296\\
2400	-0.182473132904066\\
2401	0.936052391280682\\
2402	1.22620965987108\\
2403	0.797916234064247\\
2404	0.780999018754321\\
2405	0.968257776807517\\
2406	0.903002450049678\\
2407	0.743276177215165\\
2408	0.325620124990151\\
2409	0.0154870360363162\\
2410	-0.046178989558428\\
2411	-8.42037653515093e-05\\
2412	-0.459227745769321\\
2413	-0.810949580713104\\
2414	-1.07456533189584\\
2415	-0.88583279352635\\
2416	-1.08967518519328\\
2417	-0.844782817108985\\
2418	-0.579578648075665\\
2419	-0.351127985139472\\
2420	-0.192550494310072\\
2421	0.24007814568324\\
2422	0.606527752516758\\
2423	0.0150478092879028\\
2424	0.259146825457481\\
2425	0.623399335204238\\
2426	0.498906693628658\\
2427	0.917755014478377\\
2428	0.906063236858527\\
2429	0.954182086896042\\
2430	0.817771411283431\\
2431	0.404410772228137\\
2432	0.142160676879227\\
2433	0.245821719799994\\
2434	-0.277738101617806\\
2435	-0.746552452275737\\
2436	-0.654872628280092\\
2437	-0.7841866959625\\
2438	-1.03798000550551\\
2439	-0.84917541167376\\
2440	-0.930817375371785\\
2441	-0.945771820143866\\
2442	-0.763588368284369\\
2443	-0.455429447457228\\
2444	-0.422943480842003\\
2445	-0.30434984483157\\
2446	0.157450215397225\\
2447	0.468606174006564\\
2448	0.771178648782408\\
2449	0.823126170131848\\
2450	0.843652345810907\\
2451	1.13111402108727\\
2452	0.955937409161727\\
2453	1.07873719431824\\
2454	0.733638371520065\\
2455	0.55655293030552\\
2456	0.303606983353954\\
2457	-0.137879113795235\\
2458	-0.3996991319806\\
2459	-0.654806702406678\\
2460	-0.25777155507074\\
2461	-0.874665304247149\\
2462	-0.73555657508905\\
2463	-0.919532653346282\\
2464	-0.565274579966092\\
2465	-0.690682615238133\\
2466	-1.02830325037298\\
2467	-0.631190266369637\\
2468	-0.124631629639798\\
2469	-0.00378486879607304\\
2470	0.699170186107727\\
2471	0.428988721254206\\
2472	0.309904432194192\\
2473	0.929243763995023\\
2474	0.741955041922523\\
2475	1.00872899664484\\
2476	1.22258687249403\\
2477	0.967105948138542\\
2478	0.820787202384495\\
2479	0.513827602441201\\
2480	0.29139719990002\\
2481	0.0452096769572956\\
2482	-0.818620466781345\\
2483	-0.35169880141254\\
2484	-0.43558257634393\\
2485	-1.12025034955816\\
2486	-1.28658582989288\\
2487	-0.858697184888557\\
2488	-1.31869479656171\\
2489	-1.27395712112874\\
2490	-0.577177943510792\\
2491	-0.704375713227111\\
2492	-0.0455801733020876\\
2493	-0.35745659199676\\
2494	0.0547874641255156\\
2495	0.49379192151272\\
2496	0.780112095190884\\
2497	0.934513987494896\\
2498	0.988845157123741\\
2499	0.946241001493392\\
2500	0.85172606601214\\
2501	1.0056636329449\\
2502	0.66716927876198\\
2503	0.652309735135879\\
2504	0.567569353051619\\
2505	-0.435078746194258\\
2506	0.211538268109399\\
2507	-0.581147498735045\\
2508	-0.362583350756097\\
2509	-0.482653268231068\\
2510	-1.27471205146549\\
2511	-0.958071443111818\\
2512	-0.205666082027206\\
2513	-0.94617095512977\\
2514	-0.644914461958584\\
2515	-0.483915792737591\\
2516	-0.0404334613173568\\
2517	0.175759532450319\\
2518	-0.019529167581102\\
2519	0.341578621350295\\
2520	0.733984844473951\\
2521	0.816797729387803\\
2522	1.51327550291666\\
2523	1.09582624819643\\
2524	1.07925003959494\\
2525	0.710687569306687\\
2526	0.461142554041431\\
2527	0.735673541595971\\
2528	0.280019965435243\\
2529	0.213541638073283\\
2530	-0.378333313678499\\
2531	-0.239209758699543\\
2532	-1.06299963268808\\
2533	-0.479276128903388\\
2534	-0.756571894554144\\
2535	-0.973016200997249\\
2536	-1.24117299429704\\
2537	-0.987102166900672\\
2538	-0.69682859096563\\
2539	-0.287682908836733\\
2540	-0.555448526037896\\
2541	0.221552025929778\\
2542	0.811714886228926\\
2543	0.328857224142982\\
2544	0.762079162113514\\
2545	0.68277786255378\\
2546	0.561781879943064\\
2547	1.32673803827313\\
2548	1.13667613838785\\
2549	0.783858646964354\\
2550	0.779108080202001\\
2551	0.506534933309629\\
2552	-0.00294181708454566\\
2553	-0.576731883876741\\
2554	-0.202888985301642\\
2555	-0.543195805150013\\
2556	-0.78622053739946\\
2557	-1.01917724979197\\
2558	-1.35266045594192\\
2559	-1.11441897664821\\
2560	-0.945216725685909\\
2561	-1.13575308623641\\
2562	-0.65292325385821\\
2563	-0.619007491765421\\
2564	-0.346866850598747\\
2565	-0.388078870376708\\
2566	0.0212078472134283\\
2567	0.531485799809387\\
2568	0.480408759063721\\
2569	0.866038203308695\\
2570	1.38861470915066\\
2571	1.1756305825462\\
2572	1.21348372961564\\
2573	1.42633455693122\\
2574	1.01676324174035\\
2575	0.646966982649268\\
2576	0.760812222272496\\
2577	-0.134168426636463\\
2578	-0.344918148574158\\
2579	0.0119459529982566\\
2580	-0.555506700825947\\
2581	-0.839289223996575\\
2582	-0.843472795606177\\
2583	-1.05677649892279\\
2584	-1.04668288772495\\
2585	-0.984699325308919\\
2586	-0.877706406247789\\
2587	-0.769147590242856\\
2588	-0.337556258552288\\
2589	-0.196877243583537\\
2590	0.514923848191678\\
2591	0.492411544078042\\
2592	0.885741976069324\\
2593	0.798576032478076\\
2594	0.927751640995072\\
2595	1.27923591624316\\
2596	1.30996996539661\\
2597	0.905513449784377\\
2598	1.13247563519769\\
2599	0.664419098940563\\
2600	0.406632855589783\\
2601	0.558798798776884\\
2602	0.1513875494191\\
2603	-0.236679677111717\\
2604	-0.567161922564405\\
2605	-0.950345842451072\\
2606	-0.971783897385831\\
2607	-0.589940969254801\\
2608	-1.43371370461558\\
2609	-1.04068423493808\\
2610	-0.795984571361962\\
2611	-0.524780116895994\\
2612	-0.72335860648422\\
2613	0.0501194885519856\\
2614	0.355565485654246\\
2615	0.229323003579174\\
2616	0.599156213764155\\
2617	0.558301181713374\\
2618	1.30954138491014\\
2619	1.08125083560609\\
2620	0.925198358593898\\
2621	0.691044258344691\\
2622	0.775812472933411\\
2623	0.934867651849717\\
2624	0.385760793045916\\
2625	0.147333575932358\\
2626	-0.460885723184491\\
2627	-0.402529447324662\\
2628	-0.483490418120998\\
2629	-1.1390035982929\\
2630	-0.693515779597782\\
2631	-0.445952631862492\\
2632	-1.32997206369551\\
2633	-1.36061462244334\\
2634	-0.769980416458736\\
2635	-0.388087679859497\\
2636	-0.342438216370627\\
2637	-0.103737801316796\\
2638	0.193720850390098\\
2639	0.201468571329521\\
2640	0.289305338509461\\
2641	1.18378502996735\\
2642	1.0114820131899\\
2643	0.73190454128558\\
2644	1.22211489137309\\
2645	0.607817287069185\\
2646	0.72648471440248\\
2647	0.723730186206745\\
2648	0.13602842655887\\
2649	-0.0099294855955071\\
2650	-0.185635711983197\\
2651	-0.00280514562684286\\
2652	-0.554459335379414\\
2653	-0.919224863234008\\
2654	-0.996437454920444\\
2655	-1.06038802705754\\
2656	-0.688925021648097\\
2657	-0.961779193648523\\
2658	-0.930902634016472\\
2659	-0.504115491421361\\
2660	-0.345343280185477\\
2661	0.109234578726128\\
2662	0.40696926724952\\
2663	0.6829764993149\\
2664	0.802002072595228\\
2665	0.971418713310395\\
2666	1.37191459312225\\
2667	0.738077050190428\\
2668	0.90362214155357\\
2669	0.518821881438454\\
2670	0.959473761659804\\
2671	0.562779213553447\\
2672	0.25717736528262\\
2673	-0.31685263967626\\
2674	-0.151218501130886\\
2675	-0.497828244759233\\
2676	-0.960352711589405\\
2677	-1.08887854981773\\
2678	-1.1165177195324\\
2679	-1.05367837784105\\
2680	-1.35030147265468\\
2681	-0.938867675498889\\
2682	-0.61595639847248\\
2683	-0.642252816776251\\
2684	-0.480369208872338\\
2685	-0.000218533962268869\\
2686	0.438665430463707\\
2687	0.339764035013208\\
2688	0.552738761026341\\
2689	0.50856049362924\\
2690	0.727208616813644\\
2691	1.13325311990482\\
2692	0.955909195300149\\
2693	1.06923489150543\\
2694	0.903643419302794\\
2695	0.64205352309152\\
2696	0.495724383219793\\
2697	0.409195988523012\\
2698	-0.0631855098884721\\
2699	-0.344381707247168\\
2700	-0.666786276638757\\
2701	-1.10785972170228\\
2702	-1.05899872091353\\
2703	-0.92474783183713\\
2704	-1.23005605988266\\
2705	-0.793372093555829\\
2706	-0.681540321975993\\
2707	-0.535578266432467\\
2708	-0.256708716495904\\
2709	-0.0629435535373388\\
2710	0.187331336238879\\
2711	0.437745736016739\\
2712	0.865171880949965\\
2713	0.809630323316491\\
2714	0.354967772456274\\
2715	0.888876332548296\\
2716	1.02519661712786\\
2717	1.11290777313747\\
2718	1.06262213749082\\
2719	0.347281063914035\\
2720	0.494429885905688\\
2721	0.121516416955332\\
2722	-0.238896356659788\\
2723	-0.434169461229762\\
2724	-0.663465424143155\\
2725	-0.798457743887295\\
2726	-0.892834246818194\\
2727	-1.14794514303152\\
2728	-1.21519189953161\\
2729	-0.682023555986553\\
2730	-0.80291127307061\\
2731	-0.241417374725827\\
2732	-0.530977169082374\\
2733	-0.201231101457785\\
2734	0.510318507739588\\
2735	0.754698790542258\\
2736	0.42691032886471\\
2737	0.876080314545664\\
2738	1.24611379543537\\
2739	1.29590070816902\\
2740	0.693810088401073\\
2741	1.26030072064893\\
2742	0.901796416345399\\
2743	0.608836365388555\\
2744	0.273684254370791\\
2745	-0.147894383417543\\
2746	-0.171003711286226\\
2747	-0.497156953065338\\
2748	-0.140220028143464\\
2749	-0.787335788299029\\
2750	-0.984805270501286\\
2751	-1.00080591209241\\
2752	-1.11846075182769\\
2753	-0.434516387184504\\
2754	-0.611181655512116\\
2755	-0.541201277570866\\
2756	-0.0727997304883231\\
2757	-0.0756881274655712\\
2758	0.170392829084147\\
2759	0.315145662792883\\
2760	0.519443390099099\\
2761	1.18893270665982\\
2762	1.11797431452107\\
2763	1.01712045119978\\
2764	1.3627427091236\\
2765	0.903430737275798\\
2766	0.818639879857381\\
2767	0.240217169249123\\
2768	0.430519786889404\\
2769	-0.25560500693064\\
2770	-0.413423161315958\\
2771	-0.227254255216806\\
2772	-0.578582626417211\\
2773	-1.02506559771642\\
2774	-1.03509362738109\\
2775	-0.940281555740227\\
2776	-0.932105601706024\\
2777	-1.2811266562584\\
2778	-1.02355720772847\\
2779	-0.954702048731899\\
2780	-0.340385600909365\\
2781	-0.453424367492425\\
2782	0.349333095478197\\
2783	0.214702825790366\\
2784	0.409583991200893\\
2785	1.23055712255745\\
2786	1.42989751832408\\
2787	0.814329345142066\\
2788	0.973051780918752\\
2789	1.33742129694788\\
2790	0.632834012557123\\
2791	0.94230247903883\\
2792	0.507619467391529\\
2793	0.0806991902045883\\
2794	-0.271288805311294\\
2795	-0.0608141053156296\\
2796	-0.498060981298958\\
2797	-0.971466549794802\\
2798	-0.858117106973166\\
2799	-1.19413602866604\\
2800	-0.948011827445006\\
2801	-0.555339628103195\\
2802	-0.634717921696216\\
2803	-0.727958621950656\\
2804	-0.246291896539616\\
2805	-0.224428466679243\\
2806	0.644147226176031\\
2807	0.0383130586379294\\
2808	0.509595195942448\\
2809	1.00250234318881\\
2810	0.762425859984919\\
2811	1.00166955603002\\
2812	1.12781402790783\\
2813	1.17958870795478\\
2814	0.546273197743408\\
2815	0.519945900460232\\
2816	0.21861302602918\\
2817	0.0427117157012974\\
2818	-0.109583270591555\\
2819	-0.578138765770953\\
2820	-0.478130156085956\\
2821	-0.734381516468496\\
2822	-1.05169884068901\\
2823	-0.93591430527062\\
2824	-0.761093223052008\\
2825	-0.675905256691228\\
2826	-0.604203095057338\\
2827	-0.57250626736071\\
2828	-0.462928616482189\\
2829	-0.0181249287450512\\
2830	0.0879551738318081\\
2831	0.365391464907103\\
2832	0.864133473191875\\
2833	1.15368129763194\\
2834	1.07933586570872\\
2835	1.45714565053774\\
2836	0.639128396669057\\
2837	0.903218979531239\\
2838	0.775057102139943\\
2839	0.919258273288899\\
2840	0.436337958083874\\
2841	-0.0041938939051003\\
2842	-0.0682446152744103\\
2843	-0.295940140958724\\
2844	-0.616311532438698\\
2845	-1.28357408418893\\
2846	-0.717056913905588\\
2847	-1.77900074030977\\
2848	-0.956841210661321\\
2849	-1.24529757974639\\
2850	-0.833200761045093\\
2851	-0.244993002005957\\
2852	-0.439068462874402\\
2853	0.360750003872754\\
2854	0.366458763193472\\
2855	0.38455406784895\\
2856	0.454987695109236\\
2857	0.718894612022122\\
2858	1.02980519277105\\
2859	0.467190946362134\\
2860	1.11898074606044\\
2861	0.960680403878763\\
2862	0.936979266393903\\
2863	1.01615020677656\\
2864	0.251716255523708\\
2865	-0.15115011365501\\
2866	0.0717107815185018\\
2867	-0.239516628543711\\
2868	-0.68916411175627\\
2869	-0.962673505333217\\
2870	-0.890298602006863\\
2871	-1.25048826299694\\
2872	-0.966552110576413\\
2873	-1.04089618367491\\
2874	-0.937078301707617\\
2875	-0.34135318081511\\
2876	-0.317372009395449\\
2877	0.23407241661246\\
2878	-0.0508411606071334\\
2879	0.493471878804987\\
2880	0.423065686325951\\
2881	0.968323282813297\\
2882	1.23804049387288\\
2883	1.22670249679992\\
2884	0.507964263973236\\
2885	1.05069866571833\\
2886	0.755683304413496\\
2887	0.632764370794714\\
2888	0.20166393314443\\
2889	-0.301181051522768\\
2890	0.00752177660518508\\
2891	-0.553174026895268\\
2892	-0.730979597552387\\
2893	-0.7921931210175\\
2894	-0.378331601877341\\
2895	-0.48440077822681\\
2896	-1.27643149710984\\
2897	-1.00424343143106\\
2898	-0.532778868165932\\
2899	-1.02186102559202\\
2900	-0.0457764030569855\\
2901	-0.138818825185326\\
2902	0.505399517446142\\
2903	0.242806485573802\\
2904	0.450749139059685\\
2905	1.34778225162152\\
2906	1.0775861583189\\
2907	0.928347022850788\\
2908	0.750186888865199\\
2909	1.14504883969849\\
2910	0.830291188826091\\
2911	0.0513265456898812\\
2912	0.12202697127293\\
2913	0.732905277363642\\
2914	-0.19489977689946\\
2915	-0.887160688482047\\
2916	-0.855898514177853\\
2917	-0.923832579999704\\
2918	-0.79056893634377\\
2919	-1.03408721059\\
2920	-0.958611973454126\\
2921	-0.902138324661074\\
2922	-0.968635510313813\\
2923	-0.295511551067958\\
2924	-0.405903887216399\\
2925	-0.17197641670814\\
2926	0.235901321983277\\
2927	0.424724347643916\\
2928	0.540471661829934\\
2929	0.858309461113705\\
2930	1.41885050900479\\
2931	1.11879396641019\\
2932	0.834748294062673\\
2933	0.703559592700143\\
2934	0.700273988277356\\
2935	0.205287907014406\\
2936	0.732728503574966\\
2937	0.113985584849687\\
2938	-0.452610422308071\\
2939	-0.412715617542951\\
2940	-0.980208505612585\\
2941	-0.699550296479332\\
2942	-0.986609259861227\\
2943	-0.988265641951415\\
2944	-0.900370030383475\\
2945	-0.707098459303244\\
2946	-0.934944427091923\\
2947	-0.886256297129713\\
2948	-0.347413293869773\\
2949	-0.252425628308463\\
2950	0.287368696371141\\
2951	0.0246659246399138\\
2952	0.849310406822584\\
2953	0.898431691527187\\
2954	0.697256755206635\\
2955	0.926674701924747\\
2956	1.16883888055491\\
2957	1.44113426695804\\
2958	0.675167860933026\\
2959	0.867215192945106\\
2960	0.131566754859543\\
2961	0.114572556358902\\
2962	-0.238700276567867\\
2963	-0.363563154785715\\
2964	-0.768646323327577\\
2965	-1.14794564733458\\
2966	-1.00659467997506\\
2967	-0.899279126611921\\
2968	-1.0946071181896\\
2969	-0.939395260581988\\
2970	-0.663775362122004\\
2971	-0.437292950552076\\
2972	-0.488326335555552\\
2973	-0.128330836366477\\
2974	0.387088483123182\\
2975	0.513722878440367\\
2976	0.355548076950814\\
2977	1.17624858689567\\
2978	0.808010642080645\\
2979	0.841340952943336\\
2980	0.662771838154059\\
2981	0.873608539567254\\
2982	0.730749524093557\\
2983	0.485743736034431\\
2984	0.136281350408889\\
2985	-0.303321870891777\\
2986	0.481864895352593\\
2987	-0.103848806438563\\
2988	-0.737514054629038\\
2989	-0.855574783578925\\
2990	-0.865686489966699\\
2991	-1.0003826409059\\
2992	-1.10897731624403\\
2993	-0.689124486738109\\
2994	-1.00846589069\\
2995	-0.119971918597277\\
2996	-0.718737439118264\\
2997	-0.14063026104118\\
2998	0.387534646104798\\
2999	0.46609258107801\\
3000	0.320826928922343\\
3001	1.03615466968951\\
3002	0.982531880233366\\
3003	0.815360223723735\\
3004	1.07168459774328\\
3005	1.06735421960489\\
3006	1.24256635932278\\
3007	0.536974400688359\\
3008	0.25832630210184\\
3009	-0.0446901697168998\\
3010	-0.62348888557284\\
3011	-0.32303146621379\\
3012	-1.11691356411084\\
3013	-0.890251190666473\\
3014	-1.00754751011536\\
3015	-0.808540797794659\\
3016	-1.60147527368762\\
3017	-0.692695970480025\\
3018	-0.869932635025422\\
3019	-0.847310112682075\\
3020	-0.32611441156288\\
3021	-0.0160553267753921\\
3022	0.0973405814176081\\
3023	0.110873633180723\\
3024	0.92715090875079\\
3025	0.874028257713194\\
3026	0.787817085708044\\
3027	1.26986415995513\\
3028	1.06350778272973\\
3029	0.614498445336408\\
3030	0.892712394093011\\
3031	0.156175711240519\\
3032	0.0874410877599442\\
3033	0.102187278431762\\
3034	-0.100234410557302\\
3035	-0.423929644503217\\
3036	-1.08975023076158\\
3037	-0.646654288343653\\
3038	-0.663646169178665\\
3039	-0.736439585050204\\
3040	-1.10897065520771\\
3041	0.307407896537563\\
3042	0.70145547484566\\
3043	0.974433594396438\\
3044	0.426158479653523\\
3045	-0.568194829240244\\
3046	-0.723360891482964\\
3047	-0.727747668636796\\
3048	-0.222687083317775\\
3049	0.226944358263614\\
3050	0.62960720059392\\
3051	1.06439710346787\\
3052	0.693595096315298\\
3053	-0.684257218437068\\
3054	-1.42328262842897\\
3055	-0.597071719259793\\
3056	-0.383539161360158\\
3057	0.659122454177412\\
3058	1.14967330708766\\
3059	0.645750330703629\\
3060	0.273092862061368\\
3061	-0.582396387367197\\
3062	-0.888927315982833\\
3063	-0.657448601840911\\
3064	-0.357209724167777\\
3065	-0.074391582720541\\
3066	1.20556967950597\\
3067	0.919040574407847\\
3068	0.266903466636091\\
3069	-0.866200977359213\\
3070	-0.845460104140691\\
3071	-0.863576461958897\\
3072	-0.293971251646032\\
3073	0.154305469135065\\
3074	0.765740882406573\\
3075	0.713387824275362\\
3076	0.769755301815074\\
3077	-0.579143138958618\\
3078	-0.973155864846212\\
3079	-1.117638333497\\
3080	-0.165931678251947\\
3081	0.392809245021121\\
3082	0.94543033446807\\
3083	1.12866450544988\\
3084	0.520842232866526\\
3085	-0.21967060036308\\
3086	-1.40639882724857\\
3087	-0.869227079811421\\
3088	-0.153503673537274\\
3089	0.473875328586799\\
3090	1.13094341110131\\
3091	1.39312043867369\\
3092	0.649145845721417\\
3093	-0.646899491160784\\
3094	-1.1148916875648\\
3095	-0.945694282666554\\
3096	-0.513536676357386\\
3097	0.377886597204265\\
3098	1.20396197869657\\
3099	0.799534158754718\\
3100	0.305713926548205\\
3101	-0.446487347107308\\
3102	-0.778191766022696\\
3103	-0.686419590497869\\
3104	-0.403184802719343\\
3105	0.324597765042012\\
3106	1.13805497576186\\
3107	1.0802986257259\\
3108	0.478983781777988\\
3109	0.335353153788812\\
3110	-0.92204200507705\\
3111	-0.842072534892787\\
3112	-0.388430701476378\\
3113	0.533496126650359\\
3114	0.811652110807491\\
3115	1.14394647903443\\
3116	0.779053114545594\\
3117	-0.376468262242387\\
3118	-0.944425620410641\\
3119	-1.04453454026842\\
3120	-0.47571419973404\\
3121	0.0143384812096307\\
3122	0.917114547405003\\
3123	0.721216335167177\\
3124	0.650780500447763\\
3125	-0.360679865528887\\
3126	-0.916843185607569\\
3127	-0.605475895216315\\
3128	-0.767091256291162\\
3129	0.469323650722185\\
3130	0.692399412564995\\
3131	0.774470152094167\\
3132	0.427571864395028\\
3133	-0.196011377438447\\
3134	-1.11635460723243\\
3135	-0.794659709979373\\
3136	-0.0433766427548598\\
3137	0.204204406931792\\
3138	0.707119638234424\\
3139	0.91582465703326\\
3140	0.510027747615593\\
3141	-0.395218458461555\\
3142	-1.11181285917942\\
3143	-0.892675127088371\\
3144	-0.323400564133472\\
3145	0.296070297597156\\
3146	0.678393754601831\\
3147	1.01586676800878\\
3148	0.373562289176071\\
3149	-0.476782559728367\\
3150	-1.07039374549504\\
3151	-1.03040503159304\\
3152	-0.563797942341656\\
3153	0.162802811489951\\
3154	1.21231312535252\\
3155	0.864409513421221\\
3156	-0.0171829345689012\\
3157	-0.504812071521\\
3158	-0.69184409276162\\
3159	-1.03887036936553\\
3160	0.02799286679269\\
3161	0.22111391578337\\
3162	0.760636572072709\\
3163	0.687249056362262\\
3164	0.630180979278872\\
3165	-0.525369581408595\\
3166	-0.835752408242307\\
3167	-1.28725872380236\\
3168	-0.497891023980591\\
3169	0.543076323895867\\
3170	0.642608818987061\\
3171	0.929659458701019\\
3172	0.492459341163591\\
3173	0.00242530402405283\\
3174	-0.785674043648146\\
3175	-0.706590285164351\\
3176	-0.44901308356381\\
3177	0.266026748862745\\
3178	0.941773522741935\\
3179	0.718535768370382\\
3180	0.62083581785016\\
3181	-0.221939666042107\\
3182	-0.713749541876248\\
3183	-0.636305584498314\\
3184	-0.393888121017856\\
3185	0.266398138758715\\
3186	1.04676157916559\\
3187	1.17622727949491\\
3188	0.56182548375851\\
3189	-0.183953733845032\\
3190	-0.592095416825519\\
3191	-0.852352274486407\\
3192	-0.100286446397066\\
3193	0.392903107772734\\
3194	1.10580967432027\\
3195	0.60435033929697\\
3196	0.714829358487141\\
3197	-0.792741045491052\\
3198	-0.883114332034395\\
3199	-0.995917294212169\\
3200	-0.629094380697888\\
3201	0.143425711286557\\
3202	-0.117917662820142\\
3203	-0.123906376328669\\
3204	-0.626552155414181\\
3205	-0.700606066713868\\
3206	-0.891603058071626\\
3207	-0.669232388470788\\
3208	-0.728348317013209\\
3209	-0.895758018620001\\
3210	-0.725911065153235\\
3211	-0.15976142705382\\
3212	-0.377752380268937\\
3213	0.123726021431957\\
3214	-0.00429074228505588\\
3215	0.269197961294826\\
3216	0.850251982958416\\
3217	0.974944347643873\\
3218	1.18278004224406\\
3219	0.855971537237622\\
3220	1.20914749097433\\
3221	0.93661535137483\\
3222	0.743110640185628\\
3223	0.935956478122248\\
3224	0.645744048945146\\
3225	-0.435469554926834\\
3226	-0.112491898594501\\
3227	-0.6299718901604\\
3228	-0.724212491990273\\
3229	-0.971798784950067\\
3230	-0.582476946775067\\
3231	-0.719940961355777\\
3232	-0.857082509803716\\
3233	-0.74073179254254\\
3234	-0.651206957481895\\
3235	-0.404899635171829\\
3236	-0.267930121190912\\
3237	-0.108220232575891\\
3238	0.305885169395619\\
3239	0.370356623005186\\
3240	0.568927391355125\\
3241	0.76950533131419\\
3242	0.926279382644053\\
3243	0.999463338321975\\
3244	1.54790572283162\\
3245	0.624405707974732\\
3246	0.513898933332982\\
3247	0.613023765905738\\
3248	0.312837692082834\\
3249	-0.0951176844153402\\
3250	-0.119093831306087\\
3251	-0.271201927121477\\
3252	-0.915358023995976\\
3253	-0.921139320379935\\
3254	-1.23420765492287\\
3255	-0.978963612284879\\
3256	-0.944687841067309\\
3257	-0.745395998989427\\
3258	-0.996840748829945\\
3259	-0.49138839132573\\
3260	-0.356232880963678\\
3261	0.243215390998786\\
3262	0.260910385405219\\
3263	0.319494965007819\\
3264	0.605948963639011\\
3265	0.709872252838408\\
3266	1.26802400237494\\
3267	0.883293111388739\\
3268	0.621766469333468\\
3269	0.853107985588962\\
3270	1.00962654726752\\
3271	0.706391676819387\\
3272	0.360882140799714\\
3273	0.467811275778111\\
3274	-0.323784287827913\\
3275	-0.381738658408723\\
3276	-0.852277365995352\\
3277	-1.00618646269247\\
3278	-1.02305712283007\\
3279	-0.8527433658535\\
3280	-0.521672012593088\\
3281	-0.719355485613653\\
3282	-0.993419131339929\\
3283	-0.603870493264814\\
3284	-0.448236279357784\\
3285	-0.133829337816107\\
3286	0.456378009948408\\
3287	0.524158716433541\\
3288	0.879072867333973\\
3289	0.964907124286952\\
3290	0.768569710198529\\
3291	1.20596420716432\\
3292	0.679153980808631\\
3293	1.07711816598191\\
3294	0.887920690107098\\
3295	0.80421399238318\\
3296	0.0064548960319275\\
3297	0.0227052787777786\\
3298	-0.188258154871695\\
3299	-0.417249020007483\\
3300	-0.461068903907703\\
3301	-0.673066447988857\\
3302	-0.800093554802108\\
3303	-0.957715179361532\\
3304	-0.60628655855095\\
3305	-0.970891995758633\\
3306	-0.731099576168955\\
3307	-0.687012661148833\\
3308	-0.337932191181886\\
3309	0.0940966110455287\\
3310	-0.0823867598037447\\
3311	0.482431860136557\\
3312	0.492742205204816\\
3313	1.00629118862753\\
3314	0.506189551758558\\
3315	0.560061049241879\\
3316	1.34482569641628\\
3317	0.86470488243006\\
3318	0.313277027060633\\
3319	0.180638595245461\\
3320	0.304712224601463\\
3321	0.137061378062249\\
3322	-0.11673106225799\\
3323	-0.200935091204685\\
3324	-0.58267385826946\\
3325	-0.846311294787715\\
3326	-0.711534964523263\\
3327	-0.536936011181624\\
3328	-0.975369291166978\\
3329	-0.951966289155432\\
3330	-0.52134527843419\\
3331	-0.671467410936858\\
3332	-0.656045768386732\\
3333	-0.0921045586472413\\
3334	0.247151675489566\\
3335	0.472331015447154\\
3336	0.406975041274407\\
3337	0.69628421878605\\
3338	0.79394580900885\\
3339	1.21138780688005\\
3340	0.952072274528807\\
3341	1.04346507487503\\
3342	0.887669524214183\\
3343	0.514883369982468\\
3344	-0.119097975637918\\
3345	0.149492796305196\\
3346	-0.0347374974693068\\
3347	-0.608121066358048\\
3348	-0.660964998281895\\
3349	-0.928702413133135\\
3350	-0.565277856371541\\
3351	-1.00450730778088\\
3352	-0.935710168250705\\
3353	-1.13262003364767\\
3354	-0.918294888742807\\
3355	-0.861531372843671\\
3356	-0.161121238397686\\
3357	0.0676009588500546\\
3358	0.00691374361905142\\
3359	1.08590010066862\\
3360	0.862862542539\\
3361	0.920674304768681\\
3362	1.37480303767406\\
3363	0.448134927725852\\
3364	0.960964615187397\\
3365	0.89261909445803\\
3366	0.798475864492516\\
3367	0.325607825865333\\
3368	0.442624733358787\\
3369	-0.427870126576769\\
3370	-0.168229180484272\\
3371	-0.341486811504386\\
3372	-0.666352690448074\\
3373	-0.74834239718825\\
3374	-1.11690066421289\\
3375	-0.874694540703165\\
3376	-1.06517878617135\\
3377	-1.26942711666019\\
3378	-0.823157036354215\\
3379	-0.696691288205687\\
3380	-0.553445607305218\\
3381	0.0447333403331365\\
3382	0.201323763496278\\
3383	0.382702268693341\\
3384	0.534607354795276\\
3385	1.2050539207238\\
3386	1.2084946376763\\
3387	0.955358578143802\\
3388	1.00071585648784\\
3389	1.23662135542684\\
3390	1.15503090979398\\
3391	0.804704441374503\\
3392	0.251245624655338\\
3393	-0.0621470288262616\\
3394	-0.0240771884135555\\
3395	-0.632991290795675\\
3396	-0.896818825775148\\
3397	-0.909361768641534\\
3398	-0.732095177964014\\
3399	-0.962022644956748\\
3400	-0.926529841514518\\
3401	-0.646446470854545\\
3402	-0.701229519199938\\
3403	-0.812747496278986\\
3404	-0.135720503598599\\
3405	0.0383205747009995\\
3406	0.172486957718017\\
3407	0.347276824490949\\
3408	0.535312843442521\\
3409	0.824494992050012\\
3410	0.766862826995079\\
3411	0.883049040533745\\
3412	0.648655785938652\\
3413	0.782080743268715\\
3414	0.598604157229066\\
3415	0.687227387977648\\
3416	0.649754723604316\\
3417	0.299994683972644\\
3418	0.0419905222169101\\
3419	-0.288383771300757\\
3420	-0.534862755868711\\
3421	-0.662035375204947\\
3422	-1.10932380594829\\
3423	-0.811213236935505\\
3424	-1.24955327966455\\
3425	-0.759673384309851\\
3426	-0.940010925023865\\
3427	-0.775137459950099\\
3428	-0.425649735230038\\
3429	-0.359793536986104\\
3430	0.19584155187717\\
3431	0.688247920860225\\
3432	0.601374549158872\\
3433	0.940794866977457\\
3434	0.982261684827\\
3435	1.26884679836179\\
3436	0.945914982958375\\
3437	0.963702776946435\\
3438	1.30595732851255\\
3439	0.814464170543608\\
3440	0.115188885790266\\
3441	0.318860876930587\\
3442	-0.399089379844554\\
3443	-0.43748860679991\\
3444	-0.592424958712671\\
3445	-0.635178663846729\\
3446	-1.034899972989\\
3447	-0.877305209439735\\
3448	-0.585312571186575\\
3449	-1.32454974578619\\
3450	-0.994688107162793\\
3451	-0.290558144405291\\
3452	-0.389943013949958\\
3453	-0.076818183528877\\
3454	0.399023997758514\\
3455	0.937180264048764\\
3456	0.365842201207278\\
3457	0.626454008875957\\
3458	0.928658155362937\\
3459	0.997780781743117\\
3460	1.00525472362247\\
3461	0.828218932558317\\
3462	1.02666582215609\\
3463	0.354420120566975\\
3464	0.369809197865279\\
3465	0.289759607727803\\
3466	-0.108304891129952\\
3467	-0.578924438748861\\
3468	-0.630201529987814\\
3469	-0.900893797225009\\
3470	-1.08019780598706\\
3471	-0.694858950856317\\
3472	-1.02557660134366\\
3473	-1.02748674890363\\
3474	-0.922904547353392\\
3475	-0.530839185528484\\
3476	-0.658901565755306\\
3477	-0.436650788871861\\
3478	0.235433158873034\\
3479	0.151082063976244\\
3480	0.567701369626053\\
3481	1.00338620528261\\
3482	1.15235899570237\\
3483	1.34330144036108\\
3484	0.755423742053094\\
3485	1.08985498071623\\
3486	0.826526688950871\\
3487	0.838230630625107\\
3488	0.373430380523695\\
3489	0.250698789995585\\
3490	0.0627121813723014\\
3491	-0.531837896706279\\
3492	-0.760355967366505\\
3493	-1.12728052851184\\
3494	-0.965581790282398\\
3495	-0.918125161370533\\
3496	-1.14764164290264\\
3497	-0.882023009938543\\
3498	-0.833489983883861\\
3499	-0.654564587868614\\
3500	-0.149766103131781\\
3501	0.114401616163382\\
3502	0.449233199989943\\
3503	0.196914187984889\\
3504	0.626717897734378\\
3505	0.976463043418977\\
3506	0.656551488853507\\
3507	1.42883075010781\\
3508	1.033539578758\\
3509	0.776177347141776\\
3510	0.984856231122886\\
3511	0.642356266274738\\
3512	0.348612547801622\\
3513	-0.0839371577238804\\
3514	-0.587324693375193\\
3515	-0.655331923374602\\
3516	-0.80165833046802\\
3517	-0.728600915369099\\
3518	-0.808489199055266\\
3519	-1.17340045065434\\
3520	-0.796014158872175\\
3521	-0.7487879769778\\
3522	-0.781818828648171\\
3523	-0.770826144660578\\
3524	-0.124543551092451\\
3525	-0.0973037686014694\\
3526	-0.0345300943047692\\
3527	0.49616214639278\\
3528	0.774346689662232\\
3529	0.502784981072466\\
3530	1.36366369416858\\
3531	0.516753563161898\\
3532	0.615410870927551\\
3533	1.17205190758499\\
3534	0.826501507339716\\
3535	0.461979189240398\\
3536	0.326758447018275\\
3537	0.231928597247073\\
3538	-0.386512103065293\\
3539	-0.232002835523183\\
3540	-0.501134283650726\\
3541	-0.839970891841147\\
3542	-0.903212013277603\\
3543	-1.13837058131981\\
3544	-1.17303486292598\\
3545	-0.985127771950862\\
3546	-0.523575625392985\\
3547	-0.611605381273951\\
3548	-0.425227233666117\\
3549	-0.287414351975151\\
3550	-0.0613751207653362\\
3551	0.444655771771293\\
3552	0.889963109194865\\
3553	1.10544762370244\\
3554	1.07135753867022\\
3555	0.928510149984397\\
3556	0.899080584154083\\
3557	0.979963089531379\\
3558	0.963513565894559\\
3559	0.775976383423402\\
3560	-0.150007478730077\\
3561	0.0578769468098985\\
3562	-0.408612114062625\\
3563	-0.545161273697501\\
3564	-0.26887153702487\\
3565	-1.00386220323557\\
3566	-1.3227997933059\\
3567	-0.958051583471147\\
3568	-1.00919818105337\\
3569	-0.991134477360906\\
3570	-0.556929946325251\\
3571	-0.431273322418363\\
3572	-0.240041107495033\\
3573	-0.130251272656507\\
3574	0.270611468602868\\
3575	0.374517115774528\\
3576	0.698819608077341\\
3577	0.958898109982804\\
3578	1.02200928304597\\
3579	0.84710189206239\\
3580	0.861170251211237\\
3581	0.665376377847749\\
3582	0.976905767692601\\
3583	0.552130646685398\\
3584	0.177140776671625\\
3585	0.219940109963394\\
3586	-0.306580742796621\\
3587	-0.339899050451634\\
3588	-0.317788049811851\\
3589	-0.620016904585641\\
3590	-0.981842140847085\\
3591	-1.02314114965485\\
3592	-0.547854300169929\\
3593	-0.936363181724419\\
3594	-0.686339117889716\\
3595	-0.485011512521186\\
3596	-0.117204341145712\\
3597	-0.0200971025263423\\
3598	0.274225676104146\\
3599	0.511796226844539\\
3600	1.25892208188837\\
3601	1.02591311496193\\
3602	0.74986090418207\\
3603	0.725227326619643\\
3604	0.619975283053793\\
3605	1.30991358305565\\
3606	0.570475857996595\\
3607	0.779980211173875\\
3608	0.443620763909011\\
3609	-0.273374877591321\\
3610	0.260406940658009\\
3611	-0.316113271058692\\
3612	-0.249869339198263\\
3613	-0.7531905026666\\
3614	-0.905818962844453\\
3615	-1.22576870609372\\
3616	-0.972928730284381\\
3617	-0.867456640697195\\
3618	-1.0941141003688\\
3619	-0.562194531583012\\
3620	-0.287632285048392\\
3621	-0.2296532569827\\
3622	0.284151187405036\\
3623	0.168344048117025\\
3624	0.528787134214489\\
3625	0.85345176028068\\
3626	0.810593208090162\\
3627	1.09205538748503\\
3628	0.900153636674189\\
3629	0.702533241261481\\
3630	0.962205619822859\\
3631	0.692237866296321\\
3632	0.452022081379299\\
3633	0.221588148250421\\
3634	0.0336205236084486\\
3635	-0.217969391917477\\
3636	-0.649639983248753\\
3637	-0.930628222523626\\
3638	-0.970763641173183\\
3639	-0.72065960472143\\
3640	-1.2493987958542\\
3641	-1.02444155466814\\
3642	-0.520263658488432\\
3643	-0.368643447690071\\
3644	-0.435054142688453\\
3645	-0.114556273936317\\
3646	0.157402795800631\\
3647	0.378077236298858\\
3648	0.110431060359505\\
3649	0.734241253914575\\
3650	1.01516233988295\\
3651	1.54627791990319\\
3652	0.654037085363309\\
3653	0.755921038026671\\
3654	0.667473572034337\\
3655	0.347098927470075\\
3656	0.173387126335661\\
3657	-0.362330841052422\\
3658	-0.217423445517908\\
3659	-0.555737172144423\\
3660	-0.537461302802906\\
3661	-0.817020499501813\\
3662	-0.707493440845261\\
3663	-0.634979527729056\\
3664	-0.954074392186328\\
3665	-1.1751896831675\\
3666	-0.699104140692161\\
3667	-0.245059965143674\\
3668	-0.345528423714753\\
3669	0.0184670937682718\\
3670	-0.0239743092024576\\
3671	0.777429966649632\\
3672	0.747548112666574\\
3673	0.735400782165933\\
3674	0.57138941315192\\
3675	0.843789277788336\\
3676	0.75148004829772\\
3677	0.78791788499962\\
3678	0.831342813100923\\
3679	0.743552751104556\\
3680	0.358990804124105\\
3681	0.0870082024388582\\
3682	-0.108435303922096\\
3683	0.154025692472\\
3684	-0.582724779494643\\
3685	-0.798365613971612\\
3686	-0.592914532355304\\
3687	-0.977627638788774\\
3688	-1.13194241963055\\
3689	-0.689603667847379\\
3690	-0.833822751240361\\
3691	-0.512734845126472\\
3692	-0.128076940988007\\
3693	0.112527187326492\\
3694	-0.166525943155596\\
3695	-0.088872332591302\\
3696	0.500153382609152\\
3697	1.19952866323741\\
3698	0.998605472025058\\
3699	0.963707449326988\\
3700	0.946458088458412\\
3701	1.13046156849361\\
3702	0.827689725210813\\
3703	0.664436569139514\\
3704	-0.143647190748759\\
3705	0.0293294262927658\\
3706	-0.187972812702666\\
3707	-0.632161747333492\\
3708	-0.662314827667089\\
3709	-0.964653219658043\\
3710	-0.751335211825772\\
3711	-1.12540246900784\\
3712	-0.689382663329519\\
3713	-0.90768486179984\\
3714	-0.553333254909554\\
3715	-0.0466578836517842\\
3716	-0.309208911030615\\
3717	0.266183137302036\\
3718	-0.0477283886845118\\
3719	0.847732051093087\\
3720	0.386757998497464\\
3721	0.815233913078526\\
3722	0.87286192700796\\
3723	0.740510611579128\\
3724	0.688630272190117\\
3725	1.0427889900566\\
3726	0.442859591210438\\
3727	0.257944192277803\\
3728	0.576781482162027\\
3729	0.068654344792636\\
3730	-0.107301394843978\\
3731	-0.641404784256925\\
3732	-0.701217688033891\\
3733	-0.761180083865689\\
3734	-0.908527766359728\\
3735	-0.597531839786378\\
3736	-1.12099933207015\\
3737	-0.931700097543301\\
3738	-0.567501991492847\\
3739	-0.345797975235164\\
3740	-0.765844759541289\\
3741	-0.15596905030249\\
3742	0.0408328768592567\\
3743	0.43612041485526\\
3744	0.772464586554368\\
3745	0.929410779436893\\
3746	1.20131030684457\\
3747	1.09784822068728\\
3748	1.05193164898629\\
3749	1.10974562289249\\
3750	0.8496602068931\\
3751	0.261541592298808\\
3752	0.348739248473731\\
3753	-0.30201743060173\\
3754	0.0562785252337104\\
3755	-0.911401042109152\\
3756	-0.780221251477393\\
3757	-0.808773600016498\\
3758	-0.95455386927604\\
3759	-0.576306711821912\\
3760	-0.968931762272019\\
3761	-1.03551777474893\\
3762	-0.622372284921926\\
3763	-0.827997244659154\\
3764	-0.202129892107843\\
3765	-0.0121218037252161\\
3766	0.104378039363991\\
3767	0.300505576965336\\
3768	0.390775381416514\\
3769	0.409505556774942\\
3770	0.85111684310366\\
3771	1.17159707720026\\
3772	0.821580967473622\\
3773	1.25275125868862\\
3774	0.554948816707082\\
3775	1.03703394539718\\
3776	0.169268467488888\\
3777	0.0904954604600937\\
3778	0.128851498310046\\
3779	-0.163696138520053\\
3780	-0.826724256213424\\
3781	-0.664588813771115\\
3782	-1.19537186205926\\
3783	-1.37940445172732\\
3784	-0.833388371303965\\
3785	-1.285199756766\\
3786	-1.23219057811204\\
3787	-0.571398390779661\\
3788	-0.0140748250884938\\
3789	0.204246984649557\\
3790	-0.142294445960661\\
3791	0.448048757323822\\
3792	0.836844965632523\\
3793	1.04779529480692\\
3794	1.29526073825599\\
3795	1.35263609480338\\
3796	1.17296427644704\\
3797	0.579372405207527\\
3798	0.762618904812912\\
3799	0.249165556330129\\
3800	0.600099016519906\\
3801	0.458115121434777\\
3802	-0.640159552846608\\
3803	-0.252738788689669\\
3804	-0.733894807771839\\
3805	-0.936825129938529\\
3806	-1.14496341534807\\
3807	-0.714937487114998\\
3808	-0.895590283011635\\
3809	-0.924742007971877\\
3810	-0.660363143320622\\
3811	-0.331223082148296\\
3812	-0.518743350883494\\
3813	-0.246107691143567\\
3814	0.346339166161612\\
3815	0.794546083475773\\
3816	0.9369431552022\\
3817	0.803997535114886\\
3818	0.647978805471315\\
3819	0.473395563153904\\
3820	0.773997170010747\\
3821	1.15091560829062\\
3822	0.630851070296985\\
3823	0.621454209897041\\
3824	0.371779054754635\\
3825	-0.424657649618646\\
3826	0.08769730243883\\
3827	0.013381147172229\\
3828	-0.665883494530543\\
3829	-0.606405013235634\\
3830	-1.1154765503691\\
3831	-0.864261713772117\\
3832	-0.506211048138964\\
3833	-0.604605215571565\\
3834	-0.910366652296057\\
3835	-0.786129149072071\\
3836	-0.312277503521606\\
3837	-0.121314729008499\\
3838	0.178227721877245\\
3839	0.724153263512968\\
3840	0.706665373410634\\
3841	1.20224085422292\\
3842	0.9363875522922\\
3843	0.967851242222832\\
3844	1.15884427304625\\
3845	0.993532872326992\\
3846	0.706762695817758\\
3847	0.202564121821219\\
3848	0.0743407915481958\\
3849	0.206314070330485\\
3850	-0.0706507685411102\\
3851	-0.768190059930071\\
3852	-0.630346239079773\\
3853	-0.711408475708644\\
3854	-1.20899622412861\\
3855	-1.22104629258025\\
3856	-0.714140964652167\\
3857	-1.29058286412064\\
3858	-0.653810400644066\\
3859	-0.339091573714888\\
3860	-0.725009479772758\\
3861	-0.097408368505322\\
3862	-0.0707268619077012\\
3863	0.459837214147396\\
3864	0.562707305474319\\
3865	0.460422708255168\\
3866	1.08428956277511\\
3867	1.1310747824162\\
3868	0.842949338159132\\
3869	1.01626946645484\\
3870	0.452904010115969\\
3871	0.281646267741222\\
3872	0.0660477098374058\\
3873	0.0438094520306064\\
3874	-0.30957057182978\\
3875	-0.0119968766912575\\
3876	-0.673091267350949\\
3877	-0.704140158571098\\
3878	-0.817535809663922\\
3879	-0.914480342684939\\
3880	-0.938445617208983\\
3881	-0.619741676698046\\
3882	-1.01424901553506\\
3883	-0.192021749458784\\
3884	-0.170577226604356\\
3885	-0.325505901929608\\
3886	0.440696758941267\\
3887	0.704338667524901\\
3888	1.03924361429407\\
3889	0.73580936937642\\
3890	1.31285959624334\\
3891	1.05662530827647\\
3892	0.653845739708169\\
3893	0.724451107462656\\
3894	0.961734367570485\\
3895	0.959187497566207\\
3896	0.059398813949924\\
3897	0.0433959490952832\\
3898	-0.339237205743847\\
3899	-0.0190603096289872\\
3900	-0.857703419410567\\
3901	-0.531903997280574\\
3902	-1.08576750690493\\
3903	-0.823862030737449\\
3904	-1.27768304468189\\
3905	-0.562411371597838\\
3906	-1.24669803292423\\
3907	-0.485788917417279\\
3908	-0.229951475792856\\
3909	-0.251930571910819\\
3910	-0.0240065723951809\\
3911	0.576064987166213\\
3912	0.425296735462487\\
3913	0.823548220099869\\
3914	0.314996119180591\\
3915	0.531971099121265\\
3916	0.971327889718868\\
3917	1.14831696836164\\
3918	0.384609874449614\\
3919	0.709832534907712\\
3920	0.434782710079572\\
3921	-0.0634121989892618\\
3922	-0.197620565056376\\
3923	0.0506645385783735\\
3924	-0.970600020557219\\
3925	-0.722430941625917\\
3926	-0.899555566091831\\
3927	-0.925590704798476\\
3928	-0.982658635152935\\
3929	-0.835172970071838\\
3930	-0.809978827616885\\
3931	-0.936921859922774\\
3932	-0.454854117479344\\
3933	-0.0158113963398902\\
3934	0.250177680439135\\
3935	0.57479128494339\\
3936	1.15418491390922\\
3937	0.459707276165927\\
3938	1.09826274377469\\
3939	1.02806231835199\\
3940	0.972841906906306\\
3941	0.743032327875297\\
3942	0.725196913786225\\
3943	0.710211757308945\\
3944	0.419541415531549\\
3945	-0.291040529759236\\
3946	0.0105243271505312\\
3947	-0.383869520111095\\
3948	-0.233290482442737\\
3949	-1.07879431480671\\
3950	-1.10757895476186\\
3951	-1.38976938828971\\
3952	-0.980534345454834\\
3953	-0.675449451457971\\
3954	-0.860917695053573\\
3955	-0.674125026997399\\
3956	-0.390102184560673\\
3957	-0.104202505656488\\
3958	0.421308256004077\\
3959	0.443262487974671\\
3960	0.549170643347197\\
3961	0.97189409725446\\
3962	0.640668318207104\\
3963	1.13162629063832\\
3964	0.935132737803017\\
3965	1.37688750938218\\
3966	0.608543369177656\\
3967	0.652752079488746\\
3968	0.350539289362622\\
3969	-0.17358087962511\\
3970	0.00365787447055849\\
3971	-0.937900946902105\\
3972	-1.09267831383152\\
3973	-1.1229271398898\\
3974	-0.653829875152954\\
3975	-0.777474709815097\\
3976	-1.26137685446836\\
3977	-0.938367120376742\\
3978	-0.920093197683717\\
3979	-0.505766302569896\\
3980	0.146073625286726\\
3981	-0.264454409818372\\
3982	0.628364928933876\\
3983	0.335587373681165\\
3984	0.511293758506191\\
3985	0.966289789295847\\
3986	1.10468823028562\\
3987	1.06027906945087\\
3988	0.652956669017052\\
3989	0.743439588435373\\
3990	0.559496649844134\\
3991	0.443272441804632\\
3992	0.149148850939095\\
3993	0.163557513824954\\
3994	0.0340949254372442\\
3995	-0.495799702820903\\
3996	-0.533152757742212\\
3997	-0.660045503983731\\
3998	-0.916049882015202\\
3999	-1.05866012850738\\
4000	-1.10277144906389\\
};
\end{axis}
\end{tikzpicture}%

\section{Démodulation par filtrage}

On souhaite désormais reconstituer le signal de départ. Pour cela, nous allons procéder à un filtrage passe bas d'une part et passe haut d'autre part, avec une fréquence de coupure $F_c=\frac{F_0+F_1}{2}$. Nous ferons ensuite passer chacun des signaux filtrés par un détecteur d'énergie qui permettra de reproduire de signal binaire initial de manière fidèle.

\paragraph{Filtre passe haut}
Pour le passe haut, nous allons utiliser un filtre de réponse impulsionnelle suivante:
\[
h_\text{haut}(t) = \frac{2 F_c}{F_e} \sinc(2 F_c t)
\]
\\
et donc \[H_\text{haut}(f) = \TF(h_\text{haut}(t))
\]

\paragraph{Filtre passe bas}

\[
H_\text{bas}(f) = 1-H_\text{haut}(f)
\]

Les réponses des filtres sont les suivantes:

% This file was created by matlab2tikz.
%
%The latest updates can be retrieved from
%  http://www.mathworks.com/matlabcentral/fileexchange/22022-matlab2tikz-matlab2tikz
%where you can also make suggestions and rate matlab2tikz.
%
\definecolor{mycolor1}{rgb}{0.00000,0.44700,0.74100}%
\definecolor{mycolor2}{rgb}{0.85000,0.32500,0.09800}%
%
\begin{tikzpicture}

\begin{axis}[%
width=4.521in,
height=3.548in,
at={(0.758in,0.499in)},
scale only axis,
xmin=-0.0025,
xmax=0.0025,
ymin=-0.2,
ymax=1,
axis background/.style={fill=white}
]
\addplot [color=mycolor1, forget plot]
  table[row sep=crcr]{%
-0.00208333333333333	-0.00275664447710899\\
-0.00206249999999997	-0.00321525137559386\\
-0.00204166666666672	-0.0028129025276622\\
-0.00202083333333336	-0.00164077260919482\\
-0.00197916666666664	0.00167531519044095\\
-0.00195833333333328	0.00293260050756272\\
-0.00193750000000004	0.00342268694821279\\
-0.00191666666666668	0.00299635269250975\\
-0.00189583333333332	0.00174895541859221\\
-0.00185416666666671	-0.00178825778754943\\
-0.00183333333333335	-0.00313255054216932\\
-0.00181249999999999	-0.00365873432395158\\
-0.00179166666666664	-0.0032054005547778\\
-0.00177083333333339	-0.00187241109519876\\
-0.00172916666666667	0.00191752943484214\\
-0.00170833333333331	0.00336176155744994\\
-0.00168749999999995	0.0039297516812814\\
-0.00166666666666671	0.00344580559638619\\
-0.00164583333333335	0.00201461953280879\\
-0.00160416666666663	-0.00206694731288171\\
-0.00158333333333338	-0.00362716378566974\\
-0.00156250000000002	-0.00424413181578387\\
-0.00154166666666666	-0.00372519523933645\\
-0.0015208333333333	-0.002180204699889\\
-0.0014791666666667	0.00224161891678731\\
-0.00145833333333334	0.00393806353872705\\
-0.00143749999999998	0.00461318675628686\\
-0.00141666666666662	0.00405388893692493\\
-0.00139583333333337	0.00237544691181935\\
-0.00135416666666666	-0.00244853758602914\\
-0.0013333333333333	-0.00430725699548273\\
-0.00131250000000005	-0.00505253787593318\\
-0.00129166666666669	-0.00444620076953062\\
-0.00127083333333333	-0.00260909742773596\\
-0.00122916666666661	0.00269754140833722\\
-0.00120833333333337	0.0047528353053603\\
-0.00118750000000001	0.00558438396813665\\
-0.00116666666666665	0.00492257942340879\\
-0.00114583333333329	0.00289372623803441\\
-0.00110416666666668	-0.0030029234545641\\
-0.00108333333333333	-0.00530123937905569\\
-0.00106249999999997	-0.00624137031732919\\
-0.00104166666666672	-0.00551328895421788\\
-0.00102083333333336	-0.00324806006309997\\
-0.000979166666666642	0.00338627538493397\\
-0.000958333333333283	0.0059927053850195\\
-0.000937500000000036	0.00707355302630641\\
-0.000916666666666677	0.00626510108433853\\
-0.000895833333333318	0.00370127774632312\\
-0.000854166666666711	-0.00388182788029012\\
-0.000833333333333353	-0.00689161119277237\\
-0.000812499999999994	-0.00816179195343048\\
-0.000791666666666635	-0.00725432757133937\\
-0.000770833333333387	-0.00430148494842963\\
-0.00072916666666667	0.00454728408833982\\
-0.000708333333333311	0.00810777787384986\\
-0.000687499999999952	0.00964575412678159\\
-0.000666666666666704	0.00861451399096547\\
-0.000645833333333345	0.00513403042231919\\
-0.000624999999999987	0\\
-0.000604166666666628	-0.00548810148592738\\
-0.00058333333333338	-0.00984515884681769\\
-0.000562500000000021	-0.0117892550438441\\
-0.000541666666666663	-0.0106024787581114\\
-0.000520833333333304	-0.00636619772367586\\
-0.000499999999999945	0\\
-0.000479166666666697	0.00691978013443029\\
-0.000458333333333338	0.0125302021686771\\
-0.00043749999999998	0.0151576136277995\\
-0.000416666666666621	0.0137832223855449\\
-0.000395833333333373	0.00837657595220498\\
-0.000375000000000014	1.11022302462516e-16\\
-0.000354166666666655	-0.00936205547599389\\
-0.000333333333333297	-0.017229027981931\\
-0.000312500000000049	-0.0212206590789193\\
-0.00029166666666669	-0.0196903176936354\\
-0.000270833333333331	-0.0122426879301458\\
-0.000249999999999972	0\\
-0.000229166666666614	0.0144686311901723\\
-0.000208333333333366	0.0275664447710896\\
-0.000187500000000007	0.0353677651315323\\
-0.000166666666666648	0.034458055963862\\
-0.00014583333333329	0.0227364204416993\\
-0.000125000000000042	1.11022302462516e-16\\
-0.000104166666666683	-0.0318309886183786\\
-6.24999999999654e-05	-0.106103295394597\\
-4.16666666667176e-05	-0.137832223855448\\
-2.08333333333588e-05	-0.159154943091895\\
0	0.833333333333333\\
2.08333333333588e-05	-0.159154943091895\\
4.16666666667176e-05	-0.137832223855448\\
6.24999999999654e-05	-0.106103295394597\\
0.000104166666666683	-0.0318309886183786\\
0.000125000000000042	1.11022302462516e-16\\
0.00014583333333329	0.0227364204416993\\
0.000166666666666648	0.034458055963862\\
0.000187500000000007	0.0353677651315323\\
0.000208333333333366	0.0275664447710896\\
0.000229166666666614	0.0144686311901723\\
0.000249999999999972	0\\
0.000270833333333331	-0.0122426879301458\\
0.00029166666666669	-0.0196903176936354\\
0.000312500000000049	-0.0212206590789193\\
0.000333333333333297	-0.017229027981931\\
0.000354166666666655	-0.00936205547599389\\
0.000375000000000014	1.11022302462516e-16\\
0.000395833333333373	0.00837657595220498\\
0.000416666666666621	0.0137832223855449\\
0.00043749999999998	0.0151576136277995\\
0.000458333333333338	0.0125302021686771\\
0.000479166666666697	0.00691978013443029\\
0.000499999999999945	0\\
0.000520833333333304	-0.00636619772367586\\
0.000541666666666663	-0.0106024787581114\\
0.000562500000000021	-0.0117892550438441\\
0.00058333333333338	-0.00984515884681769\\
0.000604166666666628	-0.00548810148592738\\
0.000645833333333345	0.00513403042231919\\
0.000666666666666704	0.00861451399096547\\
0.000687499999999952	0.00964575412678159\\
0.000708333333333311	0.00810777787384986\\
0.00072916666666667	0.00454728408833982\\
0.000770833333333387	-0.00430148494842963\\
0.000791666666666635	-0.00725432757133937\\
0.000812499999999994	-0.00816179195343048\\
0.000833333333333353	-0.00689161119277237\\
0.000854166666666711	-0.00388182788029012\\
0.000895833333333318	0.00370127774632312\\
0.000916666666666677	0.00626510108433853\\
0.000937500000000036	0.00707355302630641\\
0.000958333333333283	0.0059927053850195\\
0.000979166666666642	0.00338627538493397\\
0.00102083333333336	-0.00324806006309997\\
0.00104166666666672	-0.00551328895421788\\
0.00106249999999997	-0.00624137031732919\\
0.00108333333333333	-0.00530123937905569\\
0.00110416666666668	-0.0030029234545641\\
0.00114583333333329	0.00289372623803441\\
0.00116666666666665	0.00492257942340879\\
0.00118750000000001	0.00558438396813665\\
0.00120833333333337	0.0047528353053603\\
0.00122916666666661	0.00269754140833722\\
0.00127083333333333	-0.00260909742773596\\
0.00129166666666669	-0.00444620076953062\\
0.00131250000000005	-0.00505253787593318\\
0.0013333333333333	-0.00430725699548273\\
0.00135416666666666	-0.00244853758602914\\
0.00139583333333337	0.00237544691181935\\
0.00141666666666662	0.00405388893692493\\
0.00143749999999998	0.00461318675628686\\
0.00145833333333334	0.00393806353872705\\
0.0014791666666667	0.00224161891678731\\
0.0015208333333333	-0.002180204699889\\
0.00154166666666666	-0.00372519523933645\\
0.00156250000000002	-0.00424413181578387\\
0.00158333333333338	-0.00362716378566974\\
0.00160416666666663	-0.00206694731288171\\
0.00164583333333335	0.00201461953280879\\
0.00166666666666671	0.00344580559638619\\
0.00168749999999995	0.0039297516812814\\
0.00170833333333331	0.00336176155744994\\
0.00172916666666667	0.00191752943484214\\
0.00177083333333339	-0.00187241109519876\\
0.00179166666666664	-0.0032054005547778\\
0.00181249999999999	-0.00365873432395158\\
0.00183333333333335	-0.00313255054216932\\
0.00185416666666671	-0.00178825778754943\\
0.00189583333333332	0.00174895541859221\\
0.00191666666666668	0.00299635269250975\\
0.00193750000000004	0.00342268694821279\\
0.00195833333333328	0.00293260050756272\\
0.00197916666666664	0.00167531519044095\\
0.00202083333333336	-0.00164077260919482\\
0.00204166666666672	-0.0028129025276622\\
0.00206249999999997	-0.00321525137559386\\
0.00208333333333333	-0.00275664447710899\\
};
\addplot [color=mycolor2, forget plot]
  table[row sep=crcr]{%
-0.00208333333333333	0.00275664447710897\\
-0.00206249999999999	0.00321525137559384\\
-0.00204166666666666	0.0028129025276622\\
-0.00202083333333333	0.00164077260919479\\
-0.00197916666666667	-0.00167531519044101\\
-0.00195833333333334	-0.00293260050756272\\
-0.00193750000000001	-0.00342268694821279\\
-0.00191666666666668	-0.00299635269250972\\
-0.00189583333333335	-0.00174895541859224\\
-0.00185416666666666	0.0017882577875494\\
-0.00183333333333333	0.00313255054216929\\
-0.00181249999999999	0.00365873432395161\\
-0.00179166666666666	0.00320540055477786\\
-0.00177083333333333	0.00187241109519876\\
-0.00172916666666667	-0.00191752943484214\\
-0.00170833333333334	-0.00336176155744997\\
-0.00168750000000001	-0.00392975168128137\\
-0.00166666666666668	-0.00344580559638619\\
-0.00164583333333335	-0.00201461953280879\\
-0.00160416666666666	0.00206694731288176\\
-0.00158333333333333	0.00362716378566968\\
-0.00156249999999999	0.00424413181578387\\
-0.00154166666666666	0.00372519523933643\\
-0.00152083333333333	0.00218020469988894\\
-0.00147916666666667	-0.00224161891678729\\
-0.00145833333333334	-0.00393806353872708\\
-0.00143750000000001	-0.00461318675628683\\
-0.00141666666666668	-0.00405388893692493\\
-0.00139583333333335	-0.00237544691181932\\
-0.00135416666666666	0.00244853758602917\\
-0.00133333333333333	0.00430725699548276\\
-0.00131249999999999	0.00505253787593318\\
-0.00129166666666666	0.00444620076953056\\
-0.00127083333333333	0.00260909742773599\\
-0.00122916666666667	-0.00269754140833722\\
-0.00120833333333334	-0.00475283530536028\\
-0.00118750000000001	-0.00558438396813668\\
-0.00116666666666668	-0.00492257942340885\\
-0.00114583333333335	-0.00289372623803444\\
-0.00110416666666666	0.00300292345456407\\
-0.00108333333333333	0.00530123937905572\\
-0.00106249999999999	0.00624137031732924\\
-0.00104166666666666	0.00551328895421793\\
-0.00102083333333333	0.00324806006309991\\
-0.00097916666666667	-0.00338627538493394\\
-0.000958333333333339	-0.00599270538501948\\
-0.000937500000000008	-0.00707355302630647\\
-0.000916666666666677	-0.00626510108433856\\
-0.000895833333333346	-0.00370127774632314\\
-0.000854166666666656	0.00388182788029012\\
-0.000833333333333325	0.00689161119277243\\
-0.000812499999999994	0.00816179195343053\\
-0.000791666666666663	0.00725432757133937\\
-0.000770833333333332	0.00430148494842961\\
-0.00072916666666667	-0.00454728408833988\\
-0.000708333333333339	-0.00810777787384989\\
-0.000687500000000008	-0.00964575412678154\\
-0.000666666666666677	-0.00861451399096552\\
-0.000645833333333345	-0.00513403042231919\\
-0.000624999999999987	-2.77555756156289e-17\\
-0.000604166666666656	0.0054881014859274\\
-0.000583333333333325	0.00984515884681775\\
-0.000562499999999994	0.0117892550438441\\
-0.000541666666666663	0.0106024787581114\\
-0.000520833333333331	0.00636619772367583\\
-0.0005	-0\\
-0.000479166666666669	-0.00691978013443029\\
-0.000458333333333338	-0.0125302021686771\\
-0.000437500000000007	-0.0151576136277995\\
-0.000416666666666676	-0.0137832223855448\\
-0.000395833333333345	-0.00837657595220501\\
-0.000374999999999986	-5.55111512312578e-17\\
-0.000354166666666655	0.00936205547599384\\
-0.000333333333333324	0.017229027981931\\
-0.000312499999999993	0.0212206590789194\\
-0.000291666666666662	0.0196903176936354\\
-0.000270833333333331	0.0122426879301458\\
-0.00025	-0\\
-0.000229166666666669	-0.0144686311901722\\
-0.000208333333333338	-0.0275664447710896\\
-0.000187500000000007	-0.0353677651315323\\
-0.000166666666666676	-0.034458055963862\\
-0.000145833333333345	-0.0227364204416993\\
-0.000125000000000014	-1.38777878078145e-16\\
-0.000104166666666655	0.0318309886183787\\
-6.24999999999931e-05	0.106103295394597\\
-4.16666666666621e-05	0.137832223855448\\
-2.0833333333331e-05	0.159154943091895\\
0	0.166666666666667\\
2.0833333333331e-05	0.159154943091895\\
4.16666666666621e-05	0.137832223855448\\
6.24999999999931e-05	0.106103295394597\\
0.000104166666666655	0.0318309886183787\\
0.000125000000000014	-1.38777878078145e-16\\
0.000145833333333345	-0.0227364204416993\\
0.000166666666666676	-0.034458055963862\\
0.000187500000000007	-0.0353677651315323\\
0.000208333333333338	-0.0275664447710896\\
0.000229166666666669	-0.0144686311901722\\
0.00025	-0\\
0.000270833333333331	0.0122426879301458\\
0.000291666666666662	0.0196903176936354\\
0.000312499999999993	0.0212206590789194\\
0.000333333333333324	0.017229027981931\\
0.000354166666666655	0.00936205547599384\\
0.000374999999999986	-5.55111512312578e-17\\
0.000395833333333345	-0.00837657595220501\\
0.000416666666666676	-0.0137832223855448\\
0.000437500000000007	-0.0151576136277995\\
0.000458333333333338	-0.0125302021686771\\
0.000479166666666669	-0.00691978013443029\\
0.0005	-0\\
0.000520833333333331	0.00636619772367583\\
0.000541666666666663	0.0106024787581114\\
0.000562499999999994	0.0117892550438441\\
0.000583333333333325	0.00984515884681775\\
0.000604166666666656	0.0054881014859274\\
0.000645833333333345	-0.00513403042231919\\
0.000666666666666677	-0.00861451399096552\\
0.000687500000000008	-0.00964575412678154\\
0.000708333333333339	-0.00810777787384989\\
0.00072916666666667	-0.00454728408833988\\
0.000770833333333332	0.00430148494842961\\
0.000791666666666663	0.00725432757133937\\
0.000812499999999994	0.00816179195343053\\
0.000833333333333325	0.00689161119277243\\
0.000854166666666656	0.00388182788029012\\
0.000895833333333346	-0.00370127774632314\\
0.000916666666666677	-0.00626510108433856\\
0.000937500000000008	-0.00707355302630647\\
0.000958333333333339	-0.00599270538501948\\
0.00097916666666667	-0.00338627538493394\\
0.00102083333333333	0.00324806006309991\\
0.00104166666666666	0.00551328895421793\\
0.00106249999999999	0.00624137031732924\\
0.00108333333333333	0.00530123937905572\\
0.00110416666666666	0.00300292345456407\\
0.00114583333333335	-0.00289372623803444\\
0.00116666666666668	-0.00492257942340885\\
0.00118750000000001	-0.00558438396813668\\
0.00120833333333334	-0.00475283530536028\\
0.00122916666666667	-0.00269754140833722\\
0.00127083333333333	0.00260909742773599\\
0.00129166666666666	0.00444620076953056\\
0.00131249999999999	0.00505253787593318\\
0.00133333333333333	0.00430725699548276\\
0.00135416666666666	0.00244853758602917\\
0.00139583333333335	-0.00237544691181932\\
0.00141666666666668	-0.00405388893692493\\
0.00143750000000001	-0.00461318675628683\\
0.00145833333333334	-0.00393806353872708\\
0.00147916666666667	-0.00224161891678729\\
0.00152083333333333	0.00218020469988894\\
0.00154166666666666	0.00372519523933643\\
0.00156249999999999	0.00424413181578387\\
0.00158333333333333	0.00362716378566968\\
0.00160416666666666	0.00206694731288176\\
0.00164583333333335	-0.00201461953280879\\
0.00166666666666668	-0.00344580559638619\\
0.00168750000000001	-0.00392975168128137\\
0.00170833333333334	-0.00336176155744997\\
0.00172916666666667	-0.00191752943484214\\
0.00177083333333333	0.00187241109519876\\
0.00179166666666666	0.00320540055477786\\
0.00181249999999999	0.00365873432395161\\
0.00183333333333333	0.00313255054216929\\
0.00185416666666666	0.0017882577875494\\
0.00189583333333335	-0.00174895541859224\\
0.00191666666666668	-0.00299635269250972\\
0.00193750000000001	-0.00342268694821279\\
0.00195833333333334	-0.00293260050756272\\
0.00197916666666667	-0.00167531519044101\\
0.00202083333333333	0.00164077260919479\\
0.00204166666666666	0.0028129025276622\\
0.00206249999999999	0.00321525137559384\\
0.00208333333333333	0.00275664447710897\\
};
\end{axis}
\end{tikzpicture}%

\paragraph{}

% This file was created by matlab2tikz.
%
%The latest updates can be retrieved from
%  http://www.mathworks.com/matlabcentral/fileexchange/22022-matlab2tikz-matlab2tikz
%where you can also make suggestions and rate matlab2tikz.
%
\definecolor{mycolor1}{rgb}{0.00000,0.44700,0.74100}%
\definecolor{mycolor2}{rgb}{0.85000,0.32500,0.09800}%
%
\begin{tikzpicture}

\begin{axis}[%
width=4.521in,
height=3.548in,
at={(0.758in,0.499in)},
scale only axis,
unbounded coords=jump,
xmin=-25000,
xmax=25000,
ymode=log,
ymin=1e-06,
ymax=2.01031921814129,
yminorticks=true,
axis background/.style={fill=white}
]
\addplot [color=mycolor1, forget plot]
  table[row sep=crcr]{%
-24000	0.996832895608896\\
-22118.9109998779	0.997425083103077\\
-22001.7091930167	0.999324506656028\\
-21890.3674764986	0.997816279546436\\
-21773.1656696374	0.998918637145748\\
-21655.9638627762	0.998015312415495\\
-21538.762055915	0.998715345822663\\
-21421.5602490538	0.998222183864211\\
-21304.3584421926	0.998505912788735\\
-21181.2965449884	0.998186508694978\\
-21064.0947381272	0.998541681705627\\
-20946.892931266	0.998398002424056\\
-20823.8310340618	0.998577648912566\\
-20706.6292272006	0.99836229466118\\
-20583.5673299963	0.998613826209011\\
-20466.3655231351	0.998326353057042\\
-20349.163716274	0.998400284672398\\
-20226.1018190697	0.998290165344215\\
-20108.9000122085	0.998436341329032\\
-19985.8381150043	0.998253718814034\\
-19868.6363081431	0.998472699863488\\
-19751.4345012819	0.9984698721876\\
-19628.3726040776	0.998509374151514\\
-19511.1707972165	0.998433744768065\\
-19388.1089000122	0.998546378648339\\
-19270.907093151	0.998397266383833\\
-19153.7052862898	0.998327577897565\\
-19030.6433890856	0.998360421961488\\
-18913.4415822244	0.998364099916366\\
-18790.3796850201	0.998323195719407\\
-18673.177878159	0.998401046699811\\
-18555.9760712978	0.998545354899659\\
-18432.9141740935	0.998438435657947\\
-18315.7123672323	0.998508829888668\\
-18192.6504700281	0.998476285082214\\
-18075.4486631669	0.998471828698683\\
-17958.2468563057	0.998250795477041\\
-17835.1849591015	0.998434332246732\\
-17717.9831522403	0.9982877647853\\
-17594.921255036	0.998396320441471\\
-17477.7194481748	0.998325291767864\\
-17360.5176413136	0.998626036742342\\
-17237.4557441094	0.99836339874995\\
-17120.2539372482	0.998589153788874\\
-16997.1920400439	0.998402109324027\\
-16879.9902331828	0.998551659321487\\
-16762.7884263216	0.998168263083267\\
-16639.7265291173	0.998513528801385\\
-16522.5247222561	0.998205643474438\\
-16405.3229153949	0.998750289360982\\
-16282.2610181907	0.998243726991221\\
-16165.0592113295	0.998713848285666\\
-16041.9973141253	0.998282542622693\\
-15924.7955072641	0.998676677965681\\
-15807.5937004029	0.998041023433576\\
-15684.5318031986	0.998638748484395\\
-15567.3299963374	0.998077918071155\\
-15444.2680991332	0.998600028015826\\
-15327.066292272	0.998115641198052\\
-15209.8644854108	0.998847684143466\\
-15086.8025882066	0.998154228359988\\
-14969.6007813454	0.998811231197942\\
-14846.5388841411	0.998193717476693\\
-14729.3370772799	0.998773895797101\\
-14612.1352704188	0.997940176338702\\
-14489.0733732145	0.998735639162434\\
-14371.8715663533	0.997977122489888\\
-14254.6697594921	0.998992872409103\\
-14131.6078622879	0.998015065708074\\
-14014.4060554267	0.998957631226353\\
-13891.3441582224	0.99805405247396\\
-13774.1423513612	0.998921387929662\\
-13656.9405445001	0.997789975040251\\
-13533.8786472958	0.998884095661933\\
-13416.6768404346	0.997825613544543\\
-13293.6149432304	0.998845703995176\\
-13176.4131363692	0.997862379233059\\
-13059.211329508	0.999118843297217\\
-12936.1494323037	0.99790032863796\\
-12818.9476254426	0.999084342150723\\
-12701.7458185814	0.997624166970605\\
-12578.6839213771	0.999048673577544\\
-12461.4821145159	0.997657801810784\\
-12338.4202173117	0.999011777048074\\
-12221.2184104505	0.997692683751328\\
-12104.0166035893	0.999298759283758\\
-11980.9547063851	0.997728880862959\\
-11863.7528995239	0.999266874459693\\
-11740.6910023196	0.997766467148784\\
-11623.4891954584	0.99923372514095\\
-11506.2873885972	0.997469292468153\\
-11383.225491393	0.99919924001508\\
-11266.0236845318	0.997501327008615\\
-11148.8218776706	0.999502618074376\\
-11025.7599804664	0.997534785947494\\
-10908.5581736052	0.999474562365066\\
-10785.4962764009	0.997569757679972\\
-10668.2944695397	0.999445188483724\\
-10551.0926626786	0.997254098269628\\
-10428.0307654743	0.999414414971301\\
-10310.8289586131	0.997281927184534\\
-10193.6271517519	0.999737675502979\\
-10070.5652545477	0.997311242332677\\
-9953.36344768649	0.999715189706602\\
-9830.30155048224	0.997342146588606\\
-9713.09974362105	0.99969141634205\\
-9595.89793675986	0.997003812058187\\
-9472.83603955561	0.999666268001692\\
-9355.63423269442	0.9970254512871\\
-9232.57233549017	0.999639647603724\\
-9115.37052862898	0.997048542818365\\
-8998.16872176779	1\\
-8875.10682456355	0.997073200368364\\
-8757.90501770236	0.999984587851479\\
-8640.70321084117	0.996706201494268\\
-8517.64131363692	0.999968001824271\\
-8400.43950677573	0.99671863036335\\
-8277.37760957148	0.999950151478832\\
-8160.17580271029	0.996732285126175\\
-8042.9739958491	1.00034592225815\\
-7919.91209864485	0.996747276656142\\
-7802.71029178366	1.00034325355995\\
-7679.64839457942	0.996763732379059\\
-7568.30667806129	1.0007466660933\\
-7439.38469051398	0.996781799490862\\
-7328.04297399585	1.0007629609204\\
-7199.12098644854	0.996801648986212\\
-7093.63936027347	1.00117254046307\\
-6958.8572823831	0.996823480707068\\
-6853.37565620803	1.0012124626192\\
-6718.59357831766	0.996847529653874\\
-6613.1119521426	1.00125465091802\\
-6484.18996459529	0.996341369711935\\
-6372.84824807716	1.00129941439487\\
-6243.92626052985	0.996343591974253\\
-6138.44463435478	1.00177081121673\\
-5997.80246612135	0.996936047542885\\
-5904.0410206324	1.00222236387663\\
-5751.67867171286	0.997629915316042\\
-5663.77731656696	1.00233766353038\\
-5517.27505799048	0.997012989360955\\
-5429.37370284459	1.00280084254137\\
-5265.29117323892	0.998533335895457\\
-5189.10999877915	1.00296951287786\\
-5030.88755951654	0.997879091663719\\
-4954.70638505677	1.00344890616027\\
-4778.90367476498	0.999633925359806\\
-4714.44268099133	1.00368880845461\\
-4544.50006104261	0.998976167006839\\
-4474.17897692589	1.0039569249358\\
-4310.09644732023	0.998234800181342\\
-4239.77536320352	1.00453228240155\\
-4175.31436942986	0.996409093187677\\
-4116.71346599927	0.992501187171489\\
-4046.39238188255	1.00216154085782\\
-3993.65156879502	1.00460372339531\\
-3835.42912953241	0.99751625599657\\
-3765.1080454157	1.00556365993178\\
-3712.36723232816	0.999043486767625\\
-3647.90623855451	0.991303982546033\\
-3595.16542546698	0.997648365395583\\
-3524.84434135026	1.00612798755571\\
-3472.10352826273	0.998915267203532\\
-3407.64253448907	0.990740432618655\\
-3354.90172140154	0.997811403423119\\
-3290.44072762788	1.00690301909345\\
-3243.56000488341	1.00130066977383\\
-3167.37883042364	0.990072462165594\\
-3120.49810767916	0.996678234573682\\
-3050.17702356245	1.00776420609403\\
-3003.29630081797	1.00138494970508\\
-2932.97521670126	0.989198680830231\\
-2891.95458429984	0.994010395470691\\
-2809.91331949701	1.00883862699568\\
-2768.89268709559	1.00297177443135\\
-2692.71151263582	0.988145596868668\\
-2651.6908802344	0.993807993924732\\
-2569.64961543157	1.01022545647664\\
-2528.62898303015	1.0033231583576\\
-2452.44780857038	0.986795818711662\\
-2411.42717616897	0.993585260858148\\
-2335.24600170919	1.01216671848221\\
-2300.08545965084	1.00734677840065\\
-2206.32401416188	0.985280086419699\\
-2171.16347210353	0.993360957849265\\
-2100.84238798681	1.01473416419364\\
-2065.68184592846	1.01063170281041\\
-1960.20021975339	0.983729894764392\\
-1925.03967769503	0.995867027936065\\
-1866.43877426444	1.01857184964234\\
-1837.13832254914	1.01713219173129\\
-1801.97778049078	1.00252184813393\\
-1749.23696740325	0.979469520097654\\
-1719.93651568795	0.980475766028825\\
-1690.63606397265	0.993494969204586\\
-1632.03516054206	1.02551737408785\\
-1608.59479916982	1.0270473779307\\
-1585.15443779758	1.01882562234953\\
-1544.13380539617	0.99019951403007\\
-1514.83335368087	0.9737029497909\\
-1491.39299230863	0.971612847063762\\
-1467.95263093639	0.981819134009605\\
-1438.6521792211	1.00765672763801\\
-1403.49163716274	1.03958633238676\\
-1380.0512757905	1.04860799814564\\
-1362.47100476132	1.04503319746053\\
-1339.03064338909	1.02655516198561\\
-1303.87010133073	0.982896537691829\\
-1280.42973995849	0.959708094065395\\
-1262.84946892931	0.954232975724163\\
-1245.26919790014	0.963181203408916\\
-1227.68892687096	0.98782224689618\\
-1210.10865584178	1.02641408471288\\
-1163.2279330973	1.15163037851228\\
-1151.50775241118	1.17352094110361\\
-1139.78757172506	1.18566143807447\\
-1128.06739103895	1.18533953045042\\
-1116.34721035283	1.17032911510502\\
-1104.62702966671	1.13907227488671\\
-1092.90684898059	1.09082742786468\\
-1081.18666829447	1.02577260482068\\
-1069.46648760835	0.94505563488776\\
-1063.60639726529	0.899447048543343\\
-1057.74630692223	0.850786484979452\\
-1051.88621657917	0.799477112357801\\
-1046.02612623611	0.745970830243797\\
-1040.16603589305	0.690763229858802\\
-1034.30594554999	0.634387851397492\\
-1028.44585520693	0.577409823109672\\
-1022.58576486387	0.520418977184212\\
-1016.72567452081	0.464022545893965\\
-1010.86558417775	0.408837548291054\\
-1005.00549383469	0.35548298235287\\
-999.145403491639	0.3045719404386\\
-993.285313148579	0.256703766500901\\
-987.425222805519	0.212456372314503\\
-981.56513246246	0.172378826660861\\
-975.7050421194	0.136984326241274\\
-969.84495177634	0.106743650040611\\
-958.124771090221	0.0633596428597587\\
-952.264680747161	0.0508954289468671\\
-946.404590404101	0.0449349065649443\\
-940.544500061042	0.0456614181101872\\
-934.684409717982	0.0531912055387559\\
-928.824319374922	0.0675722109178138\\
-917.104138688806	0.1167371584567\\
-911.244048345747	0.15127705573661\\
-905.383958002687	0.192183740818182\\
-899.523867659627	0.239176119363421\\
-893.663777316568	0.29191543318068\\
-887.803686973508	0.350009612126327\\
-881.943596630448	0.413018186548377\\
-876.083506287388	0.480457675957719\\
-870.223415944329	0.551807364638034\\
-864.363325601269	0.626515371693959\\
-858.503235258209	0.70400492135538\\
-852.64314491515	0.783680719516259\\
-846.78305457209	0.864935344250767\\
-840.92296422903	0.947155561231066\\
-835.062873885974	1.02972847989858\\
-829.202783542914	1.11204747212259\\
-823.342693199855	1.19351778244862\\
-811.622512513735	1.35162370894618\\
-799.902331827616	1.49971379171229\\
-788.182151141496	1.63393317599829\\
-776.461970455377	1.7510019194178\\
-764.741789769258	1.84827790080604\\
-753.021609083142	1.92378657891012\\
-741.301428397022	1.97622090119845\\
-729.581247710903	2.00491625036143\\
-717.861067024784	2.00980634185377\\
-706.140886338664	1.99136641083882\\
-694.420705652545	1.950549865115\\
-682.700524966425	1.88872389827024\\
-670.980344280306	1.80760845759489\\
-659.26016359419	1.70922157994595\\
-647.539982908071	1.5958325988802\\
-635.819802221951	1.46992324031516\\
-624.099621535832	1.33415530743284\\
-612.379440849712	1.19134262616218\\
-600.659260163593	1.04442426820744\\
-594.799169820533	0.970371190522173\\
-588.939079477474	0.896435835667216\\
-583.078989134417	0.82300721809515\\
-577.218898791358	0.750475784610004\\
-571.358808448298	0.679231927509071\\
-565.498718105238	0.609664349062848\\
-559.638627762179	0.542158276223431\\
-553.778537419119	0.477093529468452\\
-547.918447076059	0.414842454637032\\
-542.058356733	0.355767731357972\\
-536.19826638994	0.300220076248053\\
-530.33817604688	0.248535863398309\\
-524.47808570382	0.201034688337709\\
-518.617995360761	0.158016905053816\\
-512.757905017701	0.119761168221692\\
-506.897814674641	0.0865220148568635\\
-501.037724331585	0.0585275208026193\\
-495.177633988525	0.0359770680418801\\
-489.317543645466	0.0190392585477494\\
-483.457453302406	0.00785000941386517\\
-477.597362959346	0.00251086225912748\\
-471.737272616287	0.00308753750039357\\
-465.877182273227	0.00960876101515315\\
-460.017091930167	0.0220653870951406\\
-454.157001587108	0.0404098374684634\\
-448.296911244048	0.0645558716463799\\
-442.436820900988	0.0943786990291977\\
-436.576730557928	0.129715438170306\\
-430.716640214869	0.170365923483219\\
-424.856549871809	0.216093854548109\\
-418.996459528749	0.266628278185201\\
-413.136369185693	0.321665388666851\\
-407.276278842633	0.38087062692543\\
-401.416188499574	0.443881055599092\\
-395.556098156514	0.510307982980246\\
-389.696007813454	0.579739806000109\\
-383.835917470395	0.651745039672601\\
-377.975827127335	0.725875498643981\\
-372.115736784275	0.801669595071956\\
-366.255646441216	0.878655716514344\\
-360.395556098156	0.956355647412887\\
-354.535465755096	1.03428799839619\\
-342.815285068977	1.18892888879279\\
-331.095104382861	1.33879135089252\\
-319.374923696741	1.4802465076536\\
-307.654743010622	1.60991631992812\\
-295.934562324503	1.72475460601704\\
-284.214381638383	1.82211467656764\\
-272.494200952264	1.8998027206274\\
-260.774020266144	1.95611686268043\\
-249.053839580029	1.98987245605357\\
-237.333658893909	2.00041464654201\\
-225.61347820779	1.98761951181921\\
-213.89329752167	1.95188516227118\\
-202.173116835551	1.89411409936246\\
-190.452936149431	1.81568791115953\\
-178.732755463312	1.71843509222644\\
-167.012574777196	1.60459246494166\\
-155.292394091077	1.4767604070695\\
-143.572213404957	1.33785190519818\\
-131.852032718838	1.19103539109236\\
-120.131852032719	1.03967139699291\\
-114.271761689659	0.963369855247249\\
-108.411671346599	0.887243287282018\\
-102.551581003539	0.811734065523141\\
-96.6914906604798	0.737282670606966\\
-90.83140031742	0.66432514761198\\
-84.9713099743603	0.593290532616441\\
-79.1112196313043	0.524598265977831\\
-73.2511292882446	0.458655610687447\\
-67.3910389451848	0.395855095241961\\
-61.5309486021251	0.336572001914722\\
-55.6708582590654	0.28116192199057\\
-49.8107679160057	0.229958400141359\\
-43.950677572946	0.183270690263628\\
-38.0905872298863	0.141381644892136\\
-32.2304968868266	0.104545759699547\\
-26.3704065437669	0.0729873936031107\\
-20.5103162007072	0.0468991836360684\\
-14.6502258576475	0.0264406719969562\\
-8.79013551458775	0.0117371606487721\\
-2.93004517152804	0.00287880645765348\\
2.93004517152804	-8.00327355025226e-05\\
8.79013551458775	0.00287880645765348\\
14.6502258576475	0.0117371606487721\\
20.5103162007072	0.0264406719969562\\
26.3704065437669	0.0468991836360684\\
32.2304968868266	0.0729873936031107\\
38.0905872298863	0.104545759699547\\
43.950677572946	0.141381644892136\\
49.8107679160057	0.183270690263628\\
55.6708582590654	0.229958400141359\\
61.5309486021251	0.28116192199057\\
67.3910389451848	0.336572001914722\\
73.2511292882446	0.395855095241961\\
79.1112196313043	0.458655610687447\\
84.9713099743603	0.524598265977831\\
90.83140031742	0.593290532616441\\
96.6914906604798	0.66432514761198\\
102.551581003539	0.737282670606966\\
108.411671346599	0.811734065523141\\
114.271761689659	0.887243287282018\\
120.131852032719	0.963369855247249\\
125.991942375778	1.03967139699291\\
137.712123061898	1.19103539109236\\
149.432303748017	1.33785190519818\\
161.152484434137	1.4767604070695\\
172.872665120252	1.60459246494166\\
184.592845806372	1.71843509222644\\
196.313026492491	1.81568791115953\\
208.033207178611	1.89411409936246\\
219.75338786473	1.95188516227118\\
231.473568550849	1.98761951181921\\
243.193749236969	2.00041464654201\\
254.913929923085	1.98987245605357\\
266.634110609204	1.95611686268043\\
278.354291295323	1.8998027206274\\
290.074471981443	1.82211467656764\\
301.794652667562	1.72475460601704\\
313.514833353682	1.60991631992812\\
325.235014039801	1.4802465076536\\
336.955194725917	1.33879135089252\\
348.675375412036	1.18892888879279\\
360.395556098156	1.03428799839619\\
366.255646441216	0.956355647412887\\
372.115736784275	0.878655716514344\\
377.975827127335	0.801669595071956\\
383.835917470395	0.725875498643981\\
389.696007813454	0.651745039672601\\
395.556098156514	0.579739806000109\\
401.416188499574	0.510307982980246\\
407.276278842633	0.443881055599092\\
413.136369185693	0.38087062692543\\
418.996459528749	0.321665388666851\\
424.856549871809	0.266628278185201\\
430.716640214869	0.216093854548109\\
436.576730557928	0.170365923483219\\
442.436820900988	0.129715438170306\\
448.296911244048	0.0943786990291977\\
454.157001587108	0.0645558716463799\\
460.017091930167	0.0404098374684634\\
465.877182273227	0.0220653870951406\\
471.737272616287	0.00960876101515315\\
477.597362959346	0.00308753750039357\\
483.457453302406	0.00251086225912748\\
489.317543645466	0.00785000941386517\\
495.177633988525	0.0190392585477494\\
501.037724331585	0.0359770680418801\\
506.897814674641	0.0585275208026193\\
512.757905017701	0.0865220148568635\\
518.617995360761	0.119761168221692\\
524.47808570382	0.158016905053816\\
530.33817604688	0.201034688337709\\
536.19826638994	0.248535863398309\\
542.058356733	0.300220076248053\\
547.918447076059	0.355767731357972\\
553.778537419119	0.414842454637032\\
559.638627762179	0.477093529468452\\
565.498718105238	0.542158276223431\\
571.358808448298	0.609664349062848\\
577.218898791358	0.679231927509071\\
583.078989134417	0.750475784610004\\
588.939079477474	0.82300721809515\\
594.799169820533	0.896435835667216\\
600.659260163593	0.970371190522173\\
612.379440849712	1.11820883041428\\
624.099621535832	1.26344848457997\\
635.819802221951	1.40309898238801\\
647.539982908071	1.5342804818781\\
659.26016359419	1.65425181630979\\
670.980344280306	1.76043751227084\\
682.700524966425	1.85045763297059\\
694.420705652545	1.92216315329097\\
706.140886338664	1.97367872277151\\
717.861067024784	2.00345348679547\\
729.581247710903	2.01031921814129\\
741.301428397022	1.9935534943581\\
753.021609083142	1.95294419102921\\
764.741789769258	1.888850302438\\
776.461970455377	1.80225319476383\\
788.182151141496	1.69479196384835\\
799.902331827616	1.5687766965916\\
811.622512513735	1.42717416129172\\
823.342693199855	1.27356176725305\\
829.202783542914	1.19351778244862\\
835.062873885974	1.11204747212259\\
840.92296422903	1.02972847989858\\
846.78305457209	0.947155561231066\\
852.64314491515	0.864935344250767\\
858.503235258209	0.783680719516259\\
864.363325601269	0.70400492135538\\
870.223415944329	0.626515371693959\\
876.083506287388	0.551807364638034\\
881.943596630448	0.480457675957719\\
887.803686973508	0.413018186548377\\
893.663777316568	0.350009612126327\\
899.523867659627	0.29191543318068\\
905.383958002687	0.239176119363421\\
911.244048345747	0.192183740818182\\
917.104138688806	0.15127705573661\\
922.964229031862	0.1167371584567\\
934.684409717982	0.0675722109178138\\
940.544500061042	0.0531912055387559\\
946.404590404101	0.0456614181101872\\
952.264680747161	0.0449349065649443\\
958.124771090221	0.0508954289468671\\
963.98486143328	0.0633596428597587\\
975.7050421194	0.106743650040611\\
981.56513246246	0.136984326241274\\
987.425222805519	0.172378826660861\\
993.285313148579	0.212456372314503\\
999.145403491639	0.256703766500901\\
1005.00549383469	0.3045719404386\\
1010.86558417775	0.35548298235287\\
1016.72567452081	0.408837548291054\\
1022.58576486387	0.464022545893965\\
1028.44585520693	0.520418977184212\\
1034.30594554999	0.577409823109672\\
1040.16603589305	0.634387851397492\\
1046.02612623611	0.690763229858802\\
1051.88621657917	0.745970830243797\\
1057.74630692223	0.799477112357801\\
1063.60639726529	0.850786484979452\\
1069.46648760835	0.899447048543343\\
1075.32657795141	0.94505563488776\\
1087.04675863753	1.02577260482068\\
1098.76693932365	1.09082742786468\\
1110.48712000977	1.13907227488671\\
1122.20730069589	1.17032911510502\\
1133.927481382	1.18533953045042\\
1145.64766206812	1.18566143807447\\
1157.36784275424	1.17352094110361\\
1174.94811378342	1.13795970991585\\
1198.38847515566	1.0740413458022\\
1227.68892687096	0.999314778718142\\
1245.26919790014	0.96967155352373\\
1262.84946892931	0.955493607395847\\
1280.42973995849	0.956466453974541\\
1298.01001098767	0.969636504731472\\
1368.33109510438	1.04503319746053\\
1385.91136613356	1.04860799814564\\
1409.3517275058	1.03958633238676\\
1438.6521792211	1.01357300381582\\
1479.67281162251	0.978172952518464\\
1503.11317299475	0.970987312130006\\
1526.55353436699	0.976000744338978\\
1561.71407642535	0.998937489796227\\
1596.8746184837	1.0217198976992\\
1620.31497985594	1.02759576268758\\
1643.75534122818	1.02367026066286\\
1678.91588328653	1.00455161395467\\
1719.93651568795	0.982238044497972\\
1749.23696740325	0.978622546087365\\
1778.53741911854	0.987273218588935\\
1854.71859357832	1.01908115482726\\
1884.01904529361	1.01605011995891\\
1925.03967769503	0.998642735160218\\
1966.06031009645	0.983729894764392\\
1995.36076181175	0.983535542852743\\
2030.5213038701	0.994600252688535\\
2089.1222073007	1.01427702465734\\
2124.28274935905	1.01199220000341\\
2182.88365278965	0.991390914292411\\
2218.044194848	0.984990686209556\\
2253.20473690636	0.989940984337306\\
2341.10609205225	1.01216671848221\\
2376.26663411061	1.00658237790266\\
2458.30789891344	0.986795818711662\\
2499.32853131486	0.992858632969638\\
2575.50970577463	1.01022545647664\\
2616.53033817605	1.00514266838457\\
2698.57160297888	0.988145596868668\\
2739.5922353803	0.993739096028116\\
2815.77340984007	1.00883862699568\\
2856.79404224148	1.00408808834625\\
2944.69539738738	0.9893577570768\\
2991.57612013185	0.997151558963844\\
3056.03711390551	1.00776420609403\\
3102.91783664998	1.00204119091642\\
3179.09901110976	0.99004709103107\\
3225.97973385423	0.996192330290711\\
3296.30081797094	1.00690301909345\\
3343.18154071542	1.00149238370662\\
3419.36271517519	0.990747937563676\\
3472.10352826273	0.997725582765613\\
3536.56452203638	1.00619437060048\\
3589.30533512392	0.999930677343879\\
3659.62641924063	0.991338608268868\\
3712.36723232816	0.997943552822156\\
3776.82822610182	1.0055989330132\\
3829.56903918935	0.999632696094606\\
3899.89012330607	0.991844655360532\\
3958.49102673666	0.999153001859629\\
4022.95202051031	1.0049446420332\\
4087.41301428397	0.996453740372192\\
4146.01391771456	0.992508514855878\\
4222.19509217434	1.00266302664932\\
4274.93590526187	1.00376167877472\\
4415.5780734953	0.995819575132754\\
4509.33951898425	1.00379215631079\\
4667.56195824686	0.997707425594251\\
4743.74313270663	1.00373982784518\\
4919.54584299841	0.999468934484346\\
4984.00683677207	1.00342457024295\\
5159.80954706385	0.99952723307841\\
5230.13063118056	1.00283564570942\\
5382.49298010011	0.997403496660058\\
5464.53424490294	1.00288309704148\\
5634.47686485167	0.998941180613018\\
5710.65803931144	1.00234124163613\\
5857.16029788793	0.99703076558458\\
5950.92174337688	1.00212498080733\\
6097.42400195336	0.997174958504069\\
6191.18544744232	1.00192569592976\\
6337.6877060188	0.997308198965534\\
6431.44915150775	1.00174123904595\\
6577.95141008424	0.997431791028484\\
6671.71285557319	1.00156981612853\\
6818.21511414968	0.997546833978176\\
6911.97655963863	1.00140991625016\\
7052.61872787205	0.997120562294703\\
7152.24026370407	1.001260257175\\
7292.88243193749	0.997234884553152\\
7392.5039677695	1.00111974328217\\
7533.14613600293	0.99734233621052\\
7632.76767183494	1.00098743237546\\
7773.40984006837	0.997443587672973\\
7873.03137590038	1.00086250950372\\
8013.67354413381	0.997539222313923\\
8113.29507996582	1.00074426596485\\
8248.07715785618	0.997176427220803\\
8353.55878403125	1.00063208233352\\
8488.34086192162	0.997270940256605\\
8593.82248809669	1.00052541474962\\
8728.60456598706	0.997360845519608\\
8834.08619216213	1.00042378361499\\
8968.8682700525	0.997446524320757\\
9074.34989622757	1.00032676426611\\
9209.13197411794	0.997528315694111\\
9314.613600293	1.00023397930964\\
9449.39567818337	0.997606522130844\\
9554.87730435844	1.00014509213173\\
9689.65938224881	0.997681414400053\\
9795.14100842388	1.00005980151245\\
9924.06299597119	0.997369752073881\\
10041.2648028324	0.999635779522416\\
10158.4666096936	0.997088445525501\\
10281.5285068978	0.999559373508491\\
10398.730313759	0.997163127481571\\
10521.7922109633	0.999485810234261\\
10638.9940178244	0.997235173128067\\
10762.0559150287	0.99941489410079\\
10879.2577218899	0.997304759661315\\
11002.3196190941	0.999346447618324\\
11119.5214259553	0.997372048321964\\
11242.5833231596	0.999280309319404\\
11359.7851300208	0.997437186173558\\
11482.847027225	0.99921633205038\\
11600.0488340862	0.997500307590131\\
11723.1107312904	0.999154381331314\\
11840.3125381516	0.997561535644733\\
11963.3744353559	0.999094334077006\\
12080.5762422171	0.997620983265017\\
12203.6381394213	0.999036077394327\\
12320.8399462825	0.997678754264332\\
12438.0417531437	0.999297365892883\\
12561.1036503479	0.997734944223111\\
12678.3054572091	0.999239991136187\\
12801.3673544134	0.997789641337458\\
12918.5691612746	0.999184116114039\\
13041.6310584788	0.997842927092744\\
13158.83286534	0.999129657943485\\
13281.8947625443	0.997894876895867\\
13399.0965694054	0.999076539950561\\
13522.1584666097	0.99794556063267\\
13639.3602734709	0.999024691056645\\
13762.4221706751	0.997995043175541\\
13879.6239775363	0.998974045307422\\
14002.6858747406	0.998043384824429\\
14119.8876816018	0.998924541410523\\
14242.949578806	0.998090641689595\\
14360.1513856672	0.998876122298525\\
14477.3531925284	0.997843741780814\\
14600.4150897326	0.998828734809326\\
14717.6168965938	0.997890908929009\\
14840.6787937981	0.998782329333246\\
14957.8806006593	0.997937163044036\\
15080.9424978635	0.998736859518867\\
15198.1443047247	0.997982547202462\\
15321.2062019289	0.998692282004134\\
15438.4080087901	0.998027101757273\\
15561.4699059944	0.998648556189408\\
15678.6717128556	0.998070864562638\\
15801.7336100598	0.998605644010643\\
15918.935416921	0.998113871173709\\
16041.9973141253	0.99856350973776\\
16159.1991209864	0.998156154987978\\
16282.2610181907	0.998522119798214\\
16399.4628250519	0.99819774747869\\
16516.6646319131	0.99875587623264\\
16639.7265291173	0.998238678261105\\
16756.9283359785	0.998714203128856\\
16879.9902331828	0.99827897527588\\
16997.1920400439	0.998673149089442\\
17120.2539372482	0.998318664913985\\
17237.4557441094	0.998632687595478\\
17360.5176413136	0.998357772108214\\
17477.7194481748	0.998592793563823\\
17600.7813453791	0.998396320441471\\
17717.9831522403	0.998553443263037\\
17841.0450494445	0.998434332246732\\
17958.2468563057	0.998514614204243\\
18081.3087535099	0.998471828698683\\
18198.5105603711	0.998476285082214\\
18315.7123672323	0.998247530992583\\
18438.7742644366	0.998438435657947\\
18555.9760712978	0.998285571159586\\
18679.037968502	0.998401046699811\\
18796.2397753632	0.998323195719407\\
18919.3016725674	0.998364099916366\\
19036.5034794286	0.998360421961488\\
19159.5653766329	0.998327577897565\\
19276.7671834941	0.998397266383833\\
19399.8290806983	0.998291464030889\\
19517.0308875595	0.998433744768065\\
19640.0927847638	0.998255742484413\\
19757.294591625	0.9984698721876\\
19874.4963984861	0.998472699863488\\
19997.5582956904	0.998505663091104\\
20114.7601025516	0.998436341329032\\
20237.8219997558	0.998541131352518\\
20355.023806617	0.998400284672398\\
20478.0857038213	0.998576290262543\\
20595.2875106825	0.998364516555352\\
20718.3494078867	0.998611152628874\\
20835.5512147479	0.99832902416032\\
20958.6131119521	0.99864573075935\\
21075.8149188133	0.998293795106627\\
21193.0167256745	0.998433488196848\\
21316.0786228788	0.998258817458754\\
21433.28042974	0.998468763481287\\
21556.3423269442	0.998224079684414\\
21673.5441338054	0.998503839412849\\
21796.6060310096	0.998189570612652\\
21913.8078378708	0.998538726835942\\
22036.8697350751	0.998155279408667\\
22154.0715419363	0.99857343629547\\
22277.1334391405	0.998121195573732\\
22394.3352460017	0.998607978061857\\
22511.5370528629	0.998329652889421\\
22634.5989500671	0.99864236220627\\
22751.8007569283	0.998294843893876\\
22874.8626541326	0.998676598542008\\
22992.0644609938	0.998260159002638\\
23115.126358198	0.998710696724831\\
23232.3281650592	0.998225588528144\\
23349.5299719204	0.998503266104091\\
23472.5918691246	0.998191122918003\\
23589.7936759858	0.998538011093011\\
23712.8555731901	0.998156752746594\\
23830.0573800513	0.99857270127189\\
23953.1192772555	0.998122468698315\\
24000	0.996851584090231\\
};
\addplot [color=mycolor2, forget plot]
  table[row sep=crcr]{%
-24000	0.00316710439196338\\
-23994.1399096569	0.00314841591240077\\
-23988.2798193139	0.00309279157124961\\
-23982.4197289708	0.00300154425033711\\
-23976.5596386278	0.00287682762893246\\
-23970.6995482847	0.00272158535269575\\
-23964.8394579416	0.0025394815572651\\
-23958.9793675986	0.00233481438709966\\
-23953.1192772555	0.0021124145500993\\
-23947.2591869125	0.00187753130238685\\
-23941.3990965694	0.00163570855442351\\
-23935.5390062263	0.00139265402219157\\
-23929.6789158833	0.00115410451200649\\
-23923.8188255402	0.000925690518250051\\
-23917.9587351972	0.000712803329900807\\
-23912.0986448541	0.000520467782298221\\
-23906.238554511	0.000353223657520941\\
-23900.378464168	0.000215018532550082\\
-23894.5183738249	0.000109114604210599\\
-23888.6582834819	3.80116901284416e-05\\
-23882.7981931388	3.38822303215212e-06\\
-23876.9381027957	6.06163117441586e-06\\
-23871.0780124527	4.59690400613531e-05\\
-23865.2179221096	0.00012216875107912\\
-23859.3578317666	0.000232862462252826\\
-23853.4977414235	0.000375437706815771\\
-23847.6376510805	0.000546529508093835\\
-23841.7775607374	0.000742099795682522\\
-23835.9174703943	0.000957532708655188\\
-23830.0573800513	0.0011877435365919\\
-23824.1972897082	0.00142729872741364\\
-23818.3371993652	0.00167054412948655\\
-23812.4771090221	0.00191173844163099\\
-23806.617018679	0.00214518872106238\\
-23800.756928336	0.00236538475137398\\
-23794.8968379929	0.00256712909893085\\
-23789.0367476499	0.00274565978831946\\
-23783.1766573068	0.00289676270111265\\
-23777.3165669637	0.00301687104530908\\
-23771.4564766207	0.00310314954747763\\
-23765.5963862776	0.0031535613804648\\
-23759.7362959346	0.00316691624660542\\
-23753.8762055915	0.0031428984816753\\
-23748.0161152484	0.00308207451594186\\
-23742.1560249054	0.00298587951579517\\
-23736.2959345623	0.00285658352129508\\
-23730.4358442193	0.00269723787816496\\
-23724.5757538762	0.00251160322893876\\
-23718.7156635331	0.00230406076177345\\
-23712.8555731901	0.00207950881207306\\
-23706.995482847	0.00184324725662905\\
-23701.135392504	0.00160085242911215\\
-23695.2753021609	0.00135804550845667\\
-23689.4152118178	0.00112055748681614\\
-23683.5551214748	0.000893993903777696\\
-23677.6950311317	0.000683702539442639\\
-23671.8349407887	0.000494647189191627\\
-23665.9748504456	0.000331290499238187\\
-23660.1147601026	0.000197488628571264\\
-23654.2546697595	9.64002235887186e-05\\
-23648.3945794164	3.04118540724698e-05\\
-23642.5344890734	1.08167060916587e-06\\
-23636.6743987303	9.10261360959128e-06\\
-23630.8143083873	5.42860425817925e-05\\
-23624.9542180442	0.000135566172420187\\
-23619.0941277011	0.000251025212382475\\
-23613.2340373581	0.000397938614805616\\
-23607.373947015	0.00057283936599741\\
-23601.513856672	0.000771599802240223\\
-23595.6537663289	0.000989529020273607\\
-23589.7936759858	0.00122148358345847\\
-23583.9335856428	0.00146198891095097\\
-23578.0734952997	0.00170536848525696\\
-23572.2134049567	0.00194587782846629\\
-23566.3533146136	0.00217784008551419\\
-23560.4932242705	0.00239578001412959\\
-23554.6331339275	0.00259455321928503\\
-23548.7730435844	0.00276946758167119\\
-23542.9129532414	0.00291639401417057\\
-23537.0528628983	0.00303186393183529\\
-23531.1927725552	0.00311315113484939\\
-23525.3326822122	0.00315833617138655\\
-23519.4725918691	0.00316635166067952\\
-23513.6125015261	0.00313700750641909\\
-23507.752411183	0.00307099540487415\\
-23501.8923208399	0.00296987254054473\\
-23496.0322304969	0.00283602485397446\\
-23490.1721401538	0.00267261074815455\\
-23484.3120498108	0.00248348656134311\\
-23478.4519594677	0.00227311556530636\\
-23472.5918691246	0.00204646263590623\\
-23466.7317787816	0.00180887708212871\\
-23460.8716884385	0.00156596639829582\\
-23455.0115980955	0.00132346391919193\\
-23449.1515077524	0.00108709350147891\\
-23443.2914174093	0.000862434425106863\\
-23437.4313270663	0.000654789703332076\\
-23431.5712367232	0.000469060909647021\\
-23425.7111463802	0.000309632476041659\\
-23419.8510560371	0.00018026819382562\\
-23413.9909656941	8.40223600124816e-05\\
-23408.130875351	2.31676669029033e-05\\
-23402.2707850079	-8.5846270381279e-07\\
-23396.4106946649	1.25121724722545e-05\\
-23390.5506043218	6.29651148781004e-05\\
-23384.6905139788	0.000149310643285781\\
-23378.8304236357	0.000269511826509594\\
-23372.9703332926	0.00042073257420086\\
-23367.1102429496	0.000599404551240569\\
-23361.2501526065	0.000801311376971325\\
-23355.3900622635	0.0010216881225975\\
-23349.5299719204	0.00125533375879847\\
-23343.6698815773	0.00149673390007424\\
-23337.8097912343	0.0017401909490252\\
-23331.9497008912	0.001979958569241\\
-23326.0896105482	0.00221037731292248\\
-23320.2295202051	0.00242600820243765\\
-23314.369429862	0.00262176111254601\\
-23308.509339519	0.00279301492349569\\
-23302.6492491759	0.00293572660830513\\
-23296.7891588329	0.00304652667970685\\
-23290.9290684898	0.00312279874297584\\
-23285.0689781467	0.00316274127696465\\
-23279.2088878037	0.00316541018410151\\
-23273.3487974606	0.00313074110472752\\
-23267.4887071176	0.00305955096821521\\
-23261.6286167745	0.00295351874348592\\
-23255.7685264314	0.00281514584262769\\
-23249.9084360884	0.00264769711123416\\
-23244.0483457453	0.00245512379765688\\
-23238.1882554023	0.00224197031848113\\
-23232.3281650592	0.00201326702029599\\
-23226.4680747162	0.0017744114680305\\
-23220.6079843731	0.00153104106156609\\
-23214.74789403	0.00128889998659323\\
-23208.887803687	0.00105370363988343\\
-23203.0277133439	0.000831003728648045\\
-23197.1676230009	0.000626057228032108\\
-23191.3075326578	0.000443702289782391\\
-23185.4474423147	0.000288244031139553\\
-23179.5873519717	0.000163352900106792\\
-23173.7272616286	7.19780163018575e-05\\
-23167.8671712856	1.62775334424545e-05\\
-23162.0070809425	-2.43233227679195e-06\\
-23156.1469905994	1.62915959366599e-05\\
-23150.2869002564	7.20089582365438e-05\\
-23144.4268099133	0.000163406214180456\\
-23138.5667195703	0.000288327608511191\\
-23132.7066292272	0.000443826016700346\\
-23126.8465388841	0.000626232471060041\\
-23120.9864485411	0.000831242727223982\\
-23115.126358198	0.00105401882863958\\
-23109.266267855	0.00128930327248653\\
-23103.4061775119	0.00153154308315336\\
-23097.5460871688	0.00177502086512199\\
-23091.6859968258	0.00201398974254761\\
-23085.8259064827	0.00224280900082403\\
-23079.9658161397	0.0024560772285584\\
-23074.1057257966	0.00264875981775918\\
-23068.2456354535	0.00281630781211539\\
-23062.3855451105	0.00295476529839557\\
-23056.5254547674	0.00306086280545275\\
-23050.6653644244	0.00313209450585511\\
-23044.8052740813	0.0031667773971503\\
-23038.9451837382	0.00316409106521263\\
-23033.0850933952	0.00312409708999906\\
-23027.2250030521	0.00304773763518869\\
-23021.3649127091	0.00293681325323594\\
-23015.504822366	0.00279394042944539\\
-23009.6447320229	0.00262248986564705\\
-23003.7846416799	0.00242650695928725\\
-22997.9245513368	0.00221061635395005\\
-22992.0644609938	0.00197991281340175\\
-22986.2043706507	0.00173984099396714\\
-22980.3442803077	0.00149606695238417\\
-22974.4841899646	0.0012543444214817\\
-22968.6240996215	0.00102037900962845\\
-22962.7640092785	0.000799693528888538\\
-22956.9039189354	0.000597497630761689\\
-22951.0438285924	0.000418564826472744\\
-22945.1837382493	0.000267119794869069\\
-22939.3236479062	0.000146738638253502\\
-22933.4635575632	6.02644410705445e-05\\
-22927.6034672201	9.74012524708163e-06\\
-22921.7433768771	-3.63981209199257e-06\\
-22915.883286534	2.04424615879587e-05\\
-22910.0231961909	8.14205645561398e-05\\
-22904.1631058479	0.000177857220571269\\
-22898.3030155048	0.000307478134835005\\
-22892.4429251618	0.000467225627731655\\
-22886.5828348187	0.000653330761546587\\
-22880.7227444756	0.000861402258853052\\
-22874.8626541326	0.00108653011481503\\
-22869.0025637895	0.00132340145869683\\
-22863.1424734465	0.00156642593077006\\
-22857.2823831034	0.00180986761577869\\
-22851.4222927603	0.00204798041953301\\
-22845.5622024173	0.00227514369367563\\
-22839.7021120742	0.00248599490723884\\
-22833.8420217312	0.00267555623391138\\
-22827.9819313881	0.00283935206639484\\
-22822.121841045	0.00297351468418514\\
-22816.261750702	0.00307487557957131\\
-22810.4016603589	0.00314104028593801\\
-22804.5415700159	0.00317044494088894\\
-22798.6814796728	0.0031623932482611\\
-22792.8213893298	0.00311707296543532\\
-22786.9612989867	0.00303555152548288\\
-22781.1012086436	0.00291975089634746\\
-22775.2411183006	0.00277240226895486\\
-22769.3810279575	0.0025969816427352\\
-22763.5209376145	0.00239762782755336\\
-22757.6608472714	0.00217904479632731\\
-22751.8007569283	0.00194639069217638\\
-22745.9406665853	0.00170515610887546\\
-22740.0805762422	0.00146103451687346\\
-22734.2204858992	0.00121978789258141\\
-22728.3603955561	0.000987110722275153\\
-22722.500305213	0.00076849559011972\\
-22716.64021487	0.000569103523113763\\
-22710.7801245269	0.000393642153240611\\
-22704.9200341839	0.00024625457320355\\
-22699.0599438408	0.000130421509569822\\
-22693.1998534977	4.8879123236168e-05\\
-22687.3397631547	3.55437834762091e-06\\
-22681.4796728116	-4.4804937216856e-06\\
-22675.6195824686	2.49666407026223e-05\\
-22669.7594921255	9.12032237855432e-05\\
-22663.8994017824	0.000192668293299023\\
-22658.0393114394	0.000326969267657327\\
-22652.1792210963	0.000490938362023845\\
-22646.3191307533	0.000680707305269361\\
-22640.4590404102	0.000891798595481356\\
-22634.5989500671	0.00111923114136709\\
-22628.7388597241	0.00135763779703005\\
-22622.878769381	0.00160139201688716\\
-22617.018679038	0.00184474064196413\\
-22611.1585886949	0.00208193968385947\\
-22605.2984983518	0.00230738990181894\\
-22599.4384080088	0.00251576897278676\\
-22593.5783176657	0.00270215713492345\\
-22587.7182273227	0.00286215333808297\\
-22581.8581369796	0.00299197915987841\\
-22575.9980466366	0.00308856803303492\\
-22570.1379562935	0.00314963767746706\\
-22564.2778659504	0.00317374402608486\\
-22558.4177756074	0.00316031537020093\\
-22552.5576852643	0.00310966591729985\\
-22546.6975949213	0.00302298843953884\\
-22540.8375045782	0.00290232618492463\\
-22534.9774142351	0.00275052471302078\\
-22529.1173238921	0.00257116478977409\\
-22523.257233549	0.0023684779246202\\
-22517.397143206	0.0021472465417992\\
-22511.5370528629	0.00191269114091845\\
-22505.6769625198	0.00167034710918789\\
-22499.8168721768	0.00142593409216422\\
-22493.9567818337	0.00118522100549985\\
-22488.0966914907	0.000953889873499597\\
-22482.2366011476	0.000737401708231853\\
-22476.3765108045	0.000540867594888604\\
-22470.5164204615	0.00036892802650677\\
-22464.6563301184	0.000225643336778932\\
-22458.7962397754	0.000114397817855907\\
-22452.9361494323	3.78197872575473e-05\\
-22447.0760590892	-2.28050819703234e-06\\
-22441.2159687462	-4.95368442179901e-06\\
-22435.3558784031	2.98663034621566e-05\\
-22429.4957880601	0.00010136053232084\\
-22423.635697717	0.000207844369965308\\
-22417.7756073739	0.000346807168967215\\
-22411.9155170309	0.000514971460086131\\
-22406.0554266878	0.000708370249733787\\
-22400.1953363448	0.000922440598519648\\
-22394.3352460017	0.00115213127346555\\
-22388.4751556586	0.00139202193415246\\
-22382.6150653156	0.00163645104061069\\
-22376.7549749725	0.00187964946473694\\
-22370.8948846295	0.00211587665201912\\
-22365.0347942864	0.00233955612011781\\
-22359.1747039434	0.00254540709595185\\
-22353.3146136003	0.00272856918415062\\
-22347.4545232572	0.00288471712361161\\
-22341.5944329142	0.00301016292294484\\
-22335.7343425711	0.00310194296288637\\
-22329.8742522281	0.00315788800870175\\
-22324.014161885	0.00317667447853186\\
-22318.1540715419	0.00315785575616815\\
-22312.2939811989	0.00310187280745876\\
-22306.4338908558	0.00301004384776905\\
-22300.5738005128	0.00288453330298151\\
-22294.7137101697	0.00272830079464606\\
-22288.8536198266	0.0025450313515061\\
-22282.9935294836	0.00233904849334019\\
-22277.1334391405	0.00211521223675625\\
-22271.2733487975	0.00187880442921946\\
-22265.4132584544	0.00163540411647749\\
-22259.5531681113	0.00139075588434417\\
-22253.6930777683	0.00115063428147733\\
-22247.8329874252	0.000920707523120472\\
-22241.9728970822	0.000706403692488533\\
-22236.1128067391	0.000512782598102907\\
-22230.252716396	0.000344416312183844\\
-22224.392626053	0.000205281210359837\\
-22218.5325357099	9.86640620266708e-05\\
-22212.6724453669	2.7084389079859e-05\\
-22206.8123550238	-7.76507293368395e-06\\
-22200.9522646807	-5.05840432979276e-06\\
-22195.0921743377	3.51439251248846e-05\\
-22189.2320839946	0.000111896402403045\\
-22183.3719936516	0.00022339070732694\\
-22177.5119033085	0.000366998315679557\\
-22171.6518129654	0.000539332465649766\\
-22165.7917226224	0.000736328026782848\\
-22159.9316322793	0.000953337388016618\\
-22154.0715419363	0.0011852401029903\\
-22148.2114515932	0.00142656370606698\\
-22142.3513612502	0.001671612848364\\
-22136.4912709071	0.00191460370676974\\
-22130.631180564	0.00214980049387174\\
-22124.771090221	0.00237165084697883\\
-22118.9109998779	0.00257491690069267\\
-22113.0509095349	0.00275479894871231\\
-22107.1908191918	0.00290704877566157\\
-22101.3307288487	0.00302806998259028\\
-22095.4706385057	0.00311500293750803\\
-22089.6105481626	0.00316579234372959\\
-22083.7504578196	0.00317923583013728\\
-22077.8903674765	0.00315501241427375\\
-22072.0302771334	0.003093690164508\\
-22066.1701867904	0.00299671287874913\\
-22060.3100964473	0.00286636609161833\\
-22054.4500061043	0.00270572321065645\\
-22048.5899157612	0.00251857305087442\\
-22042.7298254181	0.00230933047621894\\
-22036.8697350751	0.00208293225547603\\
-22031.009644732	0.00184472058897597\\
-22025.149554389	0.00160031705445061\\
-22019.2894640459	0.00135548994578969\\
-22013.4293737028	0.00111601813571697\\
-22007.5692833598	0.000887554674900925\\
-22001.7091930167	0.000675493347235683\\
-21995.8491026737	0.000484841331038173\\
-21989.9890123306	0.000320100972490042\\
-21984.1289219875	0.000185163462725372\\
-21978.2688316445	8.32169292912925e-05\\
-21972.4087413014	1.66711127725148e-05\\
-21966.5486509584	-1.28995918351125e-05\\
-21960.6885606153	-4.79338274669962e-06\\
-21954.8284702722	4.08022931965327e-05\\
-21948.9683799292	0.000122815072568025\\
-21943.1082895861	0.000239312894783654\\
-21937.2481992431	0.000387549515836936\\
-21931.3881089	0.000564029244210892\\
-21925.5280185569	0.000764589372995192\\
-21919.6679282139	0.000984498364458058\\
-21913.8078378708	0.00121856747127369\\
-21907.9477475278	0.00146127316119429\\
-21902.0876571847	0.00170688745674643\\
-21896.2275668417	0.00194961311443229\\
-21890.3674764986	0.00218372045324131\\
-21884.5073861555	0.00240368260310614\\
-21878.6472958125	0.00260430598086248\\
-21872.7872054694	0.00278085291357972\\
-21866.9271151264	0.00292915351416795\\
-21861.0670247833	0.00304570416747892\\
-21855.2069344402	0.00312775030068174\\
-21849.3468440972	0.0031733514830158\\
-21843.4867537541	0.00318142731597603\\
-21837.6266634111	0.00315178302898263\\
-21831.766573068	0.00308511417400602\\
-21825.9064827249	0.00298299030619131\\
-21820.0463923819	0.00284781803319105\\
-21814.1863020388	0.00268278430328859\\
-21808.3262116958	0.00249178126828024\\
-21802.4661213527	0.00227931449321152\\
-21796.6060310096	0.00205039667687175\\
-21790.7459406666	0.00181042939062635\\
-21784.8858503235	0.00156507562502382\\
-21779.0257599805	0.0013201261518781\\
-21773.1656696374	0.0010813628551111\\
-21767.3055792943	0.000854422255905918\\
-21761.4454889513	0.000644662453455298\\
-21755.5853986082	0.000457036622334211\\
-21749.7253082652	0.000295976052946832\\
-21743.8652179221	0.000165285496909142\\
-21738.0051275791	6.8053288864956e-05\\
-21732.145037236	6.57836785429514e-06\\
-21726.2849468929	-1.76840772969702e-05\\
-21720.4248565499	-4.15705347971521e-06\\
-21714.5647662068	4.68445156296738e-05\\
-21708.7046758638	0.000134121119197111\\
-21702.8445855207	0.00025561686900393\\
-21696.9844951776	0.000408467925958908\\
-21691.1244048346	0.000589070002689916\\
-21685.2643144915	0.000793163351206407\\
-21679.4042241485	0.00101593323165793\\
-21673.5441338054	0.00125212349272012\\
-21667.6840434623	0.00149616058417645\\
-21661.8239531193	0.0017422850759966\\
-21655.9638627762	0.0019846875803692\\
-21650.1037724332	0.00221764586926778\\
-21644.2436820901	0.00243565995089277\\
-21638.383591747	0.00263358191700662\\
-21632.523501404	0.00280673749533747\\
-21626.6634110609	0.00295103643708406\\
-21620.8033207179	0.00306306913297706\\
-21614.9432303748	0.00314018717540569\\
-21609.0831400317	0.00318056596340672\\
-21603.2230496887	0.00318324787040729\\
-21597.3629593456	0.0031481649537589\\
-21591.5028690026	0.00307614066740391\\
-21585.6427786595	0.00296887053453593\\
-21579.7826883164	0.00282888223403618\\
-21573.9225979734	0.00265947604038275\\
-21568.0625076303	0.00246464701992132\\
-21562.2024172873	0.00224899081828803\\
-21556.3423269442	0.00201759526007206\\
-21550.4822366011	0.00177592031809374\\
-21544.6221462581	0.00152966928354407\\
-21538.762055915	0.00128465417689415\\
-21532.901965572	0.00104665857552332\\
-21527.0418752289	0.000821301095556858\\
-21521.1817848858	0.000613902750050097\\
-21515.3216945428	0.00042936131505069\\
-21509.4616041997	0.000272035669565267\\
-21503.6015138567	0.000145642840780946\\
-21497.7414235136	5.31701861906142e-05\\
-21491.8813331706	-3.19521269947748e-06\\
-21486.0212428275	-2.21182763003273e-05\\
-21480.1611524844	-3.1475492157126e-06\\
-21474.3010621414	5.32740301088044e-05\\
-21468.4409717983	0.000145819469240684\\
-21462.5808814553	0.000272308929783336\\
-21456.7207911122	0.000429761069651542\\
-21450.8607007691	0.000614463310369162\\
-21445.0006104261	0.000822059373266662\\
-21439.140520083	0.00104765202075956\\
-21433.28042974	0.00128591857954535\\
-21427.4203393969	0.00153123652077801\\
-21421.5602490538	0.0017778161343922\\
-21415.7001587108	0.00201983716683753\\
-21409.8400683677	0.00225158619786335\\
-21403.9799780247	0.00246759151377172\\
-21398.1198876816	0.00266275229325185\\
-21392.2597973385	0.00283245905618398\\
-21386.3997069955	0.00297270253069246\\
-21380.5396166524	0.00308016836798435\\
-21374.6795263094	0.00315231546681139\\
-21368.8194359663	0.00318743605714714\\
-21362.9593456232	0.00318469612222795\\
-21357.0992552802	0.00314415520234353\\
-21351.2391649371	0.00306676510985293\\
-21345.3790745941	0.00295434758344741\\
-21339.518984251	0.00280955140584624\\
-21333.6588939079	0.00263578999428017\\
-21327.7988035649	0.00243716093445656\\
-21321.9387132218	0.00221834935470449\\
-21316.0786228788	0.00198451741860817\\
-21310.2185325357	0.00174118254285544\\
-21304.3584421926	0.00149408721301988\\
-21298.4983518496	0.00124906346904916\\
-21292.6382615065	0.00101189525817956\\
-21286.7781711635	0.000788181904068069\\
-21280.9180808204	0.000583205914801907\\
-21275.0579904774	0.000401808250545425\\
-21269.1979001343	0.000248273996051639\\
-21263.3378097912	0.000126231137913003\\
-21257.4777194482	3.85648376552169e-05\\
-21251.6176291051	-1.26507737700991e-05\\
-21245.7575387621	-2.62016677201703e-05\\
-21239.897448419	-1.76269488871369e-06\\
-21234.0373580759	6.00946144661507e-05\\
-21228.1772677329	0.000157915414167216\\
-21222.3171773898	0.000289395757199251\\
-21216.4570870468	0.000451436857593549\\
-21210.5969967037	0.00064021812117586\\
-21204.7369063606	0.000851287224124466\\
-21198.8768160176	0.00107966511550895\\
-21193.0167256745	0.00131996346766029\\
-21187.1566353315	0.00156651180374478\\
-21181.2965449884	0.00181349130351671\\
-21175.4364546453	0.00205507212966843\\
-21169.5763643023	0.00228555103410204\\
-21163.7162739592	0.00249948599611892\\
-21157.8561836162	0.00269182471422838\\
-21151.9960932731	0.00285802391758802\\
-21146.13600293	0.00299415667996184\\
-21140.275912587	0.00309700520125194\\
-21134.4158222439	0.00316413686430596\\
-21128.5557319009	0.00319396176997816\\
-21122.6956415578	0.00318577038886933\\
-21116.8355512147	0.00313975043883375\\
-21110.9754608717	0.00305698258643913\\
-21105.1153705286	0.00293941507080181\\
-21099.2552801856	0.00278981784564513\\
-21093.3951898425	0.00261171731907342\\
-21087.5350994994	0.00240931322825304\\
-21081.6750091564	0.00218737960872977\\
-21075.8149188133	0.00195115219337911\\
-21069.9548284703	0.00170620489645645\\
-21064.0947381272	0.00145831829705226\\
-21058.2346477842	0.00121334322453697\\
-21052.3745574411	0.000977062665328763\\
-21046.514467098	0.000755055250273588\\
-21040.654376755	0.000552563544909791\\
-21034.7942864119	0.000374370252199059\\
-21028.9341960689	0.000224685251098001\\
-21023.0741057258	0.000107046138705955\\
-21017.2140153827	2.4234625778998e-05\\
-21011.3539250397	-2.1789238725292e-05\\
-21005.4938346966	-2.99334587582342e-05\\
nan	nan\\
-20993.7736540105	6.7310398293246e-05\\
-20987.9135636674	0.000170414625237528\\
-20982.0534733244	0.000306884430186792\\
-20976.1933829813	0.000473503608921444\\
-20970.3332926383	0.000666343797465135\\
-20964.4732022952	0.000880857087363718\\
-20958.6131119521	0.00111198327904755\\
-20952.7530216091	0.00135426924396446\\
-20946.892931266	0.00160199757994134\\
-20941.032840923	0.00184932152459245\\
-20935.1727505799	0.00209040294326255\\
-20929.3126602368	0.00231955013498935\\
-20923.4525698938	0.00253135220365635\\
-20917.5924795507	0.00272080682239459\\
-20911.7323892077	0.00288343837409154\\
-20905.8722988646	0.00301540367833292\\
-20900.0122085215	0.00311358280714593\\
-20894.1521181785	0.00317565284344941\\
-20888.2920278354	0.00320014283863114\\
-20882.4319374924	0.00318646866930303\\
-20876.5718471493	0.00313494696671604\\
-20870.7117568062	0.00304678778741879\\
-20864.8516664632	0.00292406619401881\\
-20858.9915761201	0.00276967341411233\\
-20853.1314857771	0.00258724872640782\\
-20847.271395434	0.00238109367907378\\
-20841.411305091	0.00215607066185803\\
-20835.5512147479	0.00191748822394495\\
-20829.6911244048	0.00167097584175276\\
-20823.8310340618	0.0014223510915718\\
-20817.9709437187	0.00117748236034001\\
-20812.1108533757	0.000942150334932957\\
-20806.2507630326	0.000721911538721629\\
-20800.3906726895	0.000521967137027511\\
-20794.5305823465	0.000347040108862511\\
-20788.6704920034	0.000201263685613291\\
-20782.8104016604	8.80836917304688e-05\\
-20776.9503113173	1.01770948662542e-05\\
-20771.0902209742	-3.0611309356979e-05\\
-20765.2301306312	-3.33125804714552e-05\\
-20759.3700402881	2.14335015914561e-06\\
-20753.5099499451	7.49258758179725e-05\\
-20747.649859602	0.000183323170174404\\
-20741.7897692589	0.000324782446600355\\
-20735.9296789159	0.000495970074259087\\
-20730.0695885728	0.000692850135347828\\
-20724.2094982298	0.000910779572327836\\
-20718.3494078867	0.00114461768209432\\
-20712.4893175436	0.00138884737503024\\
-20706.6292272006	0.00163770533880346\\
-20700.7691368575	0.00188531803599131\\
-20694.9090465145	0.00212584032652471\\
-20689.0489561714	0.00235359344330247\\
-20683.1888658283	0.00256319906432044\\
-20677.3287754853	0.00274970631572453\\
-20671.4686851422	0.00290870870720147\\
-20665.6085947992	0.00303644823754358\\
-20659.7485044561	0.00312990421121617\\
-20653.8884141131	0.00318686466657632\\
-20648.02832377	0.003205978726969\\
-20642.1682334269	0.00318678863591547\\
-20636.3081430839	0.00312974071630488\\
-20630.4480527408	0.00303617499155613\\
-20624.5879623978	0.00290829371006961\\
-20618.7278720547	0.0027491095121929\\
-20612.8677817116	0.00256237445937183\\
-20607.0076913686	0.00235249159792822\\
-20601.1476010255	0.00212441114117562\\
-20595.2875106825	0.00188351371818666\\
-20589.4274203394	0.00163548344248802\\
-20583.5673299963	0.0013861737950614\\
-20577.7072396533	0.00114146948591178\\
-20571.8471493102	0.000907147554284746\\
-20565.9870589672	0.00068874098589843\\
-20560.1269686241	0.000491408066685149\\
-20554.266878281	0.000319810557931723\\
-20548.406787938	0.000178003569884647\\
-20542.5466975949	6.93397352561368e-05\\
-20536.6866072519	-3.61005289553254e-06\\
-20530.8265169088	-3.91174651187153e-05\\
-20524.9664265657	-3.63376823612654e-05\\
-20519.1063362227	4.67050443822302e-06\\
-20513.2462458796	8.2945920120351e-05\\
-20507.3861555366	0.000196647531327036\\
-20501.5260651935	0.000343097744896139\\
-20495.6659748504	0.000518845460369104\\
-20489.8058845074	0.000719747391844812\\
-20483.9457941643	0.000941065743037567\\
-20478.0857038213	0.00117757993293203\\
-20472.2256134782	0.00142370973740285\\
-20466.3655231351	0.00167364694233901\\
-20460.5054327921	0.0019214924022144\\
-20454.645342449	0.00216139526995276\\
-20448.785252106	0.00238769111223189\\
-20442.9251617629	0.00259503564985561\\
-20437.0650714198	0.00277853096561077\\
-20431.2049810768	0.00293384119935575\\
-20425.3448907337	0.00305729499728184\\
-20419.4848003907	0.0031459722950326\\
-20413.6247100476	0.00319777338318522\\
-20407.7646197046	0.00321146862150076\\
-20401.9045293615	0.00318672762535123\\
-20396.0444390184	0.00312412723097415\\
-20390.1843486754	0.00302513804812941\\
-20384.3242583323	0.00289208991347358\\
-20378.4641679893	0.00272811705584226\\
-20372.6040776462	0.00253708426446659\\
-20366.7439873031	0.00232349579893282\\
-20360.8838969601	0.00209238918780621\\
-20355.023806617	0.00184921642009313\\
-20349.163716274	0.00159971533113736\\
-20343.3036259309	0.00134977421724108\\
-20337.4435355878	0.00110529287324029\\
-20331.5834452448	0.000872043332353873\\
-20325.7233549017	0.000655533595495984\\
-20319.8632645587	0.000460877566986869\\
-20314.0031742156	0.000292674268021953\\
-20308.1430838725	0.000154899180587666\\
-20302.2829935295	5.0810288928693e-05\\
-20296.4229031864	-1.71289608970521e-05\\
-20290.5628128434	-4.73079616430469e-05\\
-20284.7027225003	-3.90071259829308e-05\\
-20278.8426321572	7.58495697702066e-06\\
-20272.9825418142	9.13757987774044e-05\\
-20267.1224514711	0.000210394625440384\\
-20261.2623611281	0.000361838727574037\\
-20255.402270785	0.000542139456705485\\
-20249.5421804419	0.000747046313836898\\
-20243.6820900989	0.000971727148917599\\
-20237.8219997558	0.00121088210922227\\
-20231.9619094128	0.00145886864972079\\
-20226.1018190697	0.00170983465669336\\
-20220.2417287267	0.0019578565441099\\
-20214.3816383836	0.00219707906397015\\
-20208.5215480405	0.00242185353095053\\
-20202.6614576975	0.00262687119822895\\
-20196.8013673544	0.00280728863553673\\
-20190.9412770114	0.00295884214821608\\
-20185.0811866683	0.00307794853466199\\
-20179.2210963252	0.00316178980087591\\
-20173.3610059822	0.00320837982946083\\
-20167.5009156391	0.00321661142590622\\
-20161.6408252961	0.00318628262782528\\
-20155.780734953	0.00311810165184447\\
-20149.9206446099	0.00301367035727035\\
-20144.0605542669	0.00287544661260889\\
-20138.2004639238	0.0027066864489546\\
-20132.3403735808	0.00251136736170224\\
-20126.4802832377	0.00229409456716184\\
-20120.6201928946	0.00205999242345\\
-20114.7601025516	0.00181458357556377\\
-20108.9000122085	0.00156365867478947\\
-20103.0399218655	0.00131313974585914\\
-20097.1798315224	0.00106894042550905\\
-20091.3197411793	0.000836826370871963\\
-20085.4596508363	0.000622279132573751\\
-20079.5995604932	0.000430366706511684\\
-20073.7394701502	0.000265623821086051\\
-20067.8793798071	0.000131944787599777\\
-20062.019289464	3.24914455642226e-05\\
-20056.159199121	-3.03816211365579e-05\\
-20050.2991087779	-5.51828285233295e-05\\
-20044.4390184349	-4.1318977528854e-05\\
-20038.5789280918	1.08905596460874e-05\\
-20032.7188377487	0.000100221191034979\\
-20026.8587474057	0.000224571825151499\\
-20020.9986570626	0.000381014286463652\\
-20015.1385667196	0.000565862263930092\\
-20009.2784763765	0.000774758169158772\\
-20003.4183860334	0.00100277585766498\\
-19997.5582956904	0.00124453679189974\\
-19991.6982053473	0.0014943369067839\\
-19985.8381150043	0.00174628118556276\\
-19979.9780246612	0.00199442277095076\\
-19974.1179343182	0.00223290332879933\\
-19968.2578439751	0.00245609135167552\\
-19962.397753632	0.00265871513697177\\
-19956.537663289	0.00283598730003151\\
-19950.6775729459	0.00298371788102505\\
-19944.8174826029	0.00309841337376431\\
-19938.9573922598	0.00317735933591647\\
-19933.0973019167	0.00321868462698279\\
-19927.2372115737	0.00322140575420832\\
-19921.3771212306	0.00318545027524024\\
-19915.5170308876	0.00311165870084367\\
-19909.6569405445	0.00300176484823732\\
-19903.7968502014	0.00285835510397872\\
-19897.9367598584	0.00268480755391253\\
-19892.0766695153	0.00248521241217128\\
-19886.2165791723	0.00226427562413767\\
-19880.3564888292	0.00202720791431624\\
-19874.4963984861	0.00177960189582683\\
-19868.6363081431	0.00152730013897931\\
-19862.7762178	0.00127625731164266\\
-19856.916127457	0.00103239964388067\\
-19851.0560371139	0.000801485033775356\\
-19845.1959467708	0.000588967096438991\\
-19839.3358564278	0.000399866366243662\\
-19833.4757660847	0.000238651693966943\\
-19827.6156757417	0.000109134640544155\\
-19821.7555853986	1.43793630239751e-05\\
-19815.8954950555	-4.33698768218519e-05\\
-19810.0354047125	-6.27418663285009e-05\\
-19804.1753143694	-4.32709993287928e-05\\
-19798.3152240264	1.45915358610645e-05\\
-19792.4551336833	0.000109488206609079\\
-19786.5950433403	0.000239186982342586\\
-19780.7349529972	0.000400633830089154\\
-19774.8748626541	0.000590024624599525\\
-19769.0147723111	0.000802894779875644\\
-19763.154681968	0.00103422449039854\\
-19757.294591625	0.0012785571011974\\
-19751.4345012819	0.00153012781571654\\
-19745.5744109388	0.00178299970573255\\
-19739.7143205958	0.00203120381427884\\
-19733.8542302527	0.00226888004578474\\
-19727.9941399097	0.00249041551766442\\
-19722.1340495666	0.00269057710737819\\
-19716.2739592235	0.00286463506467592\\
-19710.4138688805	0.00300847476949458\\
-19704.5537785374	0.00311869399495326\\
-19698.6936881944	0.0031926833759882\\
-19692.8335978513	0.00322868818089843\\
-19686.9735075082	0.00322584992333993\\
-19681.1134171652	0.00318422682797309\\
-19675.2533268221	0.0031047926621266\\
-19669.3932364791	0.00298941395599161\\
-19663.533146136	0.00284080614424192\\
-19657.6730557929	0.00266246965982387\\
-19651.8129654499	0.00245860748341024\\
-19645.9528751068	0.00223402609081966\\
-19640.0927847638	0.00199402213284423\\
-19634.2326944207	0.00174425751862979\\
-19628.3726040776	0.0014906258491366\\
-19622.5125137346	0.00123911335097204\\
-19616.6524233915	0.000995657592319269\\
-19610.7923330485	0.000766007315066848\\
-19604.9322427054	0.000555586692197041\\
-19599.0721523623	0.00036936721547464\\
-19593.2120620193	0.000211750239203305\\
-19587.3519716762	8.64629550001872e-05\\
-19581.4918813332	-3.5297438650051e-06\\
-19575.6317909901	-5.60954245490624e-05\\
-19569.7717006471	-6.99846428289845e-05\\
-19563.911610304	-4.48606402078331e-05\\
-19558.0515199609	1.869249586499e-05\\
-19552.1914296179	0.000119183406251231\\
-19546.3313392748	0.00025424845348466\\
-19540.4712489318	0.000420707313197848\\
-19534.6111585887	0.000614637856179421\\
-19528.7510682456	0.000831468558037372\\
-19522.8909779026	0.00106608625917946\\
-19517.0308875595	0.00131295673521743\\
-19511.1707972165	0.00156625523458023\\
-19505.3107068734	0.00182000390460429\\
-19499.4506165303	0.00206821286369505\\
-19493.5905261873	0.00230502159057735\\
-19487.7304358442	0.00252483729308094\\
-19481.8703455012	0.00272246699002114\\
-19476.0102551581	0.00289324018648735\\
-19470.150164815	0.00303311924507743\\
-19464.290074472	0.00313879484447176\\
-19458.4299841289	0.00320776426904855\\
-19452.5698937859	0.00323839067706191\\
-19446.7098034428	0.00322994194408079\\
-19440.8497130997	0.00318260816004539\\
-19434.9896227567	0.00309749736159997\\
-19429.1295324136	0.00297660959547147\\
-19423.2694420706	0.00282278991991434\\
-19417.4093517275	0.002639661448472\\
-19411.5492613844	0.0024315400118711\\
-19405.6891710414	0.00220333244790632\\
-19399.8290806983	0.00196042091644812\\
-19393.9689903553	0.0017085359665373\\
-19388.1089000122	0.00145362134940467\\
-19382.2488096691	0.00120169376626511\\
-19376.3887193261	0.000958700860340368\\
-19370.528628983	0.000730380804708862\\
-19364.66853864	0.000522126800690162\\
-19358.8084482969	0.000338859686494413\\
-19352.9483579539	0.000184911665046173\\
-19347.0882676108	6.39238983051875e-05\\
-19341.2281772677	-2.12396113792094e-05\\
-19335.3680869247	-6.85598160528999e-05\\
-19329.5079965816	-7.69104883892955e-05\\
-19323.6479062386	-4.60850246296429e-05\\
-19317.7878158955	2.3198453580295e-05\\
-19311.9277255524	0.000129313824193651\\
-19306.0676352094	0.000269765127162093\\
-19300.2075448663	0.000441245268687923\\
-19294.3474545233	0.000639713886660138\\
-19288.4873641802	0.00086049254402621\\
-19282.6272738371	0.0010983750073472\\
-19276.7671834941	0.00134775001116313\\
-19270.907093151	0.00160273361346353\\
-19265.047002808	0.00185730802041086\\
-19259.1869124649	0.00210546360503902\\
-19253.3268221218	0.00234134076831363\\
-19247.4667317788	0.00255936829414784\\
-19241.6066414357	0.00275439493145204\\
-19235.7465510927	0.00292181109561528\\
-19229.8864607496	0.00305765781455272\\
-19224.0263704065	0.00315872034386454\\
-19218.1662800635	0.00322260423823502\\
-19212.3061897204	0.00324779207856016\\
-19206.4460993774	0.00323367951106862\\
-19200.5860090343	0.00318058974261593\\
-19194.7259186912	0.00308976614442579\\
-19188.8658283482	0.00296334313373728\\
-19183.0057380051	0.00280429601460806\\
-19177.1456476621	0.00261637095717588\\
-19171.285557319	0.00240399676273911\\
-19165.4254669759	0.00217218049308533\\
-19159.5653766329	0.0019263894235084\\
-19153.7052862898	0.00167242210244262\\
-19147.8451959468	0.0014162715587859\\
-19141.9851056037	0.00116398388359977\\
-19136.1250152607	0.00092151552327798\\
-19130.2649249176	0.000694592652493112\\
-19124.4048345745	0.000488575946729301\\
-19118.5447442315	0.000308333947919791\\
-19112.6846538884	0.000158128014557549\\
-19106.8245635454	4.15115748822309e-05\\
-19100.9644732023	-3.87539349948621e-05\\
-19095.1043828592	-8.07644595356537e-05\\
-19089.2442925162	-8.35184904910721e-05\\
-19083.3842021731	-4.69409405442995e-05\\
-19077.5241118301	2.81148451499311e-05\\
-19071.664021487	0.000139886992642582\\
-19065.8039311439	0.000285746453934804\\
-19059.9438408009	0.00046225884215683\\
-19054.0837504578	0.000665265292906495\\
-19048.2236601148	0.000889980447925988\\
-19042.3635697717	0.00113110525273996\\
-19036.5034794286	0.00138295190946436\\
-19030.6433890856	0.00163957803855868\\
-19024.7832987425	0.00189492688498233\\
-19018.9232083995	0.00214297026078808\\
-19013.0631180564	0.00237785085088631\\
-19007.2030277133	0.0025940205224798\\
-19001.3429373703	0.00278637137247749\\
-18995.4828470272	0.00295035641776533\\
-18989.6227566842	0.00308209707643156\\
-18983.7626663411	0.00317847489959295\\
-18977.902575998	0.00323720538448551\\
-18972.042485655	0.00325689212143291\\
-18966.1823953119	0.00323705999149077\\
-18960.3223049689	0.00317816662590684\\
-18954.4622146258	0.00308159185063536\\
-18948.6021242827	0.00294960535948395\\
-18942.7420339397	0.00278531337337056\\
-18936.8819435966	0.00259258553886001\\
-18931.0218532536	0.0023759637864148\\
-18925.1617629105	0.00214055529512317\\
-18919.3016725674	0.00189191208585699\\
-18913.4415822244	0.00163590008177391\\
-18907.5814918813	0.00137856072409014\\
-18901.7214015383	0.00112596840758451\\
-18895.8613111952	0.00088408709999427\\
-18890.0012208522	0.000658629529578646\\
-18884.1411305091	0.000454922265381188\\
-18878.281040166	0.00027777987656005\\
-18872.420949823	0.00013139114364504\\
-18866.5608594799	1.92200110015848e-05\\
-18860.7007691369	-5.60763775477899e-05\\
-18854.8406787938	-9.27106205544781e-05\\
-18848.9805884507	-8.98074873347542e-05\\
-18843.1204981077	-4.74248258504103e-05\\
-18837.2604077646	3.3447549295294e-05\\
-18831.4003174216	0.000150910968479387\\
-18825.5402270785	0.000302202478736994\\
-18819.6801367354	0.000483759829222655\\
-18813.8200463924	0.000691305342053608\\
-18807.9599560493	0.000919946693924104\\
-18802.0998657063	0.00116429223417216\\
-18796.2397753632	0.00141857812115274\\
-18790.3796850201	0.0016768042792222\\
-18784.5195946771	0.00193287596938492\\
-18778.659504334	0.00218074763341678\\
-18772.799413991	0.00241456561663699\\
-18766.9393236479	0.0026288064000839\\
-18761.0792333048	0.00281840707790009\\
-18755.2191429618	0.00297888499786085\\
-18749.3590526187	0.00310644373782046\\
-18743.4989622757	0.00319806291285394\\
-18737.6388719326	0.0032515696888833\\
-18731.7787815895	0.00326569030914807\\
-18725.9186912465	0.0032400804121195\\
-18720.0586009034	0.00317533341923403\\
-18714.1985105604	0.00307296678813353\\
-18708.3384202173	0.00293538644997722\\
-18702.4783298743	0.00276583026410804\\
-18696.6182395312	0.00256829181876632\\
-18690.7581491881	0.002347426371543\\
-18684.8980588451	0.00210844114424735\\
-18679.037968502	0.00185697255779655\\
-18673.177878159	0.00159895330114962\\
-18667.3177878159	0.00134047236945581\\
-18661.4576974728	0.00108763137293946\\
-18655.5976071298	0.000846400507645332\\
-18649.7375167867	0.000622477587468703\\
-18643.8774264437	0.000421153466158602\\
-18638.0173361006	0.000247187027568191\\
-18632.1572457575	0.000104692697937652\\
-18626.2971554145	-2.956861103588e-06\\
-18620.4370650714	-7.32105775005867e-05\\
-18614.5769747284	-0.000104399422478881\\
-18608.7168843853	-9.57760604676606e-05\\
-18602.8567940422	-4.75327533708477e-05\\
-18596.9967036992	3.92029096404471e-05\\
-18591.1366133561	0.000162394362353604\\
-18585.2765230131	0.000319143876072086\\
-18579.41643267	0.000505760715971023\\
-18573.5563423269	0.000717848036250807\\
-18567.6962519839	0.000950406468306214\\
-18561.8361616408	0.00119795196145568\\
-18555.9760712978	0.00145464509873329\\
-18550.1159809547	0.00171442883853689\\
-18544.2558906116	0.0019711714330792\\
-18538.3958002686	0.00221881115160648\\
-18532.5357099255	0.00245149939270799\\
-18526.6756195825	0.00266373880694001\\
-18520.8155292394	0.00285051316814547\\
-18514.9554388963	0.00300740592507665\\
-18509.0953485533	0.00313070463210016\\
-18503.2352582102	0.00321748878943333\\
-18497.3751678672	0.00326569901448781\\
-18491.5150775241	0.00327418590659551\\
-18485.654987181	0.00324273744517434\\
-18479.794896838	0.00317208426875\\
-18473.9348064949	0.00306388270339354\\
-18468.0747161519	0.00292067593503878\\
-18462.2146258088	0.00274583423561655\\
-18456.3545354658	0.00254347564764448\\
-18450.4944451227	0.00231836899430911\\
-18444.6343547796	0.00207582149875784\\
-18438.7742644366	0.00182155366113274\\
-18432.9141740935	0.00156156434323884\\
-18427.0540837505	0.00130198924221325\\
-18421.1939934074	0.00104895609258157\\
-18415.3339030643	0.000808440013231233\\
-18409.4738127213	0.000586122414200106\\
-18403.6137223782	0.000387256794770521\\
-18397.7536320352	0.000216544602743797\\
-18391.8935416921	7.80240883375843e-05\\
-18386.033451349	-2.50252198610244e-05\\
-18380.173361006	-9.01601573534626e-05\\
-18374.3132706629	-0.000115831846490102\\
-18368.4531803199	-0.000101422526372739\\
-18362.5930899768	-4.72604142253817e-05\\
-18356.7329996337	4.53877591612407e-05\\
-18350.8729092907	0.000174346370379179\\
-18345.0128189476	0.000336581988190763\\
-18339.1527286046	0.000528274722729892\\
-18333.2926382615	0.000744908160966237\\
-18327.4325479184	0.000981375771129694\\
-18321.5724575754	0.00123210126927024\\
-18315.7123672323	0.00149117011098425\\
-18309.8522768893	0.00175246900762923\\
-18303.9921865462	0.00200983017628503\\
-18298.1320962031	0.00225717691971971\\
-18292.2720058601	0.00248866710038391\\
-18286.411915517	0.00269883112095715\\
-18280.551825174	0.00288270115277156\\
-18274.6917348309	0.00303592855912138\\
-18268.8316444879	0.00315488673757002\\
-18262.9715541448	0.00323675694993573\\
-18257.1114638017	0.00327959510792965\\
-18251.2513734587	0.00328237793273344\\
-18245.3912831156	0.00324502739241868\\
-18239.5311927726	0.00316841283337592\\
-18233.6711024295	0.00305433074932621\\
-18227.8110120864	0.00290546265778688\\
-18221.9509217434	0.00272531207219984\\
-18216.0908314003	0.00251812205101473\\
-18210.2307410573	0.00228877526361677\\
-18204.3706507142	0.00204267892724433\\
-18198.5105603711	0.00178563732594442\\
-18192.6504700281	0.00152371491722304\\
-18186.790379685	0.00126309325460047\\
-18180.930289342	0.00100992510187001\\
-18175.0701989989	0.000770189181675641\\
-18169.2101086558	0.000549548987395443\\
-18163.3500183128	0.000353218992327754\\
-18157.4899279697	0.00018584141676491\\
-18151.6298376267	5.13764650934753e-05\\
-18145.7697472836	-4.69913718307363e-05\\
-18139.9096569405	-0.000106928732249591\\
-18134.0495665975	-0.000127008731112874\\
-18128.1894762544	-0.000106744926945176\\
-18122.3293859114	-4.6603099474696e-05\\
-18116.4692955683	5.20094469453597e-05\\
-18110.6092052252	0.000186776808663174\\
-18104.7491148822	0.000354528866579199\\
-18098.8890245391	0.000551315851514117\\
-18093.0289341961	0.000772501337282419\\
-18087.168843853	0.00101287147215145\\
-18081.3087535099	0.00126675787536547\\
-18075.4486631669	0.00152817130194673\\
-18069.5885728238	0.00179094292414862\\
-18063.7284824808	0.00204886989662629\\
-18057.8683921377	0.00229586177103383\\
-18052.0083017947	0.00252608430355498\\
-18046.1482114516	0.00273409726046745\\
-18040.2881211085	0.00291498296643236\\
-18034.4280307655	0.00306446255839037\\
-18028.5679404224	0.00317899719677166\\
-18022.7078500794	0.00325587184018219\\
-18016.8477597363	0.00329325960037216\\
-18010.9876693932	0.00329026515224825\\
-18005.1275790502	0.00324694616736504\\
-17999.2674887071	0.00316431225816866\\
-17993.4073983641	0.00304430145036805\\
-17987.547308021	0.00288973473190572\\
-17981.6872176779	0.00270424974416834\\
-17975.8271273349	0.0024922151740684\\
-17969.9670369918	0.00225862786163019\\
-17964.1069466488	0.00200899504622712\\
-17958.2468563057	0.00174920452650565\\
-17952.3867659626	0.00148538579471838\\
-17946.5266756196	0.00122376542092549\\
-17940.6665852765	0.000970520098580096\\
-17934.8064949335	0.000731630819955942\\
-17928.9464045904	0.000512741623865245\\
-17923.0863142473	0.000319026251571868\\
-17917.2262239043	0.000155065861107603\\
-17911.3661335612	2.47406902264884e-05\\
-17905.5060432182	-6.88617701286397e-05\\
-17899.6459528751	-0.000123519918815147\\
-17893.785862532	-0.000137930771273336\\
-17887.925772189	-0.000111741018777535\\
-17882.0656818459	-4.55556798860067e-05\\
-17876.2055915029	5.90758674657438e-05\\
-17870.3455011598	0.000199696150929326\\
-17864.4854108167	0.000372997317038402\\
-17858.6253204737	0.000574898937514028\\
-17852.7652301306	0.000800644078546268\\
-17846.9051397876	0.00104491137126863\\
-17841.0450494445	0.00130194044328263\\
-17835.1849591015	0.00156566775460915\\
-17829.3248687584	0.0018298696352209\\
-17823.4647784153	0.00208830914989589\\
-17817.6046880723	0.00233488332481833\\
-17811.7445977292	0.00256376726082426\\
-17805.8845073862	0.00276955173020013\\
-17800.0244170431	0.0029473710069704\\
-17794.1643267	0.00309301790968213\\
-17788.304236357	0.00320304333721315\\
-17782.4441460139	0.00327483794213662\\
-17776.5840556709	0.00330669400845668\\
-17770.7239653278	0.00329784606602108\\
-17764.8638749847	0.0032484892758537\\
-17759.0037846417	0.0031597751448728\\
-17753.1436942986	0.00303378466396183\\
-17747.2836039556	0.00287347949522077\\
-17741.4235136125	0.00268263235409453\\
-17735.5634232694	0.00246573822198747\\
-17729.7033329264	0.00222790847940802\\
-17723.8432425833	0.00197475045254338\\
-17717.9831522403	0.00171223521203553\\
-17712.1230618972	0.00144655674036633\\
-17706.2629715541	0.00118398578974072\\
-17700.4028812111	0.00093072187814373\\
-17694.542790868	0.000692746916920503\\
-17688.682700525	0.000475683925448585\\
-17682.8226101819	0.000284664169936385\\
-17676.9625198388	0.000124205865406549\\
-17671.1024294958	-1.89269188114981e-06\\
-17665.2423391527	-9.06430339274512e-05\\
-17659.3822488097	-0.000139937344294651\\
-17653.5221584666	-0.000148598516845465\\
-17647.6620681235	-0.000116408261157179\\
-17641.8019777805	-4.41125836659883e-05\\
-17635.9418874374	6.65954925906442e-05\\
-17630.0817970944	0.000213115569519962\\
-17624.2217067513	0.000392000948702619\\
-17618.3616164083	0.00059903970494476\\
-17612.5015260652	0.000829353851739215\\
-17606.6414357221	0.00107751426396015\\
-17600.7813453791	0.00133766865039903\\
-17594.921255036	0.0016036795596945\\
-17589.061164693	0.00186926916557996\\
-17583.2010743499	0.00212816741562344\\
-17577.3409840068	0.00237426004794859\\
-17571.4808936638	0.00260173298152088\\
-17565.6208033207	0.0028052096702421\\
-17559.7607129777	0.00297987817644177\\
-17553.9006226346	0.00312160496001904\\
-17548.0405322915	0.00322703269307582\\
-17542.1804419485	0.00329365978488761\\
-17536.3203516054	0.00331989973423141\\
-17530.4602612624	0.00330511890044406\\
-17524.6001709193	0.00324965179432829\\
-17518.7400805762	0.00315479352014854\\
-17512.8799902332	0.00302276953889241\\
-17507.0198998901	0.00285668345889799\\
-17501.1598095471	0.0026604440784261\\
-17495.299719204	0.00243867339507389\\
-17489.4396288609	0.0021965977470681\\
-17483.5795385179	0.0019399246500727\\
-17477.7194481748	0.00167470823173893\\
-17471.8593578318	0.00140720643686414\\
-17465.9992674887	0.00114373337056256\\
-17460.1391771456	0.000890510263788304\\
-17454.2790868026	0.000653518578438651\\
-17448.4189964595	0.00043835872060944\\
-17442.5589061165	0.000250117699026392\\
-17436.6988157734	9.32488559845947e-05\\
-17430.8387254303	-2.85334859471857e-05\\
-17424.9786350873	-0.00011234196916861\\
-17419.1185447442	-0.000156184656030153\\
-17413.2584544012	-0.000159012370673364\\
-17407.3983640581	-0.00012074380267877\\
-17401.5382737151	-4.22677719732014e-05\\
-17395.678183372	7.45774065927705e-05\\
-17389.8180930289	0.000227046980089\\
-17383.9580026859	0.000411554227368581\\
-17378.0979123428	0.000623754827763278\\
-17372.2378219998	0.000858649144152956\\
-17366.3777316567	0.00111070001227798\\
-17360.5176413136	0.00137396326148658\\
-17354.6575509706	0.00164222789021971\\
-17348.7974606275	0.00190916259096835\\
-17342.9373702845	0.00216846516790967\\
-17337.0772799414	0.00241401132129086\\
-17331.2171895983	0.00263999928608769\\
-17325.3570992553	0.0028410869089278\\
-17319.4970089122	0.00301251792512637\\
-17313.6369185692	0.00315023445060586\\
-17307.7768282261	0.00325097302838986\\
-17301.916737883	0.00331234195655322\\
-17296.05664754	0.0033328780652403\\
-17290.1965571969	0.00331208159513595\\
-17284.3364668539	0.00325042834598547\\
-17278.4763765108	0.00314935880011472\\
-17272.6162861677	0.00301124446952437\\
-17266.7561958247	0.00283933225223779\\
-17260.8961054816	0.00263766810384271\\
-17255.0360151386	0.00241100181814975\\
-17249.1759247955	0.00216467515785807\\
-17243.3158344524	0.00190449597017936\\
-17237.4557441094	0.00163660125346903\\
-17231.5956537663	0.00136731240367987\\
-17225.7355634233	0.00110298605450711\\
-17219.8754730802	0.000849864030840129\\
-17214.0153827372	0.000613925957206676\\
-17208.1552923941	0.000400748001395787\\
-17202.295202051	0.000215371090139906\\
-17196.435111708	6.21817112289388e-05\\
-17190.5750213649	-5.51918795326886e-05\\
-17184.7149310219	-0.000133965590637644\\
-17178.8548406788	-0.00017226553134028\\
-17172.9947503357	-0.000169172586033585\\
-17167.1346599927	-0.000124744466343105\\
-17161.2745696496	-4.00147120079053e-05\\
-17155.4144793066	8.30313444355445e-05\\
-17149.5543889635	0.000241503090366643\\
-17143.6942986204	0.000431672533579973\\
-17137.8342082774	0.000649061995683265\\
-17131.9741179343	0.000888549535702041\\
-17126.1140275913	0.00114448962183809\\
-17120.2539372482	0.00141084620853153\\
-17114.3938469051	0.00168133508208871\\
-17108.5337565621	0.00194957211792032\\
-17102.673666219	0.00220922395216201\\
-17096.813575876	0.00245415751154037\\
-17090.9534855329	0.00267858487124884\\
-17085.0933951898	0.00287720001974668\\
-17079.2333048468	0.00304530429881759\\
-17073.3732145037	0.00317891755331364\\
-17067.5131241607	0.0032748723616779\\
-17061.6530338176	0.0033308891162053\\
-17055.7929434745	0.00334563017360636\\
-17049.9328531315	0.00331873178943838\\
-17044.0727627884	0.00325081307430903\\
-17038.2126724454	0.00314346175192286\\
-17032.3525821023	0.00299919704571256\\
-17026.4924917592	0.00282141056221078\\
-17020.6324014162	0.00261428655787328\\
-17014.7723110731	0.00238270346375212\\
-17008.9122207301	0.0021321189855739\\
-17003.052130387	0.00186844148536767\\
-16997.1920400439	0.00159789067534705\\
-16991.3319497009	0.00132685090883612\\
-16985.4718593578	0.0010617205281706\\
-16979.6117690148	0.000808760824820732\\
-16973.7516786717	0.000573948176819946\\
-16967.8915883287	0.000362832855259908\\
-16962.0314979856	0.000180407835448466\\
-16956.1714076425	3.09907134717422e-05\\
-16950.3113172995	-8.18784793205741e-05\\
-16944.4512269564	-0.000155521145550106\\
-16938.5911366134	-0.000188183687877569\\
-16932.7310462703	-0.000179079263506846\\
-16926.8709559272	-0.000128406733021716\\
-16921.0108655842	-3.73463474530113e-05\\
-16915.1507752411	9.19677336562851e-05\\
-16909.2906848981	0.000256497453378197\\
-16903.430594555	0.000452372225917957\\
-16897.5705042119	0.000674979986037294\\
-16891.7104138689	0.000919075777729024\\
-16885.8503235258	0.0011789053255574\\
-16879.9902331828	0.0014483406774797\\
-16874.1301428397	0.00172102472175951\\
-16868.2700524966	0.00199052117006053\\
-16862.4099621536	0.00225046646811791\\
-16856.5498718105	0.00249472004890933\\
-16850.6897814675	0.0027175093805363\\
-16844.8296911244	0.00291356638295905\\
-16838.9696007813	0.00307825198979473\\
-16833.1095104383	0.00320766590989548\\
-16827.2494200952	0.00329873899231712\\
-16821.3893297522	0.00334930600689151\\
-16815.5292394091	0.00335815711499183\\
-16809.669149066	0.00332506680697236\\
-16803.809058723	0.00325079961408492\\
-16797.9489683799	0.00313709245131778\\
-16792.0888780369	0.00298661399816999\\
-16786.2287876938	0.00280290206770192\\
-16780.3686973508	0.00259028043329555\\
-16774.5086070077	0.00235375706848653\\
-16768.6485166646	0.00209890619477049\\
-16762.7884263216	0.00183173691520315\\
-16756.9283359785	0.00155855152981105\\
-16751.0682456355	0.00128579687304424\\
-16745.2081552924	0.00101991218018234\\
-16739.3480649493	0.000767177072449169\\
-16733.4879746063	0.00053356324939121\\
-16727.6278842632	0.000324593391174183\\
-16721.7677939202	0.000145210604301558\\
-16715.9077035771	-3.38503015062493e-07\\
-16710.047613234	-0.000108604350224984\\
-16704.187522891	-0.000177016138859087\\
-16698.3274325479	-0.00020394289452982\\
-16692.4673422049	-0.000188732347219367\\
-16686.6072518618	-0.000131726723126663\\
-16680.7471615187	-3.42550660060854e-05\\
-16674.8870711757	0.000101397740207191\\
-16669.0269808326	0.000272044525574439\\
-16663.1668904896	0.000473670710044766\\
-16657.3068001465	0.000701528742090925\\
-16651.4467098034	0.000950249878674834\\
-16645.5866194604	0.00121397067475159\\
-16639.7265291173	0.00148647120241221\\
-16633.8664387743	0.0017613217414424\\
-16628.0063484312	0.00203203448196694\\
-16622.1462580881	0.00229221665998899\\
-16616.2861677451	0.00253572151148781\\
-16610.426077402	0.00275679348066917\\
-16604.565987059	0.00295020425211911\\
-16598.7058967159	0.00311137639197449\\
-16592.8458063728	0.00323649167421322\\
-16586.9857160298	0.00332258152877746\\
-16581.1256256867	0.00336759746885296\\
-16575.2655353437	0.00337045982670993\\
-16569.4054450006	0.00333108363872912\\
-16563.5453546575	0.00325038105919412\\
-16557.6852643145	0.00313024023616807\\
-16551.8251739714	0.00297348113850494\\
-16545.9650836284	0.00278378936732268\\
-16540.1049932853	0.00256562950545911\\
-16534.2449029423	0.00232414004163137\\
-16528.3848125992	0.00206501234264672\\
-16522.5247222561	0.00179435652390311\\
-16516.6646319131	0.0015185573789269\\
-16510.80454157	0.0012441237653532\\
-16504.944451227	0.000977534999517035\\
-16499.0843608839	0.000725087884969616\\
-16493.2242705408	0.000492747986061979\\
-16487.3641801978	0.000286008659417088\\
-16481.5040898547	0.000109761174163993\\
-16475.6439995117	-3.18210083328037e-05\\
-16469.7839091686	-0.000135381057646192\\
-16463.9238188255	-0.000198458360456623\\
-16458.0637284825	-0.000219546982932227\\
-16452.2036381394	-0.000198131620406222\\
-16446.3435477964	-0.000134700176321897\\
-16440.4834574533	-3.07326637172032e-05\\
-16434.6233671102	0.000111333318655286\\
-16428.7632767672	0.000288159730377807\\
-16422.9031864241	0.000495586514094176\\
-16417.0430960811	0.000728729458506226\\
-16411.183005738	0.000982095197569767\\
-16405.3229153949	0.00124971063835898\\
-16399.4628250519	0.0015252637683092\\
-16393.6027347088	0.00180225252264397\\
-16387.7426443658	0.00207413820114466\\
-16381.8825540227	0.00233449981442621\\
-16376.0224636796	0.00257718571678646\\
-16370.1623733366	0.00279645894443526\\
-16364.3022829935	0.00298713282620357\\
-16358.4421926505	0.00314469366049456\\
-16352.5821023074	0.00326540755775076\\
-16346.7220119643	0.0033464089189049\\
-16340.8619216213	0.00338576845392555\\
-16335.0018312782	0.00338253912540536\\
-16329.1417409352	0.00333677892398197\\
-16323.2816505921	0.00324954992734402\\
-16317.4215602491	0.00312289365547601\\
-16311.561469906	0.00295978329357303\\
-16305.7013795629	0.0027640539006836\\
-16299.8412892199	0.00254031224206561\\
-16293.9811988768	0.00229382836536761\\
-16288.1211085338	0.0020304114721701\\
-16282.2610181907	0.00175627300844687\\
-16276.4009278476	0.00147788020042007\\
-16270.5408375046	0.00120180348947253\\
-16264.6807471615	0.000934561464678312\\
-16258.8206568185	0.000682466952817851\\
-16252.9605664754	0.000451477899609607\\
-16247.1004761323	0.000247056564347153\\
-16241.2403857893	7.40403555731246e-05\\
-16235.3802954462	-6.34726380712965e-05\\
-16229.5202051032	-0.000162220713228812\\
-16223.6601147601	-0.000219855914514827\\
-16217.800024417	-0.000234999859707407\\
-16211.939934074	-0.000207276700249463\\
-16206.0798437309	-0.000137322429085251\\
-16200.2197533879	-2.67703058012217e-05\\
-16194.3596630448	0.000121787267190864\\
-16188.4995727017	0.000304859527714513\\
-16182.6394823587	0.000518139371084741\\
-16176.7793920156	0.000756604674649038\\
-16170.9193016726	0.0010146365460701\\
-16165.0592113295	0.00128615171126053\\
-16159.1991209864	0.00156474592294067\\
-16153.3390306434	0.00184384500913134\\
-16147.4789403003	0.00211685999924417\\
-16141.6188499573	0.00237734266739566\\
-16135.7587596142	0.00261913782143336\\
-16129.8986692712	0.00283652874096806\\
-16124.0385789281	0.00302437232816796\\
-16118.178488585	0.0031782207765661\\
-16112.318398242	0.00329442687901623\\
-16106.4583078989	0.00337023048262889\\
-16100.5982175559	0.0034038240405048\\
-16094.7381272128	0.00339439570411013\\
-16088.8780368697	0.00334214892908698\\
-16083.0179465267	0.00324829812088709\\
-16077.1578561836	0.00311504041315398\\
-16071.2977658406	0.00294550423353621\\
-16065.4376754975	0.00274367586196693\\
-16059.5775851544	0.00251430570430218\\
-16053.7174948114	0.00226279648579603\\
-16047.8574044683	0.00199507599508508\\
-16041.9973141253	0.00171745737655641\\
-16036.1372237822	0.00143649026307191\\
-16030.2771334391	0.00115880625959995\\
-16024.4170430961	0.000890962422844709\\
-16018.556952753	0.000639286430811113\\
-16012.69686241	0.000409727098285845\\
-16006.8367720669	0.000207713769378205\\
-16000.9766817238	3.80279104543498e-05\\
-15995.1165913808	-9.53100597599009e-05\\
-15989.2565010377	-0.000189136024533497\\
-15983.3964106947	-0.000241217251211042\\
-15977.5363203516	-0.000250305519499976\\
-15971.6762300085	-0.000216167031927556\\
-15965.8161396655	-0.000139588389904397\\
-15959.9560493224	-2.23584835662776e-05\\
-15954.0959589794	0.000132773287949975\\
-15948.2358686363	0.000322161490170081\\
-15942.3757782932	0.000541350309089475\\
-15936.5156879502	0.000785178376656384\\
-15930.6555976071	0.00104790029991056\\
-15924.7955072641	0.00132332203250974\\
-15918.935416921	0.0016049468990927\\
-15913.0753265779	0.00188612883015765\\
-15907.2152362349	0.00216022919334625\\
-15901.3551458918	0.00242077352038266\\
-15895.4950555488	0.00266160442981069\\
-15889.6349652057	0.00287702713371492\\
-15883.7748748627	0.0030619440901945\\
-15877.9147845196	0.00321197561753751\\
-15872.0546941765	0.00332356361703518\\
-15866.1946038335	0.00339405594718373\\
-15860.3345134904	0.0034217694496319\\
-15854.4744231474	0.00340603012836392\\
-15848.6143328043	0.00334718952380148\\
-15842.7542424612	0.00324661688391024\\
-15836.8941521182	0.00310666730637738\\
-15831.0340617751	0.00293062659267823\\
-15825.1739714321	0.00272263410518938\\
-15819.313881089	0.00248758543855506\\
-15813.4537907459	0.00223101719329131\\
-15807.5937004029	0.00195897656392919\\
-15801.7336100598	0.00167787881306794\\
-15795.8735197168	0.00139435599053467\\
-15790.0134293737	0.00111510046472146\\
-15784.1533390306	0.000846706957782417\\
-15778.2932486876	0.000595516812678381\\
-15772.4331583445	0.000367468169909713\\
-15766.5730680015	0.000167955593274787\\
-15760.7129776584	1.70246302675327e-06\\
-15754.8528873153	-0.000127350845271362\\
-15748.9927969723	-0.00021614034906893\\
-15743.1327066292	-0.000262551201128199\\
-15737.2726162862	-0.000265468058943033\\
-15731.4125259431	-0.000224801881802328\\
-15725.5524356	-0.000141492511864545\\
-15719.692345257	-1.74869670384477e-05\\
-15713.8322549139	0.000144306053242435\\
-15707.9721645709	0.000340084386486619\\
-15702.1120742278	0.000565241750036063\\
-15696.2519838848	0.000814476109168386\\
-15690.3918935417	0.00108191452100489\\
-15684.5318031986	0.00136125151483173\\
-15678.6717128556	0.00164589774830985\\
-15672.8116225125	0.00192913543522503\\
-15666.9515321695	0.00220427687848571\\
-15661.0914418264	0.0024648223675902\\
-15655.2313514833	0.00270461371255668\\
-15649.3712611403	0.00291797978783883\\
-15643.5111707972	0.00309987064662115\\
-15637.6510804542	0.00324597703373892\\
-15631.7909901111	0.00335283246943439\\
-15625.930899768	0.00341789548531318\\
-15620.070809425	0.00343961006209267\\
-15614.2107190819	0.00341744283179781\\
-15608.3506287389	0.00335189615523155\\
-15602.4905383958	0.0032444967545569\\
-15596.6304480527	0.0030977601577299\\
-15590.7703577097	0.00291513178258332\\
-15584.9102673666	0.00270090604026635\\
-15579.0501770236	0.00246012535751445\\
-15573.1900866805	0.00219846149160917\\
-15567.3299963374	0.00192208193174272\\
-15561.4699059944	0.00163750453345362\\
-15555.6098156513	0.00135144381209565\\
-15549.7497253083	0.00107065252003196\\
-15543.8896349652	0.000801762245226182\\
-15538.0295446221	0.000551126793784171\\
-15532.1694542791	0.000324672055058073\\
-15526.309363936	0.000127755896776441\\
-15520.449273593	-3.49585975311358e-05\\
-15514.5891832499	-0.000159613549862391\\
-15508.7290929068	-0.000243247753188978\\
-15502.8690025638	-0.000283867012633818\\
-15497.0089122207	-0.000280491691646587\\
-15491.1488218777	-0.000233180329688468\\
-15485.2887315346	-0.000143028762349835\\
-15479.4286411916	-1.21447528169376e-05\\
-15473.5685508485	0.000156401278303213\\
-15467.7084605054	0.000358648273218747\\
-15461.8483701624	0.000589837618094773\\
-15455.9882798193	0.000844525097901928\\
-15450.1281894763	0.00111670909117421\\
-15444.2680991332	0.00139997198631395\\
-15438.4080087901	0.00168763148727556\\
-15432.5479184471	0.00197289824158901\\
-15426.687828104	0.00224903607266765\\
-15420.827737761	0.0025095210349352\\
-15414.9676474179	0.0027481955362655\\
-15409.1075570748	0.00295941388710672\\
-15403.2474667318	0.00313817583574399\\
-15397.3873763887	0.00328024493180674\\
-15391.5272860457	0.00338224891582324\\
-15385.6671957026	0.00344175975666725\\
-15379.8071053595	0.00345735143676508\\
-15373.9470150165	0.00342863411070367\\
-15368.0869246734	0.00335626381843553\\
-15362.2268343304	0.00324192751253394\\
-15356.3667439873	0.00308830374023165\\
-15350.5066536442	0.00289899989653999\\
-15344.6465633012	0.00267846751867429\\
-15338.7864729581	0.00243189760977451\\
-15332.9263826151	0.00216509845392739\\
-15327.066292272	0.00188435879806841\\
-15321.2062019289	0.00159629962321549\\
-15315.3461115859	0.00130771799906446\\
-15309.4860212428	0.00102542670394675\\
-15303.6259308998	0.000756093394389925\\
-15297.7658405567	0.000506083120695707\\
-15291.9057502136	0.000281307908173864\\
-15286.0456598706	8.7086958430274e-05\\
-15280.1855695275	-7.19792236303063e-05\\
-15274.3254791845	-0.000192117798626628\\
-15268.4653888414	-0.000270473076442667\\
-15262.6052984984	-0.000305174392639081\\
-15256.7452081553	-0.00029538076439031\\
-15250.8851178122	-0.000241301260081821\\
-15245.0250274692	-0.000144190589545165\\
-15239.1649371261	-6.32000663148055e-06\\
-15233.3048467831	0.000169075801332674\\
-15227.44475644	0.000377874595476299\\
-15221.5846660969	0.000615163458716024\\
-15215.7245757539	0.000875354384118947\\
-15209.8644854108	0.00115231585894787\\
-15204.0043950678	0.00143951734607849\\
-15198.1443047247	0.00173018325839858\\
-15192.2842143816	0.00201745279600168\\
-15186.4241240386	0.00229454187596013\\
-15180.5640336955	0.00255490333257139\\
-15174.7039433525	0.00279238160564078\\
-15168.8438530094	0.00300135826212091\\
-15162.9837626663	0.00317688491076186\\
-15157.1236723233	0.00331480036596573\\
-15151.2635819802	0.00341182928680324\\
-15145.4034916372	0.00346565995275887\\
-15139.5434012941	0.00347499933022859\\
-15133.683310951	0.00343960411759045\\
-15127.823220608	0.00336028702397862\\
-15121.9631302649	0.00323889812103693\\
-15116.1030399219	0.00307828169475268\\
-15110.2429495788	0.00288220960418514\\
-15104.3828592357	0.00265529270753584\\
-15098.5227688927	0.00240287243599037\\
-15092.6626785496	0.00213089506466031\\
-15086.8025882066	0.00184577163966613\\
-15080.9424978635	0.00155422686134492\\
-15075.0824075204	0.00126314048489069\\
-15069.2223171774	0.000979384979148323\\
-15063.3622268343	0.000709663273851925\\
-15057.5021364913	0.000460350426003838\\
-15051.6420461482	0.000237342945100516\\
-15045.7819558052	4.59193383597596e-05\\
-15039.9218654621	-0.000109384826114604\\
-15034.061775119	-0.000224884381242437\\
-15028.201684776	-0.000297832002020913\\
-15022.3415944329	-0.000326483551076439\\
-15016.4815040899	-0.0003101397746323\\
-15010.6214137468	-0.000249163352292922\\
-15004.7613234037	-0.000144970885391611\\
nan	nan\\
-14993.0411427176	0.000182347671641106\\
-14987.1810523746	0.000397786297766314\\
-14981.3209620315	0.00064124656956867\\
-14975.4608716884	0.000906994972593629\\
-14969.6007813454	0.00118876880093558\\
-14963.7406910023	0.0014799237351317\\
-14957.8806006593	0.00177359050624812\\
-14952.0205103162	0.00206283695243711\\
-14946.1604199731	0.00234083164499458\\
-14940.3003296301	0.00260100522197361\\
-14934.440239287	0.00283720561916546\\
-14928.580148944	0.00304384353095849\\
-14922.7200586009	0.00321602466163508\\
-14916.8599682578	0.00334966563817136\\
-14910.9998779148	0.00344159083945087\\
-14905.1397875717	0.00348960784607493\\
-14899.2796972287	0.00349255971797196\\
-14893.4196068856	0.00345035285387688\\
-14887.5595165425	0.00336395976184966\\
-14881.6994261995	0.00323539666232488\\
-14875.8393358564	0.00306767643896439\\
-14869.9792455134	0.00286473803519307\\
-14864.1191551703	0.00263135395085735\\
-14858.2590648272	0.00237301801082623\\
-14852.3989744842	0.00209581604510282\\
-14846.5388841411	0.0018062825240459\\
-14840.6787937981	0.00151124652591381\\
-14834.818703455	0.00121767066731926\\
-14828.958613112	0.00093248679559564\\
-14823.0985227689	0.000662432320091626\\
-14817.2384324258	0.000413891046680078\\
-14811.3783420828	0.000192742275460329\\
-14805.5182517397	4.22172853864268e-06\\
-14799.6581613967	-0.000147202402636761\\
-14793.7980710536	-0.000257935356010565\\
-14787.9379807105	-0.000325341134040087\\
-14782.0778903675	-0.000347805249640168\\
-14776.2178000244	-0.000324773389527042\\
-14770.3577096814	-0.000256765069308663\\
-14764.4976193383	-0.000145361944579666\\
-14758.6375289952	6.82895969455063e-06\\
-14752.7774386522	0.000196236246832203\\
-14746.9173483091	0.000418407946166413\\
-14741.0572579661	0.000668116144800087\\
-14735.197167623	0.00093947999445889\\
-14729.3370772799	0.00122610419944735\\
-14723.4769869369	0.00152122972457019\\
-14717.6168965938	0.00181789317173546\\
-14711.7568062508	0.00210909106740961\\
-14705.8967159077	0.00238794518504862\\
-14700.0366255646	0.00264786499999256\\
-14694.1765352216	0.00288270344049986\\
-14688.3164448785	0.00308690225354372\\
-14682.4563545355	0.00325562354887474\\
-14676.5962641924	0.00338486440746752\\
-14670.7361738493	0.00347155184018167\\
-14664.8760835063	0.00351361584370634\\
-14659.0159931632	0.00351003881731542\\
-14653.1559028202	0.00346088016106414\\
-14647.2958124771	0.00336727546096407\\
-14641.4357221341	0.00323141026656279\\
-14635.575631791	0.00305646906617973\\
-14629.7155414479	0.00284656065096442\\
-14623.8554511049	0.00260662161640171\\
-14617.9953607618	0.002342300268085\\
-14612.1352704188	0.00205982366122552\\
-14606.2751800757	0.00176585090394756\\
-14600.4150897326	0.00146731617983447\\
-14594.5549993896	0.00117126519018534\\
-14588.6949090465	0.00088468887352642\\
-14582.8348187035	0.000614358326495787\\
-14576.9747283604	0.000366664823968551\\
-14571.1146380173	0.000147468718030301\\
-14565.2545476743	-3.80392120851743e-05\\
-14559.3944573312	-0.000185460678968508\\
-14553.5343669882	-0.000291294164323109\\
-14547.6742766451	-0.000353018082501412\\
-14541.814186302	-0.000369150855255968\\
-14535.954095959	-0.000339286466673613\\
-14530.0940056159	-0.000264104645316144\\
-14524.2339152729	-0.000145355419131996\\
-14518.3738249298	1.4181606485196e-05\\
-14512.5137345867	0.000210762300133952\\
-14506.6536442437	0.00043976586314819\\
-14500.7935539006	0.000695803434123047\\
-14494.9334635576	0.000972844886872625\\
-14489.0733732145	0.00126436083839635\\
-14483.2132828714	0.0015634765229513\\
-14477.3531925284	0.0018631339060022\\
-14471.4931021853	0.00215625821548399\\
-14465.6330118423	0.00243592496150328\\
-14459.7729214992	0.00269552350139673\\
-14453.9128311561	0.00292891328709512\\
-14448.0527408131	0.00313056910176388\\
-14442.19265047	0.00329571185067635\\
-14436.332560127	0.00342042181080717\\
-14430.4724697839	0.00350173165535322\\
-14424.6123794408	0.0035376970457357\\
-14418.7522890978	0.00352744311202012\\
-14412.8921987547	0.0034711857108244\\
-14407.0321084117	0.00337022694442108\\
-14401.1720180686	0.00322692503256452\\
-14395.3119277256	0.00304463923383036\\
-14389.4518373825	0.00282765110239682\\
-14383.5917470394	0.00258106392615789\\
-14377.7316566964	0.00231068270777798\\
-14371.8715663533	0.00202287751120831\\
-14366.0114760103	0.00172443339015354\\
-14360.1513856672	0.00142239043416452\\
-14354.2912953241	0.00112387770259079\\
-14348.4312049811	0.000835944963785444\\
-14342.571114638	0.00056539621042337\\
-14336.711024295	0.000318628882536111\\
-14330.8509339519	0.000101482597034424\\
-14324.9908436088	-8.08990394676799e-05\\
-14319.1307532658	-0.000224190264832648\\
-14313.2706629227	-0.000324985756845948\\
-14307.4105725797	-0.000380881556884313\\
-14301.5504822366	-0.000390532398900328\\
-14295.6903918935	-0.000353684076786514\\
-14289.8303015505	-0.000271180071694326\\
-14283.9702112074	-0.000144942268043085\\
-14278.1101208644	2.20738013458808e-05\\
-14272.2500305213	0.000225948139063632\\
-14266.3899401782	0.000461888276574824\\
-14260.5298498352	0.000724341918675397\\
-14254.6697594921	0.00100712759141242\\
-14248.8096691491	0.00130358021957292\\
-14242.949578806	0.00160670820536757\\
-14237.0894884629	0.00190935830668996\\
-14231.2293981199	0.0022043844268812\\
-14225.3693077768	0.00248481633338656\\
-14219.5092174338	0.00274402432222415\\
-14213.6491270907	0.00297587593821246\\
-14207.7890367477	0.00317488104669863\\
-14201.9289464046	0.00333632182494585\\
-14196.0688560615	0.00345636459554653\\
-14190.2087657185	0.00353215085120681\\
-14184.3486753754	0.00356186530974095\\
-14178.4885850324	0.00354477937927039\\
-14172.6284946893	0.00348126899325907\\
-14166.7684043462	0.00337280637915283\\
-14160.9083140032	0.00322192594002392\\
-14155.0482236601	0.00303216503936423\\
-14149.1881333171	0.00280798107254935\\
-14143.328042974	0.00255464676876176\\
-14137.4679526309	0.00227812618235113\\
-14131.6078622879	0.00198493429025324\\
-14125.7477719448	0.00168198350024101\\
-14119.8876816018	0.0013764206862526\\
-14114.0275912587	0.00107545859238555\\
-14108.1675009156	0.000786205582809815\\
-14102.3074105726	0.000515497755686738\\
-14096.4473202295	0.000269737386289097\\
-14090.5872298865	5.47415169383387e-05\\
-14084.7271395434	-0.000124395725499868\\
-14078.8670492003	-0.000263423826120395\\
-14073.0069588573	-0.000359036732912303\\
-14067.1468685142	-0.000408951469478573\\
-14061.2867781712	-0.000411962640454991\\
-14055.4266878281	-0.000367971528561726\\
-14049.566597485	-0.000277989081314912\\
-14043.706507142	-0.000144112701387935\\
-14037.8464167989	3.05226269723719e-05\\
-14031.9863264559	0.000241817736857821\\
-14026.1262361128	0.000484805484690877\\
-14020.2661457697	0.000753767505555156\\
-14014.4060554267	0.00104236877370208\\
-14008.5459650836	0.00134380680201195\\
-14002.6858747406	0.00165097196672921\\
-13996.8257843975	0.00195661517922525\\
-13990.9656940544	0.0022535189504021\\
-13985.1056037114	0.00253466781147139\\
-13979.2455133683	0.0027934140670323\\
-13973.3854230253	0.00302363496483255\\
-13967.5253326822	0.00321987756595162\\
-13961.6652423392	0.00337748788853582\\
-13955.8051519961	0.00349272126603635\\
-13949.945061653	0.00356283130393929\\
-13944.08497131	0.00358613532134955\\
-13938.2248809669	0.00356205471872728\\
-13932.3647906239	0.00349112930359026\\
-13926.5047002808	0.00337500522006096\\
-13920.6446099377	0.00321639675151773\\
-13914.7845195947	0.00301902288247949\\
-13908.9244292516	0.00278752010166614\\
-13903.0643389086	0.00252733349123196\\
-13897.2042485655	0.00224458865951919\\
-13891.3441582224	0.00194594752989359\\
-13885.4840678794	0.00163845137996679\\
-13879.6239775363	0.00132935482955634\\
-13873.7638871933	0.00102595469092932\\
-13867.9037968502	0.000735417718974825\\
-13862.0437065071	0.000464611327181464\\
-13856.1836161641	0.000219941267876601\\
-13850.323525821	7.20011301820675e-06\\
-13844.463435478	-0.000168569880621201\\
-13838.6033451349	-0.000303196275538598\\
-13832.7432547918	-0.000393475494789959\\
-13826.8831644488	-0.000437249049686191\\
-13821.0230741057	-0.000433455140392547\\
-13815.1629837627	-0.000382154396046635\\
-13809.3028934196	-0.000284529130962334\\
-13803.4428030765	-0.000142856118191634\\
-13797.5827127335	3.9546492979655e-05\\
-13791.7226223904	0.000258396878264007\\
-13785.8625320474	0.000508550039020138\\
-13780.0024417043	0.00078411874361199\\
-13774.1423513612	0.0010786120667953\\
-13768.2822610182	0.00138508826722323\\
-13762.4221706751	0.0016963184025553\\
-13756.5620803321	0.00200495682623152\\
-13750.701989989	0.00230371454466459\\
-13744.841899646	0.00258553134405858\\
-13738.9818093029	0.00284374262236742\\
-13733.1217189598	0.00307223698416493\\
-13727.2616286168	0.00326560087258786\\
-13721.4015382737	0.0034192468150607\\
-13715.5414479307	0.00352952224543625\\
-13709.6813575876	0.00359379632114949\\
-13703.8212672445	0.00361052267204781\\
-13697.9611769015	0.00357927658441986\\
-13692.1010865584	0.0035007657267381\\
-13686.2409962154	0.00337681414709833\\
-13680.3809058723	0.00321031990360969\\
-13674.5208155292	0.00300518731189572\\
-13668.6607251862	0.00276623539278202\\
-13662.8006348431	0.002499084667487\\
-13656.9405445001	0.00221002495888296\\
-13651.080454157	0.00190586730827287\\
-13645.2203638139	0.00159378349377976\\
-13639.3602734709	0.00128113693133899\\
-13633.5001831278	0.000975308945195455\\
-13627.6400927848	0.000683524506567394\\
-13621.7800024417	0.000412681554150447\\
-13615.9199120986	0.000169187928476629\\
-13610.0598217556	-4.11902254466342e-05\\
-13604.1997314125	-0.000213465000840088\\
-13598.3396410695	-0.000343544984132847\\
-13592.4795507264	-0.000428332418795748\\
-13586.6194603833	-0.000465796970757441\\
-13580.7593700403	-0.00045502433916357\\
-13574.8992796972	-0.000396238548846174\\
-13569.0391893542	-0.00029079738163158\\
-13563.1790990111	-0.000141161037301188\\
-13557.3190086681	4.9165252847688e-05\\
-13551.458918325	0.000275713321537797\\
-13545.5988279819	0.000533156947610412\\
-13539.7387376389	0.000815437062982916\\
-13533.8786472958	0.00111590434142437\\
-13528.0185569528	0.00142747581368767\\
-13522.1584666097	0.00174280182061814\\
-13516.2983762666	0.00205443936895238\\
-13510.4382859236	0.00235502780130478\\
-13504.5781955805	0.00263746263442819\\
-13498.7181052375	0.00289506346023967\\
-13492.8580148944	0.00312173194208246\\
-13486.9979245513	0.00331209616943637\\
-13481.1378342083	0.0034616379546464\\
-13475.2777438652	0.00356680005502438\\
-13469.4176535222	0.00362507077624197\\
-13463.5575631791	0.00363504394500396\\
-13457.697472836	0.00359645281948817\\
-13451.837382493	0.00351017711966958\\
-13445.9772921499	0.00337822299508318\\
-13440.1172018069	0.00320367638486224\\
-13434.2571114638	0.00299063085426809\\
-13428.3970211207	0.00274409159504284\\
-13422.5369307777	0.00246985784069783\\
-13416.6768404346	0.00217438645866533\\
-13410.8167500916	0.0018646399278886\\
-13404.9566597485	0.00154792228059124\\
-13399.0965694054	0.00123170687408411\\
-13393.2364790624	0.000923460053034822\\
-13387.3763887193	0.000630464863146462\\
-13381.5162983763	0.000359648977902956\\
-13375.6562080332	0.000117420903896955\\
-13369.7961176901	-9.04816629876709e-05\\
-13363.9360273471	-0.000259127742386496\\
-13358.075937004	-0.000384510016431121\\
-13352.215846661	-0.000463640045495784\\
-13346.3557563179	-0.000494619490605733\\
-13340.4956659748	-0.000476685645360349\\
-13334.6355756318	-0.000410230185557357\\
-13328.7754852887	-0.000296790676461904\\
-13322.9153949457	-0.000139015020327091\\
-13317.0553046026	5.94003341502795e-05\\
-13311.1952142596	0.000293796978764833\\
-13305.3351239165	0.000558663901163979\\
-13299.4750335734	0.00084776704176819\\
-13293.6149432304	0.00115429600674359\\
-13287.7548528873	0.00147102448452132\\
-13281.8947625443	0.00179048058761845\\
-13276.0346722012	0.00210512310452079\\
-13270.1745818581	0.00240751950441071\\
-13264.3144915151	0.002690521493395\\
-13258.454401172	0.00294743397577846\\
-13252.594310829	0.00317217342681173\\
-13246.7342204859	0.00335941193135101\\
-13240.8741301428	0.00350470347765947\\
-13235.0140397998	0.00360458951335667\\
-13229.1539494567	0.00365668125752385\\
-13223.2938591137	0.00365971681014061\\
-13217.4337687706	0.00361359169415607\\
-13211.5736784275	0.00351936209084715\\
-13205.7135880845	0.00337922067474385\\
-13199.8534977414	0.0031964455997176\\
-13193.9934073984	0.002975323823287\\
-13188.1333170553	0.00272105056194462\\
-13182.2732267122	0.00243960723559769\\
-13176.4131363692	0.00213762076854656\\
-13170.5530460261	0.00182220755591233\\
-13164.6929556831	0.0015008057697849\\
-13158.83286534	0.00118099995578919\\
-13152.9727749969	0.000870342056633228\\
-13147.1126846539	0.000576173085354545\\
-13141.2525943108	0.00030544965918062\\
-13135.3925039678	6.45794925091613e-05\\
-13129.5324136247	-0.000140730260949095\\
-13123.6723232817	-0.000305608227740763\\
-13117.8122329386	-0.000426134392418144\\
-13111.9521425955	-0.000499433291546825\\
-13106.0920522525	-0.000523742608611341\\
-13100.2319619094	-0.000498455533806195\\
-13094.3718715664	-0.000424135870874628\\
-13088.5117812233	-0.000302505515995168\\
-13082.6516908802	-0.000136404585615859\\
-13076.7916005372	7.02748838355391e-05\\
-13070.9315101941	0.000312680116960369\\
-13065.0714198511	0.000585111525206799\\
-13059.211329508	0.000881156703204051\\
-13053.3512391649	0.00119384134555545\\
-13047.4911488219	0.00151579353264106\\
-13041.6310584788	0.00183941751565272\\
-13035.7709681358	0.00215707290440395\\
-13029.9108777927	0.00246125503114236\\
-13024.0507874496	0.00274477223260003\\
-13018.1906971066	0.00300091586296916\\
-13012.3306067635	0.00322361901866571\\
-13006.4705164205	0.00340760021986583\\
-13000.6104260774	0.00354848864624062\\
-12994.7503357343	0.00364292795769067\\
-12988.8902453913	0.00368865623383579\\
-12983.0301550482	0.0036845601292744\\
-12977.1700647052	0.00363070194760178\\
-12971.3099743621	0.00352831897704251\\
-12965.449884019	0.00337979508384935\\
-12959.589793676	0.00318860521565528\\
-12953.7297033329	0.00295923410579319\\
-12947.8696129899	0.00269707108066291\\
-12942.0095226468	0.00240828343697946\\
-12936.1494323037	0.00209967136398446\\
-12930.2893419607	0.00177850782262507\\
-12924.4292516176	0.00145236715229779\\
-12918.5691612746	0.00112894644337695\\
-12912.7090709315	0.000815883888288386\\
-12906.8489805884	0.00052057839741305\\
-12900.9888902454	0.000250014739631456\\
-12895.1287999023	1.05983396940968e-05\\
-12889.2687095593	-0.000191996360520902\\
-12883.4086192162	-0.000352960387325561\\
-12877.5485288732	-0.000468464380026777\\
-12871.6884385301	-0.000535749686200485\\
-12865.828348187	-0.000553194240633414\\
-12859.968257844	-0.000520351655020747\\
-12854.1081675009	-0.00043796257693283\\
-12848.2480771579	-0.000307938030400445\\
-12842.3879868148	-0.00013331511204746\\
-12836.5278964717	8.18139306052683e-05\\
-12830.6678061287	0.000332397582802178\\
-12824.8077157856	0.000612543661835639\\
-12818.9476254426	0.000915657847748411\\
-12813.0875350995	0.00123459888875648\\
-12807.2274447564	0.0015618468286643\\
-12801.3673544134	0.00188968029365477\\
-12795.5072640703	0.00221035865913803\\
-12789.6471737273	0.0025163047989812\\
-12783.7870833842	0.00280028410320546\\
-12777.9269930411	0.00305557553454391\\
-12772.0669026981	0.00327613068007132\\
-12766.206812355	0.00345671703384009\\
-12760.346722012	0.00359304211695406\\
-12754.4866316689	0.00368185549068407\\
-12748.6265413258	0.00372102623895638\\
-12742.7664509828	0.00370959407306639\\
-12736.9063606397	0.00364779283380925\\
-12731.0462702967	0.00353704581610459\\
-12725.1861799536	0.00337993300728251\\
-12719.3260896105	0.00318013099173745\\
-12713.4659992675	0.00294232692201863\\
-12707.6059089244	0.00267210856891523\\
-12701.7458185814	0.002375833029362\\
-12695.8857282383	0.00206047717666717\\
-12690.0256378953	0.00173347337133613\\
-12684.1655475522	0.00140253430026165\\
-12678.3054572091	0.00107547107263958\\
-12672.4453668661	0.000760008862021156\\
-12666.585276523	0.000463604445762696\\
-12660.72518618	0.000193269950974337\\
-12654.8650958369	-4.45930272548612e-05\\
-12649.0050054938	-0.000244345011436138\\
-12643.1449151508	-0.000401242341946952\\
-12637.2848248077	-0.000511549822465491\\
-12631.4247344647	-0.000572629635952271\\
-12625.5646441216	-0.000583004414815551\\
-12619.7045537785	-0.000542392957633614\\
-12613.8444634355	-0.000451717729446316\\
-12607.9843730924	-0.000313083948290628\\
-12602.1242827494	-0.000129730731236164\\
-12596.2641924063	9.40445667776325e-05\\
-12590.4041020632	0.000352987054251422\\
-12584.5440117202	0.000641007685132349\\
-12578.6839213771	0.000951326424858319\\
-12572.8238310341	0.00127663183456044\\
-12566.963740691	0.00160925331756385\\
-12561.1036503479	0.00194134197063009\\
-12555.2435600049	0.00226505577614567\\
-12549.3834696618	0.002572744766284\\
-12543.5233793188	0.00285713178704507\\
-12537.6632889757	0.00311148459138351\\
-12531.8031986326	0.00332977519202992\\
-12525.9431082896	0.00350682270205226\\
-12520.0830179465	0.00363841627921462\\
-12514.2229276035	0.00372141525634715\\
-12508.3628372604	0.00375382407735921\\
-12502.5027469173	0.00373484025126078\\
-12496.6426565743	0.00366487417201057\\
-12490.7825662312	0.00354554031603154\\
-12484.9224758882	0.00337962000402058\\
-12479.0623855451	0.00317099658537821\\
-12473.2022952021	0.00292456455585278\\
-12467.342204859	0.00264611473427888\\
-12461.4821145159	0.00234219819242049\\
-12455.6220241729	0.00201997213471457\\
-12449.7619338298	0.00168703135342849\\
-12443.9018434868	0.00135122922800426\\
-12438.0417531437	0.0010204924872384\\
-12432.1816628006	0.000702634103045424\\
-12426.3215724576	0.00040516873201525\\
-12420.4614821145	0.00013513506430627\\
-12414.6013917715	-0.000101070720897505\\
-12408.7413014284	-0.00029784645285955\\
-12402.8812110853	-0.000450516831741952\\
-12397.0211207423	-0.000555444505329712\\
-12391.1610303992	-0.000610116721304849\\
-12385.3009400562	-0.000613205491169877\\
-12379.4408497131	-0.00056459982558779\\
-12373.58075937	-0.000465409259387132\\
-12367.720669027	-0.00031793856165308\\
-12361.8605786839	-0.000125634206525346\\
-12356.0004883409	0.000106996152381589\\
-12350.1403979978	0.000374489322910161\\
-12344.2803076547	0.000670554855121089\\
-12338.4202173117	0.000988222950008418\\
-12332.5601269686	0.00132000851873012\\
-12326.7000366256	0.00165808753087376\\
-12320.8399462825	0.0019944814973436\\
-12314.9798559394	0.00232124573828104\\
-12309.1197655964	0.00263065699378987\\
-12303.2596752533	0.00291539594744337\\
-12297.3995849103	0.00316872034916426\\
-12291.5394945672	0.00338462464340911\\
-12285.6794042241	0.00355798232289191\\
-12279.8193138811	0.00368466763461515\\
-12273.959223538	0.00376165374866157\\
-12268.099133195	0.00378708505443428\\
-12262.2390428519	0.00376032185772527\\
-12256.3789525089	0.00368195640269436\\
-12250.5188621658	0.00355379981892677\\
-12244.6587718227	0.00337884027913275\\
-12238.7986814797	0.00316117333453156\\
-12232.9385911366	0.00290590605113968\\
-12227.0785007936	0.00261903719041244\\
-12221.2184104505	0.00230731624555055\\
-12215.3583201074	0.00197808464517036\\
-12209.4982297644	0.00163910286021765\\
-12203.6381394213	0.00129836748561333\\
-12197.7780490783	0.000963922607587565\\
-12191.9179587352	0.000643669906154229\\
-12186.0578683921	0.000345181975279903\\
-12180.1977780491	7.55232708467938e-05\\
-12174.337687706	-0.000158917080374117\\
-12168.477597363	-0.000352576654140911\\
-12162.6175070199	-0.000500851698535456\\
-12156.7574166768	-0.000600206569420244\\
-12150.8973263338	-0.000648258030506057\\
-12145.0372359907	-0.000643832408403683\\
-12139.1771456477	-0.000586994232414466\\
-12133.3170553046	-0.000479045661057906\\
-12127.4569649615	-0.000322496686323777\\
-12121.5968746185	-0.000121006796832772\\
-12115.7367842754	0.0001207005447353\\
-12109.8766939324	0.000396948611615346\\
-12104.0166035893	0.000701240715803483\\
-12098.1565132462	0.00102641297382349\\
-12092.2964229032	0.00136480294354482\\
-12086.4363325601	0.00170843016272789\\
-12080.5762422171	0.00204918433577047\\
-12074.716151874	0.00237901673159098\\
-12068.8560615309	0.0026901302757762\\
-12062.9959711879	0.00297516384823449\\
-12057.1358808448	0.00322736643019985\\
-12051.2757905018	0.00344075698087566\\
-12045.4157001587	0.00361026625855431\\
-12039.5556098157	0.00373185722264122\\
-12033.6955194726	0.00380262115837028\\
-12027.8354291295	0.0038208472346446\\
-12021.9753387865	0.00378606383305424\\
-12016.1152484434	0.00369905064929623\\
-12010.2551581004	0.00356182125941683\\
-12004.3950677573	0.00337757653881428\\
-11998.5349774142	0.0031506300112117\\
-11992.6748870712	0.00288630686836549\\
-11986.8147967281	0.00259081902417382\\
-11980.9547063851	0.00227111913406627\\
-11975.094616042	0.00193473701038974\\
-11969.2345256989	0.00158960228214142\\
-11963.3744353559	0.00124385747522906\\
-11957.5143450128	0.00090566591928914\\
-11951.6542546698	0.000583019012334543\\
-11945.7941643267	0.000283547393260629\\
-11939.9340739836	1.43404838178694e-05\\
-11934.0739836406	-0.000218221332815659\\
-11928.2138932975	-0.000408617924373255\\
-11922.3538029545	-0.000552320429514992\\
-11916.4937126114	-0.000645898976044536\\
-11910.6336222683	-0.000687104535664512\\
-11904.7735319253	-0.000674922962167231\\
-11898.9134415822	-0.000609599915083926\\
-11893.0533512392	-0.000492636057467425\\
-11887.1932608961	-0.000326752617424923\\
-11881.333170553	-0.000115828103150199\\
-11875.47308021	0.000135192357265997\\
-11869.6129898669	0.000420412932435065\\
-11863.7528995239	0.00073312554389767\\
-11857.8928091808	0.00106596761078129\\
-11852.0327188377	0.0014110953738765\\
-11846.1726284947	0.00176036871911771\\
-11840.3125381516	0.00210554314583713\\
-11834.4524478086	0.00243846435305652\\
-11828.5923574655	0.00275126085144999\\
-11822.7322671225	0.00303653005164627\\
-11816.8721767794	0.00328751343019484\\
-11811.0120864363	0.00349825662833911\\
-11805.1519960933	0.00366375069161708\\
-11799.2919057502	0.00378005110029099\\
-11793.4318154072	0.00384437176269776\\
-11787.5717250641	0.00385515173155192\\
-11781.711634721	0.00381209304655986\\
-11775.851544378	0.00371616878650527\\
-11769.9914540349	0.00356960111660585\\
-11764.1313636919	0.00337580982533867\\
-11758.2712733488	0.0031393325419885\\
-11752.4111830057	0.00286571849612794\\
-11746.5510926627	0.00256139830547651\\
-11740.6910023196	0.00223353284780547\\
-11734.8309119766	0.0018898447679716\\
-11728.9708216335	0.00153843658425448\\
-11723.1107312904	0.00118759967797026\\
-11717.2506409474	0.000845618668812081\\
-11711.3905506043	0.000520575791086038\\
-11705.5304602613	0.000220159889687517\\
-11699.6703699182	-4.85154505816385e-05\\
-11693.8102795751	-0.000279080340854924\\
-11687.9501892321	-0.000466059601298056\\
-11682.090098889	-0.00060500277171346\\
-11676.230008546	-0.000692590032835402\\
-11670.3699182029	-0.000726711517823789\\
-11664.5098278598	-0.000706518119476864\\
-11658.6497375168	-0.000632442570444803\\
-11652.7896471737	-0.000506190274248848\\
-11646.9295568307	-0.000330700078967078\\
-11641.0694664876	-0.000110075895037949\\
-11635.2093761446	0.000150509252026209\\
-11629.3492858015	0.000444934491304871\\
-11623.4891954584	0.000766274855940216\\
-11617.6291051154	0.0011069641368053\\
-11611.7690147723	0.00145897301096747\\
-11605.9089244293	0.00181399825166154\\
-11600.0488340862	0.00216365856146674\\
-11594.1887437431	0.00249969241031521\\
-11588.3286534001	0.00281415320880766\\
-11582.468563057	0.00309959720636045\\
-11576.608472714	0.00334925967178131\\
-11570.7483823709	0.00355721518658443\\
-11564.8882920278	0.00371851825338741\\
-11559.0282016848	0.00382932088312443\\
-11553.1681113417	0.00388696436550349\\
-11547.3080209987	0.00389004303458434\\
-11541.4479306556	0.00383843850080378\\
-11535.5878403125	0.00373332351644226\\
-11529.7277499695	0.00357713535795049\\
-11523.8676596264	0.00337351932864017\\
-11518.0075692834	0.00312724369047676\\
-11512.1474789403	0.00284408800978299\\
-11506.2873885972	0.00253070753122748\\
-11500.4272982542	0.00219447676199529\\
-11494.5672079111	0.00184331594227765\\
-11488.7071175681	0.00148550448483533\\
-11482.847027225	0.00112948577771722\\
-11476.9869368819	0.000783667951793509\\
-11471.1268465389	0.000456225313723368\\
-11465.2667561958	0.000154905133447802\\
-11459.4066658528	-0.00011315564734382\\
-11453.5465755097	-0.0003415994497413\\
-11447.6864851666	-0.000524998831970757\\
-11441.8263948236	-0.000658985428308061\\
-11435.9663044805	-0.000740353989706235\\
-11430.1062141375	-0.000767139048648679\\
-11424.2461237944	-0.000738662375043392\\
-11418.3860334513	-0.00065555007787772\\
-11412.5259431083	-0.000519718923417286\\
-11406.6658527652	-0.000334332166770206\\
-11400.8057624222	-0.000103725914057236\\
-11394.9456720791	0.000166692271139188\\
-11389.0855817361	0.000470570146491156\\
-11383.225491393	0.000800759982938834\\
-11377.3654010499	0.00114948666634536\\
-11371.5053107069	0.00150853075579784\\
-11365.6452203638	0.00186942218896967\\
-11359.7851300208	0.00222364006997352\\
-11353.9250396777	0.0025628138275367\\
-11348.0649493346	0.00287892099522269\\
-11342.2048589916	0.00316447694066062\\
-11336.3447686485	0.00341271205754877\\
-11330.4846783055	0.00361773222557966\\
-11324.6245879624	0.00377465873547897\\
-11318.7644976193	0.0038797443577932\\
-11312.9044072763	0.00393046279449555\\
-11307.0443169332	0.00392556937828703\\
-11301.1842265902	0.00386513156220703\\
-11295.3241362471	0.00375052845314002\\
-11289.464045904	0.00358441937418049\\
-11283.603955561	0.0033706821712818\\
-11277.7438652179	0.00311432269489252\\
-11271.8837748749	0.00282135756896189\\
-11266.0236845318	0.00249867299264071\\
-11260.1635941887	0.00215386288732921\\
-11254.3035038457	0.00179505019347766\\
-11248.4434135026	0.00143069552166349\\
-11242.5833231596	0.00106939766507679\\
-11236.7232328165	0.000719690676834046\\
-11230.8631424734	0.000389842300133102\\
-11225.0030521304	8.76585120602249e-05\\
-11219.1429617873	-0.000179701198804234\\
-11213.2828714443	-0.000405893449987297\\
-11207.4227811012	-0.000585541459919407\\
-11201.5626907581	-0.000714362849918624\\
-11195.7026004151	-0.000789271715960848\\
-11189.842510072	-0.000808452537875103\\
-11183.982419729	-0.000771404156191048\\
-11178.1223293859	-0.000678952752354025\\
-11172.2622390429	-0.00053323349861615\\
-11166.4021486998	-0.00033764128368209\\
-11160.5420583567	-9.6751650451197e-05\\
-11154.6819680137	0.000183786213358322\\
-11148.8218776706	0.000497381929508431\\
-11142.9617873276	0.000836658723243469\\
-11137.1016969845	0.00119362692170963\\
-11131.2416066414	0.00155987207647026\\
-11125.3815162984	0.00192675328124091\\
-11119.5214259553	0.00228560701108759\\
-11113.6613356123	0.00262795167483101\\
-11107.8012452692	0.00294568805187241\\
-11101.9411549261	0.00323129087659242\\
-11096.0810645831	0.00347798703931719\\
-11090.22097424	0.00367991618436731\\
-11084.360883897	0.0038322698974219\\
-11078.5007935539	0.00393140617704841\\
-11072.6407032108	0.00397493646565578\\
-11066.7806128678	0.0039617831605952\\
-11060.9205225247	0.00389220622127304\\
-11055.0604321817	0.00376779821679077\\
-11049.2003418386	0.00359144790304411\\
-11043.3402514955	0.00336727316123264\\
-11037.4801611525	0.0031005248539906\\
-11031.6200708094	0.00279746384381757\\
-11025.7599804664	0.00246521405312791\\
-11019.8998901233	0.00211159501458328\\
-11014.0397997802	0.00174493784698165\\
-11008.1797094372	0.00137388898730985\\
-11002.3196190941	0.00100720630188527\\
-10996.4595287511	0.000653552384313878\\
-10990.599438408	0.000321289918300479\\
-10984.7393480649	1.82839388482019e-05\\
-10978.8792577219	-0.000248284332806725\\
-10973.0191673788	-0.000472087674542846\\
-10967.1590770358	-0.00064780303624405\\
-10961.2989866927	-0.000771238136418995\\
-10955.4388963497	-0.000839431472165179\\
-10949.5788060066	-0.000850723357193363\\
-10943.7187156635	-0.000804796284483161\\
-10937.8586253205	-0.000702683632908234\\
-10931.9985349774	-0.000546746483764736\\
-10926.1384446344	-0.000340619065828483\\
-10920.2783542913	-8.91240887661348e-05\\
-10914.4182639482	0.000201840063112207\\
-10908.5581736052	0.000525437638647681\\
-10902.6980832621	0.000874056086533502\\
-10896.8379929191	0.00123948510972467\\
-10890.977902576	0.00161310999657833\\
-10885.1178122329	0.0019861146766046\\
-10879.2577218899	0.00234968971526866\\
-10873.3976315468	0.00269524034099343\\
-10867.5375412038	0.00301458959249164\\
-10861.6774508607	0.00330017178559698\\
-10855.8173605176	0.00354521172185579\\
-10849.9572701746	0.00374388539552924\\
-10844.0971798315	0.00389145838628969\\
-10838.2370894885	0.00398439865071868\\
-10832.3769991454	0.00402046102484192\\
-10826.5169088023	0.00399874141766316\\
-10820.6568184593	0.00391969938738359\\
-10814.7967281162	0.0037851485389755\\
-10808.9366377732	0.00359821494007103\\
-10803.0765474301	0.00336326450764631\\
-10797.216457087	0.00308580105198491\\
-10791.356366744	0.00277233735769858\\
-10785.4962764009	0.00243024232186667\\
-10779.6361860579	0.0020675677359539\\
-10773.7760957148	0.00169285878333361\\
-10767.9160053717	0.00131495271158743\\
-10762.0559150287	0.000942770422780636\\
-10756.1958246856	0.000585105895872639\\
-10750.3357343426	0.000250418411753181\\
-10744.4756439995	-5.3367510771315e-05\\
-10738.6155536565	-0.000319049736759018\\
-10732.7554633134	-0.000540319253106635\\
-10726.8953729703	-0.000711909975547922\\
-10721.0352826273	-0.000829724068009946\\
-10715.1751922842	-0.000890929792516086\\
-10709.3151019412	-0.000894029553685494\\
-10703.4550115981	-0.000838896503484505\\
-10697.594921255	-0.000726778812581885\\
-10691.734830912	-0.000560271477435844\\
-10685.8747405689	-0.000343256298508176\\
-10680.0146502259	-8.08114172161412e-05\\
-10674.1545598828	0.000220907480782324\\
-10668.2944695397	0.000554811517343699\\
-10662.4343791967	0.000913045144044303\\
-10656.5742888536	0.00128717092367804\\
-10650.7141985106	0.00166836822485437\\
-10644.8541081675	0.00204764115135044\\
-10638.9940178244	0.00241603080492403\\
-10633.1339274814	0.00276482687384703\\
-10627.2738371383	0.00308577355024298\\
-10621.4137467953	0.00337126490894524\\
-10615.5536564522	0.00361452512434587\\
-10609.6935661091	0.00380976925589593\\
-10603.8334757661	0.00395234078702719\\
-10597.973385423	0.00403882264862621\\
-10592.11329508	0.00406711907964583\\
-10586.2532047369	0.0040365063650835\\
-10580.3931143938	0.00394765122377145\\
-10574.5330240508	0.00380259638053253\\
-10568.6729337077	0.00360471363352503\\
-10562.8128433647	0.00335862549278282\\
-10556.9527530216	0.00307009721166655\\
-10551.0926626786	0.00274590173169188\\
-10545.2325723355	0.0023936607045229\\
-10539.3724819924	0.0020216653216966\\
-10533.5123916494	0.00163868116430806\\
-10527.6523013063	0.00125374166453445\\
-10521.7922109633	0.000875935045694689\\
-10515.9321206202	0.000514189765212761\\
-10510.0720302771	0.000177063525259744\\
-10504.2119399341	-0.000127459163385799\\
-10498.351849591	-0.000392156076208997\\
-10492.491759248	-0.000610738550696399\\
-10486.6316689049	-0.000778000881692425\\
-10480.7715785618	-0.000889944287736941\\
-10474.9114882188	-0.000943872496941892\\
-10469.0513978757	-0.00093845666803155\\
-10463.1913075327	-0.000873768084241219\\
-10457.3312171896	-0.00075127781698924\\
-10451.4711268465	-0.000573823335669707\\
-10445.6110365035	-0.000345542819937063\\
-10439.7509461604	-7.17786946231689e-05\\
-10433.8908558174	0.000241047364741797\\
-10428.0307654743	0.000585585031828312\\
-10422.1706751312	0.000953728003514114\\
-10416.3105847882	0.00133680469212515\\
-10410.4504944451	0.00172578245042778\\
-10404.5904041021	0.00211148052044146\\
-10398.730313759	0.00248478668600041\\
-10392.8702234159	0.00283687251706336\\
-10387.0101330729	0.00315940212184735\\
-10381.1500427298	0.00344472947134819\\
-10375.2899523868	0.00368607962577793\\
-10369.4298620437	0.0038777095678259\\
-10363.5697717006	0.00401504482546074\\
-10357.7096813576	0.00409478863462869\\
-10351.8495910145	0.00411500103698702\\
-10345.9895006715	0.00407514601630697\\
-10340.1294103284	0.00397610552871888\\
-10334.2693199853	0.00382016006354836\\
-10328.4092296423	0.00361093616065831\\
-10322.5491392992	0.00335332209138283\\
-10316.6890489562	0.00305335366302287\\
-10310.8289586131	0.00271807281358105\\
-10304.9688682701	0.00235536230955139\\
-10299.108777927	0.00197376042610718\\
-10293.2486875839	0.00158225996608609\\
-10287.3885972409	0.00119009634824351\\
-10281.5285068978	0.000806529757394639\\
-10275.6684165548	0.000440626495162082\\
-10269.8083262117	0.000101044693077294\\
-10263.9482358686	-0.000204170547546042\\
-10258.0881455256	-0.000467777742389184\\
-10252.2280551825	-0.000683510823126175\\
-10246.3679648395	-0.000846228073398956\\
-10240.5078744964	-0.000952034662309488\\
-10234.6477841533	-0.000998375855852838\\
-10228.7876938103	-0.000984098676312138\\
-10222.9276034672	-0.0009094805221939\\
-10217.0675131242	-0.000776224040201815\\
-10211.2074227811	-0.000587418336644094\\
-10205.347332438	-0.00034746741081545\\
-10199.487242095	-6.1987467481811e-05\\
-10193.6271517519	0.000262324497739244\\
-10187.7670614089	0.000617847765708381\\
-10181.9069710658	0.000996216930582045\\
-10176.0468807227	0.00138851870049675\\
-10170.1867903797	0.00178550183289332\\
-10164.3267000366	0.0021777952602686\\
-10158.4666096936	0.0025561292641822\\
-10152.6065193505	0.0029115544778712\\
-10146.7464290074	0.00323565354394054\\
-10140.8863386644	0.00352074042202343\\
-10135.0262483213	0.00376004262754251\\
-10129.1661579783	0.00394786208149846\\
-10123.3060676352	0.00407971075127527\\
-10117.4459772922	0.00415241785433444\\
-10111.5858869491	0.00416420606459005\\
-10105.725796606	0.00411473489116484\\
-10099.865706263	0.00400511017125079\\
-10094.0056159199	0.00383785941951596\\
-10088.1455255769	0.00361687358081613\\
-10082.2854352338	0.00334731652848656\\
-10076.4253448907	0.00303550441144205\\
-10070.5652545477	0.0026887576695916\\
-10064.7051642046	0.00231522918344578\\
-10058.8450738616	0.00192371259194056\\
-10052.9849835185	0.00152343528432678\\
-10047.1248931754	0.00112384093899665\\
-10041.2648028324	0.000734366733043322\\
-10035.4047124893	0.000364220478364466\\
-10029.5446221463	2.21629464914135e-05\\
-10023.6845318032	-0.00028369947249052\\
-10017.8244414601	-0.00054610687034786\\
-10011.9643511171	-0.000758818128777706\\
-10006.104260774	-0.000916759346183609\\
-10000.244170431	-0.00101614485376583\\
-9994.3840800879	-0.00105456793537535\\
-9988.52398974484	-0.00103105907773339\\
-9982.66389940178	-0.000946110342293003\\
-9976.80380905872	-0.000801665248406276\\
-9970.94371871567	-0.000601074371164269\\
-9965.08362837261	-0.000349017667190359\\
-9959.22353802955	-5.13953281799235e-05\\
-9953.36344768649	0.000284810292908342\\
-9947.50335734343	0.000651698452519214\\
-9941.64326700037	0.00104063564336161\\
-9935.78317665731	0.0014424587168023\\
-9929.92308631425	0.00184769072254565\\
-9924.06299597119	0.00224676438194026\\
-9918.20290562813	0.00263024792603116\\
-9912.34281528507	0.00298906796786863\\
-9906.48272494201	0.00331472414342831\\
-9900.62263459895	0.00359949044392328\\
-9894.76254425589	0.00383659847186479\\
-9888.90245391283	0.00402039827435218\\
-9883.04236356977	0.00414649293305666\\
-9877.18227322671	0.0042118437045222\\
-9871.32218288365	0.00421484319840697\\
-9865.46209254059	0.00415535483135437\\
-9859.60200219753	0.00403471759031168\\
-9853.74191185447	0.00385571595604088\\
-9847.88182151142	0.00362251566100787\\
-9842.02173116836	0.00334056676256645\\
-9836.1616408253	0.00301647628608772\\
-9830.30155048224	0.00265785341320656\\
-9824.44146013918	0.00227313084248649\\
-9818.58136979612	0.00187136651527814\\
-9812.72127945306	0.00146203036832846\\
-9806.86118911	0.00105478113286438\\
-9801.00109876694	0.000659238440118995\\
-9795.14100842388	0.000284755609934703\\
-9789.28091808082	-5.9801511773406e-05\\
-9783.42082773776	-0.000366264437741306\\
-9777.5607373947	-0.00062735567827998\\
-9771.70064705164	-0.000836861544688236\\
-9765.84055670858	-0.000989780014612611\\
-9759.98046636552	-0.00108244014129332\\
-9754.12037602246	-0.001112590156819\\
-9748.2602856794	-0.00107945215577781\\
-9742.40019533634	-0.000983742032508692\\
-9736.54010499328	-0.000827654164184469\\
-9730.68001465022	-0.000614811163982807\\
-9724.81992430716	-0.000350179853627983\\
-9718.95983396411	-3.99554034799944e-05\\
-9713.09974362105	0.000308583657897991\\
-9707.23965327799	0.000687246171984016\\
-9701.37956293493	0.00108712081242176\\
-9695.51947259187	0.00149878575938505\\
-9689.65938224881	0.00191253065364958\\
-9683.79929190575	0.00231858560203689\\
-9677.93920156269	0.00270735183443493\\
-9672.07911121963	0.0030696285673527\\
-9666.21902087657	0.00339683071277788\\
-9660.35893053351	0.00368119228174126\\
-9654.49884019045	0.00391595066399643\\
-9648.63874984739	0.00409550741163851\\
-9642.77865950433	0.00421556170466336\\
-9636.91856916128	0.00427321331770823\\
-9631.05847881822	0.00426703262295962\\
-9625.19838847516	0.0041970959411474\\
-9619.3382981321	0.00406498536853612\\
-9613.47820778904	0.00387375304460541\\
-9607.61811744598	0.0036278506670992\\
-9601.75802710292	0.00333302587986098\\
-9595.89793675986	0.00299618794527391\\
-9590.0378464168	0.00262524583909361\\
-9584.17775607374	0.00222892256176408\\
-9578.31766573068	0.00181655002431203\\
-9572.45757538762	0.0013978493320181\\
-9566.59748504456	0.000982701637709513\\
-9560.7373947015	0.000580914965836019\\
-9554.87730435844	0.000201992508947934\\
-9549.01721401538	-0.000145092130400868\\
-9543.15712367232	-0.000452107437206596\\
-9537.29703332927	-0.000711759189798616\\
-9531.43694298621	-0.000917863745331579\\
-9525.57685264315	-0.00106549528902926\\
-9519.71676230009	-0.00115110354139142\\
-9513.85667195703	-0.00117259911164546\\
-9507.99658161397	-0.0011294044460698\\
-9502.13649127091	-0.00102246913060506\\
-9496.27640092785	-0.000854249146834765\\
-9490.41631058479	-0.000628650531872853\\
-9484.55622024173	-0.000350938732998813\\
-9478.69612989867	-2.76157599881354e-05\\
-9472.83603955561	0.000333731999683831\\
-9466.97594921255	0.000724611741319333\\
-9461.11585886949	0.00113582380540982\\
-9455.25576852643	0.00155767815331197\\
-9449.39567818337	0.00198022266306069\\
-9443.53558784031	0.00239347786758938\\
-9437.67549749725	0.00278767259901116\\
-9431.81540715419	0.00315347497583804\\
-9425.95531681113	0.00348221327314416\\
-9420.09522646807	0.00376608144899409\\
-9414.23513612501	0.00399832445619156\\
-9408.37504578196	0.00417339894181847\\
-9402.5149554389	0.00428710551121622\\
-9396.65486509584	0.00433668940223972\\
-9390.79477475278	0.0043209071568294\\
-9384.93468440972	0.00424005767722932\\
-9379.07459406666	0.00409597689430078\\
-9373.2145037236	0.00389199613237378\\
-9367.35441338054	0.00363286511371622\\
-9361.49432303748	0.00332464138060481\\
-9355.63423269442	0.00297454870915472\\
-9349.77414235136	0.00259080782287016\\
-9343.9140520083	0.00218244337278847\\
-9338.05396166524	0.00175907171526883\\
-9332.19387132218	0.00133067447743953\\
-9326.33378097912	0.000907363241330561\\
-9320.47369063606	0.000499140893397119\\
-9314.613600293	0.000115665271268403\\
-9308.75350994994	-0.000233979308234936\\
-9302.89341960689	-0.000541497235881272\\
-9297.03332926383	-0.000799578414947775\\
-9291.17323892077	-0.00100207201605785\\
-9285.31314857771	-0.001144133052739\\
-9279.45305823465	-0.00122233828547741\\
-9273.59296789159	-0.00123476868250341\\
-9267.73287754853	-0.00118105645194892\\
-9261.87278720547	-0.00106239549452392\\
-9256.01269686241	-0.000881514988088023\\
-9250.15260651935	-0.000642616685997277\\
-9244.29251617629	-0.000351277368403233\\
-9238.43242583323	-1.43187103329364e-05\\
-9232.57233549017	0.000360352397762063\\
-9226.71224514711	0.000763929339960075\\
-9220.85215480405	0.00118691272472488\\
-9214.992064461	0.0016193339321104\\
-9209.13197411794	0.00205098999858324\\
-9203.27188377488	0.00247168430556852\\
-9197.41179343182	0.00287146739490105\\
-9191.55170308876	0.00324087222426609\\
-9185.6916127457	0.00357113830113054\\
-9179.83152240264	0.00385441938930239\\
-9173.97143205958	0.00408396986436188\\
-9168.11134171652	0.00425430529276466\\
-9162.25125137346	0.00436133341242939\\
-9156.3911610304	0.00440245238819283\\
-9150.53107068734	0.00437661398349029\\
-9144.67098034428	0.0042843501160307\\
-9138.81089000122	0.00412776212804796\\
-9132.95079965816	0.00391047298120032\\
-9127.0907093151	0.00363754346196718\\
-9121.23061897204	0.00331535433450035\\
-9115.37052862898	0.00295145718423983\\
-9109.51043828592	0.00255439743838982\\
-9103.65034794286	0.00213351370972789\\
-9097.79025759981	0.00169871817493735\\
-9091.93016725675	0.00126026315146303\\
-9086.07007691369	0.000828499369137581\\
-9080.20998657063	0.000413631634918769\\
-9074.34989622757	2.54776579995897e-05\\
-9068.48980588451	-0.000326764266075394\\
-9062.62971554145	-0.000634733215633085\\
-9056.76962519839	-0.000891104083604343\\
-9050.90953485533	-0.00108976178970174\\
-9045.04944451227	-0.00122594712163562\\
-9039.18935416921	-0.00129637072791189\\
-9033.32926382615	-0.0012992925329808\\
-9027.46917348309	-0.00123456465848646\\
-9021.60908314003	-0.00110363679407522\\
-9015.74899279697	-0.000909523846688235\\
-9009.88890245391	-0.000656736587672686\\
-9004.02881211085	-0.000351176891762859\\
nan	nan\\
-8992.30863142473	0.000388552979863155\\
-8986.44854108167	0.000805348415206443\\
-8980.58845073861	0.00124057479759203\\
-8974.72836039555	0.00168397365532782\\
-8968.8682700525	0.00212508129596976\\
-8963.00817970944	0.00255347568281955\\
-8957.14808936638	0.00295902262171462\\
-8951.28799902332	0.00333211544331347\\
-8945.42790868026	0.00366390251348384\\
-8939.5678183372	0.00394649718408987\\
-8933.70772799414	0.0041731652057437\\
-8927.84763765108	0.00433848514944135\\
-8921.98754730802	0.00443847801676287\\
-8916.12745696496	0.00447070294047248\\
-8910.2673666219	0.00443431667582854\\
-8904.40727627884	0.00433009543330857\\
-8898.54718593578	0.00416041849243725\\
-8892.68709559272	0.00392921393786518\\
-8886.82700524966	0.00364186775339929\\
-8880.96691490661	0.0033050983764149\\
-8875.10682456355	0.00292679963415491\\
-8869.24673422049	0.00251585573192684\\
-8863.38664387743	0.00208193262934456\\
-8857.52655353437	0.00163525070234832\\
-8851.66646319131	0.00118634403701513\\
-8845.80637284825	0.000745812023475895\\
-8839.94628250519	0.000324069106560004\\
-8834.08619216213	-6.89013992747874e-05\\
-8828.22610181907	-0.00042378361213721\\
-8822.36601147601	-0.000732149909718607\\
-8816.50592113295	-0.0009866610474207\\
-8810.64583078989	-0.00118124081662268\\
-8804.78574044683	-0.00131122108748717\\
-8798.92565010377	-0.00137345377493222\\
-8793.06555976071	-0.0013663870435062\\
-8787.20546941765	-0.00129010390772904\\
-8781.3453790746	-0.00114632227046114\\
-8775.48528873154	-0.000938356351398894\\
-8769.62519838848	-0.000671040369009993\\
-8763.76510804542	-0.000350616232252363\\
-8757.90501770236	1.54121500532384e-05\\
-8752.0449273593	0.000418454542535058\\
-8746.18483701624	0.000849035927562462\\
-8740.32474667318	0.00129701919275337\\
-8734.46465633012	0.00175184372891886\\
-8728.60456598706	0.00220277432339206\\
-8722.744475644	0.00263915448377542\\
-8716.88438530094	0.00305065821529802\\
-8711.02429495788	0.00342753430388004\\
-8705.16420461482	0.00376083732708757\\
-8699.30411427176	0.00404263992075257\\
-8693.4440239287	0.00426622126576131\\
-8687.58393358564	0.00442622731398859\\
-8681.72384324258	0.00451879893457501\\
-8675.86375289952	0.0045416649148009\\
-8670.00366255646	0.00449419757514533\\
-8664.1435722134	0.00437742963913729\\
-8658.28348187034	0.00419403191096226\\
-8652.42339152729	0.00394825223982305\\
-8646.56330118423	0.00364581716300942\\
-8640.70321084117	0.00329379850529249\\
-8634.84312049811	0.0029004480401003\\
-8628.98303015505	0.00247500407744168\\
-8623.12293981199	0.00202747451125621\\
-8617.26284946893	0.00156840141982746\\
-8611.40275912587	0.00110861275528065\\
-8605.54266878281	0.000658966970117599\\
-8599.68257843975	0.000230096602570953\\
-8593.82248809669	-0.000167843125083536\\
-8587.96239775363	-0.000525414751988675\\
-8582.10230741057	-0.000834122375845357\\
-8576.24221706751	-0.0010866134956365\\
-8570.38212672445	-0.00127685410565536\\
-8564.52203638139	-0.00140027287203389\\
-8558.66194603833	-0.00145387094796795\\
-8552.80185569527	-0.00143629479067399\\
-8546.94176535222	-0.00134787021387826\\
-8541.08167500916	-0.00119059682223355\\
-8535.2215846661	-0.000968102909160292\\
-8529.36149432304	-0.000685561832787853\\
-8523.50140397998	-0.000349571796148656\\
-8517.64131363692	3.19981760311385e-05\\
-8511.78122329386	0.000450192469494161\\
-8505.9211329508	0.000895179009245003\\
-8500.06104260774	0.00135648035571344\\
-8494.20095226468	0.00182322033744776\\
-8488.34086192162	0.00228438041581017\\
-8482.48077157856	0.00272905973982109\\
-8476.6206812355	0.00314673275453141\\
-8470.76059089244	0.00352749827598055\\
-8464.90050054939	0.00386231414000165\\
-8459.04041020633	0.00414321186451714\\
-8453.18031986327	0.00436348623070412\\
-8447.32022952021	0.00451785527312759\\
-8441.46013917715	0.00460258686299975\\
-8435.60004883409	0.00461558885176763\\
-8429.73995849103	0.00455646059946184\\
-8423.87986814797	0.00442650462099318\\
-8418.01977780491	0.0042286980239978\\
-8412.15968746185	0.00396762436058506\\
-8406.29959711879	0.00364936745141265\\
-8400.43950677573	0.00328136964013129\\
-8394.57941643267	0.00287225777954352\\
-8388.71932608961	0.00243164101740384\\
-8382.85923574655	0.00196988512036785\\
-8376.99914540349	0.00149786863542219\\
-8371.13905506043	0.00102672662364745\\
-8365.27896471737	0.000567588002048909\\
-8359.41887437431	0.000131312688125549\\
-8353.55878403125	-0.000271765245471788\\
-8347.69869368819	-0.000632082332671406\\
-8341.83860334514	-0.000941072596011324\\
-8335.97851300208	-0.00119137116750191\\
-8330.11842265902	-0.00137698980918325\\
-8324.25833231596	-0.0014934601513483\\
-8318.3982419729	-0.00153794122356749\\
-8312.53815162984	-0.00150928869192557\\
-8306.67806128678	-0.00140808411730959\\
-8300.81797094372	-0.00123662349096209\\
-8294.95788060066	-0.000998865264702988\\
-8289.0977902576	-0.000700339049596867\\
-8283.23769991454	-0.000348017087531137\\
-8277.37760957148	4.98485181683334e-05\\
-8271.51751922842	0.000483919014221943\\
-8265.65742888536	0.000943988128342759\\
-8259.7973385423	0.00141922197891081\\
-8253.93724819924	0.00189841412506393\\
-8248.07715785618	0.002370249753967\\
-8242.21706751312	0.00282357277835864\\
-8236.35697717006	0.00324764954019057\\
-8230.496886827	0.00363242288834444\\
-8224.63679648394	0.00396875061699897\\
-8218.77670614088	0.00424862261316535\\
-8212.91661579783	0.00446535155618922\\
-8207.05652545477	0.00461373263109226\\
-8201.19643511171	0.00469016844152466\\
-8195.33634476865	0.00469275612697605\\
-8189.47625442559	0.00462133457518301\\
-8183.61616408253	0.00447749056206788\\
-8177.75607373947	0.0042645236188018\\
-8171.89598339641	0.00398737040045396\\
-8166.03589305335	0.00365249028792505\\
-8160.17580271029	0.00326771487287228\\
-8154.31571236723	0.00284206483124809\\
-8148.45562202417	0.00238553846709102\\
-8142.59553168111	0.00190887688143375\\
-8136.73544133805	0.00142331128130615\\
-8130.87535099499	0.000940298371978892\\
-8125.01526065193	0.000471250065677629\\
-8119.15517030888	2.72638824057015e-05\\
-8113.29507996582	-0.000381139589259708\\
-8107.43498962276	-0.000744265961507545\\
-8101.5748992797	-0.00105347714186791\\
-8095.71480893664	-0.00130139679299883\\
-8089.85471859358	-0.00148208627178479\\
-8083.99462825052	-0.00159118685275138\\
-8078.13453790746	-0.00162602483077163\\
-8072.2744475644	-0.00158567697055843\\
-8066.41435722134	-0.00147099470286946\\
-8060.55426687828	-0.00128458643998159\\
-8054.69417653522	-0.001030758370533\\
-8048.83408619216	-0.000715415075374337\\
-8042.9739958491	-0.000345922256493886\\
-8037.11390550604	6.90652311472127e-05\\
-8031.25381516299	0.000519806030933051\\
-8025.39372481993	0.000995700875414336\\
-8019.53363447687	0.00148554175362422\\
-8013.67354413381	0.00197777579864329\\
-8007.81345379075	0.00246077768362429\\
-8001.95336344769	0.0029231241047975\\
-7996.09327310463	0.0033538638718244\\
-7990.23318276157	0.00374277722110838\\
-7984.37309241851	0.00408061821249624\\
-7978.51300207545	0.00435933446178328\\
-7972.65291173239	0.00457225898802852\\
-7966.79282138933	0.00471426960627653\\
-7960.93273104627	0.00478191205601213\\
-7955.07264070321	0.00477348390764992\\
-7949.21255036015	0.00468907721033001\\
-7943.35246001709	0.00453057881694765\\
-7937.49236967403	0.00430162831903393\\
-7931.63227933097	0.00400753452702635\\
-7925.77218898791	0.00365515240923434\\
-7919.91209864485	0.00325272334162619\\
-7914.05200830179	0.00280968238947026\\
-7908.19191795873	0.00233643712631276\\
-7902.33182761568	0.001844123173884\\
-7896.47173727262	0.00134434220401038\\
-7890.61164692956	0.000848888565200893\\
-7884.7515565865	0.000369470974443799\\
-7878.89146624344	-8.25641599395829e-05\\
-7873.03137590038	-0.000496501247896817\\
-7867.17128555732	-0.000862509506677961\\
-7861.31119521426	-0.0011718764103387\\
-7855.4511048712	-0.00141721505745585\\
-7849.59101452814	-0.0015926405225492\\
-7843.73092418508	-0.00169391098271643\\
-7837.87083384202	-0.00171853023641277\\
-7832.01074349896	-0.00166580913852832\\
-7826.1506531559	-0.00153688444269864\\
-7820.29056281284	-0.00133469454520824\\
-7814.43047246978	-0.00106391264226396\\
-7808.57038212672	-0.000730838818666499\\
-7802.71029178366	-0.000343253558008612\\
-7796.8502014406	8.97639209317597e-05\\
-7790.99011109754	0.000558048261289596\\
-7785.13002075449	0.00105058652195334\\
-7779.26993041143	0.00155577709121441\\
-7773.40984006837	0.0020617028745361\\
-7767.54974972531	0.00255641232483692\\
-7761.68965938225	0.00302820168944862\\
-7755.82956903919	0.00346589180955359\\
-7749.96947869613	0.0038590929269255\\
-7744.10938835307	0.00419845122673438\\
-7738.24929801001	0.00447587126749133\\
-7732.38920766695	0.00468470901040696\\
-7726.52911732389	0.00481993084626727\\
-7720.66902698083	0.00487823481579208\\
-7714.80893663777	0.00485813110556342\\
-7708.94884629471	0.00475997985931153\\
-7703.08875595166	0.00458598535043495\\
-7697.2286656086	0.00434014658960261\\
-7691.36857526554	0.00402816547112825\\
-7685.50848492248	0.00365731456558242\\
-7679.64839457942	0.00323626762295892\\
-7673.78830423636	0.00277489673474384\\
-7667.9282138933	0.00228404089697095\\
-7662.06812355024	0.0017752513990113\\
-7656.20803320718	0.00126052001710392\\
-7650.34794286412	0.000751996407223845\\
-7644.48785252106	0.000261701355847398\\
-7638.627762178	-0.000198757344909064\\
-7632.76767183494	-0.000618459683814279\\
-7626.90758149188	-0.000987432375046988\\
-7621.04749114882	-0.00129688582194969\\
-7615.18740080576	-0.00153942347109906\\
-7609.3273104627	-0.00170921857177884\\
-7603.46722011964	-0.00180215411821402\\
-7597.60712977658	-0.0018159226151982\\
-7591.74703943353	-0.00175008325178003\\
-7585.88694909047	-0.00160607506982482\\
-7580.02685874741	-0.00138718575129767\\
-7574.16676840435	-0.00109847669622303\\
-7568.30667806129	-0.000746666095729551\\
-7562.44658771823	-0.000339972699795591\\
-7556.58649737517	0.000112076088802196\\
-7550.72640703211	0.000598867314853788\\
-7544.86631668905	0.00110895154258507\\
-7539.00622634599	0.00163031205512012\\
-7533.14613600293	0.00215064785350499\\
-7527.28604565987	0.00265766379220044\\
-7521.42595531681	0.00313936100899797\\
-7515.56586497375	0.00358432079081758\\
-7509.70577463069	0.00398197516092144\\
-7503.84568428763	0.00432285777721249\\
-7497.98559394457	0.00459882918638366\\
-7492.12550360151	0.00480327107548022\\
-7486.26541325845	0.00493124488621574\\
-7480.40532291539	0.00497961099312772\\
-7474.54523257233	0.00494710557064382\\
-7468.68514222927	0.00483437327082675\\
-7462.82505188622	0.00464395487261931\\
-7456.96496154316	0.00438023012724407\\
-7451.1048712001	0.00404931708139347\\
-7445.24478085704	0.0036589301905883\\
-7439.38469051398	0.00321820051139937\\
-7433.52460017092	0.00273746216191537\\
-7427.66450982786	0.00222801004266241\\
-7421.8044194848	0.00170183449662479\\
-7415.94432914174	0.00117133914015726\\
-7410.08423879868	0.000649048503737084\\
-7404.22414845562	0.000147312371606068\\
-7398.36405811256	-0.000321986201267342\\
-7392.5039677695	-0.000747712293557661\\
-7386.64387742644	-0.0011197432806885\\
-7380.78378708338	-0.00142920949135318\\
-7374.92369674032	-0.00166870563765352\\
-7369.06360639727	-0.00183246798045045\\
-7363.20351605421	-0.00191651299321907\\
-7357.34342571115	-0.00191873419030468\\
-7351.48333536809	-0.0018389547684456\\
-7345.62324502503	-0.00167893475050007\\
-7339.76315468197	-0.00144233239245914\\
-7333.90306433891	-0.00113462069513754\\
-7328.04297399585	-0.000762960923031298\\
-7322.18288365279	-0.000336036051262834\\
-7316.32279330973	0.000136151987313901\\
-7310.46270296667	0.000642516522328329\\
-7304.60261262361	0.00117114635007881\\
-7298.74252228055	0.00170958581632914\\
-7292.88243193749	0.00224512819346433\\
-7287.02234159443	0.00276511544347561\\
-7281.16225125137	0.0032572372966923\\
-7275.30216090832	0.00370982258164542\\
-7269.44207056526	0.00411211591289026\\
-7263.5819802222	0.00445453317931075\\
-7257.72188987914	0.00472888976538742\\
-7251.86179953608	0.0049285960728339\\
-7246.00170919302	0.00504881567328607\\
-7240.14161884996	0.00508658229843938\\
-7234.2815285069	0.00504087284005571\\
-7228.42143816384	0.0049126345663638\\
-7222.56134782078	0.00470476584106593\\
-7216.70125747772	0.0044220507260332\\
-7210.84116713466	0.00407104894102709\\
-7204.9810767916	0.00365994370832216\\
-7199.12098644854	0.00319835101011038\\
-7193.26089610548	0.00269709470247947\\
-7187.40080576242	0.00216795274406222\\
-7181.54071541936	0.00162338048697094\\
-7175.6806250763	0.00107621752968296\\
-7169.82053473324	0.000539385029281543\\
-7163.96044439018	2.55806072418099e-05\\
-7158.10035404712	-0.000453021949212453\\
-7152.24026370407	-0.000885061089799094\\
-7146.38017336101	-0.00126025717732863\\
-7140.52008301795	-0.00156965703806855\\
-7134.65999267489	-0.00180584757024491\\
-7128.79990233183	-0.00196313331618809\\
-7122.93981198877	-0.00203767374653313\\
-7117.07972164571	-0.00202757694959323\\
-7111.21963130265	-0.00193294744323167\\
-7105.35954095959	-0.00175588690663076\\
-7099.49945061653	-0.00150044773905863\\
-7093.63936027347	-0.0011725404670072\\
-7087.77926993041	-0.000779797112215492\\
-7081.91917958735	-0.000331393677184145\\
-7076.05908924429	0.000162164125179634\\
-7070.19899890123	0.000689286895977135\\
-7064.33890855817	0.00123757356929216\\
-7058.47881821511	0.00179410304284405\\
-7052.61872787205	0.00234573856389811\\
-7046.75863752899	0.00287943770187893\\
-7040.89854718593	0.00338256059547219\\
-7035.03845684288	0.00384316919069466\\
-7029.17836649982	0.0042503103878975\\
-7023.31827615676	0.00459427638360516\\
-7017.4581858137	0.00486683602111729\\
-7011.59809547064	0.00506143163847697\\
-7005.73800512758	0.00517333670794662\\
-6999.87791478452	0.00519977047984608\\
-6994.01782444146	0.0051399668529639\\
-6988.1577340984	0.00499519576923716\\
-6982.29764375534	0.00476873655094517\\
-6976.43755341228	0.00446580372984344\\
-6970.57746306922	0.00409342704370376\\
-6964.71737272616	0.00366028835835199\\
-6958.8572823831	0.00317651929661852\\
-6952.99719204004	0.00265346428939738\\
-6947.13710169699	0.00210341458817649\\
-6941.27701135393	0.0015393194740748\\
-6935.41692101087	0.0009744814463379\\
-6929.55683066781	0.00042224256351902\\
-6923.69674032475	-0.000104330668892562\\
-6917.83664998169	-0.000592756424064951\\
-6911.97655963863	-0.00103143338130802\\
-6906.11646929557	-0.00140991624614854\\
-6900.25637895251	-0.00171916442158967\\
-6894.39628860945	-0.00195175791239929\\
-6888.53619826639	-0.00210207530847515\\
-6882.67610792333	-0.00216642957996259\\
-6876.81601758027	-0.00214315840660015\\
-6870.95592723721	-0.00203266683163263\\
-6865.09583689415	-0.00183742115323662\\
-6859.23574655109	-0.00156189411605015\\
-6853.37565620803	-0.0012124626159133\\
-6847.51556586497	-0.000797260254116888\\
-6841.65547552191	-0.00032598814825652\\
-6835.79538517886	0.000190311602124636\\
-6829.9352948358	0.000739514507128819\\
-6824.07520449274	0.00130869828155682\\
-6818.21511414968	0.00188444676516167\\
-6812.35502380662	0.00245316602226645\\
-6806.49493346356	0.00300140517511289\\
-6800.6348431205	0.00351617440088875\\
-6794.77475277744	0.00398525257689921\\
-6788.91466243438	0.00439747728922471\\
-6783.05457209132	0.00474301032587694\\
-6777.19448174826	0.00501357234097803\\
-6771.3343914052	0.00520264109569365\\
-6765.47430106214	0.00530560853042484\\
-6759.61421071908	0.00531989288959067\\
-6753.75412037602	0.00524500317214631\\
-6747.89403003296	0.00508255430570568\\
-6742.0339396899	0.00483623260068631\\
-6736.17384934684	0.00451171221529959\\
-6730.31375900378	0.00411652452141069\\
-6724.45366866072	0.00365988337488168\\
-6718.59357831766	0.00315247034271061\\
-6712.7334879746	0.00260618489054492\\
-6706.87339763155	0.00203386537113897\\
-6701.01330728849	0.0014489873534096\\
-6695.15321694543	0.000865346378826853\\
-6689.29312660237	0.000296732611241846\\
-6683.43303625931	-0.000243394948793349\\
-6677.57294591625	-0.000742226991104452\\
-6671.71285557319	-0.00118790762803002\\
-6665.85276523013	-0.00156981613176464\\
-6659.99267488707	-0.00187881998418719\\
-6654.13258454401	-0.0021074932120618\\
-6648.27249420095	-0.00225029479024497\\
-6642.41240385789	-0.00230370282935873\\
-6636.55231351483	-0.0022663013013571\\
-6630.69222317177	-0.00213881717160385\\
-6624.83213282871	-0.00192410697380834\\
-6618.97204248565	-0.0016270930563418\\
-6613.1119521426	-0.00125465091792605\\
-6607.25186179954	-0.000815450207793016\\
-6601.39177145648	-0.000319753064928165\\
-6595.53168111342	0.000220825515518639\\
-6589.67159077036	0.000793589695060228\\
-6583.8115004273	0.00138506081597554\\
-6577.95141008424	0.00198129444312221\\
-6572.09131974118	0.00256820896834099\\
-6566.23122939812	0.00313191803605909\\
-6560.37113905506	0.00365905894498741\\
-6554.511048712	0.00413710926157759\\
-6548.65095836894	0.00455468414590147\\
-6542.79086802588	0.00490180733291645\\
-6536.93077768282	0.00517014932109206\\
-6531.07068733976	0.00535322708461923\\
-6525.21059699671	0.00544656052279753\\
-6519.35050665365	0.005447781874524\\
-6513.49041631059	0.005356695429094\\
-6507.63032596753	0.00517528603443037\\
-6501.77023562447	0.00490767611017306\\
-6495.91014528141	0.00456003208852832\\
-6490.05005493835	0.00414042240254238\\
-6484.18996459529	0.00365863028905291\\
-6478.32987425223	0.00312592574678473\\
-6472.46978390917	0.00255480196276792\\
-6466.60969356611	0.00195868236896549\\
-6460.74960322305	0.00135160519540629\\
-6454.88951287999	0.000747892930735089\\
-6449.02942253693	0.000161814470056469\\
-6443.16933219387	-0.000392752082520558\\
-6437.30924185081	-0.000902647990680954\\
-6431.44915150775	-0.00135574606517833\\
-6425.58906116469	-0.00174123904191137\\
-6419.72897082163	-0.00204989730375426\\
-6413.86888047857	-0.0022742898028516\\
-6408.00879013551	-0.00240896289796897\\
-6402.14869979245	-0.00245057280581024\\
-6396.2886094494	-0.00239796845315791\\
-6390.42851910634	-0.0022522226819947\\
-6384.56842876328	-0.00201661097626424\\
-6378.70833842022	-0.00169653811643794\\
-6372.84824807716	-0.00129941439902993\\
-6366.9881577341	-0.000834484252292827\\
-6361.12806739104	-0.000312611208892719\\
-6355.26797704798	0.000253975765548383\\
-6349.40788670492	0.000851968667284953\\
-6343.54779636186	0.00146729286793934\\
-6337.6877060188	0.00208543821521955\\
-6331.82761567574	0.00269180103443843\\
-6325.96752533268	0.00327202897234065\\
-6320.10743498962	0.00381236054331838\\
-6314.24734464656	0.00429995134713795\\
-6308.3872543035	0.00472317922852225\\
-6302.52716396044	0.00507192113012198\\
-6296.66707361738	0.00533779504811387\\
-6290.80698327432	0.00551436130853336\\
-6284.94689293126	0.00559727833517958\\
-6279.08680258821	0.005584409144124\\
-6273.22671224515	0.00547587595770156\\
-6267.36662190209	0.00527406154989258\\
-6261.50653155903	0.00498355719262695\\
-6255.64644121597	0.00461105833162487\\
-6249.78635087291	0.00416521035809779\\
-6243.92626052985	0.00365640802571516\\
-6238.06617018679	0.00309655316352792\\
-6232.20607984373	0.00249877632929415\\
-6226.34598950067	0.0018771289102097\\
-6220.48589915761	0.00124625288840901\\
-6214.62580881455	0.000621036029345132\\
-6208.76571847149	1.62606105484162e-05\\
-6202.90562812843	-0.000553746023853492\\
-6197.04553778538	-0.00107545082509682\\
-6191.18544744232	-0.00153643628271601\\
-6185.32535709926	-0.00192569592569518\\
-6179.4652667562	-0.00223389705270557\\
-6173.60517641314	-0.00245360442278085\\
-6167.74508607008	-0.00257945954782799\\
-6161.88499572702	-0.0026083112671582\\
-6156.02490538396	-0.00253929442639755\\
-6150.1648150409	-0.00237385470245888\\
-6144.30472469784	-0.00211571888554609\\
-6138.44463435478	-0.00177081121470925\\
-6132.58454401172	-0.00134711764011586\\
-6126.72445366866	-0.000854501117769777\\
-6120.8643633256	-0.000304472205703852\\
-6115.00427298254	0.00029007970600882\\
-6109.14418263948	0.000915188258319774\\
-6103.28409229642	0.00155613801297503\\
-6097.42400195336	0.0021978106739798\\
-6091.5639116103	0.00282504149783154\\
-6085.70382126725	0.00342297749337938\\
-6079.84373092419	0.00397742895308915\\
-6073.98364058113	0.00447520599931261\\
-6068.12355023807	0.00490443216626425\\
-6062.26345989501	0.00525482756537569\\
-6056.40336955195	0.00551795488594006\\
-6050.54327920889	0.00568742234715295\\
-6044.68318886583	0.00575903872375827\\
-6038.82309852277	0.00573091668791684\\
-6032.96300817971	0.00560352192679455\\
-6027.10291783665	0.0053796667672068\\
-6021.24282749359	0.00506444834994351\\
-6015.38273715053	0.00466513270465152\\
-6009.52264680747	0.00419098735696594\\
-6003.66255646441	0.00365306632065881\\
-5997.80246612135	0.00306395245963145\\
-5991.94237577829	0.00243746322122374\\
-5986.08228543523	0.00178832661968941\\
-5980.22219509217	0.00113183506484267\\
-5974.36210474911	0.000483485169084496\\
-5968.50201440605	-0.000141387986402107\\
-5962.64192406299	-0.000727973497765935\\
-5956.78183371994	-0.00126233562435079\\
-5950.92174337688	-0.00173174479983086\\
-5945.06165303382	-0.00212498080952864\\
-5939.20156269076	-0.00243260091635069\\
-5933.3414723477	-0.00264716653143206\\
-5927.48138200464	-0.00276342299202658\\
-5921.62129166158	-0.00277842810518147\\
-5915.76120131852	-0.00269162631846643\\
-5909.90111097546	-0.00250486665543998\\
-5904.0410206324	-0.00222236387948641\\
-5898.18093028934	-0.00185060368842271\\
-5892.32083994628	-0.0013981940668296\\
-5886.46074960322	-0.000875666198067985\\
-5880.60065926016	-0.000295229537481451\\
-5874.7405689171	0.000329513260952879\\
-5868.88047857404	0.000983884911402183\\
-5863.02038823099	0.00165247809928863\\
-5857.16029788793	0.00231951803323882\\
-5851.30020754487	0.00296923441796199\\
-5845.44011720181	0.00358623408019082\\
-5839.58002685875	0.00415586544681112\\
-5833.71993651569	0.00466456624824702\\
-5827.85984617263	0.00510018619935562\\
-5821.99975582957	0.00545227698349191\\
-5816.13966548651	0.00571234262291438\\
-5810.27957514345	0.00587404423898748\\
-5804.41948480039	0.00593335427217524\\
-5798.55939445733	0.00588865641368789\\
-5792.69930411427	0.00574078877660408\\
-5786.83921377121	0.00549302916919449\\
-5780.97912342815	0.00515102269908503\\
-5775.11903308509	0.00472265329803579\\
-5769.25894274204	0.00421786208702458\\
-5763.39885239898	0.00364841676164796\\
-5757.53876205592	0.00302763734357482\\
-5751.67867171286	0.00237008468625006\\
-5745.8185813698	0.00169121901555685\\
-5739.95849102674	0.00100703650901828\\
-5734.09840068368	0.000333692453135872\\
-5728.23831034062	-0.000312880147372273\\
-5722.37821999756	-0.000917347507992998\\
-5716.5181296545	-0.00146533863688936\\
-5710.65803931144	-0.00194378692458454\\
-5704.79794896838	-0.00234124163505816\\
-5698.93785862532	-0.00264814187668898\\
-5693.07776828226	-0.00285704650173869\\
-5687.2176779392	-0.00296281441032375\\
-5681.35758759614	-0.00296273089486637\\
-5675.49749725308	-0.00285657692616216\\
-5669.63740691002	-0.002646639623084\\
-5663.77731656696	-0.00233766353442861\\
-5657.9172262239	-0.00193674375738302\\
-5652.05713588084	-0.00145316329447726\\
-5646.19704553779	-0.000898178371887775\\
-5640.33695519473	-0.000284756680071274\\
-5634.47686485167	0.000372725380382555\\
-5628.61677450861	0.00105881938305717\\
-5622.75668416555	0.00175736760918476\\
-5616.89659382249	0.00245188331715586\\
-5611.03650347943	0.00312593960227707\\
-5605.17641313637	0.00376355767896498\\
-5599.31632279331	0.00434958541164809\\
-5593.45623245025	0.00487005713112554\\
-5587.59614210719	0.00531252619536609\\
-5581.73605176413	0.00566636238044875\\
-5575.87596142107	0.00592300699925457\\
-5570.01587107801	0.00607617962991959\\
-5564.15578073495	0.00612203146568702\\
-5558.29569039189	0.00605924154656835\\
-5552.43560004883	0.00588905347689357\\
-5546.57550970577	0.00561525162976472\\
-5540.71541936271	0.00524407727078112\\
-5534.85532901965	0.00478408644935242\\
-5528.9952386766	0.00424595288912246\\
-5523.13514833354	0.00364222041250649\\
-5517.27505799048	0.00298701063816723\\
-5511.41496764742	0.00229569275805176\\
-5505.55487730436	0.00158452311235249\\
-5499.6947869613	0.000870263010159422\\
-5493.83469661824	0.000169783776807478\\
-5487.97460627518	-0.000500331670064663\\
-5482.11451593212	-0.00112418131098558\\
-5476.25442558906	-0.00168692129635067\\
-5470.394335246	-0.00217511905284908\\
-5464.53424490294	-0.00257707382942487\\
-5458.67415455988	-0.00288309703758909\\
-5452.81406421682	-0.00308574567399403\\
-5446.95397387376	-0.00318000320655496\\
-5441.09388353071	-0.00316340353342987\\
-5435.23379318765	-0.00303609495944454\\
-5429.37370284459	-0.00280084254465944\\
-5423.51361250153	-0.00246296863114866\\
-5417.65352215847	-0.00203023281513701\\
-5411.79343181541	-0.0015126540634109\\
-5405.93334147235	-0.000922279046914138\\
-5400.07325112929	-0.000272902043125599\\
-5394.21316078623	0.000420257084672758\\
-5388.35307044317	0.00114090931835258\\
-5382.49298010011	0.00187207898338601\\
-5376.63288975705	0.00259650333742868\\
-5370.77279941399	0.0032970398478728\\
-5364.91270907093	0.00395707155384661\\
-5359.05261872787	0.00456090093196473\\
-5353.19252838481	0.0050941229348565\\
-5347.33243804175	0.00554396834214876\\
-5341.47234769869	0.00589960924439551\\
-5335.61225735563	0.0061524193568112\\
-5329.75216701258	0.00629618290975367\\
-5323.89207666952	0.00632724706264815\\
-5318.03198632646	0.00624461411186288\\
-5312.1718959834	0.00604997117492418\\
-5306.31180564034	0.00574765650713192\\
-5300.45171529728	0.00534456309915285\\
-5294.59162495422	0.00484998168923948\\
-5288.73153461116	0.0042753867587451\\
-5282.8714442681	0.00363417043458561\\
-5277.01135392504	0.00294133046499977\\
-5271.15126358198	0.00221311953307835\\
-5265.29117323892	0.0014666641034942\\
-5259.43108289586	0.000719561736366034\\
-5253.5709925528	-1.05336688027332e-05\\
-5247.71090220974	-0.000706328928454045\\
-5241.85081186668	-0.00135130108202774\\
-5235.99072152362	-0.00193008965858885\\
-5230.13063118056	-0.00242886239529102\\
-5224.2705408375	-0.00283564570531639\\
-5218.41045049444	-0.00314061200521067\\
-5212.55036015138	-0.00333631701190556\\
-5206.69026980832	-0.00341788128565944\\
-5200.83017946527	-0.0033831115981668\\
-5194.97008912221	-0.00323255911665472\\
-5189.10999877915	-0.00296951287963104\\
-5183.24990843609	-0.0025999285654072\\
-5177.38981809303	-0.00213229408257217\\
-5171.52972774997	-0.00157743500799598\\
-5165.66963740691	-0.000948264325738051\\
-5159.80954706385	-0.00025948224664403\\
-5153.94945672079	0.000472766918436309\\
-5148.08936637773	0.00123127282810813\\
-5142.22927603467	0.00199816327395481\\
-5136.36918569161	0.00275532495227472\\
-5130.50909534855	0.00348483094289926\\
-5124.64900500549	0.00416936481180091\\
-5118.78891466243	0.00479263130979069\\
-5112.92882431937	0.00533974393228034\\
-5107.06873397632	0.00579758012743732\\
-5101.20864363326	0.00615509568318416\\
-5095.3485532902	0.00640359076642127\\
-5089.48846294714	0.0065369212118717\\
-5083.62837260408	0.00655164993633903\\
-5077.76828226102	0.00644713475472385\\
-5071.90819191796	0.00622555036950828\\
-5066.0481015749	0.00589184385245099\\
-5060.18801123184	0.00545362450936383\\
-5054.32792088878	0.00492099056791236\\
-5048.46783054572	0.00430629663020967\\
-5042.60774020266	0.00362386723746691\\
-5036.7476498596	0.00288966318301387\\
-5030.88755951654	0.00212090834009687\\
-5025.02746917348	0.00133568572432813\\
-5019.16737883043	0.000552512258170917\\
-5013.30728848737	-0.000210097768708227\\
-5007.44719814431	-0.00093407025742763\\
-5001.58710780125	-0.00160219954101694\\
-4995.72701745819	-0.00219855716803276\\
-4989.86692711513	-0.00270887152233677\\
-4984.00683677207	-0.00312086923776121\\
-4978.14674642901	-0.00342457024591402\\
-4972.28665608595	-0.00361252937091565\\
-4966.42656574289	-0.00368001862940849\\
-4960.56647539983	-0.00362514578034095\\
-4954.70638505677	-0.00344890616422852\\
-4948.84629471371	-0.00315516643918719\\
-4942.98620437065	-0.00275058042713254\\
-4937.12611402759	-0.00224443888751755\\
-4931.26602368453	-0.00164845660097083\\
-4925.40593334147	-0.00097650163525037\\
-4919.54584299841	-0.000244273043748826\\
-4913.68575265535	0.000531065519587303\\
-4907.82566231229	0.00133128772401368\\
-4901.96557196923	0.0021375326189662\\
-4896.10548162617	0.00293074877724391\\
-4890.24539128312	0.00369214407177429\\
-4884.38530094006	0.00440363047408418\\
-4878.525210597	0.00504825335207591\\
-4872.66512025394	0.00561059508553304\\
-4866.80502991088	0.00607714339672218\\
-4860.94493956782	0.00643661560344464\\
-4855.08484922476	0.0066802310206236\\
-4849.2247588817	0.00680192493989265\\
-4843.36466853864	0.00679849898248883\\
-4837.50457819558	0.00666970410702392\\
-4831.64448785252	0.00641825413822676\\
-4825.78439750946	0.00604976931606611\\
-4819.9243071664	0.00557265101788421\\
-4814.06421682334	0.004997890433962\\
-4808.20412648028	0.00433881554615089\\
-4802.34403613722	0.00361078222517441\\
-4796.48394579416	0.00283081660068235\\
-4790.6238554511	0.00201721702516986\\
-4784.76376510804	0.00118912493071111\\
-4778.90367476498	0.000366074636237159\\
-4773.04358442193	-0.000432467313435233\\
-4767.18349407887	-0.00118756242026539\\
-4761.32340373581	-0.00188124924158634\\
-4755.46331339275	-0.00249697049601303\\
-4749.60322304969	-0.00301996811788801\\
-4743.74313270663	-0.00343763684092504\\
-4737.88304236357	-0.00373982784746673\\
-4732.02295202051	-0.00391909517608863\\
-4726.16286167745	-0.003970878913277\\
-4720.30277133439	-0.00389362067287524\\
-4714.44268099133	-0.00368880845362635\\
-4708.58259064827	-0.00336094962661905\\
-4702.72250030521	-0.0029174724982079\\
-4696.86240996215	-0.00236855858275609\\
-4691.0023196191	-0.0017269093620529\\
-4685.14222927604	-0.00100745286604061\\
-4679.28213893298	-0.000226996845361649\\
-4673.42204858992	0.000596163413792659\\
-4667.56195824686	0.00144267348875718\\
-4661.7018679038	0.00229257440960022\\
-4655.84177756074	0.00312577276856715\\
-4649.98168721768	0.00392251528471928\\
-4644.12159687462	0.00466385662267876\\
-4638.26150653156	0.00533210939437569\\
-4632.4014161885	0.00591126566192369\\
-4626.54132584544	0.00638737990567538\\
-4620.68123550238	0.00674890430226135\\
-4614.82114515932	0.00698696826112283\\
-4608.96105481626	0.00709559546077547\\
-4603.1009644732	0.00707185308507725\\
-4597.24087413014	0.00691592954726432\\
-4591.38078378708	0.00663113866645557\\
-4585.52069344402	0.00622384999610994\\
-4579.66060310097	0.00570334674537006\\
-4573.80051275791	0.00508161444939937\\
-4567.94042241485	0.00437306518987947\\
-4562.08033207179	0.00359420370226787\\
-4556.22024172873	0.00276324309694081\\
-4550.36015138567	0.00189967913315761\\
-4544.50006104261	0.00102383298922104\\
-4538.63997069955	0.000156373243617816\\
-4532.77988035649	-0.000682171696352505\\
-4526.91979001343	-0.00147189922510477\\
-4521.05969967037	-0.00219400474851233\\
-4515.19960932731	-0.00283122927761571\\
-4509.33951898425	-0.00336827176109309\\
-4503.47942864119	-0.0037921563139937\\
-4497.61933829813	-0.00409254553806755\\
-4491.75924795507	-0.00426199237767411\\
-4485.89915761201	-0.00429612438516661\\
-4480.03906726895	-0.00419375584983331\\
-4474.17897692589	-0.00395692493507415\\
-4468.31888658283	-0.00359085473168586\\
-4462.45879623977	-0.00310383893029356\\
-4456.59870589671	-0.00250705459753157\\
-4450.73861555366	-0.0018143062706842\\
-4444.8785252106	-0.00104170721982999\\
-4439.01843486754	-0.000207305227401601\\
-4433.15834452448	0.000669338432613838\\
-4427.29825418142	0.00156760700324504\\
-4421.43816383836	0.00246631252594408\\
-4415.5780734953	0.00334419502831241\\
-4409.71798315224	0.00418042486645277\\
-4403.85789280918	0.00495509636141761\\
-4397.99780246612	0.00564970104074225\\
-4392.13771212306	0.00624756924310367\\
-4386.27762178	0.00673426955943134\\
-4380.41753143694	0.0070979565481617\\
-4374.55744109388	0.00732965835699008\\
-4368.69735075082	0.00742349727841564\\
-4362.83726040776	0.00737683782939752\\
-4356.97717006471	0.00719035864244292\\
-4351.11707972165	0.00686804624474679\\
-4345.25698937859	0.00641711063969909\\
-4339.39689903553	0.00584782445449976\\
-4333.53680869247	0.00517328922701587\\
-4327.67671834941	0.0044091341367855\\
-4321.81662800635	0.00357315409801373\\
-4315.95653766329	0.00268489558418496\\
-4310.09644732023	0.0017651998147762\\
-4304.23635697717	0.000835713969562203\\
-4298.37626663411	-8.16181154118465e-05\\
-4292.51617629105	-0.000965073798040115\\
-4286.65608594799	-0.00179366581587415\\
-4280.79599560493	-0.00254764064671001\\
-4274.93590526187	-0.00320894922013131\\
-4269.07581491882	-0.00376167877409589\\
-4263.21572457576	-0.00419243553368495\\
-4257.3556342327	-0.00449066902062229\\
-4251.49554388964	-0.00464893015194464\\
-4245.63545354658	-0.00466305682746218\\
-4239.77536320352	-0.00453228239845683\\
-4233.91527286046	-0.00425926421897166\\
-4228.0551825174	-0.00385003135867849\\
-4222.19509217434	-0.0033138524653146\\
-4216.33500183128	-0.00266302665218088\\
-4210.47491148822	-0.00191260211466732\\
-4204.61482114516	-0.00108002890107251\\
-4198.7547308021	-0.000184753838597746\\
-4192.89464045904	0.000752232993288131\\
-4187.03455011598	0.00170888967644035\\
-4181.17445977292	0.00266264026045302\\
-4175.31436942986	0.00359090680916101\\
-4169.4542790868	0.00447164315926095\\
-4163.59418874374	0.00528385778461535\\
-4157.73409840068	0.00600811337621788\\
-4151.87400805762	0.00662699126065033\\
-4146.01391771456	0.00712550957154266\\
-4140.15382737151	0.00749148514694538\\
-4134.29373702845	0.00771583042410058\\
-4128.43364668539	0.00779277810868319\\
-4122.57355634233	0.0077200280816226\\
-4116.71346599927	0.00749881282498909\\
-4110.85337565621	0.00713387956008516\\
-4104.99328531315	0.00663338925592373\\
-4099.13319497009	0.00600873462693156\\
-4093.27310462703	0.00527428116240368\\
-4087.41301428397	0.00444703705929061\\
-4081.55292394091	0.00354625962961642\\
-4075.69283359785	0.00259300727804757\\
-4069.83274325479	0.00160964746120418\\
-4063.97265291173	0.000619332112477259\\
-4058.11256256867	-0.000354547176565682\\
-4052.25247222561	-0.00128891141044684\\
-4046.39238188255	-0.00216154085385285\\
-4040.53229153949	-0.00295160316964667\\
-4034.67220119643	-0.00364015049060143\\
-4028.81211085337	-0.00421057357706133\\
-4022.95202051031	-0.00464900218843632\\
-4017.09193016726	-0.00494464203082911\\
-4011.2318398242	-0.00509004011016201\\
-4005.37174948114	-0.00508127198444811\\
-3999.51165913808	-0.00491804623215694\\
-3993.65156879502	-0.00460372339460904\\
-3987.79147845196	-0.00414524866070209\\
-3981.9313881089	-0.00355299959823261\\
-3976.07129776584	-0.00284055224740289\\
-3970.21120742278	-0.00202437083091371\\
-3964.35111707972	-0.0011234281582107\\
-3958.49102673666	-0.000158765461476173\\
-3952.6309363936	0.000846998138636516\\
-3946.77084605054	0.00187019511530769\\
-3940.91075570748	0.00288666558511696\\
-3935.05066536443	0.00387232687513467\\
-3929.19057502137	0.00480374319374668\\
-3923.33048467831	0.00565868194429951\\
-3917.47039433525	0.00641664349010055\\
-3911.61030399219	0.00705935176191942\\
-3905.75021364913	0.00757119398092691\\
-3899.89012330607	0.00793959893294186\\
-3894.03003296301	0.00815534464361101\\
-3888.16994261995	0.00821278794353416\\
-3882.30985227689	0.00811001022850894\\
-3876.44976193383	0.00784887568288021\\
-3870.58967159077	0.00743500028746927\\
-3864.72958124771	0.00687763203637733\\
-3858.86949090465	0.00618944488518528\\
-3853.00940056159	0.00538625100051817\\
-3847.14931021853	0.00448663782634899\\
-3841.28921987547	0.00351153828073021\\
-3835.42912953241	0.00248374400730378\\
-3829.56903918935	0.00142737298294409\\
-3823.7089488463	0.000367303903315773\\
-3817.84885850324	-0.00067140940488208\\
-3811.98876816018	-0.00166412977740159\\
-3806.12867781712	-0.00258722074885845\\
-3800.26858747406	-0.00341860900343866\\
-3794.408497131	-0.00413831187208375\\
-3788.54840678794	-0.00472891728670607\\
-3782.68831644488	-0.0051760046804829\\
-3776.82822610182	-0.00546849667736869\\
-3770.96813575876	-0.00559893301236882\\
-3765.1080454157	-0.00556365993117846\\
-3759.24795507264	-0.00536293029028585\\
-3753.38786472958	-0.00500091166995849\\
-3747.52777438652	-0.00448560197798168\\
-3741.66768404346	-0.00382865420313579\\
-3735.8075937004	-0.00304511413168779\\
-3729.94750335734	-0.00215307690828768\\
-3724.08741301428	-0.00117327026290317\\
-3718.22732267122	-0.000128573984568956\\
-3712.36723232816	0.000956513234835723\\
-3706.5071419851	0.00205644718068208\\
-3700.64705164204	0.00314523831719472\\
-3694.78696129899	0.00419706472580414\\
-3688.92687095593	0.00518688334262058\\
-3683.06678061287	0.00609102504420597\\
-3677.20669026981	0.00688775946002173\\
-3671.34659992675	0.00755781605084114\\
-3665.48650958369	0.00808484897791946\\
-3659.62641924063	0.00845583456773837\\
-3653.76632889757	0.00866139172788757\\
-3647.90623855451	0.00869601745434875\\
-3642.04614821145	0.00855823154619699\\
-3636.18605786839	0.00825062676708455\\
-3630.32596752533	0.00777982291194471\\
-3624.46587718227	0.00715632550136256\\
-3618.60578683921	0.00639429208248266\\
-3612.74569649615	0.00551121130918611\\
-3606.88560615309	0.00452750205278483\\
-3601.02551581004	0.00346604171332176\\
-3595.16542546698	0.00235163460776858\\
-3589.30533512392	0.00121043276809243\\
-3583.44524478086	6.93226548304334e-05\\
-3577.5851544378	-0.00104470785231902\\
-3571.72506409474	-0.00210520730384014\\
-3565.86497375168	-0.00308688831490519\\
-3560.00488340862	-0.00396623011362101\\
-3554.14479306556	-0.00472204166663773\\
-3548.2847027225	-0.00533597192109993\\
-3542.42461237944	-0.00579295489945733\\
-3536.56452203638	-0.00608157887400178\\
-3530.70443169332	-0.00619437060029849\\
-3524.84434135026	-0.00612798755876339\\
-3518.9842510072	-0.00588331330029163\\
-3513.12416066415	-0.00546545326007842\\
-3507.26407032109	-0.00488363074642171\\
-3501.40397997803	-0.0041509851657094\\
-3495.54388963497	-0.00328427686532124\\
-3489.68379929191	-0.00230350519651384\\
-3483.82370894885	-0.00123144847820071\\
-3477.96361860579	-9.3136419914711e-05\\
-3472.10352826273	0.00108473279853733\\
-3466.24343791967	0.00227441723806101\\
-3460.38334757661	0.00344778310696529\\
-3454.52325723355	0.00457696856755867\\
-3448.66316689049	0.00563504380392908\\
-3442.80307654743	0.00659665173717525\\
-3436.94298620437	0.0074386141681885\\
-3431.08289586131	0.00814048888322185\\
-3425.22280551825	0.00868506435612603\\
-3419.36271517519	0.00905878010104059\\
-3413.50262483213	0.00925206243771255\\
-3407.64253448907	0.00925956738492283\\
-3401.78244414601	0.00908032456041693\\
-3395.92235380295	0.00871777827683565\\
-3390.06226345989	0.00817972443606576\\
-3384.20217311684	0.00747814428085077\\
-3378.34208277378	0.00662893850202409\\
-3372.48199243072	0.00565156756964951\\
-3366.62190208766	0.00456860639406005\\
-3360.7618117446	0.00340522348184922\\
-3354.90172140154	0.00218859657776847\\
-3349.04163105848	0.000947278333604621\\
-3343.18154071542	-0.000289473216962629\\
-3337.32145037236	-0.00149238370437206\\
-3331.4613600293	-0.00263285479000728\\
-3325.60126968624	-0.00368364350610682\\
-3319.74117934318	-0.00461951245901207\\
-3313.88108900012	-0.00541783540592309\\
-3308.02099865706	-0.00605914370243496\\
-3302.160908314	-0.00652760045285251\\
-3296.30081797094	-0.00681139084674552\\
-3290.44072762788	-0.00690301909406241\\
-3284.58063728482	-0.00679950453741863\\
-3278.72054694176	-0.00650247186787117\\
-3272.8604565987	-0.00601813285091013\\
-3267.00036625564	-0.00535715951998051\\
-3261.14027591259	-0.00453445135847831\\
-3255.28018556953	-0.00356880150643534\\
-3249.42009522647	-0.0024824694344331\\
-3243.56000488341	-0.00130066977059606\\
-3237.69991454035	-5.09889885340821e-05\\
-3231.83982419729	0.0012372565793532\\
-3225.97973385423	0.00253370649345193\\
-3220.11964351117	0.00380766971341746\\
-3214.25955316811	0.00502884912276843\\
-3208.39946282505	0.00616805999901185\\
-3202.53937248199	0.00719792542440139\\
-3196.67928213893	0.00809353209863417\\
-3190.81919179587	0.00883303087731186\\
-3184.95910145281	0.00939816759370712\\
-3179.09901110976	0.00977473130189391\\
-3173.2389207667	0.00995290897301492\\
-3167.37883042364	0.00992753783331191\\
-3161.51874008058	0.00969824891316643\\
-3155.65864973752	0.00926949791224047\\
-3149.79855939446	0.00865048213370397\\
-3143.9384690514	0.00785494492196435\\
-3138.07837870834	0.00690087170401437\\
-3132.21828836528	0.00581008431329038\\
-3126.35819802222	0.00460774270801342\\
-3120.49810767916	0.00332176542560803\\
-3114.6380173361	0.00198218208787227\\
-3108.77792699304	0.00062043293757984\\
-3102.91783664998	-0.00073136829003325\\
-3097.05774630692	-0.00204119091566028\\
-3091.19765596386	-0.00327784549821102\\
-3085.3375656208	-0.00441172574216919\\
-3079.47747527774	-0.00541551652062163\\
-3073.61738493469	-0.0062648511204916\\
-3067.75729459163	-0.00693890193505245\\
-3061.89720424857	-0.0074208903251861\\
-3056.03711390551	-0.00769850321069726\\
-3050.17702356245	-0.007764206095785\\
-3044.31693321939	-0.00761544462933128\\
-3038.45684287633	-0.00725472939405607\\
-3032.59675253327	-0.00668960135077654\\
-3026.73666219021	-0.00593247817007214\\
-3020.87657184715	-0.0050003844974638\\
-3015.01648150409	-0.00391457195439474\\
-3009.15639116103	-0.00270003730960112\\
-3003.29630081797	-0.00138494970234729\\
nan	nan\\
-2991.57612013185	0.00142231272316089\\
-2985.71602978879	0.00284844103973382\\
-2979.85593944573	0.00424457763248576\\
-2973.99584910267	0.0055774538797864\\
-2968.13575875961	0.00681512974368829\\
-2962.27566841655	0.00792775625113928\\
-2956.41557807349	0.00888829241280049\\
-2950.55548773043	0.00967315940618472\\
-2944.69539738738	0.0102628162442474\\
-2938.83530704432	0.0106422429257624\\
-2932.97521670126	0.0108013191729026\\
-2927.1151263582	0.0107350892715334\\
-2921.25503601514	0.0104439061639191\\
-2915.39494567208	0.00993345076560541\\
-2909.53485532902	0.00921462539855497\\
-2903.67476498596	0.00830332320286377\\
-2897.8146746429	0.00722007832548872\\
-2891.95458429984	0.0059896045260749\\
-2886.09449395678	0.00464023251354548\\
-2880.23440361372	0.00320325877352022\\
-2874.37431327066	0.00171222080351519\\
-2868.5142229276	0.00020211548993306\\
-2862.65413258454	-0.00129142120849967\\
-2856.79404224148	-0.00273295409464437\\
-2850.93395189842	-0.00408808834539056\\
-2845.07386155537	-0.00532428906249935\\
-2839.21377121231	-0.00641166128931641\\
-2833.35368086925	-0.00732367184756998\\
-2827.49359052619	-0.00803779562351115\\
-2821.63350018313	-0.00853607061840782\\
-2815.77340984007	-0.00880554815098393\\
-2809.91331949701	-0.00883862699654855\\
-2804.05322915395	-0.00863326292849645\\
-2798.19313881089	-0.00819304802201049\\
-2792.33304846783	-0.00752715712211438\\
-2786.47295812477	-0.00665016200192275\\
-2780.61286778171	-0.00558171686171607\\
-2774.75277743865	-0.00434612187831032\\
-2768.89268709559	-0.00297177442992492\\
-2763.03259675254	-0.00149052032857914\\
-2757.17250640948	6.30801788299924e-05\\
-2751.31241606642	0.00165255478330148\\
-2745.45232572336	0.00324037129351002\\
-2739.5922353803	0.00478881977361752\\
-2733.73214503724	0.00626090397022931\\
-2727.87205469418	0.0076212210206767\\
-2722.01196435112	0.0088368085913962\\
-2716.15187400806	0.00987793924262647\\
-2710.291783665	0.0107188429453168\\
-2704.43169332194	0.0113383402598657\\
-2698.57160297888	0.0117203706928083\\
-2692.71151263582	0.0118544031335403\\
-2686.85142229276	0.0117357179755324\\
-2680.9913319497	0.0113655534972052\\
-2675.13124160664	0.0107511122353016\\
-2669.27115126358	0.00990542636467931\\
-2663.41106092052	0.00884708441849526\\
-2657.55097057746	0.00759982497022164\\
-2651.6908802344	0.00619200607336325\\
-2645.83078989134	0.00465596223862386\\
-2639.97069954828	0.00302726345532345\\
-2634.11060920523	0.00134389316210531\\
-2628.25051886217	-0.000354635910199807\\
-2622.39042851911	-0.00202820751994197\\
-2616.53033817605	-0.00363704770804635\\
-2610.67024783299	-0.00514266838524258\\
-2604.81015748993	-0.00650878641024377\\
-2598.95006714687	-0.00770219629457858\\
-2593.08997680381	-0.00869357562946508\\
-2587.22988646075	-0.00945820378437724\\
-2581.36979611769	-0.00997657635428583\\
-2575.50970577463	-0.010234900178849\\
-2569.64961543157	-0.0102254564807398\\
-2563.78952508851	-0.00994682270303402\\
-2557.92943474545	-0.00940394689773573\\
-2552.06934440239	-0.00860807195946099\\
-2546.20925405933	-0.00757651052092272\\
-2540.34916371627	-0.00633227485948068\\
-2534.48907337321	-0.00490356960841335\\
-2528.62898303015	-0.00332315835780078\\
-2522.76889268709	-0.00162761827576789\\
-2516.90880234403	0.000143500384176332\\
-2511.04871200098	0.00194859066912146\\
-2505.18862165792	0.00374496006777452\\
-2499.32853131486	0.00548983383634214\\
-2493.4684409718	0.0071413670297236\\
-2487.60835062874	0.00865964139479158\\
-2481.74826028568	0.0100076234812093\\
-2475.88816994262	0.0111520610785373\\
-2470.02807959956	0.0120642963894758\\
-2464.1679892565	0.0127209761646408\\
-2458.30789891344	0.0131046413218598\\
-2452.44780857038	0.013204181292042\\
-2446.58771822732	0.0130151414254016\\
-2440.72762788426	0.0125398751761778\\
-2434.8675375412	0.0117875363843739\\
-2429.00744719815	0.0107739107154888\\
-2423.14735685509	0.00952108910289104\\
-2417.28726651203	0.00805698978402546\\
-2411.42717616897	0.00641473913918165\\
-2405.56708582591	0.00463192494249867\\
-2399.70699548285	0.0027497387414227\\
-2393.84690513979	0.000812026814088753\\
-2387.98681479673	-0.00113572855030081\\
-2382.12672445367	-0.00304747390285287\\
-2376.26663411061	-0.00487767081481468\\
-2370.40654376755	-0.00658237789869062\\
-2364.54645342449	-0.00812030309476438\\
-2358.68636308143	-0.00945380116082388\\
-2352.82627273837	-0.0105497924085143\\
-2346.96618239531	-0.0113805804089319\\
-2341.10609205225	-0.0119245486059466\\
-2335.24600170919	-0.0121667184788795\\
-2329.38591136613	-0.012099155029404\\
-2323.52582102307	-0.0117212088592692\\
-2317.66573068002	-0.0110395878732599\\
-2311.80564033696	-0.0100682556059625\\
-2305.9455499939	-0.00882815723345431\\
-2300.08545965084	-0.00734677839859645\\
-2294.22536930778	-0.00565754595920392\\
-2288.36527896472	-0.00379908356424504\\
-2282.50518862166	-0.00181433848336023\\
-2276.6450982786	0.00025040072141383\\
-2270.78500793554	0.00234657336730572\\
-2264.92491759248	0.00442447812832633\\
-2259.06482724942	0.00643444061830099\\
-2253.20473690636	0.00832798981830225\\
-2247.3446465633	0.0100590156610343\\
-2241.48455622024	0.0115848803034256\\
-2235.62446587718	0.0128674564878283\\
-2229.76437553412	0.0138740678917618\\
-2223.90428519106	0.0145783084693991\\
-2218.044194848	0.014960720446091\\
-2212.18410450494	0.0150093137914393\\
-2206.32401416188	0.0147199135818754\\
-2200.46392381882	0.0140963256054732\\
-2194.60383347576	0.0131503147588297\\
-2188.74374313271	0.0119013951481239\\
-2182.88365278965	0.0103764352298235\\
-2177.02356244659	0.00860908570863427\\
-2171.16347210353	0.00663904214923177\\
-2165.30338176047	0.00451115825297398\\
-2159.44329141741	0.0022744294019185\\
-2153.58320107435	-1.91307012518548e-05\\
-2147.72311073129	-0.00231569475419591\\
-2141.86302038823	-0.00456088025260645\\
-2136.00293004517	-0.00670103237302143\\
-2130.14283970211	-0.00868450131249065\\
-2124.28274935905	-0.010462883917712\\
-2118.42265901599	-0.0119922000052796\\
-2112.56256867293	-0.0132339750428316\\
-2106.70247832987	-0.0141562028041475\\
-2100.84238798681	-0.0147341641942396\\
-2094.98229764376	-0.0149510816021221\\
-2089.1222073007	-0.0147985918175227\\
-2083.26211695764	-0.0142770246534892\\
-2077.40202661458	-0.013395478866648\\
-2071.54193627152	-0.0121716916531481\\
-2065.68184592846	-0.0106317028143412\\
-2059.8217555854	-0.00880931952503118\\
-2053.96166524234	-0.00674539237829642\\
-2048.10157489928	-0.00448691792266446\\
-2042.24148455622	-0.00208598712897101\\
-2036.38139421316	0.000401396964019029\\
-2030.5213038701	0.0029166058741285\\
-2024.66121352704	0.00539974730823386\\
-2018.80112318398	0.00779107048717147\\
-2012.94103284092	0.0100323825085114\\
-2007.08094249787	0.0120684425051182\\
-2001.22085215481	0.0138483005413317\\
-1995.36076181175	0.0153265491510133\\
-1989.50067146869	0.0164644571434375\\
-1983.64058112563	0.01723095775178\\
-1977.78049078257	0.017603466330637\\
-1971.92040043951	0.0175685065489551\\
-1966.06031009645	0.0171221282966627\\
-1960.20021975339	0.0162701052353912\\
-1954.34012941033	0.0150279049613457\\
-1948.48003906727	0.0134204300100195\\
-1942.61994872421	0.0114815332838823\\
-1936.75985838115	0.00925331681943426\\
-1930.89976803809	0.00678522797837604\\
-1925.03967769503	0.00413297206022388\\
-1919.17958735197	0.00135726484147846\\
-1913.31949700891	-0.00147754743827254\\
-1907.45940666585	-0.00430496970760822\\
-1901.59931632279	-0.00705790764695253\\
-1895.73922597973	-0.00967024958476721\\
-1889.87913563667	-0.0120784439005949\\
-1884.01904529361	-0.0142230348116931\\
-1878.15895495056	-0.0160501199614507\\
-1872.2988646075	-0.0175126946320237\\
-1866.43877426444	-0.0185718496407358\\
-1860.57868392138	-0.0191977930156998\\
-1854.71859357832	-0.019370669308774\\
-1848.85850323526	-0.0190811548232647\\
-1842.9984128922	-0.0183308120121746\\
-1837.13832254914	-0.017132191727909\\
-1831.27823220608	-0.015508677757939\\
-1825.41814186302	-0.0134940740375226\\
-1819.55805151996	-0.0111319409451808\\
-1813.6979611769	-0.00847469302921748\\
-1807.83787083384	-0.00558247624155257\\
-1801.97778049078	-0.0025218481337946\\
-1796.11769014772	0.000635710629293177\\
-1790.25759980466	0.00381542077874518\\
-1784.3975094616	0.0069409287138406\\
-1778.53741911854	0.00993609147400074\\
-1772.67732877548	0.0127267814135153\\
-1766.81723843242	0.0152426685735685\\
-1760.95714808936	0.0174189387001174\\
-1755.09705774631	0.019197905785455\\
-1749.23696740325	0.0205304799005163\\
-1743.37687706019	0.0213774539147755\\
-1737.51678671713	0.0217105764099564\\
-1731.65669637407	0.0215133826033089\\
-1725.79660603101	0.0207817603317231\\
-1719.93651568795	0.0195242339738226\\
-1714.07642534489	0.017761955499055\\
-1708.21633500183	0.0155283984775505\\
-1702.35624465877	0.0128687577205047\\
-1696.49615431571	0.0098390640935434\\
-1690.63606397265	0.00650503079562319\\
-1684.77597362959	0.00294065386667799\\
-1678.91588328653	-0.000773404269295005\\
-1673.05579294348	-0.00455161395270145\\
-1667.19570260042	-0.00830552109696161\\
-1661.33561225736	-0.0119457906926107\\
-1655.4755219143	-0.0153843050391008\\
-1649.61543157124	-0.0185362687434548\\
-1643.75534122818	-0.0213222716012969\\
-1637.89525088512	-0.0236702606628685\\
-1632.03516054206	-0.0255173740877348\\
-1626.175070199	-0.0268115918039416\\
-1620.31497985594	-0.0275131614658803\\
-1614.45488951288	-0.0275957626874273\\
-1608.59479916982	-0.0270473779280049\\
-1602.73470882676	-0.0258708446201176\\
-1596.8746184837	-0.0240840700213118\\
-1591.01452814064	-0.0217198976977746\\
-1585.15443779758	-0.0188256223530538\\
-1579.29434745452	-0.0154621577261232\\
-1573.43425711146	-0.0117028703150683\\
-1567.57416676841	-0.00763209957447714\\
-1561.71407642535	-0.00334339277894496\\
-1555.85398608229	0.00106251020646209\\
-1549.99389573923	0.00547990064286727\\
-1544.13380539617	0.00980048597372192\\
-1538.27371505311	0.013915884500524\\
-1532.41362471005	0.0177201744695551\\
-1526.55353436699	0.0211124400539121\\
-1520.69344402393	0.0239992556573947\\
-1514.83335368087	0.0262970502124335\\
-1508.97326333781	0.027934294692799\\
-1503.11317299475	0.0288534589218024\\
-1497.25308265169	0.0290126878692258\\
-1491.39299230863	0.028387152936888\\
-1485.53290196557	0.0269700401290352\\
-1479.67281162251	0.0247731443769706\\
-1473.81272127945	0.0218270474814445\\
-1467.95263093639	0.0181808659890328\\
-1462.09254059333	0.0139015646557745\\
-1456.23245025027	0.00907284075713785\\
-1450.37235990721	0.00379359419896103\\
-1444.51226956415	-0.00182399207114651\\
-1438.6521792211	-0.00765672763855815\\
-1432.79208887804	-0.013573003818232\\
-1426.93199853498	-0.019435646200381\\
-1421.07190819192	-0.0251049733031036\\
-1415.21181784886	-0.0304419982165962\\
-1409.3517275058	-0.0353117055066393\\
-1403.49163716274	-0.0395863323882404\\
-1397.63154681968	-0.0431485813924892\\
-1391.77145647662	-0.0458946914965311\\
-1385.91136613356	-0.0477372959797141\\
-1380.0512757905	-0.0486079981452268\\
-1374.19118544744	-0.048459600428606\\
-1368.33109510438	-0.0472679282845822\\
-1362.47100476132	-0.0450331974637927\\
-1356.61091441826	-0.0417808817478559\\
-1350.7508240752	-0.0375620477701387\\
-1344.89073373214	-0.032453133982438\\
-1339.03064338909	-0.0265551619885406\\
-1333.17055304603	-0.0199923800795998\\
-1327.31046270297	-0.0129103506877623\\
-1321.45037235991	-0.00547350534665286\\
-1315.59028201685	0.00213779761317895\\
-1309.73019167379	0.0097306662554505\\
-1303.87010133073	0.0171034623045789\\
-1298.01001098767	0.0240499161933233\\
-1292.14992064461	0.0303634952677014\\
-1286.28983030155	0.0358419435443458\\
-1280.42973995849	0.0402919059350923\\
-1274.56964961543	0.0435335460266654\\
-1268.70955927237	0.0454050644083306\\
-1262.84946892931	0.0457670242753695\\
-1256.98937858626	0.0445063926070338\\
-1251.1292882432	0.0415402086416198\\
-1245.26919790014	0.0368187965883548\\
-1239.40910755708	0.030328446479454\\
-1233.54901721402	0.0220934956158966\\
-1227.68892687096	0.012177753102854\\
-1221.8288365279	0.000685221285856634\\
-1215.96874618484	-0.0122399196904163\\
-1210.10865584178	-0.0264140847141357\\
-1204.24856549872	-0.0416158205683189\\
-1198.38847515566	-0.0575880992416865\\
-1192.5283848126	-0.0740413457986789\\
-1186.66829446954	-0.0906571648315894\\
-1180.80820412648	-0.107092715865975\\
-1174.94811378342	-0.122985675223848\\
-1169.08802344036	-0.137959709919137\\
-1163.2279330973	-0.15163037851408\\
-1157.36784275424	-0.163611364615539\\
-1151.50775241118	-0.173520941108077\\
-1145.64766206812	-0.180988557399433\\
-1139.78757172506	-0.185661438070636\\
-1133.927481382	-0.18721107940182\\
-1128.06739103895	-0.185339530454364\\
-1122.20730069589	-0.179785347520509\\
-1116.34721035283	-0.170329115104188\\
-1110.48712000977	-0.156798432711866\\
-1104.62702966671	-0.139072274887151\\
-1098.76693932365	-0.117084641646392\\
-1092.90684898059	-0.0908274278657117\\
-1087.04675863753	-0.0603524528168039\\
-1081.18666829447	-0.0257726048238335\\
-1075.32657795141	0.0127379292904945\\
-1069.46648760835	0.0549443651140472\\
-1063.60639726529	0.100552951453853\\
-1057.74630692223	0.149213515023878\\
-1051.88621657917	0.200522887640001\\
-1046.02612623611	0.254029169754871\\
-1040.16603589305	0.309236770140319\\
-1034.30594554999	0.365612148604064\\
-1028.44585520693	0.422590176891381\\
-1022.58576486387	0.479581022814801\\
-1016.72567452081	0.535977454104738\\
-1010.86558417775	0.591162451709148\\
-1005.00549383469	0.644517017647452\\
-999.145403491639	0.695428059563599\\
-993.285313148579	0.743296233500197\\
-987.425222805519	0.787543627687799\\
-981.56513246246	0.827621173336601\\
-975.7050421194	0.863015673755593\\
-969.84495177634	0.893256349960191\\
-963.98486143328	0.917920809728259\\
-958.124771090221	0.936640357141518\\
-952.264680747161	0.94910457105228\\
-946.404590404101	0.95506509343728\\
-940.544500061042	0.954338581889728\\
-934.684409717982	0.946808794458036\\
-928.824319374922	0.932427789082648\\
-922.964229031862	0.91121623424658\\
-917.104138688806	0.883262841547149\\
-911.244048345747	0.848722944261556\\
-905.383958002687	0.807816259181336\\
-899.523867659627	0.760823880633252\\
-893.663777316568	0.708084566821275\\
-887.803686973508	0.649990387873447\\
-881.943596630448	0.58698181345213\\
-876.083506287388	0.519542324044213\\
-870.223415944329	0.448192635362775\\
-864.363325601269	0.373484628305135\\
-858.503235258209	0.29599507864572\\
-852.64314491515	0.216319280481339\\
-846.78305457209	0.135064655748203\\
-840.92296422903	0.0528444387668077\\
-835.062873885974	-0.0297284798989344\\
-829.202783542914	-0.112047472118835\\
-823.342693199855	-0.193517782451845\\
-817.482602856795	-0.273561767257345\\
-811.622512513735	-0.35162370894472\\
-805.762422170676	-0.427174161292262\\
-799.902331827616	-0.499713791710958\\
-794.042241484556	-0.568776696596996\\
-788.182151141496	-0.633933176002452\\
-782.322060798437	-0.694791963845459\\
-776.461970455377	-0.751001919424186\\
-770.601880112317	-0.802253194759983\\
-764.741789769258	-0.84827790079681\\
-758.881699426198	-0.888850302444771\\
-753.021609083142	-0.923786578908758\\
-747.161518740082	-0.952944191032698\\
-741.301428397022	-0.976220901207955\\
-735.441338053963	-0.993553494367116\\
-729.581247710903	-1.00491625035163\\
-723.721157367843	-1.01031921814018\\
-717.861067024784	-1.00980634184636\\
-712.000976681724	-1.00345348679411\\
-706.140886338664	-0.991366410834391\\
-700.280795995604	-0.973678722777293\\
-694.420705652545	-0.950549865111324\\
-688.560615309485	-0.922163153296808\\
-682.700524966425	-0.888723898274073\\
-676.840434623366	-0.85045763296809\\
-670.980344280306	-0.807608457584923\\
-665.12025393725	-0.760437512269963\\
-659.26016359419	-0.709221579942353\\
-653.40007325113	-0.654251816316046\\
-647.539982908071	-0.595832598875509\\
-641.679892565011	-0.534280481878652\\
-635.819802221951	-0.469923240314452\\
-629.959711878892	-0.403098982382879\\
-624.099621535832	-0.334155307436597\\
-618.239531192772	-0.263448484581125\\
-612.379440849712	-0.191342626164042\\
-606.519350506653	-0.118208830417588\\
-600.659260163593	-0.0444242682066619\\
-594.799169820533	0.0296288094771863\\
-588.939079477474	0.103564164336577\\
-583.078989134417	0.176992781905603\\
-577.218898791358	0.24952421539293\\
-571.358808448298	0.320768072493506\\
-565.498718105238	0.390335650935024\\
-559.638627762179	0.457841723775397\\
};
\addplot [color=mycolor2, forget plot]
  table[row sep=crcr]{%
-559.638627762179	0.457841723775397\\
-553.778537419119	0.522906470529288\\
-547.918447076059	0.585157545364451\\
-542.058356733	0.644232268641514\\
-536.19826638994	0.699779923750387\\
-530.33817604688	0.751464136600138\\
-524.47808570382	0.798965311660523\\
-518.617995360761	0.841983094946846\\
-512.757905017701	0.8802388317818\\
-506.897814674641	0.913477985140558\\
-501.037724331585	0.941472479198873\\
-495.177633988525	0.96402293195531\\
-489.317543645466	0.980960741453607\\
-483.457453302406	0.992149990584131\\
-477.597362959346	0.99748913773928\\
-471.737272616287	0.996912462498718\\
-465.877182273227	0.990391238982957\\
-460.017091930167	0.977934612900703\\
-454.157001587108	0.959590162529187\\
-448.296911244048	0.935444128356586\\
-442.436820900988	0.905621300974736\\
-436.576730557928	0.870284561832645\\
-430.716640214869	0.829634076516639\\
-424.856549871809	0.783906145453829\\
-418.996459528749	0.733371721814328\\
-413.136369185693	0.678334611335258\\
-407.276278842633	0.619129373073812\\
-401.416188499574	0.556118944399546\\
-395.556098156514	0.489692017017881\\
-389.696007813454	0.420260194002688\\
-383.835917470395	0.348254960325842\\
-377.975827127335	0.274124501354156\\
-372.115736784275	0.198330404927245\\
-366.255646441216	0.121344283487174\\
-360.395556098156	0.0436443525862737\\
-354.535465755096	-0.0342879983924338\\
-348.675375412036	-0.111971608682172\\
-342.815285068977	-0.188928888795259\\
-336.955194725917	-0.264689072635429\\
-331.095104382861	-0.338791350891377\\
-325.235014039801	-0.41078786043124\\
-319.374923696741	-0.480246507654551\\
-313.514833353682	-0.546753607071468\\
-307.654743010622	-0.609916319920112\\
-301.794652667562	-0.669364881091267\\
-295.934562324503	-0.724754606017306\\
-290.074471981443	-0.775767672645932\\
-284.214381638383	-0.822114676567233\\
-278.354291295323	-0.863535960352843\\
-272.494200952264	-0.899802720626777\\
-266.634110609204	-0.930717898753082\\
-260.774020266144	-0.956116862678604\\
-254.913929923085	-0.975867889088106\\
-249.053839580029	-0.989872456064077\\
-243.193749236969	-0.998065356918254\\
-237.333658893909	-1.00041464654236\\
-231.473568550849	-0.996921431105025\\
-225.61347820779	-0.987619511824232\\
-219.75338786473	-0.972574892848291\\
-213.89329752167	-0.951885162271285\\
-208.033207178611	-0.925678754301471\\
-202.173116835551	-0.894114099366228\\
-196.313026492491	-0.85737866777532\\
-190.452936149431	-0.815687911153543\\
-184.592845806372	-0.769284104693474\\
-178.732755463312	-0.718435092217464\\
-172.872665120252	-0.663432935071271\\
-167.012574777196	-0.604592464941082\\
-161.152484434137	-0.542249740482189\\
-155.292394091077	-0.476760407068243\\
-149.432303748017	-0.408497959180072\\
-143.572213404957	-0.337851905193855\\
-137.712123061898	-0.265225834903588\\
-131.852032718838	-0.19103539109517\\
-125.991942375778	-0.115706147506575\\
-120.131852032719	-0.0396713969892807\\
-114.271761689659	0.0366301447516272\\
-108.411671346599	0.11275671271717\\
-102.551581003539	0.18826593447527\\
-96.6914906604798	0.262717329394353\\
-90.83140031742	0.335674852387291\\
-84.9713099743603	0.406709467386166\\
-79.1112196313043	0.475401734021133\\
-73.2511292882446	0.541344389310977\\
-67.3910389451848	0.604144904758838\\
-61.5309486021251	0.663427998087187\\
-55.6708582590654	0.718838078007769\\
-49.8107679160057	0.770041599858514\\
-43.950677572946	0.816729309733349\\
-38.0905872298863	0.858618355104467\\
-32.2304968868266	0.895454240302984\\
-26.3704065437669	0.927012606394028\\
-20.5103162007072	0.953100816365631\\
-14.6502258576475	0.973559328006949\\
-8.79013551458775	0.988262839354347\\
-2.93004517152804	0.997121193545325\\
2.93004517152804	1.00008003273341\\
8.79013551458775	0.997121193545325\\
14.6502258576475	0.988262839354347\\
20.5103162007072	0.973559328006949\\
26.3704065437669	0.953100816365631\\
32.2304968868266	0.927012606394028\\
38.0905872298863	0.895454240302984\\
43.950677572946	0.858618355104467\\
49.8107679160057	0.816729309733349\\
55.6708582590654	0.770041599858514\\
61.5309486021251	0.718838078007769\\
67.3910389451848	0.663427998087187\\
73.2511292882446	0.604144904758838\\
79.1112196313043	0.541344389310977\\
84.9713099743603	0.475401734021133\\
90.83140031742	0.406709467386166\\
96.6914906604798	0.335674852387291\\
102.551581003539	0.262717329394353\\
108.411671346599	0.18826593447527\\
114.271761689659	0.11275671271717\\
120.131852032719	0.0366301447516272\\
125.991942375778	-0.0396713969892807\\
131.852032718838	-0.115706147506575\\
137.712123061898	-0.19103539109517\\
143.572213404957	-0.265225834903588\\
149.432303748017	-0.337851905193855\\
155.292394091077	-0.408497959180072\\
161.152484434137	-0.476760407068243\\
167.012574777196	-0.542249740482189\\
172.872665120252	-0.604592464941082\\
178.732755463312	-0.663432935071271\\
184.592845806372	-0.718435092217464\\
190.452936149431	-0.769284104693474\\
196.313026492491	-0.815687911153543\\
202.173116835551	-0.85737866777532\\
208.033207178611	-0.894114099366228\\
213.89329752167	-0.925678754301471\\
219.75338786473	-0.951885162271285\\
225.61347820779	-0.972574892848291\\
231.473568550849	-0.987619511824232\\
237.333658893909	-0.996921431105025\\
243.193749236969	-1.00041464654236\\
249.053839580029	-0.998065356918254\\
254.913929923085	-0.989872456064077\\
260.774020266144	-0.975867889088106\\
266.634110609204	-0.956116862678604\\
272.494200952264	-0.930717898753082\\
278.354291295323	-0.899802720626777\\
284.214381638383	-0.863535960352843\\
290.074471981443	-0.822114676567233\\
295.934562324503	-0.775767672645932\\
301.794652667562	-0.724754606017306\\
307.654743010622	-0.669364881091267\\
313.514833353682	-0.609916319920112\\
319.374923696741	-0.546753607071468\\
325.235014039801	-0.480246507654551\\
331.095104382861	-0.41078786043124\\
336.955194725917	-0.338791350891377\\
342.815285068977	-0.264689072635429\\
348.675375412036	-0.188928888795259\\
354.535465755096	-0.111971608682172\\
360.395556098156	-0.0342879983924338\\
366.255646441216	0.0436443525862737\\
372.115736784275	0.121344283487174\\
377.975827127335	0.198330404927245\\
383.835917470395	0.274124501354156\\
389.696007813454	0.348254960325842\\
395.556098156514	0.420260194002688\\
401.416188499574	0.489692017017881\\
407.276278842633	0.556118944399546\\
413.136369185693	0.619129373073812\\
418.996459528749	0.678334611335258\\
424.856549871809	0.733371721814328\\
430.716640214869	0.783906145453829\\
436.576730557928	0.829634076516639\\
442.436820900988	0.870284561832645\\
448.296911244048	0.905621300974736\\
454.157001587108	0.935444128356586\\
460.017091930167	0.959590162529187\\
465.877182273227	0.977934612900703\\
471.737272616287	0.990391238982957\\
477.597362959346	0.996912462498718\\
483.457453302406	0.99748913773928\\
489.317543645466	0.992149990584131\\
495.177633988525	0.980960741453607\\
501.037724331585	0.96402293195531\\
506.897814674641	0.941472479198873\\
512.757905017701	0.913477985140558\\
518.617995360761	0.8802388317818\\
524.47808570382	0.841983094946846\\
530.33817604688	0.798965311660523\\
536.19826638994	0.751464136600138\\
542.058356733	0.699779923750387\\
547.918447076059	0.644232268641514\\
553.778537419119	0.585157545364451\\
559.638627762179	0.522906470529288\\
565.498718105238	0.457841723775397\\
571.358808448298	0.390335650935024\\
577.218898791358	0.320768072493506\\
583.078989134417	0.24952421539293\\
588.939079477474	0.176992781905603\\
594.799169820533	0.103564164336577\\
600.659260163593	0.0296288094771863\\
606.519350506653	-0.0444242682066619\\
612.379440849712	-0.118208830417588\\
618.239531192772	-0.191342626164042\\
624.099621535832	-0.263448484581125\\
629.959711878892	-0.334155307436597\\
635.819802221951	-0.403098982382879\\
641.679892565011	-0.469923240314452\\
647.539982908071	-0.534280481878652\\
653.40007325113	-0.595832598875509\\
659.26016359419	-0.654251816316046\\
665.12025393725	-0.709221579942353\\
670.980344280306	-0.760437512269963\\
676.840434623366	-0.807608457584923\\
682.700524966425	-0.85045763296809\\
688.560615309485	-0.888723898274073\\
694.420705652545	-0.922163153296808\\
700.280795995604	-0.950549865111324\\
706.140886338664	-0.973678722777293\\
712.000976681724	-0.991366410834391\\
717.861067024784	-1.00345348679411\\
723.721157367843	-1.00980634184636\\
729.581247710903	-1.01031921814018\\
735.441338053963	-1.00491625035163\\
741.301428397022	-0.993553494367116\\
747.161518740082	-0.976220901207955\\
753.021609083142	-0.952944191032698\\
758.881699426198	-0.923786578908758\\
764.741789769258	-0.888850302444771\\
770.601880112317	-0.84827790079681\\
776.461970455377	-0.802253194759983\\
782.322060798437	-0.751001919424186\\
788.182151141496	-0.694791963845459\\
794.042241484556	-0.633933176002452\\
799.902331827616	-0.568776696596996\\
805.762422170676	-0.499713791710958\\
811.622512513735	-0.427174161292262\\
817.482602856795	-0.35162370894472\\
823.342693199855	-0.273561767257345\\
829.202783542914	-0.193517782451845\\
835.062873885974	-0.112047472118835\\
840.92296422903	-0.0297284798989344\\
846.78305457209	0.0528444387668077\\
852.64314491515	0.135064655748203\\
858.503235258209	0.216319280481339\\
864.363325601269	0.29599507864572\\
870.223415944329	0.373484628305135\\
876.083506287388	0.448192635362775\\
881.943596630448	0.519542324044213\\
887.803686973508	0.58698181345213\\
893.663777316568	0.649990387873447\\
899.523867659627	0.708084566821275\\
905.383958002687	0.760823880633252\\
911.244048345747	0.807816259181336\\
917.104138688806	0.848722944261556\\
922.964229031862	0.883262841547149\\
928.824319374922	0.91121623424658\\
934.684409717982	0.932427789082648\\
940.544500061042	0.946808794458036\\
946.404590404101	0.954338581889728\\
952.264680747161	0.95506509343728\\
958.124771090221	0.94910457105228\\
963.98486143328	0.936640357141518\\
969.84495177634	0.917920809728259\\
975.7050421194	0.893256349960191\\
981.56513246246	0.863015673755593\\
987.425222805519	0.827621173336601\\
993.285313148579	0.787543627687799\\
999.145403491639	0.743296233500197\\
1005.00549383469	0.695428059563599\\
1010.86558417775	0.644517017647452\\
1016.72567452081	0.591162451709148\\
1022.58576486387	0.535977454104738\\
1028.44585520693	0.479581022814801\\
1034.30594554999	0.422590176891381\\
1040.16603589305	0.365612148604064\\
1046.02612623611	0.309236770140319\\
1051.88621657917	0.254029169754871\\
1057.74630692223	0.200522887640001\\
1063.60639726529	0.149213515023878\\
1069.46648760835	0.100552951453853\\
1075.32657795141	0.0549443651140472\\
1081.18666829447	0.0127379292904945\\
1087.04675863753	-0.0257726048238335\\
1092.90684898059	-0.0603524528168039\\
1098.76693932365	-0.0908274278657117\\
1104.62702966671	-0.117084641646392\\
1110.48712000977	-0.139072274887151\\
1116.34721035283	-0.156798432711866\\
1122.20730069589	-0.170329115104188\\
1128.06739103895	-0.179785347520509\\
1133.927481382	-0.185339530454364\\
1139.78757172506	-0.18721107940182\\
1145.64766206812	-0.185661438070636\\
1151.50775241118	-0.180988557399433\\
1157.36784275424	-0.173520941108077\\
1163.2279330973	-0.163611364615539\\
1169.08802344036	-0.15163037851408\\
1174.94811378342	-0.137959709919137\\
1180.80820412648	-0.122985675223848\\
1186.66829446954	-0.107092715865975\\
1192.5283848126	-0.0906571648315894\\
1198.38847515566	-0.0740413457986789\\
1204.24856549872	-0.0575880992416865\\
1210.10865584178	-0.0416158205683189\\
1215.96874618484	-0.0264140847141357\\
1221.8288365279	-0.0122399196904163\\
1227.68892687096	0.000685221285856634\\
1233.54901721402	0.012177753102854\\
1239.40910755708	0.0220934956158966\\
1245.26919790014	0.030328446479454\\
1251.1292882432	0.0368187965883548\\
1256.98937858626	0.0415402086416198\\
1262.84946892931	0.0445063926070338\\
1268.70955927237	0.0457670242753695\\
1274.56964961543	0.0454050644083306\\
1280.42973995849	0.0435335460266654\\
1286.28983030155	0.0402919059350923\\
1292.14992064461	0.0358419435443458\\
1298.01001098767	0.0303634952677014\\
1303.87010133073	0.0240499161933233\\
1309.73019167379	0.0171034623045789\\
1315.59028201685	0.0097306662554505\\
1321.45037235991	0.00213779761317895\\
1327.31046270297	-0.00547350534665286\\
1333.17055304603	-0.0129103506877623\\
1339.03064338909	-0.0199923800795998\\
1344.89073373214	-0.0265551619885406\\
1350.7508240752	-0.032453133982438\\
1356.61091441826	-0.0375620477701387\\
1362.47100476132	-0.0417808817478559\\
1368.33109510438	-0.0450331974637927\\
1374.19118544744	-0.0472679282845822\\
1380.0512757905	-0.048459600428606\\
1385.91136613356	-0.0486079981452268\\
1391.77145647662	-0.0477372959797141\\
1397.63154681968	-0.0458946914965311\\
1403.49163716274	-0.0431485813924892\\
1409.3517275058	-0.0395863323882404\\
1415.21181784886	-0.0353117055066393\\
1421.07190819192	-0.0304419982165962\\
1426.93199853498	-0.0251049733031036\\
1432.79208887804	-0.019435646200381\\
1438.6521792211	-0.013573003818232\\
1444.51226956415	-0.00765672763855815\\
1450.37235990721	-0.00182399207114651\\
1456.23245025027	0.00379359419896103\\
1462.09254059333	0.00907284075713785\\
1467.95263093639	0.0139015646557745\\
1473.81272127945	0.0181808659890328\\
1479.67281162251	0.0218270474814445\\
1485.53290196557	0.0247731443769706\\
1491.39299230863	0.0269700401290352\\
1497.25308265169	0.028387152936888\\
1503.11317299475	0.0290126878692258\\
1508.97326333781	0.0288534589218024\\
1514.83335368087	0.027934294692799\\
1520.69344402393	0.0262970502124335\\
1526.55353436699	0.0239992556573947\\
1532.41362471005	0.0211124400539121\\
1538.27371505311	0.0177201744695551\\
1544.13380539617	0.013915884500524\\
1549.99389573923	0.00980048597372192\\
1555.85398608229	0.00547990064286727\\
1561.71407642535	0.00106251020646209\\
1567.57416676841	-0.00334339277894496\\
1573.43425711146	-0.00763209957447714\\
1579.29434745452	-0.0117028703150683\\
1585.15443779758	-0.0154621577261232\\
1591.01452814064	-0.0188256223530538\\
1596.8746184837	-0.0217198976977746\\
1602.73470882676	-0.0240840700213118\\
1608.59479916982	-0.0258708446201176\\
1614.45488951288	-0.0270473779280049\\
1620.31497985594	-0.0275957626874273\\
1626.175070199	-0.0275131614658803\\
1632.03516054206	-0.0268115918039416\\
1637.89525088512	-0.0255173740877348\\
1643.75534122818	-0.0236702606628685\\
1649.61543157124	-0.0213222716012969\\
1655.4755219143	-0.0185362687434548\\
1661.33561225736	-0.0153843050391008\\
1667.19570260042	-0.0119457906926107\\
1673.05579294348	-0.00830552109696161\\
1678.91588328653	-0.00455161395270145\\
1684.77597362959	-0.000773404269295005\\
1690.63606397265	0.00294065386667799\\
1696.49615431571	0.00650503079562319\\
1702.35624465877	0.0098390640935434\\
1708.21633500183	0.0128687577205047\\
1714.07642534489	0.0155283984775505\\
1719.93651568795	0.017761955499055\\
1725.79660603101	0.0195242339738226\\
1731.65669637407	0.0207817603317231\\
1737.51678671713	0.0215133826033089\\
1743.37687706019	0.0217105764099564\\
1749.23696740325	0.0213774539147755\\
1755.09705774631	0.0205304799005163\\
1760.95714808936	0.019197905785455\\
1766.81723843242	0.0174189387001174\\
1772.67732877548	0.0152426685735685\\
1778.53741911854	0.0127267814135153\\
1784.3975094616	0.00993609147400074\\
1790.25759980466	0.0069409287138406\\
1796.11769014772	0.00381542077874518\\
1801.97778049078	0.000635710629293177\\
1807.83787083384	-0.0025218481337946\\
1813.6979611769	-0.00558247624155257\\
1819.55805151996	-0.00847469302921748\\
1825.41814186302	-0.0111319409451808\\
1831.27823220608	-0.0134940740375226\\
1837.13832254914	-0.015508677757939\\
1842.9984128922	-0.017132191727909\\
1848.85850323526	-0.0183308120121746\\
1854.71859357832	-0.0190811548232647\\
1860.57868392138	-0.019370669308774\\
1866.43877426444	-0.0191977930156998\\
1872.2988646075	-0.0185718496407358\\
1878.15895495056	-0.0175126946320237\\
1884.01904529361	-0.0160501199614507\\
1889.87913563667	-0.0142230348116931\\
1895.73922597973	-0.0120784439005949\\
1901.59931632279	-0.00967024958476721\\
1907.45940666585	-0.00705790764695253\\
1913.31949700891	-0.00430496970760822\\
1919.17958735197	-0.00147754743827254\\
1925.03967769503	0.00135726484147846\\
1930.89976803809	0.00413297206022388\\
1936.75985838115	0.00678522797837604\\
1942.61994872421	0.00925331681943426\\
1948.48003906727	0.0114815332838823\\
1954.34012941033	0.0134204300100195\\
1960.20021975339	0.0150279049613457\\
1966.06031009645	0.0162701052353912\\
1971.92040043951	0.0171221282966627\\
1977.78049078257	0.0175685065489551\\
1983.64058112563	0.017603466330637\\
1989.50067146869	0.01723095775178\\
1995.36076181175	0.0164644571434375\\
2001.22085215481	0.0153265491510133\\
2007.08094249787	0.0138483005413317\\
2012.94103284092	0.0120684425051182\\
2018.80112318398	0.0100323825085114\\
2024.66121352704	0.00779107048717147\\
2030.5213038701	0.00539974730823386\\
2036.38139421316	0.0029166058741285\\
2042.24148455622	0.000401396964019029\\
2048.10157489928	-0.00208598712897101\\
2053.96166524234	-0.00448691792266446\\
2059.8217555854	-0.00674539237829642\\
2065.68184592846	-0.00880931952503118\\
2071.54193627152	-0.0106317028143412\\
2077.40202661458	-0.0121716916531481\\
2083.26211695764	-0.013395478866648\\
2089.1222073007	-0.0142770246534892\\
2094.98229764376	-0.0147985918175227\\
2100.84238798681	-0.0149510816021221\\
2106.70247832987	-0.0147341641942396\\
2112.56256867293	-0.0141562028041475\\
2118.42265901599	-0.0132339750428316\\
2124.28274935905	-0.0119922000052796\\
2130.14283970211	-0.010462883917712\\
2136.00293004517	-0.00868450131249065\\
2141.86302038823	-0.00670103237302143\\
2147.72311073129	-0.00456088025260645\\
2153.58320107435	-0.00231569475419591\\
2159.44329141741	-1.91307012518548e-05\\
2165.30338176047	0.0022744294019185\\
2171.16347210353	0.00451115825297398\\
2177.02356244659	0.00663904214923177\\
2182.88365278965	0.00860908570863427\\
2188.74374313271	0.0103764352298235\\
2194.60383347576	0.0119013951481239\\
2200.46392381882	0.0131503147588297\\
2206.32401416188	0.0140963256054732\\
2212.18410450494	0.0147199135818754\\
2218.044194848	0.0150093137914393\\
2223.90428519106	0.014960720446091\\
2229.76437553412	0.0145783084693991\\
2235.62446587718	0.0138740678917618\\
2241.48455622024	0.0128674564878283\\
2247.3446465633	0.0115848803034256\\
2253.20473690636	0.0100590156610343\\
2259.06482724942	0.00832798981830225\\
2264.92491759248	0.00643444061830099\\
2270.78500793554	0.00442447812832633\\
2276.6450982786	0.00234657336730572\\
2282.50518862166	0.00025040072141383\\
2288.36527896472	-0.00181433848336023\\
2294.22536930778	-0.00379908356424504\\
2300.08545965084	-0.00565754595920392\\
2305.9455499939	-0.00734677839859645\\
2311.80564033696	-0.00882815723345431\\
2317.66573068002	-0.0100682556059625\\
2323.52582102307	-0.0110395878732599\\
2329.38591136613	-0.0117212088592692\\
2335.24600170919	-0.012099155029404\\
2341.10609205225	-0.0121667184788795\\
2346.96618239531	-0.0119245486059466\\
2352.82627273837	-0.0113805804089319\\
2358.68636308143	-0.0105497924085143\\
2364.54645342449	-0.00945380116082388\\
2370.40654376755	-0.00812030309476438\\
2376.26663411061	-0.00658237789869062\\
2382.12672445367	-0.00487767081481468\\
2387.98681479673	-0.00304747390285287\\
2393.84690513979	-0.00113572855030081\\
2399.70699548285	0.000812026814088753\\
2405.56708582591	0.0027497387414227\\
2411.42717616897	0.00463192494249867\\
2417.28726651203	0.00641473913918165\\
2423.14735685509	0.00805698978402546\\
2429.00744719815	0.00952108910289104\\
2434.8675375412	0.0107739107154888\\
2440.72762788426	0.0117875363843739\\
2446.58771822732	0.0125398751761778\\
2452.44780857038	0.0130151414254016\\
2458.30789891344	0.013204181292042\\
2464.1679892565	0.0131046413218598\\
2470.02807959956	0.0127209761646408\\
2475.88816994262	0.0120642963894758\\
2481.74826028568	0.0111520610785373\\
2487.60835062874	0.0100076234812093\\
2493.4684409718	0.00865964139479158\\
2499.32853131486	0.0071413670297236\\
2505.18862165792	0.00548983383634214\\
2511.04871200098	0.00374496006777452\\
2516.90880234403	0.00194859066912146\\
2522.76889268709	0.000143500384176332\\
2528.62898303015	-0.00162761827576789\\
2534.48907337321	-0.00332315835780078\\
2540.34916371627	-0.00490356960841335\\
2546.20925405933	-0.00633227485948068\\
2552.06934440239	-0.00757651052092272\\
2557.92943474545	-0.00860807195946099\\
2563.78952508851	-0.00940394689773573\\
2569.64961543157	-0.00994682270303402\\
2575.50970577463	-0.0102254564807398\\
2581.36979611769	-0.010234900178849\\
2587.22988646075	-0.00997657635428583\\
2593.08997680381	-0.00945820378437724\\
2598.95006714687	-0.00869357562946508\\
2604.81015748993	-0.00770219629457858\\
2610.67024783299	-0.00650878641024377\\
2616.53033817605	-0.00514266838524258\\
2622.39042851911	-0.00363704770804635\\
2628.25051886217	-0.00202820751994197\\
2634.11060920523	-0.000354635910199807\\
2639.97069954828	0.00134389316210531\\
2645.83078989134	0.00302726345532345\\
2651.6908802344	0.00465596223862386\\
2657.55097057746	0.00619200607336325\\
2663.41106092052	0.00759982497022164\\
2669.27115126358	0.00884708441849526\\
2675.13124160664	0.00990542636467931\\
2680.9913319497	0.0107511122353016\\
2686.85142229276	0.0113655534972052\\
2692.71151263582	0.0117357179755324\\
2698.57160297888	0.0118544031335403\\
2704.43169332194	0.0117203706928083\\
2710.291783665	0.0113383402598657\\
2716.15187400806	0.0107188429453168\\
2722.01196435112	0.00987793924262647\\
2727.87205469418	0.0088368085913962\\
2733.73214503724	0.0076212210206767\\
2739.5922353803	0.00626090397022931\\
2745.45232572336	0.00478881977361752\\
2751.31241606642	0.00324037129351002\\
2757.17250640948	0.00165255478330148\\
2763.03259675254	6.30801788299924e-05\\
2768.89268709559	-0.00149052032857914\\
2774.75277743865	-0.00297177442992492\\
2780.61286778171	-0.00434612187831032\\
2786.47295812477	-0.00558171686171607\\
2792.33304846783	-0.00665016200192275\\
2798.19313881089	-0.00752715712211438\\
2804.05322915395	-0.00819304802201049\\
2809.91331949701	-0.00863326292849645\\
2815.77340984007	-0.00883862699654855\\
2821.63350018313	-0.00880554815098393\\
2827.49359052619	-0.00853607061840782\\
2833.35368086925	-0.00803779562351115\\
2839.21377121231	-0.00732367184756998\\
2845.07386155537	-0.00641166128931641\\
2850.93395189842	-0.00532428906249935\\
2856.79404224148	-0.00408808834539056\\
2862.65413258454	-0.00273295409464437\\
2868.5142229276	-0.00129142120849967\\
2874.37431327066	0.00020211548993306\\
2880.23440361372	0.00171222080351519\\
2886.09449395678	0.00320325877352022\\
2891.95458429984	0.00464023251354548\\
2897.8146746429	0.0059896045260749\\
2903.67476498596	0.00722007832548872\\
2909.53485532902	0.00830332320286377\\
2915.39494567208	0.00921462539855497\\
2921.25503601514	0.00993345076560541\\
2927.1151263582	0.0104439061639191\\
2932.97521670126	0.0107350892715334\\
2938.83530704432	0.0108013191729026\\
2944.69539738738	0.0106422429257624\\
2950.55548773043	0.0102628162442474\\
2956.41557807349	0.00967315940618472\\
2962.27566841655	0.00888829241280049\\
2968.13575875961	0.00792775625113928\\
2973.99584910267	0.00681512974368829\\
2979.85593944573	0.0055774538797864\\
2985.71602978879	0.00424457763248576\\
2991.57612013185	0.00284844103973382\\
2997.43621047491	0.00142231272316089\\
nan	nan\\
3009.15639116103	-0.00138494970234729\\
3015.01648150409	-0.00270003730960112\\
3020.87657184715	-0.00391457195439474\\
3026.73666219021	-0.0050003844974638\\
3032.59675253327	-0.00593247817007214\\
3038.45684287633	-0.00668960135077654\\
3044.31693321939	-0.00725472939405607\\
3050.17702356245	-0.00761544462933128\\
3056.03711390551	-0.007764206095785\\
3061.89720424857	-0.00769850321069726\\
3067.75729459163	-0.0074208903251861\\
3073.61738493469	-0.00693890193505245\\
3079.47747527774	-0.0062648511204916\\
3085.3375656208	-0.00541551652062163\\
3091.19765596386	-0.00441172574216919\\
3097.05774630692	-0.00327784549821102\\
3102.91783664998	-0.00204119091566028\\
3108.77792699304	-0.00073136829003325\\
3114.6380173361	0.00062043293757984\\
3120.49810767916	0.00198218208787227\\
3126.35819802222	0.00332176542560803\\
3132.21828836528	0.00460774270801342\\
3138.07837870834	0.00581008431329038\\
3143.9384690514	0.00690087170401437\\
3149.79855939446	0.00785494492196435\\
3155.65864973752	0.00865048213370397\\
3161.51874008058	0.00926949791224047\\
3167.37883042364	0.00969824891316643\\
3173.2389207667	0.00992753783331191\\
3179.09901110976	0.00995290897301492\\
3184.95910145281	0.00977473130189391\\
3190.81919179587	0.00939816759370712\\
3196.67928213893	0.00883303087731186\\
3202.53937248199	0.00809353209863417\\
3208.39946282505	0.00719792542440139\\
3214.25955316811	0.00616805999901185\\
3220.11964351117	0.00502884912276843\\
3225.97973385423	0.00380766971341746\\
3231.83982419729	0.00253370649345193\\
3237.69991454035	0.0012372565793532\\
3243.56000488341	-5.09889885340821e-05\\
3249.42009522647	-0.00130066977059606\\
3255.28018556953	-0.0024824694344331\\
3261.14027591259	-0.00356880150643534\\
3267.00036625564	-0.00453445135847831\\
3272.8604565987	-0.00535715951998051\\
3278.72054694176	-0.00601813285091013\\
3284.58063728482	-0.00650247186787117\\
3290.44072762788	-0.00679950453741863\\
3296.30081797094	-0.00690301909406241\\
3302.160908314	-0.00681139084674552\\
3308.02099865706	-0.00652760045285251\\
3313.88108900012	-0.00605914370243496\\
3319.74117934318	-0.00541783540592309\\
3325.60126968624	-0.00461951245901207\\
3331.4613600293	-0.00368364350610682\\
3337.32145037236	-0.00263285479000728\\
3343.18154071542	-0.00149238370437206\\
3349.04163105848	-0.000289473216962629\\
3354.90172140154	0.000947278333604621\\
3360.7618117446	0.00218859657776847\\
3366.62190208766	0.00340522348184922\\
3372.48199243072	0.00456860639406005\\
3378.34208277378	0.00565156756964951\\
3384.20217311684	0.00662893850202409\\
3390.06226345989	0.00747814428085077\\
3395.92235380295	0.00817972443606576\\
3401.78244414601	0.00871777827683565\\
3407.64253448907	0.00908032456041693\\
3413.50262483213	0.00925956738492283\\
3419.36271517519	0.00925206243771255\\
3425.22280551825	0.00905878010104059\\
3431.08289586131	0.00868506435612603\\
3436.94298620437	0.00814048888322185\\
3442.80307654743	0.0074386141681885\\
3448.66316689049	0.00659665173717525\\
3454.52325723355	0.00563504380392908\\
3460.38334757661	0.00457696856755867\\
3466.24343791967	0.00344778310696529\\
3472.10352826273	0.00227441723806101\\
3477.96361860579	0.00108473279853733\\
3483.82370894885	-9.3136419914711e-05\\
3489.68379929191	-0.00123144847820071\\
3495.54388963497	-0.00230350519651384\\
3501.40397997803	-0.00328427686532124\\
3507.26407032109	-0.0041509851657094\\
3513.12416066415	-0.00488363074642171\\
3518.9842510072	-0.00546545326007842\\
3524.84434135026	-0.00588331330029163\\
3530.70443169332	-0.00612798755876339\\
3536.56452203638	-0.00619437060029849\\
3542.42461237944	-0.00608157887400178\\
3548.2847027225	-0.00579295489945733\\
3554.14479306556	-0.00533597192109993\\
3560.00488340862	-0.00472204166663773\\
3565.86497375168	-0.00396623011362101\\
3571.72506409474	-0.00308688831490519\\
3577.5851544378	-0.00210520730384014\\
3583.44524478086	-0.00104470785231902\\
3589.30533512392	6.93226548304334e-05\\
3595.16542546698	0.00121043276809243\\
3601.02551581004	0.00235163460776858\\
3606.88560615309	0.00346604171332176\\
3612.74569649615	0.00452750205278483\\
3618.60578683921	0.00551121130918611\\
3624.46587718227	0.00639429208248266\\
3630.32596752533	0.00715632550136256\\
3636.18605786839	0.00777982291194471\\
3642.04614821145	0.00825062676708455\\
3647.90623855451	0.00855823154619699\\
3653.76632889757	0.00869601745434875\\
3659.62641924063	0.00866139172788757\\
3665.48650958369	0.00845583456773837\\
3671.34659992675	0.00808484897791946\\
3677.20669026981	0.00755781605084114\\
3683.06678061287	0.00688775946002173\\
3688.92687095593	0.00609102504420597\\
3694.78696129899	0.00518688334262058\\
3700.64705164204	0.00419706472580414\\
3706.5071419851	0.00314523831719472\\
3712.36723232816	0.00205644718068208\\
3718.22732267122	0.000956513234835723\\
3724.08741301428	-0.000128573984568956\\
3729.94750335734	-0.00117327026290317\\
3735.8075937004	-0.00215307690828768\\
3741.66768404346	-0.00304511413168779\\
3747.52777438652	-0.00382865420313579\\
3753.38786472958	-0.00448560197798168\\
3759.24795507264	-0.00500091166995849\\
3765.1080454157	-0.00536293029028585\\
3770.96813575876	-0.00556365993117846\\
3776.82822610182	-0.00559893301236882\\
3782.68831644488	-0.00546849667736869\\
3788.54840678794	-0.0051760046804829\\
3794.408497131	-0.00472891728670607\\
3800.26858747406	-0.00413831187208375\\
3806.12867781712	-0.00341860900343866\\
3811.98876816018	-0.00258722074885845\\
3817.84885850324	-0.00166412977740159\\
3823.7089488463	-0.00067140940488208\\
3829.56903918935	0.000367303903315773\\
3835.42912953241	0.00142737298294409\\
3841.28921987547	0.00248374400730378\\
3847.14931021853	0.00351153828073021\\
3853.00940056159	0.00448663782634899\\
3858.86949090465	0.00538625100051817\\
3864.72958124771	0.00618944488518528\\
3870.58967159077	0.00687763203637733\\
3876.44976193383	0.00743500028746927\\
3882.30985227689	0.00784887568288021\\
3888.16994261995	0.00811001022850894\\
3894.03003296301	0.00821278794353416\\
3899.89012330607	0.00815534464361101\\
3905.75021364913	0.00793959893294186\\
3911.61030399219	0.00757119398092691\\
3917.47039433525	0.00705935176191942\\
3923.33048467831	0.00641664349010055\\
3929.19057502137	0.00565868194429951\\
3935.05066536443	0.00480374319374668\\
3940.91075570748	0.00387232687513467\\
3946.77084605054	0.00288666558511696\\
3952.6309363936	0.00187019511530769\\
3958.49102673666	0.000846998138636516\\
3964.35111707972	-0.000158765461476173\\
3970.21120742278	-0.0011234281582107\\
3976.07129776584	-0.00202437083091371\\
3981.9313881089	-0.00284055224740289\\
3987.79147845196	-0.00355299959823261\\
3993.65156879502	-0.00414524866070209\\
3999.51165913808	-0.00460372339460904\\
4005.37174948114	-0.00491804623215694\\
4011.2318398242	-0.00508127198444811\\
4017.09193016726	-0.00509004011016201\\
4022.95202051031	-0.00494464203082911\\
4028.81211085337	-0.00464900218843632\\
4034.67220119643	-0.00421057357706133\\
4040.53229153949	-0.00364015049060143\\
4046.39238188255	-0.00295160316964667\\
4052.25247222561	-0.00216154085385285\\
4058.11256256867	-0.00128891141044684\\
4063.97265291173	-0.000354547176565682\\
4069.83274325479	0.000619332112477259\\
4075.69283359785	0.00160964746120418\\
4081.55292394091	0.00259300727804757\\
4087.41301428397	0.00354625962961642\\
4093.27310462703	0.00444703705929061\\
4099.13319497009	0.00527428116240368\\
4104.99328531315	0.00600873462693156\\
4110.85337565621	0.00663338925592373\\
4116.71346599927	0.00713387956008516\\
4122.57355634233	0.00749881282498909\\
4128.43364668539	0.0077200280816226\\
4134.29373702845	0.00779277810868319\\
4140.15382737151	0.00771583042410058\\
4146.01391771456	0.00749148514694538\\
4151.87400805762	0.00712550957154266\\
4157.73409840068	0.00662699126065033\\
4163.59418874374	0.00600811337621788\\
4169.4542790868	0.00528385778461535\\
4175.31436942986	0.00447164315926095\\
4181.17445977292	0.00359090680916101\\
4187.03455011598	0.00266264026045302\\
4192.89464045904	0.00170888967644035\\
4198.7547308021	0.000752232993288131\\
4204.61482114516	-0.000184753838597746\\
4210.47491148822	-0.00108002890107251\\
4216.33500183128	-0.00191260211466732\\
4222.19509217434	-0.00266302665218088\\
4228.0551825174	-0.0033138524653146\\
4233.91527286046	-0.00385003135867849\\
4239.77536320352	-0.00425926421897166\\
4245.63545354658	-0.00453228239845683\\
4251.49554388964	-0.00466305682746218\\
4257.3556342327	-0.00464893015194464\\
4263.21572457576	-0.00449066902062229\\
4269.07581491882	-0.00419243553368495\\
4274.93590526187	-0.00376167877409589\\
4280.79599560493	-0.00320894922013131\\
4286.65608594799	-0.00254764064671001\\
4292.51617629105	-0.00179366581587415\\
4298.37626663411	-0.000965073798040115\\
4304.23635697717	-8.16181154118465e-05\\
4310.09644732023	0.000835713969562203\\
4315.95653766329	0.0017651998147762\\
4321.81662800635	0.00268489558418496\\
4327.67671834941	0.00357315409801373\\
4333.53680869247	0.0044091341367855\\
4339.39689903553	0.00517328922701587\\
4345.25698937859	0.00584782445449976\\
4351.11707972165	0.00641711063969909\\
4356.97717006471	0.00686804624474679\\
4362.83726040776	0.00719035864244292\\
4368.69735075082	0.00737683782939752\\
4374.55744109388	0.00742349727841564\\
4380.41753143694	0.00732965835699008\\
4386.27762178	0.0070979565481617\\
4392.13771212306	0.00673426955943134\\
4397.99780246612	0.00624756924310367\\
4403.85789280918	0.00564970104074225\\
4409.71798315224	0.00495509636141761\\
4415.5780734953	0.00418042486645277\\
4421.43816383836	0.00334419502831241\\
4427.29825418142	0.00246631252594408\\
4433.15834452448	0.00156760700324504\\
4439.01843486754	0.000669338432613838\\
4444.8785252106	-0.000207305227401601\\
4450.73861555366	-0.00104170721982999\\
4456.59870589671	-0.0018143062706842\\
4462.45879623977	-0.00250705459753157\\
4468.31888658283	-0.00310383893029356\\
4474.17897692589	-0.00359085473168586\\
4480.03906726895	-0.00395692493507415\\
4485.89915761201	-0.00419375584983331\\
4491.75924795507	-0.00429612438516661\\
4497.61933829813	-0.00426199237767411\\
4503.47942864119	-0.00409254553806755\\
4509.33951898425	-0.0037921563139937\\
4515.19960932731	-0.00336827176109309\\
4521.05969967037	-0.00283122927761571\\
4526.91979001343	-0.00219400474851233\\
4532.77988035649	-0.00147189922510477\\
4538.63997069955	-0.000682171696352505\\
4544.50006104261	0.000156373243617816\\
4550.36015138567	0.00102383298922104\\
4556.22024172873	0.00189967913315761\\
4562.08033207179	0.00276324309694081\\
4567.94042241485	0.00359420370226787\\
4573.80051275791	0.00437306518987947\\
4579.66060310097	0.00508161444939937\\
4585.52069344402	0.00570334674537006\\
4591.38078378708	0.00622384999610994\\
4597.24087413014	0.00663113866645557\\
4603.1009644732	0.00691592954726432\\
4608.96105481626	0.00707185308507725\\
4614.82114515932	0.00709559546077547\\
4620.68123550238	0.00698696826112283\\
4626.54132584544	0.00674890430226135\\
4632.4014161885	0.00638737990567538\\
4638.26150653156	0.00591126566192369\\
4644.12159687462	0.00533210939437569\\
4649.98168721768	0.00466385662267876\\
4655.84177756074	0.00392251528471928\\
4661.7018679038	0.00312577276856715\\
4667.56195824686	0.00229257440960022\\
4673.42204858992	0.00144267348875718\\
4679.28213893298	0.000596163413792659\\
4685.14222927604	-0.000226996845361649\\
4691.0023196191	-0.00100745286604061\\
4696.86240996215	-0.0017269093620529\\
4702.72250030521	-0.00236855858275609\\
4708.58259064827	-0.0029174724982079\\
4714.44268099133	-0.00336094962661905\\
4720.30277133439	-0.00368880845362635\\
4726.16286167745	-0.00389362067287524\\
4732.02295202051	-0.003970878913277\\
4737.88304236357	-0.00391909517608863\\
4743.74313270663	-0.00373982784746673\\
4749.60322304969	-0.00343763684092504\\
4755.46331339275	-0.00301996811788801\\
4761.32340373581	-0.00249697049601303\\
4767.18349407887	-0.00188124924158634\\
4773.04358442193	-0.00118756242026539\\
4778.90367476498	-0.000432467313435233\\
4784.76376510804	0.000366074636237159\\
4790.6238554511	0.00118912493071111\\
4796.48394579416	0.00201721702516986\\
4802.34403613722	0.00283081660068235\\
4808.20412648028	0.00361078222517441\\
4814.06421682334	0.00433881554615089\\
4819.9243071664	0.004997890433962\\
4825.78439750946	0.00557265101788421\\
4831.64448785252	0.00604976931606611\\
4837.50457819558	0.00641825413822676\\
4843.36466853864	0.00666970410702392\\
4849.2247588817	0.00679849898248883\\
4855.08484922476	0.00680192493989265\\
4860.94493956782	0.0066802310206236\\
4866.80502991088	0.00643661560344464\\
4872.66512025394	0.00607714339672218\\
4878.525210597	0.00561059508553304\\
4884.38530094006	0.00504825335207591\\
4890.24539128312	0.00440363047408418\\
4896.10548162617	0.00369214407177429\\
4901.96557196923	0.00293074877724391\\
4907.82566231229	0.0021375326189662\\
4913.68575265535	0.00133128772401368\\
4919.54584299841	0.000531065519587303\\
4925.40593334147	-0.000244273043748826\\
4931.26602368453	-0.00097650163525037\\
4937.12611402759	-0.00164845660097083\\
4942.98620437065	-0.00224443888751755\\
4948.84629471371	-0.00275058042713254\\
4954.70638505677	-0.00315516643918719\\
4960.56647539983	-0.00344890616422852\\
4966.42656574289	-0.00362514578034095\\
4972.28665608595	-0.00368001862940849\\
4978.14674642901	-0.00361252937091565\\
4984.00683677207	-0.00342457024591402\\
4989.86692711513	-0.00312086923776121\\
4995.72701745819	-0.00270887152233677\\
5001.58710780125	-0.00219855716803276\\
5007.44719814431	-0.00160219954101694\\
5013.30728848737	-0.00093407025742763\\
5019.16737883043	-0.000210097768708227\\
5025.02746917348	0.000552512258170917\\
5030.88755951654	0.00133568572432813\\
5036.7476498596	0.00212090834009687\\
5042.60774020266	0.00288966318301387\\
5048.46783054572	0.00362386723746691\\
5054.32792088878	0.00430629663020967\\
5060.18801123184	0.00492099056791236\\
5066.0481015749	0.00545362450936383\\
5071.90819191796	0.00589184385245099\\
5077.76828226102	0.00622555036950828\\
5083.62837260408	0.00644713475472385\\
5089.48846294714	0.00655164993633903\\
5095.3485532902	0.0065369212118717\\
5101.20864363326	0.00640359076642127\\
5107.06873397632	0.00615509568318416\\
5112.92882431937	0.00579758012743732\\
5118.78891466243	0.00533974393228034\\
5124.64900500549	0.00479263130979069\\
5130.50909534855	0.00416936481180091\\
5136.36918569161	0.00348483094289926\\
5142.22927603467	0.00275532495227472\\
5148.08936637773	0.00199816327395481\\
5153.94945672079	0.00123127282810813\\
5159.80954706385	0.000472766918436309\\
5165.66963740691	-0.00025948224664403\\
5171.52972774997	-0.000948264325738051\\
5177.38981809303	-0.00157743500799598\\
5183.24990843609	-0.00213229408257217\\
5189.10999877915	-0.0025999285654072\\
5194.97008912221	-0.00296951287963104\\
5200.83017946527	-0.00323255911665472\\
5206.69026980832	-0.0033831115981668\\
5212.55036015138	-0.00341788128565944\\
5218.41045049444	-0.00333631701190556\\
5224.2705408375	-0.00314061200521067\\
5230.13063118056	-0.00283564570531639\\
5235.99072152362	-0.00242886239529102\\
5241.85081186668	-0.00193008965858885\\
5247.71090220974	-0.00135130108202774\\
5253.5709925528	-0.000706328928454045\\
5259.43108289586	-1.05336688027332e-05\\
5265.29117323892	0.000719561736366034\\
5271.15126358198	0.0014666641034942\\
5277.01135392504	0.00221311953307835\\
5282.8714442681	0.00294133046499977\\
5288.73153461116	0.00363417043458561\\
5294.59162495422	0.0042753867587451\\
5300.45171529728	0.00484998168923948\\
5306.31180564034	0.00534456309915285\\
5312.1718959834	0.00574765650713192\\
5318.03198632646	0.00604997117492418\\
5323.89207666952	0.00624461411186288\\
5329.75216701258	0.00632724706264815\\
5335.61225735563	0.00629618290975367\\
5341.47234769869	0.0061524193568112\\
5347.33243804175	0.00589960924439551\\
5353.19252838481	0.00554396834214876\\
5359.05261872787	0.0050941229348565\\
5364.91270907093	0.00456090093196473\\
5370.77279941399	0.00395707155384661\\
5376.63288975705	0.0032970398478728\\
5382.49298010011	0.00259650333742868\\
5388.35307044317	0.00187207898338601\\
5394.21316078623	0.00114090931835258\\
5400.07325112929	0.000420257084672758\\
5405.93334147235	-0.000272902043125599\\
5411.79343181541	-0.000922279046914138\\
5417.65352215847	-0.0015126540634109\\
5423.51361250153	-0.00203023281513701\\
5429.37370284459	-0.00246296863114866\\
5435.23379318765	-0.00280084254465944\\
5441.09388353071	-0.00303609495944454\\
5446.95397387376	-0.00316340353342987\\
5452.81406421682	-0.00318000320655496\\
5458.67415455988	-0.00308574567399403\\
5464.53424490294	-0.00288309703758909\\
5470.394335246	-0.00257707382942487\\
5476.25442558906	-0.00217511905284908\\
5482.11451593212	-0.00168692129635067\\
5487.97460627518	-0.00112418131098558\\
5493.83469661824	-0.000500331670064663\\
5499.6947869613	0.000169783776807478\\
5505.55487730436	0.000870263010159422\\
5511.41496764742	0.00158452311235249\\
5517.27505799048	0.00229569275805176\\
5523.13514833354	0.00298701063816723\\
5528.9952386766	0.00364222041250649\\
5534.85532901965	0.00424595288912246\\
5540.71541936271	0.00478408644935242\\
5546.57550970577	0.00524407727078112\\
5552.43560004883	0.00561525162976472\\
5558.29569039189	0.00588905347689357\\
5564.15578073495	0.00605924154656835\\
5570.01587107801	0.00612203146568702\\
5575.87596142107	0.00607617962991959\\
5581.73605176413	0.00592300699925457\\
5587.59614210719	0.00566636238044875\\
5593.45623245025	0.00531252619536609\\
5599.31632279331	0.00487005713112554\\
5605.17641313637	0.00434958541164809\\
5611.03650347943	0.00376355767896498\\
5616.89659382249	0.00312593960227707\\
5622.75668416555	0.00245188331715586\\
5628.61677450861	0.00175736760918476\\
5634.47686485167	0.00105881938305717\\
5640.33695519473	0.000372725380382555\\
5646.19704553779	-0.000284756680071274\\
5652.05713588084	-0.000898178371887775\\
5657.9172262239	-0.00145316329447726\\
5663.77731656696	-0.00193674375738302\\
5669.63740691002	-0.00233766353442861\\
5675.49749725308	-0.002646639623084\\
5681.35758759614	-0.00285657692616216\\
5687.2176779392	-0.00296273089486637\\
5693.07776828226	-0.00296281441032375\\
5698.93785862532	-0.00285704650173869\\
5704.79794896838	-0.00264814187668898\\
5710.65803931144	-0.00234124163505816\\
5716.5181296545	-0.00194378692458454\\
5722.37821999756	-0.00146533863688936\\
5728.23831034062	-0.000917347507992998\\
5734.09840068368	-0.000312880147372273\\
5739.95849102674	0.000333692453135872\\
5745.8185813698	0.00100703650901828\\
5751.67867171286	0.00169121901555685\\
5757.53876205592	0.00237008468625006\\
5763.39885239898	0.00302763734357482\\
5769.25894274204	0.00364841676164796\\
5775.11903308509	0.00421786208702458\\
5780.97912342815	0.00472265329803579\\
5786.83921377121	0.00515102269908503\\
5792.69930411427	0.00549302916919449\\
5798.55939445733	0.00574078877660408\\
5804.41948480039	0.00588865641368789\\
5810.27957514345	0.00593335427217524\\
5816.13966548651	0.00587404423898748\\
5821.99975582957	0.00571234262291438\\
5827.85984617263	0.00545227698349191\\
5833.71993651569	0.00510018619935562\\
5839.58002685875	0.00466456624824702\\
5845.44011720181	0.00415586544681112\\
5851.30020754487	0.00358623408019082\\
5857.16029788793	0.00296923441796199\\
5863.02038823099	0.00231951803323882\\
5868.88047857404	0.00165247809928863\\
5874.7405689171	0.000983884911402183\\
5880.60065926016	0.000329513260952879\\
5886.46074960322	-0.000295229537481451\\
5892.32083994628	-0.000875666198067985\\
5898.18093028934	-0.0013981940668296\\
5904.0410206324	-0.00185060368842271\\
5909.90111097546	-0.00222236387948641\\
5915.76120131852	-0.00250486665543998\\
5921.62129166158	-0.00269162631846643\\
5927.48138200464	-0.00277842810518147\\
5933.3414723477	-0.00276342299202658\\
5939.20156269076	-0.00264716653143206\\
5945.06165303382	-0.00243260091635069\\
5950.92174337688	-0.00212498080952864\\
5956.78183371994	-0.00173174479983086\\
5962.64192406299	-0.00126233562435079\\
5968.50201440605	-0.000727973497765935\\
5974.36210474911	-0.000141387986402107\\
5980.22219509217	0.000483485169084496\\
5986.08228543523	0.00113183506484267\\
5991.94237577829	0.00178832661968941\\
5997.80246612135	0.00243746322122374\\
6003.66255646441	0.00306395245963145\\
6009.52264680747	0.00365306632065881\\
6015.38273715053	0.00419098735696594\\
6021.24282749359	0.00466513270465152\\
6027.10291783665	0.00506444834994351\\
6032.96300817971	0.0053796667672068\\
6038.82309852277	0.00560352192679455\\
6044.68318886583	0.00573091668791684\\
6050.54327920889	0.00575903872375827\\
6056.40336955195	0.00568742234715295\\
6062.26345989501	0.00551795488594006\\
6068.12355023807	0.00525482756537569\\
6073.98364058113	0.00490443216626425\\
6079.84373092419	0.00447520599931261\\
6085.70382126725	0.00397742895308915\\
6091.5639116103	0.00342297749337938\\
6097.42400195336	0.00282504149783154\\
6103.28409229642	0.0021978106739798\\
6109.14418263948	0.00155613801297503\\
6115.00427298254	0.000915188258319774\\
6120.8643633256	0.00029007970600882\\
6126.72445366866	-0.000304472205703852\\
6132.58454401172	-0.000854501117769777\\
6138.44463435478	-0.00134711764011586\\
6144.30472469784	-0.00177081121470925\\
6150.1648150409	-0.00211571888554609\\
6156.02490538396	-0.00237385470245888\\
6161.88499572702	-0.00253929442639755\\
6167.74508607008	-0.0026083112671582\\
6173.60517641314	-0.00257945954782799\\
6179.4652667562	-0.00245360442278085\\
6185.32535709926	-0.00223389705270557\\
6191.18544744232	-0.00192569592569518\\
6197.04553778538	-0.00153643628271601\\
6202.90562812843	-0.00107545082509682\\
6208.76571847149	-0.000553746023853492\\
6214.62580881455	1.62606105484162e-05\\
6220.48589915761	0.000621036029345132\\
6226.34598950067	0.00124625288840901\\
6232.20607984373	0.0018771289102097\\
6238.06617018679	0.00249877632929415\\
6243.92626052985	0.00309655316352792\\
6249.78635087291	0.00365640802571516\\
6255.64644121597	0.00416521035809779\\
6261.50653155903	0.00461105833162487\\
6267.36662190209	0.00498355719262695\\
6273.22671224515	0.00527406154989258\\
6279.08680258821	0.00547587595770156\\
6284.94689293126	0.005584409144124\\
6290.80698327432	0.00559727833517958\\
6296.66707361738	0.00551436130853336\\
6302.52716396044	0.00533779504811387\\
6308.3872543035	0.00507192113012198\\
6314.24734464656	0.00472317922852225\\
6320.10743498962	0.00429995134713795\\
6325.96752533268	0.00381236054331838\\
6331.82761567574	0.00327202897234065\\
6337.6877060188	0.00269180103443843\\
6343.54779636186	0.00208543821521955\\
6349.40788670492	0.00146729286793934\\
6355.26797704798	0.000851968667284953\\
6361.12806739104	0.000253975765548383\\
6366.9881577341	-0.000312611208892719\\
6372.84824807716	-0.000834484252292827\\
6378.70833842022	-0.00129941439902993\\
6384.56842876328	-0.00169653811643794\\
6390.42851910634	-0.00201661097626424\\
6396.2886094494	-0.0022522226819947\\
6402.14869979245	-0.00239796845315791\\
6408.00879013551	-0.00245057280581024\\
6413.86888047857	-0.00240896289796897\\
6419.72897082163	-0.0022742898028516\\
6425.58906116469	-0.00204989730375426\\
6431.44915150775	-0.00174123904191137\\
6437.30924185081	-0.00135574606517833\\
6443.16933219387	-0.000902647990680954\\
6449.02942253693	-0.000392752082520558\\
6454.88951287999	0.000161814470056469\\
6460.74960322305	0.000747892930735089\\
6466.60969356611	0.00135160519540629\\
6472.46978390917	0.00195868236896549\\
6478.32987425223	0.00255480196276792\\
6484.18996459529	0.00312592574678473\\
6490.05005493835	0.00365863028905291\\
6495.91014528141	0.00414042240254238\\
6501.77023562447	0.00456003208852832\\
6507.63032596753	0.00490767611017306\\
6513.49041631059	0.00517528603443037\\
6519.35050665365	0.005356695429094\\
6525.21059699671	0.005447781874524\\
6531.07068733976	0.00544656052279753\\
6536.93077768282	0.00535322708461923\\
6542.79086802588	0.00517014932109206\\
6548.65095836894	0.00490180733291645\\
6554.511048712	0.00455468414590147\\
6560.37113905506	0.00413710926157759\\
6566.23122939812	0.00365905894498741\\
6572.09131974118	0.00313191803605909\\
6577.95141008424	0.00256820896834099\\
6583.8115004273	0.00198129444312221\\
6589.67159077036	0.00138506081597554\\
6595.53168111342	0.000793589695060228\\
6601.39177145648	0.000220825515518639\\
6607.25186179954	-0.000319753064928165\\
6613.1119521426	-0.000815450207793016\\
6618.97204248565	-0.00125465091792605\\
6624.83213282871	-0.0016270930563418\\
6630.69222317177	-0.00192410697380834\\
6636.55231351483	-0.00213881717160385\\
6642.41240385789	-0.0022663013013571\\
6648.27249420095	-0.00230370282935873\\
6654.13258454401	-0.00225029479024497\\
6659.99267488707	-0.0021074932120618\\
6665.85276523013	-0.00187881998418719\\
6671.71285557319	-0.00156981613176464\\
6677.57294591625	-0.00118790762803002\\
6683.43303625931	-0.000742226991104452\\
6689.29312660237	-0.000243394948793349\\
6695.15321694543	0.000296732611241846\\
6701.01330728849	0.000865346378826853\\
6706.87339763155	0.0014489873534096\\
6712.7334879746	0.00203386537113897\\
6718.59357831766	0.00260618489054492\\
6724.45366866072	0.00315247034271061\\
6730.31375900378	0.00365988337488168\\
6736.17384934684	0.00411652452141069\\
6742.0339396899	0.00451171221529959\\
6747.89403003296	0.00483623260068631\\
6753.75412037602	0.00508255430570568\\
6759.61421071908	0.00524500317214631\\
6765.47430106214	0.00531989288959067\\
6771.3343914052	0.00530560853042484\\
6777.19448174826	0.00520264109569365\\
6783.05457209132	0.00501357234097803\\
6788.91466243438	0.00474301032587694\\
6794.77475277744	0.00439747728922471\\
6800.6348431205	0.00398525257689921\\
6806.49493346356	0.00351617440088875\\
6812.35502380662	0.00300140517511289\\
6818.21511414968	0.00245316602226645\\
6824.07520449274	0.00188444676516167\\
6829.9352948358	0.00130869828155682\\
6835.79538517886	0.000739514507128819\\
6841.65547552191	0.000190311602124636\\
6847.51556586497	-0.00032598814825652\\
6853.37565620803	-0.000797260254116888\\
6859.23574655109	-0.0012124626159133\\
6865.09583689415	-0.00156189411605015\\
6870.95592723721	-0.00183742115323662\\
6876.81601758027	-0.00203266683163263\\
6882.67610792333	-0.00214315840660015\\
6888.53619826639	-0.00216642957996259\\
6894.39628860945	-0.00210207530847515\\
6900.25637895251	-0.00195175791239929\\
6906.11646929557	-0.00171916442158967\\
6911.97655963863	-0.00140991624614854\\
6917.83664998169	-0.00103143338130802\\
6923.69674032475	-0.000592756424064951\\
6929.55683066781	-0.000104330668892562\\
6935.41692101087	0.00042224256351902\\
6941.27701135393	0.0009744814463379\\
6947.13710169699	0.0015393194740748\\
6952.99719204004	0.00210341458817649\\
6958.8572823831	0.00265346428939738\\
6964.71737272616	0.00317651929661852\\
6970.57746306922	0.00366028835835199\\
6976.43755341228	0.00409342704370376\\
6982.29764375534	0.00446580372984344\\
6988.1577340984	0.00476873655094517\\
6994.01782444146	0.00499519576923716\\
6999.87791478452	0.0051399668529639\\
7005.73800512758	0.00519977047984608\\
7011.59809547064	0.00517333670794662\\
7017.4581858137	0.00506143163847697\\
7023.31827615676	0.00486683602111729\\
7029.17836649982	0.00459427638360516\\
7035.03845684288	0.0042503103878975\\
7040.89854718593	0.00384316919069466\\
7046.75863752899	0.00338256059547219\\
7052.61872787205	0.00287943770187893\\
7058.47881821511	0.00234573856389811\\
7064.33890855817	0.00179410304284405\\
7070.19899890123	0.00123757356929216\\
7076.05908924429	0.000689286895977135\\
7081.91917958735	0.000162164125179634\\
7087.77926993041	-0.000331393677184145\\
7093.63936027347	-0.000779797112215492\\
7099.49945061653	-0.0011725404670072\\
7105.35954095959	-0.00150044773905863\\
7111.21963130265	-0.00175588690663076\\
7117.07972164571	-0.00193294744323167\\
7122.93981198877	-0.00202757694959323\\
7128.79990233183	-0.00203767374653313\\
7134.65999267489	-0.00196313331618809\\
7140.52008301795	-0.00180584757024491\\
7146.38017336101	-0.00156965703806855\\
7152.24026370407	-0.00126025717732863\\
7158.10035404712	-0.000885061089799094\\
7163.96044439018	-0.000453021949212453\\
7169.82053473324	2.55806072418099e-05\\
7175.6806250763	0.000539385029281543\\
7181.54071541936	0.00107621752968296\\
7187.40080576242	0.00162338048697094\\
7193.26089610548	0.00216795274406222\\
7199.12098644854	0.00269709470247947\\
7204.9810767916	0.00319835101011038\\
7210.84116713466	0.00365994370832216\\
7216.70125747772	0.00407104894102709\\
7222.56134782078	0.0044220507260332\\
7228.42143816384	0.00470476584106593\\
7234.2815285069	0.0049126345663638\\
7240.14161884996	0.00504087284005571\\
7246.00170919302	0.00508658229843938\\
7251.86179953608	0.00504881567328607\\
7257.72188987914	0.0049285960728339\\
7263.5819802222	0.00472888976538742\\
7269.44207056526	0.00445453317931075\\
7275.30216090832	0.00411211591289026\\
7281.16225125137	0.00370982258164542\\
7287.02234159443	0.0032572372966923\\
7292.88243193749	0.00276511544347561\\
7298.74252228055	0.00224512819346433\\
7304.60261262361	0.00170958581632914\\
7310.46270296667	0.00117114635007881\\
7316.32279330973	0.000642516522328329\\
7322.18288365279	0.000136151987313901\\
7328.04297399585	-0.000336036051262834\\
7333.90306433891	-0.000762960923031298\\
7339.76315468197	-0.00113462069513754\\
7345.62324502503	-0.00144233239245914\\
7351.48333536809	-0.00167893475050007\\
7357.34342571115	-0.0018389547684456\\
7363.20351605421	-0.00191873419030468\\
7369.06360639727	-0.00191651299321907\\
7374.92369674032	-0.00183246798045045\\
7380.78378708338	-0.00166870563765352\\
7386.64387742644	-0.00142920949135318\\
7392.5039677695	-0.0011197432806885\\
7398.36405811256	-0.000747712293557661\\
7404.22414845562	-0.000321986201267342\\
7410.08423879868	0.000147312371606068\\
7415.94432914174	0.000649048503737084\\
7421.8044194848	0.00117133914015726\\
7427.66450982786	0.00170183449662479\\
7433.52460017092	0.00222801004266241\\
7439.38469051398	0.00273746216191537\\
7445.24478085704	0.00321820051139937\\
7451.1048712001	0.0036589301905883\\
7456.96496154316	0.00404931708139347\\
7462.82505188622	0.00438023012724407\\
7468.68514222927	0.00464395487261931\\
7474.54523257233	0.00483437327082675\\
7480.40532291539	0.00494710557064382\\
7486.26541325845	0.00497961099312772\\
7492.12550360151	0.00493124488621574\\
7497.98559394457	0.00480327107548022\\
7503.84568428763	0.00459882918638366\\
7509.70577463069	0.00432285777721249\\
7515.56586497375	0.00398197516092144\\
7521.42595531681	0.00358432079081758\\
7527.28604565987	0.00313936100899797\\
7533.14613600293	0.00265766379220044\\
7539.00622634599	0.00215064785350499\\
7544.86631668905	0.00163031205512012\\
7550.72640703211	0.00110895154258507\\
7556.58649737517	0.000598867314853788\\
7562.44658771823	0.000112076088802196\\
7568.30667806129	-0.000339972699795591\\
7574.16676840435	-0.000746666095729551\\
7580.02685874741	-0.00109847669622303\\
7585.88694909047	-0.00138718575129767\\
7591.74703943353	-0.00160607506982482\\
7597.60712977658	-0.00175008325178003\\
7603.46722011964	-0.0018159226151982\\
7609.3273104627	-0.00180215411821402\\
7615.18740080576	-0.00170921857177884\\
7621.04749114882	-0.00153942347109906\\
7626.90758149188	-0.00129688582194969\\
7632.76767183494	-0.000987432375046988\\
7638.627762178	-0.000618459683814279\\
7644.48785252106	-0.000198757344909064\\
7650.34794286412	0.000261701355847398\\
7656.20803320718	0.000751996407223845\\
7662.06812355024	0.00126052001710392\\
7667.9282138933	0.0017752513990113\\
7673.78830423636	0.00228404089697095\\
7679.64839457942	0.00277489673474384\\
7685.50848492248	0.00323626762295892\\
7691.36857526554	0.00365731456558242\\
7697.2286656086	0.00402816547112825\\
7703.08875595166	0.00434014658960261\\
7708.94884629471	0.00458598535043495\\
7714.80893663777	0.00475997985931153\\
7720.66902698083	0.00485813110556342\\
7726.52911732389	0.00487823481579208\\
7732.38920766695	0.00481993084626727\\
7738.24929801001	0.00468470901040696\\
7744.10938835307	0.00447587126749133\\
7749.96947869613	0.00419845122673438\\
7755.82956903919	0.0038590929269255\\
7761.68965938225	0.00346589180955359\\
7767.54974972531	0.00302820168944862\\
7773.40984006837	0.00255641232483692\\
7779.26993041143	0.0020617028745361\\
7785.13002075449	0.00155577709121441\\
7790.99011109754	0.00105058652195334\\
7796.8502014406	0.000558048261289596\\
7802.71029178366	8.97639209317597e-05\\
7808.57038212672	-0.000343253558008612\\
7814.43047246978	-0.000730838818666499\\
7820.29056281284	-0.00106391264226396\\
7826.1506531559	-0.00133469454520824\\
7832.01074349896	-0.00153688444269864\\
7837.87083384202	-0.00166580913852832\\
7843.73092418508	-0.00171853023641277\\
7849.59101452814	-0.00169391098271643\\
7855.4511048712	-0.0015926405225492\\
7861.31119521426	-0.00141721505745585\\
7867.17128555732	-0.0011718764103387\\
7873.03137590038	-0.000862509506677961\\
7878.89146624344	-0.000496501247896817\\
7884.7515565865	-8.25641599395829e-05\\
7890.61164692956	0.000369470974443799\\
7896.47173727262	0.000848888565200893\\
7902.33182761568	0.00134434220401038\\
7908.19191795873	0.001844123173884\\
7914.05200830179	0.00233643712631276\\
7919.91209864485	0.00280968238947026\\
7925.77218898791	0.00325272334162619\\
7931.63227933097	0.00365515240923434\\
7937.49236967403	0.00400753452702635\\
7943.35246001709	0.00430162831903393\\
7949.21255036015	0.00453057881694765\\
7955.07264070321	0.00468907721033001\\
7960.93273104627	0.00477348390764992\\
7966.79282138933	0.00478191205601213\\
7972.65291173239	0.00471426960627653\\
7978.51300207545	0.00457225898802852\\
7984.37309241851	0.00435933446178328\\
7990.23318276157	0.00408061821249624\\
7996.09327310463	0.00374277722110838\\
8001.95336344769	0.0033538638718244\\
8007.81345379075	0.0029231241047975\\
8013.67354413381	0.00246077768362429\\
8019.53363447687	0.00197777579864329\\
8025.39372481993	0.00148554175362422\\
8031.25381516299	0.000995700875414336\\
8037.11390550604	0.000519806030933051\\
8042.9739958491	6.90652311472127e-05\\
8048.83408619216	-0.000345922256493886\\
8054.69417653522	-0.000715415075374337\\
8060.55426687828	-0.001030758370533\\
8066.41435722134	-0.00128458643998159\\
8072.2744475644	-0.00147099470286946\\
8078.13453790746	-0.00158567697055843\\
8083.99462825052	-0.00162602483077163\\
8089.85471859358	-0.00159118685275138\\
8095.71480893664	-0.00148208627178479\\
8101.5748992797	-0.00130139679299883\\
8107.43498962276	-0.00105347714186791\\
8113.29507996582	-0.000744265961507545\\
8119.15517030888	-0.000381139589259708\\
8125.01526065193	2.72638824057015e-05\\
8130.87535099499	0.000471250065677629\\
8136.73544133805	0.000940298371978892\\
8142.59553168111	0.00142331128130615\\
8148.45562202417	0.00190887688143375\\
8154.31571236723	0.00238553846709102\\
8160.17580271029	0.00284206483124809\\
8166.03589305335	0.00326771487287228\\
8171.89598339641	0.00365249028792505\\
8177.75607373947	0.00398737040045396\\
8183.61616408253	0.0042645236188018\\
8189.47625442559	0.00447749056206788\\
8195.33634476865	0.00462133457518301\\
8201.19643511171	0.00469275612697605\\
8207.05652545477	0.00469016844152466\\
8212.91661579783	0.00461373263109226\\
8218.77670614088	0.00446535155618922\\
8224.63679648394	0.00424862261316535\\
8230.496886827	0.00396875061699897\\
8236.35697717006	0.00363242288834444\\
8242.21706751312	0.00324764954019057\\
8248.07715785618	0.00282357277835864\\
8253.93724819924	0.002370249753967\\
8259.7973385423	0.00189841412506393\\
8265.65742888536	0.00141922197891081\\
8271.51751922842	0.000943988128342759\\
8277.37760957148	0.000483919014221943\\
8283.23769991454	4.98485181683334e-05\\
8289.0977902576	-0.000348017087531137\\
8294.95788060066	-0.000700339049596867\\
8300.81797094372	-0.000998865264702988\\
8306.67806128678	-0.00123662349096209\\
8312.53815162984	-0.00140808411730959\\
8318.3982419729	-0.00150928869192557\\
8324.25833231596	-0.00153794122356749\\
8330.11842265902	-0.0014934601513483\\
8335.97851300208	-0.00137698980918325\\
8341.83860334514	-0.00119137116750191\\
8347.69869368819	-0.000941072596011324\\
8353.55878403125	-0.000632082332671406\\
8359.41887437431	-0.000271765245471788\\
8365.27896471737	0.000131312688125549\\
8371.13905506043	0.000567588002048909\\
8376.99914540349	0.00102672662364745\\
8382.85923574655	0.00149786863542219\\
8388.71932608961	0.00196988512036785\\
8394.57941643267	0.00243164101740384\\
8400.43950677573	0.00287225777954352\\
8406.29959711879	0.00328136964013129\\
8412.15968746185	0.00364936745141265\\
8418.01977780491	0.00396762436058506\\
8423.87986814797	0.0042286980239978\\
8429.73995849103	0.00442650462099318\\
8435.60004883409	0.00455646059946184\\
8441.46013917715	0.00461558885176763\\
8447.32022952021	0.00460258686299975\\
8453.18031986327	0.00451785527312759\\
8459.04041020633	0.00436348623070412\\
8464.90050054939	0.00414321186451714\\
8470.76059089244	0.00386231414000165\\
8476.6206812355	0.00352749827598055\\
8482.48077157856	0.00314673275453141\\
8488.34086192162	0.00272905973982109\\
8494.20095226468	0.00228438041581017\\
8500.06104260774	0.00182322033744776\\
8505.9211329508	0.00135648035571344\\
8511.78122329386	0.000895179009245003\\
8517.64131363692	0.000450192469494161\\
8523.50140397998	3.19981760311385e-05\\
8529.36149432304	-0.000349571796148656\\
8535.2215846661	-0.000685561832787853\\
8541.08167500916	-0.000968102909160292\\
8546.94176535222	-0.00119059682223355\\
8552.80185569527	-0.00134787021387826\\
8558.66194603833	-0.00143629479067399\\
8564.52203638139	-0.00145387094796795\\
8570.38212672445	-0.00140027287203389\\
8576.24221706751	-0.00127685410565536\\
8582.10230741057	-0.0010866134956365\\
8587.96239775363	-0.000834122375845357\\
8593.82248809669	-0.000525414751988675\\
8599.68257843975	-0.000167843125083536\\
8605.54266878281	0.000230096602570953\\
8611.40275912587	0.000658966970117599\\
8617.26284946893	0.00110861275528065\\
8623.12293981199	0.00156840141982746\\
8628.98303015505	0.00202747451125621\\
8634.84312049811	0.00247500407744168\\
8640.70321084117	0.0029004480401003\\
8646.56330118423	0.00329379850529249\\
8652.42339152729	0.00364581716300942\\
8658.28348187034	0.00394825223982305\\
8664.1435722134	0.00419403191096226\\
8670.00366255646	0.00437742963913729\\
8675.86375289952	0.00449419757514533\\
8681.72384324258	0.0045416649148009\\
8687.58393358564	0.00451879893457501\\
8693.4440239287	0.00442622731398859\\
8699.30411427176	0.00426622126576131\\
8705.16420461482	0.00404263992075257\\
8711.02429495788	0.00376083732708757\\
8716.88438530094	0.00342753430388004\\
8722.744475644	0.00305065821529802\\
8728.60456598706	0.00263915448377542\\
8734.46465633012	0.00220277432339206\\
8740.32474667318	0.00175184372891886\\
8746.18483701624	0.00129701919275337\\
8752.0449273593	0.000849035927562462\\
8757.90501770236	0.000418454542535058\\
8763.76510804542	1.54121500532384e-05\\
8769.62519838848	-0.000350616232252363\\
8775.48528873154	-0.000671040369009993\\
8781.3453790746	-0.000938356351398894\\
8787.20546941765	-0.00114632227046114\\
8793.06555976071	-0.00129010390772904\\
8798.92565010377	-0.0013663870435062\\
8804.78574044683	-0.00137345377493222\\
8810.64583078989	-0.00131122108748717\\
8816.50592113295	-0.00118124081662268\\
8822.36601147601	-0.0009866610474207\\
8828.22610181907	-0.000732149909718607\\
8834.08619216213	-0.00042378361213721\\
8839.94628250519	-6.89013992747874e-05\\
8845.80637284825	0.000324069106560004\\
8851.66646319131	0.000745812023475895\\
8857.52655353437	0.00118634403701513\\
8863.38664387743	0.00163525070234832\\
8869.24673422049	0.00208193262934456\\
8875.10682456355	0.00251585573192684\\
8880.96691490661	0.00292679963415491\\
8886.82700524966	0.0033050983764149\\
8892.68709559272	0.00364186775339929\\
8898.54718593578	0.00392921393786518\\
8904.40727627884	0.00416041849243725\\
8910.2673666219	0.00433009543330857\\
8916.12745696496	0.00443431667582854\\
8921.98754730802	0.00447070294047248\\
8927.84763765108	0.00443847801676287\\
8933.70772799414	0.00433848514944135\\
8939.5678183372	0.0041731652057437\\
8945.42790868026	0.00394649718408987\\
8951.28799902332	0.00366390251348384\\
8957.14808936638	0.00333211544331347\\
8963.00817970944	0.00295902262171462\\
8968.8682700525	0.00255347568281955\\
8974.72836039555	0.00212508129596976\\
8980.58845073861	0.00168397365532782\\
8986.44854108167	0.00124057479759203\\
8992.30863142473	0.000805348415206443\\
8998.16872176779	0.000388552979863155\\
nan	nan\\
9009.88890245391	-0.000351176891762859\\
9015.74899279697	-0.000656736587672686\\
9021.60908314003	-0.000909523846688235\\
9027.46917348309	-0.00110363679407522\\
9033.32926382615	-0.00123456465848646\\
9039.18935416921	-0.0012992925329808\\
9045.04944451227	-0.00129637072791189\\
9050.90953485533	-0.00122594712163562\\
9056.76962519839	-0.00108976178970174\\
9062.62971554145	-0.000891104083604343\\
9068.48980588451	-0.000634733215633085\\
9074.34989622757	-0.000326764266075394\\
9080.20998657063	2.54776579995897e-05\\
9086.07007691369	0.000413631634918769\\
9091.93016725675	0.000828499369137581\\
9097.79025759981	0.00126026315146303\\
9103.65034794286	0.00169871817493735\\
9109.51043828592	0.00213351370972789\\
9115.37052862898	0.00255439743838982\\
9121.23061897204	0.00295145718423983\\
9127.0907093151	0.00331535433450035\\
9132.95079965816	0.00363754346196718\\
9138.81089000122	0.00391047298120032\\
9144.67098034428	0.00412776212804796\\
9150.53107068734	0.0042843501160307\\
9156.3911610304	0.00437661398349029\\
9162.25125137346	0.00440245238819283\\
9168.11134171652	0.00436133341242939\\
9173.97143205958	0.00425430529276466\\
9179.83152240264	0.00408396986436188\\
9185.6916127457	0.00385441938930239\\
9191.55170308876	0.00357113830113054\\
9197.41179343182	0.00324087222426609\\
9203.27188377488	0.00287146739490105\\
9209.13197411794	0.00247168430556852\\
9214.992064461	0.00205098999858324\\
9220.85215480405	0.0016193339321104\\
9226.71224514711	0.00118691272472488\\
9232.57233549017	0.000763929339960075\\
9238.43242583323	0.000360352397762063\\
9244.29251617629	-1.43187103329364e-05\\
9250.15260651935	-0.000351277368403233\\
9256.01269686241	-0.000642616685997277\\
9261.87278720547	-0.000881514988088023\\
9267.73287754853	-0.00106239549452392\\
9273.59296789159	-0.00118105645194892\\
9279.45305823465	-0.00123476868250341\\
9285.31314857771	-0.00122233828547741\\
9291.17323892077	-0.001144133052739\\
9297.03332926383	-0.00100207201605785\\
9302.89341960689	-0.000799578414947775\\
9308.75350994994	-0.000541497235881272\\
9314.613600293	-0.000233979308234936\\
9320.47369063606	0.000115665271268403\\
9326.33378097912	0.000499140893397119\\
9332.19387132218	0.000907363241330561\\
9338.05396166524	0.00133067447743953\\
9343.9140520083	0.00175907171526883\\
9349.77414235136	0.00218244337278847\\
9355.63423269442	0.00259080782287016\\
9361.49432303748	0.00297454870915472\\
9367.35441338054	0.00332464138060481\\
9373.2145037236	0.00363286511371622\\
9379.07459406666	0.00389199613237378\\
9384.93468440972	0.00409597689430078\\
9390.79477475278	0.00424005767722932\\
9396.65486509584	0.0043209071568294\\
9402.5149554389	0.00433668940223972\\
9408.37504578196	0.00428710551121622\\
9414.23513612501	0.00417339894181847\\
9420.09522646807	0.00399832445619156\\
9425.95531681113	0.00376608144899409\\
9431.81540715419	0.00348221327314416\\
9437.67549749725	0.00315347497583804\\
9443.53558784031	0.00278767259901116\\
9449.39567818337	0.00239347786758938\\
9455.25576852643	0.00198022266306069\\
9461.11585886949	0.00155767815331197\\
9466.97594921255	0.00113582380540982\\
9472.83603955561	0.000724611741319333\\
9478.69612989867	0.000333731999683831\\
9484.55622024173	-2.76157599881354e-05\\
9490.41631058479	-0.000350938732998813\\
9496.27640092785	-0.000628650531872853\\
9502.13649127091	-0.000854249146834765\\
9507.99658161397	-0.00102246913060506\\
9513.85667195703	-0.0011294044460698\\
9519.71676230009	-0.00117259911164546\\
9525.57685264315	-0.00115110354139142\\
9531.43694298621	-0.00106549528902926\\
9537.29703332927	-0.000917863745331579\\
9543.15712367232	-0.000711759189798616\\
9549.01721401538	-0.000452107437206596\\
9554.87730435844	-0.000145092130400868\\
9560.7373947015	0.000201992508947934\\
9566.59748504456	0.000580914965836019\\
9572.45757538762	0.000982701637709513\\
9578.31766573068	0.0013978493320181\\
9584.17775607374	0.00181655002431203\\
9590.0378464168	0.00222892256176408\\
9595.89793675986	0.00262524583909361\\
9601.75802710292	0.00299618794527391\\
9607.61811744598	0.00333302587986098\\
9613.47820778904	0.0036278506670992\\
9619.3382981321	0.00387375304460541\\
9625.19838847516	0.00406498536853612\\
9631.05847881822	0.0041970959411474\\
9636.91856916128	0.00426703262295962\\
9642.77865950433	0.00427321331770823\\
9648.63874984739	0.00421556170466336\\
9654.49884019045	0.00409550741163851\\
9660.35893053351	0.00391595066399643\\
9666.21902087657	0.00368119228174126\\
9672.07911121963	0.00339683071277788\\
9677.93920156269	0.0030696285673527\\
9683.79929190575	0.00270735183443493\\
9689.65938224881	0.00231858560203689\\
9695.51947259187	0.00191253065364958\\
9701.37956293493	0.00149878575938505\\
9707.23965327799	0.00108712081242176\\
9713.09974362105	0.000687246171984016\\
9718.95983396411	0.000308583657897991\\
9724.81992430716	-3.99554034799944e-05\\
9730.68001465022	-0.000350179853627983\\
9736.54010499328	-0.000614811163982807\\
9742.40019533634	-0.000827654164184469\\
9748.2602856794	-0.000983742032508692\\
9754.12037602246	-0.00107945215577781\\
9759.98046636552	-0.001112590156819\\
9765.84055670858	-0.00108244014129332\\
9771.70064705164	-0.000989780014612611\\
9777.5607373947	-0.000836861544688236\\
9783.42082773776	-0.00062735567827998\\
9789.28091808082	-0.000366264437741306\\
9795.14100842388	-5.9801511773406e-05\\
9801.00109876694	0.000284755609934703\\
9806.86118911	0.000659238440118995\\
9812.72127945306	0.00105478113286438\\
9818.58136979612	0.00146203036832846\\
9824.44146013918	0.00187136651527814\\
9830.30155048224	0.00227313084248649\\
9836.1616408253	0.00265785341320656\\
9842.02173116836	0.00301647628608772\\
9847.88182151142	0.00334056676256645\\
9853.74191185447	0.00362251566100787\\
9859.60200219753	0.00385571595604088\\
9865.46209254059	0.00403471759031168\\
9871.32218288365	0.00415535483135437\\
9877.18227322671	0.00421484319840697\\
9883.04236356977	0.0042118437045222\\
9888.90245391283	0.00414649293305666\\
9894.76254425589	0.00402039827435218\\
9900.62263459895	0.00383659847186479\\
9906.48272494201	0.00359949044392328\\
9912.34281528507	0.00331472414342831\\
9918.20290562813	0.00298906796786863\\
9924.06299597119	0.00263024792603116\\
9929.92308631425	0.00224676438194026\\
9935.78317665731	0.00184769072254565\\
9941.64326700037	0.0014424587168023\\
9947.50335734343	0.00104063564336161\\
9953.36344768649	0.000651698452519214\\
9959.22353802955	0.000284810292908342\\
9965.08362837261	-5.13953281799235e-05\\
9970.94371871567	-0.000349017667190359\\
9976.80380905872	-0.000601074371164269\\
9982.66389940178	-0.000801665248406276\\
9988.52398974484	-0.000946110342293003\\
9994.3840800879	-0.00103105907773339\\
10000.244170431	-0.00105456793537535\\
10006.104260774	-0.00101614485376583\\
10011.9643511171	-0.000916759346183609\\
10017.8244414601	-0.000758818128777706\\
10023.6845318032	-0.00054610687034786\\
10029.5446221463	-0.00028369947249052\\
10035.4047124893	2.21629464914135e-05\\
10041.2648028324	0.000364220478364466\\
10047.1248931754	0.000734366733043322\\
10052.9849835185	0.00112384093899665\\
10058.8450738616	0.00152343528432678\\
10064.7051642046	0.00192371259194056\\
10070.5652545477	0.00231522918344578\\
10076.4253448907	0.0026887576695916\\
10082.2854352338	0.00303550441144205\\
10088.1455255769	0.00334731652848656\\
10094.0056159199	0.00361687358081613\\
10099.865706263	0.00383785941951596\\
10105.725796606	0.00400511017125079\\
10111.5858869491	0.00411473489116484\\
10117.4459772922	0.00416420606459005\\
10123.3060676352	0.00415241785433444\\
10129.1661579783	0.00407971075127527\\
10135.0262483213	0.00394786208149846\\
10140.8863386644	0.00376004262754251\\
10146.7464290074	0.00352074042202343\\
10152.6065193505	0.00323565354394054\\
10158.4666096936	0.0029115544778712\\
10164.3267000366	0.0025561292641822\\
10170.1867903797	0.0021777952602686\\
10176.0468807227	0.00178550183289332\\
10181.9069710658	0.00138851870049675\\
10187.7670614089	0.000996216930582045\\
10193.6271517519	0.000617847765708381\\
10199.487242095	0.000262324497739244\\
10205.347332438	-6.1987467481811e-05\\
10211.2074227811	-0.00034746741081545\\
10217.0675131242	-0.000587418336644094\\
10222.9276034672	-0.000776224040201815\\
10228.7876938103	-0.0009094805221939\\
10234.6477841533	-0.000984098676312138\\
10240.5078744964	-0.000998375855852838\\
10246.3679648395	-0.000952034662309488\\
10252.2280551825	-0.000846228073398956\\
10258.0881455256	-0.000683510823126175\\
10263.9482358686	-0.000467777742389184\\
10269.8083262117	-0.000204170547546042\\
10275.6684165548	0.000101044693077294\\
10281.5285068978	0.000440626495162082\\
10287.3885972409	0.000806529757394639\\
10293.2486875839	0.00119009634824351\\
10299.108777927	0.00158225996608609\\
10304.9688682701	0.00197376042610718\\
10310.8289586131	0.00235536230955139\\
10316.6890489562	0.00271807281358105\\
10322.5491392992	0.00305335366302287\\
10328.4092296423	0.00335332209138283\\
10334.2693199853	0.00361093616065831\\
10340.1294103284	0.00382016006354836\\
10345.9895006715	0.00397610552871888\\
10351.8495910145	0.00407514601630697\\
10357.7096813576	0.00411500103698702\\
10363.5697717006	0.00409478863462869\\
10369.4298620437	0.00401504482546074\\
10375.2899523868	0.0038777095678259\\
10381.1500427298	0.00368607962577793\\
10387.0101330729	0.00344472947134819\\
10392.8702234159	0.00315940212184735\\
10398.730313759	0.00283687251706336\\
10404.5904041021	0.00248478668600041\\
10410.4504944451	0.00211148052044146\\
10416.3105847882	0.00172578245042778\\
10422.1706751312	0.00133680469212515\\
10428.0307654743	0.000953728003514114\\
10433.8908558174	0.000585585031828312\\
10439.7509461604	0.000241047364741797\\
10445.6110365035	-7.17786946231689e-05\\
10451.4711268465	-0.000345542819937063\\
10457.3312171896	-0.000573823335669707\\
10463.1913075327	-0.00075127781698924\\
10469.0513978757	-0.000873768084241219\\
10474.9114882188	-0.00093845666803155\\
10480.7715785618	-0.000943872496941892\\
10486.6316689049	-0.000889944287736941\\
10492.491759248	-0.000778000881692425\\
10498.351849591	-0.000610738550696399\\
10504.2119399341	-0.000392156076208997\\
10510.0720302771	-0.000127459163385799\\
10515.9321206202	0.000177063525259744\\
10521.7922109633	0.000514189765212761\\
10527.6523013063	0.000875935045694689\\
10533.5123916494	0.00125374166453445\\
10539.3724819924	0.00163868116430806\\
10545.2325723355	0.0020216653216966\\
10551.0926626786	0.0023936607045229\\
10556.9527530216	0.00274590173169188\\
10562.8128433647	0.00307009721166655\\
10568.6729337077	0.00335862549278282\\
10574.5330240508	0.00360471363352503\\
10580.3931143938	0.00380259638053253\\
10586.2532047369	0.00394765122377145\\
10592.11329508	0.0040365063650835\\
10597.973385423	0.00406711907964583\\
10603.8334757661	0.00403882264862621\\
10609.6935661091	0.00395234078702719\\
10615.5536564522	0.00380976925589593\\
10621.4137467953	0.00361452512434587\\
10627.2738371383	0.00337126490894524\\
10633.1339274814	0.00308577355024298\\
10638.9940178244	0.00276482687384703\\
10644.8541081675	0.00241603080492403\\
10650.7141985106	0.00204764115135044\\
10656.5742888536	0.00166836822485437\\
10662.4343791967	0.00128717092367804\\
10668.2944695397	0.000913045144044303\\
10674.1545598828	0.000554811517343699\\
10680.0146502259	0.000220907480782324\\
10685.8747405689	-8.08114172161412e-05\\
10691.734830912	-0.000343256298508176\\
10697.594921255	-0.000560271477435844\\
10703.4550115981	-0.000726778812581885\\
10709.3151019412	-0.000838896503484505\\
10715.1751922842	-0.000894029553685494\\
10721.0352826273	-0.000890929792516086\\
10726.8953729703	-0.000829724068009946\\
10732.7554633134	-0.000711909975547922\\
10738.6155536565	-0.000540319253106635\\
10744.4756439995	-0.000319049736759018\\
10750.3357343426	-5.3367510771315e-05\\
10756.1958246856	0.000250418411753181\\
10762.0559150287	0.000585105895872639\\
10767.9160053717	0.000942770422780636\\
10773.7760957148	0.00131495271158743\\
10779.6361860579	0.00169285878333361\\
10785.4962764009	0.0020675677359539\\
10791.356366744	0.00243024232186667\\
10797.216457087	0.00277233735769858\\
10803.0765474301	0.00308580105198491\\
10808.9366377732	0.00336326450764631\\
10814.7967281162	0.00359821494007103\\
10820.6568184593	0.0037851485389755\\
10826.5169088023	0.00391969938738359\\
10832.3769991454	0.00399874141766316\\
10838.2370894885	0.00402046102484192\\
10844.0971798315	0.00398439865071868\\
10849.9572701746	0.00389145838628969\\
10855.8173605176	0.00374388539552924\\
10861.6774508607	0.00354521172185579\\
10867.5375412038	0.00330017178559698\\
10873.3976315468	0.00301458959249164\\
10879.2577218899	0.00269524034099343\\
10885.1178122329	0.00234968971526866\\
10890.977902576	0.0019861146766046\\
10896.8379929191	0.00161310999657833\\
10902.6980832621	0.00123948510972467\\
10908.5581736052	0.000874056086533502\\
10914.4182639482	0.000525437638647681\\
10920.2783542913	0.000201840063112207\\
10926.1384446344	-8.91240887661348e-05\\
10931.9985349774	-0.000340619065828483\\
10937.8586253205	-0.000546746483764736\\
10943.7187156635	-0.000702683632908234\\
10949.5788060066	-0.000804796284483161\\
10955.4388963497	-0.000850723357193363\\
10961.2989866927	-0.000839431472165179\\
10967.1590770358	-0.000771238136418995\\
10973.0191673788	-0.00064780303624405\\
10978.8792577219	-0.000472087674542846\\
10984.7393480649	-0.000248284332806725\\
10990.599438408	1.82839388482019e-05\\
10996.4595287511	0.000321289918300479\\
11002.3196190941	0.000653552384313878\\
11008.1797094372	0.00100720630188527\\
11014.0397997802	0.00137388898730985\\
11019.8998901233	0.00174493784698165\\
11025.7599804664	0.00211159501458328\\
11031.6200708094	0.00246521405312791\\
11037.4801611525	0.00279746384381757\\
11043.3402514955	0.0031005248539906\\
11049.2003418386	0.00336727316123264\\
11055.0604321817	0.00359144790304411\\
11060.9205225247	0.00376779821679077\\
11066.7806128678	0.00389220622127304\\
11072.6407032108	0.0039617831605952\\
11078.5007935539	0.00397493646565578\\
11084.360883897	0.00393140617704841\\
11090.22097424	0.0038322698974219\\
11096.0810645831	0.00367991618436731\\
11101.9411549261	0.00347798703931719\\
11107.8012452692	0.00323129087659242\\
11113.6613356123	0.00294568805187241\\
11119.5214259553	0.00262795167483101\\
11125.3815162984	0.00228560701108759\\
11131.2416066414	0.00192675328124091\\
11137.1016969845	0.00155987207647026\\
11142.9617873276	0.00119362692170963\\
11148.8218776706	0.000836658723243469\\
11154.6819680137	0.000497381929508431\\
11160.5420583567	0.000183786213358322\\
11166.4021486998	-9.6751650451197e-05\\
11172.2622390429	-0.00033764128368209\\
11178.1223293859	-0.00053323349861615\\
11183.982419729	-0.000678952752354025\\
11189.842510072	-0.000771404156191048\\
11195.7026004151	-0.000808452537875103\\
11201.5626907581	-0.000789271715960848\\
11207.4227811012	-0.000714362849918624\\
11213.2828714443	-0.000585541459919407\\
11219.1429617873	-0.000405893449987297\\
11225.0030521304	-0.000179701198804234\\
11230.8631424734	8.76585120602249e-05\\
11236.7232328165	0.000389842300133102\\
11242.5833231596	0.000719690676834046\\
11248.4434135026	0.00106939766507679\\
11254.3035038457	0.00143069552166349\\
11260.1635941887	0.00179505019347766\\
11266.0236845318	0.00215386288732921\\
11271.8837748749	0.00249867299264071\\
11277.7438652179	0.00282135756896189\\
11283.603955561	0.00311432269489252\\
11289.464045904	0.0033706821712818\\
11295.3241362471	0.00358441937418049\\
11301.1842265902	0.00375052845314002\\
11307.0443169332	0.00386513156220703\\
11312.9044072763	0.00392556937828703\\
11318.7644976193	0.00393046279449555\\
11324.6245879624	0.0038797443577932\\
11330.4846783055	0.00377465873547897\\
11336.3447686485	0.00361773222557966\\
11342.2048589916	0.00341271205754877\\
11348.0649493346	0.00316447694066062\\
11353.9250396777	0.00287892099522269\\
11359.7851300208	0.0025628138275367\\
11365.6452203638	0.00222364006997352\\
11371.5053107069	0.00186942218896967\\
11377.3654010499	0.00150853075579784\\
11383.225491393	0.00114948666634536\\
11389.0855817361	0.000800759982938834\\
11394.9456720791	0.000470570146491156\\
11400.8057624222	0.000166692271139188\\
11406.6658527652	-0.000103725914057236\\
11412.5259431083	-0.000334332166770206\\
11418.3860334513	-0.000519718923417286\\
11424.2461237944	-0.00065555007787772\\
11430.1062141375	-0.000738662375043392\\
11435.9663044805	-0.000767139048648679\\
11441.8263948236	-0.000740353989706235\\
11447.6864851666	-0.000658985428308061\\
11453.5465755097	-0.000524998831970757\\
11459.4066658528	-0.0003415994497413\\
11465.2667561958	-0.00011315564734382\\
11471.1268465389	0.000154905133447802\\
11476.9869368819	0.000456225313723368\\
11482.847027225	0.000783667951793509\\
11488.7071175681	0.00112948577771722\\
11494.5672079111	0.00148550448483533\\
11500.4272982542	0.00184331594227765\\
11506.2873885972	0.00219447676199529\\
11512.1474789403	0.00253070753122748\\
11518.0075692834	0.00284408800978299\\
11523.8676596264	0.00312724369047676\\
11529.7277499695	0.00337351932864017\\
11535.5878403125	0.00357713535795049\\
11541.4479306556	0.00373332351644226\\
11547.3080209987	0.00383843850080378\\
11553.1681113417	0.00389004303458434\\
11559.0282016848	0.00388696436550349\\
11564.8882920278	0.00382932088312443\\
11570.7483823709	0.00371851825338741\\
11576.608472714	0.00355721518658443\\
11582.468563057	0.00334925967178131\\
11588.3286534001	0.00309959720636045\\
11594.1887437431	0.00281415320880766\\
11600.0488340862	0.00249969241031521\\
11605.9089244293	0.00216365856146674\\
11611.7690147723	0.00181399825166154\\
11617.6291051154	0.00145897301096747\\
11623.4891954584	0.0011069641368053\\
11629.3492858015	0.000766274855940216\\
11635.2093761446	0.000444934491304871\\
11641.0694664876	0.000150509252026209\\
11646.9295568307	-0.000110075895037949\\
11652.7896471737	-0.000330700078967078\\
11658.6497375168	-0.000506190274248848\\
11664.5098278598	-0.000632442570444803\\
11670.3699182029	-0.000706518119476864\\
11676.230008546	-0.000726711517823789\\
11682.090098889	-0.000692590032835402\\
11687.9501892321	-0.00060500277171346\\
11693.8102795751	-0.000466059601298056\\
11699.6703699182	-0.000279080340854924\\
11705.5304602613	-4.85154505816385e-05\\
11711.3905506043	0.000220159889687517\\
11717.2506409474	0.000520575791086038\\
11723.1107312904	0.000845618668812081\\
11728.9708216335	0.00118759967797026\\
11734.8309119766	0.00153843658425448\\
11740.6910023196	0.0018898447679716\\
11746.5510926627	0.00223353284780547\\
11752.4111830057	0.00256139830547651\\
11758.2712733488	0.00286571849612794\\
11764.1313636919	0.0031393325419885\\
11769.9914540349	0.00337580982533867\\
11775.851544378	0.00356960111660585\\
11781.711634721	0.00371616878650527\\
11787.5717250641	0.00381209304655986\\
11793.4318154072	0.00385515173155192\\
11799.2919057502	0.00384437176269776\\
11805.1519960933	0.00378005110029099\\
11811.0120864363	0.00366375069161708\\
11816.8721767794	0.00349825662833911\\
11822.7322671225	0.00328751343019484\\
11828.5923574655	0.00303653005164627\\
11834.4524478086	0.00275126085144999\\
11840.3125381516	0.00243846435305652\\
11846.1726284947	0.00210554314583713\\
11852.0327188377	0.00176036871911771\\
11857.8928091808	0.0014110953738765\\
11863.7528995239	0.00106596761078129\\
11869.6129898669	0.00073312554389767\\
11875.47308021	0.000420412932435065\\
11881.333170553	0.000135192357265997\\
11887.1932608961	-0.000115828103150199\\
11893.0533512392	-0.000326752617424923\\
11898.9134415822	-0.000492636057467425\\
11904.7735319253	-0.000609599915083926\\
11910.6336222683	-0.000674922962167231\\
11916.4937126114	-0.000687104535664512\\
11922.3538029545	-0.000645898976044536\\
11928.2138932975	-0.000552320429514992\\
11934.0739836406	-0.000408617924373255\\
11939.9340739836	-0.000218221332815659\\
11945.7941643267	1.43404838178694e-05\\
11951.6542546698	0.000283547393260629\\
11957.5143450128	0.000583019012334543\\
11963.3744353559	0.00090566591928914\\
11969.2345256989	0.00124385747522906\\
11975.094616042	0.00158960228214142\\
11980.9547063851	0.00193473701038974\\
11986.8147967281	0.00227111913406627\\
11992.6748870712	0.00259081902417382\\
11998.5349774142	0.00288630686836549\\
12004.3950677573	0.0031506300112117\\
12010.2551581004	0.00337757653881428\\
12016.1152484434	0.00356182125941683\\
12021.9753387865	0.00369905064929623\\
12027.8354291295	0.00378606383305424\\
12033.6955194726	0.0038208472346446\\
12039.5556098157	0.00380262115837028\\
12045.4157001587	0.00373185722264122\\
12051.2757905018	0.00361026625855431\\
12057.1358808448	0.00344075698087566\\
12062.9959711879	0.00322736643019985\\
12068.8560615309	0.00297516384823449\\
12074.716151874	0.0026901302757762\\
12080.5762422171	0.00237901673159098\\
12086.4363325601	0.00204918433577047\\
12092.2964229032	0.00170843016272789\\
12098.1565132462	0.00136480294354482\\
12104.0166035893	0.00102641297382349\\
12109.8766939324	0.000701240715803483\\
12115.7367842754	0.000396948611615346\\
12121.5968746185	0.0001207005447353\\
12127.4569649615	-0.000121006796832772\\
12133.3170553046	-0.000322496686323777\\
12139.1771456477	-0.000479045661057906\\
12145.0372359907	-0.000586994232414466\\
12150.8973263338	-0.000643832408403683\\
12156.7574166768	-0.000648258030506057\\
12162.6175070199	-0.000600206569420244\\
12168.477597363	-0.000500851698535456\\
12174.337687706	-0.000352576654140911\\
12180.1977780491	-0.000158917080374117\\
12186.0578683921	7.55232708467938e-05\\
12191.9179587352	0.000345181975279903\\
12197.7780490783	0.000643669906154229\\
12203.6381394213	0.000963922607587565\\
12209.4982297644	0.00129836748561333\\
12215.3583201074	0.00163910286021765\\
12221.2184104505	0.00197808464517036\\
12227.0785007936	0.00230731624555055\\
12232.9385911366	0.00261903719041244\\
12238.7986814797	0.00290590605113968\\
12244.6587718227	0.00316117333453156\\
12250.5188621658	0.00337884027913275\\
12256.3789525089	0.00355379981892677\\
12262.2390428519	0.00368195640269436\\
12268.099133195	0.00376032185772527\\
12273.959223538	0.00378708505443428\\
12279.8193138811	0.00376165374866157\\
12285.6794042241	0.00368466763461515\\
12291.5394945672	0.00355798232289191\\
12297.3995849103	0.00338462464340911\\
12303.2596752533	0.00316872034916426\\
12309.1197655964	0.00291539594744337\\
12314.9798559394	0.00263065699378987\\
12320.8399462825	0.00232124573828104\\
12326.7000366256	0.0019944814973436\\
12332.5601269686	0.00165808753087376\\
12338.4202173117	0.00132000851873012\\
12344.2803076547	0.000988222950008418\\
12350.1403979978	0.000670554855121089\\
12356.0004883409	0.000374489322910161\\
12361.8605786839	0.000106996152381589\\
12367.720669027	-0.000125634206525346\\
12373.58075937	-0.00031793856165308\\
12379.4408497131	-0.000465409259387132\\
12385.3009400562	-0.00056459982558779\\
12391.1610303992	-0.000613205491169877\\
12397.0211207423	-0.000610116721304849\\
12402.8812110853	-0.000555444505329712\\
12408.7413014284	-0.000450516831741952\\
12414.6013917715	-0.00029784645285955\\
12420.4614821145	-0.000101070720897505\\
12426.3215724576	0.00013513506430627\\
12432.1816628006	0.00040516873201525\\
12438.0417531437	0.000702634103045424\\
12443.9018434868	0.0010204924872384\\
12449.7619338298	0.00135122922800426\\
12455.6220241729	0.00168703135342849\\
12461.4821145159	0.00201997213471457\\
12467.342204859	0.00234219819242049\\
12473.2022952021	0.00264611473427888\\
12479.0623855451	0.00292456455585278\\
12484.9224758882	0.00317099658537821\\
12490.7825662312	0.00337962000402058\\
12496.6426565743	0.00354554031603154\\
12502.5027469173	0.00366487417201057\\
12508.3628372604	0.00373484025126078\\
12514.2229276035	0.00375382407735921\\
12520.0830179465	0.00372141525634715\\
12525.9431082896	0.00363841627921462\\
12531.8031986326	0.00350682270205226\\
12537.6632889757	0.00332977519202992\\
12543.5233793188	0.00311148459138351\\
12549.3834696618	0.00285713178704507\\
12555.2435600049	0.002572744766284\\
12561.1036503479	0.00226505577614567\\
12566.963740691	0.00194134197063009\\
12572.8238310341	0.00160925331756385\\
12578.6839213771	0.00127663183456044\\
12584.5440117202	0.000951326424858319\\
12590.4041020632	0.000641007685132349\\
12596.2641924063	0.000352987054251422\\
12602.1242827494	9.40445667776325e-05\\
12607.9843730924	-0.000129730731236164\\
12613.8444634355	-0.000313083948290628\\
12619.7045537785	-0.000451717729446316\\
12625.5646441216	-0.000542392957633614\\
12631.4247344647	-0.000583004414815551\\
12637.2848248077	-0.000572629635952271\\
12643.1449151508	-0.000511549822465491\\
12649.0050054938	-0.000401242341946952\\
12654.8650958369	-0.000244345011436138\\
12660.72518618	-4.45930272548612e-05\\
12666.585276523	0.000193269950974337\\
12672.4453668661	0.000463604445762696\\
12678.3054572091	0.000760008862021156\\
12684.1655475522	0.00107547107263958\\
12690.0256378953	0.00140253430026165\\
12695.8857282383	0.00173347337133613\\
12701.7458185814	0.00206047717666717\\
12707.6059089244	0.002375833029362\\
12713.4659992675	0.00267210856891523\\
12719.3260896105	0.00294232692201863\\
12725.1861799536	0.00318013099173745\\
12731.0462702967	0.00337993300728251\\
12736.9063606397	0.00353704581610459\\
12742.7664509828	0.00364779283380925\\
12748.6265413258	0.00370959407306639\\
12754.4866316689	0.00372102623895638\\
12760.346722012	0.00368185549068407\\
12766.206812355	0.00359304211695406\\
12772.0669026981	0.00345671703384009\\
12777.9269930411	0.00327613068007132\\
12783.7870833842	0.00305557553454391\\
12789.6471737273	0.00280028410320546\\
12795.5072640703	0.0025163047989812\\
12801.3673544134	0.00221035865913803\\
12807.2274447564	0.00188968029365477\\
12813.0875350995	0.0015618468286643\\
12818.9476254426	0.00123459888875648\\
12824.8077157856	0.000915657847748411\\
12830.6678061287	0.000612543661835639\\
12836.5278964717	0.000332397582802178\\
12842.3879868148	8.18139306052683e-05\\
12848.2480771579	-0.00013331511204746\\
12854.1081675009	-0.000307938030400445\\
12859.968257844	-0.00043796257693283\\
12865.828348187	-0.000520351655020747\\
12871.6884385301	-0.000553194240633414\\
12877.5485288732	-0.000535749686200485\\
12883.4086192162	-0.000468464380026777\\
12889.2687095593	-0.000352960387325561\\
12895.1287999023	-0.000191996360520902\\
12900.9888902454	1.05983396940968e-05\\
12906.8489805884	0.000250014739631456\\
12912.7090709315	0.00052057839741305\\
12918.5691612746	0.000815883888288386\\
12924.4292516176	0.00112894644337695\\
12930.2893419607	0.00145236715229779\\
12936.1494323037	0.00177850782262507\\
12942.0095226468	0.00209967136398446\\
12947.8696129899	0.00240828343697946\\
12953.7297033329	0.00269707108066291\\
12959.589793676	0.00295923410579319\\
12965.449884019	0.00318860521565528\\
12971.3099743621	0.00337979508384935\\
12977.1700647052	0.00352831897704251\\
12983.0301550482	0.00363070194760178\\
12988.8902453913	0.0036845601292744\\
12994.7503357343	0.00368865623383579\\
13000.6104260774	0.00364292795769067\\
13006.4705164205	0.00354848864624062\\
13012.3306067635	0.00340760021986583\\
13018.1906971066	0.00322361901866571\\
13024.0507874496	0.00300091586296916\\
13029.9108777927	0.00274477223260003\\
13035.7709681358	0.00246125503114236\\
13041.6310584788	0.00215707290440395\\
13047.4911488219	0.00183941751565272\\
13053.3512391649	0.00151579353264106\\
13059.211329508	0.00119384134555545\\
13065.0714198511	0.000881156703204051\\
13070.9315101941	0.000585111525206799\\
13076.7916005372	0.000312680116960369\\
13082.6516908802	7.02748838355391e-05\\
13088.5117812233	-0.000136404585615859\\
13094.3718715664	-0.000302505515995168\\
13100.2319619094	-0.000424135870874628\\
13106.0920522525	-0.000498455533806195\\
13111.9521425955	-0.000523742608611341\\
13117.8122329386	-0.000499433291546825\\
13123.6723232817	-0.000426134392418144\\
13129.5324136247	-0.000305608227740763\\
13135.3925039678	-0.000140730260949095\\
13141.2525943108	6.45794925091613e-05\\
13147.1126846539	0.00030544965918062\\
13152.9727749969	0.000576173085354545\\
13158.83286534	0.000870342056633228\\
13164.6929556831	0.00118099995578919\\
13170.5530460261	0.0015008057697849\\
13176.4131363692	0.00182220755591233\\
13182.2732267122	0.00213762076854656\\
13188.1333170553	0.00243960723559769\\
13193.9934073984	0.00272105056194462\\
13199.8534977414	0.002975323823287\\
13205.7135880845	0.0031964455997176\\
13211.5736784275	0.00337922067474385\\
13217.4337687706	0.00351936209084715\\
13223.2938591137	0.00361359169415607\\
13229.1539494567	0.00365971681014061\\
13235.0140397998	0.00365668125752385\\
13240.8741301428	0.00360458951335667\\
13246.7342204859	0.00350470347765947\\
13252.594310829	0.00335941193135101\\
13258.454401172	0.00317217342681173\\
13264.3144915151	0.00294743397577846\\
13270.1745818581	0.002690521493395\\
13276.0346722012	0.00240751950441071\\
13281.8947625443	0.00210512310452079\\
13287.7548528873	0.00179048058761845\\
13293.6149432304	0.00147102448452132\\
13299.4750335734	0.00115429600674359\\
13305.3351239165	0.00084776704176819\\
13311.1952142596	0.000558663901163979\\
13317.0553046026	0.000293796978764833\\
13322.9153949457	5.94003341502795e-05\\
13328.7754852887	-0.000139015020327091\\
13334.6355756318	-0.000296790676461904\\
13340.4956659748	-0.000410230185557357\\
13346.3557563179	-0.000476685645360349\\
13352.215846661	-0.000494619490605733\\
13358.075937004	-0.000463640045495784\\
13363.9360273471	-0.000384510016431121\\
13369.7961176901	-0.000259127742386496\\
13375.6562080332	-9.04816629876709e-05\\
13381.5162983763	0.000117420903896955\\
13387.3763887193	0.000359648977902956\\
13393.2364790624	0.000630464863146462\\
13399.0965694054	0.000923460053034822\\
13404.9566597485	0.00123170687408411\\
13410.8167500916	0.00154792228059124\\
13416.6768404346	0.0018646399278886\\
13422.5369307777	0.00217438645866533\\
13428.3970211207	0.00246985784069783\\
13434.2571114638	0.00274409159504284\\
13440.1172018069	0.00299063085426809\\
13445.9772921499	0.00320367638486224\\
13451.837382493	0.00337822299508318\\
13457.697472836	0.00351017711966958\\
13463.5575631791	0.00359645281948817\\
13469.4176535222	0.00363504394500396\\
13475.2777438652	0.00362507077624197\\
13481.1378342083	0.00356680005502438\\
13486.9979245513	0.0034616379546464\\
13492.8580148944	0.00331209616943637\\
13498.7181052375	0.00312173194208246\\
13504.5781955805	0.00289506346023967\\
13510.4382859236	0.00263746263442819\\
13516.2983762666	0.00235502780130478\\
13522.1584666097	0.00205443936895238\\
13528.0185569528	0.00174280182061814\\
13533.8786472958	0.00142747581368767\\
13539.7387376389	0.00111590434142437\\
13545.5988279819	0.000815437062982916\\
13551.458918325	0.000533156947610412\\
13557.3190086681	0.000275713321537797\\
13563.1790990111	4.9165252847688e-05\\
13569.0391893542	-0.000141161037301188\\
13574.8992796972	-0.00029079738163158\\
13580.7593700403	-0.000396238548846174\\
13586.6194603833	-0.00045502433916357\\
13592.4795507264	-0.000465796970757441\\
13598.3396410695	-0.000428332418795748\\
13604.1997314125	-0.000343544984132847\\
13610.0598217556	-0.000213465000840088\\
13615.9199120986	-4.11902254466342e-05\\
13621.7800024417	0.000169187928476629\\
13627.6400927848	0.000412681554150447\\
13633.5001831278	0.000683524506567394\\
13639.3602734709	0.000975308945195455\\
13645.2203638139	0.00128113693133899\\
13651.080454157	0.00159378349377976\\
13656.9405445001	0.00190586730827287\\
13662.8006348431	0.00221002495888296\\
13668.6607251862	0.002499084667487\\
13674.5208155292	0.00276623539278202\\
13680.3809058723	0.00300518731189572\\
13686.2409962154	0.00321031990360969\\
13692.1010865584	0.00337681414709833\\
13697.9611769015	0.0035007657267381\\
13703.8212672445	0.00357927658441986\\
13709.6813575876	0.00361052267204781\\
13715.5414479307	0.00359379632114949\\
13721.4015382737	0.00352952224543625\\
13727.2616286168	0.0034192468150607\\
13733.1217189598	0.00326560087258786\\
13738.9818093029	0.00307223698416493\\
13744.841899646	0.00284374262236742\\
13750.701989989	0.00258553134405858\\
13756.5620803321	0.00230371454466459\\
13762.4221706751	0.00200495682623152\\
13768.2822610182	0.0016963184025553\\
13774.1423513612	0.00138508826722323\\
13780.0024417043	0.0010786120667953\\
13785.8625320474	0.00078411874361199\\
13791.7226223904	0.000508550039020138\\
13797.5827127335	0.000258396878264007\\
13803.4428030765	3.9546492979655e-05\\
13809.3028934196	-0.000142856118191634\\
13815.1629837627	-0.000284529130962334\\
13821.0230741057	-0.000382154396046635\\
13826.8831644488	-0.000433455140392547\\
13832.7432547918	-0.000437249049686191\\
13838.6033451349	-0.000393475494789959\\
13844.463435478	-0.000303196275538598\\
13850.323525821	-0.000168569880621201\\
13856.1836161641	7.20011301820675e-06\\
13862.0437065071	0.000219941267876601\\
13867.9037968502	0.000464611327181464\\
13873.7638871933	0.000735417718974825\\
13879.6239775363	0.00102595469092932\\
13885.4840678794	0.00132935482955634\\
13891.3441582224	0.00163845137996679\\
13897.2042485655	0.00194594752989359\\
13903.0643389086	0.00224458865951919\\
13908.9244292516	0.00252733349123196\\
13914.7845195947	0.00278752010166614\\
13920.6446099377	0.00301902288247949\\
13926.5047002808	0.00321639675151773\\
13932.3647906239	0.00337500522006096\\
13938.2248809669	0.00349112930359026\\
13944.08497131	0.00356205471872728\\
13949.945061653	0.00358613532134955\\
13955.8051519961	0.00356283130393929\\
13961.6652423392	0.00349272126603635\\
13967.5253326822	0.00337748788853582\\
13973.3854230253	0.00321987756595162\\
13979.2455133683	0.00302363496483255\\
13985.1056037114	0.0027934140670323\\
13990.9656940544	0.00253466781147139\\
13996.8257843975	0.0022535189504021\\
14002.6858747406	0.00195661517922525\\
14008.5459650836	0.00165097196672921\\
14014.4060554267	0.00134380680201195\\
14020.2661457697	0.00104236877370208\\
14026.1262361128	0.000753767505555156\\
14031.9863264559	0.000484805484690877\\
14037.8464167989	0.000241817736857821\\
14043.706507142	3.05226269723719e-05\\
14049.566597485	-0.000144112701387935\\
14055.4266878281	-0.000277989081314912\\
14061.2867781712	-0.000367971528561726\\
14067.1468685142	-0.000411962640454991\\
14073.0069588573	-0.000408951469478573\\
14078.8670492003	-0.000359036732912303\\
14084.7271395434	-0.000263423826120395\\
14090.5872298865	-0.000124395725499868\\
14096.4473202295	5.47415169383387e-05\\
14102.3074105726	0.000269737386289097\\
14108.1675009156	0.000515497755686738\\
14114.0275912587	0.000786205582809815\\
14119.8876816018	0.00107545859238555\\
14125.7477719448	0.0013764206862526\\
14131.6078622879	0.00168198350024101\\
14137.4679526309	0.00198493429025324\\
14143.328042974	0.00227812618235113\\
14149.1881333171	0.00255464676876176\\
14155.0482236601	0.00280798107254935\\
14160.9083140032	0.00303216503936423\\
14166.7684043462	0.00322192594002392\\
14172.6284946893	0.00337280637915283\\
14178.4885850324	0.00348126899325907\\
14184.3486753754	0.00354477937927039\\
14190.2087657185	0.00356186530974095\\
14196.0688560615	0.00353215085120681\\
14201.9289464046	0.00345636459554653\\
14207.7890367477	0.00333632182494585\\
14213.6491270907	0.00317488104669863\\
14219.5092174338	0.00297587593821246\\
14225.3693077768	0.00274402432222415\\
14231.2293981199	0.00248481633338656\\
14237.0894884629	0.0022043844268812\\
14242.949578806	0.00190935830668996\\
14248.8096691491	0.00160670820536757\\
14254.6697594921	0.00130358021957292\\
14260.5298498352	0.00100712759141242\\
14266.3899401782	0.000724341918675397\\
14272.2500305213	0.000461888276574824\\
14278.1101208644	0.000225948139063632\\
14283.9702112074	2.20738013458808e-05\\
14289.8303015505	-0.000144942268043085\\
14295.6903918935	-0.000271180071694326\\
14301.5504822366	-0.000353684076786514\\
14307.4105725797	-0.000390532398900328\\
14313.2706629227	-0.000380881556884313\\
14319.1307532658	-0.000324985756845948\\
14324.9908436088	-0.000224190264832648\\
14330.8509339519	-8.08990394676799e-05\\
14336.711024295	0.000101482597034424\\
14342.571114638	0.000318628882536111\\
14348.4312049811	0.00056539621042337\\
14354.2912953241	0.000835944963785444\\
14360.1513856672	0.00112387770259079\\
14366.0114760103	0.00142239043416452\\
14371.8715663533	0.00172443339015354\\
14377.7316566964	0.00202287751120831\\
14383.5917470394	0.00231068270777798\\
14389.4518373825	0.00258106392615789\\
14395.3119277256	0.00282765110239682\\
14401.1720180686	0.00304463923383036\\
14407.0321084117	0.00322692503256452\\
14412.8921987547	0.00337022694442108\\
14418.7522890978	0.0034711857108244\\
14424.6123794408	0.00352744311202012\\
14430.4724697839	0.0035376970457357\\
14436.332560127	0.00350173165535322\\
14442.19265047	0.00342042181080717\\
14448.0527408131	0.00329571185067635\\
14453.9128311561	0.00313056910176388\\
14459.7729214992	0.00292891328709512\\
14465.6330118423	0.00269552350139673\\
14471.4931021853	0.00243592496150328\\
14477.3531925284	0.00215625821548399\\
14483.2132828714	0.0018631339060022\\
14489.0733732145	0.0015634765229513\\
14494.9334635576	0.00126436083839635\\
14500.7935539006	0.000972844886872625\\
14506.6536442437	0.000695803434123047\\
14512.5137345867	0.00043976586314819\\
14518.3738249298	0.000210762300133952\\
14524.2339152729	1.4181606485196e-05\\
14530.0940056159	-0.000145355419131996\\
14535.954095959	-0.000264104645316144\\
14541.814186302	-0.000339286466673613\\
14547.6742766451	-0.000369150855255968\\
14553.5343669882	-0.000353018082501412\\
14559.3944573312	-0.000291294164323109\\
14565.2545476743	-0.000185460678968508\\
14571.1146380173	-3.80392120851743e-05\\
14576.9747283604	0.000147468718030301\\
14582.8348187035	0.000366664823968551\\
14588.6949090465	0.000614358326495787\\
14594.5549993896	0.00088468887352642\\
14600.4150897326	0.00117126519018534\\
14606.2751800757	0.00146731617983447\\
14612.1352704188	0.00176585090394756\\
14617.9953607618	0.00205982366122552\\
14623.8554511049	0.002342300268085\\
14629.7155414479	0.00260662161640171\\
14635.575631791	0.00284656065096442\\
14641.4357221341	0.00305646906617973\\
14647.2958124771	0.00323141026656279\\
14653.1559028202	0.00336727546096407\\
14659.0159931632	0.00346088016106414\\
14664.8760835063	0.00351003881731542\\
14670.7361738493	0.00351361584370634\\
14676.5962641924	0.00347155184018167\\
14682.4563545355	0.00338486440746752\\
14688.3164448785	0.00325562354887474\\
14694.1765352216	0.00308690225354372\\
14700.0366255646	0.00288270344049986\\
14705.8967159077	0.00264786499999256\\
14711.7568062508	0.00238794518504862\\
14717.6168965938	0.00210909106740961\\
14723.4769869369	0.00181789317173546\\
14729.3370772799	0.00152122972457019\\
14735.197167623	0.00122610419944735\\
14741.0572579661	0.00093947999445889\\
14746.9173483091	0.000668116144800087\\
14752.7774386522	0.000418407946166413\\
14758.6375289952	0.000196236246832203\\
14764.4976193383	6.82895969455063e-06\\
14770.3577096814	-0.000145361944579666\\
14776.2178000244	-0.000256765069308663\\
14782.0778903675	-0.000324773389527042\\
14787.9379807105	-0.000347805249640168\\
14793.7980710536	-0.000325341134040087\\
14799.6581613967	-0.000257935356010565\\
14805.5182517397	-0.000147202402636761\\
14811.3783420828	4.22172853864268e-06\\
14817.2384324258	0.000192742275460329\\
14823.0985227689	0.000413891046680078\\
14828.958613112	0.000662432320091626\\
14834.818703455	0.00093248679559564\\
14840.6787937981	0.00121767066731926\\
14846.5388841411	0.00151124652591381\\
14852.3989744842	0.0018062825240459\\
14858.2590648272	0.00209581604510282\\
14864.1191551703	0.00237301801082623\\
14869.9792455134	0.00263135395085735\\
14875.8393358564	0.00286473803519307\\
14881.6994261995	0.00306767643896439\\
14887.5595165425	0.00323539666232488\\
14893.4196068856	0.00336395976184966\\
14899.2796972287	0.00345035285387688\\
14905.1397875717	0.00349255971797196\\
14910.9998779148	0.00348960784607493\\
14916.8599682578	0.00344159083945087\\
14922.7200586009	0.00334966563817136\\
14928.580148944	0.00321602466163508\\
14934.440239287	0.00304384353095849\\
14940.3003296301	0.00283720561916546\\
14946.1604199731	0.00260100522197361\\
14952.0205103162	0.00234083164499458\\
14957.8806006593	0.00206283695243711\\
14963.7406910023	0.00177359050624812\\
14969.6007813454	0.0014799237351317\\
14975.4608716884	0.00118876880093558\\
14981.3209620315	0.000906994972593629\\
14987.1810523746	0.00064124656956867\\
14993.0411427176	0.000397786297766314\\
14998.9012330607	0.000182347671641106\\
nan	nan\\
15010.6214137468	-0.000144970885391611\\
15016.4815040899	-0.000249163352292922\\
15022.3415944329	-0.0003101397746323\\
15028.201684776	-0.000326483551076439\\
15034.061775119	-0.000297832002020913\\
15039.9218654621	-0.000224884381242437\\
15045.7819558052	-0.000109384826114604\\
15051.6420461482	4.59193383597596e-05\\
15057.5021364913	0.000237342945100516\\
15063.3622268343	0.000460350426003838\\
15069.2223171774	0.000709663273851925\\
15075.0824075204	0.000979384979148323\\
15080.9424978635	0.00126314048489069\\
15086.8025882066	0.00155422686134492\\
15092.6626785496	0.00184577163966613\\
15098.5227688927	0.00213089506466031\\
15104.3828592357	0.00240287243599037\\
15110.2429495788	0.00265529270753584\\
15116.1030399219	0.00288220960418514\\
15121.9631302649	0.00307828169475268\\
15127.823220608	0.00323889812103693\\
15133.683310951	0.00336028702397862\\
15139.5434012941	0.00343960411759045\\
15145.4034916372	0.00347499933022859\\
15151.2635819802	0.00346565995275887\\
15157.1236723233	0.00341182928680324\\
15162.9837626663	0.00331480036596573\\
15168.8438530094	0.00317688491076186\\
15174.7039433525	0.00300135826212091\\
15180.5640336955	0.00279238160564078\\
15186.4241240386	0.00255490333257139\\
15192.2842143816	0.00229454187596013\\
15198.1443047247	0.00201745279600168\\
15204.0043950678	0.00173018325839858\\
15209.8644854108	0.00143951734607849\\
15215.7245757539	0.00115231585894787\\
15221.5846660969	0.000875354384118947\\
15227.44475644	0.000615163458716024\\
15233.3048467831	0.000377874595476299\\
15239.1649371261	0.000169075801332674\\
15245.0250274692	-6.32000663148055e-06\\
15250.8851178122	-0.000144190589545165\\
15256.7452081553	-0.000241301260081821\\
15262.6052984984	-0.00029538076439031\\
15268.4653888414	-0.000305174392639081\\
15274.3254791845	-0.000270473076442667\\
15280.1855695275	-0.000192117798626628\\
15286.0456598706	-7.19792236303063e-05\\
15291.9057502136	8.7086958430274e-05\\
15297.7658405567	0.000281307908173864\\
15303.6259308998	0.000506083120695707\\
15309.4860212428	0.000756093394389925\\
15315.3461115859	0.00102542670394675\\
15321.2062019289	0.00130771799906446\\
15327.066292272	0.00159629962321549\\
15332.9263826151	0.00188435879806841\\
15338.7864729581	0.00216509845392739\\
15344.6465633012	0.00243189760977451\\
15350.5066536442	0.00267846751867429\\
15356.3667439873	0.00289899989653999\\
15362.2268343304	0.00308830374023165\\
15368.0869246734	0.00324192751253394\\
15373.9470150165	0.00335626381843553\\
15379.8071053595	0.00342863411070367\\
15385.6671957026	0.00345735143676508\\
15391.5272860457	0.00344175975666725\\
15397.3873763887	0.00338224891582324\\
15403.2474667318	0.00328024493180674\\
15409.1075570748	0.00313817583574399\\
15414.9676474179	0.00295941388710672\\
15420.827737761	0.0027481955362655\\
15426.687828104	0.0025095210349352\\
15432.5479184471	0.00224903607266765\\
15438.4080087901	0.00197289824158901\\
15444.2680991332	0.00168763148727556\\
15450.1281894763	0.00139997198631395\\
15455.9882798193	0.00111670909117421\\
15461.8483701624	0.000844525097901928\\
15467.7084605054	0.000589837618094773\\
15473.5685508485	0.000358648273218747\\
15479.4286411916	0.000156401278303213\\
15485.2887315346	-1.21447528169376e-05\\
15491.1488218777	-0.000143028762349835\\
15497.0089122207	-0.000233180329688468\\
15502.8690025638	-0.000280491691646587\\
15508.7290929068	-0.000283867012633818\\
15514.5891832499	-0.000243247753188978\\
15520.449273593	-0.000159613549862391\\
15526.309363936	-3.49585975311358e-05\\
15532.1694542791	0.000127755896776441\\
15538.0295446221	0.000324672055058073\\
15543.8896349652	0.000551126793784171\\
15549.7497253083	0.000801762245226182\\
15555.6098156513	0.00107065252003196\\
15561.4699059944	0.00135144381209565\\
15567.3299963374	0.00163750453345362\\
15573.1900866805	0.00192208193174272\\
15579.0501770236	0.00219846149160917\\
15584.9102673666	0.00246012535751445\\
15590.7703577097	0.00270090604026635\\
15596.6304480527	0.00291513178258332\\
15602.4905383958	0.0030977601577299\\
15608.3506287389	0.0032444967545569\\
15614.2107190819	0.00335189615523155\\
15620.070809425	0.00341744283179781\\
15625.930899768	0.00343961006209267\\
15631.7909901111	0.00341789548531318\\
15637.6510804542	0.00335283246943439\\
15643.5111707972	0.00324597703373892\\
15649.3712611403	0.00309987064662115\\
15655.2313514833	0.00291797978783883\\
15661.0914418264	0.00270461371255668\\
15666.9515321695	0.0024648223675902\\
15672.8116225125	0.00220427687848571\\
15678.6717128556	0.00192913543522503\\
15684.5318031986	0.00164589774830985\\
15690.3918935417	0.00136125151483173\\
15696.2519838848	0.00108191452100489\\
15702.1120742278	0.000814476109168386\\
15707.9721645709	0.000565241750036063\\
15713.8322549139	0.000340084386486619\\
15719.692345257	0.000144306053242435\\
15725.5524356	-1.74869670384477e-05\\
15731.4125259431	-0.000141492511864545\\
15737.2726162862	-0.000224801881802328\\
15743.1327066292	-0.000265468058943033\\
15748.9927969723	-0.000262551201128199\\
15754.8528873153	-0.00021614034906893\\
15760.7129776584	-0.000127350845271362\\
15766.5730680015	1.70246302675327e-06\\
15772.4331583445	0.000167955593274787\\
15778.2932486876	0.000367468169909713\\
15784.1533390306	0.000595516812678381\\
15790.0134293737	0.000846706957782417\\
15795.8735197168	0.00111510046472146\\
15801.7336100598	0.00139435599053467\\
15807.5937004029	0.00167787881306794\\
15813.4537907459	0.00195897656392919\\
15819.313881089	0.00223101719329131\\
15825.1739714321	0.00248758543855506\\
15831.0340617751	0.00272263410518938\\
15836.8941521182	0.00293062659267823\\
15842.7542424612	0.00310666730637738\\
15848.6143328043	0.00324661688391024\\
15854.4744231474	0.00334718952380148\\
15860.3345134904	0.00340603012836392\\
15866.1946038335	0.0034217694496319\\
15872.0546941765	0.00339405594718373\\
15877.9147845196	0.00332356361703518\\
15883.7748748627	0.00321197561753751\\
15889.6349652057	0.0030619440901945\\
15895.4950555488	0.00287702713371492\\
15901.3551458918	0.00266160442981069\\
15907.2152362349	0.00242077352038266\\
15913.0753265779	0.00216022919334625\\
15918.935416921	0.00188612883015765\\
15924.7955072641	0.0016049468990927\\
15930.6555976071	0.00132332203250974\\
15936.5156879502	0.00104790029991056\\
15942.3757782932	0.000785178376656384\\
15948.2358686363	0.000541350309089475\\
15954.0959589794	0.000322161490170081\\
15959.9560493224	0.000132773287949975\\
15965.8161396655	-2.23584835662776e-05\\
15971.6762300085	-0.000139588389904397\\
15977.5363203516	-0.000216167031927556\\
15983.3964106947	-0.000250305519499976\\
15989.2565010377	-0.000241217251211042\\
15995.1165913808	-0.000189136024533497\\
16000.9766817238	-9.53100597599009e-05\\
16006.8367720669	3.80279104543498e-05\\
16012.69686241	0.000207713769378205\\
16018.556952753	0.000409727098285845\\
16024.4170430961	0.000639286430811113\\
16030.2771334391	0.000890962422844709\\
16036.1372237822	0.00115880625959995\\
16041.9973141253	0.00143649026307191\\
16047.8574044683	0.00171745737655641\\
16053.7174948114	0.00199507599508508\\
16059.5775851544	0.00226279648579603\\
16065.4376754975	0.00251430570430218\\
16071.2977658406	0.00274367586196693\\
16077.1578561836	0.00294550423353621\\
16083.0179465267	0.00311504041315398\\
16088.8780368697	0.00324829812088709\\
16094.7381272128	0.00334214892908698\\
16100.5982175559	0.00339439570411013\\
16106.4583078989	0.0034038240405048\\
16112.318398242	0.00337023048262889\\
16118.178488585	0.00329442687901623\\
16124.0385789281	0.0031782207765661\\
16129.8986692712	0.00302437232816796\\
16135.7587596142	0.00283652874096806\\
16141.6188499573	0.00261913782143336\\
16147.4789403003	0.00237734266739566\\
16153.3390306434	0.00211685999924417\\
16159.1991209864	0.00184384500913134\\
16165.0592113295	0.00156474592294067\\
16170.9193016726	0.00128615171126053\\
16176.7793920156	0.0010146365460701\\
16182.6394823587	0.000756604674649038\\
16188.4995727017	0.000518139371084741\\
16194.3596630448	0.000304859527714513\\
16200.2197533879	0.000121787267190864\\
16206.0798437309	-2.67703058012217e-05\\
16211.939934074	-0.000137322429085251\\
16217.800024417	-0.000207276700249463\\
16223.6601147601	-0.000234999859707407\\
16229.5202051032	-0.000219855914514827\\
16235.3802954462	-0.000162220713228812\\
16241.2403857893	-6.34726380712965e-05\\
16247.1004761323	7.40403555731246e-05\\
16252.9605664754	0.000247056564347153\\
16258.8206568185	0.000451477899609607\\
16264.6807471615	0.000682466952817851\\
16270.5408375046	0.000934561464678312\\
16276.4009278476	0.00120180348947253\\
16282.2610181907	0.00147788020042007\\
16288.1211085338	0.00175627300844687\\
16293.9811988768	0.0020304114721701\\
16299.8412892199	0.00229382836536761\\
16305.7013795629	0.00254031224206561\\
16311.561469906	0.0027640539006836\\
16317.4215602491	0.00295978329357303\\
16323.2816505921	0.00312289365547601\\
16329.1417409352	0.00324954992734402\\
16335.0018312782	0.00333677892398197\\
16340.8619216213	0.00338253912540536\\
16346.7220119643	0.00338576845392555\\
16352.5821023074	0.0033464089189049\\
16358.4421926505	0.00326540755775076\\
16364.3022829935	0.00314469366049456\\
16370.1623733366	0.00298713282620357\\
16376.0224636796	0.00279645894443526\\
16381.8825540227	0.00257718571678646\\
16387.7426443658	0.00233449981442621\\
16393.6027347088	0.00207413820114466\\
16399.4628250519	0.00180225252264397\\
16405.3229153949	0.0015252637683092\\
16411.183005738	0.00124971063835898\\
16417.0430960811	0.000982095197569767\\
16422.9031864241	0.000728729458506226\\
16428.7632767672	0.000495586514094176\\
16434.6233671102	0.000288159730377807\\
16440.4834574533	0.000111333318655286\\
16446.3435477964	-3.07326637172032e-05\\
16452.2036381394	-0.000134700176321897\\
16458.0637284825	-0.000198131620406222\\
16463.9238188255	-0.000219546982932227\\
16469.7839091686	-0.000198458360456623\\
16475.6439995117	-0.000135381057646192\\
16481.5040898547	-3.18210083328037e-05\\
16487.3641801978	0.000109761174163993\\
16493.2242705408	0.000286008659417088\\
16499.0843608839	0.000492747986061979\\
16504.944451227	0.000725087884969616\\
16510.80454157	0.000977534999517035\\
16516.6646319131	0.0012441237653532\\
16522.5247222561	0.0015185573789269\\
16528.3848125992	0.00179435652390311\\
16534.2449029423	0.00206501234264672\\
16540.1049932853	0.00232414004163137\\
16545.9650836284	0.00256562950545911\\
16551.8251739714	0.00278378936732268\\
16557.6852643145	0.00297348113850494\\
16563.5453546575	0.00313024023616807\\
16569.4054450006	0.00325038105919412\\
16575.2655353437	0.00333108363872912\\
16581.1256256867	0.00337045982670993\\
16586.9857160298	0.00336759746885296\\
16592.8458063728	0.00332258152877746\\
16598.7058967159	0.00323649167421322\\
16604.565987059	0.00311137639197449\\
16610.426077402	0.00295020425211911\\
16616.2861677451	0.00275679348066917\\
16622.1462580881	0.00253572151148781\\
16628.0063484312	0.00229221665998899\\
16633.8664387743	0.00203203448196694\\
16639.7265291173	0.0017613217414424\\
16645.5866194604	0.00148647120241221\\
16651.4467098034	0.00121397067475159\\
16657.3068001465	0.000950249878674834\\
16663.1668904896	0.000701528742090925\\
16669.0269808326	0.000473670710044766\\
16674.8870711757	0.000272044525574439\\
16680.7471615187	0.000101397740207191\\
16686.6072518618	-3.42550660060854e-05\\
16692.4673422049	-0.000131726723126663\\
16698.3274325479	-0.000188732347219367\\
16704.187522891	-0.00020394289452982\\
16710.047613234	-0.000177016138859087\\
16715.9077035771	-0.000108604350224984\\
16721.7677939202	-3.38503015062493e-07\\
16727.6278842632	0.000145210604301558\\
16733.4879746063	0.000324593391174183\\
16739.3480649493	0.00053356324939121\\
16745.2081552924	0.000767177072449169\\
16751.0682456355	0.00101991218018234\\
16756.9283359785	0.00128579687304424\\
16762.7884263216	0.00155855152981105\\
16768.6485166646	0.00183173691520315\\
16774.5086070077	0.00209890619477049\\
16780.3686973508	0.00235375706848653\\
16786.2287876938	0.00259028043329555\\
16792.0888780369	0.00280290206770192\\
16797.9489683799	0.00298661399816999\\
16803.809058723	0.00313709245131778\\
16809.669149066	0.00325079961408492\\
16815.5292394091	0.00332506680697236\\
16821.3893297522	0.00335815711499183\\
16827.2494200952	0.00334930600689151\\
16833.1095104383	0.00329873899231712\\
16838.9696007813	0.00320766590989548\\
16844.8296911244	0.00307825198979473\\
16850.6897814675	0.00291356638295905\\
16856.5498718105	0.0027175093805363\\
16862.4099621536	0.00249472004890933\\
16868.2700524966	0.00225046646811791\\
16874.1301428397	0.00199052117006053\\
16879.9902331828	0.00172102472175951\\
16885.8503235258	0.0014483406774797\\
16891.7104138689	0.0011789053255574\\
16897.5705042119	0.000919075777729024\\
16903.430594555	0.000674979986037294\\
16909.2906848981	0.000452372225917957\\
16915.1507752411	0.000256497453378197\\
16921.0108655842	9.19677336562851e-05\\
16926.8709559272	-3.73463474530113e-05\\
16932.7310462703	-0.000128406733021716\\
16938.5911366134	-0.000179079263506846\\
16944.4512269564	-0.000188183687877569\\
16950.3113172995	-0.000155521145550106\\
16956.1714076425	-8.18784793205741e-05\\
16962.0314979856	3.09907134717422e-05\\
16967.8915883287	0.000180407835448466\\
16973.7516786717	0.000362832855259908\\
16979.6117690148	0.000573948176819946\\
16985.4718593578	0.000808760824820732\\
16991.3319497009	0.0010617205281706\\
16997.1920400439	0.00132685090883612\\
17003.052130387	0.00159789067534705\\
17008.9122207301	0.00186844148536767\\
17014.7723110731	0.0021321189855739\\
17020.6324014162	0.00238270346375212\\
17026.4924917592	0.00261428655787328\\
17032.3525821023	0.00282141056221078\\
17038.2126724454	0.00299919704571256\\
17044.0727627884	0.00314346175192286\\
17049.9328531315	0.00325081307430903\\
17055.7929434745	0.00331873178943838\\
17061.6530338176	0.00334563017360636\\
17067.5131241607	0.0033308891162053\\
17073.3732145037	0.0032748723616779\\
17079.2333048468	0.00317891755331364\\
17085.0933951898	0.00304530429881759\\
17090.9534855329	0.00287720001974668\\
17096.813575876	0.00267858487124884\\
17102.673666219	0.00245415751154037\\
17108.5337565621	0.00220922395216201\\
17114.3938469051	0.00194957211792032\\
17120.2539372482	0.00168133508208871\\
17126.1140275913	0.00141084620853153\\
17131.9741179343	0.00114448962183809\\
17137.8342082774	0.000888549535702041\\
17143.6942986204	0.000649061995683265\\
17149.5543889635	0.000431672533579973\\
17155.4144793066	0.000241503090366643\\
17161.2745696496	8.30313444355445e-05\\
17167.1346599927	-4.00147120079053e-05\\
17172.9947503357	-0.000124744466343105\\
17178.8548406788	-0.000169172586033585\\
17184.7149310219	-0.00017226553134028\\
17190.5750213649	-0.000133965590637644\\
17196.435111708	-5.51918795326886e-05\\
17202.295202051	6.21817112289388e-05\\
17208.1552923941	0.000215371090139906\\
17214.0153827372	0.000400748001395787\\
17219.8754730802	0.000613925957206676\\
17225.7355634233	0.000849864030840129\\
17231.5956537663	0.00110298605450711\\
17237.4557441094	0.00136731240367987\\
17243.3158344524	0.00163660125346903\\
17249.1759247955	0.00190449597017936\\
17255.0360151386	0.00216467515785807\\
17260.8961054816	0.00241100181814975\\
17266.7561958247	0.00263766810384271\\
17272.6162861677	0.00283933225223779\\
17278.4763765108	0.00301124446952437\\
17284.3364668539	0.00314935880011472\\
17290.1965571969	0.00325042834598547\\
17296.05664754	0.00331208159513595\\
17301.916737883	0.0033328780652403\\
17307.7768282261	0.00331234195655322\\
17313.6369185692	0.00325097302838986\\
17319.4970089122	0.00315023445060586\\
17325.3570992553	0.00301251792512637\\
17331.2171895983	0.0028410869089278\\
17337.0772799414	0.00263999928608769\\
17342.9373702845	0.00241401132129086\\
17348.7974606275	0.00216846516790967\\
17354.6575509706	0.00190916259096835\\
17360.5176413136	0.00164222789021971\\
17366.3777316567	0.00137396326148658\\
17372.2378219998	0.00111070001227798\\
17378.0979123428	0.000858649144152956\\
17383.9580026859	0.000623754827763278\\
17389.8180930289	0.000411554227368581\\
17395.678183372	0.000227046980089\\
17401.5382737151	7.45774065927705e-05\\
17407.3983640581	-4.22677719732014e-05\\
17413.2584544012	-0.00012074380267877\\
17419.1185447442	-0.000159012370673364\\
17424.9786350873	-0.000156184656030153\\
17430.8387254303	-0.00011234196916861\\
17436.6988157734	-2.85334859471857e-05\\
17442.5589061165	9.32488559845947e-05\\
17448.4189964595	0.000250117699026392\\
17454.2790868026	0.00043835872060944\\
17460.1391771456	0.000653518578438651\\
17465.9992674887	0.000890510263788304\\
17471.8593578318	0.00114373337056256\\
17477.7194481748	0.00140720643686414\\
17483.5795385179	0.00167470823173893\\
17489.4396288609	0.0019399246500727\\
17495.299719204	0.0021965977470681\\
17501.1598095471	0.00243867339507389\\
17507.0198998901	0.0026604440784261\\
17512.8799902332	0.00285668345889799\\
17518.7400805762	0.00302276953889241\\
17524.6001709193	0.00315479352014854\\
17530.4602612624	0.00324965179432829\\
17536.3203516054	0.00330511890044406\\
17542.1804419485	0.00331989973423141\\
17548.0405322915	0.00329365978488761\\
17553.9006226346	0.00322703269307582\\
17559.7607129777	0.00312160496001904\\
17565.6208033207	0.00297987817644177\\
17571.4808936638	0.0028052096702421\\
17577.3409840068	0.00260173298152088\\
17583.2010743499	0.00237426004794859\\
17589.061164693	0.00212816741562344\\
17594.921255036	0.00186926916557996\\
17600.7813453791	0.0016036795596945\\
17606.6414357221	0.00133766865039903\\
17612.5015260652	0.00107751426396015\\
17618.3616164083	0.000829353851739215\\
17624.2217067513	0.00059903970494476\\
17630.0817970944	0.000392000948702619\\
17635.9418874374	0.000213115569519962\\
17641.8019777805	6.65954925906442e-05\\
17647.6620681235	-4.41125836659883e-05\\
17653.5221584666	-0.000116408261157179\\
17659.3822488097	-0.000148598516845465\\
17665.2423391527	-0.000139937344294651\\
17671.1024294958	-9.06430339274512e-05\\
17676.9625198388	-1.89269188114981e-06\\
17682.8226101819	0.000124205865406549\\
17688.682700525	0.000284664169936385\\
17694.542790868	0.000475683925448585\\
17700.4028812111	0.000692746916920503\\
17706.2629715541	0.00093072187814373\\
17712.1230618972	0.00118398578974072\\
17717.9831522403	0.00144655674036633\\
17723.8432425833	0.00171223521203553\\
17729.7033329264	0.00197475045254338\\
17735.5634232694	0.00222790847940802\\
17741.4235136125	0.00246573822198747\\
17747.2836039556	0.00268263235409453\\
17753.1436942986	0.00287347949522077\\
17759.0037846417	0.00303378466396183\\
17764.8638749847	0.0031597751448728\\
17770.7239653278	0.0032484892758537\\
17776.5840556709	0.00329784606602108\\
17782.4441460139	0.00330669400845668\\
17788.304236357	0.00327483794213662\\
17794.1643267	0.00320304333721315\\
17800.0244170431	0.00309301790968213\\
17805.8845073862	0.0029473710069704\\
17811.7445977292	0.00276955173020013\\
17817.6046880723	0.00256376726082426\\
17823.4647784153	0.00233488332481833\\
17829.3248687584	0.00208830914989589\\
17835.1849591015	0.0018298696352209\\
17841.0450494445	0.00156566775460915\\
17846.9051397876	0.00130194044328263\\
17852.7652301306	0.00104491137126863\\
17858.6253204737	0.000800644078546268\\
17864.4854108167	0.000574898937514028\\
17870.3455011598	0.000372997317038402\\
17876.2055915029	0.000199696150929326\\
17882.0656818459	5.90758674657438e-05\\
17887.925772189	-4.55556798860067e-05\\
17893.785862532	-0.000111741018777535\\
17899.6459528751	-0.000137930771273336\\
17905.5060432182	-0.000123519918815147\\
17911.3661335612	-6.88617701286397e-05\\
17917.2262239043	2.47406902264884e-05\\
17923.0863142473	0.000155065861107603\\
17928.9464045904	0.000319026251571868\\
17934.8064949335	0.000512741623865245\\
17940.6665852765	0.000731630819955942\\
17946.5266756196	0.000970520098580096\\
17952.3867659626	0.00122376542092549\\
17958.2468563057	0.00148538579471838\\
17964.1069466488	0.00174920452650565\\
17969.9670369918	0.00200899504622712\\
17975.8271273349	0.00225862786163019\\
17981.6872176779	0.0024922151740684\\
17987.547308021	0.00270424974416834\\
17993.4073983641	0.00288973473190572\\
17999.2674887071	0.00304430145036805\\
18005.1275790502	0.00316431225816866\\
18010.9876693932	0.00324694616736504\\
18016.8477597363	0.00329026515224825\\
18022.7078500794	0.00329325960037216\\
18028.5679404224	0.00325587184018219\\
18034.4280307655	0.00317899719677166\\
18040.2881211085	0.00306446255839037\\
18046.1482114516	0.00291498296643236\\
18052.0083017947	0.00273409726046745\\
18057.8683921377	0.00252608430355498\\
18063.7284824808	0.00229586177103383\\
18069.5885728238	0.00204886989662629\\
18075.4486631669	0.00179094292414862\\
18081.3087535099	0.00152817130194673\\
18087.168843853	0.00126675787536547\\
18093.0289341961	0.00101287147215145\\
18098.8890245391	0.000772501337282419\\
18104.7491148822	0.000551315851514117\\
18110.6092052252	0.000354528866579199\\
18116.4692955683	0.000186776808663174\\
18122.3293859114	5.20094469453597e-05\\
18128.1894762544	-4.6603099474696e-05\\
18134.0495665975	-0.000106744926945176\\
18139.9096569405	-0.000127008731112874\\
18145.7697472836	-0.000106928732249591\\
18151.6298376267	-4.69913718307363e-05\\
18157.4899279697	5.13764650934753e-05\\
18163.3500183128	0.00018584141676491\\
18169.2101086558	0.000353218992327754\\
18175.0701989989	0.000549548987395443\\
18180.930289342	0.000770189181675641\\
18186.790379685	0.00100992510187001\\
18192.6504700281	0.00126309325460047\\
18198.5105603711	0.00152371491722304\\
18204.3706507142	0.00178563732594442\\
18210.2307410573	0.00204267892724433\\
18216.0908314003	0.00228877526361677\\
18221.9509217434	0.00251812205101473\\
18227.8110120864	0.00272531207219984\\
18233.6711024295	0.00290546265778688\\
18239.5311927726	0.00305433074932621\\
18245.3912831156	0.00316841283337592\\
18251.2513734587	0.00324502739241868\\
18257.1114638017	0.00328237793273344\\
18262.9715541448	0.00327959510792965\\
18268.8316444879	0.00323675694993573\\
18274.6917348309	0.00315488673757002\\
18280.551825174	0.00303592855912138\\
18286.411915517	0.00288270115277156\\
18292.2720058601	0.00269883112095715\\
18298.1320962031	0.00248866710038391\\
18303.9921865462	0.00225717691971971\\
18309.8522768893	0.00200983017628503\\
18315.7123672323	0.00175246900762923\\
18321.5724575754	0.00149117011098425\\
18327.4325479184	0.00123210126927024\\
18333.2926382615	0.000981375771129694\\
18339.1527286046	0.000744908160966237\\
18345.0128189476	0.000528274722729892\\
18350.8729092907	0.000336581988190763\\
18356.7329996337	0.000174346370379179\\
18362.5930899768	4.53877591612407e-05\\
18368.4531803199	-4.72604142253817e-05\\
18374.3132706629	-0.000101422526372739\\
18380.173361006	-0.000115831846490102\\
18386.033451349	-9.01601573534626e-05\\
18391.8935416921	-2.50252198610244e-05\\
18397.7536320352	7.80240883375843e-05\\
18403.6137223782	0.000216544602743797\\
18409.4738127213	0.000387256794770521\\
18415.3339030643	0.000586122414200106\\
18421.1939934074	0.000808440013231233\\
18427.0540837505	0.00104895609258157\\
18432.9141740935	0.00130198924221325\\
18438.7742644366	0.00156156434323884\\
18444.6343547796	0.00182155366113274\\
18450.4944451227	0.00207582149875784\\
18456.3545354658	0.00231836899430911\\
18462.2146258088	0.00254347564764448\\
18468.0747161519	0.00274583423561655\\
18473.9348064949	0.00292067593503878\\
18479.794896838	0.00306388270339354\\
18485.654987181	0.00317208426875\\
18491.5150775241	0.00324273744517434\\
18497.3751678672	0.00327418590659551\\
18503.2352582102	0.00326569901448781\\
18509.0953485533	0.00321748878943333\\
18514.9554388963	0.00313070463210016\\
18520.8155292394	0.00300740592507665\\
18526.6756195825	0.00285051316814547\\
18532.5357099255	0.00266373880694001\\
18538.3958002686	0.00245149939270799\\
18544.2558906116	0.00221881115160648\\
18550.1159809547	0.0019711714330792\\
18555.9760712978	0.00171442883853689\\
18561.8361616408	0.00145464509873329\\
18567.6962519839	0.00119795196145568\\
18573.5563423269	0.000950406468306214\\
18579.41643267	0.000717848036250807\\
18585.2765230131	0.000505760715971023\\
18591.1366133561	0.000319143876072086\\
18596.9967036992	0.000162394362353604\\
18602.8567940422	3.92029096404471e-05\\
18608.7168843853	-4.75327533708477e-05\\
18614.5769747284	-9.57760604676606e-05\\
18620.4370650714	-0.000104399422478881\\
18626.2971554145	-7.32105775005867e-05\\
18632.1572457575	-2.956861103588e-06\\
18638.0173361006	0.000104692697937652\\
18643.8774264437	0.000247187027568191\\
18649.7375167867	0.000421153466158602\\
18655.5976071298	0.000622477587468703\\
18661.4576974728	0.000846400507645332\\
18667.3177878159	0.00108763137293946\\
18673.177878159	0.00134047236945581\\
18679.037968502	0.00159895330114962\\
18684.8980588451	0.00185697255779655\\
18690.7581491881	0.00210844114424735\\
18696.6182395312	0.002347426371543\\
18702.4783298743	0.00256829181876632\\
18708.3384202173	0.00276583026410804\\
18714.1985105604	0.00293538644997722\\
18720.0586009034	0.00307296678813353\\
18725.9186912465	0.00317533341923403\\
18731.7787815895	0.0032400804121195\\
18737.6388719326	0.00326569030914807\\
18743.4989622757	0.0032515696888833\\
18749.3590526187	0.00319806291285394\\
18755.2191429618	0.00310644373782046\\
18761.0792333048	0.00297888499786085\\
18766.9393236479	0.00281840707790009\\
18772.799413991	0.0026288064000839\\
18778.659504334	0.00241456561663699\\
18784.5195946771	0.00218074763341678\\
18790.3796850201	0.00193287596938492\\
18796.2397753632	0.0016768042792222\\
18802.0998657063	0.00141857812115274\\
18807.9599560493	0.00116429223417216\\
18813.8200463924	0.000919946693924104\\
18819.6801367354	0.000691305342053608\\
18825.5402270785	0.000483759829222655\\
18831.4003174216	0.000302202478736994\\
18837.2604077646	0.000150910968479387\\
18843.1204981077	3.3447549295294e-05\\
18848.9805884507	-4.74248258504103e-05\\
18854.8406787938	-8.98074873347542e-05\\
18860.7007691369	-9.27106205544781e-05\\
18866.5608594799	-5.60763775477899e-05\\
18872.420949823	1.92200110015848e-05\\
18878.281040166	0.00013139114364504\\
18884.1411305091	0.00027777987656005\\
18890.0012208522	0.000454922265381188\\
18895.8613111952	0.000658629529578646\\
18901.7214015383	0.00088408709999427\\
18907.5814918813	0.00112596840758451\\
18913.4415822244	0.00137856072409014\\
18919.3016725674	0.00163590008177391\\
18925.1617629105	0.00189191208585699\\
18931.0218532536	0.00214055529512317\\
18936.8819435966	0.0023759637864148\\
18942.7420339397	0.00259258553886001\\
18948.6021242827	0.00278531337337056\\
18954.4622146258	0.00294960535948395\\
18960.3223049689	0.00308159185063536\\
18966.1823953119	0.00317816662590684\\
18972.042485655	0.00323705999149077\\
18977.902575998	0.00325689212143291\\
18983.7626663411	0.00323720538448551\\
18989.6227566842	0.00317847489959295\\
18995.4828470272	0.00308209707643156\\
19001.3429373703	0.00295035641776533\\
19007.2030277133	0.00278637137247749\\
19013.0631180564	0.0025940205224798\\
19018.9232083995	0.00237785085088631\\
19024.7832987425	0.00214297026078808\\
19030.6433890856	0.00189492688498233\\
19036.5034794286	0.00163957803855868\\
19042.3635697717	0.00138295190946436\\
19048.2236601148	0.00113110525273996\\
19054.0837504578	0.000889980447925988\\
19059.9438408009	0.000665265292906495\\
19065.8039311439	0.00046225884215683\\
19071.664021487	0.000285746453934804\\
19077.5241118301	0.000139886992642582\\
19083.3842021731	2.81148451499311e-05\\
19089.2442925162	-4.69409405442995e-05\\
19095.1043828592	-8.35184904910721e-05\\
19100.9644732023	-8.07644595356537e-05\\
19106.8245635454	-3.87539349948621e-05\\
19112.6846538884	4.15115748822309e-05\\
19118.5447442315	0.000158128014557549\\
19124.4048345745	0.000308333947919791\\
19130.2649249176	0.000488575946729301\\
19136.1250152607	0.000694592652493112\\
19141.9851056037	0.00092151552327798\\
19147.8451959468	0.00116398388359977\\
19153.7052862898	0.0014162715587859\\
19159.5653766329	0.00167242210244262\\
19165.4254669759	0.0019263894235084\\
19171.285557319	0.00217218049308533\\
19177.1456476621	0.00240399676273911\\
19183.0057380051	0.00261637095717588\\
19188.8658283482	0.00280429601460806\\
19194.7259186912	0.00296334313373728\\
19200.5860090343	0.00308976614442579\\
19206.4460993774	0.00318058974261593\\
19212.3061897204	0.00323367951106862\\
19218.1662800635	0.00324779207856016\\
19224.0263704065	0.00322260423823502\\
19229.8864607496	0.00315872034386454\\
19235.7465510927	0.00305765781455272\\
19241.6066414357	0.00292181109561528\\
19247.4667317788	0.00275439493145204\\
19253.3268221218	0.00255936829414784\\
19259.1869124649	0.00234134076831363\\
19265.047002808	0.00210546360503902\\
19270.907093151	0.00185730802041086\\
19276.7671834941	0.00160273361346353\\
19282.6272738371	0.00134775001116313\\
19288.4873641802	0.0010983750073472\\
19294.3474545233	0.00086049254402621\\
19300.2075448663	0.000639713886660138\\
19306.0676352094	0.000441245268687923\\
19311.9277255524	0.000269765127162093\\
19317.7878158955	0.000129313824193651\\
19323.6479062386	2.3198453580295e-05\\
19329.5079965816	-4.60850246296429e-05\\
19335.3680869247	-7.69104883892955e-05\\
19341.2281772677	-6.85598160528999e-05\\
19347.0882676108	-2.12396113792094e-05\\
19352.9483579539	6.39238983051875e-05\\
19358.8084482969	0.000184911665046173\\
19364.66853864	0.000338859686494413\\
19370.528628983	0.000522126800690162\\
19376.3887193261	0.000730380804708862\\
19382.2488096691	0.000958700860340368\\
19388.1089000122	0.00120169376626511\\
19393.9689903553	0.00145362134940467\\
19399.8290806983	0.0017085359665373\\
19405.6891710414	0.00196042091644812\\
19411.5492613844	0.00220333244790632\\
19417.4093517275	0.0024315400118711\\
19423.2694420706	0.002639661448472\\
19429.1295324136	0.00282278991991434\\
19434.9896227567	0.00297660959547147\\
19440.8497130997	0.00309749736159997\\
19446.7098034428	0.00318260816004539\\
19452.5698937859	0.00322994194408079\\
19458.4299841289	0.00323839067706191\\
19464.290074472	0.00320776426904855\\
19470.150164815	0.00313879484447176\\
19476.0102551581	0.00303311924507743\\
19481.8703455012	0.00289324018648735\\
19487.7304358442	0.00272246699002114\\
19493.5905261873	0.00252483729308094\\
19499.4506165303	0.00230502159057735\\
19505.3107068734	0.00206821286369505\\
19511.1707972165	0.00182000390460429\\
19517.0308875595	0.00156625523458023\\
19522.8909779026	0.00131295673521743\\
19528.7510682456	0.00106608625917946\\
19534.6111585887	0.000831468558037372\\
19540.4712489318	0.000614637856179421\\
19546.3313392748	0.000420707313197848\\
19552.1914296179	0.00025424845348466\\
19558.0515199609	0.000119183406251231\\
19563.911610304	1.869249586499e-05\\
19569.7717006471	-4.48606402078331e-05\\
19575.6317909901	-6.99846428289845e-05\\
19581.4918813332	-5.60954245490624e-05\\
19587.3519716762	-3.5297438650051e-06\\
19593.2120620193	8.64629550001872e-05\\
19599.0721523623	0.000211750239203305\\
19604.9322427054	0.00036936721547464\\
19610.7923330485	0.000555586692197041\\
19616.6524233915	0.000766007315066848\\
19622.5125137346	0.000995657592319269\\
19628.3726040776	0.00123911335097204\\
19634.2326944207	0.0014906258491366\\
19640.0927847638	0.00174425751862979\\
19645.9528751068	0.00199402213284423\\
19651.8129654499	0.00223402609081966\\
19657.6730557929	0.00245860748341024\\
19663.533146136	0.00266246965982387\\
19669.3932364791	0.00284080614424192\\
19675.2533268221	0.00298941395599161\\
19681.1134171652	0.0031047926621266\\
19686.9735075082	0.00318422682797309\\
19692.8335978513	0.00322584992333993\\
19698.6936881944	0.00322868818089843\\
19704.5537785374	0.0031926833759882\\
19710.4138688805	0.00311869399495326\\
19716.2739592235	0.00300847476949458\\
19722.1340495666	0.00286463506467592\\
19727.9941399097	0.00269057710737819\\
19733.8542302527	0.00249041551766442\\
19739.7143205958	0.00226888004578474\\
19745.5744109388	0.00203120381427884\\
19751.4345012819	0.00178299970573255\\
19757.294591625	0.00153012781571654\\
19763.154681968	0.0012785571011974\\
19769.0147723111	0.00103422449039854\\
19774.8748626541	0.000802894779875644\\
19780.7349529972	0.000590024624599525\\
19786.5950433403	0.000400633830089154\\
19792.4551336833	0.000239186982342586\\
19798.3152240264	0.000109488206609079\\
19804.1753143694	1.45915358610645e-05\\
19810.0354047125	-4.32709993287928e-05\\
19815.8954950555	-6.27418663285009e-05\\
19821.7555853986	-4.33698768218519e-05\\
19827.6156757417	1.43793630239751e-05\\
19833.4757660847	0.000109134640544155\\
19839.3358564278	0.000238651693966943\\
19845.1959467708	0.000399866366243662\\
19851.0560371139	0.000588967096438991\\
19856.916127457	0.000801485033775356\\
19862.7762178	0.00103239964388067\\
19868.6363081431	0.00127625731164266\\
19874.4963984861	0.00152730013897931\\
19880.3564888292	0.00177960189582683\\
19886.2165791723	0.00202720791431624\\
19892.0766695153	0.00226427562413767\\
19897.9367598584	0.00248521241217128\\
19903.7968502014	0.00268480755391253\\
19909.6569405445	0.00285835510397872\\
19915.5170308876	0.00300176484823732\\
19921.3771212306	0.00311165870084367\\
19927.2372115737	0.00318545027524024\\
19933.0973019167	0.00322140575420832\\
19938.9573922598	0.00321868462698279\\
19944.8174826029	0.00317735933591647\\
19950.6775729459	0.00309841337376431\\
19956.537663289	0.00298371788102505\\
19962.397753632	0.00283598730003151\\
19968.2578439751	0.00265871513697177\\
19974.1179343182	0.00245609135167552\\
19979.9780246612	0.00223290332879933\\
19985.8381150043	0.00199442277095076\\
19991.6982053473	0.00174628118556276\\
19997.5582956904	0.0014943369067839\\
20003.4183860334	0.00124453679189974\\
20009.2784763765	0.00100277585766498\\
20015.1385667196	0.000774758169158772\\
20020.9986570626	0.000565862263930092\\
20026.8587474057	0.000381014286463652\\
20032.7188377487	0.000224571825151499\\
20038.5789280918	0.000100221191034979\\
20044.4390184349	1.08905596460874e-05\\
20050.2991087779	-4.1318977528854e-05\\
20056.159199121	-5.51828285233295e-05\\
20062.019289464	-3.03816211365579e-05\\
20067.8793798071	3.24914455642226e-05\\
20073.7394701502	0.000131944787599777\\
20079.5995604932	0.000265623821086051\\
20085.4596508363	0.000430366706511684\\
20091.3197411793	0.000622279132573751\\
20097.1798315224	0.000836826370871963\\
20103.0399218655	0.00106894042550905\\
20108.9000122085	0.00131313974585914\\
20114.7601025516	0.00156365867478947\\
20120.6201928946	0.00181458357556377\\
20126.4802832377	0.00205999242345\\
20132.3403735808	0.00229409456716184\\
20138.2004639238	0.00251136736170224\\
20144.0605542669	0.0027066864489546\\
20149.9206446099	0.00287544661260889\\
20155.780734953	0.00301367035727035\\
20161.6408252961	0.00311810165184447\\
20167.5009156391	0.00318628262782528\\
20173.3610059822	0.00321661142590622\\
20179.2210963252	0.00320837982946083\\
20185.0811866683	0.00316178980087591\\
20190.9412770114	0.00307794853466199\\
20196.8013673544	0.00295884214821608\\
20202.6614576975	0.00280728863553673\\
20208.5215480405	0.00262687119822895\\
20214.3816383836	0.00242185353095053\\
20220.2417287267	0.00219707906397015\\
20226.1018190697	0.0019578565441099\\
20231.9619094128	0.00170983465669336\\
20237.8219997558	0.00145886864972079\\
20243.6820900989	0.00121088210922227\\
20249.5421804419	0.000971727148917599\\
20255.402270785	0.000747046313836898\\
20261.2623611281	0.000542139456705485\\
20267.1224514711	0.000361838727574037\\
20272.9825418142	0.000210394625440384\\
20278.8426321572	9.13757987774044e-05\\
20284.7027225003	7.58495697702066e-06\\
20290.5628128434	-3.90071259829308e-05\\
20296.4229031864	-4.73079616430469e-05\\
20302.2829935295	-1.71289608970521e-05\\
20308.1430838725	5.0810288928693e-05\\
20314.0031742156	0.000154899180587666\\
20319.8632645587	0.000292674268021953\\
20325.7233549017	0.000460877566986869\\
20331.5834452448	0.000655533595495984\\
20337.4435355878	0.000872043332353873\\
20343.3036259309	0.00110529287324029\\
20349.163716274	0.00134977421724108\\
20355.023806617	0.00159971533113736\\
20360.8838969601	0.00184921642009313\\
20366.7439873031	0.00209238918780621\\
20372.6040776462	0.00232349579893282\\
20378.4641679893	0.00253708426446659\\
20384.3242583323	0.00272811705584226\\
20390.1843486754	0.00289208991347358\\
20396.0444390184	0.00302513804812941\\
20401.9045293615	0.00312412723097415\\
20407.7646197046	0.00318672762535123\\
20413.6247100476	0.00321146862150076\\
20419.4848003907	0.00319777338318522\\
20425.3448907337	0.0031459722950326\\
20431.2049810768	0.00305729499728184\\
20437.0650714198	0.00293384119935575\\
20442.9251617629	0.00277853096561077\\
20448.785252106	0.00259503564985561\\
20454.645342449	0.00238769111223189\\
20460.5054327921	0.00216139526995276\\
20466.3655231351	0.0019214924022144\\
20472.2256134782	0.00167364694233901\\
20478.0857038213	0.00142370973740285\\
20483.9457941643	0.00117757993293203\\
20489.8058845074	0.000941065743037567\\
20495.6659748504	0.000719747391844812\\
20501.5260651935	0.000518845460369104\\
20507.3861555366	0.000343097744896139\\
20513.2462458796	0.000196647531327036\\
20519.1063362227	8.2945920120351e-05\\
20524.9664265657	4.67050443822302e-06\\
20530.8265169088	-3.63376823612654e-05\\
20536.6866072519	-3.91174651187153e-05\\
20542.5466975949	-3.61005289553254e-06\\
20548.406787938	6.93397352561368e-05\\
20554.266878281	0.000178003569884647\\
20560.1269686241	0.000319810557931723\\
20565.9870589672	0.000491408066685149\\
20571.8471493102	0.00068874098589843\\
20577.7072396533	0.000907147554284746\\
20583.5673299963	0.00114146948591178\\
20589.4274203394	0.0013861737950614\\
20595.2875106825	0.00163548344248802\\
20601.1476010255	0.00188351371818666\\
20607.0076913686	0.00212441114117562\\
20612.8677817116	0.00235249159792822\\
20618.7278720547	0.00256237445937183\\
20624.5879623978	0.0027491095121929\\
20630.4480527408	0.00290829371006961\\
20636.3081430839	0.00303617499155613\\
20642.1682334269	0.00312974071630488\\
20648.02832377	0.00318678863591547\\
20653.8884141131	0.003205978726969\\
20659.7485044561	0.00318686466657632\\
20665.6085947992	0.00312990421121617\\
20671.4686851422	0.00303644823754358\\
20677.3287754853	0.00290870870720147\\
20683.1888658283	0.00274970631572453\\
20689.0489561714	0.00256319906432044\\
20694.9090465145	0.00235359344330247\\
20700.7691368575	0.00212584032652471\\
20706.6292272006	0.00188531803599131\\
20712.4893175436	0.00163770533880346\\
20718.3494078867	0.00138884737503024\\
20724.2094982298	0.00114461768209432\\
20730.0695885728	0.000910779572327836\\
20735.9296789159	0.000692850135347828\\
20741.7897692589	0.000495970074259087\\
20747.649859602	0.000324782446600355\\
20753.5099499451	0.000183323170174404\\
20759.3700402881	7.49258758179725e-05\\
20765.2301306312	2.14335015914561e-06\\
20771.0902209742	-3.33125804714552e-05\\
20776.9503113173	-3.0611309356979e-05\\
20782.8104016604	1.01770948662542e-05\\
20788.6704920034	8.80836917304688e-05\\
20794.5305823465	0.000201263685613291\\
20800.3906726895	0.000347040108862511\\
20806.2507630326	0.000521967137027511\\
20812.1108533757	0.000721911538721629\\
20817.9709437187	0.000942150334932957\\
20823.8310340618	0.00117748236034001\\
20829.6911244048	0.0014223510915718\\
20835.5512147479	0.00167097584175276\\
20841.411305091	0.00191748822394495\\
20847.271395434	0.00215607066185803\\
20853.1314857771	0.00238109367907378\\
20858.9915761201	0.00258724872640782\\
20864.8516664632	0.00276967341411233\\
20870.7117568062	0.00292406619401881\\
20876.5718471493	0.00304678778741879\\
20882.4319374924	0.00313494696671604\\
20888.2920278354	0.00318646866930303\\
20894.1521181785	0.00320014283863114\\
20900.0122085215	0.00317565284344941\\
20905.8722988646	0.00311358280714593\\
20911.7323892077	0.00301540367833292\\
20917.5924795507	0.00288343837409154\\
20923.4525698938	0.00272080682239459\\
20929.3126602368	0.00253135220365635\\
20935.1727505799	0.00231955013498935\\
20941.032840923	0.00209040294326255\\
20946.892931266	0.00184932152459245\\
20952.7530216091	0.00160199757994134\\
20958.6131119521	0.00135426924396446\\
20964.4732022952	0.00111198327904755\\
20970.3332926383	0.000880857087363718\\
20976.1933829813	0.000666343797465135\\
20982.0534733244	0.000473503608921444\\
20987.9135636674	0.000306884430186792\\
20993.7736540105	0.000170414625237528\\
20999.6337443536	6.7310398293246e-05\\
nan	nan\\
21011.3539250397	-2.99334587582342e-05\\
21017.2140153827	-2.1789238725292e-05\\
21023.0741057258	2.4234625778998e-05\\
21028.9341960689	0.000107046138705955\\
21034.7942864119	0.000224685251098001\\
21040.654376755	0.000374370252199059\\
21046.514467098	0.000552563544909791\\
21052.3745574411	0.000755055250273588\\
21058.2346477842	0.000977062665328763\\
21064.0947381272	0.00121334322453697\\
21069.9548284703	0.00145831829705226\\
21075.8149188133	0.00170620489645645\\
21081.6750091564	0.00195115219337911\\
21087.5350994994	0.00218737960872977\\
21093.3951898425	0.00240931322825304\\
21099.2552801856	0.00261171731907342\\
21105.1153705286	0.00278981784564513\\
21110.9754608717	0.00293941507080181\\
21116.8355512147	0.00305698258643913\\
21122.6956415578	0.00313975043883375\\
21128.5557319009	0.00318577038886933\\
21134.4158222439	0.00319396176997816\\
21140.275912587	0.00316413686430596\\
21146.13600293	0.00309700520125194\\
21151.9960932731	0.00299415667996184\\
21157.8561836162	0.00285802391758802\\
21163.7162739592	0.00269182471422838\\
21169.5763643023	0.00249948599611892\\
21175.4364546453	0.00228555103410204\\
21181.2965449884	0.00205507212966843\\
21187.1566353315	0.00181349130351671\\
21193.0167256745	0.00156651180374478\\
21198.8768160176	0.00131996346766029\\
21204.7369063606	0.00107966511550895\\
21210.5969967037	0.000851287224124466\\
21216.4570870468	0.00064021812117586\\
21222.3171773898	0.000451436857593549\\
21228.1772677329	0.000289395757199251\\
21234.0373580759	0.000157915414167216\\
21239.897448419	6.00946144661507e-05\\
21245.7575387621	-1.76269488871369e-06\\
21251.6176291051	-2.62016677201703e-05\\
21257.4777194482	-1.26507737700991e-05\\
21263.3378097912	3.85648376552169e-05\\
21269.1979001343	0.000126231137913003\\
21275.0579904774	0.000248273996051639\\
21280.9180808204	0.000401808250545425\\
21286.7781711635	0.000583205914801907\\
21292.6382615065	0.000788181904068069\\
21298.4983518496	0.00101189525817956\\
21304.3584421926	0.00124906346904916\\
21310.2185325357	0.00149408721301988\\
21316.0786228788	0.00174118254285544\\
21321.9387132218	0.00198451741860817\\
21327.7988035649	0.00221834935470449\\
21333.6588939079	0.00243716093445656\\
21339.518984251	0.00263578999428017\\
21345.3790745941	0.00280955140584624\\
21351.2391649371	0.00295434758344741\\
21357.0992552802	0.00306676510985293\\
21362.9593456232	0.00314415520234353\\
21368.8194359663	0.00318469612222795\\
21374.6795263094	0.00318743605714714\\
21380.5396166524	0.00315231546681139\\
21386.3997069955	0.00308016836798435\\
21392.2597973385	0.00297270253069246\\
21398.1198876816	0.00283245905618398\\
21403.9799780247	0.00266275229325185\\
21409.8400683677	0.00246759151377172\\
21415.7001587108	0.00225158619786335\\
21421.5602490538	0.00201983716683753\\
21427.4203393969	0.0017778161343922\\
21433.28042974	0.00153123652077801\\
21439.140520083	0.00128591857954535\\
21445.0006104261	0.00104765202075956\\
21450.8607007691	0.000822059373266662\\
21456.7207911122	0.000614463310369162\\
21462.5808814553	0.000429761069651542\\
21468.4409717983	0.000272308929783336\\
21474.3010621414	0.000145819469240684\\
21480.1611524844	5.32740301088044e-05\\
21486.0212428275	-3.1475492157126e-06\\
21491.8813331706	-2.21182763003273e-05\\
21497.7414235136	-3.19521269947748e-06\\
21503.6015138567	5.31701861906142e-05\\
21509.4616041997	0.000145642840780946\\
21515.3216945428	0.000272035669565267\\
21521.1817848858	0.00042936131505069\\
21527.0418752289	0.000613902750050097\\
21532.901965572	0.000821301095556858\\
21538.762055915	0.00104665857552332\\
21544.6221462581	0.00128465417689415\\
21550.4822366011	0.00152966928354407\\
21556.3423269442	0.00177592031809374\\
21562.2024172873	0.00201759526007206\\
21568.0625076303	0.00224899081828803\\
21573.9225979734	0.00246464701992132\\
21579.7826883164	0.00265947604038275\\
21585.6427786595	0.00282888223403618\\
21591.5028690026	0.00296887053453593\\
21597.3629593456	0.00307614066740391\\
21603.2230496887	0.0031481649537589\\
21609.0831400317	0.00318324787040729\\
21614.9432303748	0.00318056596340672\\
21620.8033207179	0.00314018717540569\\
21626.6634110609	0.00306306913297706\\
21632.523501404	0.00295103643708406\\
21638.383591747	0.00280673749533747\\
21644.2436820901	0.00263358191700662\\
21650.1037724332	0.00243565995089277\\
21655.9638627762	0.00221764586926778\\
21661.8239531193	0.0019846875803692\\
21667.6840434623	0.0017422850759966\\
21673.5441338054	0.00149616058417645\\
21679.4042241485	0.00125212349272012\\
21685.2643144915	0.00101593323165793\\
21691.1244048346	0.000793163351206407\\
21696.9844951776	0.000589070002689916\\
21702.8445855207	0.000408467925958908\\
21708.7046758638	0.00025561686900393\\
21714.5647662068	0.000134121119197111\\
21720.4248565499	4.68445156296738e-05\\
21726.2849468929	-4.15705347971521e-06\\
21732.145037236	-1.76840772969702e-05\\
21738.0051275791	6.57836785429514e-06\\
21743.8652179221	6.8053288864956e-05\\
21749.7253082652	0.000165285496909142\\
21755.5853986082	0.000295976052946832\\
21761.4454889513	0.000457036622334211\\
21767.3055792943	0.000644662453455298\\
21773.1656696374	0.000854422255905918\\
21779.0257599805	0.0010813628551111\\
21784.8858503235	0.0013201261518781\\
21790.7459406666	0.00156507562502382\\
21796.6060310096	0.00181042939062635\\
21802.4661213527	0.00205039667687175\\
21808.3262116958	0.00227931449321152\\
21814.1863020388	0.00249178126828024\\
21820.0463923819	0.00268278430328859\\
21825.9064827249	0.00284781803319105\\
21831.766573068	0.00298299030619131\\
21837.6266634111	0.00308511417400602\\
21843.4867537541	0.00315178302898263\\
21849.3468440972	0.00318142731597603\\
21855.2069344402	0.0031733514830158\\
21861.0670247833	0.00312775030068174\\
21866.9271151264	0.00304570416747892\\
21872.7872054694	0.00292915351416795\\
21878.6472958125	0.00278085291357972\\
21884.5073861555	0.00260430598086248\\
21890.3674764986	0.00240368260310614\\
21896.2275668417	0.00218372045324131\\
21902.0876571847	0.00194961311443229\\
21907.9477475278	0.00170688745674643\\
21913.8078378708	0.00146127316119429\\
21919.6679282139	0.00121856747127369\\
21925.5280185569	0.000984498364458058\\
21931.3881089	0.000764589372995192\\
21937.2481992431	0.000564029244210892\\
21943.1082895861	0.000387549515836936\\
21948.9683799292	0.000239312894783654\\
21954.8284702722	0.000122815072568025\\
21960.6885606153	4.08022931965327e-05\\
21966.5486509584	-4.79338274669962e-06\\
21972.4087413014	-1.28995918351125e-05\\
21978.2688316445	1.66711127725148e-05\\
21984.1289219875	8.32169292912925e-05\\
21989.9890123306	0.000185163462725372\\
21995.8491026737	0.000320100972490042\\
22001.7091930167	0.000484841331038173\\
22007.5692833598	0.000675493347235683\\
22013.4293737028	0.000887554674900925\\
22019.2894640459	0.00111601813571697\\
22025.149554389	0.00135548994578969\\
22031.009644732	0.00160031705445061\\
22036.8697350751	0.00184472058897597\\
22042.7298254181	0.00208293225547603\\
22048.5899157612	0.00230933047621894\\
22054.4500061043	0.00251857305087442\\
22060.3100964473	0.00270572321065645\\
22066.1701867904	0.00286636609161833\\
22072.0302771334	0.00299671287874913\\
22077.8903674765	0.003093690164508\\
22083.7504578196	0.00315501241427375\\
22089.6105481626	0.00317923583013728\\
22095.4706385057	0.00316579234372959\\
22101.3307288487	0.00311500293750803\\
22107.1908191918	0.00302806998259028\\
22113.0509095349	0.00290704877566157\\
22118.9109998779	0.00275479894871231\\
22124.771090221	0.00257491690069267\\
22130.631180564	0.00237165084697883\\
22136.4912709071	0.00214980049387174\\
22142.3513612502	0.00191460370676974\\
22148.2114515932	0.001671612848364\\
22154.0715419363	0.00142656370606698\\
22159.9316322793	0.0011852401029903\\
22165.7917226224	0.000953337388016618\\
22171.6518129654	0.000736328026782848\\
22177.5119033085	0.000539332465649766\\
22183.3719936516	0.000366998315679557\\
22189.2320839946	0.00022339070732694\\
22195.0921743377	0.000111896402403045\\
22200.9522646807	3.51439251248846e-05\\
22206.8123550238	-5.05840432979276e-06\\
22212.6724453669	-7.76507293368395e-06\\
22218.5325357099	2.7084389079859e-05\\
22224.392626053	9.86640620266708e-05\\
22230.252716396	0.000205281210359837\\
22236.1128067391	0.000344416312183844\\
22241.9728970822	0.000512782598102907\\
22247.8329874252	0.000706403692488533\\
22253.6930777683	0.000920707523120472\\
22259.5531681113	0.00115063428147733\\
22265.4132584544	0.00139075588434417\\
22271.2733487975	0.00163540411647749\\
22277.1334391405	0.00187880442921946\\
22282.9935294836	0.00211521223675625\\
22288.8536198266	0.00233904849334019\\
22294.7137101697	0.0025450313515061\\
22300.5738005128	0.00272830079464606\\
22306.4338908558	0.00288453330298151\\
22312.2939811989	0.00301004384776905\\
22318.1540715419	0.00310187280745876\\
22324.014161885	0.00315785575616815\\
22329.8742522281	0.00317667447853186\\
22335.7343425711	0.00315788800870175\\
22341.5944329142	0.00310194296288637\\
22347.4545232572	0.00301016292294484\\
22353.3146136003	0.00288471712361161\\
22359.1747039434	0.00272856918415062\\
22365.0347942864	0.00254540709595185\\
22370.8948846295	0.00233955612011781\\
22376.7549749725	0.00211587665201912\\
22382.6150653156	0.00187964946473694\\
22388.4751556586	0.00163645104061069\\
22394.3352460017	0.00139202193415246\\
22400.1953363448	0.00115213127346555\\
22406.0554266878	0.000922440598519648\\
22411.9155170309	0.000708370249733787\\
22417.7756073739	0.000514971460086131\\
22423.635697717	0.000346807168967215\\
22429.4957880601	0.000207844369965308\\
22435.3558784031	0.00010136053232084\\
22441.2159687462	2.98663034621566e-05\\
22447.0760590892	-4.95368442179901e-06\\
22452.9361494323	-2.28050819703234e-06\\
22458.7962397754	3.78197872575473e-05\\
22464.6563301184	0.000114397817855907\\
22470.5164204615	0.000225643336778932\\
22476.3765108045	0.00036892802650677\\
22482.2366011476	0.000540867594888604\\
22488.0966914907	0.000737401708231853\\
22493.9567818337	0.000953889873499597\\
22499.8168721768	0.00118522100549985\\
22505.6769625198	0.00142593409216422\\
22511.5370528629	0.00167034710918789\\
22517.397143206	0.00191269114091845\\
22523.257233549	0.0021472465417992\\
22529.1173238921	0.0023684779246202\\
22534.9774142351	0.00257116478977409\\
22540.8375045782	0.00275052471302078\\
22546.6975949213	0.00290232618492463\\
22552.5576852643	0.00302298843953884\\
22558.4177756074	0.00310966591729985\\
22564.2778659504	0.00316031537020093\\
22570.1379562935	0.00317374402608486\\
22575.9980466366	0.00314963767746706\\
22581.8581369796	0.00308856803303492\\
22587.7182273227	0.00299197915987841\\
22593.5783176657	0.00286215333808297\\
22599.4384080088	0.00270215713492345\\
22605.2984983518	0.00251576897278676\\
22611.1585886949	0.00230738990181894\\
22617.018679038	0.00208193968385947\\
22622.878769381	0.00184474064196413\\
22628.7388597241	0.00160139201688716\\
22634.5989500671	0.00135763779703005\\
22640.4590404102	0.00111923114136709\\
22646.3191307533	0.000891798595481356\\
22652.1792210963	0.000680707305269361\\
22658.0393114394	0.000490938362023845\\
22663.8994017824	0.000326969267657327\\
22669.7594921255	0.000192668293299023\\
22675.6195824686	9.12032237855432e-05\\
22681.4796728116	2.49666407026223e-05\\
22687.3397631547	-4.4804937216856e-06\\
22693.1998534977	3.55437834762091e-06\\
22699.0599438408	4.8879123236168e-05\\
22704.9200341839	0.000130421509569822\\
22710.7801245269	0.00024625457320355\\
22716.64021487	0.000393642153240611\\
22722.500305213	0.000569103523113763\\
22728.3603955561	0.00076849559011972\\
22734.2204858992	0.000987110722275153\\
22740.0805762422	0.00121978789258141\\
22745.9406665853	0.00146103451687346\\
22751.8007569283	0.00170515610887546\\
22757.6608472714	0.00194639069217638\\
22763.5209376145	0.00217904479632731\\
22769.3810279575	0.00239762782755336\\
22775.2411183006	0.0025969816427352\\
22781.1012086436	0.00277240226895486\\
22786.9612989867	0.00291975089634746\\
22792.8213893298	0.00303555152548288\\
22798.6814796728	0.00311707296543532\\
22804.5415700159	0.0031623932482611\\
22810.4016603589	0.00317044494088894\\
22816.261750702	0.00314104028593801\\
22822.121841045	0.00307487557957131\\
22827.9819313881	0.00297351468418514\\
22833.8420217312	0.00283935206639484\\
22839.7021120742	0.00267555623391138\\
22845.5622024173	0.00248599490723884\\
22851.4222927603	0.00227514369367563\\
22857.2823831034	0.00204798041953301\\
22863.1424734465	0.00180986761577869\\
22869.0025637895	0.00156642593077006\\
22874.8626541326	0.00132340145869683\\
22880.7227444756	0.00108653011481503\\
};
\addplot [color=mycolor2, forget plot]
  table[row sep=crcr]{%
22880.7227444756	0.00108653011481503\\
22886.5828348187	0.000861402258853052\\
22892.4429251618	0.000653330761546587\\
22898.3030155048	0.000467225627731655\\
22904.1631058479	0.000307478134835005\\
22910.0231961909	0.000177857220571269\\
22915.883286534	8.14205645561398e-05\\
22921.7433768771	2.04424615879587e-05\\
22927.6034672201	-3.63981209199257e-06\\
22933.4635575632	9.74012524708163e-06\\
22939.3236479062	6.02644410705445e-05\\
22945.1837382493	0.000146738638253502\\
22951.0438285924	0.000267119794869069\\
22956.9039189354	0.000418564826472744\\
22962.7640092785	0.000597497630761689\\
22968.6240996215	0.000799693528888538\\
22974.4841899646	0.00102037900962845\\
22980.3442803077	0.0012543444214817\\
22986.2043706507	0.00149606695238417\\
22992.0644609938	0.00173984099396714\\
22997.9245513368	0.00197991281340175\\
23003.7846416799	0.00221061635395005\\
23009.6447320229	0.00242650695928725\\
23015.504822366	0.00262248986564705\\
23021.3649127091	0.00279394042944539\\
23027.2250030521	0.00293681325323594\\
23033.0850933952	0.00304773763518869\\
23038.9451837382	0.00312409708999906\\
23044.8052740813	0.00316409106521263\\
23050.6653644244	0.0031667773971503\\
23056.5254547674	0.00313209450585511\\
23062.3855451105	0.00306086280545275\\
23068.2456354535	0.00295476529839557\\
23074.1057257966	0.00281630781211539\\
23079.9658161397	0.00264875981775918\\
23085.8259064827	0.0024560772285584\\
23091.6859968258	0.00224280900082403\\
23097.5460871688	0.00201398974254761\\
23103.4061775119	0.00177502086512199\\
23109.266267855	0.00153154308315336\\
23115.126358198	0.00128930327248653\\
23120.9864485411	0.00105401882863958\\
23126.8465388841	0.000831242727223982\\
23132.7066292272	0.000626232471060041\\
23138.5667195703	0.000443826016700346\\
23144.4268099133	0.000288327608511191\\
23150.2869002564	0.000163406214180456\\
23156.1469905994	7.20089582365438e-05\\
23162.0070809425	1.62915959366599e-05\\
23167.8671712856	-2.43233227679195e-06\\
23173.7272616286	1.62775334424545e-05\\
23179.5873519717	7.19780163018575e-05\\
23185.4474423147	0.000163352900106792\\
23191.3075326578	0.000288244031139553\\
23197.1676230009	0.000443702289782391\\
23203.0277133439	0.000626057228032108\\
23208.887803687	0.000831003728648045\\
23214.74789403	0.00105370363988343\\
23220.6079843731	0.00128889998659323\\
23226.4680747162	0.00153104106156609\\
23232.3281650592	0.0017744114680305\\
23238.1882554023	0.00201326702029599\\
23244.0483457453	0.00224197031848113\\
23249.9084360884	0.00245512379765688\\
23255.7685264314	0.00264769711123416\\
23261.6286167745	0.00281514584262769\\
23267.4887071176	0.00295351874348592\\
23273.3487974606	0.00305955096821521\\
23279.2088878037	0.00313074110472752\\
23285.0689781467	0.00316541018410151\\
23290.9290684898	0.00316274127696465\\
23296.7891588329	0.00312279874297584\\
23302.6492491759	0.00304652667970685\\
23308.509339519	0.00293572660830513\\
23314.369429862	0.00279301492349569\\
23320.2295202051	0.00262176111254601\\
23326.0896105482	0.00242600820243765\\
23331.9497008912	0.00221037731292248\\
23337.8097912343	0.001979958569241\\
23343.6698815773	0.0017401909490252\\
23349.5299719204	0.00149673390007424\\
23355.3900622635	0.00125533375879847\\
23361.2501526065	0.0010216881225975\\
23367.1102429496	0.000801311376971325\\
23372.9703332926	0.000599404551240569\\
23378.8304236357	0.00042073257420086\\
23384.6905139788	0.000269511826509594\\
23390.5506043218	0.000149310643285781\\
23396.4106946649	6.29651148781004e-05\\
23402.2707850079	1.25121724722545e-05\\
23408.130875351	-8.5846270381279e-07\\
23413.9909656941	2.31676669029033e-05\\
23419.8510560371	8.40223600124816e-05\\
23425.7111463802	0.00018026819382562\\
23431.5712367232	0.000309632476041659\\
23437.4313270663	0.000469060909647021\\
23443.2914174093	0.000654789703332076\\
23449.1515077524	0.000862434425106863\\
23455.0115980955	0.00108709350147891\\
23460.8716884385	0.00132346391919193\\
23466.7317787816	0.00156596639829582\\
23472.5918691246	0.00180887708212871\\
23478.4519594677	0.00204646263590623\\
23484.3120498108	0.00227311556530636\\
23490.1721401538	0.00248348656134311\\
23496.0322304969	0.00267261074815455\\
23501.8923208399	0.00283602485397446\\
23507.752411183	0.00296987254054473\\
23513.6125015261	0.00307099540487415\\
23519.4725918691	0.00313700750641909\\
23525.3326822122	0.00316635166067952\\
23531.1927725552	0.00315833617138655\\
23537.0528628983	0.00311315113484939\\
23542.9129532414	0.00303186393183529\\
23548.7730435844	0.00291639401417057\\
23554.6331339275	0.00276946758167119\\
23560.4932242705	0.00259455321928503\\
23566.3533146136	0.00239578001412959\\
23572.2134049567	0.00217784008551419\\
23578.0734952997	0.00194587782846629\\
23583.9335856428	0.00170536848525696\\
23589.7936759858	0.00146198891095097\\
23595.6537663289	0.00122148358345847\\
23601.513856672	0.000989529020273607\\
23607.373947015	0.000771599802240223\\
23613.2340373581	0.00057283936599741\\
23619.0941277011	0.000397938614805616\\
23624.9542180442	0.000251025212382475\\
23630.8143083873	0.000135566172420187\\
23636.6743987303	5.42860425817925e-05\\
23642.5344890734	9.10261360959128e-06\\
23648.3945794164	1.08167060916587e-06\\
23654.2546697595	3.04118540724698e-05\\
23660.1147601026	9.64002235887186e-05\\
23665.9748504456	0.000197488628571264\\
23671.8349407887	0.000331290499238187\\
23677.6950311317	0.000494647189191627\\
23683.5551214748	0.000683702539442639\\
23689.4152118178	0.000893993903777696\\
23695.2753021609	0.00112055748681614\\
23701.135392504	0.00135804550845667\\
23706.995482847	0.00160085242911215\\
23712.8555731901	0.00184324725662905\\
23718.7156635331	0.00207950881207306\\
23724.5757538762	0.00230406076177345\\
23730.4358442193	0.00251160322893876\\
23736.2959345623	0.00269723787816496\\
23742.1560249054	0.00285658352129508\\
23748.0161152484	0.00298587951579517\\
23753.8762055915	0.00308207451594186\\
23759.7362959346	0.0031428984816753\\
23765.5963862776	0.00316691624660542\\
23771.4564766207	0.0031535613804648\\
23777.3165669637	0.00310314954747763\\
23783.1766573068	0.00301687104530908\\
23789.0367476499	0.00289676270111265\\
23794.8968379929	0.00274565978831946\\
23800.756928336	0.00256712909893085\\
23806.617018679	0.00236538475137398\\
23812.4771090221	0.00214518872106238\\
23818.3371993652	0.00191173844163099\\
23824.1972897082	0.00167054412948655\\
23830.0573800513	0.00142729872741364\\
23835.9174703943	0.0011877435365919\\
23841.7775607374	0.000957532708655188\\
23847.6376510805	0.000742099795682522\\
23853.4977414235	0.000546529508093835\\
23859.3578317666	0.000375437706815771\\
23865.2179221096	0.000232862462252826\\
23871.0780124527	0.00012216875107912\\
23876.9381027957	4.59690400613531e-05\\
23882.7981931388	6.06163117441586e-06\\
23888.6582834819	3.38822303215212e-06\\
23894.5183738249	3.80116901284416e-05\\
23900.378464168	0.000109114604210599\\
23906.238554511	0.000215018532550082\\
23912.0986448541	0.000353223657520941\\
23917.9587351972	0.000520467782298221\\
23923.8188255402	0.000712803329900807\\
23929.6789158833	0.000925690518250051\\
23935.5390062263	0.00115410451200649\\
23941.3990965694	0.00139265402219157\\
23947.2591869125	0.00163570855442351\\
23953.1192772555	0.00187753130238685\\
23958.9793675986	0.0021124145500993\\
23964.8394579416	0.00233481438709966\\
23970.6995482847	0.0025394815572651\\
23976.5596386278	0.00272158535269575\\
23982.4197289708	0.00287682762893246\\
23988.2798193139	0.00300154425033711\\
23994.1399096569	0.00309279157124961\\
24000	0.00314841591240077\\
};
\end{axis}
\end{tikzpicture}%

Les sorties des filtres sont les suivantes

% This file was created by matlab2tikz.
%
%The latest updates can be retrieved from
%  http://www.mathworks.com/matlabcentral/fileexchange/22022-matlab2tikz-matlab2tikz
%where you can also make suggestions and rate matlab2tikz.
%
\definecolor{mycolor1}{rgb}{0.00000,0.44700,0.74100}%
\definecolor{mycolor2}{rgb}{0.85000,0.32500,0.09800}%
%
\begin{tikzpicture}

\begin{axis}[%
width=4.521in,
height=3.559in,
at={(0.758in,0.488in)},
scale only axis,
xmin=0,
xmax=4000,
xlabel style={font=\color{white!15!black}},
xlabel={Temps [s]},
ymin=-1.5,
ymax=1.5,
ylabel style={font=\color{white!15!black}},
ylabel={Amplitude},
axis background/.style={fill=white},
title style={font=\bfseries},
title={Signal démodulé},
legend style={legend cell align=left, align=left, draw=white!15!black}
]
\addplot [color=mycolor1]
  table[row sep=crcr]{%
1	-0.0037331097405513\\
3	-0.0142427619216505\\
7	-0.0381347271045342\\
8	-0.0423112335402038\\
9	-0.0450159428287407\\
10	-0.0457074914238547\\
11	-0.043993347050673\\
12	-0.0395898552928884\\
13	-0.0320865960025003\\
14	-0.0213516150829491\\
15	-0.0071884278017933\\
16	0.0104635209931985\\
17	0.0316027244630277\\
18	0.0561464217275898\\
19	0.0838477803672504\\
20	0.114429365314663\\
21	0.147480861340227\\
22	0.182532521439498\\
24	0.256414529294943\\
26	0.330863627591498\\
27	0.366472803108991\\
28	0.399955722209597\\
29	0.430543546280205\\
30	0.457581967224996\\
31	0.480300474701835\\
32	0.498098621252666\\
33	0.510371490871421\\
34	0.516721402921121\\
35	0.51670537803102\\
36	0.51008032604841\\
37	0.49671337237578\\
38	0.476604925274387\\
39	0.449927518028289\\
40	0.416949165145525\\
41	0.378018212678853\\
42	0.333684092299336\\
43	0.28451378337104\\
44	0.231400341564949\\
45	0.175101027343317\\
46	0.116595889903238\\
49	-0.0621510681089603\\
50	-0.119299367629992\\
51	-0.173546185672876\\
52	-0.223846554468309\\
53	-0.269290443138289\\
54	-0.309073430755234\\
55	-0.342400575179454\\
56	-0.368678904026183\\
57	-0.387458599237107\\
58	-0.398322735374677\\
59	-0.401173747006851\\
60	-0.395959271737411\\
61	-0.382791665147579\\
62	-0.358165948637179\\
63	-0.324968717777665\\
64	-0.284056380010497\\
65	-0.236369629347337\\
66	-0.183109011037686\\
67	-0.125523806878846\\
68	-0.0648720119443169\\
71	0.120465004425569\\
72	0.17844569236695\\
73	0.232254784275938\\
74	0.280414415665291\\
75	0.321971381833919\\
76	0.355840774361241\\
77	0.381266160304676\\
78	0.397598172271955\\
79	0.40445984342432\\
80	0.401687938582199\\
81	0.389346045969887\\
82	0.367732689977856\\
83	0.337416978330111\\
84	0.299110888256564\\
85	0.253613764194142\\
86	0.202052633882431\\
87	0.145705864016236\\
88	0.0858487071845957\\
91	-0.0999200173569079\\
92	-0.159040550459849\\
93	-0.214366168223023\\
94	-0.264646885224465\\
95	-0.30862879047163\\
96	-0.345250894141373\\
97	-0.373590352721749\\
98	-0.393036851441138\\
99	-0.403120685456543\\
100	-0.403667421890532\\
101	-0.394608724420777\\
102	-0.376161317522019\\
103	-0.348707885113981\\
104	-0.312916926128764\\
105	-0.269730201140646\\
106	-0.220078088334049\\
107	-0.165252350938317\\
108	-0.106465190241124\\
110	0.0172041697046552\\
111	0.0791175250428751\\
112	0.139202586380179\\
113	0.195970649540868\\
114	0.248062711702914\\
115	0.294286775466389\\
116	0.333508902288486\\
117	0.364753787419886\\
118	0.38735283864844\\
119	0.400678011920263\\
120	0.404495521086574\\
121	0.398668677444221\\
122	0.383354262726243\\
123	0.35885493339083\\
124	0.325787717114054\\
125	0.284970824333868\\
126	0.237391025639226\\
127	0.184183478862906\\
128	0.126587107000887\\
129	0.0659267407568223\\
132	-0.119598698627215\\
133	-0.177688389699142\\
134	-0.23159189440139\\
135	-0.279922525361599\\
136	-0.321629480944011\\
137	-0.355745295966699\\
138	-0.381395739969776\\
139	-0.397929087109787\\
140	-0.404985823710831\\
141	-0.402397762282817\\
142	-0.390250491425377\\
143	-0.36882316543506\\
144	-0.338622190556634\\
145	-0.300393687788983\\
146	-0.255019089970119\\
147	-0.203577359522569\\
148	-0.147355046755365\\
149	-0.0875957218204348\\
152	0.0981783935071689\\
153	0.157336329874852\\
154	0.212759459284371\\
155	0.263152256094145\\
156	0.307249956834312\\
157	0.344019745219157\\
158	0.372514303503067\\
159	0.392128629129729\\
160	0.402443836198017\\
161	0.398601925595358\\
162	0.387921739129524\\
163	0.370668597360691\\
164	0.347072746440062\\
165	0.317654085929917\\
166	0.282884811656004\\
167	0.243343954683041\\
168	0.199756146990239\\
169	0.152759277686982\\
170	0.10308791875832\\
172	-0.0011791187162089\\
174	-0.107239132772975\\
175	-0.159214336961213\\
176	-0.209754585496285\\
177	-0.258311580264035\\
178	-0.304390317698108\\
179	-0.347685846478726\\
180	-0.387901326410883\\
181	-0.424828415690627\\
182	-0.458387751883492\\
183	-0.488583131761516\\
184	-0.515359251099653\\
185	-0.538959558910847\\
186	-0.559455265442466\\
187	-0.5771159424703\\
188	-0.592194123734771\\
189	-0.604898914041769\\
190	-0.615601930921912\\
191	-0.624538894007856\\
192	-0.631944246886178\\
194	-0.643005041166816\\
196	-0.650138717338905\\
198	-0.653433578108434\\
199	-0.653517768100755\\
200	-0.652385223929741\\
201	-0.649805690514768\\
202	-0.645425042655461\\
203	-0.638928101118381\\
204	-0.629924074562496\\
205	-0.618057853160735\\
206	-0.602953210952819\\
207	-0.584180111497972\\
208	-0.56135498522508\\
209	-0.534188886016182\\
210	-0.502414736710307\\
211	-0.465773913938392\\
212	-0.424250331694111\\
213	-0.377821381527156\\
214	-0.326438582513674\\
215	-0.270465367070756\\
216	-0.210070419600015\\
217	-0.145757658014645\\
218	-0.0780536283468791\\
219	-0.00757796570542268\\
221	0.138171413222153\\
222	0.216368339002202\\
223	0.290994829807005\\
224	0.360836160359213\\
225	0.424725155917713\\
226	0.481623529230546\\
227	0.530596241625062\\
228	0.570868289254577\\
229	0.601751080874237\\
230	0.622708292295101\\
231	0.633533792376511\\
232	0.63382367575241\\
233	0.623718337003538\\
234	0.603361274037525\\
235	0.573130531389324\\
236	0.533425305115088\\
237	0.484983462857144\\
238	0.428537248983048\\
239	0.365083611372484\\
240	0.295586458536945\\
241	0.221256484431706\\
242	0.14323085578917\\
244	-0.0185623218148976\\
245	-0.0996988886786312\\
246	-0.179152096396137\\
247	-0.255685030636414\\
248	-0.328032327379788\\
249	-0.394972912064532\\
250	-0.455429840618308\\
251	-0.508434014089289\\
252	-0.553068363005877\\
253	-0.58857723814981\\
254	-0.614475392921122\\
255	-0.630289364400141\\
256	-0.635685482216559\\
257	-0.630663600461958\\
258	-0.615303619588303\\
259	-0.589846624977326\\
260	-0.554655757568526\\
261	-0.510346184085392\\
262	-0.457680639179216\\
263	-0.39744577147394\\
264	-0.330699035836915\\
265	-0.258478027123147\\
266	-0.182108617744689\\
267	-0.102736340170395\\
270	0.140314969451993\\
271	0.218471294087067\\
272	0.293031436286128\\
273	0.362832692386746\\
274	0.426656749630638\\
275	0.483442202000788\\
276	0.532296781873356\\
277	0.572386333978557\\
278	0.603057891953085\\
279	0.623856534303286\\
280	0.634340093661194\\
281	0.634437411390536\\
282	0.624129145451661\\
283	0.603537933676762\\
284	0.573018666254484\\
285	0.533086802226535\\
286	0.484408104122394\\
287	0.427748510679521\\
288	0.364128929618346\\
289	0.294470872913735\\
290	0.219990027846052\\
291	0.141920857502555\\
293	-0.0200346199094383\\
294	-0.101146270895697\\
295	-0.18059664985185\\
296	-0.257153218464282\\
297	-0.329455993173724\\
298	-0.396354657201755\\
299	-0.456730497659009\\
300	-0.509621465755117\\
301	-0.554112543225528\\
302	-0.589496984669495\\
303	-0.615241924146176\\
304	-0.630830105138557\\
305	-0.636051505709929\\
306	-0.630829962477947\\
307	-0.615223830533068\\
308	-0.589579574981599\\
309	-0.554222108847625\\
310	-0.509847538642589\\
311	-0.457086372751746\\
312	-0.396694603442938\\
313	-0.329873635304466\\
314	-0.257600752665439\\
315	-0.181105884686531\\
316	-0.101657716398222\\
319	0.141313405595611\\
320	0.219424707362577\\
321	0.293884828563932\\
322	0.363501223379899\\
323	0.427188820310221\\
324	0.483771861011974\\
325	0.532443597091515\\
326	0.572340380731021\\
327	0.602909423332221\\
328	0.623482478154074\\
329	0.633837891299663\\
330	0.633753817841352\\
331	0.62318660578967\\
332	0.602447448494331\\
333	0.571764074598832\\
334	0.531665498088842\\
335	0.482867936200364\\
336	0.426073928695132\\
337	0.362352570627991\\
338	0.292642511773465\\
339	0.218133540959116\\
340	0.140013926401025\\
342	-0.0217607776448858\\
343	-0.102787291380992\\
344	-0.182105250348286\\
345	-0.258462310102914\\
346	-0.330576029471104\\
347	-0.397277157275312\\
348	-0.457437268682952\\
349	-0.510124379265108\\
350	-0.554411031528161\\
351	-0.589620214151637\\
352	-0.615077608107185\\
353	-0.630469436979183\\
354	-0.635527644387821\\
355	-0.630151460333764\\
356	-0.614383084501242\\
357	-0.588523196030565\\
358	-0.552971697704379\\
359	-0.508384185925934\\
360	-0.455420509732903\\
361	-0.394998856969778\\
362	-0.328035592471224\\
363	-0.255750745473506\\
364	-0.179174937145945\\
365	-0.0996986443283276\\
368	0.143228040053145\\
369	0.221267332517527\\
370	0.295689489172219\\
371	0.36526105609164\\
372	0.42882343222027\\
373	0.485241953151217\\
374	0.533757067028546\\
375	0.573455391817788\\
376	0.603794949767689\\
377	0.624209194741525\\
378	0.634364108465434\\
379	0.634058746587016\\
380	0.623350265899717\\
381	0.602382874423711\\
382	0.571498530362987\\
383	0.531245333197603\\
384	0.482233252890182\\
385	0.425334660928911\\
386	0.361449516232824\\
387	0.291615811799147\\
388	0.216929908431666\\
389	0.138741953709086\\
391	-0.0231931182652261\\
392	-0.10424578787206\\
393	-0.183580091208114\\
394	-0.25993484645187\\
395	-0.332017761810675\\
396	-0.398644076764413\\
397	-0.458684138890021\\
398	-0.511252760359639\\
399	-0.555375562028075\\
400	-0.590419505783757\\
401	-0.615741519276071\\
402	-0.630995858204642\\
403	-0.635811543496857\\
404	-0.630202672915402\\
405	-0.614251720920493\\
406	-0.588190612739254\\
407	-0.552485819387584\\
408	-0.507628084524185\\
409	-0.454439257822287\\
410	-0.393808573012848\\
411	-0.326734318068247\\
412	-0.254300491056711\\
413	-0.177697125728628\\
414	-0.0981375251058125\\
417	0.144927852137243\\
418	0.222953857527045\\
419	0.297234946557182\\
420	0.366726875363383\\
421	0.430136879780548\\
422	0.486519380444861\\
423	0.534848943543693\\
424	0.57442860934043\\
425	0.604591062653981\\
426	0.624800110167598\\
427	0.634701784351364\\
428	0.634177882373024\\
429	0.623257363931771\\
430	0.602098926173767\\
431	0.570995350183694\\
432	0.530584672702389\\
433	0.481397771429783\\
434	0.424352860045929\\
435	0.360273820204839\\
436	0.290341911786072\\
437	0.215568828486539\\
438	0.137299354513743\\
440	-0.02468732811667\\
441	-0.105736315924787\\
442	-0.185089397455613\\
443	-0.26135364844049\\
444	-0.333325538073041\\
445	-0.399830475055296\\
446	-0.459787007948307\\
447	-0.512190731126339\\
448	-0.556206008082881\\
449	-0.591043282356623\\
450	-0.616124146198672\\
451	-0.631142045382148\\
452	-0.63576422336655\\
453	-0.629936372996326\\
454	-0.613791707585278\\
455	-0.587538592866167\\
456	-0.551628748082294\\
457	-0.50666564065159\\
458	-0.453389142263859\\
459	-0.392621684972255\\
460	-0.325425283259847\\
461	-0.252833265993559\\
462	-0.1761295300239\\
463	-0.0965047313693503\\
466	0.146433148353935\\
467	0.224379532702642\\
468	0.298603194177304\\
469	0.367909440837138\\
470	0.431189032297425\\
471	0.487400263430118\\
472	0.535583668323397\\
473	0.574990439787598\\
474	0.604930535716903\\
475	0.624933562936803\\
476	0.634677458209808\\
477	0.633955472058915\\
478	0.622851073445872\\
479	0.601480623754469\\
480	0.570283378790464\\
481	0.534136505053539\\
482	0.489323347469963\\
483	0.436269066231944\\
484	0.375725556972611\\
485	0.30851831466407\\
486	0.235571542559228\\
487	0.158030039785444\\
488	0.0770741234673551\\
490	-0.0902890703159756\\
491	-0.174065850409079\\
492	-0.256143285971575\\
493	-0.335224467415628\\
494	-0.409928058745209\\
495	-0.479153373808003\\
496	-0.541776086119171\\
497	-0.596757129545495\\
498	-0.643203878055374\\
499	-0.680374699647473\\
500	-0.707653016545464\\
501	-0.724513708199083\\
502	-0.730713684909006\\
503	-0.726206850000381\\
504	-0.711108671960119\\
505	-0.685708340583915\\
506	-0.650429580886339\\
507	-0.605949752180095\\
508	-0.553143653465668\\
509	-0.492882385159191\\
510	-0.426319057105047\\
511	-0.354734323178036\\
512	-0.279272071641117\\
514	-0.122545924438782\\
515	-0.0440331006120687\\
516	0.0326358006659575\\
517	0.106163601890785\\
518	0.175200563353883\\
519	0.238567508979941\\
520	0.295141334464915\\
521	0.34399639821504\\
522	0.384295367163759\\
523	0.415408472448235\\
524	0.436903260573672\\
525	0.448493324572155\\
526	0.45013505700399\\
527	0.442016321330357\\
528	0.424398028202631\\
529	0.397919361986624\\
530	0.363128845364372\\
531	0.321003471540735\\
532	0.272486670185117\\
533	0.218789741043565\\
534	0.161134353666512\\
535	0.100776574236988\\
537	-0.0226144491584819\\
538	-0.0828560874520008\\
539	-0.140408427921557\\
540	-0.193964810680882\\
541	-0.242427567066898\\
542	-0.289319003186847\\
543	-0.329341801479131\\
544	-0.361595030891749\\
545	-0.385232320848445\\
546	-0.399665458844993\\
547	-0.404603522041725\\
548	-0.399866256110272\\
549	-0.385577713065231\\
550	-0.362069976462863\\
551	-0.32998149155037\\
552	-0.290046442458333\\
553	-0.243169448572189\\
554	-0.190510383486526\\
555	-0.133342074423126\\
556	-0.0730316537546969\\
559	0.112653054613475\\
560	0.171121467752073\\
561	0.225582357150415\\
562	0.274655691427597\\
563	0.317163233934934\\
564	0.352157501718921\\
565	0.378774566246648\\
566	0.396344757390125\\
567	0.404487576850897\\
568	0.402987590418888\\
569	0.391922425549637\\
570	0.371512071903453\\
571	0.342260737101697\\
572	0.304914357649068\\
573	0.260266446915921\\
574	0.209447850144443\\
575	0.153649986101755\\
576	0.0942027006490207\\
579	-0.091646681983093\\
580	-0.151179866332768\\
581	-0.207147583016649\\
582	-0.25817748063082\\
583	-0.303087929275534\\
584	-0.340754183707304\\
585	-0.37029178368266\\
586	-0.390986733524642\\
587	-0.402368154893793\\
588	-0.404151727471799\\
589	-0.39631303419219\\
590	-0.379081798790139\\
591	-0.352775211033531\\
592	-0.318110391018308\\
593	-0.275875605572764\\
594	-0.227024847851681\\
595	-0.172777739390312\\
596	-0.11441868222073\\
597	-0.0532703249023143\\
599	0.0713519013543191\\
600	0.131833334763996\\
601	0.189192700535841\\
602	0.242068843016568\\
603	0.289243702496606\\
604	0.329457435875611\\
605	0.361886148330996\\
606	0.385765281822842\\
607	0.400408859557956\\
608	0.40558382150175\\
609	0.40110183870911\\
610	0.387038170119922\\
611	0.363795557410413\\
612	0.331952789847946\\
613	0.292226529112213\\
614	0.245548655156199\\
615	0.193048753006678\\
616	0.135960721710489\\
617	0.0756253827316868\\
620	-0.110209755823234\\
621	-0.168802579742987\\
622	-0.223376722745343\\
623	-0.272635643664671\\
624	-0.315345960590548\\
625	-0.350585596983365\\
626	-0.377492373974292\\
627	-0.395389378780237\\
628	-0.403838232263297\\
629	-0.402639202814953\\
630	-0.391881822749838\\
631	-0.371814974771041\\
632	-0.342877969571873\\
633	-0.305759998558642\\
634	-0.261425593388594\\
635	-0.210795880890146\\
636	-0.155176858087088\\
637	-0.095886788745247\\
641	0.149179870078569\\
642	0.205116612808979\\
643	0.25613897613357\\
644	0.301070899030492\\
645	0.33883058475476\\
646	0.368455850458304\\
647	0.389273577687163\\
648	0.400792112322961\\
649	0.402736987134631\\
650	0.395105034735025\\
651	0.37808142791755\\
652	0.352051537621719\\
653	0.317614361799315\\
654	0.275629747913172\\
655	0.227037172742257\\
656	0.172984514515974\\
657	0.11482956758573\\
658	0.0539285986337745\\
660	-0.0703652073548255\\
661	-0.130724844094402\\
662	-0.187979484926473\\
663	-0.240780681997421\\
664	-0.287885154507876\\
665	-0.328144985955532\\
666	-0.360552025714242\\
667	-0.384422428791822\\
668	-0.399084761113954\\
669	-0.404254461023811\\
670	-0.399853258342318\\
671	-0.385901095273312\\
672	-0.362816962421221\\
673	-0.331070028824797\\
674	-0.29147978277706\\
675	-0.244906274754157\\
676	-0.192524425737702\\
677	-0.135542552030074\\
678	-0.0752910119304033\\
681	0.110267550950084\\
682	0.168862580389487\\
683	0.223430198245296\\
684	0.272674987643768\\
685	0.315372218264656\\
686	0.350601415193523\\
687	0.377503749705284\\
688	0.395455954454519\\
689	0.403900177007472\\
690	0.402708380047443\\
691	0.391976430287286\\
692	0.371891184910965\\
693	0.34297702734284\\
694	0.30584642685244\\
695	0.261500626517318\\
696	0.210831597978995\\
697	0.155200740672171\\
698	0.095829846698507\\
702	-0.149637908684326\\
703	-0.205687254251188\\
704	-0.256863396280551\\
705	-0.301907862788084\\
706	-0.339813912918544\\
707	-0.369542950517371\\
708	-0.390535814952727\\
709	-0.40219100696595\\
710	-0.404317703521883\\
711	-0.396841039965238\\
712	-0.379945607133322\\
713	-0.354090310555421\\
714	-0.31974084221747\\
715	-0.277804753397504\\
716	-0.229233975310763\\
717	-0.175217107112985\\
718	-0.117046866512737\\
719	-0.0561404504637721\\
721	0.0681578263634037\\
722	0.128631603989561\\
723	0.18601588049205\\
724	0.238914442259556\\
725	0.286089528298362\\
726	0.326502774854816\\
727	0.359113345140486\\
728	0.383108212868137\\
729	0.397976680734246\\
730	0.403340146776372\\
731	0.39908585432886\\
732	0.385344232036914\\
733	0.362433279489778\\
734	0.330789664854819\\
735	0.291308924970508\\
736	0.244839292465713\\
737	0.192530014428485\\
738	0.135576015246443\\
739	0.0753770190908654\\
742	-0.110132110263748\\
743	-0.168735466980252\\
744	-0.223316741532926\\
745	-0.272645509818631\\
746	-0.315446866170078\\
747	-0.350772472361768\\
748	-0.377743181080405\\
749	-0.395740671863678\\
750	-0.404321037864975\\
751	-0.403289469916217\\
752	-0.39270570323788\\
753	-0.372749939204823\\
754	-0.343974934470225\\
755	-0.306953503098157\\
756	-0.26262995133493\\
757	-0.212054305554375\\
758	-0.156497814353315\\
759	-0.0972469070470652\\
763	0.148318554592151\\
764	0.204499544311602\\
765	0.255777448394838\\
766	0.300973191255252\\
767	0.338997214852043\\
768	0.368862321055076\\
769	0.390010952489774\\
770	0.401854145099605\\
771	0.404191177879056\\
772	0.396872447163787\\
773	0.380071017554656\\
774	0.354304439954376\\
775	0.320093390135753\\
776	0.278239509239938\\
777	0.22973951455424\\
778	0.175774871459453\\
779	0.117607491228227\\
780	0.0566711118299281\\
782	-0.0676964963422506\\
783	-0.128224546794627\\
784	-0.185747950119094\\
785	-0.238721839961272\\
786	-0.286030510862929\\
787	-0.326575963242249\\
788	-0.359346122128954\\
789	-0.383533589958461\\
790	-0.398611271828031\\
791	-0.404173242204706\\
792	-0.400102247961968\\
793	-0.386540227400474\\
794	-0.363781008761634\\
795	-0.332299826775397\\
796	-0.292919973195239\\
797	-0.246528550095718\\
798	-0.194287218502268\\
799	-0.137401544090608\\
800	-0.0772365417187757\\
801	-0.0253320103629449\\
803	0.0835265398227421\\
804	0.138178199015329\\
805	0.191508667836843\\
806	0.242419403878557\\
807	0.289767863331235\\
808	0.332581254878278\\
809	0.369855506486147\\
810	0.400791437973567\\
811	0.424612939689268\\
812	0.440760845940531\\
813	0.448763067171058\\
814	0.448167367583665\\
815	0.438911093217939\\
816	0.420885383922268\\
817	0.394286472901967\\
818	0.359393116973024\\
819	0.316711491874685\\
820	0.26669886489799\\
821	0.210056461954082\\
822	0.147582371258068\\
823	0.0802403490979486\\
824	0.00899463215455398\\
825	-0.065120624422434\\
829	-0.36766294414565\\
830	-0.439259122655585\\
831	-0.507101745706677\\
832	-0.570276924911013\\
833	-0.62774647822016\\
834	-0.678709804744813\\
835	-0.722583492645299\\
836	-0.758720662954602\\
837	-0.786606514930099\\
838	-0.805922628418557\\
839	-0.816394902265984\\
840	-0.817970868055909\\
841	-0.810638037648914\\
842	-0.794581815720449\\
843	-0.770031622580518\\
844	-0.737354183021125\\
845	-0.69698357113748\\
846	-0.649608877176888\\
847	-0.595727573786917\\
848	-0.536079372696349\\
849	-0.471447948225887\\
850	-0.402641443210541\\
851	-0.330551501941045\\
852	-0.256173663945901\\
856	0.0477788200587383\\
857	0.120895701952577\\
858	0.191132573928371\\
859	0.257883454583862\\
860	0.320296310613685\\
861	0.377720219531966\\
862	0.439952284378705\\
863	0.494850437918103\\
864	0.541657531485725\\
865	0.579510337854117\\
866	0.607852276426001\\
867	0.62618413605378\\
868	0.634253588464162\\
869	0.63186011629432\\
870	0.619094576640691\\
871	0.596069707644801\\
872	0.563268493269788\\
873	0.521194299901254\\
874	0.470502114671945\\
875	0.41215513340876\\
876	0.346979776313674\\
877	0.276079124857006\\
878	0.200636776290594\\
879	0.121896914742592\\
882	-0.121119242344321\\
883	-0.199911599874667\\
884	-0.275437224844154\\
885	-0.346507285158623\\
886	-0.411844166635547\\
887	-0.470381595232084\\
888	-0.52116584675332\\
889	-0.563465030936186\\
890	-0.596520047072318\\
891	-0.619767269076874\\
892	-0.632839399698241\\
893	-0.635523575630486\\
894	-0.62779346332627\\
895	-0.609737772295375\\
896	-0.581633508638788\\
897	-0.544033090396169\\
898	-0.497486622160977\\
899	-0.442769400491215\\
900	-0.380793573572191\\
901	-0.312496298612132\\
902	-0.239125478403366\\
903	-0.161802767316658\\
904	-0.0818150925365444\\
906	0.0807989980025923\\
907	0.160836718920564\\
908	0.238223317370284\\
909	0.311648258320474\\
910	0.379877155949089\\
911	0.441898969752856\\
912	0.496641745222405\\
913	0.543344987623641\\
914	0.581026781242144\\
915	0.609176494013809\\
916	0.627277975044308\\
917	0.635096515367877\\
918	0.632473959494746\\
919	0.619488959665432\\
920	0.59628539624282\\
921	0.563281920589816\\
922	0.52103911488939\\
923	0.470179774395092\\
924	0.411661595426722\\
925	0.346350101152893\\
926	0.275414879707114\\
927	0.199889087863085\\
928	0.121072407035172\\
931	-0.122090177486371\\
932	-0.200885879411089\\
933	-0.276329149057347\\
934	-0.347259951184242\\
935	-0.412451805500041\\
936	-0.47091641462066\\
937	-0.521594140785965\\
938	-0.563679181581392\\
939	-0.596549735087592\\
940	-0.619629069657549\\
941	-0.632528650948643\\
942	-0.635006718688146\\
943	-0.627026657142778\\
944	-0.608750890211468\\
945	-0.580472755547817\\
946	-0.542595895738032\\
947	-0.495869772749757\\
948	-0.440980226136162\\
949	-0.378871294795772\\
950	-0.310495014173284\\
951	-0.237032254618498\\
952	-0.159713981873665\\
953	-0.0796663349206028\\
955	0.0830748083262733\\
956	0.163029694825582\\
957	0.240334553177945\\
958	0.313651950534222\\
959	0.381860862853955\\
960	0.443814198714335\\
961	0.498505704591025\\
962	0.544970332342928\\
963	0.582545037198088\\
964	0.61051228264887\\
965	0.628453977850768\\
966	0.636041383861993\\
967	0.63324880635173\\
968	0.619969979481539\\
969	0.596509139366844\\
970	0.56331558445072\\
971	0.520923432652125\\
972	0.469908199927431\\
973	0.411236229035239\\
974	0.345750337709887\\
975	0.274677798979155\\
976	0.199062801713353\\
977	0.120201024119524\\
980	-0.122894145928512\\
981	-0.201688452340477\\
982	-0.277112335170386\\
983	-0.348000196063367\\
984	-0.413132626019433\\
985	-0.471487799367878\\
986	-0.522024718221019\\
987	-0.564058536917855\\
988	-0.596829092678945\\
989	-0.6197420592257\\
990	-0.63243799495649\\
991	-0.634770863718131\\
992	-0.62659829241511\\
993	-0.608140530509445\\
994	-0.579708295455021\\
995	-0.541734242502571\\
996	-0.494871655182578\\
997	-0.439842668890833\\
998	-0.377607200367038\\
999	-0.309185459395394\\
1000	-0.235685117327193\\
1001	-0.158233812495837\\
1002	-0.0781997769822738\\
1004	0.0843683584230348\\
1005	0.164200940322644\\
1006	0.241431692075821\\
1007	0.314588655843636\\
1008	0.382683205416924\\
1009	0.44445468584172\\
1010	0.498895278487907\\
1011	0.545153815252888\\
1012	0.582417936791444\\
1013	0.610173394817139\\
1014	0.627883552094318\\
1015	0.635289636109974\\
1016	0.632299004850211\\
1017	0.618902789219646\\
1018	0.595309231894589\\
1019	0.562013127758291\\
1020	0.519414317653627\\
1021	0.46823442314826\\
1022	0.409400751011162\\
1023	0.343826401394381\\
1024	0.272558991198366\\
1025	0.196844671701001\\
1026	0.117902337694886\\
1029	-0.125079747966993\\
1030	-0.203702937714752\\
1031	-0.278985497008762\\
1032	-0.349674392699853\\
1033	-0.414612630618194\\
1034	-0.472767743541226\\
1035	-0.523102553927401\\
1036	-0.564872959179866\\
1037	-0.597362331285694\\
1038	-0.620020317313447\\
1039	-0.632450190907093\\
1040	-0.634509381017779\\
1041	-0.626134715848821\\
1042	-0.607410012059518\\
1043	-0.578749260542736\\
1044	-0.540564881191131\\
1045	-0.493461723159726\\
1046	-0.43827940420897\\
1047	-0.375901278488527\\
1048	-0.307270759883522\\
1049	-0.233537307718962\\
1050	-0.155976309121343\\
1051	-0.0758424850300798\\
1053	0.0867312526770547\\
1054	0.166577253423839\\
1055	0.243684957351434\\
1056	0.31676710734564\\
1057	0.384744597765803\\
1058	0.446361205396443\\
1059	0.500633471747278\\
1060	0.546708131194464\\
1061	0.583887583194155\\
1062	0.611390416009272\\
1063	0.62884616376914\\
1064	0.635961287462578\\
1065	0.632677722799599\\
1066	0.61904001660605\\
1067	0.595184284177321\\
1068	0.561580275023971\\
1069	0.518680823909108\\
1070	0.467283463011427\\
1071	0.408300345424777\\
1072	0.342588436401002\\
1073	0.271301057364781\\
1074	0.195540371594689\\
1075	0.116527686902373\\
1078	-0.126720645164369\\
1079	-0.205418329569966\\
1080	-0.280766235934152\\
1081	-0.351428922934247\\
1082	-0.41626069854874\\
1083	-0.474334678625382\\
1084	-0.524572251969857\\
1085	-0.566193493781157\\
1086	-0.598525875775522\\
1087	-0.621024863578896\\
1088	-0.633262455500699\\
1089	-0.635128238553079\\
1090	-0.62659124975653\\
1091	-0.607787783651929\\
1092	-0.578995168466918\\
1093	-0.540683465805159\\
1094	-0.493486526472225\\
1095	-0.438145606453418\\
1096	-0.375629936908808\\
1097	-0.306973528959588\\
1098	-0.233263524751237\\
1099	-0.155720745588951\\
1100	-0.0756230268734726\\
1102	0.0869635437766192\\
1103	0.166771217898713\\
1104	0.243835927873079\\
1105	0.316908375862567\\
1106	0.384707946181152\\
1107	0.446266893302891\\
1108	0.500419521441927\\
1109	0.546338959663444\\
1110	0.583297376259907\\
1111	0.610644844542094\\
1112	0.627919099393239\\
1113	0.634931233241332\\
1114	0.631542392774008\\
1115	0.6177337064114\\
1116	0.593794271117076\\
1117	0.560090495412169\\
1118	0.517110825239342\\
1119	0.465679068649933\\
1120	0.406595399539128\\
1121	0.340819551479854\\
1122	0.269420287264438\\
1123	0.193575870426685\\
1124	0.114669804531786\\
1127	-0.128185209531694\\
1128	-0.20669985905397\\
1129	-0.281802859001345\\
1130	-0.352259716074059\\
1131	-0.41690515450091\\
1132	-0.47476375491442\\
1133	-0.524784029007151\\
1134	-0.566186056692914\\
1135	-0.598255344331392\\
1136	-0.62048668295256\\
1137	-0.63249480839977\\
1138	-0.634204068168856\\
1139	-0.625429389976034\\
1140	-0.606352113757112\\
1141	-0.577245388918072\\
1142	-0.538738593254038\\
1143	-0.491331975072171\\
1144	-0.435845616809729\\
1145	-0.373204268834797\\
1146	-0.304411968467321\\
1147	-0.230581616191103\\
1148	-0.152966895698228\\
1149	-0.0729412016034985\\
1151	0.0895936391102623\\
1152	0.169313408202015\\
1153	0.2463209547509\\
1154	0.319229323939453\\
1155	0.386961776972839\\
1156	0.448245081275218\\
1157	0.502197648661877\\
1158	0.547934213722328\\
1159	0.584694480407506\\
1160	0.611824220720791\\
1161	0.62887411970496\\
1162	0.635651616257292\\
1163	0.632001945919455\\
1164	0.617893240867943\\
1165	0.593688163489787\\
1166	0.559717549949255\\
1167	0.516597450606241\\
1168	0.464945503017589\\
1169	0.405668741251247\\
1170	0.33973968090595\\
1171	0.268197433716523\\
1172	0.192263673251091\\
1173	0.113215609915642\\
1176	-0.129946762911004\\
1177	-0.208481514181585\\
1178	-0.283550297077909\\
1179	-0.353988609656426\\
1180	-0.418639247595365\\
1181	-0.476400901557099\\
1182	-0.526293567413632\\
1183	-0.567558621148692\\
1184	-0.599477520998789\\
1185	-0.621579391651721\\
1186	-0.633453179400931\\
1187	-0.634877126936772\\
1188	-0.625922345189792\\
1189	-0.606722824644748\\
1190	-0.5775086079434\\
1191	-0.538838403547288\\
1192	-0.491275880206558\\
1193	-0.435590759744628\\
1194	-0.372860001997196\\
1195	-0.303987603685528\\
1196	-0.230074243267609\\
1197	-0.152413160429205\\
1198	-0.0722405833166704\\
1200	0.0903571616981935\\
1201	0.170098708830665\\
1202	0.247003543136998\\
1203	0.319893939225949\\
1204	0.387493044343955\\
1205	0.448698610395695\\
1206	0.502540153883729\\
1207	0.548129983300896\\
1208	0.584731659870158\\
1209	0.611724372055505\\
1210	0.628705157241257\\
1211	0.635278771243065\\
1212	0.631391655926564\\
1213	0.617198878728232\\
1214	0.592781684716101\\
1215	0.558700178213712\\
1216	0.515390208011013\\
1217	0.463629648904771\\
1218	0.404246594061533\\
1219	0.338192556625927\\
1220	0.266560881372243\\
1221	0.190602028832927\\
1222	0.111507697668003\\
1225	-0.131609600143747\\
1226	-0.210075614650123\\
1227	-0.285059438613189\\
1228	-0.35539438602018\\
1229	-0.419856980328404\\
1230	-0.477390984820431\\
1231	-0.527097700063223\\
1232	-0.56813254065537\\
1233	-0.599806710361918\\
1234	-0.621673940961955\\
1235	-0.633234732801156\\
1236	-0.634484388789588\\
1237	-0.625173233696842\\
1238	-0.605644857885181\\
1239	-0.576199652293326\\
1240	-0.537282122783836\\
1241	-0.489438878714736\\
1242	-0.433515514139799\\
1243	-0.37051021584648\\
1244	-0.301356801877773\\
1245	-0.227254819962127\\
1246	-0.149463253773774\\
1248	0.01216676530521\\
1249	0.0934183530043811\\
1250	0.173158556334783\\
1251	0.250089638303507\\
1252	0.32293875461437\\
1253	0.3904245526046\\
1254	0.45149732033633\\
1255	0.505258052279714\\
1256	0.55072763917633\\
1257	0.587097119853752\\
1258	0.613839427192033\\
1259	0.630507443569968\\
1260	0.636868291906922\\
1261	0.632785628807596\\
1262	0.618321880666826\\
1263	0.593669032794423\\
1264	0.55931823347737\\
1265	0.515784319799877\\
1266	0.463784769793165\\
1267	0.404183652448864\\
1268	0.337977589466391\\
1269	0.266241348655967\\
1270	0.190113997050958\\
1271	0.110902282472125\\
1274	-0.132103804616236\\
1275	-0.210515793181457\\
1276	-0.285457987982227\\
1277	-0.355724204569469\\
1278	-0.420113629513708\\
1279	-0.477545605657269\\
1280	-0.527080966138328\\
1281	-0.556921870156202\\
1282	-0.576983385053154\\
1283	-0.587174971330569\\
1284	-0.587742556677767\\
1285	-0.579112933839951\\
1286	-0.561849016095948\\
1287	-0.536757688222224\\
1288	-0.50474908753722\\
1289	-0.466765150991705\\
1290	-0.423940710460101\\
1291	-0.377538234441545\\
1293	-0.27874514814448\\
1294	-0.228870779230874\\
1295	-0.180210813113717\\
1296	-0.133992042824502\\
1297	-0.0911327189837721\\
1298	-0.0526832710497729\\
1299	-0.0193383441437618\\
1300	0.00828997263079145\\
1301	0.0297521190518637\\
1302	0.0446699885119415\\
1303	0.0529540518282374\\
1304	0.0547190980551022\\
1305	0.0501105517787437\\
1306	0.0396261952837449\\
1307	0.0238100193400896\\
1308	0.00332375478774338\\
1309	-0.0210194871774547\\
1310	-0.0483369378616771\\
1312	-0.10793835565164\\
1314	-0.167327663233209\\
1315	-0.194380060842832\\
1316	-0.218395352183961\\
1317	-0.238461938564342\\
1318	-0.253791803817421\\
1319	-0.263771177471426\\
1320	-0.267858585803879\\
1321	-0.2657837304846\\
1322	-0.25721522038566\\
1323	-0.242176790607573\\
1324	-0.220725780930479\\
1325	-0.193266082639184\\
1326	-0.160222439103563\\
1327	-0.122096009795769\\
1328	-0.0797858402161182\\
1329	-0.0341159861395681\\
1330	0.0139667542052848\\
1333	0.162094136659562\\
1334	0.209068988182935\\
1335	0.253064077966428\\
1336	0.292956875056916\\
1337	0.327731666608088\\
1338	0.35664095965376\\
1339	0.378904121803771\\
1340	0.39383320255547\\
1341	0.401018455668691\\
1342	0.388885767550619\\
1343	0.367519323734086\\
1344	0.337366664503406\\
1345	0.299170854055774\\
1346	0.253801343453233\\
1347	0.20239695561304\\
1348	0.146181731096476\\
1349	0.0865114365474255\\
1352	-0.0990960339463527\\
1353	-0.158185512221735\\
1354	-0.213549288302602\\
1355	-0.263846338876647\\
1356	-0.307855162148371\\
1357	-0.344559240962099\\
1358	-0.373048059638222\\
1359	-0.392625037882681\\
1360	-0.402848689845086\\
1361	-0.40347555057042\\
1362	-0.394513260504937\\
1363	-0.376164681485989\\
1364	-0.348830206145522\\
1365	-0.313193370121098\\
1366	-0.270104351987811\\
1367	-0.220574756216593\\
1368	-0.165804614701301\\
1369	-0.107026768192554\\
1371	0.0166350335803145\\
1372	0.0785939398374467\\
1373	0.138733640686041\\
1374	0.195605695019822\\
1375	0.247765455331319\\
1376	0.294069706641494\\
1377	0.333336023530137\\
1378	0.364750526714943\\
1379	0.387431023454155\\
1380	0.400931110668807\\
1381	0.40491314590372\\
1382	0.399273645847188\\
1383	0.384116189457473\\
1384	0.359780549730658\\
1385	0.326890112657111\\
1386	0.286268277352974\\
1387	0.238856847693114\\
1388	0.185682216358146\\
1389	0.128094777739989\\
1390	0.0675067484926331\\
1392	-0.0570334490457753\\
1393	-0.117940537739742\\
1394	-0.176092689795041\\
1395	-0.230004054347773\\
1396	-0.278520992610538\\
1397	-0.320359431534598\\
1398	-0.354547422575706\\
1399	-0.380258207281258\\
1400	-0.396996648354616\\
1401	-0.404251495822336\\
1402	-0.401896936129106\\
1403	-0.389981227214321\\
1404	-0.368771063843724\\
1405	-0.338750603802964\\
1406	-0.300651657697017\\
1407	-0.255411703692516\\
1408	-0.204108387857104\\
1409	-0.147963599165905\\
1410	-0.0882659846770366\\
1413	0.0974352397838629\\
1414	0.156600053036072\\
1415	0.212074897366165\\
1416	0.262521287514573\\
1417	0.306644323514774\\
1418	0.343524441494083\\
1419	0.372213968790675\\
1420	0.392006432885864\\
1421	0.402469836128603\\
1422	0.40334124123865\\
1423	0.394610080997609\\
1424	0.376513405617061\\
1425	0.349430352376658\\
1426	0.314016144036486\\
1427	0.271119859515238\\
1428	0.22179768937167\\
1429	0.16720730769066\\
1430	0.108584418819646\\
1432	-0.0149209413871176\\
1433	-0.0768964010421769\\
1434	-0.137057428953085\\
1435	-0.193939800712542\\
1436	-0.246149154746945\\
1437	-0.29251241960219\\
1438	-0.33185492294524\\
1439	-0.363370958717042\\
1440	-0.386192190484962\\
1441	-0.39448487726213\\
1442	-0.393878227285313\\
1443	-0.384427125703951\\
1444	-0.36645169532585\\
1445	-0.340426744732667\\
1446	-0.307061003689341\\
1447	-0.267180883752189\\
1448	-0.221829002354298\\
1449	-0.171954078697581\\
1450	-0.118870593065367\\
1453	0.0465073526784181\\
1454	0.0991744106927399\\
1455	0.148528200571491\\
1456	0.193257431215443\\
1457	0.232337153794106\\
1458	0.264688780425786\\
1459	0.289535077656637\\
1460	0.306192542048393\\
1461	0.314374899536688\\
1462	0.313644261342688\\
1463	0.303999647474029\\
1464	0.285508716604909\\
1465	0.258521411069978\\
1466	0.223577611939618\\
1467	0.181334757110562\\
1468	0.132801101519362\\
1469	0.0789694955128653\\
1470	0.0209141893669766\\
1471	-0.0400065215267205\\
1474	-0.226308355005131\\
1475	-0.28492125572393\\
1476	-0.339513958796942\\
1477	-0.388791820882489\\
1478	-0.431525646551108\\
1479	-0.466715366142125\\
1480	-0.493484344799981\\
1481	-0.5110105808958\\
1482	-0.518790653294673\\
1483	-0.516497221520694\\
1484	-0.503967314507008\\
1485	-0.481229940377943\\
1486	-0.448585471538536\\
1487	-0.406508587839653\\
1488	-0.355663976870801\\
1489	-0.296978046101231\\
1490	-0.231436483619746\\
1491	-0.160179220538339\\
1492	-0.0845737207073398\\
1493	-0.00596495971285549\\
1495	0.154324112502763\\
1496	0.232952329641194\\
1497	0.308560934470279\\
1498	0.379745008565351\\
1499	0.445129845612428\\
1500	0.503435052678469\\
1501	0.553539103344065\\
1502	0.588971444329673\\
1503	0.614721642098175\\
1504	0.630396726640356\\
1505	0.635668842914129\\
1506	0.630531908241664\\
1507	0.615033609775764\\
1508	0.589408001852462\\
1509	0.55412716192177\\
1510	0.509599239518593\\
1511	0.456726775587867\\
1512	0.39637470009302\\
1513	0.329541839815192\\
1514	0.257317483044517\\
1515	0.180866349116968\\
1516	0.101441446998706\\
1519	-0.141558792568958\\
1520	-0.21958999342587\\
1521	-0.293991201260724\\
1522	-0.363635870261533\\
1523	-0.427315931049179\\
1524	-0.483963438846331\\
1525	-0.53263677726045\\
1526	-0.572584270367315\\
1527	-0.603137885415435\\
1528	-0.623688206542738\\
1529	-0.634045412312389\\
1530	-0.63398382538071\\
1531	-0.623372899238348\\
1532	-0.602584214654144\\
1533	-0.571896709686825\\
1534	-0.531813527002669\\
1535	-0.483025472496138\\
1536	-0.426261248303945\\
1537	-0.362461066855758\\
1538	-0.292683244172167\\
1539	-0.218139335182968\\
1540	-0.140000166222762\\
1542	0.0217847038525179\\
1543	0.102836206834127\\
1544	0.182199379545182\\
1545	0.258597502443536\\
1546	0.330754342069213\\
1547	0.397444415540122\\
1548	0.457617661872973\\
1549	0.510280957936175\\
1550	0.554554735979764\\
1551	0.589701682050872\\
1552	0.615139109048414\\
1553	0.630516578151855\\
1554	0.635503002733458\\
1555	0.630133565766755\\
1556	0.614324829889483\\
1557	0.588402393466367\\
1558	0.552875187457175\\
1559	0.508202459762288\\
1560	0.455149707738201\\
1561	0.394661194799937\\
1562	0.327701074307697\\
1563	0.255346012797872\\
1564	0.178780940388151\\
1565	0.0992686158792822\\
1568	-0.143645302287041\\
1569	-0.221695301081127\\
1570	-0.296086191674476\\
1571	-0.365536505868022\\
1572	-0.428966927695001\\
1573	-0.485434147989508\\
1574	-0.533877229760947\\
1575	-0.573645778476475\\
1576	-0.60398830695749\\
1577	-0.624378145623723\\
1578	-0.634504262759947\\
1579	-0.634216195661338\\
1580	-0.62349916855419\\
1581	-0.602563718126476\\
1582	-0.571753195451493\\
1583	-0.531503657093253\\
1584	-0.482538604197089\\
1585	-0.425669323137754\\
1586	-0.361798779002129\\
1587	-0.29197143602687\\
1588	-0.217328318874934\\
1589	-0.139203158886176\\
1591	0.0226410583295547\\
1592	0.103617656539427\\
1593	0.182905370293611\\
1594	0.259216510778515\\
1595	0.331323802181942\\
1596	0.397935349712952\\
1597	0.457977484302319\\
1598	0.510418461441077\\
1599	0.554473670576044\\
1600	0.589467631413754\\
1601	0.614763672040226\\
1602	0.629981731868156\\
1603	0.634835626429322\\
1604	0.629231891098698\\
1605	0.613269492675499\\
1606	0.587285060246813\\
1607	0.551573939756508\\
1608	0.506812613576585\\
1609	0.453765803713395\\
1610	0.39320798311519\\
1611	0.326197347274956\\
1612	0.253821396607236\\
1613	0.177254900812841\\
1614	0.097745141094947\\
1617	-0.145091669857266\\
1618	-0.22299878408694\\
1619	-0.29724877413264\\
1620	-0.366599978312479\\
1621	-0.429915866083775\\
1622	-0.48623020621244\\
1623	-0.534593269933339\\
1624	-0.574139902071693\\
1625	-0.604204726284024\\
1626	-0.624350151006183\\
1627	-0.634297732564846\\
1628	-0.633858438537118\\
1629	-0.622919842131978\\
1630	-0.601725537884249\\
1631	-0.570627847401738\\
1632	-0.530195801949503\\
1633	-0.481051080564157\\
1634	-0.42398103198002\\
1635	-0.359954280599595\\
1636	-0.289984130739867\\
1637	-0.215214067288798\\
1638	-0.136976994619545\\
1640	0.0250488730121106\\
1641	0.106105104882772\\
1642	0.185406353664803\\
1643	0.261687217159761\\
1644	0.333673950889988\\
1645	0.400211645745458\\
1646	0.460191719038448\\
1647	0.512564308390665\\
1648	0.556515825961469\\
1649	0.591342712419191\\
1650	0.616496533968075\\
1651	0.631446117266478\\
1652	0.636067717689912\\
1653	0.630229950582816\\
1654	0.614070089804954\\
1655	0.587807606053957\\
1656	0.551797724425342\\
1657	0.506765066079424\\
1658	0.453445752366861\\
1659	0.392685850632461\\
1660	0.325485466944883\\
1661	0.252908954375016\\
1662	0.176153579419861\\
1663	0.0965105721465989\\
1666	-0.146533624363656\\
1667	-0.224493455399625\\
1668	-0.298788250067446\\
1669	-0.3681494590287\\
1670	-0.431491840944545\\
1671	-0.487690405582725\\
1672	-0.535914017231789\\
1673	-0.57536958593073\\
1674	-0.605288161844328\\
1675	-0.625283754367501\\
1676	-0.634986272507831\\
1677	-0.634336409061234\\
1678	-0.623223384536686\\
1679	-0.601846671100702\\
1680	-0.570631809493534\\
1681	-0.530023106886347\\
1682	-0.480716298242896\\
1683	-0.423471945470283\\
1684	-0.359212527915133\\
1685	-0.289101226831008\\
1686	-0.214294156121014\\
1687	-0.135911385499185\\
1689	0.0262794176824173\\
1690	0.107404676426995\\
1691	0.186695592386059\\
1692	0.26289546666294\\
1693	0.334838265329381\\
1694	0.401285697101684\\
1695	0.461141984971619\\
1696	0.513374596714129\\
1697	0.557178709384516\\
1698	0.591815178202978\\
1699	0.616791517667934\\
1700	0.631553468575021\\
1701	0.635985667730893\\
1702	0.629980021029041\\
1703	0.613645660638213\\
1704	0.587191278461432\\
1705	0.55107496975188\\
1706	0.505899949753257\\
1707	0.452424592205261\\
1708	0.39150160166264\\
1709	0.324236239385755\\
1710	0.251578450057877\\
1711	0.174778353109105\\
1712	0.0951481309643896\\
1715	-0.147780046306252\\
1716	-0.225670695156623\\
1717	-0.299808307836429\\
1718	-0.369002337911752\\
1719	-0.432154914482908\\
1720	-0.488118800745724\\
1721	-0.536131559931619\\
1722	-0.575288844212992\\
1723	-0.604985745833346\\
1724	-0.62474519901707\\
1725	-0.634237466838385\\
1726	-0.63327851315853\\
1727	-0.621844082400003\\
1728	-0.600262114720408\\
1729	-0.568814731960629\\
1730	-0.528057341251042\\
1731	-0.478601966628048\\
1732	-0.421277209427899\\
1733	-0.356980484089945\\
1734	-0.286830858923622\\
1735	-0.211991732560364\\
1736	-0.133671045957271\\
1738	0.0283039739524611\\
1739	0.109240258355385\\
1740	0.188370184463565\\
1741	0.264445014618559\\
1742	0.336131222040876\\
1743	0.402308320799875\\
1744	0.461886701238654\\
1745	0.513858131455891\\
1746	0.557399897552386\\
1747	0.591811448070985\\
1748	0.616428726986214\\
1749	0.630958660115539\\
1750	0.635125798995887\\
1751	0.628785871597756\\
1752	0.612186066544837\\
1753	0.585529823458273\\
1754	0.549270039589373\\
1755	0.503960337232002\\
1756	0.450390558086383\\
1757	0.389416154720038\\
1758	0.322036897620364\\
1759	0.249390756045159\\
1760	0.172541887118768\\
1762	0.00733138782516107\\
1763	-0.0732609938290807\\
1764	-0.15072705410239\\
1765	-0.223776617989188\\
1766	-0.291428110351717\\
1767	-0.352719365218036\\
1768	-0.406846510235482\\
1769	-0.453168267397814\\
1770	-0.491387626248979\\
1771	-0.521189087831772\\
1772	-0.542522639690105\\
1773	-0.555585775472537\\
1774	-0.560612669564762\\
1775	-0.558180924816497\\
1776	-0.548878286454965\\
1777	-0.533432195409659\\
1778	-0.512670594714564\\
1779	-0.487596841070626\\
1780	-0.459152651186287\\
1781	-0.428485723187805\\
1784	-0.332948148394735\\
1785	-0.303177080776095\\
1786	-0.275925160744009\\
1787	-0.251821027914502\\
1788	-0.23155465992113\\
1789	-0.215549466515768\\
1790	-0.204045850127386\\
1791	-0.197245810243203\\
1792	-0.195160668394692\\
1793	-0.197632643931684\\
1794	-0.204418001276281\\
1795	-0.214991730901147\\
1796	-0.228877712662779\\
1797	-0.245378504843302\\
1799	-0.2831546802704\\
1801	-0.321644261174242\\
1802	-0.338993992992982\\
1803	-0.353889306001747\\
1804	-0.365548162204959\\
1805	-0.373231959288205\\
1806	-0.376266008011953\\
1807	-0.374093784720117\\
1808	-0.366337891445255\\
1809	-0.352715642292878\\
1810	-0.333067696521994\\
1811	-0.307451844777916\\
1812	-0.2760476781541\\
1813	-0.239149146289037\\
1814	-0.197311855812131\\
1815	-0.151140104670503\\
1816	-0.101313872240553\\
1817	-0.0487179970955367\\
1819	0.0608424412521344\\
1820	0.115733480666677\\
1822	0.22382731598691\\
1823	0.272873425657508\\
1824	0.315456648484542\\
1825	0.350540507060487\\
1826	0.377253766188005\\
1827	0.3949434065471\\
1828	0.403272115122036\\
1829	0.401943753065098\\
1830	0.391025500916385\\
1831	0.370821703906131\\
1832	0.341747665571347\\
1833	0.304521291423498\\
1834	0.260049077534404\\
1835	0.209352011392184\\
1836	0.153691581188923\\
1837	0.0944054495735145\\
1841	-0.150884097104608\\
1842	-0.20686846806575\\
1843	-0.257875540279656\\
1844	-0.302772828394609\\
1845	-0.340472009011592\\
1846	-0.370027920091616\\
1847	-0.390808944989658\\
1848	-0.402305750940741\\
1849	-0.404164721039706\\
1850	-0.396448164035974\\
1851	-0.379245705836638\\
1852	-0.352982636628894\\
1853	-0.318383176831958\\
1854	-0.276180956717781\\
1855	-0.227373693192931\\
1856	-0.173156571689105\\
1857	-0.114814887568627\\
1858	-0.0537132366848709\\
1860	0.0707954306599277\\
1861	0.131255981603317\\
1862	0.188697925558245\\
1863	0.241590082399398\\
1864	0.288735835786156\\
1865	0.329076033719502\\
1866	0.361547538528157\\
1867	0.385411745264264\\
1868	0.400092868585489\\
1869	0.405272393916675\\
1870	0.400867249057228\\
1871	0.386885111081938\\
1872	0.363711536110713\\
1873	0.331920033027927\\
1874	0.292140781719809\\
1875	0.245459207933891\\
1876	0.192967544229305\\
1877	0.13586362957858\\
1878	0.0754945670832967\\
1881	-0.110267190925242\\
1882	-0.168897073214794\\
1883	-0.22356962064714\\
1884	-0.272874304911511\\
1885	-0.315717346717975\\
1886	-0.351002277943735\\
1887	-0.377979327858156\\
1888	-0.39587993062014\\
1889	-0.404408613756914\\
1890	-0.403294337616899\\
1891	-0.392592216245703\\
1892	-0.372600209314442\\
1893	-0.343694815575873\\
1894	-0.306601272755415\\
1895	-0.262194401519537\\
1896	-0.211512722640691\\
1897	-0.155840890141917\\
1898	-0.0964754624078523\\
1902	0.149181573478472\\
1903	0.205350729423117\\
1904	0.256557967664321\\
1905	0.301657698244981\\
1906	0.339622212267841\\
1907	0.369463156854636\\
1908	0.390573826656691\\
1909	0.402324604981459\\
1910	0.404465394800809\\
1911	0.397024082447388\\
1912	0.380132975644301\\
1913	0.354144594931313\\
1914	0.31980322511663\\
1915	0.277815041883059\\
1916	0.229193087771364\\
1917	0.175146217148722\\
1918	0.116937689088445\\
1919	0.0559399805811154\\
1920	-0.0063943320733415\\
1921	-0.0665803764791235\\
1922	-0.124652956341833\\
1923	-0.179392048497448\\
1924	-0.229570184457771\\
1925	-0.274141504133695\\
1926	-0.312253624141704\\
1927	-0.342997685075716\\
1928	-0.36590841401221\\
1929	-0.38049242778925\\
1930	-0.386602725268858\\
1931	-0.384172674436741\\
1932	-0.373408286925041\\
1933	-0.354617876318116\\
1934	-0.328417622621146\\
1935	-0.295476722958483\\
1936	-0.256715412266203\\
1937	-0.213169920907148\\
1938	-0.165993324028022\\
1940	-0.0654999529770066\\
1941	-0.0147274056653259\\
1942	0.0346062538314982\\
1943	0.0813518663521791\\
1944	0.124181532086368\\
1945	0.162007684818946\\
1946	0.193846968335038\\
1947	0.218732393524533\\
1948	0.236032652355789\\
1949	0.245154279895814\\
1950	0.245808170192959\\
1951	0.237780720522096\\
1952	0.221157866536487\\
1953	0.19612697215598\\
1954	0.163081763188984\\
1955	0.122627990393084\\
1956	0.0755724622986236\\
1957	0.0228799242795503\\
1958	-0.0343415662896405\\
1959	-0.0950176481442213\\
1963	-0.344341046050431\\
1964	-0.401539843182945\\
1965	-0.453970395734359\\
1966	-0.500459326635792\\
1967	-0.539925721221152\\
1968	-0.571248615503009\\
1969	-0.593780481900467\\
1970	-0.606694329605034\\
1971	-0.609541036774772\\
1972	-0.602087308844602\\
1973	-0.584318389980581\\
1974	-0.556263869324994\\
1975	-0.518355544940732\\
1976	-0.471069362127309\\
1977	-0.415088244747949\\
1978	-0.351385476853011\\
1979	-0.281032529774166\\
1980	-0.205165605852017\\
1982	-0.0443332681124957\\
1984	0.118038052575685\\
1985	0.196975797478444\\
1986	0.272749647982891\\
1987	0.344048102056149\\
1988	0.409643658339974\\
1989	0.468617128311053\\
1990	0.519800407825187\\
1991	0.56245978193374\\
1992	0.595910542234833\\
1993	0.619586245351002\\
1994	0.633086360893685\\
1995	0.6361804361336\\
1996	0.628856058903239\\
1997	0.611202900546687\\
1998	0.58347366515909\\
1999	0.546255720620593\\
2000	0.500053743521221\\
2001	0.445662653547515\\
2002	0.383914227268178\\
2003	0.315862930716776\\
2004	0.242551610625014\\
2005	0.165388090482338\\
2006	0.0855041764175439\\
2008	-0.0771508439124773\\
2009	-0.157254328896215\\
2010	-0.234750780167815\\
2011	-0.308368234756472\\
2012	-0.376903437377678\\
2013	-0.439242433142681\\
2014	-0.494385771114139\\
2015	-0.541383978488284\\
2016	-0.579443478785379\\
2017	-0.607990951569263\\
2018	-0.626545728770907\\
2019	-0.634770154228136\\
2020	-0.632487275508993\\
2021	-0.619890885251607\\
2022	-0.597009227621584\\
2023	-0.564331970689182\\
2024	-0.522453888268956\\
2025	-0.471907315140925\\
2026	-0.413626381750419\\
2027	-0.348517753576743\\
2028	-0.277665369270835\\
2029	-0.202278113466036\\
2030	-0.123411369702808\\
2033	0.119826753297275\\
2034	0.198734467284339\\
2035	0.27440339121631\\
2036	0.34561721648879\\
2037	0.411173342746224\\
2038	0.46999098346987\\
2039	0.52103179597907\\
2040	0.563581880525817\\
2041	0.596908558350151\\
2042	0.620420904506773\\
2043	0.633721514142508\\
2044	0.636662713156966\\
2045	0.62913063649512\\
2046	0.611325381834831\\
2047	0.58344414116209\\
2048	0.546026762071961\\
2049	0.499622352514962\\
2050	0.444975884430278\\
2051	0.383085505541658\\
2052	0.314990085011686\\
2053	0.241639387615123\\
2054	0.164369387836359\\
2055	0.0843238411471248\\
2057	-0.0784758681229505\\
2058	-0.158593494168599\\
2059	-0.236065062683338\\
2060	-0.309655324919277\\
2061	-0.378193247179297\\
2062	-0.440499555080351\\
2063	-0.495483762436379\\
2064	-0.542324498642756\\
2065	-0.580221873995924\\
2066	-0.608659175730736\\
2067	-0.627059873105281\\
2068	-0.635149831892249\\
2069	-0.632823060773262\\
2070	-0.620033439169674\\
2071	-0.59704976456851\\
2072	-0.564249433539317\\
2073	-0.522203192306279\\
2074	-0.471580791224824\\
2075	-0.413168235561443\\
2076	-0.347983828710767\\
2077	-0.277098702495095\\
2078	-0.201630486401882\\
2079	-0.122893193682103\\
2080	-0.0421453141807433\\
2082	0.0886736286520318\\
2083	0.151951212268159\\
2084	0.213105251051729\\
2085	0.271728536255523\\
2086	0.327363499308831\\
2087	0.379812989192942\\
2088	0.428761376509101\\
2089	0.474026862618757\\
2090	0.515338556305778\\
2091	0.55259187906222\\
2092	0.58580820844827\\
2093	0.614878138630502\\
2094	0.639778453748931\\
2095	0.660449414114282\\
2096	0.676895936277106\\
2097	0.68905919954932\\
2098	0.697059832915784\\
2099	0.700768633697407\\
2100	0.700379356700523\\
2101	0.695785349569178\\
2102	0.686997878427064\\
2103	0.67413385579539\\
2104	0.657247448640192\\
2105	0.636290047925286\\
2106	0.611482979350512\\
2107	0.582820317994447\\
2108	0.550436661500953\\
2109	0.514476984510111\\
2110	0.47524710832522\\
2111	0.43287968706818\\
2112	0.387712101556644\\
2113	0.340010429657923\\
2114	0.290199159981512\\
2115	0.238681491920033\\
2117	0.13239169358485\\
2119	0.0252298848981809\\
2120	-0.0272830087801594\\
2121	-0.0783547463061041\\
2122	-0.127272962924053\\
2123	-0.173463144444213\\
2124	-0.216413970464146\\
2125	-0.255511743291208\\
2126	-0.290304596927854\\
2127	-0.32028303028801\\
2128	-0.345060463080245\\
2129	-0.364238398939506\\
2130	-0.377659130980192\\
2131	-0.385054973363367\\
2132	-0.386408193772695\\
2133	-0.381705351640903\\
2134	-0.371016848486761\\
2135	-0.354647893877882\\
2136	-0.332878293565955\\
2137	-0.306074827281009\\
2138	-0.274688732792583\\
2139	-0.239306272679642\\
2140	-0.200653320161109\\
2141	-0.159358220896138\\
2142	-0.100185503120883\\
2144	0.0237464755800829\\
2145	0.0855856701014091\\
2146	0.145434174450656\\
2147	0.201829430482121\\
2148	0.253392202503619\\
2149	0.298937473738079\\
2150	0.337309706559608\\
2151	0.367715538773609\\
2152	0.38936973927548\\
2153	0.401729895999324\\
2154	0.404545488799613\\
2155	0.397717843315149\\
2156	0.3814388199512\\
2157	0.356056928341786\\
2158	0.322219823706291\\
2159	0.280695463324264\\
2160	0.232571755720983\\
2161	0.178877412772181\\
2162	0.120955081430566\\
2163	0.0601647856819909\\
2165	-0.0642320268129879\\
2166	-0.124862268194192\\
2167	-0.182533625845281\\
2168	-0.235886507920895\\
2169	-0.283608796394674\\
2170	-0.324544523602981\\
2171	-0.357760367694027\\
2172	-0.382433350592692\\
2173	-0.39801517226806\\
2174	-0.404079475294111\\
2175	-0.400500809159439\\
2176	-0.387456461009151\\
2177	-0.365040638789651\\
2178	-0.333988573603619\\
2179	-0.295023824742202\\
2180	-0.249000537335633\\
2181	-0.197016997333321\\
2182	-0.140302777781926\\
2183	-0.0802797927408392\\
2186	0.105360350823048\\
2187	0.164201266604778\\
2188	0.219139028796235\\
2189	0.268870034478368\\
2190	0.312171104457775\\
2191	0.348149228066632\\
2192	0.375765749403854\\
2193	0.394496308061662\\
2194	0.403761911479251\\
2195	0.403427581478354\\
2196	0.393510630833589\\
2197	0.37426264187161\\
2198	0.346107094165291\\
2199	0.309650064638845\\
2200	0.265864877374952\\
2201	0.215819185271812\\
2202	0.160608156939361\\
2203	0.101488550194063\\
2205	-0.0224672399899646\\
2206	-0.0843793731401092\\
2207	-0.144263674328158\\
2208	-0.200683506892346\\
2209	-0.252338071702525\\
2210	-0.298000976665662\\
2211	-0.336581203192054\\
2212	-0.367129576083244\\
2213	-0.388940618795004\\
2214	-0.401506157124459\\
2215	-0.404509203246562\\
2216	-0.397939132255033\\
2217	-0.381856803883693\\
2218	-0.356670667492381\\
2219	-0.322977587400601\\
2220	-0.281635630698929\\
2221	-0.233604072116123\\
2222	-0.180029940681379\\
2223	-0.122176188415324\\
2224	-0.0614111640829833\\
2226	0.0629403589136928\\
2227	0.123590521248843\\
2228	0.18131105151997\\
2229	0.234703846441789\\
2230	0.28248876033831\\
2231	0.323557041337153\\
2232	0.356854826449762\\
2233	0.381665627461189\\
2234	0.397408309712773\\
2235	0.403644757597249\\
2236	0.400223577012639\\
2237	0.387296542380227\\
2238	0.365049300189185\\
2239	0.33415325776059\\
2240	0.295323647048917\\
2241	0.249425230892939\\
2242	0.197548741490209\\
2243	0.140913527354314\\
2244	0.0809869345848711\\
2247	-0.10466461599799\\
2248	-0.163529941678462\\
2249	-0.218472373856457\\
2250	-0.268256994648709\\
2251	-0.311655251385673\\
2252	-0.347673757674329\\
2253	-0.375396734200876\\
2254	-0.394245343954935\\
2255	-0.403607260748231\\
2256	-0.40342256068061\\
2257	-0.393569456452042\\
2258	-0.374418193574456\\
2259	-0.34637638841923\\
2260	-0.310051403608213\\
2261	-0.266389897579302\\
2262	-0.21637782263906\\
2263	-0.161232426767128\\
2264	-0.102201134449388\\
2266	0.0216719030099739\\
2267	0.0836515827550102\\
2268	0.143638665357685\\
2269	0.200144881736378\\
2270	0.251889558919629\\
2271	0.297636689651881\\
2272	0.336362092687978\\
2273	0.367004482821358\\
2274	0.388964666807169\\
2275	0.401657108940526\\
2276	0.404795975824072\\
2277	0.398327846328357\\
2278	0.3823776942113\\
2279	0.35730103081687\\
2280	0.32375783927273\\
2281	0.282498536880212\\
2282	0.234476927863398\\
2283	0.180888509365559\\
2284	0.123031110656939\\
2285	0.0621642928740584\\
2287	-0.0623944392277735\\
2288	-0.123184003615279\\
2289	-0.181103103664555\\
2290	-0.234635847404661\\
2291	-0.28258494176589\\
2292	-0.323820812494887\\
2293	-0.357309024751885\\
2294	-0.38232089240546\\
2295	-0.398242438450325\\
2296	-0.404678458932722\\
2297	-0.401466626613455\\
2298	-0.388749401665791\\
2299	-0.366730232874033\\
2300	-0.336014389320098\\
2301	-0.297253020335575\\
2302	-0.251457396086153\\
2303	-0.199661937722794\\
2304	-0.143110304161382\\
2305	-0.083161245089741\\
2308	0.102646295474187\\
2309	0.161623315668294\\
2310	0.216724515116766\\
2311	0.266670184050781\\
2312	0.310262515509294\\
2313	0.34650074981846\\
2314	0.37449506477833\\
2315	0.393544449029832\\
2316	0.403168653194371\\
2317	0.403250832313915\\
2318	0.393634756860138\\
2319	0.374712382762027\\
2320	0.346905117497045\\
2321	0.310812620253728\\
2322	0.267288104299041\\
2323	0.217375237809392\\
2324	0.162343014500493\\
2325	0.103487649061663\\
2327	-0.0202380590435496\\
2328	-0.0821788811635997\\
2329	-0.142148436285879\\
2330	-0.198750264392856\\
2331	-0.250557359314371\\
2332	-0.296394687852626\\
2333	-0.335227676944669\\
2334	-0.365980397994008\\
2335	-0.388124136835359\\
2336	-0.4009955056531\\
2337	-0.40431519663116\\
2338	-0.397991852542418\\
2339	-0.382245718836657\\
2340	-0.357319620227827\\
2341	-0.323920868062032\\
2342	-0.282788129638902\\
2343	-0.234904424116394\\
2344	-0.181410241325466\\
2345	-0.123707285853015\\
2346	-0.0629343327805145\\
2348	0.061553523872135\\
2349	0.12233946896913\\
2350	0.180263130798721\\
2351	0.233814659629843\\
2352	0.281853026398494\\
2353	0.323151114230768\\
2354	0.356781700958436\\
2355	0.381885654213875\\
2356	0.397896357828813\\
2357	0.404447700434957\\
2358	0.401357880198248\\
2359	0.388770229414149\\
2360	0.366848188997665\\
2361	0.336229123251542\\
2362	0.297559797779741\\
2363	0.251828301456953\\
2364	0.200114852158094\\
2365	0.143642117436229\\
2366	0.0837223832832024\\
2369	-0.10210483425135\\
2370	-0.16112240652501\\
2371	-0.216338210151662\\
2372	-0.266356665469175\\
2373	-0.310026441816262\\
2374	-0.346399699745234\\
2375	-0.374513624407427\\
2376	-0.393660185589397\\
2377	-0.403432750631964\\
2378	-0.403614077461498\\
2379	-0.394176983384568\\
2380	-0.375372960752884\\
2381	-0.347646049232935\\
2382	-0.311603880951225\\
2383	-0.268194571274762\\
2384	-0.218367312936607\\
2385	-0.163338213244515\\
2386	-0.104511558192826\\
2388	0.0192619554095472\\
2389	0.0812163146142666\\
2390	0.141217742578192\\
2391	0.197939507937463\\
2392	0.249822643853349\\
2393	0.295765722104989\\
2394	0.334712377230971\\
2395	0.365614114601613\\
2396	0.387873663928985\\
2397	0.400859645052606\\
2398	0.404295654493126\\
2399	0.398120449037378\\
2400	0.382473707389636\\
2401	0.353565748056553\\
2402	0.317010604090228\\
2403	0.273724144012249\\
2404	0.224700169055268\\
2405	0.171062531701864\\
2406	0.114121943542614\\
2409	-0.0635215131060249\\
2410	-0.120414960860671\\
2411	-0.174019934068383\\
2412	-0.22297484268347\\
2413	-0.266251854684469\\
2414	-0.302799583648266\\
2415	-0.3318836409303\\
2416	-0.352709731971572\\
2417	-0.364885850056453\\
2418	-0.368162749830844\\
2419	-0.362405156165551\\
2420	-0.347757341087345\\
2421	-0.324550957977408\\
2422	-0.293350252253276\\
2423	-0.254735380571219\\
2424	-0.209716125996238\\
2425	-0.159325225055454\\
2426	-0.104652998169058\\
2427	-0.0470319370888319\\
2430	0.130241773558737\\
2431	0.186218184024256\\
2432	0.238416150804369\\
2433	0.285482666995904\\
2434	0.326357255172297\\
2435	0.36005519737455\\
2436	0.385600762941976\\
2437	0.402328444777595\\
2438	0.409743386504033\\
2439	0.407485322718912\\
2440	0.395527873646643\\
2441	0.373923521682173\\
2442	0.342987471340621\\
2443	0.303166642571341\\
2444	0.255287118146498\\
2445	0.200139855702218\\
2446	0.138699918626116\\
2447	0.0722556191080912\\
2448	0.00200236261071041\\
2450	-0.144161974871622\\
2451	-0.217243733591204\\
2452	-0.288364989851743\\
2453	-0.356002796610028\\
2454	-0.418826260909555\\
2455	-0.475547258333336\\
2456	-0.524950852864094\\
2457	-0.566020152442889\\
2458	-0.597830885331859\\
2459	-0.619709413688724\\
2460	-0.631204704711763\\
2461	-0.631826204213212\\
2462	-0.617351309617334\\
2463	-0.59270333558834\\
2464	-0.558395275526436\\
2465	-0.514893058046255\\
2466	-0.462859917801779\\
2467	-0.40323476743697\\
2468	-0.337007819477549\\
2469	-0.265249457201662\\
2470	-0.189181597971583\\
2471	-0.109937788749448\\
2474	0.133344833618594\\
2475	0.211775262007905\\
2476	0.286725013850628\\
2477	0.356957283377142\\
2478	0.421338227468823\\
2479	0.478816925356568\\
2480	0.528428946676286\\
2481	0.569349043680631\\
2482	0.600987155464736\\
2483	0.622690941642304\\
2484	0.634069255821487\\
2485	0.635078804879413\\
2486	0.625674504125982\\
2487	0.605919505504517\\
2488	0.576288615593512\\
2489	0.537191952966623\\
2490	0.489228292573443\\
2491	0.433258273766114\\
2492	0.370108253451235\\
2493	0.300928706485593\\
2494	0.226824515363205\\
2495	0.148950365472956\\
2497	-0.0128428792972954\\
2498	-0.0940649313733957\\
2499	-0.173744808384527\\
2500	-0.250538751846761\\
2501	-0.323247291828466\\
2502	-0.390634745342595\\
2503	-0.451616922087396\\
2504	-0.505181055831599\\
2505	-0.550392258815009\\
2506	-0.586660309870695\\
2507	-0.613217272206384\\
2508	-0.629733859389944\\
2509	-0.635899209920353\\
2510	-0.63156293302518\\
2511	-0.616894784842771\\
2512	-0.592137787657975\\
2513	-0.557587137129303\\
2514	-0.51387037597442\\
2515	-0.461726376239312\\
2516	-0.402014374854843\\
2517	-0.335694974640774\\
2518	-0.263834799371125\\
2519	-0.187652160305333\\
2520	-0.108404944538051\\
2523	0.134763773855411\\
2524	0.213163706034265\\
2525	0.288120947968309\\
2526	0.358333930424124\\
2527	0.422567361483743\\
2528	0.479897849114423\\
2529	0.529351750824844\\
2530	0.570127573478658\\
2531	0.601542592321039\\
2532	0.623083642295114\\
2533	0.634283083229093\\
2534	0.635121293439624\\
2535	0.625513376452545\\
2536	0.605678158810406\\
2537	0.575900206657934\\
2538	0.536633490539771\\
2539	0.488533771066614\\
2540	0.432449414071471\\
2541	0.369216162297562\\
2542	0.299892965935214\\
2543	0.225661211966781\\
2544	0.147776415540648\\
2546	-0.0138860752977052\\
2547	-0.0951179942426279\\
2548	-0.174714544517883\\
2549	-0.25145037952916\\
2550	-0.324062365268219\\
2551	-0.391279049666991\\
2552	-0.452060841208095\\
2553	-0.505334106027021\\
2554	-0.550409835643222\\
2555	-0.58640882254258\\
2556	-0.612742093795077\\
2557	-0.62898553355717\\
2558	-0.634866352640529\\
2559	-0.630341298492567\\
2560	-0.615474929334141\\
2561	-0.590470065019417\\
2562	-0.555766352298633\\
2563	-0.511927811159239\\
2564	-0.459658925128224\\
2565	-0.399790220940304\\
2566	-0.333435795985224\\
2567	-0.26153859495389\\
2568	-0.185368668698175\\
2569	-0.106130906322051\\
2572	0.136569818363569\\
2573	0.214796528495299\\
2574	0.289439273895823\\
2575	0.359354868704031\\
2576	0.423326208821891\\
2577	0.480419605128645\\
2578	0.529606119830987\\
2579	0.569984490373827\\
2580	0.601037073179668\\
2581	0.622236432262525\\
2582	0.633187606128104\\
2583	0.633801598126411\\
2584	0.623952802956865\\
2585	0.603848815334004\\
2586	0.573793968034352\\
2587	0.534303752568121\\
2588	0.4860557590805\\
2589	0.429807119472571\\
2590	0.366467902094428\\
2591	0.297098544778692\\
2592	0.222865399607144\\
2593	0.144935008542234\\
2595	-0.0166568190065846\\
2596	-0.0977523181695688\\
2597	-0.177237689577851\\
2598	-0.253828883804999\\
2599	-0.326206804486901\\
2600	-0.39319869810879\\
2601	-0.453812781911893\\
2602	-0.506900718204633\\
2603	-0.551607580347536\\
2604	-0.587303013036035\\
2605	-0.613300781817543\\
2606	-0.629247110523465\\
2607	-0.634910060479342\\
2608	-0.629980719386822\\
2609	-0.614736557871765\\
2610	-0.589416743101538\\
2611	-0.554395117040258\\
2612	-0.51021913186878\\
2613	-0.45772465139089\\
2614	-0.397732549873126\\
2615	-0.331096838914164\\
2616	-0.259031889114794\\
2617	-0.182680561157213\\
2618	-0.103381598992655\\
2621	0.139580175064566\\
2622	0.217758525128829\\
2623	0.292369726303605\\
2624	0.362204125812241\\
2625	0.426100737634897\\
2626	0.483011424581946\\
2627	0.531980584677967\\
2628	0.572231457500948\\
2629	0.603110218461097\\
2630	0.624055145516195\\
2631	0.63476726657882\\
2632	0.635120210863988\\
2633	0.625061285165884\\
2634	0.604725554890592\\
2635	0.574418659035473\\
2636	0.534682257977693\\
2637	0.486202542079354\\
2638	0.429675793847309\\
2639	0.366122114891368\\
2640	0.296628442094971\\
2641	0.222164408858134\\
2642	0.144072963984399\\
2644	-0.0178543936071947\\
2645	-0.0990222641958098\\
2646	-0.17858623494385\\
2647	-0.25523350284584\\
2648	-0.327657578870912\\
2649	-0.394685998443947\\
2650	-0.455244944231254\\
2651	-0.508317396395796\\
2652	-0.553027080531137\\
2653	-0.58861136775613\\
2654	-0.614550075084026\\
2655	-0.630386565763729\\
2656	-0.635867807035993\\
2657	-0.63085970219754\\
2658	-0.615504364439403\\
2659	-0.590020912994987\\
2660	-0.55486042574239\\
2661	-0.510603556064325\\
2662	-0.457920059175194\\
2663	-0.397727695233243\\
2664	-0.331003323840832\\
2665	-0.258850263030126\\
2666	-0.182490911449804\\
2667	-0.103058310367942\\
2670	0.13999756526573\\
2671	0.218176108247917\\
2672	0.29278384324607\\
2673	0.362568851606738\\
2674	0.426404378637471\\
2675	0.483247624333671\\
2676	0.532137485097337\\
2677	0.572292660400308\\
2678	0.603014023984088\\
2679	0.623876227559322\\
2680	0.634489723820025\\
2681	0.63463135820939\\
2682	0.62431224117654\\
2683	0.603776729245055\\
2684	0.573360133540064\\
2685	0.53346971431256\\
2686	0.484797249391704\\
2687	0.428157190791808\\
2688	0.364509599498888\\
2689	0.294904610242156\\
2690	0.220395255595577\\
2691	0.142282309552229\\
2693	-0.0196563611270903\\
2694	-0.100875997883122\\
2695	-0.180391238535321\\
2696	-0.256934256149179\\
2697	-0.329290699882222\\
2698	-0.396218993150342\\
2699	-0.456631475570703\\
2700	-0.509555521459788\\
2701	-0.554096206186841\\
2702	-0.589481063515905\\
2703	-0.615226624764091\\
2704	-0.630866462780432\\
2705	-0.636127874334761\\
2706	-0.630986555388517\\
2707	-0.615449779257233\\
2708	-0.589788160319586\\
2709	-0.554465845045797\\
2710	-0.510024279362824\\
2711	-0.457197249449109\\
2712	-0.396838756113539\\
2713	-0.329971915542501\\
2714	-0.257650931855096\\
2715	-0.181143439167954\\
2716	-0.101679250558846\\
2719	0.141385272898788\\
2720	0.219518228835113\\
2721	0.294044667237358\\
2722	0.363764477888253\\
2723	0.427502246318454\\
2724	0.484216085836579\\
2725	0.532957177930712\\
2726	0.572935343035624\\
2727	0.603543057475235\\
2728	0.624172250732954\\
2729	0.634516008196442\\
2730	0.634427048129965\\
2731	0.623932222562416\\
2732	0.603214479941926\\
2733	0.572568745720673\\
2734	0.53243938533933\\
2735	0.483601794429887\\
2736	0.426860238698282\\
2737	0.363053964545543\\
2738	0.293262145527024\\
2739	0.21866135824439\\
2740	0.140535801447641\\
2742	-0.0214472457419106\\
2743	-0.102522764167588\\
2744	-0.181922174310785\\
2745	-0.25832069905573\\
2746	-0.330472683573134\\
2747	-0.397165754494836\\
2748	-0.457416177391224\\
2749	-0.510100652392339\\
2750	-0.55441205658235\\
2751	-0.589602749541882\\
2752	-0.615053413393071\\
2753	-0.630464562989346\\
2754	-0.635459772642207\\
2755	-0.629996644222501\\
2756	-0.614203524970435\\
2757	-0.58827776732187\\
2758	-0.552660795367046\\
2759	-0.507966034372657\\
2760	-0.454919787365725\\
2761	-0.394448750419997\\
2762	-0.32749648304025\\
2763	-0.255125139606207\\
2764	-0.178570170218791\\
2765	-0.0990673407150098\\
2768	0.1440240401148\\
2769	0.222132422545656\\
2770	0.296582853938617\\
2771	0.366125798747817\\
2772	0.429674583852375\\
2773	0.486217969827976\\
2774	0.534748149253574\\
2775	0.574430291076624\\
2776	0.604728187302044\\
2777	0.625153239332121\\
2778	0.635319042254196\\
2779	0.635074067992264\\
2780	0.624314313224659\\
2781	0.603374101465306\\
2782	0.572462733980956\\
2783	0.532177876260448\\
2784	0.483160645240332\\
2785	0.426146563531347\\
2786	0.362132089416718\\
2787	0.292238585273481\\
2788	0.217511120891231\\
2789	0.139163183031997\\
2791	-0.0229359390427817\\
2792	-0.104065825707949\\
2793	-0.183515495946722\\
2794	-0.259930962651651\\
2795	-0.332072109065393\\
2796	-0.398731421061257\\
2797	-0.458859287175528\\
2798	-0.511440332039001\\
2799	-0.555563064915532\\
2800	-0.590576240015707\\
2801	-0.615968300320219\\
2802	-0.631152389279578\\
2803	-0.63595566492495\\
2804	-0.630340016552054\\
2805	-0.614352270877589\\
2806	-0.588341784242857\\
2807	-0.552557536113454\\
2808	-0.507724500574568\\
2809	-0.45452198646808\\
2810	-0.39385998042826\\
2811	-0.326750766112582\\
2812	-0.254281211780381\\
2813	-0.177657360412923\\
2814	-0.0980079793853292\\
2817	0.145132033695063\\
2818	0.223152852085605\\
2819	0.297527701859508\\
2820	0.366974494039368\\
2821	0.430399389287686\\
2822	0.486820682994221\\
2823	0.535210806520809\\
2824	0.574778842010346\\
2825	0.60493954606136\\
2826	0.625133905140046\\
2827	0.635080144168569\\
2828	0.634568173998559\\
2829	0.623654126442489\\
2830	0.602468726524876\\
2831	0.571393185657598\\
2832	0.530947112545618\\
2833	0.481757656615173\\
2834	0.424648850379072\\
2835	0.3605464117054\\
2836	0.290615681773943\\
2837	0.21588072994291\\
2838	0.137560836277316\\
2840	-0.0245495089779979\\
2841	-0.105599582857394\\
2842	-0.184961765993194\\
2843	-0.261227032875468\\
2844	-0.333218665658933\\
2845	-0.399682899051641\\
2846	-0.459575025892718\\
2847	-0.511802543790509\\
2848	-0.555754382753548\\
2849	-0.590515108187901\\
2850	-0.615548190675781\\
2851	-0.630550267038871\\
2852	-0.635094434318489\\
2853	-0.629266324501259\\
2854	-0.613082491203841\\
2855	-0.586804622465479\\
2856	-0.550819906485231\\
2857	-0.505807407947032\\
2858	-0.452524968795842\\
2859	-0.391720679961054\\
2860	-0.324566010663148\\
2861	-0.252043564595624\\
2862	-0.17535564893933\\
2863	-0.0958278831203643\\
2866	0.146969542066472\\
2867	0.224876812123966\\
2868	0.299033481620882\\
2869	0.368313115592628\\
2870	0.431556607829407\\
2871	0.487707071181831\\
2872	0.535838183589021\\
2873	0.575195710010576\\
2874	0.605120714893019\\
2875	0.625019175588932\\
2876	0.634685238533166\\
2877	0.633889107593404\\
2878	0.622738127586672\\
2879	0.6013372921675\\
2880	0.570062368964045\\
2881	0.529415631082429\\
2882	0.480048385655209\\
2883	0.422781061626665\\
2884	0.358650196588769\\
2885	0.288581687706028\\
2886	0.21378698209719\\
2887	0.135466314517998\\
2889	-0.0265108821663489\\
2890	-0.107545844142805\\
2891	-0.186797942267276\\
2892	-0.262981504553863\\
2893	-0.334826901036649\\
2894	-0.401208045084786\\
2895	-0.461006924572757\\
2896	-0.513119954191097\\
2897	-0.556895670094036\\
2898	-0.591517646239936\\
2899	-0.616351551608204\\
2900	-0.631094421582929\\
2901	-0.635455237408223\\
2902	-0.629426691464687\\
2903	-0.61298568228267\\
2904	-0.586467546904714\\
2905	-0.550389187923429\\
2906	-0.505274624797039\\
2907	-0.451770397071869\\
2908	-0.390922304409742\\
2909	-0.323617235203528\\
2910	-0.251010708649119\\
2911	-0.174196088356894\\
2912	-0.0945083961714772\\
2915	0.148434211672338\\
2916	0.226313282591491\\
2917	0.300450794607514\\
2918	0.369637804352351\\
2919	0.432789994791165\\
2920	0.488767345755605\\
2921	0.536781161984891\\
2922	0.575989228442722\\
2923	0.605700278238601\\
2924	0.625520034746387\\
2925	0.635013789856202\\
2926	0.634045267810052\\
2927	0.622711008425085\\
2928	0.601149186600196\\
2929	0.56967387711893\\
2930	0.528786578467134\\
2931	0.479263543360958\\
2932	0.421862090130617\\
2933	0.35755917753113\\
2934	0.287352856001689\\
2935	0.212449650593499\\
2936	0.134022015735809\\
2938	-0.0280234635210945\\
2939	-0.10907421278489\\
2940	-0.188270440980887\\
2941	-0.264443653317812\\
2942	-0.336218683160041\\
2943	-0.402457840100396\\
2944	-0.462095385422799\\
2945	-0.514217061485397\\
2946	-0.557793345666141\\
2947	-0.592204308066357\\
2948	-0.616906551167631\\
2949	-0.631475603707258\\
2950	-0.635719547442022\\
2951	-0.629414934227498\\
2952	-0.612856778671812\\
2953	-0.586213372023394\\
2954	-0.549897721446996\\
2955	-0.504513701585893\\
2956	-0.450870135146033\\
2957	-0.38993062053396\\
2958	-0.322508565428507\\
2959	-0.249794996325818\\
2960	-0.172997163263062\\
2961	-0.0933116433157011\\
2964	0.149532936964533\\
2965	0.227330242004882\\
2966	0.301432955015571\\
2967	0.370527151581427\\
2968	0.433529740087579\\
2969	0.489381331230561\\
2970	0.537220417948902\\
2971	0.576208979334751\\
2972	0.605695764938446\\
2973	0.625243526319082\\
2974	0.634572433146332\\
2975	0.633418604162671\\
2976	0.621844703064198\\
2977	0.600009587970362\\
2978	0.56836760959095\\
2979	0.527407830432367\\
2980	0.477774629161559\\
2981	0.420263882928793\\
2982	0.355840852449091\\
2983	0.285551087539716\\
2984	0.21062206745637\\
2985	0.132212586546302\\
2989	-0.190092519257178\\
2990	-0.266129216011905\\
2991	-0.337744145164834\\
2992	-0.403832872227667\\
2993	-0.463345978078451\\
2994	-0.515208140262075\\
2995	-0.558659122371409\\
2996	-0.59288112786453\\
2997	-0.617320795288833\\
2998	-0.631677700117507\\
2999	-0.635669290346414\\
3000	-0.629155138779424\\
3001	-0.612389835497197\\
3002	-0.585502980360161\\
3003	-0.548991584279065\\
3004	-0.503481658023702\\
3005	-0.449699803127714\\
3006	-0.388548334888583\\
3007	-0.320999299822688\\
3008	-0.248186862888815\\
3009	-0.17126378134617\\
3010	-0.0915130484140718\\
3013	0.151365633297701\\
3014	0.22911573064448\\
3015	0.303044967228288\\
3016	0.372086427526028\\
3017	0.434957066817333\\
3018	0.490744057701704\\
3019	0.53842156812334\\
3020	0.577258258463644\\
3021	0.606553175220142\\
3022	0.625914603084311\\
3023	0.635000476453115\\
3024	0.633602671167409\\
3025	0.621793732651895\\
3026	0.599762552139055\\
3027	0.567855335126751\\
3028	0.526647199686067\\
3029	0.476795753660554\\
3030	0.419071594295019\\
3031	0.354521173985177\\
3032	0.284127439508666\\
3033	0.209006768127892\\
3034	0.13045125424469\\
3036	-0.0316842050488049\\
3037	-0.112727913577601\\
3038	-0.191871260877178\\
3039	-0.26791037771909\\
3040	-0.339517047921618\\
3041	-0.387231241084464\\
3042	-0.429572710312186\\
3043	-0.466396705818624\\
3044	-0.497554079024212\\
3045	-0.522983269368979\\
3046	-0.542775623123362\\
3047	-0.556864148239583\\
3048	-0.565525320623692\\
3049	-0.568935581025016\\
3050	-0.567299494171493\\
3051	-0.560926763298994\\
3052	-0.550089044197648\\
3053	-0.535008157045468\\
3054	-0.515979890880317\\
3055	-0.493480280579661\\
3056	-0.467584147661\\
3057	-0.438788220400056\\
3058	-0.407232751235824\\
3059	-0.373152026572825\\
3060	-0.336916457134521\\
3061	-0.298774985463751\\
3062	-0.258919223139401\\
3063	-0.217660674740728\\
3065	-0.131723181950292\\
3068	0.00173431216398967\\
3070	0.0905368064777576\\
3072	0.176290466019054\\
3073	0.217204755071634\\
3074	0.256362716607782\\
3075	0.293366553763008\\
3076	0.32780292827465\\
3077	0.359307879677999\\
3078	0.387599401684383\\
3079	0.412239816343117\\
3080	0.432853972446537\\
3081	0.449238754083581\\
3082	0.461080054035847\\
3083	0.468115881285485\\
3084	0.470188215147118\\
3085	0.467254639223029\\
3086	0.459250041289124\\
3087	0.44608151818511\\
3088	0.427939069862077\\
3089	0.404962764994252\\
3090	0.37735763549199\\
3091	0.345510963945799\\
3092	0.309775239325063\\
3093	0.270719895282127\\
3094	0.228832594260439\\
3095	0.184676529740955\\
3097	0.0921394092938499\\
3099	-0.00126147690025391\\
3100	-0.0464922280884821\\
3101	-0.0898438534013621\\
3102	-0.149454153983697\\
3103	-0.205529604071216\\
3104	-0.256661090983016\\
3105	-0.301679582970337\\
3106	-0.339564940403761\\
3107	-0.369324777707789\\
3108	-0.390286692968857\\
3109	-0.402017495030577\\
3110	-0.404135176493583\\
3111	-0.396648178102623\\
3112	-0.379662832511258\\
3113	-0.353631528157621\\
3114	-0.319210093616221\\
3115	-0.277236556218668\\
3116	-0.228595311431491\\
3117	-0.174498516269978\\
3118	-0.116209426239038\\
3119	-0.0551381846257755\\
3121	0.0694300682789617\\
3122	0.129967391932496\\
3123	0.187460672601901\\
3124	0.240478542728852\\
3125	0.28778390025218\\
3126	0.328205903396793\\
3127	0.360869827648912\\
3128	0.384997394885886\\
3129	0.399917298956098\\
3130	0.405323660135764\\
3131	0.401137508376451\\
3132	0.387393513803545\\
3133	0.364429899766492\\
3134	0.332810935942689\\
3135	0.293260011377697\\
3136	0.246718271877398\\
3137	0.194424652042471\\
3138	0.137463183186355\\
3139	0.0772347625520524\\
3142	-0.108467014815233\\
3143	-0.167132437472446\\
3144	-0.221789505314064\\
3145	-0.271145646971036\\
3146	-0.314018115289855\\
3147	-0.349499843653575\\
3148	-0.376591576770807\\
3149	-0.394683550805894\\
3150	-0.403377428708609\\
3151	-0.402450565265553\\
3152	-0.391900400776649\\
3153	-0.372027815760703\\
3154	-0.343392178183421\\
3155	-0.30653800115806\\
3156	-0.262335823259036\\
3157	-0.211923727227713\\
3158	-0.156485615851125\\
3159	-0.0972587479236608\\
3161	0.0265622758611244\\
3162	0.0882930484208373\\
3163	0.147942855977362\\
3164	0.204026951234027\\
3165	0.255260228289444\\
3166	0.300384585599659\\
3167	0.33842077153713\\
3168	0.368364077965452\\
3169	0.389497621969895\\
3170	0.401450540681708\\
3171	0.403738938950482\\
3172	0.396401265922577\\
3173	0.379580932793488\\
3174	0.353752910618823\\
3175	0.319461286488149\\
3176	0.277613806325917\\
3177	0.22916883759035\\
3178	0.175206521905238\\
3179	0.117079113540967\\
3180	0.0561001074183878\\
3182	-0.0684226576668152\\
3183	-0.128981823040249\\
3184	-0.186467868218188\\
3185	-0.23946768434098\\
3186	-0.286821327462349\\
3187	-0.327369115104375\\
3188	-0.360094035623661\\
3189	-0.384321682716291\\
3190	-0.399371843227982\\
3191	-0.404922272212389\\
3192	-0.400847189082924\\
3193	-0.387186599496545\\
3194	-0.364313387071888\\
3195	-0.332758289013782\\
3196	-0.293314664673744\\
3197	-0.246796362394889\\
3198	-0.19449277798185\\
3199	-0.137547663366149\\
3200	-0.0772864477194162\\
3201	-0.0252336618864319\\
3203	0.0837836173691358\\
3204	0.13858823434839\\
3205	0.192054815660867\\
3206	0.243063127306414\\
3207	0.290494661379853\\
3208	0.333405023593514\\
3209	0.370819205787484\\
3210	0.401857346435918\\
3211	0.425748031696912\\
3212	0.44196453318682\\
3213	0.449952338663934\\
3214	0.449446469361646\\
3215	0.440282783605653\\
3216	0.422324772416232\\
3217	0.395745112765781\\
3218	0.360874933698597\\
3219	0.318198198952359\\
3220	0.268144111621041\\
3221	0.211546709228969\\
3222	0.149096310796267\\
3223	0.0817140041940547\\
3224	0.0104267257861466\\
3225	-0.0635849459499696\\
3229	-0.366211062285856\\
3230	-0.437855683173439\\
3231	-0.505775361027645\\
3232	-0.568914015708287\\
3233	-0.626406010669143\\
3234	-0.677417150064684\\
3235	-0.72130135478028\\
3236	-0.757417621457989\\
3237	-0.785323011712535\\
3238	-0.804663234607233\\
3239	-0.815157151006133\\
3240	-0.816750966923337\\
3241	-0.809405745164895\\
3242	-0.793315134370459\\
3243	-0.768724137308254\\
3244	-0.736049147045833\\
3245	-0.695660145711372\\
3246	-0.64818442446267\\
3247	-0.594255863955823\\
3248	-0.534563644960599\\
3249	-0.469894117418335\\
3250	-0.401099371142209\\
3251	-0.329014350870239\\
3252	-0.254511829188687\\
3256	0.0495005147936354\\
3257	0.122632932959732\\
3258	0.192907609548911\\
3259	0.259600253959434\\
3260	0.321975894977186\\
3261	0.379321637382873\\
3262	0.441451653711283\\
3263	0.496352528525222\\
3264	0.543120207976244\\
3265	0.580948294074915\\
3266	0.60920276389561\\
3267	0.627486485554527\\
3268	0.635513011516196\\
3269	0.633063102088272\\
3270	0.620161433048452\\
3271	0.597085251097269\\
3272	0.564247167387748\\
3273	0.522062564832595\\
3274	0.471376427065024\\
3275	0.412891146077527\\
3276	0.347649222560904\\
3277	0.276729069903013\\
3278	0.201249232450664\\
3279	0.122451874186936\\
3282	-0.120726941935118\\
3283	-0.19957810222013\\
3284	-0.275103450835559\\
3285	-0.346117381057411\\
3286	-0.411553219743382\\
3287	-0.470151880023423\\
3288	-0.521062011081085\\
3289	-0.563387809816504\\
3290	-0.596431410587229\\
3291	-0.61966040422476\\
3292	-0.632663532243896\\
3293	-0.635328987457797\\
3294	-0.627544251527524\\
3295	-0.609460474149728\\
3296	-0.581300403505793\\
3297	-0.543621344285384\\
3298	-0.49702082550948\\
3299	-0.442237407993616\\
3300	-0.380218049619089\\
3301	-0.311945626069246\\
3302	-0.238540049198946\\
3303	-0.16119705584606\\
3304	-0.081231051058694\\
3306	0.0814863914301895\\
3307	0.161483189986484\\
3308	0.23884762904936\\
3309	0.312268820005556\\
3310	0.380590338135335\\
3311	0.442654521830718\\
3312	0.497478820303968\\
3313	0.544059647787435\\
3314	0.581779502759218\\
3315	0.609923613703359\\
3316	0.628002336037298\\
3317	0.635806582796704\\
3318	0.633217490492825\\
3319	0.620182457114879\\
3320	0.596955181085377\\
3321	0.563904679539519\\
3322	0.521624427642564\\
3323	0.470738171337871\\
3324	0.412117955956091\\
3325	0.346744170003149\\
3326	0.275645253429502\\
3327	0.200041874534691\\
3328	0.121157173625306\\
3331	-0.122083278531591\\
3332	-0.200866436028264\\
3333	-0.276386567956251\\
3334	-0.347334681337543\\
3335	-0.412621180803399\\
3336	-0.471080234887268\\
3337	-0.521832319904206\\
3338	-0.564004825153006\\
3339	-0.596933396716395\\
3340	-0.620006469035161\\
3341	-0.632875662536208\\
3342	-0.635368770823789\\
3343	-0.627443446388952\\
3344	-0.609151084270707\\
3345	-0.580906813775073\\
3346	-0.543118186439642\\
3347	-0.496341688635766\\
3348	-0.441442077302327\\
3349	-0.379266540900517\\
3350	-0.310912996606476\\
3351	-0.237443736346904\\
3352	-0.160041740467022\\
3353	-0.0800335015237579\\
3355	0.0826732088576136\\
3356	0.162624996859904\\
3357	0.239851109168285\\
3358	0.313189897889515\\
3359	0.38130552546636\\
3360	0.443198195496734\\
3361	0.497834139772294\\
3362	0.544252292823785\\
3363	0.58181785726174\\
3364	0.60975928547532\\
3365	0.627709911252623\\
3366	0.635296392017153\\
3367	0.632476839987248\\
3368	0.619213978107837\\
3369	0.595843556765885\\
3370	0.562647057300182\\
3371	0.520154500958597\\
3372	0.469153818673021\\
3373	0.410414195129761\\
3374	0.344986343956407\\
3375	0.273801260933396\\
3376	0.198135464632287\\
3377	0.119317110503289\\
3380	-0.123833941896464\\
3381	-0.202590837132448\\
3382	-0.278000521181639\\
3383	-0.348835513254016\\
3384	-0.413917052047054\\
3385	-0.472260655092668\\
3386	-0.522841967786462\\
3387	-0.56477899266838\\
3388	-0.597419888322293\\
3389	-0.620318077491902\\
3390	-0.633034949036755\\
3391	-0.6353071530134\\
3392	-0.627142576548067\\
3393	-0.608689319316909\\
3394	-0.580233691729518\\
3395	-0.54220149494131\\
3396	-0.495258006332278\\
3397	-0.440219645024627\\
3398	-0.377964901902487\\
3399	-0.309484145859642\\
3400	-0.235884495693881\\
3401	-0.158450255675234\\
3402	-0.0783759542123335\\
3404	0.0842964423040939\\
3405	0.164162386212411\\
3406	0.241393087334927\\
3407	0.314671974575504\\
3408	0.382751775185625\\
3409	0.444546733835068\\
3410	0.499044406728444\\
3411	0.545381800590803\\
3412	0.582741290642844\\
3413	0.610508935354119\\
3414	0.62822097724802\\
3415	0.63558899802274\\
3416	0.632481029270821\\
3417	0.619034697901043\\
3418	0.595415017020969\\
3419	0.561989740401714\\
3420	0.519414043968936\\
3421	0.468259184038743\\
3422	0.409432085454682\\
3423	0.343878074155327\\
3424	0.272624624097261\\
3425	0.196904207593434\\
3426	0.117983008751253\\
3429	-0.125075748539984\\
3430	-0.203818420619882\\
3431	-0.279182739418502\\
3432	-0.349919581284666\\
3433	-0.414929281476361\\
3434	-0.473101366420451\\
3435	-0.523542383585209\\
3436	-0.56531922414888\\
3437	-0.597832514681613\\
3438	-0.620592409669371\\
3439	-0.633092987182863\\
3440	-0.635148630741242\\
3441	-0.626816358138058\\
3442	-0.608103456217123\\
3443	-0.579424100171764\\
3444	-0.541231871826312\\
3445	-0.49416070103598\\
3446	-0.438890194507167\\
3447	-0.376434002122096\\
3448	-0.307843153973408\\
3449	-0.234105435891706\\
3450	-0.156504848639543\\
3451	-0.0763708435106309\\
3453	0.0863345142565777\\
3454	0.166188025869815\\
3455	0.243316034806867\\
3456	0.316490852048446\\
3457	0.384454987918616\\
3458	0.446102614531355\\
3459	0.500450537640063\\
3460	0.546560132951981\\
3461	0.583703631722074\\
3462	0.61124653744173\\
3463	0.628776698089041\\
3464	0.635925359792054\\
3465	0.632654860934963\\
3466	0.618999357790472\\
3467	0.595186591938273\\
3468	0.561575527570767\\
3469	0.518748039963157\\
3470	0.46741856287781\\
3471	0.408355389172812\\
3472	0.342627159158383\\
3473	0.271261203163249\\
3474	0.195467444180849\\
3475	0.116444682672409\\
3478	-0.126565397694321\\
3479	-0.205163679057932\\
3480	-0.280412989796332\\
3481	-0.351077582375638\\
3482	-0.415966857607145\\
3483	-0.474061016564065\\
3484	-0.524272719635974\\
3485	-0.565949133937465\\
3486	-0.598271322660366\\
3487	-0.620809347190061\\
3488	-0.633123253568556\\
3489	-0.635031603639618\\
3490	-0.626526404665583\\
3491	-0.607661052177264\\
3492	-0.578782554958252\\
3493	-0.540401251651474\\
3494	-0.493148395033131\\
3495	-0.437806204321078\\
3496	-0.375221710472943\\
3497	-0.306516751703384\\
3498	-0.232755091545187\\
3499	-0.155119191869289\\
3500	-0.0749871458656344\\
3502	0.087585334549658\\
3503	0.167403197999192\\
3504	0.244469167508669\\
3505	0.317506958518607\\
3506	0.385381347112798\\
3507	0.446826656484518\\
3508	0.500986090237802\\
3509	0.546956327675616\\
3510	0.583825807590074\\
3511	0.611132743325015\\
3512	0.628451868348293\\
3513	0.6354291460284\\
3514	0.631992768069267\\
3515	0.618174095941413\\
3516	0.594218620935862\\
3517	0.560399570730624\\
3518	0.517376353892814\\
3519	0.465899042134424\\
3520	0.406733244415591\\
3521	0.340890451825544\\
3522	0.269454720423255\\
3523	0.193642392083348\\
3524	0.114563747770717\\
3527	-0.128465863709607\\
3528	-0.20706531010228\\
3529	-0.282202873295319\\
3530	-0.352782486778324\\
3531	-0.417483926190926\\
3532	-0.4752937722551\\
3533	-0.525386542803062\\
3534	-0.566833462814884\\
3535	-0.598952816879319\\
3536	-0.621219734079204\\
3537	-0.633327699149504\\
3538	-0.634997383882364\\
3539	-0.626231032194028\\
3540	-0.607200018567255\\
3541	-0.578155812003843\\
3542	-0.539590301538283\\
3543	-0.492136733397729\\
3544	-0.436577883269365\\
3545	-0.373910605071615\\
3546	-0.305090202850806\\
3547	-0.231241192313973\\
3548	-0.153569070721915\\
3549	-0.0733859965057491\\
3551	0.0892674220490335\\
3552	0.168994033524086\\
3553	0.245936083717424\\
3554	0.318829818975246\\
3555	0.386526561341725\\
3556	0.447879509996255\\
3557	0.501830347543546\\
3558	0.547537270121666\\
3559	0.584234347059464\\
3560	0.611392202008119\\
3561	0.628481336214918\\
3562	0.635260434507472\\
3563	0.631593883448204\\
3564	0.617450470383574\\
3565	0.59323165628166\\
3566	0.559297166799297\\
3567	0.516086489655208\\
3568	0.464467645439981\\
3569	0.405183563326318\\
3570	0.339193669520228\\
3571	0.267667996978616\\
3572	0.19173166377368\\
3573	0.1126503736632\\
3576	-0.130489510363532\\
3577	-0.209024561534989\\
3578	-0.284097460014891\\
3579	-0.354477894328284\\
3580	-0.419067974250993\\
3581	-0.476728048502082\\
3582	-0.526602236816416\\
3583	-0.56781884087377\\
3584	-0.599710498689547\\
3585	-0.621737484646019\\
3586	-0.633544831629479\\
3587	-0.634982082340684\\
3588	-0.62598113709646\\
3589	-0.606675101967539\\
3590	-0.57741926834251\\
3591	-0.538604833086993\\
3592	-0.49106945881249\\
3593	-0.435408776703298\\
3594	-0.372541719646961\\
3595	-0.303571558372823\\
3596	-0.229641044440086\\
3597	-0.151911324647699\\
3599	0.00971336512429843\\
3600	0.0909289570895453\\
3601	0.170687497616655\\
3602	0.247665110102844\\
3603	0.320587064289612\\
3604	0.38823341943089\\
3605	0.449423516861316\\
3606	0.503298461933809\\
3607	0.548898084224675\\
3608	0.585464852090809\\
3609	0.612454010491092\\
3610	0.629295583904877\\
3611	0.635829287986326\\
3612	0.631917875341514\\
3613	0.617682520029575\\
3614	0.59331145950182\\
3615	0.559208625362771\\
3616	0.515881635670212\\
3617	0.464080813354485\\
3618	0.404712134496094\\
3619	0.338680545453826\\
3620	0.267088807472192\\
3621	0.191068344172436\\
3622	0.11193601041532\\
3625	-0.131046861368759\\
3626	-0.209504599324646\\
3627	-0.284552831271412\\
3628	-0.354859743960333\\
3629	-0.419322206847937\\
3630	-0.476930314386664\\
3631	-0.526657007723315\\
3632	-0.567726079913427\\
3633	-0.599463357832519\\
3634	-0.621361972127488\\
3635	-0.633027274451706\\
3636	-0.634275013429487\\
3637	-0.625071471096362\\
3638	-0.605587442413253\\
3639	-0.576177897618436\\
3640	-0.537262506935804\\
3641	-0.489522953316737\\
3642	-0.433778018995326\\
3643	-0.370849964832814\\
3644	-0.301814381541135\\
3645	-0.227833354765608\\
3646	-0.150075405505959\\
3647	-0.0698690807967068\\
3649	0.0927754930698939\\
3650	0.17244616547805\\
3651	0.249217843564111\\
3652	0.321982264815688\\
3653	0.389484396082025\\
3654	0.450538033758221\\
3655	0.50421153081561\\
3656	0.549595279809182\\
3657	0.585987790217132\\
3658	0.612696973534639\\
3659	0.629349709945927\\
3660	0.635636733477895\\
3661	0.631539678859099\\
3662	0.617004366824403\\
3663	0.592279922012494\\
3664	0.557846089609939\\
3665	0.514263513062815\\
3666	0.462270797972451\\
3667	0.402599125795405\\
3668	0.336381218628958\\
3669	0.264614219496707\\
3670	0.188481437462087\\
3671	0.109272828701251\\
3674	-0.133739986757064\\
3675	-0.212136210361678\\
3676	-0.287067819411277\\
3677	-0.357273864011859\\
3678	-0.421627019417429\\
3679	-0.479099836149089\\
3680	-0.528655339259785\\
3681	-0.569520076860954\\
3682	-0.601053309606868\\
3683	-0.622735343682052\\
3684	-0.634170307360819\\
3685	-0.635168236924528\\
3686	-0.625745890276903\\
3687	-0.606025035428047\\
3688	-0.576308492505632\\
3689	-0.537173770725531\\
3690	-0.48919619706021\\
3691	-0.433153859348295\\
3692	-0.370015074128787\\
3693	-0.300794756217783\\
3694	-0.226577650558283\\
3695	-0.148615050605258\\
3697	0.0131715974471263\\
3698	0.0943940594879678\\
3699	0.174080934890753\\
3700	0.250936740589623\\
3701	0.32359196861762\\
3702	0.390959205679337\\
3703	0.451934835876273\\
3704	0.50553615617855\\
3705	0.550761176718424\\
3706	0.586939065924525\\
3707	0.613493305918837\\
3708	0.629947199273374\\
3709	0.636016412323897\\
3710	0.631647352485288\\
3711	0.616947765788609\\
3712	0.592119210349665\\
3713	0.557508604103987\\
3714	0.513736500834966\\
3715	0.461501421031244\\
3716	0.40172174448162\\
3717	0.335319349640486\\
3718	0.263438141958886\\
3719	0.187208371810812\\
3720	0.107962758453596\\
3723	-0.135037415060651\\
3724	-0.2133205990026\\
3725	-0.288168185561062\\
3726	-0.358262594274947\\
3727	-0.4224363163612\\
3728	-0.479692396246719\\
3729	-0.529069451635678\\
3730	-0.569746532748013\\
3731	-0.601059585327675\\
3732	-0.622454502846267\\
3733	-0.633651618791191\\
3734	-0.63444775666494\\
3735	-0.624876850959026\\
3736	-0.604961993129109\\
3737	-0.575121140008832\\
3738	-0.535836455215303\\
3739	-0.487727790432018\\
3740	-0.431528892906044\\
3741	-0.3682853951982\\
3742	-0.298983858799147\\
3743	-0.224775734775449\\
3744	-0.146834352572569\\
3746	0.014856919070553\\
3747	0.0960432159727134\\
3748	0.175623299324343\\
3749	0.252288150620188\\
3750	0.324853630170765\\
3751	0.39209117681412\\
3752	0.452866852513125\\
3753	0.506253116657263\\
3754	0.551260003976495\\
3755	0.587225061477966\\
3756	0.613488942477488\\
3757	0.629692409798281\\
3758	0.635601916003452\\
3759	0.630996157786285\\
3760	0.616052734837922\\
3761	0.590999767951871\\
3762	0.556234980242152\\
3763	0.512352957329313\\
3764	0.460016822127272\\
3765	0.40007210874046\\
3766	0.333607122332069\\
3767	0.261668625905713\\
3768	0.185422537312661\\
3769	0.106160609017934\\
3772	-0.136882511112162\\
3773	-0.215175580141022\\
3774	-0.2899092693483\\
3775	-0.359924496894109\\
3776	-0.42393476569714\\
3777	-0.481044180543449\\
3778	-0.530247465487719\\
3779	-0.570788622183954\\
3780	-0.601846381537598\\
3781	-0.62311226795191\\
3782	-0.634076081821604\\
3783	-0.634615209710773\\
3784	-0.62479605228873\\
3785	-0.604664023997884\\
3786	-0.574547565042849\\
3787	-0.535080213397578\\
3788	-0.486851990699051\\
3789	-0.430567877893736\\
3790	-0.367175932500686\\
3791	-0.29777743411023\\
3792	-0.22353173126703\\
3793	-0.145603558804396\\
3795	0.0160991257630485\\
3796	0.0972691173524254\\
3797	0.176835624785781\\
3798	0.253471912727946\\
3799	0.325991716956196\\
3800	0.39309891257335\\
3801	0.45367094273297\\
3802	0.506899072236592\\
3803	0.551727813755861\\
3804	0.587527531763953\\
3805	0.613684851498874\\
3806	0.629762931646383\\
3807	0.635459743822139\\
3808	0.630717395187276\\
3809	0.615617297853078\\
3810	0.590402036137675\\
3811	0.555460712659624\\
3812	0.511393463494187\\
3813	0.458950523083786\\
3814	0.398895590983557\\
3815	0.332326558728255\\
3816	0.260236012762562\\
3817	0.183932624078352\\
3818	0.104647744700287\\
3821	-0.138192812436955\\
3822	-0.216392408402953\\
3823	-0.291026025478459\\
3824	-0.36091178800234\\
3825	-0.42477560736188\\
3826	-0.481736951262519\\
3827	-0.530808743682883\\
3828	-0.571115248540536\\
3829	-0.602068125575897\\
3830	-0.623101757715176\\
3831	-0.633890960571989\\
3832	-0.634265888727896\\
3833	-0.624227686777431\\
3834	-0.603834819764415\\
3835	-0.573561693680858\\
3836	-0.533817418271155\\
3837	-0.485351345637355\\
3838	-0.428892815282325\\
3839	-0.365381730143781\\
3840	-0.295843209566101\\
3841	-0.221512189876648\\
3842	-0.143467907391823\\
3844	0.0182766137691033\\
3845	0.0994159611987016\\
3846	0.178888608397301\\
3847	0.25544573533989\\
3848	0.327843165498507\\
3849	0.394852109790918\\
3850	0.455376230272122\\
3851	0.508406396223563\\
3852	0.553084487497927\\
3853	0.588686297006006\\
3854	0.614659110065531\\
3855	0.630544229881707\\
3856	0.636025999536287\\
3857	0.631107724411777\\
3858	0.615725467083848\\
3859	0.590248640348364\\
3860	0.555104274712448\\
3861	0.510861294212191\\
3862	0.458260459351095\\
3863	0.398024376692319\\
3864	0.331321877231403\\
3865	0.259178143485315\\
3866	0.18272574481216\\
3867	0.1032645263208\\
3870	-0.139792951266827\\
3871	-0.217970398375655\\
3872	-0.292554630386348\\
3873	-0.362394312464403\\
3874	-0.426269271762521\\
3875	-0.483160073559702\\
3876	-0.532049263461886\\
3877	-0.572216598517116\\
3878	-0.60301148701592\\
3879	-0.623921107868227\\
3880	-0.634598816228845\\
3881	-0.634843266039752\\
3882	-0.624562181596957\\
3883	-0.604116331693149\\
3884	-0.573707500296678\\
3885	-0.533836589980638\\
3886	-0.485297677388189\\
3887	-0.428714868143743\\
3888	-0.365083880256861\\
3889	-0.295455310997568\\
3890	-0.220997018133403\\
3891	-0.142918741678386\\
3893	0.0190981128516796\\
3894	0.100255553102215\\
3895	0.179700801978925\\
3896	0.256246014213957\\
3897	0.328576411740869\\
3898	0.395509483088063\\
3899	0.455890070007172\\
3900	0.508865385457284\\
3901	0.553406811255627\\
3902	0.588926055218508\\
3903	0.614700117153461\\
3904	0.6304149051889\\
3905	0.635729338585406\\
3906	0.630668644882462\\
3907	0.615176499153222\\
3908	0.589544294004554\\
3909	0.554261648632291\\
3910	0.5099126077248\\
3911	0.457151541964777\\
3912	0.396872711114611\\
3913	0.330087007146176\\
3914	0.257948365167067\\
3915	0.181520303249727\\
3916	0.102071653422172\\
3919	-0.140925171599974\\
3920	-0.219041254997137\\
3921	-0.293590887139544\\
3922	-0.363289890786291\\
3923	-0.427085363469359\\
3924	-0.483755367591584\\
3925	-0.532524527190617\\
3926	-0.572560891046123\\
3927	-0.60313981830086\\
3928	-0.623798162289404\\
3929	-0.634235931308467\\
3930	-0.634213855693815\\
3931	-0.623778685508114\\
3932	-0.603115164695282\\
3933	-0.572557037101433\\
3934	-0.532568665230883\\
3935	-0.483845485122856\\
3936	-0.427153639303469\\
3937	-0.363402698383652\\
3938	-0.293718654105305\\
3939	-0.219197964526757\\
3940	-0.141078429881418\\
3942	0.0208142614283133\\
3943	0.101847422026822\\
3944	0.181268585091857\\
3945	0.257729984426533\\
3946	0.329857803793402\\
3947	0.396623013483349\\
3948	0.456833149543399\\
3949	0.509607880970634\\
3950	0.553949001322962\\
3951	0.589247747017453\\
3952	0.61479100153565\\
3953	0.630167844169137\\
3954	0.6352300141657\\
3955	0.629885843391548\\
3956	0.614200229959351\\
3957	0.588350108382656\\
3958	0.552824585774488\\
3959	0.508220270415677\\
3960	0.455325649560564\\
3961	0.394861711735757\\
3962	0.327995884039865\\
3963	0.255652623843162\\
3964	0.179152438631263\\
3965	0.099610436983312\\
3968	-0.143298150936062\\
3969	-0.221294834734181\\
3970	-0.295724989697646\\
3971	-0.365234081875315\\
3972	-0.42871577869937\\
3973	-0.485203412495139\\
3974	-0.533753518492631\\
3975	-0.573586442926626\\
3976	-0.603932018297201\\
3977	-0.624334141712552\\
3978	-0.634440354906474\\
3979	-0.634202422958424\\
3980	-0.623533556266011\\
3981	-0.60255885415927\\
3982	-0.571767180857933\\
3983	-0.531517008160336\\
3984	-0.482482242664901\\
3985	-0.425624503106064\\
3986	-0.361736091228977\\
3987	-0.291901353321009\\
3988	-0.217232123171925\\
3989	-0.139019544748862\\
3991	0.0229283960184148\\
3992	0.103966473600849\\
3993	0.183290306528761\\
3994	0.259604449246126\\
3995	0.331677349206984\\
3996	0.398287015872484\\
3997	0.458379682013401\\
3998	0.510867050037177\\
3999	0.55501428470734\\
4000	0.590031574483874\\
};
\addlegendentry{1}

\addplot [color=mycolor2]
  table[row sep=crcr]{%
1	0.0037331097405513\\
3	0.0142427619216505\\
7	0.0381347271045342\\
8	0.0423112335402038\\
9	0.0450159428287407\\
10	0.0457074914238547\\
11	0.043993347050673\\
12	0.0395898552928884\\
13	0.0320865960025003\\
14	0.0213516150829491\\
15	0.0071884278017933\\
16	-0.0104635209931985\\
17	-0.0316027244630277\\
18	-0.0561464217275898\\
19	-0.0838477803672504\\
20	-0.114429365314663\\
21	-0.147480861340227\\
22	-0.182532521439498\\
24	-0.256414529294943\\
26	-0.330863627591498\\
27	-0.366472803108991\\
28	-0.399955722209597\\
29	-0.430543546280205\\
30	-0.457581967224996\\
31	-0.0854242369550775\\
32	0.0347031395781414\\
33	0.143936139799735\\
34	0.249627704551585\\
35	0.335196330195231\\
36	0.41174907423283\\
37	0.473441832381468\\
38	0.513385485182425\\
39	0.550725129406601\\
40	0.558341064181604\\
41	0.548778117834445\\
42	0.529669161066977\\
43	0.483282395452079\\
44	0.432942969570377\\
45	0.366060194824513\\
46	0.290047898050943\\
47	0.204360803959389\\
51	-0.171996153593227\\
52	-0.260817452463016\\
53	-0.34390079686591\\
54	-0.422427499259811\\
55	-0.483530127380618\\
56	-0.536997327606969\\
57	-0.573353529848646\\
58	-0.59255970005006\\
59	-0.595218765312893\\
60	-0.590717048515216\\
61	-0.563025181457306\\
62	-0.528132548893154\\
63	-0.476387133452135\\
64	-0.421143570026288\\
65	-0.348507869274727\\
66	-0.268861534787447\\
67	-0.182685741981459\\
68	-0.0924125838055261\\
70	0.0840607340105635\\
71	0.175260773609352\\
72	0.259914275829942\\
73	0.346578770628639\\
75	0.475178451316424\\
76	0.522967032438373\\
77	0.562779028035493\\
78	0.583145703819355\\
79	0.591858807321842\\
80	0.587786295291153\\
81	0.57395978976183\\
82	0.540739807334376\\
83	0.496025894247396\\
84	0.443096343963589\\
85	0.373634110911553\\
86	0.297625252146645\\
87	0.217828376677062\\
88	0.12452325603499\\
89	0.0368633923617381\\
90	-0.0546865793808138\\
91	-0.147632230775798\\
92	-0.233625887075959\\
93	-0.318164729038472\\
94	-0.392004904158057\\
95	-0.453811259055328\\
96	-0.506236607307983\\
97	-0.543848884929503\\
98	-0.579796858869031\\
99	-0.592807005690247\\
100	-0.594140734314351\\
101	-0.584855863855864\\
102	-0.554270571337838\\
103	-0.51692489642619\\
104	-0.4584875065284\\
105	-0.398778768821103\\
106	-0.323597591000635\\
107	-0.244285879139625\\
109	-0.0676258504822727\\
110	0.0266468618178806\\
111	0.118488956600231\\
112	0.209108510394344\\
113	0.288413455476984\\
114	0.360765045317294\\
115	0.431617128407197\\
116	0.494141851419499\\
117	0.534688129754159\\
118	0.572367397025573\\
119	0.590109772315827\\
120	0.598575632818665\\
121	0.582358121424022\\
122	0.56369095473292\\
124	0.486113010135796\\
125	0.421499926771958\\
126	0.352760392958317\\
128	0.185521342270022\\
130	0.00832247174366785\\
131	-0.0863556151316516\\
132	-0.173717547850629\\
133	-0.265273614604212\\
134	-0.340635892778664\\
135	-0.412805959932939\\
136	-0.475782642874492\\
137	-0.521107557188316\\
138	-0.559153592179882\\
139	-0.583956590171056\\
140	-0.597353173146985\\
141	-0.588048480955422\\
142	-0.571892448752351\\
143	-0.541802233536146\\
144	-0.496649559504021\\
145	-0.440989877441098\\
146	-0.3775246967366\\
147	-0.298563400087914\\
148	-0.216931216614739\\
149	-0.129317609434111\\
150	-0.0387232668676916\\
151	0.0545107316297617\\
152	0.141819071362079\\
153	0.235556806249406\\
154	0.316107139272845\\
155	0.388908391471432\\
156	0.453650779249983\\
157	0.505929366550845\\
158	0.551965388431654\\
159	0.578292530452018\\
160	0.592079381203348\\
161	0.602217159038446\\
162	0.587300260109714\\
163	0.558430043542558\\
164	0.51522146538764\\
165	0.455390804023864\\
166	0.383088453103028\\
167	0.308391944130562\\
168	0.214651756387866\\
169	0.113474828820017\\
170	0.00816548081138535\\
171	-0.0935763847237467\\
172	-0.192602416449517\\
174	-0.378262076655574\\
175	-0.457523578359087\\
176	-0.519462996667698\\
177	-0.564085138963947\\
178	-0.595437857289653\\
179	-0.60812161734566\\
180	-0.602537183797722\\
181	-0.574663723014964\\
182	-0.528524289038614\\
183	-0.457812339022439\\
184	-0.375894419265933\\
185	-0.2669790429768\\
186	-0.148923202001697\\
187	-0.0128209283038814\\
188	0.135562990429662\\
190	0.447206595397347\\
191	1.09551415370152\\
192	0.984686816462272\\
193	0.86649316261628\\
194	0.744752226341916\\
196	0.495870586619276\\
197	0.373627771013616\\
198	0.254237379238475\\
199	0.142044670163159\\
200	0.0369059529534752\\
201	-0.0634421613635823\\
202	-0.152996620313843\\
203	-0.22817819580132\\
204	-0.294445990327858\\
205	-0.350100535684305\\
206	-0.385991857070167\\
207	-0.415118983617504\\
208	-0.430814423569245\\
209	-0.433308159838816\\
210	-0.425907334029034\\
211	-0.410227518504598\\
212	-0.380642377363074\\
213	-0.344445051704952\\
214	-0.302235843796097\\
215	-0.24849809064699\\
216	-0.201249685736002\\
220	0.0260023077344158\\
221	0.0791058677109504\\
222	0.12800444471668\\
224	0.205134855444612\\
225	0.247793345876744\\
226	0.279667686846551\\
227	0.304555435740895\\
229	0.344941290356473\\
230	0.357677928417161\\
231	0.364749341145398\\
232	0.363796345090577\\
233	0.359607970432535\\
234	0.347839238436791\\
236	0.304692104844435\\
237	0.277325214822667\\
238	0.245467026993083\\
239	0.209418439165347\\
240	0.170286080373444\\
241	0.120793881662848\\
242	0.0818030758860004\\
243	0.0395037750763549\\
244	-0.0123540404206324\\
245	-0.0543932561013207\\
246	-0.101320084675535\\
247	-0.146089350612328\\
248	-0.187123226706717\\
249	-0.229466451963162\\
250	-0.260879396075779\\
251	-0.287512145265737\\
252	-0.316509321574358\\
253	-0.337102633731774\\
254	-0.353697731070042\\
255	-0.361190333850118\\
256	-0.363791445187417\\
257	-0.360531359571269\\
258	-0.353523107698038\\
259	-0.338614716650682\\
260	-0.313447537767843\\
261	-0.297670218702933\\
262	-0.257466272359579\\
263	-0.227245928901993\\
264	-0.187554077312598\\
265	-0.152213011251206\\
267	-0.0597451939506755\\
268	-0.0116908448358117\\
269	0.0303244325341439\\
270	0.0811821336706089\\
271	0.122235320297932\\
272	0.167859936970672\\
273	0.209124266276376\\
274	0.244954717083147\\
275	0.27921280280907\\
276	0.307049795459989\\
278	0.346034071854319\\
279	0.355276819345363\\
281	0.365671149485934\\
282	0.357713661242087\\
283	0.345846523825003\\
284	0.330583950026721\\
285	0.30705335947232\\
286	0.278646032154484\\
287	0.248248707498078\\
288	0.211022235937889\\
289	0.169591142951504\\
291	0.0801275869671372\\
292	0.0382329978210691\\
293	-0.0147160771125527\\
294	-0.0576712550964658\\
295	-0.107521910455034\\
296	-0.144673262433571\\
297	-0.186423934727372\\
298	-0.226636664971011\\
300	-0.297029637526521\\
301	-0.315675546240982\\
302	-0.342158584734079\\
303	-0.357139744563028\\
304	-0.360741240384868\\
305	-0.366066826743463\\
306	-0.360067423111104\\
307	-0.350597909231965\\
308	-0.3354589490732\\
309	-0.318663012857542\\
310	-0.290558217733633\\
311	-0.260996150299889\\
312	-0.224861174868238\\
313	-0.187178367562865\\
315	-0.103453513811473\\
316	-0.0579475211411591\\
317	-0.00885607671352773\\
318	0.0308105534863898\\
319	0.0805562085042766\\
320	0.123610529814869\\
321	0.168326212482953\\
322	0.210595718418062\\
323	0.247475496255447\\
325	0.305033179806742\\
326	0.329308151097848\\
327	0.345782496103766\\
328	0.358935494630259\\
329	0.365065689466064\\
330	0.362045078587926\\
331	0.357761161946655\\
332	0.341473144884276\\
333	0.331225955796072\\
334	0.304313508393079\\
336	0.244625576424369\\
337	0.205383775589326\\
338	0.168516079684196\\
339	0.124518194493248\\
340	0.0858011052509937\\
341	0.0372791775726\\
342	-0.0175832442400861\\
343	-0.0575741697398371\\
344	-0.107292741245601\\
346	-0.189099210794211\\
348	-0.26293775092654\\
349	-0.289017433773552\\
350	-0.319031995136356\\
351	-0.337387518574815\\
352	-0.351301152083124\\
353	-0.361422368763215\\
354	-0.362502029035113\\
355	-0.360605443358963\\
356	-0.354034981714904\\
357	-0.341477809741718\\
358	-0.312931155857768\\
359	-0.292806790913346\\
360	-0.260832058260803\\
361	-0.225094947036268\\
362	-0.187719409909732\\
363	-0.148455824795292\\
364	-0.103623433274606\\
365	-0.0572716465153462\\
366	-0.00715879174140355\\
367	0.0342893526235457\\
368	0.084072338756414\\
369	0.127750203943378\\
370	0.169172519672429\\
372	0.245296571128165\\
373	0.278192549159485\\
374	0.302913577562322\\
375	0.329447111583249\\
376	0.344531713908964\\
377	0.357818805017814\\
378	0.364558807943013\\
379	0.361622707408515\\
380	0.356841923864067\\
381	0.346250305766716\\
382	0.323691247016995\\
383	0.307018581033844\\
384	0.282054136252555\\
385	0.246131969309317\\
387	0.16822215575985\\
388	0.122200748722662\\
389	0.0838491984445682\\
390	0.0356178282786459\\
391	-0.0114321815171934\\
392	-0.0616406867616206\\
393	-0.104488112689523\\
394	-0.150201618321717\\
395	-0.190954569157839\\
396	-0.22971008908462\\
397	-0.265186942765922\\
398	-0.294997936637628\\
400	-0.342087928269848\\
402	-0.361674701627635\\
403	-0.362614550845137\\
404	-0.360329424354404\\
406	-0.34009893839584\\
407	-0.318553224156858\\
408	-0.291636957582341\\
409	-0.259970465492643\\
410	-0.222968662243147\\
411	-0.187677711880951\\
412	-0.142669549611128\\
413	-0.100481915366345\\
414	-0.0533453352268225\\
417	0.0798611802833875\\
418	0.128174503758601\\
419	0.172293161645484\\
420	0.209552411160985\\
421	0.245339799060275\\
422	0.278107114037084\\
423	0.304072128401913\\
424	0.328068161243209\\
425	0.345981377264252\\
426	0.358028617667514\\
427	0.363482570908218\\
428	0.364708625023468\\
430	0.348966255094183\\
431	0.327843507043781\\
432	0.307956040396675\\
433	0.274152172039066\\
434	0.244199637669681\\
436	0.167551481391456\\
437	0.124740306827789\\
438	0.0736598476419204\\
439	0.0294214447412742\\
440	-0.0175309900882894\\
442	-0.102915259116799\\
443	-0.150653556767793\\
446	-0.265951959760059\\
448	-0.320389816574334\\
449	-0.335250946146061\\
450	-0.353115441196223\\
451	-0.357930123933784\\
452	-0.365156281818145\\
453	-0.359530275220095\\
454	-0.351661406244602\\
455	-0.340956349068165\\
456	-0.31802563643123\\
458	-0.259554405480685\\
459	-0.227117461458874\\
460	-0.187490290316873\\
461	-0.14399603899983\\
462	-0.102814371161912\\
463	-0.0528357078846966\\
464	-0.00992386841335247\\
465	0.0422009715794047\\
466	0.0818008741080121\\
467	0.129296865558445\\
468	0.168012035650463\\
469	0.20954351711498\\
470	0.246095709429937\\
471	0.28049437794698\\
472	0.309302791548362\\
473	0.330396951226703\\
474	0.34601873793963\\
475	0.360095248681773\\
476	0.365581571412804\\
477	0.36364649084453\\
478	0.358227163491392\\
479	0.345610905271769\\
480	0.324500061158687\\
481	0.300625087824301\\
483	0.227806656871053\\
484	0.189131103855743\\
488	0.011809397873094\\
489	-0.035927637063196\\
491	-0.120258092603308\\
492	-0.157006129897127\\
493	-0.190527548160844\\
494	-0.222600735758988\\
495	-0.245428102558435\\
496	-0.264203338967036\\
497	-0.279218053168279\\
498	-0.283574540524114\\
499	-0.290793089186991\\
500	-0.287031007012956\\
501	-0.279342384726988\\
503	-0.236482600756062\\
504	-0.212044401471303\\
506	-0.144262255093508\\
507	-0.104383191328907\\
508	-0.0617827449846118\\
510	0.0315632921469842\\
511	-0.390298419222745\\
512	-0.364106440211799\\
513	-0.317759999973987\\
514	-0.257025855330539\\
515	-0.188449590185883\\
516	-0.111184156401578\\
517	-0.0298116669737283\\
518	0.0535207028142395\\
519	0.138874310684514\\
520	0.220025813582197\\
521	0.297798186853015\\
522	0.370370208846452\\
523	0.430719541407143\\
524	0.476247376842366\\
525	0.513130856146745\\
526	0.546493032731178\\
527	0.552602799101805\\
528	0.554794667722945\\
529	0.538216177852973\\
530	0.512625941491024\\
531	0.46361261198399\\
532	0.40645485998175\\
533	0.340922200037312\\
534	0.264395825813608\\
535	0.180037745893515\\
536	0.0892353256022034\\
537	-0.00455750314813486\\
538	-0.094389629318357\\
539	-0.189817369249795\\
540	-0.277049494782659\\
541	-0.353630219827664\\
542	-0.426705997796944\\
543	-0.485092457913652\\
544	-0.530162557820859\\
545	-0.567570161194453\\
546	-0.585861434952676\\
547	-0.596165651722458\\
548	-0.586231832342037\\
549	-0.566204597983415\\
550	-0.533725082018009\\
551	-0.488084257103765\\
552	-0.429795585315787\\
553	-0.359241407282298\\
554	-0.282241565928871\\
555	-0.198076381276223\\
556	-0.106140023759053\\
557	-0.0177080356020269\\
558	0.0772818393852504\\
559	0.163483100975554\\
560	0.25288622939479\\
561	0.329554492980151\\
562	0.403263526920909\\
563	0.467618725188458\\
564	0.515986023934602\\
565	0.55507553026564\\
566	0.586460246398019\\
567	0.596173568748327\\
568	0.591015155569039\\
569	0.577233789921593\\
570	0.545087761060586\\
571	0.504572725299568\\
572	0.451231226336404\\
573	0.378930541296995\\
574	0.310122514445084\\
575	0.22905393138808\\
576	0.138491098139184\\
577	0.0515432563270224\\
578	-0.0438238734154766\\
579	-0.137776606423358\\
580	-0.227525013599916\\
581	-0.305647825454798\\
582	-0.378332099965064\\
583	-0.445164979273159\\
584	-0.500308919577947\\
585	-0.545960477103108\\
586	-0.574845342410299\\
587	-0.595630612675905\\
588	-0.596393825711402\\
589	-0.585089082924696\\
590	-0.556731654740361\\
591	-0.519321184717228\\
592	-0.470860977559369\\
593	-0.406137299494731\\
594	-0.335869582552277\\
595	-0.256060832898129\\
597	-0.0787236320866214\\
599	0.105998167063717\\
600	0.193320869277159\\
601	0.278432372871521\\
602	0.356389129266063\\
603	0.426627667740831\\
604	0.484203180839359\\
605	0.533857941441056\\
606	0.565613870152902\\
607	0.586899489154348\\
608	0.59820651188511\\
609	0.586514760978844\\
610	0.566931697632754\\
611	0.53474125594812\\
612	0.486024120297316\\
613	0.43024470166938\\
614	0.360657431920117\\
615	0.281682557882505\\
617	0.10958260150619\\
618	0.0139332061248751\\
621	-0.250182040117579\\
622	-0.328784436977458\\
623	-0.398407658383803\\
624	-0.463571983544171\\
625	-0.516588286795013\\
626	-0.554970615035472\\
627	-0.581445494420223\\
628	-0.597490303269296\\
629	-0.59351732487994\\
630	-0.573746396012211\\
631	-0.546725109940326\\
632	-0.502474199182871\\
633	-0.451191493995793\\
634	-0.383214185349516\\
635	-0.307374659939796\\
636	-0.229732538132339\\
637	-0.138470058682287\\
639	0.0413860119879246\\
640	0.134363618629777\\
641	0.217961243070476\\
642	0.299930974515064\\
643	0.377317766433407\\
644	0.443941350953992\\
645	0.499121116154583\\
646	0.541594315522616\\
647	0.576845187085837\\
648	0.589704728983179\\
649	0.596479331420596\\
650	0.586749310445157\\
651	0.559799341593134\\
652	0.521499415845028\\
653	0.469546140426701\\
654	0.404962786595661\\
655	0.336383410588951\\
656	0.258876393523224\\
657	0.174732232673705\\
659	-0.0145232475447301\\
662	-0.276604759298152\\
663	-0.356615515591329\\
664	-0.419013135558089\\
665	-0.488194045989985\\
666	-0.531993497047097\\
667	-0.569047919707828\\
668	-0.589333832363536\\
669	-0.595011334837181\\
670	-0.588981544015951\\
671	-0.570444345777105\\
672	-0.533086914246724\\
673	-0.486686665200523\\
674	-0.428223908246309\\
675	-0.362025426147284\\
676	-0.281517219493253\\
677	-0.198424707796221\\
678	-0.105620817110776\\
679	-0.0174665220952193\\
680	0.0745565278225513\\
681	0.16213987267929\\
683	0.327685361681688\\
684	0.399297042814851\\
685	0.462711925840267\\
686	0.51710444727405\\
687	0.554209571863794\\
688	0.579777931024637\\
689	0.593425991488402\\
690	0.592410877290604\\
691	0.577951248740192\\
692	0.545456320487119\\
693	0.505774900196684\\
694	0.450746358651486\\
695	0.388240059424788\\
696	0.31043874930856\\
697	0.228785740058356\\
698	0.139230967889944\\
699	0.0481091496121735\\
700	-0.0399898099112761\\
701	-0.132154948560128\\
703	-0.303852732512041\\
704	-0.37293712252449\\
705	-0.443301274969144\\
706	-0.496998150620129\\
707	-0.543801793652619\\
708	-0.574616865411826\\
709	-0.590036754730136\\
710	-0.593526565359298\\
711	-0.586273827498189\\
712	-0.562268381829199\\
713	-0.522976947735515\\
714	-0.473035393343707\\
715	-0.407873857476261\\
716	-0.337363774341156\\
717	-0.257682271485464\\
718	-0.174180208731286\\
719	-0.0795538736119852\\
722	0.193444119022843\\
724	0.354735429863922\\
725	0.424265909776295\\
726	0.483812287131059\\
727	0.529474927260708\\
728	0.567323769246741\\
729	0.587447031073225\\
730	0.594611718359374\\
731	0.587751190902509\\
732	0.57250699649876\\
733	0.534467376371595\\
734	0.48946974209548\\
735	0.428952424763793\\
736	0.362558657913723\\
737	0.27893523482453\\
738	0.199904396439706\\
740	0.0200827489802577\\
742	-0.160363625969694\\
743	-0.241623469818023\\
744	-0.327719947244077\\
745	-0.399640324319535\\
746	-0.465758725109481\\
747	-0.515449762860044\\
748	-0.557256300225163\\
749	-0.58086591115898\\
750	-0.591855365261836\\
751	-0.59153150646307\\
752	-0.578515886065361\\
753	-0.551393023710261\\
754	-0.502191886749642\\
755	-0.446596206188588\\
756	-0.385878139982651\\
757	-0.314134641151213\\
758	-0.227529524089732\\
759	-0.144295861586215\\
760	-0.0534061823877892\\
761	0.0414660128385549\\
763	0.213820957838379\\
764	0.301728848693983\\
765	0.377154003901978\\
766	0.442973920951317\\
767	0.49983583571202\\
768	0.545320610022827\\
769	0.576254322338173\\
770	0.590349732617142\\
771	0.596811864260417\\
772	0.584636992986361\\
773	0.558489362573255\\
774	0.51891189259004\\
775	0.472569481532901\\
776	0.408685325655824\\
777	0.338200366186356\\
778	0.256891106740568\\
779	0.169622933809478\\
780	0.0837402251140702\\
781	-0.00853206631472858\\
782	-0.0972637547702107\\
783	-0.187298283593464\\
784	-0.271332583234198\\
785	-0.353563815509915\\
786	-0.423790787584039\\
787	-0.484253758027535\\
788	-0.528563026531629\\
789	-0.559116106881447\\
790	-0.585906467709719\\
791	-0.597478824686277\\
792	-0.587928205768094\\
793	-0.572281886038127\\
794	-0.53933505267787\\
795	-0.48823661955339\\
796	-0.433584478375906\\
797	-0.362903592508701\\
798	-0.283552086450072\\
799	-0.202073471816675\\
800	-0.112343651549509\\
801	-0.0166330533347718\\
802	0.0886367228990821\\
803	0.188496897295863\\
804	0.275348330695124\\
805	0.352620811609995\\
806	0.426912589790845\\
807	0.485167708169683\\
808	0.532114342095156\\
809	0.562123918424732\\
810	0.576222956592119\\
811	0.572490793221732\\
812	0.553863007069594\\
813	0.520213627477915\\
814	0.475260013496154\\
815	0.407756617420091\\
816	0.338484563853854\\
817	0.257084630202826\\
819	0.0734698375931657\\
821	-0.121528674193542\\
822	-0.210892596156555\\
823	-0.29728791242178\\
824	-0.373982768529459\\
825	-0.439495962997171\\
826	-0.488889875447057\\
827	-0.528626309917399\\
828	-0.543774254434993\\
829	-0.544238493949251\\
830	-0.526176882235177\\
831	0.576834288993723\\
832	0.50871913268702\\
833	0.441828129623445\\
834	0.369630654180128\\
835	0.291956813996876\\
838	0.0680507287775072\\
840	-0.0697154836971094\\
841	-0.124073202042837\\
842	-0.181742926694369\\
843	-0.229082658423977\\
844	-0.259146459589374\\
845	-0.290885520637858\\
846	-0.309021674251198\\
847	-0.320850847861038\\
848	-0.323051675278748\\
849	-0.319270349425551\\
850	-0.293936552854575\\
851	-0.270716991295558\\
852	-0.237676240125893\\
854	-0.156045095775426\\
855	-0.105050126795049\\
856	-0.0480944334212836\\
857	0.00551476704231391\\
858	0.0569787832259863\\
859	0.109942147459151\\
860	0.166822138926818\\
861	0.218773768399842\\
862	0.255995530947075\\
863	0.285626646546007\\
865	0.334613078726306\\
866	0.347527737137625\\
867	0.358483261807123\\
868	0.363698357564317\\
869	0.360980527067568\\
870	0.355699932593325\\
871	0.341103922838101\\
872	0.323626763260563\\
873	0.299070537914758\\
874	0.270147680967511\\
876	0.2028252558448\\
877	0.160749207262143\\
878	0.116218844981177\\
879	0.0676039563927588\\
881	-0.026589432307901\\
882	-0.0627591545162431\\
883	-0.111662168086696\\
884	-0.156626924123429\\
885	-0.20008949119665\\
886	-0.23797299995249\\
887	-0.26785130920689\\
888	-0.294841890077805\\
889	-0.326399980095175\\
890	-0.342584929678651\\
892	-0.366896576443196\\
893	-0.362112257215358\\
894	-0.360449388032976\\
895	-0.34488507978358\\
896	-0.333951835254084\\
898	-0.287291591482244\\
899	-0.254725747229713\\
900	-0.220349674323643\\
901	-0.177757251844923\\
902	-0.133623587904822\\
903	-0.0935326129388159\\
904	-0.0453984035220856\\
905	-0.000898666176908591\\
907	0.0953413727393126\\
908	0.137219808726968\\
910	0.216324737536524\\
911	0.253447615956702\\
912	0.284832669127354\\
913	0.309950766572911\\
914	0.333878678477504\\
915	0.352398039151467\\
916	0.359395073194719\\
917	0.363944948265271\\
918	0.357946394448845\\
919	0.353600447861027\\
920	0.338884155744381\\
921	0.32217950391032\\
922	0.29993599155523\\
923	0.273028462292132\\
924	0.240008505125843\\
926	0.15697702006355\\
927	0.11617616912099\\
930	-0.022171157192588\\
931	-0.0734827065584795\\
932	-0.114137480513818\\
933	-0.158872208396133\\
934	-0.201053034675624\\
935	-0.23834753616029\\
936	-0.267437467959553\\
937	-0.301215090380083\\
938	-0.326376683591661\\
939	-0.341975611988801\\
940	-0.351164895549118\\
941	-0.362951728323878\\
942	-0.363734434724392\\
943	-0.361966568953449\\
944	-0.345657257149014\\
945	-0.332899311149049\\
947	-0.286849612156402\\
950	-0.180934995302778\\
951	-0.136108089071513\\
952	-0.0927689256818667\\
953	-0.0456809461147714\\
957	0.136543351580258\\
958	0.181016650340098\\
959	0.222247950046494\\
960	0.251197986630359\\
961	0.290335949413475\\
962	0.315021390388665\\
963	0.333389328185604\\
964	0.348301740359602\\
965	0.357633244975204\\
966	0.365396396337019\\
967	0.360281210550056\\
968	0.353230521999194\\
969	0.342836979524236\\
970	0.322766780575876\\
971	0.299378639005681\\
972	0.269261232492681\\
973	0.233524724616927\\
975	0.157846677632733\\
976	0.11119108277353\\
977	0.0700860977922275\\
980	-0.0719338615863307\\
981	-0.116207750720605\\
982	-0.152363818650883\\
983	-0.20105632903369\\
984	-0.241583115835056\\
985	-0.267205712144005\\
986	-0.299105840108496\\
987	-0.325886436293786\\
988	-0.337227225703828\\
989	-0.354887421138756\\
990	-0.362686015872896\\
991	-0.36630428729768\\
992	-0.362252083720705\\
993	-0.35320374160392\\
995	-0.313393826070751\\
996	-0.282635612183185\\
997	-0.253197612116765\\
998	-0.212772141247569\\
1000	-0.134747585711921\\
1001	-0.091339528161825\\
1002	-0.0422205273580403\\
1003	-0.000732623234853236\\
1004	0.0499998493651219\\
1005	0.092872441057807\\
1006	0.137697053742158\\
1008	0.220906547725008\\
1009	0.253284978681677\\
1011	0.314012527741852\\
1012	0.333244161495713\\
1013	0.350855272463377\\
1014	0.359384978228263\\
1015	0.36496284683335\\
1016	0.356792094652519\\
1017	0.355775949977669\\
1018	0.345288056493246\\
1019	0.324195361571583\\
1020	0.29941174968144\\
1022	0.234658441029751\\
1023	0.197178292429726\\
1024	0.156037997641761\\
1025	0.110564888755562\\
1026	0.0663755986597607\\
1027	0.0203806888825966\\
1028	-0.0270236924261553\\
1029	-0.0704536002790519\\
1030	-0.111820993222864\\
1031	-0.163675655644056\\
1032	-0.201031002995023\\
1033	-0.241832510061158\\
1034	-0.269043691924253\\
1035	-0.297727345123803\\
1036	-0.328536423121477\\
1037	-0.339089215952754\\
1038	-0.358830688318903\\
1039	-0.363061134975396\\
1040	-0.362275451007463\\
1041	-0.359575928642698\\
1042	-0.344098108862454\\
1043	-0.326613373804321\\
1044	-0.310531995116435\\
1045	-0.288107198145099\\
1047	-0.218215697214873\\
1048	-0.178711032083356\\
1049	-0.137347695608696\\
1050	-0.0902544434284209\\
1051	-0.0398475555266486\\
1052	0.000501285909649596\\
1053	0.0481001090897735\\
1054	0.0971238610982255\\
1055	0.141466912922169\\
1057	0.221938855359895\\
1058	0.256929545093499\\
1060	0.31352316645598\\
1061	0.333095625936039\\
1063	0.361026006430166\\
1064	0.363905633937065\\
1065	0.361453406101646\\
1066	0.354649306317697\\
1067	0.342557601311\\
1068	0.321957162712806\\
1069	0.297032132442382\\
1070	0.267840458132014\\
1071	0.233463201377617\\
1072	0.193936474853217\\
1073	0.15647078783968\\
1074	0.114379971486414\\
1075	0.0647707053531121\\
1076	0.0239965989367192\\
1077	-0.0273945173430548\\
1079	-0.120257729147397\\
1080	-0.16305879166066\\
1081	-0.203874367141452\\
1082	-0.238152577919209\\
1084	-0.299709101976532\\
1085	-0.3227871157751\\
1086	-0.340154064059334\\
1087	-0.361547802546738\\
1088	-0.363359676264736\\
1089	-0.363217014480142\\
1091	-0.3509959741491\\
1092	-0.333681806795084\\
1094	-0.282433759505693\\
1096	-0.216626838208867\\
1097	-0.178657040491998\\
1098	-0.130967604826765\\
1099	-0.0854813535224821\\
1100	-0.0385275778376126\\
1101	0.00105278739647474\\
1102	0.0481019603153072\\
1103	0.0935592078139962\\
1104	0.140931796664063\\
1105	0.181491660904612\\
1106	0.224702692313713\\
1107	0.253693116553677\\
1109	0.314729170096598\\
1110	0.338055456830716\\
1111	0.349232126056904\\
1112	0.358101985265421\\
1113	0.364510448097462\\
1115	0.354735637707108\\
1116	0.342663849383825\\
1117	0.32182214091381\\
1118	0.292384656879221\\
1119	0.26584190691392\\
1121	0.199277344863731\\
1122	0.156267594572455\\
1123	0.110577481997097\\
1124	0.0681711752422416\\
1125	0.0164340726614682\\
1126	-0.0288011382294826\\
1127	-0.0711925257014627\\
1129	-0.160968086233879\\
1130	-0.201998171559808\\
1131	-0.241583138899387\\
1132	-0.271777211195513\\
1133	-0.303999945704163\\
1134	-0.324605298837014\\
1135	-0.343669818475064\\
1137	-0.364927398705731\\
1138	-0.363396472114346\\
1139	-0.357953113622898\\
1140	-0.350751391737958\\
1141	-0.331750002167155\\
1142	-0.306341877201703\\
1143	-0.282421824798803\\
1144	-0.251485983748353\\
1146	-0.175888549076262\\
1148	-0.088661574591697\\
1149	-0.0422952564363186\\
1151	0.0540927193524112\\
1152	0.0978424602240011\\
1153	0.144636391859422\\
1154	0.178281074639472\\
1155	0.22058989613015\\
1156	0.257350728532856\\
1157	0.285932618942297\\
1159	0.336403043696009\\
1160	0.351928807594504\\
1161	0.361006496221762\\
1162	0.365058579000561\\
1163	0.36004317631523\\
1164	0.352811985344033\\
1165	0.339747198977875\\
1166	0.318933010375076\\
1167	0.296379060937852\\
1168	0.271272858471548\\
1169	0.230878608948842\\
1170	0.195540821585382\\
1171	0.150763972251298\\
1172	0.112514072247905\\
1173	0.0619606064915388\\
1174	0.0182177227861757\\
1175	-0.030338754211698\\
1176	-0.0740053183985765\\
1177	-0.120114130052116\\
1178	-0.163293882631024\\
1179	-0.19695400098044\\
1180	-0.241918155733856\\
1181	-0.273642837125863\\
1182	-0.296117688456889\\
1183	-0.328256922787205\\
1184	-0.342713345792617\\
1185	-0.358918293045008\\
1187	-0.363293451595382\\
1188	-0.357322377003584\\
1189	-0.348218069109407\\
1190	-0.330195810345685\\
1191	-0.305132949745712\\
1192	-0.286150016092506\\
1193	-0.252793769212985\\
1194	-0.212088121229954\\
1195	-0.174862101747294\\
1196	-0.131641024098371\\
1197	-0.0897161456032336\\
1200	0.0512570270607284\\
1202	0.141742453793995\\
1203	0.182303718391267\\
1204	0.226134337289295\\
1205	0.256016627994995\\
1206	0.293118078722728\\
1207	0.314298032179977\\
1208	0.332770006909868\\
1209	0.349986433253434\\
1210	0.362996016494435\\
1211	0.366952086458241\\
1212	0.362665651378848\\
1213	0.35463929958587\\
1214	0.341364704470834\\
1215	0.316982345393171\\
1216	0.294015926465818\\
1217	0.261429493638843\\
1218	0.230355796778895\\
1219	0.195546480627399\\
1220	0.153937405832494\\
1221	0.113796657554303\\
1222	0.0636282987247796\\
1223	0.0120584775454518\\
1224	-0.0260487657565136\\
1225	-0.0731837822781927\\
1226	-0.122318120473665\\
1227	-0.163529984635716\\
1228	-0.202565400669755\\
1229	-0.240400541802046\\
1230	-0.272722014078681\\
1231	-0.303021229613478\\
1232	-0.326382519660001\\
1233	-0.344287349775641\\
1234	-0.358724815532241\\
1235	-0.36048407644239\\
1236	-0.364277496160412\\
1238	-0.34979714709516\\
1239	-0.333355306145222\\
1240	-0.308787632663552\\
1241	-0.281804680504138\\
1243	-0.211888597861162\\
1244	-0.172521275339477\\
1245	-0.134583113711869\\
1246	-0.0889409884039196\\
1247	-0.0403309618432104\\
1248	0.00512114754383219\\
1249	0.0552886822897563\\
1250	0.101026923133304\\
1251	0.140711047904915\\
1252	0.185577814060252\\
1253	0.225393600940151\\
1255	0.291562633873582\\
1256	0.316228003519427\\
1257	0.335143396761396\\
1258	0.351912824834017\\
1259	0.359913726729701\\
1260	0.362115964465829\\
1261	0.361417591025202\\
1262	0.3523281644907\\
1264	0.322326796687776\\
1265	0.293879728361844\\
1266	0.269665851879836\\
1267	0.228737010091663\\
1268	0.192682406266158\\
1269	0.153810595286814\\
1270	0.112771514494398\\
1271	0.0619610289759294\\
1272	0.0130663651602845\\
1273	-0.0301028848934948\\
1274	-0.0810615848386078\\
1276	-0.165359038406677\\
1278	-0.239755322270867\\
1279	-0.275908334080668\\
1280	-0.302180577721174\\
1281	-0.339391767478901\\
1282	-0.366689215365113\\
1284	-0.409006892136404\\
1285	-0.420201407002423\\
1286	-0.423043102284282\\
1287	-0.414382650938933\\
1288	-0.398312907310356\\
1289	-0.372954556438344\\
1290	-0.344946706418796\\
1291	-0.303009016780834\\
1292	-0.254324285622261\\
1294	-0.127463402389822\\
1295	-0.0545443988380612\\
1296	0.0314800385131093\\
1297	0.113623907945112\\
1299	0.290801531154102\\
1300	0.387184171383069\\
1301	0.480227104693768\\
1302	0.567659008892406\\
1303	0.657158148592316\\
1304	0.739163065535195\\
1305	0.816963709361971\\
1306	0.880899131835577\\
1307	0.94311099853121\\
1308	0.991015774097832\\
1309	1.02028348441945\\
1310	1.04141650942256\\
1311	-0.11719790073721\\
1312	-0.233861979285393\\
1313	-0.345450671564322\\
1314	-0.445389186669672\\
1315	-0.534477330494155\\
1316	-0.606747042224015\\
1317	-0.66200460965274\\
1318	-0.700023678677553\\
1319	-0.725470082670654\\
1320	-0.734697388583754\\
1321	-0.718655391842731\\
1322	-0.690477287586873\\
1323	-0.64537487432608\\
1324	-0.585169655355912\\
1325	-0.513326497231901\\
1326	-0.427565635671272\\
1327	-0.335192644742619\\
1328	-0.234184347758855\\
1331	0.080442925278021\\
1332	0.18134356246901\\
1333	0.276360519002992\\
1334	0.363392041066618\\
1335	0.440687418284142\\
1336	0.502317929619494\\
1337	0.549615446459484\\
1338	0.585873790767437\\
1339	0.602507950318795\\
1340	0.601549964843343\\
1341	0.586703567628774\\
1342	0.570212078110671\\
1343	0.541908524830433\\
1344	0.496572317174014\\
1345	0.440019134502563\\
1346	0.374924270680822\\
1347	0.297634033261147\\
1348	0.214839521565864\\
1349	0.129576460782118\\
1350	0.0388948000127129\\
1351	-0.049309810638988\\
1352	-0.144698772568063\\
1353	-0.232423831244432\\
1354	-0.315438080460353\\
1355	-0.388695032816486\\
1356	-0.453236090829705\\
1357	-0.511093915018591\\
1358	-0.549731259946384\\
1359	-0.576419604024068\\
1360	-0.594545143327196\\
1361	-0.59322147212697\\
1362	-0.578996436014222\\
1363	-0.551843797277797\\
1364	-0.512385061801069\\
1365	-0.466792500277279\\
1366	-0.399631137459437\\
1367	-0.323506079657363\\
1368	-0.244024369658746\\
1369	-0.158509637917632\\
1370	-0.067366708366535\\
1371	0.0260298899838745\\
1372	0.11513675491824\\
1373	0.20160080650794\\
1374	0.286518193476695\\
1375	0.363304780231829\\
1376	0.433914235753036\\
1377	0.490726429500228\\
1378	0.537174556440732\\
1379	0.569715414206257\\
1380	0.588407319348335\\
1381	0.595125804444251\\
1382	0.587058937212532\\
1383	0.561081937489689\\
1384	0.528831365315909\\
1385	0.482136460890615\\
1386	0.419968581993544\\
1387	0.353843917655922\\
1388	0.273435336872353\\
1389	0.186606719431438\\
1390	0.0980485924851564\\
1391	0.00630104279434818\\
1392	-0.084202268736135\\
1394	-0.260584159655536\\
1395	-0.33825912664588\\
1396	-0.406552682066831\\
1397	-0.469855405492581\\
1398	-0.520340587320334\\
1399	-0.561852866102072\\
1400	-0.585658520155903\\
1401	-0.594470051447388\\
1402	-0.591598874442298\\
1403	-0.577276200088818\\
1404	-0.545651317365355\\
1405	-0.495716749622716\\
1406	-0.44359602147324\\
1407	-0.370615125880704\\
1408	-0.304620294647975\\
1409	-0.215902337401076\\
1410	-0.129015147486371\\
1411	-0.0396920913403846\\
1412	0.0547988598859774\\
1413	0.14272256703498\\
1414	0.23364860675747\\
1415	0.313553928583588\\
1416	0.383935625386584\\
1417	0.447217460476168\\
1418	0.506198710062563\\
1419	0.551534095183797\\
1420	0.578484669623776\\
1421	0.593536560344546\\
1422	0.59616577633642\\
1423	0.578571856658073\\
1424	0.558040576634539\\
1425	0.51284346756438\\
1426	0.462658884959183\\
1427	0.398487926784583\\
1428	0.326821870575714\\
1429	0.241545798252901\\
1430	0.162889431882377\\
1431	0.0704923301709641\\
1432	-0.0181149917707444\\
1434	-0.202484447346706\\
1435	-0.288389067472053\\
1436	-0.367195398321201\\
1437	-0.431627882584053\\
1438	-0.489307782151172\\
1439	-0.533855071064863\\
1440	-0.571983709873621\\
1441	-0.593517541001347\\
1442	-0.60589778505755\\
1443	-0.603584933271577\\
1444	-0.580967477833383\\
1445	-0.548392102049093\\
1446	-0.501962421196367\\
1447	-0.436959658607975\\
1448	-0.368112533156818\\
1449	-0.288483827504933\\
1450	-0.198633107249861\\
1451	-0.104415775652797\\
1452	-0.0061218123796607\\
1453	0.0936879226815108\\
1454	0.189353584099081\\
1455	0.283359076112902\\
1456	0.374680837041069\\
1457	0.458967262311489\\
1458	0.525321160264411\\
1459	0.583672046840093\\
1460	0.632671631264202\\
1461	0.665501029258394\\
1462	0.684784007681628\\
1463	0.690633203059406\\
1464	0.678525269725924\\
1465	0.652267487213976\\
1466	0.613621209729899\\
1467	0.560016867314971\\
1468	0.497977422152417\\
1469	0.423900493957717\\
1470	0.345237865002218\\
1471	-0.305832393780747\\
1472	-0.359887991653068\\
1473	-0.40614212905075\\
1474	-0.447253746365277\\
1475	-0.475359006760755\\
1476	-0.496825421343146\\
1477	-0.513565025676598\\
1478	-0.514504331628814\\
1479	-0.518069645728701\\
1480	-0.50523257941677\\
1481	-0.490348388980692\\
1482	-0.463821742864639\\
1483	-0.435966767608988\\
1484	-0.398884216289389\\
1485	-0.35923657694093\\
1486	-0.316333134015622\\
1487	-0.271308002428668\\
1489	-0.162081296411543\\
1490	-0.109958937890951\\
1491	-0.0644236680450376\\
1492	-0.0125022643910597\\
1494	0.0847831534792931\\
1495	0.127389825170212\\
1496	0.166866367629609\\
1497	0.209162183607987\\
1498	0.240512675997252\\
1499	0.268566466020729\\
1500	0.302965827400385\\
1501	0.31702583265087\\
1502	0.338689513763256\\
1503	0.355143371069062\\
1504	0.357590353604337\\
1505	0.365253861753445\\
1506	0.362405126052181\\
1507	0.352485631247873\\
1508	0.33789863912807\\
1510	0.293425076270978\\
1512	0.22524737971662\\
1513	0.190327623577559\\
1514	0.151030768189685\\
1515	0.102535390842149\\
1516	0.0567069688390802\\
1518	-0.0329034286219212\\
1520	-0.125679676793879\\
1521	-0.170491219175346\\
1523	-0.245681144460832\\
1524	-0.274896420730329\\
1525	-0.309037246118805\\
1526	-0.328442794802413\\
1528	-0.359161081284128\\
1529	-0.36536045357434\\
1530	-0.36597728271272\\
1531	-0.358592319293166\\
1532	-0.34654612219947\\
1533	-0.328229833150999\\
1534	-0.307543408130641\\
1536	-0.246354137161688\\
1537	-0.209289591645756\\
1538	-0.170393926640372\\
1539	-0.126395494876306\\
1540	-0.072977896937573\\
1541	-0.0271806968271449\\
1542	0.0145442570014893\\
1543	0.0602692122010922\\
1544	0.101221337511106\\
1545	0.146595865361633\\
1547	0.228790787058188\\
1548	0.263226244349426\\
1549	0.292623683348211\\
1550	0.317054148764328\\
1551	0.335969708185075\\
1552	0.352736174170332\\
1553	0.364109701507914\\
1554	0.365448699046738\\
1555	0.361749534687988\\
1556	0.353810335635444\\
1557	0.338871308455509\\
1558	0.313568472638053\\
1559	0.291680272397571\\
1560	0.260382353288605\\
1561	0.222147545049665\\
1562	0.185340409380387\\
1563	0.1461254721562\\
1564	0.104816363564169\\
1565	0.0569106324228414\\
1566	0.0114792237059191\\
1568	-0.0867207595838408\\
1569	-0.126423190530659\\
1570	-0.171422524917944\\
1572	-0.24191246881901\\
1573	-0.279024128288711\\
1575	-0.326697592743585\\
1576	-0.349335295041328\\
1577	-0.357934832947194\\
1578	-0.363084556190643\\
1579	-0.362317668326341\\
1580	-0.355825265413841\\
1581	-0.343781405478239\\
1582	-0.326021688708806\\
1583	-0.30514460208633\\
1584	-0.27548690511685\\
1585	-0.248686244265627\\
1586	-0.209586451574978\\
1587	-0.16422472602062\\
1588	-0.125591851863192\\
1589	-0.0784623311042196\\
1590	-0.0296667358747982\\
1591	0.0117574071623494\\
1594	0.150658111736902\\
1595	0.190495646149429\\
1596	0.226961874912377\\
1598	0.292975662077879\\
1599	0.318678113886563\\
1600	0.338050147322065\\
1601	0.353169675377103\\
1603	0.366378668651123\\
1604	0.357045583929903\\
1605	0.353046606627231\\
1606	0.339186282908031\\
1608	0.291326589468099\\
1609	0.261540518753463\\
1611	0.187536599286886\\
1612	0.146886656625611\\
1613	0.0998929101056092\\
1614	0.0582252924327804\\
1615	0.0122358009562049\\
1616	-0.0361881610351702\\
1618	-0.130383116407756\\
1619	-0.16934596416786\\
1620	-0.213048297549449\\
1621	-0.243215707333547\\
1622	-0.277909772900784\\
1623	-0.306185512949469\\
1624	-0.330337045130818\\
1625	-0.351632019448061\\
1626	-0.361046604886724\\
1627	-0.363213142311452\\
1628	-0.358274670536957\\
1629	-0.355749419416497\\
1630	-0.344596369092869\\
1631	-0.328724616977524\\
1632	-0.30404158888814\\
1634	-0.243962694705715\\
1635	-0.202441187311251\\
1636	-0.169294238611201\\
1637	-0.124608864402489\\
1638	-0.0787474378439583\\
1639	-0.0370887225208207\\
1640	0.0152138690937136\\
1643	0.151364995804215\\
1644	0.193164600841556\\
1645	0.230698211669733\\
1647	0.293105902932439\\
1648	0.319190106874885\\
1649	0.335531016084587\\
1650	0.350356414267026\\
1651	0.360326351993535\\
1652	0.364076916523118\\
1653	0.36455307416054\\
1654	0.353023415928419\\
1655	0.335183136688556\\
1657	0.293347708692636\\
1658	0.265480383261092\\
1660	0.184961621723232\\
1661	0.141322566977578\\
1662	0.102750050889881\\
1663	0.0556544177870819\\
1664	0.0110858178736635\\
1665	-0.0401591483587254\\
1666	-0.0835748966233041\\
1667	-0.130198834077873\\
1668	-0.168376564502069\\
1669	-0.212150724566072\\
1670	-0.24731310570678\\
1671	-0.280084711484051\\
1673	-0.328389808202701\\
1674	-0.348737117118617\\
1675	-0.359336784760671\\
1676	-0.365824019108913\\
1677	-0.361867622291811\\
1678	-0.355837615029486\\
1679	-0.347975486833548\\
1680	-0.330614736063126\\
1681	-0.303132270720198\\
1682	-0.273178825522336\\
1683	-0.241575578731954\\
1684	-0.208696896170295\\
1685	-0.168586851334567\\
1686	-0.120287769937477\\
1687	-0.0776978039298228\\
1688	-0.0321381127851055\\
1689	0.0182760127277106\\
1690	0.0599096307823856\\
1691	0.107398619677042\\
1692	0.15169987948093\\
1693	0.193727599090835\\
1694	0.227957231448727\\
1695	0.264851366438052\\
1696	0.300085856743408\\
1697	0.315623507877262\\
1698	0.339795982730266\\
1699	0.35250393021397\\
1700	0.359637143398231\\
1701	0.362104016245212\\
1702	0.361557086649555\\
1703	0.356152148441652\\
1704	0.33401040540457\\
1705	0.316288650472416\\
1706	0.287953550399379\\
1707	0.262972333593098\\
1708	0.224893512003291\\
1710	0.144788819993664\\
1711	0.0994185368144826\\
1712	0.0562543601058678\\
1714	-0.0419189551262207\\
1715	-0.0848619680450611\\
1716	-0.126522373364423\\
1717	-0.175185785042686\\
1718	-0.212740468210086\\
1719	-0.246131182997033\\
1720	-0.284845154388222\\
1722	-0.328857949589747\\
1723	-0.347119823135927\\
1724	-0.360656931039102\\
1725	-0.364394356285629\\
1726	-0.36365237837208\\
1727	-0.355342621651289\\
1728	-0.343574930967861\\
1729	-0.327794868039291\\
1730	-0.299938863395255\\
1731	-0.27524591590236\\
1732	-0.244005906657549\\
1733	-0.206297179117882\\
1734	-0.163139836558003\\
1735	-0.121624295561105\\
1736	-0.0750697690641573\\
1737	-0.0306273535461514\\
1738	0.0214815288127284\\
1739	0.0587125108181681\\
1740	0.106994033330921\\
1741	0.151411161127726\\
1743	0.234123457739315\\
1744	0.265598852809489\\
1745	0.294749382051577\\
1746	0.320193488164023\\
1747	0.341463677697448\\
1749	0.363391037461497\\
1750	0.361296525688431\\
1751	0.361098415878132\\
1752	0.349643059798836\\
1753	0.334975969143215\\
1754	0.31620502052283\\
1755	0.287038529530037\\
1756	0.256539606801653\\
1757	0.220310832734413\\
1758	0.179996343829316\\
1759	0.142847180060016\\
1760	0.101745797147487\\
1762	0.0107310806779424\\
1763	-0.0385525916631195\\
1764	-0.0814741708823021\\
1765	-0.134905542355682\\
1766	-0.180309257265435\\
1767	-0.229802145734538\\
1768	-0.276187232463144\\
1769	-0.314762260773023\\
1770	-0.352007175364179\\
1771	-0.385994286552886\\
1772	-0.40795865434302\\
1773	-0.424585340917474\\
1774	-0.438284037926678\\
1775	-0.439327827855323\\
1776	-0.428959920519446\\
1777	-0.410244056982265\\
1778	-0.380186754397982\\
1779	-0.343131329229436\\
1780	-0.291261471193138\\
1781	-0.230679254787447\\
1782	-0.163183848279004\\
1783	-0.084832432899475\\
1784	-0.00143658315255379\\
1785	0.0923916113415544\\
1786	0.191333143296561\\
1787	0.295888634780567\\
1788	0.405459659348253\\
1789	0.510735252775248\\
1790	0.627485795366738\\
1791	1.07096171345574\\
1792	0.976933300487417\\
1793	0.877045667374659\\
1794	0.765532382242327\\
1796	0.511020255701624\\
1797	0.380062449067736\\
1798	0.241535650485275\\
1799	0.10762872018222\\
1800	-0.0242823387529825\\
1801	-0.145991548165512\\
1802	-0.26095723314711\\
1803	-0.358554150600867\\
1804	-0.448059627219664\\
1805	-0.517608413660128\\
1806	-0.57229936752401\\
1807	-0.610915629804822\\
1808	-0.633730444599223\\
1809	-0.637164983680123\\
1810	-0.623642308795752\\
1811	-0.588765827674251\\
1812	-0.53945172730937\\
1813	-0.480945707812225\\
1814	-0.406971687194527\\
1815	-0.321563539966519\\
1817	-0.13838889004046\\
1818	-0.0357385483043799\\
1819	0.0632055302467052\\
1820	0.158736212993517\\
1821	0.252109840939738\\
1822	0.329485148028198\\
1823	0.400878329680836\\
1824	0.466141286529819\\
1825	0.51825644353994\\
1826	0.556156543133966\\
1827	0.584619843840755\\
1828	0.591729679169475\\
1829	0.590881996748976\\
1830	0.570476504487488\\
1831	0.548785565189974\\
1832	0.502691922735266\\
1833	0.449616914527269\\
1834	0.386988918594398\\
1835	0.31027317160806\\
1836	0.227193290123978\\
1838	0.0494196764643675\\
1839	-0.0406151334823335\\
1840	-0.136185993108029\\
1841	-0.220644931805509\\
1843	-0.381382871144524\\
1844	-0.44295672593671\\
1845	-0.498573616489011\\
1846	-0.545846058671486\\
1847	-0.578026298018813\\
1848	-0.590784202068789\\
1849	-0.593000248489261\\
1850	-0.582355304781231\\
1851	-0.55693626660468\\
1852	-0.523000563375717\\
1853	-0.468317929676232\\
1854	-0.409018442983324\\
1855	-0.337398530455175\\
1856	-0.26003338662531\\
1857	-0.168407424553152\\
1858	-0.0807375974200113\\
1860	0.105377067001427\\
1861	0.188478478220077\\
1862	0.276815577434263\\
1863	0.353211600864142\\
1864	0.424725595762538\\
1865	0.484467125405445\\
1866	0.531691463073003\\
1867	0.563280765304171\\
1868	0.58662248076962\\
1869	0.595293245772154\\
1870	0.595330226519309\\
1871	0.570612516862184\\
1872	0.539110701382924\\
1873	0.488152486649597\\
1874	0.429481900527662\\
1875	0.361891339809517\\
1877	0.202352303553198\\
1878	0.110668373548833\\
1879	0.0166812876937001\\
1880	-0.0700460958464646\\
1881	-0.165525118167807\\
1882	-0.252304390439349\\
1884	-0.402913994900246\\
1885	-0.465660439482235\\
1886	-0.515825423381102\\
1887	-0.556069446849961\\
1888	-0.581850695876255\\
1889	-0.593224118134003\\
1890	-0.594794081562213\\
1891	-0.57915548495339\\
1892	-0.551563981885465\\
1893	-0.50521998522936\\
1894	-0.448333633242783\\
1895	-0.386156819471125\\
1896	-0.309114045931437\\
1897	-0.227239186129282\\
1898	-0.138988720504585\\
1899	-0.0483417662353531\\
1900	0.0397585034952499\\
1902	0.219870848369283\\
1904	0.380749678716256\\
1905	0.446906222751295\\
1906	0.499740807788839\\
1907	0.542420818263054\\
1908	0.575356770628332\\
1909	0.589978598138259\\
1910	0.594340901888245\\
1911	0.582404012955976\\
1912	0.558893669995996\\
1913	0.522990744208528\\
1914	0.47146925228526\\
1915	0.410550773826344\\
1917	0.25733280960776\\
1918	0.169053530772544\\
1919	0.0823549409378757\\
1922	-0.192686979368318\\
1923	-0.280905796687421\\
1924	-0.364281529683012\\
1925	-0.436498813813614\\
1926	-0.501863422089173\\
1927	-0.548220567346561\\
1928	-0.587258082673998\\
1929	-0.604004535142394\\
1930	-0.611336869798834\\
1931	-0.605042931167645\\
1932	-0.58297019999145\\
1933	-0.543852613014224\\
1934	-0.488643233189578\\
1935	-0.424470910206765\\
1936	-0.352347714398093\\
1937	-0.265137724903525\\
1938	-0.174417286865264\\
1940	0.0313881754141221\\
1941	0.13515069613004\\
1942	0.231742971307995\\
1943	0.334476783033551\\
1944	0.423647977715518\\
1945	0.508878710340468\\
1946	0.585613632446893\\
1947	0.647089662285907\\
1948	0.695612845194773\\
1949	0.732875068811154\\
1950	0.755924081936882\\
1951	0.547922236221439\\
1953	0.406083515614682\\
1955	0.258796627492302\\
1956	0.187313509831256\\
1957	0.10843491507103\\
1958	0.0429825681608236\\
1959	-0.0265995200102225\\
1960	-0.0858160970205972\\
1961	-0.148782881373336\\
1962	-0.200085046693403\\
1963	-0.249556523065621\\
1964	-0.29004538907293\\
1965	-0.323592856294454\\
1966	-0.351184506824666\\
1967	-0.372542622858418\\
1968	-0.382715036596892\\
1969	-0.390943447473546\\
1970	-0.388381118798861\\
1971	-0.387142443944413\\
1972	-0.372071001460426\\
1973	-0.361338956904547\\
1974	-0.332243708679925\\
1975	-0.305088296017857\\
1976	-0.274833907172251\\
1978	-0.19808047355491\\
1980	-0.115582570737388\\
1983	0.0177084000306422\\
1984	0.065917671223815\\
1985	0.11221046358105\\
1986	0.157285812610553\\
1987	0.19420775430126\\
1988	0.232519867158771\\
1989	0.273947158648298\\
1990	0.295300502969894\\
1991	0.325472972658645\\
1992	0.342511002740139\\
1993	0.352687415892433\\
1994	0.361441253114208\\
1995	0.363003115988249\\
1996	0.36033592976446\\
1997	0.352213382388982\\
1998	0.329684057116083\\
1999	0.316462585944464\\
2000	0.287161325600664\\
2002	0.218523638315219\\
2003	0.180490744316103\\
2005	0.0963640486093027\\
2006	0.0503097681544205\\
2007	0.000510234604462312\\
2008	-0.0456166592184672\\
2009	-0.0895448693281651\\
2010	-0.135296308923444\\
2011	-0.17520194389499\\
2012	-0.218542247012465\\
2013	-0.251659516368818\\
2014	-0.283597906273371\\
2015	-0.308011644036469\\
2016	-0.33593265556101\\
2017	-0.349251667027602\\
2019	-0.369779157000721\\
2020	-0.360553884324872\\
2021	-0.34980835893839\\
2022	-0.343588787370209\\
2023	-0.323763460491591\\
2024	-0.301372146864651\\
2026	-0.238855812103338\\
2027	-0.20094975423126\\
2028	-0.156541098209345\\
2029	-0.11881992893359\\
2030	-0.0732202433177918\\
2032	0.0206850571266841\\
2033	0.0705136596761804\\
2034	0.117060840375416\\
2035	0.153688183045688\\
2036	0.195174111435335\\
2037	0.235248105717801\\
2038	0.266849790457854\\
2039	0.301419230844203\\
2040	0.324938222870514\\
2041	0.341531265519734\\
2042	0.354952563846382\\
2043	0.360552403750262\\
2044	0.364635663809622\\
2045	0.361116423379826\\
2047	0.336629321318469\\
2048	0.312322707015483\\
2049	0.283986713273862\\
2050	0.253829380588741\\
2051	0.219991865682005\\
2052	0.177529797814259\\
2053	0.136251331028689\\
2054	0.0974719619512143\\
2055	0.0454017521969945\\
2056	0.00202684669420705\\
2057	-0.0447720048655356\\
2058	-0.0888306596175426\\
2059	-0.135506565541618\\
2060	-0.184244810727705\\
2061	-0.219726856346824\\
2063	-0.28420827033824\\
2065	-0.330076853905211\\
2066	-0.348229522616293\\
2067	-0.359568780411337\\
2068	-0.367720603371254\\
2069	-0.358602110362426\\
2070	-0.355066912694383\\
2071	-0.345586730943069\\
2073	-0.298864490766391\\
2074	-0.271258059506181\\
2075	-0.23537635264347\\
2076	-0.200646901694654\\
2077	-0.161377275506311\\
2078	-0.11901131172317\\
2079	-0.0706965428880721\\
2080	-0.0249179271240791\\
2081	0.0397576750938242\\
2082	0.0976672472384053\\
2083	0.161329207798644\\
2084	0.216171254046458\\
2085	0.276734272025351\\
2087	0.3623816824188\\
2088	0.391847287574365\\
2089	0.414085466557026\\
2090	0.420184855257503\\
2091	0.424919877466891\\
2092	0.412722910687535\\
2094	0.347144112094611\\
2095	0.296987780611744\\
2096	0.238165969855345\\
2097	0.164540850536014\\
2098	0.0857900765499835\\
2099	-0.00333550843970443\\
2100	-0.101299812228717\\
2101	-0.204919328235064\\
2102	-0.314221046955026\\
2103	-0.419995768719218\\
2104	-0.530222805864469\\
2105	-0.638886565134726\\
2106	-0.739280742776373\\
2107	-0.83679497363164\\
2108	-0.92518209883383\\
2109	-1.0029378133795\\
2110	-1.07207628094511\\
2111	0.517249917361369\\
2112	0.595919605830204\\
2113	0.657906458606703\\
2114	0.700216613115572\\
2115	0.722227671299152\\
2116	0.719187763519585\\
2117	0.69243561440453\\
2118	0.648394613217533\\
2119	0.585651798113304\\
2120	0.514348163147133\\
2121	0.423695332608077\\
2122	0.321435251566527\\
2123	0.216326083857439\\
2124	0.101717202579948\\
2125	-0.00572596131451064\\
2126	-0.117626113288679\\
2127	-0.219255877417254\\
2128	-0.323864254258297\\
2129	-0.4063334546845\\
2130	-0.483210657193922\\
2131	-0.543896510301693\\
2132	-0.586948878459225\\
2133	-0.613193777151992\\
2134	-0.627793973060761\\
2135	-0.613953370936542\\
2136	-0.593365862715928\\
2137	-0.549870137741891\\
2138	-0.490039908982908\\
2139	-0.413782750379141\\
2140	-0.332387689014013\\
2141	-0.2338276700184\\
2142	-0.146730217441473\\
2143	-0.0568105940178611\\
2144	0.0359365259378137\\
2145	0.123128649000591\\
2146	0.216007301626178\\
2147	0.296919824050292\\
2148	0.374168962137901\\
2149	0.437327465080671\\
2150	0.496215435315662\\
2151	0.542756540081882\\
2152	0.574877804575408\\
2153	0.590628990450114\\
2154	0.590793249090439\\
2155	0.582443572381635\\
2156	0.561396221886753\\
2157	0.52536116174997\\
2158	0.474810965961296\\
2160	0.346724486621497\\
2162	0.179177150370833\\
2164	-0.0065687222431734\\
2166	-0.18267573776393\\
2168	-0.349552082885566\\
2169	-0.422186757386953\\
2171	-0.527047603376559\\
2172	-0.566454694097956\\
2173	-0.588043058578478\\
2174	-0.597525787169161\\
2175	-0.587888135600224\\
2176	-0.569790012154499\\
2177	-0.537748547961201\\
2178	-0.492058908146646\\
2179	-0.436966929942173\\
2180	-0.365123081245656\\
2181	-0.2900609826861\\
2182	-0.208025802160591\\
2183	-0.119491670116986\\
2184	-0.0281539032248475\\
2185	0.0619689590139387\\
2186	0.155966223176165\\
2187	0.244058130387657\\
2188	0.3273684639762\\
2189	0.396188526196511\\
2190	0.457067007000205\\
2191	0.51215661933702\\
2192	0.550149886526924\\
2193	0.578506350141197\\
2194	0.592145015207734\\
2195	0.591348091766577\\
2196	0.579497681508656\\
2197	0.547788845904506\\
2198	0.510290241066286\\
2199	0.457041731006484\\
2200	0.39316273811437\\
2201	0.317003674442276\\
2202	0.237112632996286\\
2203	0.152909564112633\\
2204	0.0585839348627815\\
2205	-0.0337607554447459\\
2206	-0.120436113685173\\
2207	-0.218086157638481\\
2208	-0.296560744068756\\
2209	-0.366120635584139\\
2210	-0.439745696789032\\
2211	-0.496629477977876\\
2212	-0.54302461144789\\
2213	-0.5705526134102\\
2214	-0.595431090441707\\
2215	-0.596835777301294\\
2216	-0.583888386579019\\
2217	-0.561366020869627\\
2218	-0.522822223969342\\
2219	-0.474318395285991\\
2220	-0.41212293460012\\
2221	-0.345696080437847\\
2223	-0.182019655760541\\
2224	-0.0872284743172713\\
2225	0.0020032364564031\\
2226	0.0958286737300114\\
2228	0.269529606772267\\
2229	0.349780800343069\\
2230	0.413382855295367\\
2231	0.474053419246502\\
2232	0.523230671638885\\
2233	0.565751214655847\\
2234	0.587848037798722\\
2235	0.59452917371982\\
2236	0.594453807454556\\
2237	0.572322589919168\\
2238	0.535857865758771\\
2239	0.49162572854766\\
2240	0.433524840378595\\
2241	0.37110860722305\\
2242	0.291243968591061\\
2243	0.206667900258708\\
2244	0.119647804694978\\
2245	0.0279023235034401\\
2246	-0.0601549624011568\\
2247	-0.151410981501158\\
2248	-0.238549957864052\\
2249	-0.31937755420995\\
2250	-0.39362012831316\\
2251	-0.459869377739778\\
2252	-0.509634043427468\\
2253	-0.55362449940867\\
2254	-0.581928580139902\\
2255	-0.595977471445622\\
2256	-0.591483109454202\\
2257	-0.577013179644382\\
2258	-0.555047041044872\\
2259	-0.510216706231404\\
2260	-0.455985084094664\\
2261	-0.393655892141851\\
2262	-0.317004585987888\\
2263	-0.236939954912941\\
2264	-0.151045056073144\\
2265	-0.061570465699333\\
2266	0.0330567603155032\\
2267	0.124458451317423\\
2268	0.211709140485254\\
2269	0.293269448130104\\
2270	0.372006047936793\\
2271	0.435258687847636\\
2272	0.494464323134252\\
2273	0.542425974519119\\
2274	0.571150998746816\\
2275	0.58835385381235\\
2276	0.599178409584965\\
2277	0.584690481649432\\
2278	0.562694618756268\\
2279	0.523026691682844\\
2280	0.476039374835182\\
2281	0.415703023028982\\
2282	0.344095375271536\\
2283	0.266029370761771\\
2284	0.182852074635321\\
2285	0.0876764570316482\\
2286	-0.000300429634535249\\
2287	-0.0967722332011363\\
2288	-0.181625607737715\\
2290	-0.343261968155275\\
2291	-0.415852393043679\\
2292	-0.477148463307458\\
2293	-0.524463042177558\\
2294	-0.55998461462832\\
2295	-0.585611322218938\\
2296	-0.591395104915591\\
2297	-0.592025506140544\\
2298	-0.575759734344047\\
2299	-0.540505697058961\\
2300	-0.493950849266184\\
2301	-0.437303127205723\\
2302	-0.370891576008034\\
2303	-0.289766987264557\\
2304	-0.213149399727172\\
2306	-0.0323899138320485\\
2307	0.0605233627493362\\
2309	0.239802563663034\\
2310	0.317521523153573\\
2311	0.39150551027933\\
2312	0.457693875384393\\
2313	0.513092288591452\\
2314	0.54756640672349\\
2315	0.583656162701118\\
2316	0.59185569526835\\
2317	0.595247880055922\\
2318	0.582086523462294\\
2319	0.554767823929524\\
2320	0.509551095694405\\
2321	0.457519320265419\\
2322	0.392999372689246\\
2323	0.31856390970961\\
2324	0.239403960680647\\
2325	0.152750219525387\\
2327	-0.0300684522176198\\
2328	-0.115500617985617\\
2329	-0.209718721834179\\
2330	-0.29133517839773\\
2331	-0.370261941610806\\
2332	-0.437927207994562\\
2333	-0.494587331311777\\
2334	-0.537960881633808\\
2335	-0.573473919042044\\
2336	-0.594702504289216\\
2337	-0.595267214067917\\
2338	-0.587021911062948\\
2339	-0.562074805782686\\
2340	-0.525844054316622\\
2341	-0.475176555355119\\
2342	-0.414233101597347\\
2343	-0.348751762875963\\
2344	-0.270525791088858\\
2346	-0.0919447605438108\\
2347	-0.00312079462082693\\
2348	0.0933426064466403\\
2349	0.1801129206101\\
2350	0.265437752841081\\
2351	0.344662297458399\\
2352	0.416377166300208\\
2353	0.478894791375751\\
2354	0.524353029370104\\
2355	0.558052976759427\\
2356	0.587782376378982\\
2357	0.593090637411024\\
2358	0.593423581833122\\
2359	0.570689791515179\\
2360	0.54315529687301\\
2361	0.492950132265833\\
2362	0.435455177882432\\
2363	0.37281639336652\\
2364	0.291533327731031\\
2365	0.212976284038632\\
2366	0.124304481964373\\
2368	-0.0603023343096538\\
2369	-0.14537744970994\\
2370	-0.237671162542938\\
2371	-0.318446720941211\\
2372	-0.393779458784593\\
2373	-0.457840726284303\\
2374	-0.510724529143772\\
2375	-0.546881478843716\\
2376	-0.579787199308157\\
2377	-0.594888078557233\\
2378	-0.594636495572558\\
2379	-0.57820138886882\\
2380	-0.555072601994652\\
2381	-0.509908503479437\\
2382	-0.461320295236874\\
2383	-0.394710827019026\\
2384	-0.325210944743503\\
2386	-0.151908105762686\\
2387	-0.0646845396845492\\
2388	0.0294346455270897\\
2390	0.210769529328445\\
2391	0.291737712324448\\
2392	0.370228518569547\\
2393	0.435333453415296\\
2394	0.491415630592201\\
2395	0.538387538188545\\
2396	0.570750982952177\\
2397	0.592326260180926\\
2398	0.595340777979345\\
2399	0.584502269102359\\
2400	0.562979799457025\\
2401	0.533665580015168\\
2402	0.48435410303955\\
2403	0.425707583887288\\
2404	0.360322342719428\\
2405	0.280032882050364\\
2406	0.193246149789047\\
2407	0.101392522755759\\
2408	0.00561398988293149\\
2409	-0.0856803814208433\\
2410	-0.180175261934437\\
2412	-0.356004778148417\\
2413	-0.425460908041259\\
2414	-0.493175062853425\\
2415	-0.552097024708473\\
2416	-0.587864895761868\\
2417	-0.619363505464662\\
2418	-0.630696130953311\\
2419	-0.62943549061265\\
2420	-0.615703400730126\\
2421	-0.589613398960864\\
2422	-0.537886522913595\\
2423	-0.483299657541011\\
2424	-0.416173329757385\\
2425	-0.336705140421145\\
2426	-0.254431874800957\\
2428	-0.0686155472044447\\
2429	0.023559482723158\\
2430	0.121171177149336\\
2431	0.644708786061528\\
2432	0.658808824784501\\
2433	0.656814522374589\\
2434	0.650278326956141\\
2435	0.637541487055387\\
2436	0.614665035536291\\
2437	0.583762730401304\\
2438	0.543803721714085\\
2439	0.499141323647109\\
2440	0.451796743265731\\
2441	0.401143135904022\\
2442	0.34181456005399\\
2443	0.280774579670833\\
2444	0.218515023258078\\
2445	0.159587064581501\\
2446	0.0964618330895064\\
2447	0.038269769604085\\
2448	-0.018329073523546\\
2449	-0.0731531198307493\\
2450	-0.125714565755061\\
2451	-0.1717766035149\\
2452	-0.214062486943931\\
2453	-0.260937018925233\\
2454	-0.293918402941927\\
2455	-0.319708371060187\\
2456	-0.344094150083492\\
2457	-0.356539733151749\\
2458	-0.366714415442402\\
2459	-0.369718253022256\\
2460	-0.368067061735928\\
2461	-0.363108932730484\\
2462	-0.35503431604775\\
2464	-0.320147353815628\\
2465	-0.297339281218228\\
2466	-0.264469574539362\\
2467	-0.22989144515941\\
2468	-0.193309858273096\\
2471	-0.0627229032829746\\
2473	0.0317560322650934\\
2476	0.164351721772618\\
2477	0.20519939427686\\
2478	0.242878252634455\\
2479	0.274508179112672\\
2480	0.301874640695587\\
2481	0.327578910617831\\
2482	0.343908463500156\\
2483	0.358281523685946\\
2484	0.362829736503045\\
2485	0.36355025567218\\
2486	0.357895656954952\\
2487	0.344763501030229\\
2488	0.327689838823972\\
2489	0.305128493226221\\
2490	0.2837661250569\\
2491	0.247205998663958\\
2492	0.213616875042135\\
2493	0.172631086123602\\
2494	0.133493729960719\\
2495	0.0868184549922262\\
2496	0.036048681286502\\
2498	-0.0523588575715621\\
2499	-0.0993206093485242\\
2500	-0.138906956881328\\
2501	-0.185677187117108\\
2502	-0.227599684604684\\
2503	-0.258007745352643\\
2504	-0.291641956385774\\
2505	-0.315392656651966\\
2506	-0.333706258290022\\
2507	-0.350583184333573\\
2508	-0.359950054187721\\
2509	-0.364483675603879\\
2510	-0.361863086172889\\
2512	-0.344945674472001\\
2514	-0.291546787605967\\
2515	-0.266725184319967\\
2516	-0.232939316390912\\
2517	-0.190094874878014\\
2518	-0.15366790255348\\
2519	-0.110387584902583\\
2520	-0.0594387736182398\\
2521	-0.0162873559838772\\
2522	0.0346706197042295\\
2523	0.075126292654204\\
2524	0.121492757975375\\
2525	0.166177594737746\\
2526	0.206924984148372\\
2527	0.243513198520759\\
2528	0.275641073699262\\
2529	0.302912270994057\\
2530	0.325336457762205\\
2531	0.345531604744338\\
2532	0.355750632871604\\
2533	0.363953959953506\\
2534	0.365811378139824\\
2535	0.352765171095598\\
2536	0.350293978118316\\
2537	0.327625897668895\\
2538	0.309368884320975\\
2539	0.282263568974486\\
2540	0.243261602006896\\
2542	0.178320117878684\\
2543	0.127577575325631\\
2544	0.0844616350359502\\
2545	0.038017111582576\\
2546	-0.00658188950546901\\
2547	-0.0534501256793192\\
2548	-0.103625425696919\\
2549	-0.146438146692617\\
2551	-0.225682562412658\\
2552	-0.25473349546246\\
2553	-0.290058844545911\\
2554	-0.315814941026019\\
2555	-0.339242033668143\\
2556	-0.355355043449435\\
2558	-0.365541303324335\\
2560	-0.355677012077194\\
2561	-0.337401087694616\\
2562	-0.323452264681691\\
2563	-0.290655348185282\\
2564	-0.262199763844819\\
2565	-0.229486261600869\\
2566	-0.194107819519559\\
2567	-0.151100674377631\\
2568	-0.105849234435027\\
2569	-0.0581967790076305\\
2570	-0.0166459934293925\\
2571	0.0353869823420609\\
2572	0.0849461531993256\\
2573	0.122590333478911\\
2574	0.167697815571046\\
2575	0.20536896668591\\
2576	0.239828083775137\\
2577	0.279689772082293\\
2578	0.306167949921473\\
2579	0.326788020491222\\
2580	0.34606266059609\\
2581	0.357470040181852\\
2582	0.361032107854953\\
2583	0.358117659932759\\
2584	0.358763664161415\\
2585	0.346255158699933\\
2586	0.328738729383076\\
2587	0.305458106145124\\
2589	0.246095037403848\\
2590	0.211250232979182\\
2591	0.168689640702723\\
2592	0.129408932617935\\
2594	0.0372639319821246\\
2595	-0.0124519258843065\\
2596	-0.0575460980603566\\
2597	-0.100587620630449\\
2598	-0.14724612833561\\
2599	-0.186750241931804\\
2600	-0.221273458095766\\
2601	-0.258701718640623\\
2602	-0.288735678199828\\
2604	-0.334851114028424\\
2605	-0.351669367306386\\
2606	-0.357460000173432\\
2607	-0.367095323107606\\
2608	-0.364014132689135\\
2609	-0.349403541022184\\
2610	-0.338474416598729\\
2611	-0.319597609200173\\
2612	-0.293287722498462\\
2613	-0.264805526526288\\
2614	-0.230405794554372\\
2615	-0.192487366876776\\
2616	-0.151569640329399\\
2617	-0.108707975813104\\
2618	-0.0611954793739642\\
2619	-0.0158219281765923\\
2620	0.0353907981025259\\
2621	0.078835255219019\\
2622	0.125478121550259\\
2625	0.245646760425643\\
2626	0.278802090042063\\
2627	0.30375751849806\\
2628	0.330610982823146\\
2629	0.345637704692308\\
2630	0.357502445248429\\
2631	0.367794968800354\\
2632	0.36651294663443\\
2633	0.359300972843812\\
2634	0.346498107923253\\
2635	0.327076132135517\\
2636	0.305295030964317\\
2637	0.281830654837449\\
2638	0.240947762832548\\
2639	0.207673273897854\\
2640	0.168794569215606\\
2641	0.126982576040518\\
2642	0.0779975061391269\\
2643	0.0370984234236857\\
2644	-0.00904975994126289\\
2645	-0.059255023399146\\
2646	-0.103247463067873\\
2647	-0.149186463992464\\
2648	-0.184277150794969\\
2649	-0.225423707902337\\
2650	-0.261435828782851\\
2651	-0.293278006887704\\
2652	-0.316549905979628\\
2654	-0.351368560563515\\
2655	-0.360235264918174\\
2656	-0.36680900654892\\
2657	-0.360916527938116\\
2659	-0.340997663623057\\
2660	-0.315242709692029\\
2661	-0.286862582510366\\
2662	-0.265685663864133\\
2663	-0.232333644283244\\
2666	-0.104465217880715\\
2667	-0.0591561258593174\\
2668	-0.0123645736030085\\
2670	0.0767746054775671\\
2671	0.128771597457217\\
2672	0.168613232465304\\
2673	0.205278674448437\\
2674	0.246879652231655\\
2675	0.274091804679301\\
2676	0.304690123678938\\
2677	0.329567338471406\\
2679	0.356585809484386\\
2680	0.363418335201004\\
2681	0.367772158569096\\
2682	0.358555651301685\\
2683	0.344838550419354\\
2684	0.327704162473765\\
2685	0.304716493620163\\
2686	0.280393215316963\\
2687	0.244412546186595\\
2689	0.16915931661606\\
2691	0.0829929690976314\\
2693	-0.00906645637496695\\
2694	-0.0564414877835588\\
2695	-0.101961950417262\\
2696	-0.142920600822436\\
2697	-0.191104647740303\\
2699	-0.265099599700989\\
2700	-0.289489254897035\\
2701	-0.316941802748261\\
2702	-0.337946410855238\\
2703	-0.355818861895386\\
2705	-0.364452185850951\\
2706	-0.363855692713059\\
2707	-0.354469009501827\\
2708	-0.338867893205588\\
2710	-0.295067528292748\\
2711	-0.261516494568696\\
2713	-0.189118832629902\\
2714	-0.148489552847423\\
2715	-0.102442048996636\\
2716	-0.0551380373872234\\
2717	-0.0121382902775622\\
2718	0.0345024074736102\\
2719	0.0784044943184199\\
2721	0.170374073773473\\
2723	0.247025461139401\\
2724	0.279215263243714\\
2725	0.306410987559957\\
2726	0.330119418986669\\
2727	0.349203286860302\\
2728	0.358859928009679\\
2729	0.364445735727259\\
2730	0.363352969396601\\
2731	0.354635975272686\\
2732	0.344340223945437\\
2733	0.328590965220428\\
2734	0.302462364816165\\
2735	0.278018724226968\\
2736	0.245126967612578\\
2737	0.208096555956217\\
2738	0.168616263130389\\
2739	0.125295791901863\\
2742	-0.0102442805155079\\
2743	-0.0590263442231844\\
2744	-0.110251988962773\\
2746	-0.189132057235383\\
2747	-0.225730417695104\\
2748	-0.259266152400414\\
2749	-0.294664410866972\\
2751	-0.337605079952027\\
2752	-0.353275371824111\\
2753	-0.361562252267049\\
2754	-0.364600187053384\\
2755	-0.361168651177195\\
2756	-0.35193257309993\\
2757	-0.339190920043166\\
2758	-0.319727031047933\\
2759	-0.289717793983982\\
2760	-0.262054438273026\\
2761	-0.223743580382688\\
2762	-0.190576336917729\\
2763	-0.148387759798879\\
2764	-0.100009272401167\\
2765	-0.0544802233880546\\
2766	-0.0134519326875306\\
2768	0.0845191150560822\\
2769	0.129172355341325\\
2770	0.165931742122666\\
2771	0.212281162755062\\
2772	0.24655776386453\\
2773	0.277917911656004\\
2774	0.304700478399809\\
2775	0.325885507526891\\
2776	0.345647226970868\\
2778	0.368012482801078\\
2779	0.363052353060993\\
2780	0.356134391309752\\
2781	0.344445905514476\\
2782	0.325458798662567\\
2783	0.308399478880347\\
2784	0.277134627604028\\
2786	0.209152450292549\\
2788	0.123715023481054\\
2790	0.0317560072680863\\
2791	-0.00986418480897555\\
2792	-0.0581480636287779\\
2793	-0.105073132216148\\
2794	-0.145152023140326\\
2795	-0.190151881395195\\
2797	-0.265892135900231\\
2798	-0.291719254504187\\
2799	-0.32131511892112\\
2800	-0.340344153914884\\
2802	-0.360470641962365\\
2803	-0.365980135570226\\
2804	-0.361683511531282\\
2805	-0.351068109961943\\
2806	-0.336143907748465\\
2807	-0.320025201373028\\
2808	-0.293235179710791\\
2809	-0.264092330955464\\
2810	-0.225600114191366\\
2811	-0.190857641813636\\
2812	-0.143938811747375\\
2813	-0.103829151721584\\
2814	-0.0585683163326394\\
2815	-0.00564448592240296\\
2816	0.0418024636433074\\
2817	0.081135801144228\\
2819	0.174556461644897\\
2820	0.208751620370094\\
2821	0.25012102359824\\
2822	0.280326126258842\\
2823	0.306283432179043\\
2825	0.349818093752219\\
2826	0.359180521378221\\
2828	0.363830085839254\\
2830	0.344877636710862\\
2831	0.330170131195246\\
2832	0.304751234242758\\
2834	0.243504839530942\\
2835	0.204515667799114\\
2836	0.170447333907305\\
2837	0.119095799167098\\
2839	0.0339117159956004\\
2840	-0.016282580382267\\
2842	-0.104864623962385\\
2843	-0.147241272812153\\
2844	-0.193359448105639\\
2845	-0.229814744716805\\
2846	-0.264868641907924\\
2847	-0.294226860908111\\
2848	-0.318336914059273\\
2849	-0.340547231102391\\
2850	-0.351831660231255\\
2851	-0.361222797732808\\
2852	-0.365931650179391\\
2853	-0.3609004630257\\
2854	-0.35003218005113\\
2855	-0.334956378533661\\
2856	-0.315148557024713\\
2857	-0.291024909036423\\
2858	-0.261680941747727\\
2860	-0.187991672788485\\
2861	-0.146084713849632\\
2863	-0.0524582088091847\\
2864	-0.00779175400930399\\
2865	0.0421390591313866\\
2866	0.0802608122971833\\
2867	0.128405381494304\\
2869	0.214361954352626\\
2870	0.24817267594608\\
2872	0.308181891305139\\
2873	0.330955419258316\\
2874	0.347069666605876\\
2875	0.353660071782997\\
2876	0.366074512676278\\
2877	0.355646269477347\\
2878	0.357308804999775\\
2880	0.326132266161949\\
2881	0.306773099610382\\
2882	0.274257060311811\\
2883	0.24692535871327\\
2884	0.207549791654401\\
2885	0.165090107450851\\
2888	0.0324585830899196\\
2889	-0.0203993164395797\\
2890	-0.0601736122998773\\
2891	-0.106369605325199\\
2892	-0.148851821803873\\
2893	-0.187334855852441\\
2894	-0.230666082771222\\
2895	-0.266378353615892\\
2897	-0.317287687673797\\
2898	-0.33944347832221\\
2899	-0.354782541650366\\
2901	-0.367015912100669\\
2902	-0.360850774602113\\
2903	-0.353065097215222\\
2904	-0.338336958048785\\
2906	-0.289759717594279\\
2907	-0.256194425895956\\
2908	-0.226682689530207\\
2909	-0.185111427388165\\
2910	-0.146308685393251\\
2912	-0.0515065393306031\\
2913	-0.00552832946732451\\
2914	0.0340767668126318\\
2915	0.0862808245137785\\
2917	0.172617303155221\\
2918	0.210341132137273\\
2919	0.244052957822532\\
2920	0.281902133664062\\
2921	0.306274892928286\\
2922	0.329102684475856\\
2923	0.347258098177008\\
2924	0.363827463447706\\
2925	0.368724754069717\\
2926	0.360148611935983\\
2927	0.355505567183627\\
2928	0.346375029174851\\
2929	0.321523542299929\\
2930	0.305605955501505\\
2931	0.273766741317104\\
2932	0.244731149377003\\
2933	0.202774151387075\\
2934	0.16247520711795\\
2935	0.126771436567651\\
2938	-0.0185149504113724\\
2939	-0.0601343973025905\\
2940	-0.107258646444734\\
2941	-0.156657103554608\\
2942	-0.194757028124513\\
2943	-0.224088381989986\\
2944	-0.264926313665001\\
2945	-0.299112853748284\\
2946	-0.321630141615969\\
2947	-0.340330229016672\\
2948	-0.351985893335041\\
2949	-0.362230729916973\\
2950	-0.36397128655608\\
2951	-0.360685224091867\\
2952	-0.353220598240114\\
2953	-0.333131161521578\\
2954	-0.31576214038887\\
2955	-0.290032946362771\\
2956	-0.257833338909677\\
2957	-0.223331452341426\\
2958	-0.185666700377169\\
2959	-0.142497484432624\\
2960	-0.0939743326366624\\
2961	-0.051819776726461\\
2962	-0.00792799352439033\\
2963	0.0382409380381432\\
2964	0.0856709587446858\\
2965	0.127275012798236\\
2966	0.177272202343829\\
2967	0.213259034022485\\
2968	0.246318129586598\\
2969	0.281235464913152\\
2970	0.305144583977381\\
2971	0.33202612208288\\
2972	0.347274654504872\\
2973	0.358999814478921\\
2974	0.36498587689448\\
2975	0.365425815292383\\
2976	0.355088665621679\\
2977	0.341175733066848\\
2978	0.32606121426943\\
2980	0.275185815386976\\
2981	0.237100755688516\\
2982	0.206208364506438\\
2983	0.164684570991994\\
2984	0.118478621818213\\
2987	-0.0114621494535641\\
2988	-0.0641043566379267\\
2989	-0.105613549028021\\
2990	-0.154216513089978\\
2992	-0.231834301476738\\
2993	-0.264684633327306\\
2994	-0.296305037752973\\
2995	-0.323208437483117\\
2996	-0.340268771937644\\
2997	-0.352769452171287\\
2998	-0.363093750283497\\
2999	-0.364623122742159\\
3000	-0.359635700229774\\
3001	-0.349638316105029\\
3002	-0.337082318369085\\
3003	-0.31513455510094\\
3005	-0.256852142261323\\
3006	-0.225504348772574\\
3007	-0.180295812488112\\
3008	-0.143357539210228\\
3010	-0.0554164779637176\\
3011	-0.00592661722885168\\
3012	0.0407328849960322\\
3013	0.0861944587604739\\
3014	0.129577343888741\\
3015	0.170176471122431\\
3016	0.220096145540992\\
3017	0.252752433698333\\
3019	0.30830553265605\\
3020	0.331577571832895\\
3021	0.347582232464902\\
3022	0.357421956724011\\
3023	0.365963213981104\\
3024	0.360584541931985\\
3025	0.360779268217811\\
3026	0.339897710744026\\
3027	0.324910746285695\\
3028	0.303996549702333\\
3029	0.273750653867864\\
3030	0.237112784004694\\
3031	0.205498657438511\\
3032	0.163448433575923\\
3033	0.11828598948432\\
3034	0.0760551512539678\\
3035	0.0306278100442796\\
3036	-0.0128985944620581\\
3037	-0.0643102879103026\\
3038	-0.110075783926277\\
3040	-0.198181294045753\\
3041	-0.248827848893143\\
3042	-0.305596362796223\\
3043	-0.346137677634033\\
3044	-0.382999944146377\\
3046	-0.435107634086762\\
3048	-0.435471214669633\\
3049	-0.423262567424899\\
3050	-0.39524578273722\\
3051	-0.35847557295665\\
3052	-0.313061896581985\\
3053	-0.258588157314989\\
3055	-0.115046729202732\\
3056	-0.036485038074261\\
3057	0.0537495417597711\\
3059	0.229162386083317\\
3060	0.324953856053071\\
3061	0.409443732550244\\
3062	0.497379235874178\\
3063	0.581008023761569\\
3064	0.655351068154232\\
3065	0.719455440652837\\
3066	0.76955631316514\\
3067	0.81850448110572\\
3068	0.849456281176572\\
3069	0.865357977931126\\
3070	0.863132182746995\\
3071	-1.08956401204296\\
3072	-1.07620209408196\\
3073	-1.03579270521504\\
3074	-0.976949184572732\\
3075	-0.901070680373323\\
3076	-0.802132764874841\\
3077	-0.692847548219106\\
3078	-0.572646753857953\\
3079	-0.447068610307269\\
3080	-0.314831153888917\\
3081	-0.17549295563731\\
3082	-0.0407653431539075\\
3083	0.0809473688741491\\
3085	0.315296537766244\\
3086	0.407078217258913\\
3087	0.489505877209467\\
3088	0.551495064165465\\
3089	0.590555998430773\\
3090	0.617156381135374\\
3091	0.621637711829862\\
3092	0.61010121529489\\
3093	0.579928181075047\\
3094	0.52771097952791\\
3095	0.457415936118196\\
3096	0.385599697704492\\
3097	0.292153386801601\\
3098	0.191479478963174\\
3099	0.0819719354244626\\
3100	-0.0215677300820971\\
3101	-0.131166021644731\\
3102	-0.218733999047799\\
3103	-0.304424469086825\\
3104	-0.379032724075842\\
3105	-0.445849101748536\\
3107	-0.545345470183747\\
3108	-0.574942391701825\\
3109	-0.593739523673321\\
3110	-0.592916242809224\\
3111	-0.583992071346984\\
3112	-0.558965367172732\\
3113	-0.5188855979045\\
3114	-0.468896015133396\\
3115	-0.406819695410832\\
3116	-0.341847265830893\\
3117	-0.257087738421887\\
3119	-0.081515561570086\\
3120	0.0113975409126397\\
3121	0.105526023030052\\
3122	0.192616227779581\\
3124	0.35039118789382\\
3125	0.421521334306817\\
3126	0.479893940739657\\
3127	0.529170838524806\\
3128	0.567447516691573\\
3129	0.585469596532675\\
3130	0.593991359935444\\
3131	0.587878611229826\\
3132	0.569646326277052\\
3133	0.536675823625046\\
3134	0.487500847902083\\
3135	0.428992599501271\\
3136	0.363428409276821\\
3137	0.285874533420156\\
3138	0.201535203117146\\
3139	0.119146474334229\\
3140	0.0206806017522467\\
3142	-0.161488640440894\\
3143	-0.246192442883512\\
3144	-0.329142716934257\\
3145	-0.39833111458438\\
3146	-0.4595973465116\\
3147	-0.515465705640054\\
3148	-0.555572225296601\\
3149	-0.583249163757046\\
3150	-0.595761979060626\\
3151	-0.597078008913286\\
3152	-0.577876382020804\\
3153	-0.550441863705601\\
3154	-0.503230534765407\\
3155	-0.451242274666129\\
3156	-0.386318709443003\\
3157	-0.308930773557677\\
3158	-0.234182648931437\\
3159	-0.142224104875368\\
3160	-0.054575059787112\\
3162	0.131287943698226\\
3163	0.220558467436149\\
3165	0.378011621688984\\
3166	0.446607288952237\\
3167	0.497393202451349\\
3168	0.541185907215549\\
3169	0.574574095414391\\
3170	0.593423359112421\\
3171	0.59545843531987\\
3172	0.582898029310854\\
3173	0.560365715494299\\
3174	0.522591060831019\\
3175	0.470763722739321\\
3176	0.407598284165488\\
3177	0.33963967658201\\
3178	0.259268615640394\\
3179	0.172803574543195\\
3181	-0.00887373249724988\\
3182	-0.101152354389797\\
3183	-0.19065453329722\\
3184	-0.270138894480624\\
3185	-0.35168468633583\\
3186	-0.424770422264828\\
3187	-0.481716641204912\\
3188	-0.526559023503523\\
3189	-0.565759162201175\\
3190	-0.582801317803842\\
3191	-0.596744210778979\\
3192	-0.59073353831991\\
3193	-0.571422898206492\\
3194	-0.533021867188381\\
3195	-0.490676141333097\\
3196	-0.430520373200125\\
3197	-0.36702248524125\\
3198	-0.287766956112137\\
3199	-0.201350887609806\\
3200	-0.117229764980948\\
3201	-0.0130330057445462\\
3202	0.0882074464698235\\
3203	0.187435660752271\\
3204	0.275161521731661\\
3205	0.357630501875519\\
3206	0.425931367780322\\
3207	0.484264799771154\\
3208	0.530561245346689\\
3209	0.558244792701316\\
3210	0.576757491580338\\
3211	0.573900939778468\\
3212	0.556147419219087\\
3213	0.52315703543627\\
3214	0.471833559161041\\
3215	0.409234450400618\\
3216	0.338682136388343\\
3217	0.257594941260322\\
3218	0.167248296444996\\
3220	-0.0220206524190871\\
3221	-0.120263674081798\\
3222	-0.210077663200536\\
3223	-0.298189056495175\\
3224	-0.372736146018269\\
3225	-0.442577754882223\\
3226	-0.487118157759596\\
3227	-0.53006423843226\\
3228	-0.543958420145373\\
3229	-0.54524963464246\\
3230	-0.528132167397871\\
3231	0.574800192181556\\
3232	0.509738209005263\\
3233	0.44243203359747\\
3234	0.366464200604696\\
3235	0.29231512753222\\
3236	0.215737531877494\\
3237	0.143341950775266\\
3238	0.0694972074170437\\
3239	-0.00281918657992719\\
3240	-0.0678378434527076\\
3241	-0.123811313282204\\
3242	-0.181334333308314\\
3243	-0.224389299407449\\
3244	-0.265505420939007\\
3245	-0.293770603550456\\
3246	-0.309521795918499\\
3247	-0.320063300741367\\
3248	-0.319924825559156\\
3249	-0.315375063061765\\
3250	-0.294603766954424\\
3251	-0.271229868378668\\
3252	-0.239003353305179\\
3253	-0.195807310786677\\
3254	-0.151435746643983\\
3255	-0.110220705463234\\
3256	-0.051582411390882\\
3257	0.000543291646863509\\
3258	0.057396373741085\\
3259	0.111729635359552\\
3260	0.170046550625102\\
3261	0.219466248087429\\
3262	0.253631608760315\\
3263	0.285486260876041\\
3264	0.311772856959124\\
3265	0.334018970969282\\
3266	0.349356815811461\\
3267	0.358928488230504\\
3268	0.365032976495968\\
3269	0.361831168327626\\
3270	0.354357459445964\\
3271	0.341495912975006\\
3272	0.32308931664511\\
3274	0.275843184675523\\
3275	0.234022633893346\\
3277	0.159285742842712\\
3278	0.115361418060274\\
3279	0.0686945797810949\\
3280	0.0248148182272416\\
3281	-0.0209350169475329\\
3282	-0.0715467351546977\\
3283	-0.115078339897536\\
3284	-0.160411096644566\\
3286	-0.235337787415574\\
3288	-0.300867558405116\\
3289	-0.322030807447391\\
3290	-0.341989937447124\\
3292	-0.361938357895269\\
3293	-0.365385472583966\\
3294	-0.36028379222671\\
3295	-0.350630711734539\\
3296	-0.330197941387269\\
3297	-0.313019131540386\\
3298	-0.288794609477009\\
3299	-0.254311612250604\\
3301	-0.177703299486893\\
3302	-0.136750114043934\\
3303	-0.0888248878691229\\
3304	-0.0482696717635918\\
3305	0.000146694391787605\\
3306	0.0442386128938779\\
3308	0.135557794897977\\
3309	0.179854721692664\\
3310	0.222113584947238\\
3311	0.254855847116232\\
3312	0.282031494832154\\
3313	0.310711643920058\\
3314	0.331484146093317\\
3315	0.348167948211994\\
3316	0.361867647334748\\
3317	0.364534659426226\\
3318	0.364334421773037\\
3319	0.355984539332894\\
3320	0.339555292288878\\
3321	0.3244982905303\\
3322	0.295160819878674\\
3323	0.270854223560491\\
3324	0.236874580838503\\
3325	0.200570716431685\\
3326	0.154290887671323\\
3327	0.113742709670078\\
3328	0.0690904838420465\\
3329	0.0230078636195685\\
3330	-0.0219058793845761\\
3331	-0.0686184951546238\\
3333	-0.15821160825999\\
3334	-0.195537081129714\\
3335	-0.237355146209211\\
3336	-0.269909787767574\\
3337	-0.300456420616683\\
3338	-0.323426111041954\\
3340	-0.358019597574184\\
3342	-0.365702113641873\\
3343	-0.357693467877198\\
3344	-0.35338585238469\\
3345	-0.337111932574317\\
3346	-0.30739449798466\\
3347	-0.284585037451961\\
3348	-0.257171510867011\\
3349	-0.22093299779408\\
3351	-0.135098838705744\\
3352	-0.0906809018219974\\
3353	-0.0431747253351205\\
3355	0.0474309831643041\\
3356	0.0957530465548189\\
3357	0.142347172597965\\
3359	0.218369487026393\\
3360	0.256725292415013\\
3361	0.284564877886169\\
3363	0.333502325002428\\
3364	0.35027524129373\\
3365	0.360263325866072\\
3366	0.361784287071259\\
3367	0.361271302914702\\
3368	0.35350051739124\\
3369	0.343662112392394\\
3371	0.299595146906086\\
3372	0.270225071575624\\
3373	0.234812197782958\\
3374	0.193249706397182\\
3375	0.157755181438006\\
3376	0.115189698829909\\
3377	0.0666388180979993\\
3378	0.0219287055192581\\
3379	-0.0257005077419308\\
3380	-0.0672014580268296\\
3381	-0.11619022070272\\
3383	-0.202310103616128\\
3384	-0.238964544094415\\
3385	-0.273739892307276\\
3387	-0.322362805426565\\
3388	-0.344331983239499\\
3389	-0.349092559712972\\
3390	-0.360693320269547\\
3391	-0.363124436422368\\
3392	-0.355049688929284\\
3393	-0.354189930436405\\
3395	-0.310629663799773\\
3396	-0.283249434054596\\
3397	-0.25443023907701\\
3398	-0.215183055301623\\
3399	-0.182030267141272\\
3400	-0.134674605955297\\
3402	-0.0447091882124369\\
3403	0.00277482063665957\\
3404	0.0468614408250687\\
3405	0.0956327362173397\\
3406	0.137793024322036\\
3407	0.176855129679552\\
3408	0.218859476608031\\
3409	0.253546148788701\\
3411	0.313677382880087\\
3412	0.333956900372868\\
3413	0.349143229909714\\
3415	0.367806040429969\\
3416	0.364957409733506\\
3417	0.354181310892272\\
3418	0.341235825951117\\
3419	0.325275766242157\\
3420	0.301382744611601\\
3421	0.27047656012428\\
3422	0.233495102350844\\
3424	0.157435695738513\\
3425	0.110431055968547\\
3427	0.0198973549663606\\
3428	-0.0238697644167587\\
3429	-0.071970871785652\\
3430	-0.116827676463345\\
3431	-0.158010788123192\\
3432	-0.200548428329057\\
3433	-0.240837305801506\\
3435	-0.299372578350358\\
3436	-0.324574396332991\\
3437	-0.34388260293008\\
3438	-0.357052286368798\\
3439	-0.362978540360018\\
3440	-0.365878474743113\\
3441	-0.360366963732304\\
3442	-0.351852260560463\\
3443	-0.33324453123987\\
3444	-0.311722206344257\\
3445	-0.281858775541423\\
3446	-0.248299655665505\\
3447	-0.213476095523674\\
3448	-0.174092607152488\\
3449	-0.132750670337828\\
3450	-0.0886797523394307\\
3451	-0.0423491602800823\\
3452	0.000924175244108483\\
3453	0.0509842594074144\\
3454	0.092230968880358\\
3455	0.140509070722146\\
3457	0.217769396146196\\
3458	0.254230545170685\\
3460	0.312748940566053\\
3461	0.336428244601393\\
3463	0.360784927395798\\
3464	0.364152808172094\\
3465	0.364600706529927\\
3466	0.353684825326127\\
3467	0.341003676605396\\
3468	0.326595786545113\\
3469	0.299090432794401\\
3470	0.264994804416347\\
3471	0.235794738264303\\
3472	0.193563272451229\\
3473	0.154775546342989\\
3475	0.0679896728706808\\
3476	0.0187467207051668\\
3477	-0.0259274655245463\\
3478	-0.0694501823854807\\
3479	-0.122763419192779\\
3480	-0.164047250552358\\
3481	-0.199428487698242\\
3482	-0.239893355491859\\
3483	-0.272489381296964\\
3484	-0.299123158849397\\
3485	-0.319974224638827\\
3486	-0.346313230817941\\
3487	-0.358210952709669\\
3488	-0.363297737345874\\
3489	-0.364100944486836\\
3490	-0.358835014725628\\
3491	-0.34890627807772\\
3492	-0.328923885993845\\
3493	-0.311659916057579\\
3494	-0.282055969374142\\
3496	-0.214051549170563\\
3497	-0.17685430938036\\
3500	-0.0441478972334153\\
3501	0.00690152931474586\\
3502	0.0500244594145443\\
3504	0.138703203320802\\
3505	0.182538460430806\\
3508	0.287832470854937\\
3510	0.33396966363398\\
3511	0.352269085549779\\
3512	0.362455485016199\\
3513	0.367795669837051\\
3514	0.360025108871923\\
3515	0.356169904871422\\
3516	0.341115859101592\\
3517	0.323823996962346\\
3518	0.296875002741217\\
3519	0.268641572682554\\
3520	0.235450532165032\\
3522	0.153160146867322\\
3523	0.107670108544426\\
3524	0.0651574373432595\\
3525	0.0197403023771585\\
3526	-0.0294679675871521\\
3527	-0.0739058530652983\\
3528	-0.119877976237149\\
3529	-0.163163469463143\\
3531	-0.237872091120607\\
3532	-0.270346218251689\\
3533	-0.303970987935372\\
3534	-0.325545064417838\\
3536	-0.359214282091216\\
3537	-0.358840122060883\\
3538	-0.363502979035275\\
3539	-0.360140248952575\\
3541	-0.330346232816282\\
3542	-0.308813510202071\\
3543	-0.283387345971278\\
3544	-0.254007306004041\\
3545	-0.216237114409068\\
3546	-0.176037871699918\\
3548	-0.0863859720147957\\
3549	-0.0435634410782768\\
3550	0.00666257583179686\\
3551	0.052866929065658\\
3552	0.0968723070081978\\
3553	0.139215209198028\\
3554	0.185203094167264\\
3555	0.22173877146588\\
3557	0.288728686257855\\
3558	0.315572019903811\\
3559	0.332220624039564\\
3560	0.355160887163038\\
3561	0.355953208862957\\
3563	0.365129341868396\\
3564	0.355092561719175\\
3565	0.339484002827703\\
3566	0.320915511095791\\
3567	0.29778213796726\\
3568	0.264529482722992\\
3569	0.234513795467137\\
3570	0.196198210922375\\
3572	0.110481336518205\\
3574	0.016383053620757\\
3575	-0.0292542839797534\\
3576	-0.0723090202013736\\
3577	-0.120131840107206\\
3578	-0.163657142182728\\
3579	-0.205238434492912\\
3580	-0.242613740390425\\
3581	-0.273161582829289\\
3582	-0.29947806731343\\
3583	-0.322685215181991\\
3584	-0.342464330562052\\
3585	-0.357046221461587\\
3586	-0.363963578944549\\
3587	-0.363103394558948\\
3588	-0.35662635637982\\
3589	-0.345467069369079\\
3590	-0.335565237588526\\
3591	-0.308759052368714\\
3592	-0.283434394379128\\
3593	-0.250380412972845\\
3594	-0.209458814799291\\
3595	-0.175552325840727\\
3596	-0.135172993937431\\
3597	-0.0864793395176093\\
3599	0.00480577053349407\\
3600	0.0542704602075901\\
3601	0.0992462027065812\\
3602	0.142862636702375\\
3604	0.22317887898862\\
3605	0.256848707414974\\
3606	0.288715114580555\\
3607	0.31565400411364\\
3608	0.336037777442925\\
3609	0.349135865074913\\
3610	0.359119064257357\\
3612	0.364008940635813\\
3613	0.353455579986075\\
3614	0.33795401712905\\
3615	0.321246264410547\\
3617	0.266061960658135\\
3618	0.234280594693701\\
3619	0.195032766195709\\
3621	0.107966386696262\\
3622	0.0671540478438146\\
3623	0.0159972539058799\\
3625	-0.0758823328310427\\
3627	-0.164052947948676\\
3628	-0.204022901017197\\
3629	-0.241075030317006\\
3630	-0.268732269370958\\
3631	-0.301253540741072\\
3632	-0.328659849824362\\
3633	-0.347546525276812\\
3634	-0.360879774818386\\
3635	-0.359759926579954\\
3636	-0.366330026591186\\
3638	-0.346752882899182\\
3639	-0.334385436337016\\
3640	-0.304086164251203\\
3641	-0.279922543026714\\
3642	-0.244983193781536\\
3643	-0.2121371773801\\
3645	-0.133053983661739\\
3647	-0.0398175008267572\\
3648	0.00311138117240262\\
3649	0.0530079674167609\\
3650	0.0991085403520628\\
3651	0.141269374008516\\
3652	0.185496803417209\\
3653	0.220508790070653\\
3654	0.256952643897876\\
3655	0.289218829703259\\
3656	0.313636052085712\\
3657	0.336716586186867\\
3658	0.350286691021211\\
3660	0.366354805798892\\
3661	0.363203100524515\\
3662	0.354840871864781\\
3663	0.341035901773921\\
3664	0.322097962122825\\
3665	0.297119377898071\\
3666	0.265156960812874\\
3667	0.229977354737912\\
3669	0.154667722672002\\
3670	0.105633191321886\\
3671	0.0616519584873458\\
3672	0.0188210862947926\\
3673	-0.0283557782258868\\
3674	-0.0775842374678177\\
3675	-0.121887369164142\\
3676	-0.164704362719476\\
3677	-0.205209453314637\\
3678	-0.246908514881397\\
3679	-0.275246942961985\\
3680	-0.301979476033921\\
3681	-0.320600580657356\\
3682	-0.347362545115629\\
3683	-0.357854126611528\\
3684	-0.363790179919761\\
3685	-0.365579387992511\\
3686	-0.359534260564487\\
3687	-0.351013157604029\\
3689	-0.308451549748042\\
3690	-0.278617081120501\\
3691	-0.247629304722977\\
3692	-0.209229281911576\\
3693	-0.168365417564473\\
3694	-0.129315869839047\\
3695	-0.0878482972748316\\
3697	0.0106336909084348\\
3698	0.0538079266034401\\
3699	0.100493853707576\\
3700	0.141409928147823\\
3701	0.188596213130495\\
3702	0.224711171711988\\
3703	0.257738837592569\\
3705	0.313598373494642\\
3706	0.333541095531928\\
3707	0.349895353529973\\
3708	0.361180508381494\\
3709	0.365778300548754\\
3710	0.361799236900879\\
3711	0.353174550135464\\
3713	0.32484909303048\\
3716	0.233356721030304\\
3717	0.192057484071938\\
3718	0.149194550825541\\
3719	0.109161864770158\\
3720	0.0608437598525597\\
3722	-0.0295543950146566\\
3723	-0.0759343377276309\\
3724	-0.126288913263579\\
3725	-0.170866026209296\\
3726	-0.207336585565827\\
3727	-0.238910403911177\\
3728	-0.274893901832456\\
3729	-0.304046752836712\\
3730	-0.327359305365462\\
3731	-0.342622821815894\\
3732	-0.35650562355795\\
3734	-0.368724589915928\\
3735	-0.3588141175278\\
3736	-0.346937107517533\\
3737	-0.33173743842417\\
3738	-0.307276957824797\\
3739	-0.280949410406265\\
3740	-0.245519962100389\\
3741	-0.212482642486975\\
3742	-0.168573663799634\\
3743	-0.129037649286602\\
3744	-0.0823175140749299\\
3745	-0.0331576695261901\\
3746	0.00859207111443538\\
3747	0.0582985530727456\\
3748	0.098207565044504\\
3749	0.14869333164097\\
3750	0.183516964271348\\
3751	0.22459394779753\\
3752	0.258859207536261\\
3753	0.287552969849003\\
3755	0.337998555056402\\
3756	0.348585289198127\\
3758	0.366886422127209\\
3759	0.361796186383799\\
3760	0.35409016419635\\
3762	0.318493814533667\\
3763	0.294414812998639\\
3765	0.233443946528041\\
3766	0.189949408981192\\
3767	0.149747847817707\\
3768	0.108222008876965\\
3769	0.0629820854878744\\
3770	0.0100023870922996\\
3771	-0.0331606592380922\\
3772	-0.0799624377041255\\
3773	-0.123353386986309\\
3774	-0.165030257897342\\
3775	-0.205273869280063\\
3777	-0.27478103018575\\
3778	-0.303271783144282\\
3779	-0.325012581975443\\
3780	-0.344042720185826\\
3781	-0.360065880170623\\
3782	-0.363163032025568\\
3783	-0.367437251699812\\
3784	-0.355545821417763\\
3785	-0.351180598089286\\
3786	-0.330501053388616\\
3787	-0.306360531968039\\
3788	-0.278883921263059\\
3789	-0.242294140783997\\
3790	-0.210057779193903\\
3791	-0.171753312991768\\
3792	-0.126425393787031\\
3793	-0.0856956321799771\\
3794	-0.0356790669807197\\
3796	0.0554448694456369\\
3797	0.100516411724584\\
3798	0.143179992696787\\
3799	0.183130910188083\\
3800	0.22484376130069\\
3801	0.262355645826574\\
3802	0.289459706539219\\
3803	0.320313984766017\\
3804	0.335267776943056\\
3805	0.356955016999109\\
3806	0.359756225617275\\
3807	0.364746926441512\\
3808	0.364958841917542\\
3809	0.355785230529364\\
3810	0.336811721079357\\
3811	0.320087539716496\\
3812	0.290647827388511\\
3813	0.259179767384921\\
3814	0.230067285036057\\
3815	0.18653072523739\\
3816	0.144352973597506\\
3817	0.105214049949154\\
3818	0.0628905106905222\\
3819	0.0159606929655638\\
3820	-0.036883036774725\\
3823	-0.165217349608156\\
3824	-0.203950283941595\\
3825	-0.240618906153031\\
3827	-0.308037272082402\\
3829	-0.348699227532052\\
3830	-0.35487876839079\\
3831	-0.359884787333158\\
3832	-0.368407622959239\\
3833	-0.356242244275109\\
3834	-0.347256085139179\\
3835	-0.330203031322526\\
3836	-0.308402942586781\\
3837	-0.275790759913889\\
3838	-0.245389782719485\\
3839	-0.210101785459301\\
3840	-0.169069588098864\\
3841	-0.12504067261716\\
3842	-0.0845287414954328\\
3843	-0.038253681574588\\
3844	0.0117950670537539\\
3845	0.06030309060543\\
3846	0.105158177724206\\
3847	0.146005671163948\\
3848	0.184745467780431\\
3849	0.220859317289523\\
3850	0.258674778640852\\
3851	0.293644876147482\\
3853	0.337719674978416\\
3854	0.352349526098806\\
3855	0.355946929650599\\
3856	0.366764043265448\\
3857	0.365561580334543\\
3859	0.339914863467584\\
3860	0.315827341293698\\
3861	0.293442680817861\\
3862	0.266607821487469\\
3863	0.230350896037635\\
3865	0.145518864007045\\
3866	0.104166621507375\\
3868	0.011966441536515\\
3869	-0.0317016872859313\\
3870	-0.0801568922415754\\
3871	-0.121728545756923\\
3872	-0.168284859895721\\
3873	-0.208403226812152\\
3874	-0.242840172641081\\
3875	-0.276124894583518\\
3876	-0.305519111840113\\
3877	-0.331535981459638\\
3878	-0.34799971828761\\
3879	-0.355943745084005\\
3880	-0.362029272287145\\
3881	-0.366478203984116\\
3882	-0.357012706843307\\
3884	-0.33048500303039\\
3885	-0.307205054697988\\
3886	-0.274307498586495\\
3887	-0.248406762168088\\
3889	-0.165886781480367\\
3890	-0.129364699649614\\
3891	-0.0808756705973792\\
3892	-0.037167764095102\\
3893	0.0122535972041078\\
3894	0.0575644607074537\\
3895	0.100377898480929\\
3896	0.149239867978849\\
3897	0.190559582064907\\
3898	0.226103197765497\\
3899	0.263234169230145\\
3901	0.314921523716748\\
3902	0.335256735602798\\
3903	0.352257423793617\\
3905	0.368539379673621\\
3907	0.353412043631579\\
3908	0.338692065216037\\
3909	0.31791005829291\\
3910	0.291924937238491\\
3911	0.263934316348696\\
3912	0.223927967156214\\
3913	0.191814726324537\\
3914	0.148165975914708\\
3915	0.100177122807509\\
3916	0.0596842789905168\\
3918	-0.0321685487997456\\
3919	-0.0830101463648134\\
3920	-0.123170330381981\\
3921	-0.168886191855108\\
3922	-0.212626858617568\\
3923	-0.247468272846163\\
3924	-0.276103865339337\\
3925	-0.301999905225784\\
3926	-0.331424092393718\\
3927	-0.34645226122575\\
3928	-0.359412560485453\\
3929	-0.359576176120299\\
3930	-0.365663379455327\\
3931	-0.354590194946013\\
3932	-0.346962254243408\\
3933	-0.326140249370383\\
3934	-0.307773000181896\\
3935	-0.273378612914257\\
3936	-0.249033063848401\\
3937	-0.206804117049614\\
3939	-0.126538956066724\\
3940	-0.0820312029359229\\
3941	-0.0323725896696487\\
3942	0.0106388321146369\\
3943	0.0594366455557065\\
3944	0.0987090099133638\\
3946	0.190084668531654\\
3947	0.230913506166416\\
3948	0.25924798721644\\
3949	0.294570244765509\\
3950	0.319523922255485\\
3951	0.337177157623046\\
3952	0.353018961995531\\
3953	0.366808369498358\\
3954	0.361639105765789\\
3955	0.36266520315985\\
3956	0.352930050313716\\
3957	0.338338075556749\\
3958	0.317216172163626\\
3959	0.292248993470366\\
3960	0.260596416465887\\
3961	0.222707662602261\\
3962	0.186708469817404\\
3963	0.147056914908717\\
3966	0.0154644961285157\\
3967	-0.039834910292484\\
3968	-0.0807300759529426\\
3969	-0.126730048111312\\
3970	-0.169744171609182\\
3971	-0.211106423462297\\
3972	-0.246323651945659\\
3973	-0.276905165691915\\
3975	-0.332472985848199\\
3976	-0.34420722153618\\
3977	-0.357347065365957\\
3978	-0.359501025492591\\
3979	-0.365983933266762\\
3980	-0.358915064082339\\
3981	-0.349205872501443\\
3982	-0.327514792322745\\
3984	-0.277460025934488\\
3985	-0.244858749621926\\
3986	-0.207697811393246\\
3987	-0.167337892362411\\
3988	-0.121926382013953\\
3989	-0.0793420257773505\\
3990	-0.0343232768464077\\
3991	0.0148728871445201\\
3992	0.056916419375284\\
3993	0.106829468335491\\
3994	0.148823904920846\\
3995	0.195377551682668\\
3996	0.227344992071266\\
3997	0.264146558783068\\
3998	0.293617270978757\\
3999	0.316217569240962\\
4000	0.340636992789769\\
};
\addlegendentry{0}

\end{axis}
\end{tikzpicture}%

\subsection{Détection d'énergie}


Avec ces signaux ainsi filtrés, nous désirons reconstituer le signal de base. Pour cela nous allons utiliser un détecteur d'énergie.
Nous divisons nos signaux en périodes $T_s$ et sur chaque période nous calculons l'énergie suivant la formule suivante:
\[
E=\sum_{i=1}^{N_s} x_n^2
\]
Enfin, on compare cette énergie à un seuil $K$ qu'on fixera à la moyenne des énergies du signal.
Pour le signal en sortie du passe bas par exemple, si $E>K$ alors le signal reconstitué sera égal à 1 sur cette période $T_s$, sinon il sera égal à 0.

Voici les figures obtenues grâce à cette méthode.
% This file was created by matlab2tikz.
%
%The latest updates can be retrieved from
%  http://www.mathworks.com/matlabcentral/fileexchange/22022-matlab2tikz-matlab2tikz
%where you can also make suggestions and rate matlab2tikz.
%
\definecolor{mycolor1}{rgb}{0.00000,0.44700,0.74100}%
%
\begin{tikzpicture}

\begin{axis}[%
width=4.521in,
height=3.559in,
at={(0.758in,0.488in)},
scale only axis,
xmin=0,
xmax=4000,
xlabel style={font=\color{white!15!black}},
xlabel={temps [s]},
ymin=-0.1,
ymax=1.1,
ylabel style={font=\color{white!15!black}},
ylabel={$\text{\$m}_\text{i}\text{(t)\$}$},
axis background/.style={fill=white},
title style={font=\bfseries},
title={Signal NRZ aléatoire}
]
\addplot [color=mycolor1, forget plot]
  table[row sep=crcr]{%
1	1\\
160	1\\
161	0\\
320	0\\
321	1\\
480	1\\
481	0\\
640	0\\
641	1\\
1440	1\\
1441	0\\
1600	0\\
1601	1\\
1920	1\\
1921	0\\
2240	0\\
2241	1\\
2400	1\\
2401	0\\
2720	0\\
2721	1\\
2880	1\\
2881	0\\
3040	0\\
3041	1\\
3840	1\\
3841	0\\
4000	0\\
};
\end{axis}
\end{tikzpicture}%

% This file was created by matlab2tikz.
%
%The latest updates can be retrieved from
%  http://www.mathworks.com/matlabcentral/fileexchange/22022-matlab2tikz-matlab2tikz
%where you can also make suggestions and rate matlab2tikz.
%
\definecolor{mycolor1}{rgb}{0.00000,0.44700,0.74100}%
%
\begin{tikzpicture}

\begin{axis}[%
width=4.521in,
height=3.566in,
at={(0.758in,0.481in)},
scale only axis,
xmin=0,
xmax=0.018,
ymin=-0.1,
ymax=1.1,
axis background/.style={fill=white},
title style={font=\bfseries},
title={Signal reconstitué}
]
\addplot [color=mycolor1, forget plot]
  table[row sep=crcr]{%
0	0\\
0.00404166666666672	0\\
0.00406250000000008	1\\
0.00425000000000009	1\\
0.00427083333333322	0\\
0.00435416666666666	0\\
0.00437500000000002	1\\
0.00445833333333323	1\\
0.00447916666666659	0\\
0.00456250000000002	0\\
0.00458333333333338	1\\
0.00477083333333339	1\\
0.00479166666666675	0\\
0.00487499999999996	0\\
0.00489583333333332	1\\
0.00497916666666676	1\\
0.00499999999999989	0\\
0.00508333333333333	0\\
0.00510416666666669	1\\
0.00550000000000006	1\\
0.00552083333333342	0\\
0.00560416666666663	0\\
0.00562499999999999	1\\
0.00570833333333343	1\\
0.00572916666666656	0\\
0.0058125	0\\
0.00583333333333336	1\\
0.00602083333333336	1\\
0.00604166666666672	0\\
0.00612499999999994	0\\
0.00614583333333329	1\\
0.00622916666666673	1\\
0.00625000000000009	0\\
0.0063333333333333	0\\
0.00635416666666666	1\\
0.00675000000000003	1\\
0.00677083333333339	0\\
0.00685416666666661	0\\
0.00687499999999996	1\\
0.0069583333333334	1\\
0.00697916666666676	0\\
0.00706249999999997	0\\
0.00708333333333333	1\\
0.00727083333333334	1\\
0.0072916666666667	0\\
0.0106041666666667	0\\
0.0106250000000001	1\\
0.0107083333333333	1\\
0.0107291666666667	0\\
0.0108124999999999	0\\
0.0108333333333333	1\\
0.0110208333333333	1\\
0.0110416666666666	0\\
0.0111250000000001	0\\
0.0111458333333334	1\\
0.0112291666666666	1\\
0.01125	0\\
0.0113333333333334	0\\
0.0113541666666668	1\\
0.0117499999999999	1\\
0.0117708333333333	0\\
0.0118541666666667	0\\
0.0118750000000001	1\\
0.0119583333333333	1\\
0.0119791666666667	0\\
0.0120625000000001	0\\
0.0120833333333332	1\\
0.0122708333333332	1\\
0.0122916666666666	0\\
0.012375	0\\
0.0123958333333334	1\\
0.0124791666666666	1\\
0.0125	0\\
0.0125833333333334	0\\
0.0126041666666667	1\\
0.0129999999999999	1\\
0.0130208333333333	0\\
0.0131041666666667	0\\
0.0131250000000001	1\\
0.0132083333333333	1\\
0.0132291666666666	0\\
0.0133125000000001	0\\
0.0133333333333334	1\\
0.0135208333333334	1\\
0.0135416666666666	0\\
0.013625	0\\
0.0136458333333334	1\\
0.0137291666666666	1\\
0.0137499999999999	0\\
0.0138333333333334	0\\
0.0138541666666667	1\\
0.0142500000000001	1\\
0.0142708333333332	0\\
0.0143541666666667	0\\
0.014375	1\\
0.0144583333333332	1\\
0.0144791666666666	0\\
0.0145625	0\\
0.0145833333333334	1\\
0.0147708333333334	1\\
0.0147916666666668	0\\
0.014875	0\\
0.0148958333333333	1\\
0.0149791666666668	1\\
0.0149999999999999	0\\
0.0150833333333333	0\\
0.0151041666666667	1\\
0.0155000000000001	1\\
0.0155208333333334	0\\
0.0156041666666666	0\\
0.015625	1\\
0.0157083333333334	1\\
0.0157291666666666	0\\
0.0158125	0\\
0.0158333333333334	1\\
0.0160208333333334	1\\
0.0160416666666667	0\\
0.0161249999999999	0\\
0.0161458333333333	1\\
0.0162291666666667	1\\
0.0162500000000001	0\\
0.0163333333333333	0\\
0.0163541666666667	1\\
0.0166458333333332	1\\
};
\end{axis}
\end{tikzpicture}%

On peut désormais calculer le taux d'erreur binaire $\eta$ correspondant au nombre de bits mal transmits sur le nombre de bits total.
Avec $F_1=980Hz$ et $F_0=1180Hz$ on obtient un taux d'erreur binaire $\eta=9\%$
\subsection{Modification du démodulateur} 

\subsubsection{Modification du nombre de coefficients}

On remarque qu'avec $F_1=980Hz$ et $F_0=1180Hz$, le nombre de coefficient n'a que très peu d'influence sur le taux d'erreur binaire.

%%%%%%%%%%%%%%%%%%%%%%%%%% A CHECK %%%%%%%%%%%%%%%%%%%%%%%%%%%%%%%%%%%%%%%%%%%%%%%%%%%%%%%%%%%%%%%%%%%%%%%%%%%%%%%%%%%%%%%%%%%%%%%%%%
En passant à 201 coefficients, le taux d'erreur reste de 0.2662.
En passant à 61 coefficients, le taux d'erreur reste de 0.0725. 
% V21 : 0.0912 pour les deux quantités de coefficients

% This file was created by matlab2tikz.
%
%The latest updates can be retrieved from
%  http://www.mathworks.com/matlabcentral/fileexchange/22022-matlab2tikz-matlab2tikz
%where you can also make suggestions and rate matlab2tikz.
%
\definecolor{mycolor1}{rgb}{0.00000,0.44700,0.74100}%
%
\begin{tikzpicture}

\begin{axis}[%
width=4.521in,
height=3.566in,
at={(0.758in,0.481in)},
scale only axis,
xmin=0,
xmax=0.018,
ymin=-0.1,
ymax=1.1,
axis background/.style={fill=white},
title style={font=\bfseries},
title={Signal reconstitué}
]
\addplot [color=mycolor1, forget plot]
  table[row sep=crcr]{%
0	0\\
0.00404166666666672	0\\
0.00406250000000008	1\\
0.00425000000000009	1\\
0.00427083333333322	0\\
0.00435416666666666	0\\
0.00437500000000002	1\\
0.00445833333333323	1\\
0.00447916666666659	0\\
0.00456250000000002	0\\
0.00458333333333338	1\\
0.00477083333333339	1\\
0.00479166666666675	0\\
0.00487499999999996	0\\
0.00489583333333332	1\\
0.00497916666666676	1\\
0.00499999999999989	0\\
0.00508333333333333	0\\
0.00510416666666669	1\\
0.00550000000000006	1\\
0.00552083333333342	0\\
0.00560416666666663	0\\
0.00562499999999999	1\\
0.00570833333333343	1\\
0.00572916666666656	0\\
0.0058125	0\\
0.00583333333333336	1\\
0.00602083333333336	1\\
0.00604166666666672	0\\
0.00612499999999994	0\\
0.00614583333333329	1\\
0.00622916666666673	1\\
0.00625000000000009	0\\
0.0063333333333333	0\\
0.00635416666666666	1\\
0.00675000000000003	1\\
0.00677083333333339	0\\
0.00685416666666661	0\\
0.00687499999999996	1\\
0.0069583333333334	1\\
0.00697916666666676	0\\
0.00706249999999997	0\\
0.00708333333333333	1\\
0.00727083333333334	1\\
0.0072916666666667	0\\
0.0106041666666667	0\\
0.0106250000000001	1\\
0.0107083333333333	1\\
0.0107291666666667	0\\
0.0108124999999999	0\\
0.0108333333333333	1\\
0.0110208333333333	1\\
0.0110416666666666	0\\
0.0111250000000001	0\\
0.0111458333333334	1\\
0.0112291666666666	1\\
0.01125	0\\
0.0113333333333334	0\\
0.0113541666666668	1\\
0.0117499999999999	1\\
0.0117708333333333	0\\
0.0118541666666667	0\\
0.0118750000000001	1\\
0.0119583333333333	1\\
0.0119791666666667	0\\
0.0120625000000001	0\\
0.0120833333333332	1\\
0.0122708333333332	1\\
0.0122916666666666	0\\
0.012375	0\\
0.0123958333333334	1\\
0.0124791666666666	1\\
0.0125	0\\
0.0125833333333334	0\\
0.0126041666666667	1\\
0.0129999999999999	1\\
0.0130208333333333	0\\
0.0131041666666667	0\\
0.0131250000000001	1\\
0.0132083333333333	1\\
0.0132291666666666	0\\
0.0133125000000001	0\\
0.0133333333333334	1\\
0.0135208333333334	1\\
0.0135416666666666	0\\
0.013625	0\\
0.0136458333333334	1\\
0.0137291666666666	1\\
0.0137499999999999	0\\
0.0138333333333334	0\\
0.0138541666666667	1\\
0.0142500000000001	1\\
0.0142708333333332	0\\
0.0143541666666667	0\\
0.014375	1\\
0.0144583333333332	1\\
0.0144791666666666	0\\
0.0145625	0\\
0.0145833333333334	1\\
0.0147708333333334	1\\
0.0147916666666668	0\\
0.014875	0\\
0.0148958333333333	1\\
0.0149791666666668	1\\
0.0149999999999999	0\\
0.0150833333333333	0\\
0.0151041666666667	1\\
0.0155000000000001	1\\
0.0155208333333334	0\\
0.0156041666666666	0\\
0.015625	1\\
0.0157083333333334	1\\
0.0157291666666666	0\\
0.0158125	0\\
0.0158333333333334	1\\
0.0160208333333334	1\\
0.0160416666666667	0\\
0.0161249999999999	0\\
0.0161458333333333	1\\
0.0162291666666667	1\\
0.0162500000000001	0\\
0.0163333333333333	0\\
0.0163541666666667	1\\
0.0166458333333332	1\\
};
\end{axis}
\end{tikzpicture}%

% This file was created by matlab2tikz.
%
%The latest updates can be retrieved from
%  http://www.mathworks.com/matlabcentral/fileexchange/22022-matlab2tikz-matlab2tikz
%where you can also make suggestions and rate matlab2tikz.
%
\definecolor{mycolor1}{rgb}{0.00000,0.44700,0.74100}%
\definecolor{mycolor2}{rgb}{0.85000,0.32500,0.09800}%
%
\begin{tikzpicture}

\begin{axis}[%
width=4.521in,
height=3.548in,
at={(0.758in,0.499in)},
scale only axis,
unbounded coords=jump,
xmin=-25000,
xmax=25000,
ymode=log,
ymin=1e-06,
ymax=2.01031921814129,
yminorticks=true,
axis background/.style={fill=white}
]
\addplot [color=mycolor1, forget plot]
  table[row sep=crcr]{%
-24000	0.996832895608896\\
-22118.9109998779	0.997425083103077\\
-22001.7091930167	0.999324506656028\\
-21890.3674764986	0.997816279546436\\
-21773.1656696374	0.998918637145748\\
-21655.9638627762	0.998015312415495\\
-21538.762055915	0.998715345822663\\
-21421.5602490538	0.998222183864211\\
-21304.3584421926	0.998505912788735\\
-21181.2965449884	0.998186508694978\\
-21064.0947381272	0.998541681705627\\
-20946.892931266	0.998398002424056\\
-20823.8310340618	0.998577648912566\\
-20706.6292272006	0.99836229466118\\
-20583.5673299963	0.998613826209011\\
-20466.3655231351	0.998326353057042\\
-20349.163716274	0.998400284672398\\
-20226.1018190697	0.998290165344215\\
-20108.9000122085	0.998436341329032\\
-19985.8381150043	0.998253718814034\\
-19868.6363081431	0.998472699863488\\
-19751.4345012819	0.9984698721876\\
-19628.3726040776	0.998509374151514\\
-19511.1707972165	0.998433744768065\\
-19388.1089000122	0.998546378648339\\
-19270.907093151	0.998397266383833\\
-19153.7052862898	0.998327577897565\\
-19030.6433890856	0.998360421961488\\
-18913.4415822244	0.998364099916366\\
-18790.3796850201	0.998323195719407\\
-18673.177878159	0.998401046699811\\
-18555.9760712978	0.998545354899659\\
-18432.9141740935	0.998438435657947\\
-18315.7123672323	0.998508829888668\\
-18192.6504700281	0.998476285082214\\
-18075.4486631669	0.998471828698683\\
-17958.2468563057	0.998250795477041\\
-17835.1849591015	0.998434332246732\\
-17717.9831522403	0.9982877647853\\
-17594.921255036	0.998396320441471\\
-17477.7194481748	0.998325291767864\\
-17360.5176413136	0.998626036742342\\
-17237.4557441094	0.99836339874995\\
-17120.2539372482	0.998589153788874\\
-16997.1920400439	0.998402109324027\\
-16879.9902331828	0.998551659321487\\
-16762.7884263216	0.998168263083267\\
-16639.7265291173	0.998513528801385\\
-16522.5247222561	0.998205643474438\\
-16405.3229153949	0.998750289360982\\
-16282.2610181907	0.998243726991221\\
-16165.0592113295	0.998713848285666\\
-16041.9973141253	0.998282542622693\\
-15924.7955072641	0.998676677965681\\
-15807.5937004029	0.998041023433576\\
-15684.5318031986	0.998638748484395\\
-15567.3299963374	0.998077918071155\\
-15444.2680991332	0.998600028015826\\
-15327.066292272	0.998115641198052\\
-15209.8644854108	0.998847684143466\\
-15086.8025882066	0.998154228359988\\
-14969.6007813454	0.998811231197942\\
-14846.5388841411	0.998193717476693\\
-14729.3370772799	0.998773895797101\\
-14612.1352704188	0.997940176338702\\
-14489.0733732145	0.998735639162434\\
-14371.8715663533	0.997977122489888\\
-14254.6697594921	0.998992872409103\\
-14131.6078622879	0.998015065708074\\
-14014.4060554267	0.998957631226353\\
-13891.3441582224	0.99805405247396\\
-13774.1423513612	0.998921387929662\\
-13656.9405445001	0.997789975040251\\
-13533.8786472958	0.998884095661933\\
-13416.6768404346	0.997825613544543\\
-13293.6149432304	0.998845703995176\\
-13176.4131363692	0.997862379233059\\
-13059.211329508	0.999118843297217\\
-12936.1494323037	0.99790032863796\\
-12818.9476254426	0.999084342150723\\
-12701.7458185814	0.997624166970605\\
-12578.6839213771	0.999048673577544\\
-12461.4821145159	0.997657801810784\\
-12338.4202173117	0.999011777048074\\
-12221.2184104505	0.997692683751328\\
-12104.0166035893	0.999298759283758\\
-11980.9547063851	0.997728880862959\\
-11863.7528995239	0.999266874459693\\
-11740.6910023196	0.997766467148784\\
-11623.4891954584	0.99923372514095\\
-11506.2873885972	0.997469292468153\\
-11383.225491393	0.99919924001508\\
-11266.0236845318	0.997501327008615\\
-11148.8218776706	0.999502618074376\\
-11025.7599804664	0.997534785947494\\
-10908.5581736052	0.999474562365066\\
-10785.4962764009	0.997569757679972\\
-10668.2944695397	0.999445188483724\\
-10551.0926626786	0.997254098269628\\
-10428.0307654743	0.999414414971301\\
-10310.8289586131	0.997281927184534\\
-10193.6271517519	0.999737675502979\\
-10070.5652545477	0.997311242332677\\
-9953.36344768649	0.999715189706602\\
-9830.30155048224	0.997342146588606\\
-9713.09974362105	0.99969141634205\\
-9595.89793675986	0.997003812058187\\
-9472.83603955561	0.999666268001692\\
-9355.63423269442	0.9970254512871\\
-9232.57233549017	0.999639647603724\\
-9115.37052862898	0.997048542818365\\
-8998.16872176779	1\\
-8875.10682456355	0.997073200368364\\
-8757.90501770236	0.999984587851479\\
-8640.70321084117	0.996706201494268\\
-8517.64131363692	0.999968001824271\\
-8400.43950677573	0.99671863036335\\
-8277.37760957148	0.999950151478832\\
-8160.17580271029	0.996732285126175\\
-8042.9739958491	1.00034592225815\\
-7919.91209864485	0.996747276656142\\
-7802.71029178366	1.00034325355995\\
-7679.64839457942	0.996763732379059\\
-7568.30667806129	1.0007466660933\\
-7439.38469051398	0.996781799490862\\
-7328.04297399585	1.0007629609204\\
-7199.12098644854	0.996801648986212\\
-7093.63936027347	1.00117254046307\\
-6958.8572823831	0.996823480707068\\
-6853.37565620803	1.0012124626192\\
-6718.59357831766	0.996847529653874\\
-6613.1119521426	1.00125465091802\\
-6484.18996459529	0.996341369711935\\
-6372.84824807716	1.00129941439487\\
-6243.92626052985	0.996343591974253\\
-6138.44463435478	1.00177081121673\\
-5997.80246612135	0.996936047542885\\
-5904.0410206324	1.00222236387663\\
-5751.67867171286	0.997629915316042\\
-5663.77731656696	1.00233766353038\\
-5517.27505799048	0.997012989360955\\
-5429.37370284459	1.00280084254137\\
-5265.29117323892	0.998533335895457\\
-5189.10999877915	1.00296951287786\\
-5030.88755951654	0.997879091663719\\
-4954.70638505677	1.00344890616027\\
-4778.90367476498	0.999633925359806\\
-4714.44268099133	1.00368880845461\\
-4544.50006104261	0.998976167006839\\
-4474.17897692589	1.0039569249358\\
-4310.09644732023	0.998234800181342\\
-4239.77536320352	1.00453228240155\\
-4175.31436942986	0.996409093187677\\
-4116.71346599927	0.992501187171489\\
-4046.39238188255	1.00216154085782\\
-3993.65156879502	1.00460372339531\\
-3835.42912953241	0.99751625599657\\
-3765.1080454157	1.00556365993178\\
-3712.36723232816	0.999043486767625\\
-3647.90623855451	0.991303982546033\\
-3595.16542546698	0.997648365395583\\
-3524.84434135026	1.00612798755571\\
-3472.10352826273	0.998915267203532\\
-3407.64253448907	0.990740432618655\\
-3354.90172140154	0.997811403423119\\
-3290.44072762788	1.00690301909345\\
-3243.56000488341	1.00130066977383\\
-3167.37883042364	0.990072462165594\\
-3120.49810767916	0.996678234573682\\
-3050.17702356245	1.00776420609403\\
-3003.29630081797	1.00138494970508\\
-2932.97521670126	0.989198680830231\\
-2891.95458429984	0.994010395470691\\
-2809.91331949701	1.00883862699568\\
-2768.89268709559	1.00297177443135\\
-2692.71151263582	0.988145596868668\\
-2651.6908802344	0.993807993924732\\
-2569.64961543157	1.01022545647664\\
-2528.62898303015	1.0033231583576\\
-2452.44780857038	0.986795818711662\\
-2411.42717616897	0.993585260858148\\
-2335.24600170919	1.01216671848221\\
-2300.08545965084	1.00734677840065\\
-2206.32401416188	0.985280086419699\\
-2171.16347210353	0.993360957849265\\
-2100.84238798681	1.01473416419364\\
-2065.68184592846	1.01063170281041\\
-1960.20021975339	0.983729894764392\\
-1925.03967769503	0.995867027936065\\
-1866.43877426444	1.01857184964234\\
-1837.13832254914	1.01713219173129\\
-1801.97778049078	1.00252184813393\\
-1749.23696740325	0.979469520097654\\
-1719.93651568795	0.980475766028825\\
-1690.63606397265	0.993494969204586\\
-1632.03516054206	1.02551737408785\\
-1608.59479916982	1.0270473779307\\
-1585.15443779758	1.01882562234953\\
-1544.13380539617	0.99019951403007\\
-1514.83335368087	0.9737029497909\\
-1491.39299230863	0.971612847063762\\
-1467.95263093639	0.981819134009605\\
-1438.6521792211	1.00765672763801\\
-1403.49163716274	1.03958633238676\\
-1380.0512757905	1.04860799814564\\
-1362.47100476132	1.04503319746053\\
-1339.03064338909	1.02655516198561\\
-1303.87010133073	0.982896537691829\\
-1280.42973995849	0.959708094065395\\
-1262.84946892931	0.954232975724163\\
-1245.26919790014	0.963181203408916\\
-1227.68892687096	0.98782224689618\\
-1210.10865584178	1.02641408471288\\
-1163.2279330973	1.15163037851228\\
-1151.50775241118	1.17352094110361\\
-1139.78757172506	1.18566143807447\\
-1128.06739103895	1.18533953045042\\
-1116.34721035283	1.17032911510502\\
-1104.62702966671	1.13907227488671\\
-1092.90684898059	1.09082742786468\\
-1081.18666829447	1.02577260482068\\
-1069.46648760835	0.94505563488776\\
-1063.60639726529	0.899447048543343\\
-1057.74630692223	0.850786484979452\\
-1051.88621657917	0.799477112357801\\
-1046.02612623611	0.745970830243797\\
-1040.16603589305	0.690763229858802\\
-1034.30594554999	0.634387851397492\\
-1028.44585520693	0.577409823109672\\
-1022.58576486387	0.520418977184212\\
-1016.72567452081	0.464022545893965\\
-1010.86558417775	0.408837548291054\\
-1005.00549383469	0.35548298235287\\
-999.145403491639	0.3045719404386\\
-993.285313148579	0.256703766500901\\
-987.425222805519	0.212456372314503\\
-981.56513246246	0.172378826660861\\
-975.7050421194	0.136984326241274\\
-969.84495177634	0.106743650040611\\
-958.124771090221	0.0633596428597587\\
-952.264680747161	0.0508954289468671\\
-946.404590404101	0.0449349065649443\\
-940.544500061042	0.0456614181101872\\
-934.684409717982	0.0531912055387559\\
-928.824319374922	0.0675722109178138\\
-917.104138688806	0.1167371584567\\
-911.244048345747	0.15127705573661\\
-905.383958002687	0.192183740818182\\
-899.523867659627	0.239176119363421\\
-893.663777316568	0.29191543318068\\
-887.803686973508	0.350009612126327\\
-881.943596630448	0.413018186548377\\
-876.083506287388	0.480457675957719\\
-870.223415944329	0.551807364638034\\
-864.363325601269	0.626515371693959\\
-858.503235258209	0.70400492135538\\
-852.64314491515	0.783680719516259\\
-846.78305457209	0.864935344250767\\
-840.92296422903	0.947155561231066\\
-835.062873885974	1.02972847989858\\
-829.202783542914	1.11204747212259\\
-823.342693199855	1.19351778244862\\
-811.622512513735	1.35162370894618\\
-799.902331827616	1.49971379171229\\
-788.182151141496	1.63393317599829\\
-776.461970455377	1.7510019194178\\
-764.741789769258	1.84827790080604\\
-753.021609083142	1.92378657891012\\
-741.301428397022	1.97622090119845\\
-729.581247710903	2.00491625036143\\
-717.861067024784	2.00980634185377\\
-706.140886338664	1.99136641083882\\
-694.420705652545	1.950549865115\\
-682.700524966425	1.88872389827024\\
-670.980344280306	1.80760845759489\\
-659.26016359419	1.70922157994595\\
-647.539982908071	1.5958325988802\\
-635.819802221951	1.46992324031516\\
-624.099621535832	1.33415530743284\\
-612.379440849712	1.19134262616218\\
-600.659260163593	1.04442426820744\\
-594.799169820533	0.970371190522173\\
-588.939079477474	0.896435835667216\\
-583.078989134417	0.82300721809515\\
-577.218898791358	0.750475784610004\\
-571.358808448298	0.679231927509071\\
-565.498718105238	0.609664349062848\\
-559.638627762179	0.542158276223431\\
-553.778537419119	0.477093529468452\\
-547.918447076059	0.414842454637032\\
-542.058356733	0.355767731357972\\
-536.19826638994	0.300220076248053\\
-530.33817604688	0.248535863398309\\
-524.47808570382	0.201034688337709\\
-518.617995360761	0.158016905053816\\
-512.757905017701	0.119761168221692\\
-506.897814674641	0.0865220148568635\\
-501.037724331585	0.0585275208026193\\
-495.177633988525	0.0359770680418801\\
-489.317543645466	0.0190392585477494\\
-483.457453302406	0.00785000941386517\\
-477.597362959346	0.00251086225912748\\
-471.737272616287	0.00308753750039357\\
-465.877182273227	0.00960876101515315\\
-460.017091930167	0.0220653870951406\\
-454.157001587108	0.0404098374684634\\
-448.296911244048	0.0645558716463799\\
-442.436820900988	0.0943786990291977\\
-436.576730557928	0.129715438170306\\
-430.716640214869	0.170365923483219\\
-424.856549871809	0.216093854548109\\
-418.996459528749	0.266628278185201\\
-413.136369185693	0.321665388666851\\
-407.276278842633	0.38087062692543\\
-401.416188499574	0.443881055599092\\
-395.556098156514	0.510307982980246\\
-389.696007813454	0.579739806000109\\
-383.835917470395	0.651745039672601\\
-377.975827127335	0.725875498643981\\
-372.115736784275	0.801669595071956\\
-366.255646441216	0.878655716514344\\
-360.395556098156	0.956355647412887\\
-354.535465755096	1.03428799839619\\
-342.815285068977	1.18892888879279\\
-331.095104382861	1.33879135089252\\
-319.374923696741	1.4802465076536\\
-307.654743010622	1.60991631992812\\
-295.934562324503	1.72475460601704\\
-284.214381638383	1.82211467656764\\
-272.494200952264	1.8998027206274\\
-260.774020266144	1.95611686268043\\
-249.053839580029	1.98987245605357\\
-237.333658893909	2.00041464654201\\
-225.61347820779	1.98761951181921\\
-213.89329752167	1.95188516227118\\
-202.173116835551	1.89411409936246\\
-190.452936149431	1.81568791115953\\
-178.732755463312	1.71843509222644\\
-167.012574777196	1.60459246494166\\
-155.292394091077	1.4767604070695\\
-143.572213404957	1.33785190519818\\
-131.852032718838	1.19103539109236\\
-120.131852032719	1.03967139699291\\
-114.271761689659	0.963369855247249\\
-108.411671346599	0.887243287282018\\
-102.551581003539	0.811734065523141\\
-96.6914906604798	0.737282670606966\\
-90.83140031742	0.66432514761198\\
-84.9713099743603	0.593290532616441\\
-79.1112196313043	0.524598265977831\\
-73.2511292882446	0.458655610687447\\
-67.3910389451848	0.395855095241961\\
-61.5309486021251	0.336572001914722\\
-55.6708582590654	0.28116192199057\\
-49.8107679160057	0.229958400141359\\
-43.950677572946	0.183270690263628\\
-38.0905872298863	0.141381644892136\\
-32.2304968868266	0.104545759699547\\
-26.3704065437669	0.0729873936031107\\
-20.5103162007072	0.0468991836360684\\
-14.6502258576475	0.0264406719969562\\
-8.79013551458775	0.0117371606487721\\
-2.93004517152804	0.00287880645765348\\
2.93004517152804	-8.00327355025226e-05\\
8.79013551458775	0.00287880645765348\\
14.6502258576475	0.0117371606487721\\
20.5103162007072	0.0264406719969562\\
26.3704065437669	0.0468991836360684\\
32.2304968868266	0.0729873936031107\\
38.0905872298863	0.104545759699547\\
43.950677572946	0.141381644892136\\
49.8107679160057	0.183270690263628\\
55.6708582590654	0.229958400141359\\
61.5309486021251	0.28116192199057\\
67.3910389451848	0.336572001914722\\
73.2511292882446	0.395855095241961\\
79.1112196313043	0.458655610687447\\
84.9713099743603	0.524598265977831\\
90.83140031742	0.593290532616441\\
96.6914906604798	0.66432514761198\\
102.551581003539	0.737282670606966\\
108.411671346599	0.811734065523141\\
114.271761689659	0.887243287282018\\
120.131852032719	0.963369855247249\\
125.991942375778	1.03967139699291\\
137.712123061898	1.19103539109236\\
149.432303748017	1.33785190519818\\
161.152484434137	1.4767604070695\\
172.872665120252	1.60459246494166\\
184.592845806372	1.71843509222644\\
196.313026492491	1.81568791115953\\
208.033207178611	1.89411409936246\\
219.75338786473	1.95188516227118\\
231.473568550849	1.98761951181921\\
243.193749236969	2.00041464654201\\
254.913929923085	1.98987245605357\\
266.634110609204	1.95611686268043\\
278.354291295323	1.8998027206274\\
290.074471981443	1.82211467656764\\
301.794652667562	1.72475460601704\\
313.514833353682	1.60991631992812\\
325.235014039801	1.4802465076536\\
336.955194725917	1.33879135089252\\
348.675375412036	1.18892888879279\\
360.395556098156	1.03428799839619\\
366.255646441216	0.956355647412887\\
372.115736784275	0.878655716514344\\
377.975827127335	0.801669595071956\\
383.835917470395	0.725875498643981\\
389.696007813454	0.651745039672601\\
395.556098156514	0.579739806000109\\
401.416188499574	0.510307982980246\\
407.276278842633	0.443881055599092\\
413.136369185693	0.38087062692543\\
418.996459528749	0.321665388666851\\
424.856549871809	0.266628278185201\\
430.716640214869	0.216093854548109\\
436.576730557928	0.170365923483219\\
442.436820900988	0.129715438170306\\
448.296911244048	0.0943786990291977\\
454.157001587108	0.0645558716463799\\
460.017091930167	0.0404098374684634\\
465.877182273227	0.0220653870951406\\
471.737272616287	0.00960876101515315\\
477.597362959346	0.00308753750039357\\
483.457453302406	0.00251086225912748\\
489.317543645466	0.00785000941386517\\
495.177633988525	0.0190392585477494\\
501.037724331585	0.0359770680418801\\
506.897814674641	0.0585275208026193\\
512.757905017701	0.0865220148568635\\
518.617995360761	0.119761168221692\\
524.47808570382	0.158016905053816\\
530.33817604688	0.201034688337709\\
536.19826638994	0.248535863398309\\
542.058356733	0.300220076248053\\
547.918447076059	0.355767731357972\\
553.778537419119	0.414842454637032\\
559.638627762179	0.477093529468452\\
565.498718105238	0.542158276223431\\
571.358808448298	0.609664349062848\\
577.218898791358	0.679231927509071\\
583.078989134417	0.750475784610004\\
588.939079477474	0.82300721809515\\
594.799169820533	0.896435835667216\\
600.659260163593	0.970371190522173\\
612.379440849712	1.11820883041428\\
624.099621535832	1.26344848457997\\
635.819802221951	1.40309898238801\\
647.539982908071	1.5342804818781\\
659.26016359419	1.65425181630979\\
670.980344280306	1.76043751227084\\
682.700524966425	1.85045763297059\\
694.420705652545	1.92216315329097\\
706.140886338664	1.97367872277151\\
717.861067024784	2.00345348679547\\
729.581247710903	2.01031921814129\\
741.301428397022	1.9935534943581\\
753.021609083142	1.95294419102921\\
764.741789769258	1.888850302438\\
776.461970455377	1.80225319476383\\
788.182151141496	1.69479196384835\\
799.902331827616	1.5687766965916\\
811.622512513735	1.42717416129172\\
823.342693199855	1.27356176725305\\
829.202783542914	1.19351778244862\\
835.062873885974	1.11204747212259\\
840.92296422903	1.02972847989858\\
846.78305457209	0.947155561231066\\
852.64314491515	0.864935344250767\\
858.503235258209	0.783680719516259\\
864.363325601269	0.70400492135538\\
870.223415944329	0.626515371693959\\
876.083506287388	0.551807364638034\\
881.943596630448	0.480457675957719\\
887.803686973508	0.413018186548377\\
893.663777316568	0.350009612126327\\
899.523867659627	0.29191543318068\\
905.383958002687	0.239176119363421\\
911.244048345747	0.192183740818182\\
917.104138688806	0.15127705573661\\
922.964229031862	0.1167371584567\\
934.684409717982	0.0675722109178138\\
940.544500061042	0.0531912055387559\\
946.404590404101	0.0456614181101872\\
952.264680747161	0.0449349065649443\\
958.124771090221	0.0508954289468671\\
963.98486143328	0.0633596428597587\\
975.7050421194	0.106743650040611\\
981.56513246246	0.136984326241274\\
987.425222805519	0.172378826660861\\
993.285313148579	0.212456372314503\\
999.145403491639	0.256703766500901\\
1005.00549383469	0.3045719404386\\
1010.86558417775	0.35548298235287\\
1016.72567452081	0.408837548291054\\
1022.58576486387	0.464022545893965\\
1028.44585520693	0.520418977184212\\
1034.30594554999	0.577409823109672\\
1040.16603589305	0.634387851397492\\
1046.02612623611	0.690763229858802\\
1051.88621657917	0.745970830243797\\
1057.74630692223	0.799477112357801\\
1063.60639726529	0.850786484979452\\
1069.46648760835	0.899447048543343\\
1075.32657795141	0.94505563488776\\
1087.04675863753	1.02577260482068\\
1098.76693932365	1.09082742786468\\
1110.48712000977	1.13907227488671\\
1122.20730069589	1.17032911510502\\
1133.927481382	1.18533953045042\\
1145.64766206812	1.18566143807447\\
1157.36784275424	1.17352094110361\\
1174.94811378342	1.13795970991585\\
1198.38847515566	1.0740413458022\\
1227.68892687096	0.999314778718142\\
1245.26919790014	0.96967155352373\\
1262.84946892931	0.955493607395847\\
1280.42973995849	0.956466453974541\\
1298.01001098767	0.969636504731472\\
1368.33109510438	1.04503319746053\\
1385.91136613356	1.04860799814564\\
1409.3517275058	1.03958633238676\\
1438.6521792211	1.01357300381582\\
1479.67281162251	0.978172952518464\\
1503.11317299475	0.970987312130006\\
1526.55353436699	0.976000744338978\\
1561.71407642535	0.998937489796227\\
1596.8746184837	1.0217198976992\\
1620.31497985594	1.02759576268758\\
1643.75534122818	1.02367026066286\\
1678.91588328653	1.00455161395467\\
1719.93651568795	0.982238044497972\\
1749.23696740325	0.978622546087365\\
1778.53741911854	0.987273218588935\\
1854.71859357832	1.01908115482726\\
1884.01904529361	1.01605011995891\\
1925.03967769503	0.998642735160218\\
1966.06031009645	0.983729894764392\\
1995.36076181175	0.983535542852743\\
2030.5213038701	0.994600252688535\\
2089.1222073007	1.01427702465734\\
2124.28274935905	1.01199220000341\\
2182.88365278965	0.991390914292411\\
2218.044194848	0.984990686209556\\
2253.20473690636	0.989940984337306\\
2341.10609205225	1.01216671848221\\
2376.26663411061	1.00658237790266\\
2458.30789891344	0.986795818711662\\
2499.32853131486	0.992858632969638\\
2575.50970577463	1.01022545647664\\
2616.53033817605	1.00514266838457\\
2698.57160297888	0.988145596868668\\
2739.5922353803	0.993739096028116\\
2815.77340984007	1.00883862699568\\
2856.79404224148	1.00408808834625\\
2944.69539738738	0.9893577570768\\
2991.57612013185	0.997151558963844\\
3056.03711390551	1.00776420609403\\
3102.91783664998	1.00204119091642\\
3179.09901110976	0.99004709103107\\
3225.97973385423	0.996192330290711\\
3296.30081797094	1.00690301909345\\
3343.18154071542	1.00149238370662\\
3419.36271517519	0.990747937563676\\
3472.10352826273	0.997725582765613\\
3536.56452203638	1.00619437060048\\
3589.30533512392	0.999930677343879\\
3659.62641924063	0.991338608268868\\
3712.36723232816	0.997943552822156\\
3776.82822610182	1.0055989330132\\
3829.56903918935	0.999632696094606\\
3899.89012330607	0.991844655360532\\
3958.49102673666	0.999153001859629\\
4022.95202051031	1.0049446420332\\
4087.41301428397	0.996453740372192\\
4146.01391771456	0.992508514855878\\
4222.19509217434	1.00266302664932\\
4274.93590526187	1.00376167877472\\
4415.5780734953	0.995819575132754\\
4509.33951898425	1.00379215631079\\
4667.56195824686	0.997707425594251\\
4743.74313270663	1.00373982784518\\
4919.54584299841	0.999468934484346\\
4984.00683677207	1.00342457024295\\
5159.80954706385	0.99952723307841\\
5230.13063118056	1.00283564570942\\
5382.49298010011	0.997403496660058\\
5464.53424490294	1.00288309704148\\
5634.47686485167	0.998941180613018\\
5710.65803931144	1.00234124163613\\
5857.16029788793	0.99703076558458\\
5950.92174337688	1.00212498080733\\
6097.42400195336	0.997174958504069\\
6191.18544744232	1.00192569592976\\
6337.6877060188	0.997308198965534\\
6431.44915150775	1.00174123904595\\
6577.95141008424	0.997431791028484\\
6671.71285557319	1.00156981612853\\
6818.21511414968	0.997546833978176\\
6911.97655963863	1.00140991625016\\
7052.61872787205	0.997120562294703\\
7152.24026370407	1.001260257175\\
7292.88243193749	0.997234884553152\\
7392.5039677695	1.00111974328217\\
7533.14613600293	0.99734233621052\\
7632.76767183494	1.00098743237546\\
7773.40984006837	0.997443587672973\\
7873.03137590038	1.00086250950372\\
8013.67354413381	0.997539222313923\\
8113.29507996582	1.00074426596485\\
8248.07715785618	0.997176427220803\\
8353.55878403125	1.00063208233352\\
8488.34086192162	0.997270940256605\\
8593.82248809669	1.00052541474962\\
8728.60456598706	0.997360845519608\\
8834.08619216213	1.00042378361499\\
8968.8682700525	0.997446524320757\\
9074.34989622757	1.00032676426611\\
9209.13197411794	0.997528315694111\\
9314.613600293	1.00023397930964\\
9449.39567818337	0.997606522130844\\
9554.87730435844	1.00014509213173\\
9689.65938224881	0.997681414400053\\
9795.14100842388	1.00005980151245\\
9924.06299597119	0.997369752073881\\
10041.2648028324	0.999635779522416\\
10158.4666096936	0.997088445525501\\
10281.5285068978	0.999559373508491\\
10398.730313759	0.997163127481571\\
10521.7922109633	0.999485810234261\\
10638.9940178244	0.997235173128067\\
10762.0559150287	0.99941489410079\\
10879.2577218899	0.997304759661315\\
11002.3196190941	0.999346447618324\\
11119.5214259553	0.997372048321964\\
11242.5833231596	0.999280309319404\\
11359.7851300208	0.997437186173558\\
11482.847027225	0.99921633205038\\
11600.0488340862	0.997500307590131\\
11723.1107312904	0.999154381331314\\
11840.3125381516	0.997561535644733\\
11963.3744353559	0.999094334077006\\
12080.5762422171	0.997620983265017\\
12203.6381394213	0.999036077394327\\
12320.8399462825	0.997678754264332\\
12438.0417531437	0.999297365892883\\
12561.1036503479	0.997734944223111\\
12678.3054572091	0.999239991136187\\
12801.3673544134	0.997789641337458\\
12918.5691612746	0.999184116114039\\
13041.6310584788	0.997842927092744\\
13158.83286534	0.999129657943485\\
13281.8947625443	0.997894876895867\\
13399.0965694054	0.999076539950561\\
13522.1584666097	0.99794556063267\\
13639.3602734709	0.999024691056645\\
13762.4221706751	0.997995043175541\\
13879.6239775363	0.998974045307422\\
14002.6858747406	0.998043384824429\\
14119.8876816018	0.998924541410523\\
14242.949578806	0.998090641689595\\
14360.1513856672	0.998876122298525\\
14477.3531925284	0.997843741780814\\
14600.4150897326	0.998828734809326\\
14717.6168965938	0.997890908929009\\
14840.6787937981	0.998782329333246\\
14957.8806006593	0.997937163044036\\
15080.9424978635	0.998736859518867\\
15198.1443047247	0.997982547202462\\
15321.2062019289	0.998692282004134\\
15438.4080087901	0.998027101757273\\
15561.4699059944	0.998648556189408\\
15678.6717128556	0.998070864562638\\
15801.7336100598	0.998605644010643\\
15918.935416921	0.998113871173709\\
16041.9973141253	0.99856350973776\\
16159.1991209864	0.998156154987978\\
16282.2610181907	0.998522119798214\\
16399.4628250519	0.99819774747869\\
16516.6646319131	0.99875587623264\\
16639.7265291173	0.998238678261105\\
16756.9283359785	0.998714203128856\\
16879.9902331828	0.99827897527588\\
16997.1920400439	0.998673149089442\\
17120.2539372482	0.998318664913985\\
17237.4557441094	0.998632687595478\\
17360.5176413136	0.998357772108214\\
17477.7194481748	0.998592793563823\\
17600.7813453791	0.998396320441471\\
17717.9831522403	0.998553443263037\\
17841.0450494445	0.998434332246732\\
17958.2468563057	0.998514614204243\\
18081.3087535099	0.998471828698683\\
18198.5105603711	0.998476285082214\\
18315.7123672323	0.998247530992583\\
18438.7742644366	0.998438435657947\\
18555.9760712978	0.998285571159586\\
18679.037968502	0.998401046699811\\
18796.2397753632	0.998323195719407\\
18919.3016725674	0.998364099916366\\
19036.5034794286	0.998360421961488\\
19159.5653766329	0.998327577897565\\
19276.7671834941	0.998397266383833\\
19399.8290806983	0.998291464030889\\
19517.0308875595	0.998433744768065\\
19640.0927847638	0.998255742484413\\
19757.294591625	0.9984698721876\\
19874.4963984861	0.998472699863488\\
19997.5582956904	0.998505663091104\\
20114.7601025516	0.998436341329032\\
20237.8219997558	0.998541131352518\\
20355.023806617	0.998400284672398\\
20478.0857038213	0.998576290262543\\
20595.2875106825	0.998364516555352\\
20718.3494078867	0.998611152628874\\
20835.5512147479	0.99832902416032\\
20958.6131119521	0.99864573075935\\
21075.8149188133	0.998293795106627\\
21193.0167256745	0.998433488196848\\
21316.0786228788	0.998258817458754\\
21433.28042974	0.998468763481287\\
21556.3423269442	0.998224079684414\\
21673.5441338054	0.998503839412849\\
21796.6060310096	0.998189570612652\\
21913.8078378708	0.998538726835942\\
22036.8697350751	0.998155279408667\\
22154.0715419363	0.99857343629547\\
22277.1334391405	0.998121195573732\\
22394.3352460017	0.998607978061857\\
22511.5370528629	0.998329652889421\\
22634.5989500671	0.99864236220627\\
22751.8007569283	0.998294843893876\\
22874.8626541326	0.998676598542008\\
22992.0644609938	0.998260159002638\\
23115.126358198	0.998710696724831\\
23232.3281650592	0.998225588528144\\
23349.5299719204	0.998503266104091\\
23472.5918691246	0.998191122918003\\
23589.7936759858	0.998538011093011\\
23712.8555731901	0.998156752746594\\
23830.0573800513	0.99857270127189\\
23953.1192772555	0.998122468698315\\
24000	0.996851584090231\\
};
\addplot [color=mycolor2, forget plot]
  table[row sep=crcr]{%
-24000	0.00316710439196338\\
-23994.1399096569	0.00314841591240077\\
-23988.2798193139	0.00309279157124961\\
-23982.4197289708	0.00300154425033711\\
-23976.5596386278	0.00287682762893246\\
-23970.6995482847	0.00272158535269575\\
-23964.8394579416	0.0025394815572651\\
-23958.9793675986	0.00233481438709966\\
-23953.1192772555	0.0021124145500993\\
-23947.2591869125	0.00187753130238685\\
-23941.3990965694	0.00163570855442351\\
-23935.5390062263	0.00139265402219157\\
-23929.6789158833	0.00115410451200649\\
-23923.8188255402	0.000925690518250051\\
-23917.9587351972	0.000712803329900807\\
-23912.0986448541	0.000520467782298221\\
-23906.238554511	0.000353223657520941\\
-23900.378464168	0.000215018532550082\\
-23894.5183738249	0.000109114604210599\\
-23888.6582834819	3.80116901284416e-05\\
-23882.7981931388	3.38822303215212e-06\\
-23876.9381027957	6.06163117441586e-06\\
-23871.0780124527	4.59690400613531e-05\\
-23865.2179221096	0.00012216875107912\\
-23859.3578317666	0.000232862462252826\\
-23853.4977414235	0.000375437706815771\\
-23847.6376510805	0.000546529508093835\\
-23841.7775607374	0.000742099795682522\\
-23835.9174703943	0.000957532708655188\\
-23830.0573800513	0.0011877435365919\\
-23824.1972897082	0.00142729872741364\\
-23818.3371993652	0.00167054412948655\\
-23812.4771090221	0.00191173844163099\\
-23806.617018679	0.00214518872106238\\
-23800.756928336	0.00236538475137398\\
-23794.8968379929	0.00256712909893085\\
-23789.0367476499	0.00274565978831946\\
-23783.1766573068	0.00289676270111265\\
-23777.3165669637	0.00301687104530908\\
-23771.4564766207	0.00310314954747763\\
-23765.5963862776	0.0031535613804648\\
-23759.7362959346	0.00316691624660542\\
-23753.8762055915	0.0031428984816753\\
-23748.0161152484	0.00308207451594186\\
-23742.1560249054	0.00298587951579517\\
-23736.2959345623	0.00285658352129508\\
-23730.4358442193	0.00269723787816496\\
-23724.5757538762	0.00251160322893876\\
-23718.7156635331	0.00230406076177345\\
-23712.8555731901	0.00207950881207306\\
-23706.995482847	0.00184324725662905\\
-23701.135392504	0.00160085242911215\\
-23695.2753021609	0.00135804550845667\\
-23689.4152118178	0.00112055748681614\\
-23683.5551214748	0.000893993903777696\\
-23677.6950311317	0.000683702539442639\\
-23671.8349407887	0.000494647189191627\\
-23665.9748504456	0.000331290499238187\\
-23660.1147601026	0.000197488628571264\\
-23654.2546697595	9.64002235887186e-05\\
-23648.3945794164	3.04118540724698e-05\\
-23642.5344890734	1.08167060916587e-06\\
-23636.6743987303	9.10261360959128e-06\\
-23630.8143083873	5.42860425817925e-05\\
-23624.9542180442	0.000135566172420187\\
-23619.0941277011	0.000251025212382475\\
-23613.2340373581	0.000397938614805616\\
-23607.373947015	0.00057283936599741\\
-23601.513856672	0.000771599802240223\\
-23595.6537663289	0.000989529020273607\\
-23589.7936759858	0.00122148358345847\\
-23583.9335856428	0.00146198891095097\\
-23578.0734952997	0.00170536848525696\\
-23572.2134049567	0.00194587782846629\\
-23566.3533146136	0.00217784008551419\\
-23560.4932242705	0.00239578001412959\\
-23554.6331339275	0.00259455321928503\\
-23548.7730435844	0.00276946758167119\\
-23542.9129532414	0.00291639401417057\\
-23537.0528628983	0.00303186393183529\\
-23531.1927725552	0.00311315113484939\\
-23525.3326822122	0.00315833617138655\\
-23519.4725918691	0.00316635166067952\\
-23513.6125015261	0.00313700750641909\\
-23507.752411183	0.00307099540487415\\
-23501.8923208399	0.00296987254054473\\
-23496.0322304969	0.00283602485397446\\
-23490.1721401538	0.00267261074815455\\
-23484.3120498108	0.00248348656134311\\
-23478.4519594677	0.00227311556530636\\
-23472.5918691246	0.00204646263590623\\
-23466.7317787816	0.00180887708212871\\
-23460.8716884385	0.00156596639829582\\
-23455.0115980955	0.00132346391919193\\
-23449.1515077524	0.00108709350147891\\
-23443.2914174093	0.000862434425106863\\
-23437.4313270663	0.000654789703332076\\
-23431.5712367232	0.000469060909647021\\
-23425.7111463802	0.000309632476041659\\
-23419.8510560371	0.00018026819382562\\
-23413.9909656941	8.40223600124816e-05\\
-23408.130875351	2.31676669029033e-05\\
-23402.2707850079	-8.5846270381279e-07\\
-23396.4106946649	1.25121724722545e-05\\
-23390.5506043218	6.29651148781004e-05\\
-23384.6905139788	0.000149310643285781\\
-23378.8304236357	0.000269511826509594\\
-23372.9703332926	0.00042073257420086\\
-23367.1102429496	0.000599404551240569\\
-23361.2501526065	0.000801311376971325\\
-23355.3900622635	0.0010216881225975\\
-23349.5299719204	0.00125533375879847\\
-23343.6698815773	0.00149673390007424\\
-23337.8097912343	0.0017401909490252\\
-23331.9497008912	0.001979958569241\\
-23326.0896105482	0.00221037731292248\\
-23320.2295202051	0.00242600820243765\\
-23314.369429862	0.00262176111254601\\
-23308.509339519	0.00279301492349569\\
-23302.6492491759	0.00293572660830513\\
-23296.7891588329	0.00304652667970685\\
-23290.9290684898	0.00312279874297584\\
-23285.0689781467	0.00316274127696465\\
-23279.2088878037	0.00316541018410151\\
-23273.3487974606	0.00313074110472752\\
-23267.4887071176	0.00305955096821521\\
-23261.6286167745	0.00295351874348592\\
-23255.7685264314	0.00281514584262769\\
-23249.9084360884	0.00264769711123416\\
-23244.0483457453	0.00245512379765688\\
-23238.1882554023	0.00224197031848113\\
-23232.3281650592	0.00201326702029599\\
-23226.4680747162	0.0017744114680305\\
-23220.6079843731	0.00153104106156609\\
-23214.74789403	0.00128889998659323\\
-23208.887803687	0.00105370363988343\\
-23203.0277133439	0.000831003728648045\\
-23197.1676230009	0.000626057228032108\\
-23191.3075326578	0.000443702289782391\\
-23185.4474423147	0.000288244031139553\\
-23179.5873519717	0.000163352900106792\\
-23173.7272616286	7.19780163018575e-05\\
-23167.8671712856	1.62775334424545e-05\\
-23162.0070809425	-2.43233227679195e-06\\
-23156.1469905994	1.62915959366599e-05\\
-23150.2869002564	7.20089582365438e-05\\
-23144.4268099133	0.000163406214180456\\
-23138.5667195703	0.000288327608511191\\
-23132.7066292272	0.000443826016700346\\
-23126.8465388841	0.000626232471060041\\
-23120.9864485411	0.000831242727223982\\
-23115.126358198	0.00105401882863958\\
-23109.266267855	0.00128930327248653\\
-23103.4061775119	0.00153154308315336\\
-23097.5460871688	0.00177502086512199\\
-23091.6859968258	0.00201398974254761\\
-23085.8259064827	0.00224280900082403\\
-23079.9658161397	0.0024560772285584\\
-23074.1057257966	0.00264875981775918\\
-23068.2456354535	0.00281630781211539\\
-23062.3855451105	0.00295476529839557\\
-23056.5254547674	0.00306086280545275\\
-23050.6653644244	0.00313209450585511\\
-23044.8052740813	0.0031667773971503\\
-23038.9451837382	0.00316409106521263\\
-23033.0850933952	0.00312409708999906\\
-23027.2250030521	0.00304773763518869\\
-23021.3649127091	0.00293681325323594\\
-23015.504822366	0.00279394042944539\\
-23009.6447320229	0.00262248986564705\\
-23003.7846416799	0.00242650695928725\\
-22997.9245513368	0.00221061635395005\\
-22992.0644609938	0.00197991281340175\\
-22986.2043706507	0.00173984099396714\\
-22980.3442803077	0.00149606695238417\\
-22974.4841899646	0.0012543444214817\\
-22968.6240996215	0.00102037900962845\\
-22962.7640092785	0.000799693528888538\\
-22956.9039189354	0.000597497630761689\\
-22951.0438285924	0.000418564826472744\\
-22945.1837382493	0.000267119794869069\\
-22939.3236479062	0.000146738638253502\\
-22933.4635575632	6.02644410705445e-05\\
-22927.6034672201	9.74012524708163e-06\\
-22921.7433768771	-3.63981209199257e-06\\
-22915.883286534	2.04424615879587e-05\\
-22910.0231961909	8.14205645561398e-05\\
-22904.1631058479	0.000177857220571269\\
-22898.3030155048	0.000307478134835005\\
-22892.4429251618	0.000467225627731655\\
-22886.5828348187	0.000653330761546587\\
-22880.7227444756	0.000861402258853052\\
-22874.8626541326	0.00108653011481503\\
-22869.0025637895	0.00132340145869683\\
-22863.1424734465	0.00156642593077006\\
-22857.2823831034	0.00180986761577869\\
-22851.4222927603	0.00204798041953301\\
-22845.5622024173	0.00227514369367563\\
-22839.7021120742	0.00248599490723884\\
-22833.8420217312	0.00267555623391138\\
-22827.9819313881	0.00283935206639484\\
-22822.121841045	0.00297351468418514\\
-22816.261750702	0.00307487557957131\\
-22810.4016603589	0.00314104028593801\\
-22804.5415700159	0.00317044494088894\\
-22798.6814796728	0.0031623932482611\\
-22792.8213893298	0.00311707296543532\\
-22786.9612989867	0.00303555152548288\\
-22781.1012086436	0.00291975089634746\\
-22775.2411183006	0.00277240226895486\\
-22769.3810279575	0.0025969816427352\\
-22763.5209376145	0.00239762782755336\\
-22757.6608472714	0.00217904479632731\\
-22751.8007569283	0.00194639069217638\\
-22745.9406665853	0.00170515610887546\\
-22740.0805762422	0.00146103451687346\\
-22734.2204858992	0.00121978789258141\\
-22728.3603955561	0.000987110722275153\\
-22722.500305213	0.00076849559011972\\
-22716.64021487	0.000569103523113763\\
-22710.7801245269	0.000393642153240611\\
-22704.9200341839	0.00024625457320355\\
-22699.0599438408	0.000130421509569822\\
-22693.1998534977	4.8879123236168e-05\\
-22687.3397631547	3.55437834762091e-06\\
-22681.4796728116	-4.4804937216856e-06\\
-22675.6195824686	2.49666407026223e-05\\
-22669.7594921255	9.12032237855432e-05\\
-22663.8994017824	0.000192668293299023\\
-22658.0393114394	0.000326969267657327\\
-22652.1792210963	0.000490938362023845\\
-22646.3191307533	0.000680707305269361\\
-22640.4590404102	0.000891798595481356\\
-22634.5989500671	0.00111923114136709\\
-22628.7388597241	0.00135763779703005\\
-22622.878769381	0.00160139201688716\\
-22617.018679038	0.00184474064196413\\
-22611.1585886949	0.00208193968385947\\
-22605.2984983518	0.00230738990181894\\
-22599.4384080088	0.00251576897278676\\
-22593.5783176657	0.00270215713492345\\
-22587.7182273227	0.00286215333808297\\
-22581.8581369796	0.00299197915987841\\
-22575.9980466366	0.00308856803303492\\
-22570.1379562935	0.00314963767746706\\
-22564.2778659504	0.00317374402608486\\
-22558.4177756074	0.00316031537020093\\
-22552.5576852643	0.00310966591729985\\
-22546.6975949213	0.00302298843953884\\
-22540.8375045782	0.00290232618492463\\
-22534.9774142351	0.00275052471302078\\
-22529.1173238921	0.00257116478977409\\
-22523.257233549	0.0023684779246202\\
-22517.397143206	0.0021472465417992\\
-22511.5370528629	0.00191269114091845\\
-22505.6769625198	0.00167034710918789\\
-22499.8168721768	0.00142593409216422\\
-22493.9567818337	0.00118522100549985\\
-22488.0966914907	0.000953889873499597\\
-22482.2366011476	0.000737401708231853\\
-22476.3765108045	0.000540867594888604\\
-22470.5164204615	0.00036892802650677\\
-22464.6563301184	0.000225643336778932\\
-22458.7962397754	0.000114397817855907\\
-22452.9361494323	3.78197872575473e-05\\
-22447.0760590892	-2.28050819703234e-06\\
-22441.2159687462	-4.95368442179901e-06\\
-22435.3558784031	2.98663034621566e-05\\
-22429.4957880601	0.00010136053232084\\
-22423.635697717	0.000207844369965308\\
-22417.7756073739	0.000346807168967215\\
-22411.9155170309	0.000514971460086131\\
-22406.0554266878	0.000708370249733787\\
-22400.1953363448	0.000922440598519648\\
-22394.3352460017	0.00115213127346555\\
-22388.4751556586	0.00139202193415246\\
-22382.6150653156	0.00163645104061069\\
-22376.7549749725	0.00187964946473694\\
-22370.8948846295	0.00211587665201912\\
-22365.0347942864	0.00233955612011781\\
-22359.1747039434	0.00254540709595185\\
-22353.3146136003	0.00272856918415062\\
-22347.4545232572	0.00288471712361161\\
-22341.5944329142	0.00301016292294484\\
-22335.7343425711	0.00310194296288637\\
-22329.8742522281	0.00315788800870175\\
-22324.014161885	0.00317667447853186\\
-22318.1540715419	0.00315785575616815\\
-22312.2939811989	0.00310187280745876\\
-22306.4338908558	0.00301004384776905\\
-22300.5738005128	0.00288453330298151\\
-22294.7137101697	0.00272830079464606\\
-22288.8536198266	0.0025450313515061\\
-22282.9935294836	0.00233904849334019\\
-22277.1334391405	0.00211521223675625\\
-22271.2733487975	0.00187880442921946\\
-22265.4132584544	0.00163540411647749\\
-22259.5531681113	0.00139075588434417\\
-22253.6930777683	0.00115063428147733\\
-22247.8329874252	0.000920707523120472\\
-22241.9728970822	0.000706403692488533\\
-22236.1128067391	0.000512782598102907\\
-22230.252716396	0.000344416312183844\\
-22224.392626053	0.000205281210359837\\
-22218.5325357099	9.86640620266708e-05\\
-22212.6724453669	2.7084389079859e-05\\
-22206.8123550238	-7.76507293368395e-06\\
-22200.9522646807	-5.05840432979276e-06\\
-22195.0921743377	3.51439251248846e-05\\
-22189.2320839946	0.000111896402403045\\
-22183.3719936516	0.00022339070732694\\
-22177.5119033085	0.000366998315679557\\
-22171.6518129654	0.000539332465649766\\
-22165.7917226224	0.000736328026782848\\
-22159.9316322793	0.000953337388016618\\
-22154.0715419363	0.0011852401029903\\
-22148.2114515932	0.00142656370606698\\
-22142.3513612502	0.001671612848364\\
-22136.4912709071	0.00191460370676974\\
-22130.631180564	0.00214980049387174\\
-22124.771090221	0.00237165084697883\\
-22118.9109998779	0.00257491690069267\\
-22113.0509095349	0.00275479894871231\\
-22107.1908191918	0.00290704877566157\\
-22101.3307288487	0.00302806998259028\\
-22095.4706385057	0.00311500293750803\\
-22089.6105481626	0.00316579234372959\\
-22083.7504578196	0.00317923583013728\\
-22077.8903674765	0.00315501241427375\\
-22072.0302771334	0.003093690164508\\
-22066.1701867904	0.00299671287874913\\
-22060.3100964473	0.00286636609161833\\
-22054.4500061043	0.00270572321065645\\
-22048.5899157612	0.00251857305087442\\
-22042.7298254181	0.00230933047621894\\
-22036.8697350751	0.00208293225547603\\
-22031.009644732	0.00184472058897597\\
-22025.149554389	0.00160031705445061\\
-22019.2894640459	0.00135548994578969\\
-22013.4293737028	0.00111601813571697\\
-22007.5692833598	0.000887554674900925\\
-22001.7091930167	0.000675493347235683\\
-21995.8491026737	0.000484841331038173\\
-21989.9890123306	0.000320100972490042\\
-21984.1289219875	0.000185163462725372\\
-21978.2688316445	8.32169292912925e-05\\
-21972.4087413014	1.66711127725148e-05\\
-21966.5486509584	-1.28995918351125e-05\\
-21960.6885606153	-4.79338274669962e-06\\
-21954.8284702722	4.08022931965327e-05\\
-21948.9683799292	0.000122815072568025\\
-21943.1082895861	0.000239312894783654\\
-21937.2481992431	0.000387549515836936\\
-21931.3881089	0.000564029244210892\\
-21925.5280185569	0.000764589372995192\\
-21919.6679282139	0.000984498364458058\\
-21913.8078378708	0.00121856747127369\\
-21907.9477475278	0.00146127316119429\\
-21902.0876571847	0.00170688745674643\\
-21896.2275668417	0.00194961311443229\\
-21890.3674764986	0.00218372045324131\\
-21884.5073861555	0.00240368260310614\\
-21878.6472958125	0.00260430598086248\\
-21872.7872054694	0.00278085291357972\\
-21866.9271151264	0.00292915351416795\\
-21861.0670247833	0.00304570416747892\\
-21855.2069344402	0.00312775030068174\\
-21849.3468440972	0.0031733514830158\\
-21843.4867537541	0.00318142731597603\\
-21837.6266634111	0.00315178302898263\\
-21831.766573068	0.00308511417400602\\
-21825.9064827249	0.00298299030619131\\
-21820.0463923819	0.00284781803319105\\
-21814.1863020388	0.00268278430328859\\
-21808.3262116958	0.00249178126828024\\
-21802.4661213527	0.00227931449321152\\
-21796.6060310096	0.00205039667687175\\
-21790.7459406666	0.00181042939062635\\
-21784.8858503235	0.00156507562502382\\
-21779.0257599805	0.0013201261518781\\
-21773.1656696374	0.0010813628551111\\
-21767.3055792943	0.000854422255905918\\
-21761.4454889513	0.000644662453455298\\
-21755.5853986082	0.000457036622334211\\
-21749.7253082652	0.000295976052946832\\
-21743.8652179221	0.000165285496909142\\
-21738.0051275791	6.8053288864956e-05\\
-21732.145037236	6.57836785429514e-06\\
-21726.2849468929	-1.76840772969702e-05\\
-21720.4248565499	-4.15705347971521e-06\\
-21714.5647662068	4.68445156296738e-05\\
-21708.7046758638	0.000134121119197111\\
-21702.8445855207	0.00025561686900393\\
-21696.9844951776	0.000408467925958908\\
-21691.1244048346	0.000589070002689916\\
-21685.2643144915	0.000793163351206407\\
-21679.4042241485	0.00101593323165793\\
-21673.5441338054	0.00125212349272012\\
-21667.6840434623	0.00149616058417645\\
-21661.8239531193	0.0017422850759966\\
-21655.9638627762	0.0019846875803692\\
-21650.1037724332	0.00221764586926778\\
-21644.2436820901	0.00243565995089277\\
-21638.383591747	0.00263358191700662\\
-21632.523501404	0.00280673749533747\\
-21626.6634110609	0.00295103643708406\\
-21620.8033207179	0.00306306913297706\\
-21614.9432303748	0.00314018717540569\\
-21609.0831400317	0.00318056596340672\\
-21603.2230496887	0.00318324787040729\\
-21597.3629593456	0.0031481649537589\\
-21591.5028690026	0.00307614066740391\\
-21585.6427786595	0.00296887053453593\\
-21579.7826883164	0.00282888223403618\\
-21573.9225979734	0.00265947604038275\\
-21568.0625076303	0.00246464701992132\\
-21562.2024172873	0.00224899081828803\\
-21556.3423269442	0.00201759526007206\\
-21550.4822366011	0.00177592031809374\\
-21544.6221462581	0.00152966928354407\\
-21538.762055915	0.00128465417689415\\
-21532.901965572	0.00104665857552332\\
-21527.0418752289	0.000821301095556858\\
-21521.1817848858	0.000613902750050097\\
-21515.3216945428	0.00042936131505069\\
-21509.4616041997	0.000272035669565267\\
-21503.6015138567	0.000145642840780946\\
-21497.7414235136	5.31701861906142e-05\\
-21491.8813331706	-3.19521269947748e-06\\
-21486.0212428275	-2.21182763003273e-05\\
-21480.1611524844	-3.1475492157126e-06\\
-21474.3010621414	5.32740301088044e-05\\
-21468.4409717983	0.000145819469240684\\
-21462.5808814553	0.000272308929783336\\
-21456.7207911122	0.000429761069651542\\
-21450.8607007691	0.000614463310369162\\
-21445.0006104261	0.000822059373266662\\
-21439.140520083	0.00104765202075956\\
-21433.28042974	0.00128591857954535\\
-21427.4203393969	0.00153123652077801\\
-21421.5602490538	0.0017778161343922\\
-21415.7001587108	0.00201983716683753\\
-21409.8400683677	0.00225158619786335\\
-21403.9799780247	0.00246759151377172\\
-21398.1198876816	0.00266275229325185\\
-21392.2597973385	0.00283245905618398\\
-21386.3997069955	0.00297270253069246\\
-21380.5396166524	0.00308016836798435\\
-21374.6795263094	0.00315231546681139\\
-21368.8194359663	0.00318743605714714\\
-21362.9593456232	0.00318469612222795\\
-21357.0992552802	0.00314415520234353\\
-21351.2391649371	0.00306676510985293\\
-21345.3790745941	0.00295434758344741\\
-21339.518984251	0.00280955140584624\\
-21333.6588939079	0.00263578999428017\\
-21327.7988035649	0.00243716093445656\\
-21321.9387132218	0.00221834935470449\\
-21316.0786228788	0.00198451741860817\\
-21310.2185325357	0.00174118254285544\\
-21304.3584421926	0.00149408721301988\\
-21298.4983518496	0.00124906346904916\\
-21292.6382615065	0.00101189525817956\\
-21286.7781711635	0.000788181904068069\\
-21280.9180808204	0.000583205914801907\\
-21275.0579904774	0.000401808250545425\\
-21269.1979001343	0.000248273996051639\\
-21263.3378097912	0.000126231137913003\\
-21257.4777194482	3.85648376552169e-05\\
-21251.6176291051	-1.26507737700991e-05\\
-21245.7575387621	-2.62016677201703e-05\\
-21239.897448419	-1.76269488871369e-06\\
-21234.0373580759	6.00946144661507e-05\\
-21228.1772677329	0.000157915414167216\\
-21222.3171773898	0.000289395757199251\\
-21216.4570870468	0.000451436857593549\\
-21210.5969967037	0.00064021812117586\\
-21204.7369063606	0.000851287224124466\\
-21198.8768160176	0.00107966511550895\\
-21193.0167256745	0.00131996346766029\\
-21187.1566353315	0.00156651180374478\\
-21181.2965449884	0.00181349130351671\\
-21175.4364546453	0.00205507212966843\\
-21169.5763643023	0.00228555103410204\\
-21163.7162739592	0.00249948599611892\\
-21157.8561836162	0.00269182471422838\\
-21151.9960932731	0.00285802391758802\\
-21146.13600293	0.00299415667996184\\
-21140.275912587	0.00309700520125194\\
-21134.4158222439	0.00316413686430596\\
-21128.5557319009	0.00319396176997816\\
-21122.6956415578	0.00318577038886933\\
-21116.8355512147	0.00313975043883375\\
-21110.9754608717	0.00305698258643913\\
-21105.1153705286	0.00293941507080181\\
-21099.2552801856	0.00278981784564513\\
-21093.3951898425	0.00261171731907342\\
-21087.5350994994	0.00240931322825304\\
-21081.6750091564	0.00218737960872977\\
-21075.8149188133	0.00195115219337911\\
-21069.9548284703	0.00170620489645645\\
-21064.0947381272	0.00145831829705226\\
-21058.2346477842	0.00121334322453697\\
-21052.3745574411	0.000977062665328763\\
-21046.514467098	0.000755055250273588\\
-21040.654376755	0.000552563544909791\\
-21034.7942864119	0.000374370252199059\\
-21028.9341960689	0.000224685251098001\\
-21023.0741057258	0.000107046138705955\\
-21017.2140153827	2.4234625778998e-05\\
-21011.3539250397	-2.1789238725292e-05\\
-21005.4938346966	-2.99334587582342e-05\\
nan	nan\\
-20993.7736540105	6.7310398293246e-05\\
-20987.9135636674	0.000170414625237528\\
-20982.0534733244	0.000306884430186792\\
-20976.1933829813	0.000473503608921444\\
-20970.3332926383	0.000666343797465135\\
-20964.4732022952	0.000880857087363718\\
-20958.6131119521	0.00111198327904755\\
-20952.7530216091	0.00135426924396446\\
-20946.892931266	0.00160199757994134\\
-20941.032840923	0.00184932152459245\\
-20935.1727505799	0.00209040294326255\\
-20929.3126602368	0.00231955013498935\\
-20923.4525698938	0.00253135220365635\\
-20917.5924795507	0.00272080682239459\\
-20911.7323892077	0.00288343837409154\\
-20905.8722988646	0.00301540367833292\\
-20900.0122085215	0.00311358280714593\\
-20894.1521181785	0.00317565284344941\\
-20888.2920278354	0.00320014283863114\\
-20882.4319374924	0.00318646866930303\\
-20876.5718471493	0.00313494696671604\\
-20870.7117568062	0.00304678778741879\\
-20864.8516664632	0.00292406619401881\\
-20858.9915761201	0.00276967341411233\\
-20853.1314857771	0.00258724872640782\\
-20847.271395434	0.00238109367907378\\
-20841.411305091	0.00215607066185803\\
-20835.5512147479	0.00191748822394495\\
-20829.6911244048	0.00167097584175276\\
-20823.8310340618	0.0014223510915718\\
-20817.9709437187	0.00117748236034001\\
-20812.1108533757	0.000942150334932957\\
-20806.2507630326	0.000721911538721629\\
-20800.3906726895	0.000521967137027511\\
-20794.5305823465	0.000347040108862511\\
-20788.6704920034	0.000201263685613291\\
-20782.8104016604	8.80836917304688e-05\\
-20776.9503113173	1.01770948662542e-05\\
-20771.0902209742	-3.0611309356979e-05\\
-20765.2301306312	-3.33125804714552e-05\\
-20759.3700402881	2.14335015914561e-06\\
-20753.5099499451	7.49258758179725e-05\\
-20747.649859602	0.000183323170174404\\
-20741.7897692589	0.000324782446600355\\
-20735.9296789159	0.000495970074259087\\
-20730.0695885728	0.000692850135347828\\
-20724.2094982298	0.000910779572327836\\
-20718.3494078867	0.00114461768209432\\
-20712.4893175436	0.00138884737503024\\
-20706.6292272006	0.00163770533880346\\
-20700.7691368575	0.00188531803599131\\
-20694.9090465145	0.00212584032652471\\
-20689.0489561714	0.00235359344330247\\
-20683.1888658283	0.00256319906432044\\
-20677.3287754853	0.00274970631572453\\
-20671.4686851422	0.00290870870720147\\
-20665.6085947992	0.00303644823754358\\
-20659.7485044561	0.00312990421121617\\
-20653.8884141131	0.00318686466657632\\
-20648.02832377	0.003205978726969\\
-20642.1682334269	0.00318678863591547\\
-20636.3081430839	0.00312974071630488\\
-20630.4480527408	0.00303617499155613\\
-20624.5879623978	0.00290829371006961\\
-20618.7278720547	0.0027491095121929\\
-20612.8677817116	0.00256237445937183\\
-20607.0076913686	0.00235249159792822\\
-20601.1476010255	0.00212441114117562\\
-20595.2875106825	0.00188351371818666\\
-20589.4274203394	0.00163548344248802\\
-20583.5673299963	0.0013861737950614\\
-20577.7072396533	0.00114146948591178\\
-20571.8471493102	0.000907147554284746\\
-20565.9870589672	0.00068874098589843\\
-20560.1269686241	0.000491408066685149\\
-20554.266878281	0.000319810557931723\\
-20548.406787938	0.000178003569884647\\
-20542.5466975949	6.93397352561368e-05\\
-20536.6866072519	-3.61005289553254e-06\\
-20530.8265169088	-3.91174651187153e-05\\
-20524.9664265657	-3.63376823612654e-05\\
-20519.1063362227	4.67050443822302e-06\\
-20513.2462458796	8.2945920120351e-05\\
-20507.3861555366	0.000196647531327036\\
-20501.5260651935	0.000343097744896139\\
-20495.6659748504	0.000518845460369104\\
-20489.8058845074	0.000719747391844812\\
-20483.9457941643	0.000941065743037567\\
-20478.0857038213	0.00117757993293203\\
-20472.2256134782	0.00142370973740285\\
-20466.3655231351	0.00167364694233901\\
-20460.5054327921	0.0019214924022144\\
-20454.645342449	0.00216139526995276\\
-20448.785252106	0.00238769111223189\\
-20442.9251617629	0.00259503564985561\\
-20437.0650714198	0.00277853096561077\\
-20431.2049810768	0.00293384119935575\\
-20425.3448907337	0.00305729499728184\\
-20419.4848003907	0.0031459722950326\\
-20413.6247100476	0.00319777338318522\\
-20407.7646197046	0.00321146862150076\\
-20401.9045293615	0.00318672762535123\\
-20396.0444390184	0.00312412723097415\\
-20390.1843486754	0.00302513804812941\\
-20384.3242583323	0.00289208991347358\\
-20378.4641679893	0.00272811705584226\\
-20372.6040776462	0.00253708426446659\\
-20366.7439873031	0.00232349579893282\\
-20360.8838969601	0.00209238918780621\\
-20355.023806617	0.00184921642009313\\
-20349.163716274	0.00159971533113736\\
-20343.3036259309	0.00134977421724108\\
-20337.4435355878	0.00110529287324029\\
-20331.5834452448	0.000872043332353873\\
-20325.7233549017	0.000655533595495984\\
-20319.8632645587	0.000460877566986869\\
-20314.0031742156	0.000292674268021953\\
-20308.1430838725	0.000154899180587666\\
-20302.2829935295	5.0810288928693e-05\\
-20296.4229031864	-1.71289608970521e-05\\
-20290.5628128434	-4.73079616430469e-05\\
-20284.7027225003	-3.90071259829308e-05\\
-20278.8426321572	7.58495697702066e-06\\
-20272.9825418142	9.13757987774044e-05\\
-20267.1224514711	0.000210394625440384\\
-20261.2623611281	0.000361838727574037\\
-20255.402270785	0.000542139456705485\\
-20249.5421804419	0.000747046313836898\\
-20243.6820900989	0.000971727148917599\\
-20237.8219997558	0.00121088210922227\\
-20231.9619094128	0.00145886864972079\\
-20226.1018190697	0.00170983465669336\\
-20220.2417287267	0.0019578565441099\\
-20214.3816383836	0.00219707906397015\\
-20208.5215480405	0.00242185353095053\\
-20202.6614576975	0.00262687119822895\\
-20196.8013673544	0.00280728863553673\\
-20190.9412770114	0.00295884214821608\\
-20185.0811866683	0.00307794853466199\\
-20179.2210963252	0.00316178980087591\\
-20173.3610059822	0.00320837982946083\\
-20167.5009156391	0.00321661142590622\\
-20161.6408252961	0.00318628262782528\\
-20155.780734953	0.00311810165184447\\
-20149.9206446099	0.00301367035727035\\
-20144.0605542669	0.00287544661260889\\
-20138.2004639238	0.0027066864489546\\
-20132.3403735808	0.00251136736170224\\
-20126.4802832377	0.00229409456716184\\
-20120.6201928946	0.00205999242345\\
-20114.7601025516	0.00181458357556377\\
-20108.9000122085	0.00156365867478947\\
-20103.0399218655	0.00131313974585914\\
-20097.1798315224	0.00106894042550905\\
-20091.3197411793	0.000836826370871963\\
-20085.4596508363	0.000622279132573751\\
-20079.5995604932	0.000430366706511684\\
-20073.7394701502	0.000265623821086051\\
-20067.8793798071	0.000131944787599777\\
-20062.019289464	3.24914455642226e-05\\
-20056.159199121	-3.03816211365579e-05\\
-20050.2991087779	-5.51828285233295e-05\\
-20044.4390184349	-4.1318977528854e-05\\
-20038.5789280918	1.08905596460874e-05\\
-20032.7188377487	0.000100221191034979\\
-20026.8587474057	0.000224571825151499\\
-20020.9986570626	0.000381014286463652\\
-20015.1385667196	0.000565862263930092\\
-20009.2784763765	0.000774758169158772\\
-20003.4183860334	0.00100277585766498\\
-19997.5582956904	0.00124453679189974\\
-19991.6982053473	0.0014943369067839\\
-19985.8381150043	0.00174628118556276\\
-19979.9780246612	0.00199442277095076\\
-19974.1179343182	0.00223290332879933\\
-19968.2578439751	0.00245609135167552\\
-19962.397753632	0.00265871513697177\\
-19956.537663289	0.00283598730003151\\
-19950.6775729459	0.00298371788102505\\
-19944.8174826029	0.00309841337376431\\
-19938.9573922598	0.00317735933591647\\
-19933.0973019167	0.00321868462698279\\
-19927.2372115737	0.00322140575420832\\
-19921.3771212306	0.00318545027524024\\
-19915.5170308876	0.00311165870084367\\
-19909.6569405445	0.00300176484823732\\
-19903.7968502014	0.00285835510397872\\
-19897.9367598584	0.00268480755391253\\
-19892.0766695153	0.00248521241217128\\
-19886.2165791723	0.00226427562413767\\
-19880.3564888292	0.00202720791431624\\
-19874.4963984861	0.00177960189582683\\
-19868.6363081431	0.00152730013897931\\
-19862.7762178	0.00127625731164266\\
-19856.916127457	0.00103239964388067\\
-19851.0560371139	0.000801485033775356\\
-19845.1959467708	0.000588967096438991\\
-19839.3358564278	0.000399866366243662\\
-19833.4757660847	0.000238651693966943\\
-19827.6156757417	0.000109134640544155\\
-19821.7555853986	1.43793630239751e-05\\
-19815.8954950555	-4.33698768218519e-05\\
-19810.0354047125	-6.27418663285009e-05\\
-19804.1753143694	-4.32709993287928e-05\\
-19798.3152240264	1.45915358610645e-05\\
-19792.4551336833	0.000109488206609079\\
-19786.5950433403	0.000239186982342586\\
-19780.7349529972	0.000400633830089154\\
-19774.8748626541	0.000590024624599525\\
-19769.0147723111	0.000802894779875644\\
-19763.154681968	0.00103422449039854\\
-19757.294591625	0.0012785571011974\\
-19751.4345012819	0.00153012781571654\\
-19745.5744109388	0.00178299970573255\\
-19739.7143205958	0.00203120381427884\\
-19733.8542302527	0.00226888004578474\\
-19727.9941399097	0.00249041551766442\\
-19722.1340495666	0.00269057710737819\\
-19716.2739592235	0.00286463506467592\\
-19710.4138688805	0.00300847476949458\\
-19704.5537785374	0.00311869399495326\\
-19698.6936881944	0.0031926833759882\\
-19692.8335978513	0.00322868818089843\\
-19686.9735075082	0.00322584992333993\\
-19681.1134171652	0.00318422682797309\\
-19675.2533268221	0.0031047926621266\\
-19669.3932364791	0.00298941395599161\\
-19663.533146136	0.00284080614424192\\
-19657.6730557929	0.00266246965982387\\
-19651.8129654499	0.00245860748341024\\
-19645.9528751068	0.00223402609081966\\
-19640.0927847638	0.00199402213284423\\
-19634.2326944207	0.00174425751862979\\
-19628.3726040776	0.0014906258491366\\
-19622.5125137346	0.00123911335097204\\
-19616.6524233915	0.000995657592319269\\
-19610.7923330485	0.000766007315066848\\
-19604.9322427054	0.000555586692197041\\
-19599.0721523623	0.00036936721547464\\
-19593.2120620193	0.000211750239203305\\
-19587.3519716762	8.64629550001872e-05\\
-19581.4918813332	-3.5297438650051e-06\\
-19575.6317909901	-5.60954245490624e-05\\
-19569.7717006471	-6.99846428289845e-05\\
-19563.911610304	-4.48606402078331e-05\\
-19558.0515199609	1.869249586499e-05\\
-19552.1914296179	0.000119183406251231\\
-19546.3313392748	0.00025424845348466\\
-19540.4712489318	0.000420707313197848\\
-19534.6111585887	0.000614637856179421\\
-19528.7510682456	0.000831468558037372\\
-19522.8909779026	0.00106608625917946\\
-19517.0308875595	0.00131295673521743\\
-19511.1707972165	0.00156625523458023\\
-19505.3107068734	0.00182000390460429\\
-19499.4506165303	0.00206821286369505\\
-19493.5905261873	0.00230502159057735\\
-19487.7304358442	0.00252483729308094\\
-19481.8703455012	0.00272246699002114\\
-19476.0102551581	0.00289324018648735\\
-19470.150164815	0.00303311924507743\\
-19464.290074472	0.00313879484447176\\
-19458.4299841289	0.00320776426904855\\
-19452.5698937859	0.00323839067706191\\
-19446.7098034428	0.00322994194408079\\
-19440.8497130997	0.00318260816004539\\
-19434.9896227567	0.00309749736159997\\
-19429.1295324136	0.00297660959547147\\
-19423.2694420706	0.00282278991991434\\
-19417.4093517275	0.002639661448472\\
-19411.5492613844	0.0024315400118711\\
-19405.6891710414	0.00220333244790632\\
-19399.8290806983	0.00196042091644812\\
-19393.9689903553	0.0017085359665373\\
-19388.1089000122	0.00145362134940467\\
-19382.2488096691	0.00120169376626511\\
-19376.3887193261	0.000958700860340368\\
-19370.528628983	0.000730380804708862\\
-19364.66853864	0.000522126800690162\\
-19358.8084482969	0.000338859686494413\\
-19352.9483579539	0.000184911665046173\\
-19347.0882676108	6.39238983051875e-05\\
-19341.2281772677	-2.12396113792094e-05\\
-19335.3680869247	-6.85598160528999e-05\\
-19329.5079965816	-7.69104883892955e-05\\
-19323.6479062386	-4.60850246296429e-05\\
-19317.7878158955	2.3198453580295e-05\\
-19311.9277255524	0.000129313824193651\\
-19306.0676352094	0.000269765127162093\\
-19300.2075448663	0.000441245268687923\\
-19294.3474545233	0.000639713886660138\\
-19288.4873641802	0.00086049254402621\\
-19282.6272738371	0.0010983750073472\\
-19276.7671834941	0.00134775001116313\\
-19270.907093151	0.00160273361346353\\
-19265.047002808	0.00185730802041086\\
-19259.1869124649	0.00210546360503902\\
-19253.3268221218	0.00234134076831363\\
-19247.4667317788	0.00255936829414784\\
-19241.6066414357	0.00275439493145204\\
-19235.7465510927	0.00292181109561528\\
-19229.8864607496	0.00305765781455272\\
-19224.0263704065	0.00315872034386454\\
-19218.1662800635	0.00322260423823502\\
-19212.3061897204	0.00324779207856016\\
-19206.4460993774	0.00323367951106862\\
-19200.5860090343	0.00318058974261593\\
-19194.7259186912	0.00308976614442579\\
-19188.8658283482	0.00296334313373728\\
-19183.0057380051	0.00280429601460806\\
-19177.1456476621	0.00261637095717588\\
-19171.285557319	0.00240399676273911\\
-19165.4254669759	0.00217218049308533\\
-19159.5653766329	0.0019263894235084\\
-19153.7052862898	0.00167242210244262\\
-19147.8451959468	0.0014162715587859\\
-19141.9851056037	0.00116398388359977\\
-19136.1250152607	0.00092151552327798\\
-19130.2649249176	0.000694592652493112\\
-19124.4048345745	0.000488575946729301\\
-19118.5447442315	0.000308333947919791\\
-19112.6846538884	0.000158128014557549\\
-19106.8245635454	4.15115748822309e-05\\
-19100.9644732023	-3.87539349948621e-05\\
-19095.1043828592	-8.07644595356537e-05\\
-19089.2442925162	-8.35184904910721e-05\\
-19083.3842021731	-4.69409405442995e-05\\
-19077.5241118301	2.81148451499311e-05\\
-19071.664021487	0.000139886992642582\\
-19065.8039311439	0.000285746453934804\\
-19059.9438408009	0.00046225884215683\\
-19054.0837504578	0.000665265292906495\\
-19048.2236601148	0.000889980447925988\\
-19042.3635697717	0.00113110525273996\\
-19036.5034794286	0.00138295190946436\\
-19030.6433890856	0.00163957803855868\\
-19024.7832987425	0.00189492688498233\\
-19018.9232083995	0.00214297026078808\\
-19013.0631180564	0.00237785085088631\\
-19007.2030277133	0.0025940205224798\\
-19001.3429373703	0.00278637137247749\\
-18995.4828470272	0.00295035641776533\\
-18989.6227566842	0.00308209707643156\\
-18983.7626663411	0.00317847489959295\\
-18977.902575998	0.00323720538448551\\
-18972.042485655	0.00325689212143291\\
-18966.1823953119	0.00323705999149077\\
-18960.3223049689	0.00317816662590684\\
-18954.4622146258	0.00308159185063536\\
-18948.6021242827	0.00294960535948395\\
-18942.7420339397	0.00278531337337056\\
-18936.8819435966	0.00259258553886001\\
-18931.0218532536	0.0023759637864148\\
-18925.1617629105	0.00214055529512317\\
-18919.3016725674	0.00189191208585699\\
-18913.4415822244	0.00163590008177391\\
-18907.5814918813	0.00137856072409014\\
-18901.7214015383	0.00112596840758451\\
-18895.8613111952	0.00088408709999427\\
-18890.0012208522	0.000658629529578646\\
-18884.1411305091	0.000454922265381188\\
-18878.281040166	0.00027777987656005\\
-18872.420949823	0.00013139114364504\\
-18866.5608594799	1.92200110015848e-05\\
-18860.7007691369	-5.60763775477899e-05\\
-18854.8406787938	-9.27106205544781e-05\\
-18848.9805884507	-8.98074873347542e-05\\
-18843.1204981077	-4.74248258504103e-05\\
-18837.2604077646	3.3447549295294e-05\\
-18831.4003174216	0.000150910968479387\\
-18825.5402270785	0.000302202478736994\\
-18819.6801367354	0.000483759829222655\\
-18813.8200463924	0.000691305342053608\\
-18807.9599560493	0.000919946693924104\\
-18802.0998657063	0.00116429223417216\\
-18796.2397753632	0.00141857812115274\\
-18790.3796850201	0.0016768042792222\\
-18784.5195946771	0.00193287596938492\\
-18778.659504334	0.00218074763341678\\
-18772.799413991	0.00241456561663699\\
-18766.9393236479	0.0026288064000839\\
-18761.0792333048	0.00281840707790009\\
-18755.2191429618	0.00297888499786085\\
-18749.3590526187	0.00310644373782046\\
-18743.4989622757	0.00319806291285394\\
-18737.6388719326	0.0032515696888833\\
-18731.7787815895	0.00326569030914807\\
-18725.9186912465	0.0032400804121195\\
-18720.0586009034	0.00317533341923403\\
-18714.1985105604	0.00307296678813353\\
-18708.3384202173	0.00293538644997722\\
-18702.4783298743	0.00276583026410804\\
-18696.6182395312	0.00256829181876632\\
-18690.7581491881	0.002347426371543\\
-18684.8980588451	0.00210844114424735\\
-18679.037968502	0.00185697255779655\\
-18673.177878159	0.00159895330114962\\
-18667.3177878159	0.00134047236945581\\
-18661.4576974728	0.00108763137293946\\
-18655.5976071298	0.000846400507645332\\
-18649.7375167867	0.000622477587468703\\
-18643.8774264437	0.000421153466158602\\
-18638.0173361006	0.000247187027568191\\
-18632.1572457575	0.000104692697937652\\
-18626.2971554145	-2.956861103588e-06\\
-18620.4370650714	-7.32105775005867e-05\\
-18614.5769747284	-0.000104399422478881\\
-18608.7168843853	-9.57760604676606e-05\\
-18602.8567940422	-4.75327533708477e-05\\
-18596.9967036992	3.92029096404471e-05\\
-18591.1366133561	0.000162394362353604\\
-18585.2765230131	0.000319143876072086\\
-18579.41643267	0.000505760715971023\\
-18573.5563423269	0.000717848036250807\\
-18567.6962519839	0.000950406468306214\\
-18561.8361616408	0.00119795196145568\\
-18555.9760712978	0.00145464509873329\\
-18550.1159809547	0.00171442883853689\\
-18544.2558906116	0.0019711714330792\\
-18538.3958002686	0.00221881115160648\\
-18532.5357099255	0.00245149939270799\\
-18526.6756195825	0.00266373880694001\\
-18520.8155292394	0.00285051316814547\\
-18514.9554388963	0.00300740592507665\\
-18509.0953485533	0.00313070463210016\\
-18503.2352582102	0.00321748878943333\\
-18497.3751678672	0.00326569901448781\\
-18491.5150775241	0.00327418590659551\\
-18485.654987181	0.00324273744517434\\
-18479.794896838	0.00317208426875\\
-18473.9348064949	0.00306388270339354\\
-18468.0747161519	0.00292067593503878\\
-18462.2146258088	0.00274583423561655\\
-18456.3545354658	0.00254347564764448\\
-18450.4944451227	0.00231836899430911\\
-18444.6343547796	0.00207582149875784\\
-18438.7742644366	0.00182155366113274\\
-18432.9141740935	0.00156156434323884\\
-18427.0540837505	0.00130198924221325\\
-18421.1939934074	0.00104895609258157\\
-18415.3339030643	0.000808440013231233\\
-18409.4738127213	0.000586122414200106\\
-18403.6137223782	0.000387256794770521\\
-18397.7536320352	0.000216544602743797\\
-18391.8935416921	7.80240883375843e-05\\
-18386.033451349	-2.50252198610244e-05\\
-18380.173361006	-9.01601573534626e-05\\
-18374.3132706629	-0.000115831846490102\\
-18368.4531803199	-0.000101422526372739\\
-18362.5930899768	-4.72604142253817e-05\\
-18356.7329996337	4.53877591612407e-05\\
-18350.8729092907	0.000174346370379179\\
-18345.0128189476	0.000336581988190763\\
-18339.1527286046	0.000528274722729892\\
-18333.2926382615	0.000744908160966237\\
-18327.4325479184	0.000981375771129694\\
-18321.5724575754	0.00123210126927024\\
-18315.7123672323	0.00149117011098425\\
-18309.8522768893	0.00175246900762923\\
-18303.9921865462	0.00200983017628503\\
-18298.1320962031	0.00225717691971971\\
-18292.2720058601	0.00248866710038391\\
-18286.411915517	0.00269883112095715\\
-18280.551825174	0.00288270115277156\\
-18274.6917348309	0.00303592855912138\\
-18268.8316444879	0.00315488673757002\\
-18262.9715541448	0.00323675694993573\\
-18257.1114638017	0.00327959510792965\\
-18251.2513734587	0.00328237793273344\\
-18245.3912831156	0.00324502739241868\\
-18239.5311927726	0.00316841283337592\\
-18233.6711024295	0.00305433074932621\\
-18227.8110120864	0.00290546265778688\\
-18221.9509217434	0.00272531207219984\\
-18216.0908314003	0.00251812205101473\\
-18210.2307410573	0.00228877526361677\\
-18204.3706507142	0.00204267892724433\\
-18198.5105603711	0.00178563732594442\\
-18192.6504700281	0.00152371491722304\\
-18186.790379685	0.00126309325460047\\
-18180.930289342	0.00100992510187001\\
-18175.0701989989	0.000770189181675641\\
-18169.2101086558	0.000549548987395443\\
-18163.3500183128	0.000353218992327754\\
-18157.4899279697	0.00018584141676491\\
-18151.6298376267	5.13764650934753e-05\\
-18145.7697472836	-4.69913718307363e-05\\
-18139.9096569405	-0.000106928732249591\\
-18134.0495665975	-0.000127008731112874\\
-18128.1894762544	-0.000106744926945176\\
-18122.3293859114	-4.6603099474696e-05\\
-18116.4692955683	5.20094469453597e-05\\
-18110.6092052252	0.000186776808663174\\
-18104.7491148822	0.000354528866579199\\
-18098.8890245391	0.000551315851514117\\
-18093.0289341961	0.000772501337282419\\
-18087.168843853	0.00101287147215145\\
-18081.3087535099	0.00126675787536547\\
-18075.4486631669	0.00152817130194673\\
-18069.5885728238	0.00179094292414862\\
-18063.7284824808	0.00204886989662629\\
-18057.8683921377	0.00229586177103383\\
-18052.0083017947	0.00252608430355498\\
-18046.1482114516	0.00273409726046745\\
-18040.2881211085	0.00291498296643236\\
-18034.4280307655	0.00306446255839037\\
-18028.5679404224	0.00317899719677166\\
-18022.7078500794	0.00325587184018219\\
-18016.8477597363	0.00329325960037216\\
-18010.9876693932	0.00329026515224825\\
-18005.1275790502	0.00324694616736504\\
-17999.2674887071	0.00316431225816866\\
-17993.4073983641	0.00304430145036805\\
-17987.547308021	0.00288973473190572\\
-17981.6872176779	0.00270424974416834\\
-17975.8271273349	0.0024922151740684\\
-17969.9670369918	0.00225862786163019\\
-17964.1069466488	0.00200899504622712\\
-17958.2468563057	0.00174920452650565\\
-17952.3867659626	0.00148538579471838\\
-17946.5266756196	0.00122376542092549\\
-17940.6665852765	0.000970520098580096\\
-17934.8064949335	0.000731630819955942\\
-17928.9464045904	0.000512741623865245\\
-17923.0863142473	0.000319026251571868\\
-17917.2262239043	0.000155065861107603\\
-17911.3661335612	2.47406902264884e-05\\
-17905.5060432182	-6.88617701286397e-05\\
-17899.6459528751	-0.000123519918815147\\
-17893.785862532	-0.000137930771273336\\
-17887.925772189	-0.000111741018777535\\
-17882.0656818459	-4.55556798860067e-05\\
-17876.2055915029	5.90758674657438e-05\\
-17870.3455011598	0.000199696150929326\\
-17864.4854108167	0.000372997317038402\\
-17858.6253204737	0.000574898937514028\\
-17852.7652301306	0.000800644078546268\\
-17846.9051397876	0.00104491137126863\\
-17841.0450494445	0.00130194044328263\\
-17835.1849591015	0.00156566775460915\\
-17829.3248687584	0.0018298696352209\\
-17823.4647784153	0.00208830914989589\\
-17817.6046880723	0.00233488332481833\\
-17811.7445977292	0.00256376726082426\\
-17805.8845073862	0.00276955173020013\\
-17800.0244170431	0.0029473710069704\\
-17794.1643267	0.00309301790968213\\
-17788.304236357	0.00320304333721315\\
-17782.4441460139	0.00327483794213662\\
-17776.5840556709	0.00330669400845668\\
-17770.7239653278	0.00329784606602108\\
-17764.8638749847	0.0032484892758537\\
-17759.0037846417	0.0031597751448728\\
-17753.1436942986	0.00303378466396183\\
-17747.2836039556	0.00287347949522077\\
-17741.4235136125	0.00268263235409453\\
-17735.5634232694	0.00246573822198747\\
-17729.7033329264	0.00222790847940802\\
-17723.8432425833	0.00197475045254338\\
-17717.9831522403	0.00171223521203553\\
-17712.1230618972	0.00144655674036633\\
-17706.2629715541	0.00118398578974072\\
-17700.4028812111	0.00093072187814373\\
-17694.542790868	0.000692746916920503\\
-17688.682700525	0.000475683925448585\\
-17682.8226101819	0.000284664169936385\\
-17676.9625198388	0.000124205865406549\\
-17671.1024294958	-1.89269188114981e-06\\
-17665.2423391527	-9.06430339274512e-05\\
-17659.3822488097	-0.000139937344294651\\
-17653.5221584666	-0.000148598516845465\\
-17647.6620681235	-0.000116408261157179\\
-17641.8019777805	-4.41125836659883e-05\\
-17635.9418874374	6.65954925906442e-05\\
-17630.0817970944	0.000213115569519962\\
-17624.2217067513	0.000392000948702619\\
-17618.3616164083	0.00059903970494476\\
-17612.5015260652	0.000829353851739215\\
-17606.6414357221	0.00107751426396015\\
-17600.7813453791	0.00133766865039903\\
-17594.921255036	0.0016036795596945\\
-17589.061164693	0.00186926916557996\\
-17583.2010743499	0.00212816741562344\\
-17577.3409840068	0.00237426004794859\\
-17571.4808936638	0.00260173298152088\\
-17565.6208033207	0.0028052096702421\\
-17559.7607129777	0.00297987817644177\\
-17553.9006226346	0.00312160496001904\\
-17548.0405322915	0.00322703269307582\\
-17542.1804419485	0.00329365978488761\\
-17536.3203516054	0.00331989973423141\\
-17530.4602612624	0.00330511890044406\\
-17524.6001709193	0.00324965179432829\\
-17518.7400805762	0.00315479352014854\\
-17512.8799902332	0.00302276953889241\\
-17507.0198998901	0.00285668345889799\\
-17501.1598095471	0.0026604440784261\\
-17495.299719204	0.00243867339507389\\
-17489.4396288609	0.0021965977470681\\
-17483.5795385179	0.0019399246500727\\
-17477.7194481748	0.00167470823173893\\
-17471.8593578318	0.00140720643686414\\
-17465.9992674887	0.00114373337056256\\
-17460.1391771456	0.000890510263788304\\
-17454.2790868026	0.000653518578438651\\
-17448.4189964595	0.00043835872060944\\
-17442.5589061165	0.000250117699026392\\
-17436.6988157734	9.32488559845947e-05\\
-17430.8387254303	-2.85334859471857e-05\\
-17424.9786350873	-0.00011234196916861\\
-17419.1185447442	-0.000156184656030153\\
-17413.2584544012	-0.000159012370673364\\
-17407.3983640581	-0.00012074380267877\\
-17401.5382737151	-4.22677719732014e-05\\
-17395.678183372	7.45774065927705e-05\\
-17389.8180930289	0.000227046980089\\
-17383.9580026859	0.000411554227368581\\
-17378.0979123428	0.000623754827763278\\
-17372.2378219998	0.000858649144152956\\
-17366.3777316567	0.00111070001227798\\
-17360.5176413136	0.00137396326148658\\
-17354.6575509706	0.00164222789021971\\
-17348.7974606275	0.00190916259096835\\
-17342.9373702845	0.00216846516790967\\
-17337.0772799414	0.00241401132129086\\
-17331.2171895983	0.00263999928608769\\
-17325.3570992553	0.0028410869089278\\
-17319.4970089122	0.00301251792512637\\
-17313.6369185692	0.00315023445060586\\
-17307.7768282261	0.00325097302838986\\
-17301.916737883	0.00331234195655322\\
-17296.05664754	0.0033328780652403\\
-17290.1965571969	0.00331208159513595\\
-17284.3364668539	0.00325042834598547\\
-17278.4763765108	0.00314935880011472\\
-17272.6162861677	0.00301124446952437\\
-17266.7561958247	0.00283933225223779\\
-17260.8961054816	0.00263766810384271\\
-17255.0360151386	0.00241100181814975\\
-17249.1759247955	0.00216467515785807\\
-17243.3158344524	0.00190449597017936\\
-17237.4557441094	0.00163660125346903\\
-17231.5956537663	0.00136731240367987\\
-17225.7355634233	0.00110298605450711\\
-17219.8754730802	0.000849864030840129\\
-17214.0153827372	0.000613925957206676\\
-17208.1552923941	0.000400748001395787\\
-17202.295202051	0.000215371090139906\\
-17196.435111708	6.21817112289388e-05\\
-17190.5750213649	-5.51918795326886e-05\\
-17184.7149310219	-0.000133965590637644\\
-17178.8548406788	-0.00017226553134028\\
-17172.9947503357	-0.000169172586033585\\
-17167.1346599927	-0.000124744466343105\\
-17161.2745696496	-4.00147120079053e-05\\
-17155.4144793066	8.30313444355445e-05\\
-17149.5543889635	0.000241503090366643\\
-17143.6942986204	0.000431672533579973\\
-17137.8342082774	0.000649061995683265\\
-17131.9741179343	0.000888549535702041\\
-17126.1140275913	0.00114448962183809\\
-17120.2539372482	0.00141084620853153\\
-17114.3938469051	0.00168133508208871\\
-17108.5337565621	0.00194957211792032\\
-17102.673666219	0.00220922395216201\\
-17096.813575876	0.00245415751154037\\
-17090.9534855329	0.00267858487124884\\
-17085.0933951898	0.00287720001974668\\
-17079.2333048468	0.00304530429881759\\
-17073.3732145037	0.00317891755331364\\
-17067.5131241607	0.0032748723616779\\
-17061.6530338176	0.0033308891162053\\
-17055.7929434745	0.00334563017360636\\
-17049.9328531315	0.00331873178943838\\
-17044.0727627884	0.00325081307430903\\
-17038.2126724454	0.00314346175192286\\
-17032.3525821023	0.00299919704571256\\
-17026.4924917592	0.00282141056221078\\
-17020.6324014162	0.00261428655787328\\
-17014.7723110731	0.00238270346375212\\
-17008.9122207301	0.0021321189855739\\
-17003.052130387	0.00186844148536767\\
-16997.1920400439	0.00159789067534705\\
-16991.3319497009	0.00132685090883612\\
-16985.4718593578	0.0010617205281706\\
-16979.6117690148	0.000808760824820732\\
-16973.7516786717	0.000573948176819946\\
-16967.8915883287	0.000362832855259908\\
-16962.0314979856	0.000180407835448466\\
-16956.1714076425	3.09907134717422e-05\\
-16950.3113172995	-8.18784793205741e-05\\
-16944.4512269564	-0.000155521145550106\\
-16938.5911366134	-0.000188183687877569\\
-16932.7310462703	-0.000179079263506846\\
-16926.8709559272	-0.000128406733021716\\
-16921.0108655842	-3.73463474530113e-05\\
-16915.1507752411	9.19677336562851e-05\\
-16909.2906848981	0.000256497453378197\\
-16903.430594555	0.000452372225917957\\
-16897.5705042119	0.000674979986037294\\
-16891.7104138689	0.000919075777729024\\
-16885.8503235258	0.0011789053255574\\
-16879.9902331828	0.0014483406774797\\
-16874.1301428397	0.00172102472175951\\
-16868.2700524966	0.00199052117006053\\
-16862.4099621536	0.00225046646811791\\
-16856.5498718105	0.00249472004890933\\
-16850.6897814675	0.0027175093805363\\
-16844.8296911244	0.00291356638295905\\
-16838.9696007813	0.00307825198979473\\
-16833.1095104383	0.00320766590989548\\
-16827.2494200952	0.00329873899231712\\
-16821.3893297522	0.00334930600689151\\
-16815.5292394091	0.00335815711499183\\
-16809.669149066	0.00332506680697236\\
-16803.809058723	0.00325079961408492\\
-16797.9489683799	0.00313709245131778\\
-16792.0888780369	0.00298661399816999\\
-16786.2287876938	0.00280290206770192\\
-16780.3686973508	0.00259028043329555\\
-16774.5086070077	0.00235375706848653\\
-16768.6485166646	0.00209890619477049\\
-16762.7884263216	0.00183173691520315\\
-16756.9283359785	0.00155855152981105\\
-16751.0682456355	0.00128579687304424\\
-16745.2081552924	0.00101991218018234\\
-16739.3480649493	0.000767177072449169\\
-16733.4879746063	0.00053356324939121\\
-16727.6278842632	0.000324593391174183\\
-16721.7677939202	0.000145210604301558\\
-16715.9077035771	-3.38503015062493e-07\\
-16710.047613234	-0.000108604350224984\\
-16704.187522891	-0.000177016138859087\\
-16698.3274325479	-0.00020394289452982\\
-16692.4673422049	-0.000188732347219367\\
-16686.6072518618	-0.000131726723126663\\
-16680.7471615187	-3.42550660060854e-05\\
-16674.8870711757	0.000101397740207191\\
-16669.0269808326	0.000272044525574439\\
-16663.1668904896	0.000473670710044766\\
-16657.3068001465	0.000701528742090925\\
-16651.4467098034	0.000950249878674834\\
-16645.5866194604	0.00121397067475159\\
-16639.7265291173	0.00148647120241221\\
-16633.8664387743	0.0017613217414424\\
-16628.0063484312	0.00203203448196694\\
-16622.1462580881	0.00229221665998899\\
-16616.2861677451	0.00253572151148781\\
-16610.426077402	0.00275679348066917\\
-16604.565987059	0.00295020425211911\\
-16598.7058967159	0.00311137639197449\\
-16592.8458063728	0.00323649167421322\\
-16586.9857160298	0.00332258152877746\\
-16581.1256256867	0.00336759746885296\\
-16575.2655353437	0.00337045982670993\\
-16569.4054450006	0.00333108363872912\\
-16563.5453546575	0.00325038105919412\\
-16557.6852643145	0.00313024023616807\\
-16551.8251739714	0.00297348113850494\\
-16545.9650836284	0.00278378936732268\\
-16540.1049932853	0.00256562950545911\\
-16534.2449029423	0.00232414004163137\\
-16528.3848125992	0.00206501234264672\\
-16522.5247222561	0.00179435652390311\\
-16516.6646319131	0.0015185573789269\\
-16510.80454157	0.0012441237653532\\
-16504.944451227	0.000977534999517035\\
-16499.0843608839	0.000725087884969616\\
-16493.2242705408	0.000492747986061979\\
-16487.3641801978	0.000286008659417088\\
-16481.5040898547	0.000109761174163993\\
-16475.6439995117	-3.18210083328037e-05\\
-16469.7839091686	-0.000135381057646192\\
-16463.9238188255	-0.000198458360456623\\
-16458.0637284825	-0.000219546982932227\\
-16452.2036381394	-0.000198131620406222\\
-16446.3435477964	-0.000134700176321897\\
-16440.4834574533	-3.07326637172032e-05\\
-16434.6233671102	0.000111333318655286\\
-16428.7632767672	0.000288159730377807\\
-16422.9031864241	0.000495586514094176\\
-16417.0430960811	0.000728729458506226\\
-16411.183005738	0.000982095197569767\\
-16405.3229153949	0.00124971063835898\\
-16399.4628250519	0.0015252637683092\\
-16393.6027347088	0.00180225252264397\\
-16387.7426443658	0.00207413820114466\\
-16381.8825540227	0.00233449981442621\\
-16376.0224636796	0.00257718571678646\\
-16370.1623733366	0.00279645894443526\\
-16364.3022829935	0.00298713282620357\\
-16358.4421926505	0.00314469366049456\\
-16352.5821023074	0.00326540755775076\\
-16346.7220119643	0.0033464089189049\\
-16340.8619216213	0.00338576845392555\\
-16335.0018312782	0.00338253912540536\\
-16329.1417409352	0.00333677892398197\\
-16323.2816505921	0.00324954992734402\\
-16317.4215602491	0.00312289365547601\\
-16311.561469906	0.00295978329357303\\
-16305.7013795629	0.0027640539006836\\
-16299.8412892199	0.00254031224206561\\
-16293.9811988768	0.00229382836536761\\
-16288.1211085338	0.0020304114721701\\
-16282.2610181907	0.00175627300844687\\
-16276.4009278476	0.00147788020042007\\
-16270.5408375046	0.00120180348947253\\
-16264.6807471615	0.000934561464678312\\
-16258.8206568185	0.000682466952817851\\
-16252.9605664754	0.000451477899609607\\
-16247.1004761323	0.000247056564347153\\
-16241.2403857893	7.40403555731246e-05\\
-16235.3802954462	-6.34726380712965e-05\\
-16229.5202051032	-0.000162220713228812\\
-16223.6601147601	-0.000219855914514827\\
-16217.800024417	-0.000234999859707407\\
-16211.939934074	-0.000207276700249463\\
-16206.0798437309	-0.000137322429085251\\
-16200.2197533879	-2.67703058012217e-05\\
-16194.3596630448	0.000121787267190864\\
-16188.4995727017	0.000304859527714513\\
-16182.6394823587	0.000518139371084741\\
-16176.7793920156	0.000756604674649038\\
-16170.9193016726	0.0010146365460701\\
-16165.0592113295	0.00128615171126053\\
-16159.1991209864	0.00156474592294067\\
-16153.3390306434	0.00184384500913134\\
-16147.4789403003	0.00211685999924417\\
-16141.6188499573	0.00237734266739566\\
-16135.7587596142	0.00261913782143336\\
-16129.8986692712	0.00283652874096806\\
-16124.0385789281	0.00302437232816796\\
-16118.178488585	0.0031782207765661\\
-16112.318398242	0.00329442687901623\\
-16106.4583078989	0.00337023048262889\\
-16100.5982175559	0.0034038240405048\\
-16094.7381272128	0.00339439570411013\\
-16088.8780368697	0.00334214892908698\\
-16083.0179465267	0.00324829812088709\\
-16077.1578561836	0.00311504041315398\\
-16071.2977658406	0.00294550423353621\\
-16065.4376754975	0.00274367586196693\\
-16059.5775851544	0.00251430570430218\\
-16053.7174948114	0.00226279648579603\\
-16047.8574044683	0.00199507599508508\\
-16041.9973141253	0.00171745737655641\\
-16036.1372237822	0.00143649026307191\\
-16030.2771334391	0.00115880625959995\\
-16024.4170430961	0.000890962422844709\\
-16018.556952753	0.000639286430811113\\
-16012.69686241	0.000409727098285845\\
-16006.8367720669	0.000207713769378205\\
-16000.9766817238	3.80279104543498e-05\\
-15995.1165913808	-9.53100597599009e-05\\
-15989.2565010377	-0.000189136024533497\\
-15983.3964106947	-0.000241217251211042\\
-15977.5363203516	-0.000250305519499976\\
-15971.6762300085	-0.000216167031927556\\
-15965.8161396655	-0.000139588389904397\\
-15959.9560493224	-2.23584835662776e-05\\
-15954.0959589794	0.000132773287949975\\
-15948.2358686363	0.000322161490170081\\
-15942.3757782932	0.000541350309089475\\
-15936.5156879502	0.000785178376656384\\
-15930.6555976071	0.00104790029991056\\
-15924.7955072641	0.00132332203250974\\
-15918.935416921	0.0016049468990927\\
-15913.0753265779	0.00188612883015765\\
-15907.2152362349	0.00216022919334625\\
-15901.3551458918	0.00242077352038266\\
-15895.4950555488	0.00266160442981069\\
-15889.6349652057	0.00287702713371492\\
-15883.7748748627	0.0030619440901945\\
-15877.9147845196	0.00321197561753751\\
-15872.0546941765	0.00332356361703518\\
-15866.1946038335	0.00339405594718373\\
-15860.3345134904	0.0034217694496319\\
-15854.4744231474	0.00340603012836392\\
-15848.6143328043	0.00334718952380148\\
-15842.7542424612	0.00324661688391024\\
-15836.8941521182	0.00310666730637738\\
-15831.0340617751	0.00293062659267823\\
-15825.1739714321	0.00272263410518938\\
-15819.313881089	0.00248758543855506\\
-15813.4537907459	0.00223101719329131\\
-15807.5937004029	0.00195897656392919\\
-15801.7336100598	0.00167787881306794\\
-15795.8735197168	0.00139435599053467\\
-15790.0134293737	0.00111510046472146\\
-15784.1533390306	0.000846706957782417\\
-15778.2932486876	0.000595516812678381\\
-15772.4331583445	0.000367468169909713\\
-15766.5730680015	0.000167955593274787\\
-15760.7129776584	1.70246302675327e-06\\
-15754.8528873153	-0.000127350845271362\\
-15748.9927969723	-0.00021614034906893\\
-15743.1327066292	-0.000262551201128199\\
-15737.2726162862	-0.000265468058943033\\
-15731.4125259431	-0.000224801881802328\\
-15725.5524356	-0.000141492511864545\\
-15719.692345257	-1.74869670384477e-05\\
-15713.8322549139	0.000144306053242435\\
-15707.9721645709	0.000340084386486619\\
-15702.1120742278	0.000565241750036063\\
-15696.2519838848	0.000814476109168386\\
-15690.3918935417	0.00108191452100489\\
-15684.5318031986	0.00136125151483173\\
-15678.6717128556	0.00164589774830985\\
-15672.8116225125	0.00192913543522503\\
-15666.9515321695	0.00220427687848571\\
-15661.0914418264	0.0024648223675902\\
-15655.2313514833	0.00270461371255668\\
-15649.3712611403	0.00291797978783883\\
-15643.5111707972	0.00309987064662115\\
-15637.6510804542	0.00324597703373892\\
-15631.7909901111	0.00335283246943439\\
-15625.930899768	0.00341789548531318\\
-15620.070809425	0.00343961006209267\\
-15614.2107190819	0.00341744283179781\\
-15608.3506287389	0.00335189615523155\\
-15602.4905383958	0.0032444967545569\\
-15596.6304480527	0.0030977601577299\\
-15590.7703577097	0.00291513178258332\\
-15584.9102673666	0.00270090604026635\\
-15579.0501770236	0.00246012535751445\\
-15573.1900866805	0.00219846149160917\\
-15567.3299963374	0.00192208193174272\\
-15561.4699059944	0.00163750453345362\\
-15555.6098156513	0.00135144381209565\\
-15549.7497253083	0.00107065252003196\\
-15543.8896349652	0.000801762245226182\\
-15538.0295446221	0.000551126793784171\\
-15532.1694542791	0.000324672055058073\\
-15526.309363936	0.000127755896776441\\
-15520.449273593	-3.49585975311358e-05\\
-15514.5891832499	-0.000159613549862391\\
-15508.7290929068	-0.000243247753188978\\
-15502.8690025638	-0.000283867012633818\\
-15497.0089122207	-0.000280491691646587\\
-15491.1488218777	-0.000233180329688468\\
-15485.2887315346	-0.000143028762349835\\
-15479.4286411916	-1.21447528169376e-05\\
-15473.5685508485	0.000156401278303213\\
-15467.7084605054	0.000358648273218747\\
-15461.8483701624	0.000589837618094773\\
-15455.9882798193	0.000844525097901928\\
-15450.1281894763	0.00111670909117421\\
-15444.2680991332	0.00139997198631395\\
-15438.4080087901	0.00168763148727556\\
-15432.5479184471	0.00197289824158901\\
-15426.687828104	0.00224903607266765\\
-15420.827737761	0.0025095210349352\\
-15414.9676474179	0.0027481955362655\\
-15409.1075570748	0.00295941388710672\\
-15403.2474667318	0.00313817583574399\\
-15397.3873763887	0.00328024493180674\\
-15391.5272860457	0.00338224891582324\\
-15385.6671957026	0.00344175975666725\\
-15379.8071053595	0.00345735143676508\\
-15373.9470150165	0.00342863411070367\\
-15368.0869246734	0.00335626381843553\\
-15362.2268343304	0.00324192751253394\\
-15356.3667439873	0.00308830374023165\\
-15350.5066536442	0.00289899989653999\\
-15344.6465633012	0.00267846751867429\\
-15338.7864729581	0.00243189760977451\\
-15332.9263826151	0.00216509845392739\\
-15327.066292272	0.00188435879806841\\
-15321.2062019289	0.00159629962321549\\
-15315.3461115859	0.00130771799906446\\
-15309.4860212428	0.00102542670394675\\
-15303.6259308998	0.000756093394389925\\
-15297.7658405567	0.000506083120695707\\
-15291.9057502136	0.000281307908173864\\
-15286.0456598706	8.7086958430274e-05\\
-15280.1855695275	-7.19792236303063e-05\\
-15274.3254791845	-0.000192117798626628\\
-15268.4653888414	-0.000270473076442667\\
-15262.6052984984	-0.000305174392639081\\
-15256.7452081553	-0.00029538076439031\\
-15250.8851178122	-0.000241301260081821\\
-15245.0250274692	-0.000144190589545165\\
-15239.1649371261	-6.32000663148055e-06\\
-15233.3048467831	0.000169075801332674\\
-15227.44475644	0.000377874595476299\\
-15221.5846660969	0.000615163458716024\\
-15215.7245757539	0.000875354384118947\\
-15209.8644854108	0.00115231585894787\\
-15204.0043950678	0.00143951734607849\\
-15198.1443047247	0.00173018325839858\\
-15192.2842143816	0.00201745279600168\\
-15186.4241240386	0.00229454187596013\\
-15180.5640336955	0.00255490333257139\\
-15174.7039433525	0.00279238160564078\\
-15168.8438530094	0.00300135826212091\\
-15162.9837626663	0.00317688491076186\\
-15157.1236723233	0.00331480036596573\\
-15151.2635819802	0.00341182928680324\\
-15145.4034916372	0.00346565995275887\\
-15139.5434012941	0.00347499933022859\\
-15133.683310951	0.00343960411759045\\
-15127.823220608	0.00336028702397862\\
-15121.9631302649	0.00323889812103693\\
-15116.1030399219	0.00307828169475268\\
-15110.2429495788	0.00288220960418514\\
-15104.3828592357	0.00265529270753584\\
-15098.5227688927	0.00240287243599037\\
-15092.6626785496	0.00213089506466031\\
-15086.8025882066	0.00184577163966613\\
-15080.9424978635	0.00155422686134492\\
-15075.0824075204	0.00126314048489069\\
-15069.2223171774	0.000979384979148323\\
-15063.3622268343	0.000709663273851925\\
-15057.5021364913	0.000460350426003838\\
-15051.6420461482	0.000237342945100516\\
-15045.7819558052	4.59193383597596e-05\\
-15039.9218654621	-0.000109384826114604\\
-15034.061775119	-0.000224884381242437\\
-15028.201684776	-0.000297832002020913\\
-15022.3415944329	-0.000326483551076439\\
-15016.4815040899	-0.0003101397746323\\
-15010.6214137468	-0.000249163352292922\\
-15004.7613234037	-0.000144970885391611\\
nan	nan\\
-14993.0411427176	0.000182347671641106\\
-14987.1810523746	0.000397786297766314\\
-14981.3209620315	0.00064124656956867\\
-14975.4608716884	0.000906994972593629\\
-14969.6007813454	0.00118876880093558\\
-14963.7406910023	0.0014799237351317\\
-14957.8806006593	0.00177359050624812\\
-14952.0205103162	0.00206283695243711\\
-14946.1604199731	0.00234083164499458\\
-14940.3003296301	0.00260100522197361\\
-14934.440239287	0.00283720561916546\\
-14928.580148944	0.00304384353095849\\
-14922.7200586009	0.00321602466163508\\
-14916.8599682578	0.00334966563817136\\
-14910.9998779148	0.00344159083945087\\
-14905.1397875717	0.00348960784607493\\
-14899.2796972287	0.00349255971797196\\
-14893.4196068856	0.00345035285387688\\
-14887.5595165425	0.00336395976184966\\
-14881.6994261995	0.00323539666232488\\
-14875.8393358564	0.00306767643896439\\
-14869.9792455134	0.00286473803519307\\
-14864.1191551703	0.00263135395085735\\
-14858.2590648272	0.00237301801082623\\
-14852.3989744842	0.00209581604510282\\
-14846.5388841411	0.0018062825240459\\
-14840.6787937981	0.00151124652591381\\
-14834.818703455	0.00121767066731926\\
-14828.958613112	0.00093248679559564\\
-14823.0985227689	0.000662432320091626\\
-14817.2384324258	0.000413891046680078\\
-14811.3783420828	0.000192742275460329\\
-14805.5182517397	4.22172853864268e-06\\
-14799.6581613967	-0.000147202402636761\\
-14793.7980710536	-0.000257935356010565\\
-14787.9379807105	-0.000325341134040087\\
-14782.0778903675	-0.000347805249640168\\
-14776.2178000244	-0.000324773389527042\\
-14770.3577096814	-0.000256765069308663\\
-14764.4976193383	-0.000145361944579666\\
-14758.6375289952	6.82895969455063e-06\\
-14752.7774386522	0.000196236246832203\\
-14746.9173483091	0.000418407946166413\\
-14741.0572579661	0.000668116144800087\\
-14735.197167623	0.00093947999445889\\
-14729.3370772799	0.00122610419944735\\
-14723.4769869369	0.00152122972457019\\
-14717.6168965938	0.00181789317173546\\
-14711.7568062508	0.00210909106740961\\
-14705.8967159077	0.00238794518504862\\
-14700.0366255646	0.00264786499999256\\
-14694.1765352216	0.00288270344049986\\
-14688.3164448785	0.00308690225354372\\
-14682.4563545355	0.00325562354887474\\
-14676.5962641924	0.00338486440746752\\
-14670.7361738493	0.00347155184018167\\
-14664.8760835063	0.00351361584370634\\
-14659.0159931632	0.00351003881731542\\
-14653.1559028202	0.00346088016106414\\
-14647.2958124771	0.00336727546096407\\
-14641.4357221341	0.00323141026656279\\
-14635.575631791	0.00305646906617973\\
-14629.7155414479	0.00284656065096442\\
-14623.8554511049	0.00260662161640171\\
-14617.9953607618	0.002342300268085\\
-14612.1352704188	0.00205982366122552\\
-14606.2751800757	0.00176585090394756\\
-14600.4150897326	0.00146731617983447\\
-14594.5549993896	0.00117126519018534\\
-14588.6949090465	0.00088468887352642\\
-14582.8348187035	0.000614358326495787\\
-14576.9747283604	0.000366664823968551\\
-14571.1146380173	0.000147468718030301\\
-14565.2545476743	-3.80392120851743e-05\\
-14559.3944573312	-0.000185460678968508\\
-14553.5343669882	-0.000291294164323109\\
-14547.6742766451	-0.000353018082501412\\
-14541.814186302	-0.000369150855255968\\
-14535.954095959	-0.000339286466673613\\
-14530.0940056159	-0.000264104645316144\\
-14524.2339152729	-0.000145355419131996\\
-14518.3738249298	1.4181606485196e-05\\
-14512.5137345867	0.000210762300133952\\
-14506.6536442437	0.00043976586314819\\
-14500.7935539006	0.000695803434123047\\
-14494.9334635576	0.000972844886872625\\
-14489.0733732145	0.00126436083839635\\
-14483.2132828714	0.0015634765229513\\
-14477.3531925284	0.0018631339060022\\
-14471.4931021853	0.00215625821548399\\
-14465.6330118423	0.00243592496150328\\
-14459.7729214992	0.00269552350139673\\
-14453.9128311561	0.00292891328709512\\
-14448.0527408131	0.00313056910176388\\
-14442.19265047	0.00329571185067635\\
-14436.332560127	0.00342042181080717\\
-14430.4724697839	0.00350173165535322\\
-14424.6123794408	0.0035376970457357\\
-14418.7522890978	0.00352744311202012\\
-14412.8921987547	0.0034711857108244\\
-14407.0321084117	0.00337022694442108\\
-14401.1720180686	0.00322692503256452\\
-14395.3119277256	0.00304463923383036\\
-14389.4518373825	0.00282765110239682\\
-14383.5917470394	0.00258106392615789\\
-14377.7316566964	0.00231068270777798\\
-14371.8715663533	0.00202287751120831\\
-14366.0114760103	0.00172443339015354\\
-14360.1513856672	0.00142239043416452\\
-14354.2912953241	0.00112387770259079\\
-14348.4312049811	0.000835944963785444\\
-14342.571114638	0.00056539621042337\\
-14336.711024295	0.000318628882536111\\
-14330.8509339519	0.000101482597034424\\
-14324.9908436088	-8.08990394676799e-05\\
-14319.1307532658	-0.000224190264832648\\
-14313.2706629227	-0.000324985756845948\\
-14307.4105725797	-0.000380881556884313\\
-14301.5504822366	-0.000390532398900328\\
-14295.6903918935	-0.000353684076786514\\
-14289.8303015505	-0.000271180071694326\\
-14283.9702112074	-0.000144942268043085\\
-14278.1101208644	2.20738013458808e-05\\
-14272.2500305213	0.000225948139063632\\
-14266.3899401782	0.000461888276574824\\
-14260.5298498352	0.000724341918675397\\
-14254.6697594921	0.00100712759141242\\
-14248.8096691491	0.00130358021957292\\
-14242.949578806	0.00160670820536757\\
-14237.0894884629	0.00190935830668996\\
-14231.2293981199	0.0022043844268812\\
-14225.3693077768	0.00248481633338656\\
-14219.5092174338	0.00274402432222415\\
-14213.6491270907	0.00297587593821246\\
-14207.7890367477	0.00317488104669863\\
-14201.9289464046	0.00333632182494585\\
-14196.0688560615	0.00345636459554653\\
-14190.2087657185	0.00353215085120681\\
-14184.3486753754	0.00356186530974095\\
-14178.4885850324	0.00354477937927039\\
-14172.6284946893	0.00348126899325907\\
-14166.7684043462	0.00337280637915283\\
-14160.9083140032	0.00322192594002392\\
-14155.0482236601	0.00303216503936423\\
-14149.1881333171	0.00280798107254935\\
-14143.328042974	0.00255464676876176\\
-14137.4679526309	0.00227812618235113\\
-14131.6078622879	0.00198493429025324\\
-14125.7477719448	0.00168198350024101\\
-14119.8876816018	0.0013764206862526\\
-14114.0275912587	0.00107545859238555\\
-14108.1675009156	0.000786205582809815\\
-14102.3074105726	0.000515497755686738\\
-14096.4473202295	0.000269737386289097\\
-14090.5872298865	5.47415169383387e-05\\
-14084.7271395434	-0.000124395725499868\\
-14078.8670492003	-0.000263423826120395\\
-14073.0069588573	-0.000359036732912303\\
-14067.1468685142	-0.000408951469478573\\
-14061.2867781712	-0.000411962640454991\\
-14055.4266878281	-0.000367971528561726\\
-14049.566597485	-0.000277989081314912\\
-14043.706507142	-0.000144112701387935\\
-14037.8464167989	3.05226269723719e-05\\
-14031.9863264559	0.000241817736857821\\
-14026.1262361128	0.000484805484690877\\
-14020.2661457697	0.000753767505555156\\
-14014.4060554267	0.00104236877370208\\
-14008.5459650836	0.00134380680201195\\
-14002.6858747406	0.00165097196672921\\
-13996.8257843975	0.00195661517922525\\
-13990.9656940544	0.0022535189504021\\
-13985.1056037114	0.00253466781147139\\
-13979.2455133683	0.0027934140670323\\
-13973.3854230253	0.00302363496483255\\
-13967.5253326822	0.00321987756595162\\
-13961.6652423392	0.00337748788853582\\
-13955.8051519961	0.00349272126603635\\
-13949.945061653	0.00356283130393929\\
-13944.08497131	0.00358613532134955\\
-13938.2248809669	0.00356205471872728\\
-13932.3647906239	0.00349112930359026\\
-13926.5047002808	0.00337500522006096\\
-13920.6446099377	0.00321639675151773\\
-13914.7845195947	0.00301902288247949\\
-13908.9244292516	0.00278752010166614\\
-13903.0643389086	0.00252733349123196\\
-13897.2042485655	0.00224458865951919\\
-13891.3441582224	0.00194594752989359\\
-13885.4840678794	0.00163845137996679\\
-13879.6239775363	0.00132935482955634\\
-13873.7638871933	0.00102595469092932\\
-13867.9037968502	0.000735417718974825\\
-13862.0437065071	0.000464611327181464\\
-13856.1836161641	0.000219941267876601\\
-13850.323525821	7.20011301820675e-06\\
-13844.463435478	-0.000168569880621201\\
-13838.6033451349	-0.000303196275538598\\
-13832.7432547918	-0.000393475494789959\\
-13826.8831644488	-0.000437249049686191\\
-13821.0230741057	-0.000433455140392547\\
-13815.1629837627	-0.000382154396046635\\
-13809.3028934196	-0.000284529130962334\\
-13803.4428030765	-0.000142856118191634\\
-13797.5827127335	3.9546492979655e-05\\
-13791.7226223904	0.000258396878264007\\
-13785.8625320474	0.000508550039020138\\
-13780.0024417043	0.00078411874361199\\
-13774.1423513612	0.0010786120667953\\
-13768.2822610182	0.00138508826722323\\
-13762.4221706751	0.0016963184025553\\
-13756.5620803321	0.00200495682623152\\
-13750.701989989	0.00230371454466459\\
-13744.841899646	0.00258553134405858\\
-13738.9818093029	0.00284374262236742\\
-13733.1217189598	0.00307223698416493\\
-13727.2616286168	0.00326560087258786\\
-13721.4015382737	0.0034192468150607\\
-13715.5414479307	0.00352952224543625\\
-13709.6813575876	0.00359379632114949\\
-13703.8212672445	0.00361052267204781\\
-13697.9611769015	0.00357927658441986\\
-13692.1010865584	0.0035007657267381\\
-13686.2409962154	0.00337681414709833\\
-13680.3809058723	0.00321031990360969\\
-13674.5208155292	0.00300518731189572\\
-13668.6607251862	0.00276623539278202\\
-13662.8006348431	0.002499084667487\\
-13656.9405445001	0.00221002495888296\\
-13651.080454157	0.00190586730827287\\
-13645.2203638139	0.00159378349377976\\
-13639.3602734709	0.00128113693133899\\
-13633.5001831278	0.000975308945195455\\
-13627.6400927848	0.000683524506567394\\
-13621.7800024417	0.000412681554150447\\
-13615.9199120986	0.000169187928476629\\
-13610.0598217556	-4.11902254466342e-05\\
-13604.1997314125	-0.000213465000840088\\
-13598.3396410695	-0.000343544984132847\\
-13592.4795507264	-0.000428332418795748\\
-13586.6194603833	-0.000465796970757441\\
-13580.7593700403	-0.00045502433916357\\
-13574.8992796972	-0.000396238548846174\\
-13569.0391893542	-0.00029079738163158\\
-13563.1790990111	-0.000141161037301188\\
-13557.3190086681	4.9165252847688e-05\\
-13551.458918325	0.000275713321537797\\
-13545.5988279819	0.000533156947610412\\
-13539.7387376389	0.000815437062982916\\
-13533.8786472958	0.00111590434142437\\
-13528.0185569528	0.00142747581368767\\
-13522.1584666097	0.00174280182061814\\
-13516.2983762666	0.00205443936895238\\
-13510.4382859236	0.00235502780130478\\
-13504.5781955805	0.00263746263442819\\
-13498.7181052375	0.00289506346023967\\
-13492.8580148944	0.00312173194208246\\
-13486.9979245513	0.00331209616943637\\
-13481.1378342083	0.0034616379546464\\
-13475.2777438652	0.00356680005502438\\
-13469.4176535222	0.00362507077624197\\
-13463.5575631791	0.00363504394500396\\
-13457.697472836	0.00359645281948817\\
-13451.837382493	0.00351017711966958\\
-13445.9772921499	0.00337822299508318\\
-13440.1172018069	0.00320367638486224\\
-13434.2571114638	0.00299063085426809\\
-13428.3970211207	0.00274409159504284\\
-13422.5369307777	0.00246985784069783\\
-13416.6768404346	0.00217438645866533\\
-13410.8167500916	0.0018646399278886\\
-13404.9566597485	0.00154792228059124\\
-13399.0965694054	0.00123170687408411\\
-13393.2364790624	0.000923460053034822\\
-13387.3763887193	0.000630464863146462\\
-13381.5162983763	0.000359648977902956\\
-13375.6562080332	0.000117420903896955\\
-13369.7961176901	-9.04816629876709e-05\\
-13363.9360273471	-0.000259127742386496\\
-13358.075937004	-0.000384510016431121\\
-13352.215846661	-0.000463640045495784\\
-13346.3557563179	-0.000494619490605733\\
-13340.4956659748	-0.000476685645360349\\
-13334.6355756318	-0.000410230185557357\\
-13328.7754852887	-0.000296790676461904\\
-13322.9153949457	-0.000139015020327091\\
-13317.0553046026	5.94003341502795e-05\\
-13311.1952142596	0.000293796978764833\\
-13305.3351239165	0.000558663901163979\\
-13299.4750335734	0.00084776704176819\\
-13293.6149432304	0.00115429600674359\\
-13287.7548528873	0.00147102448452132\\
-13281.8947625443	0.00179048058761845\\
-13276.0346722012	0.00210512310452079\\
-13270.1745818581	0.00240751950441071\\
-13264.3144915151	0.002690521493395\\
-13258.454401172	0.00294743397577846\\
-13252.594310829	0.00317217342681173\\
-13246.7342204859	0.00335941193135101\\
-13240.8741301428	0.00350470347765947\\
-13235.0140397998	0.00360458951335667\\
-13229.1539494567	0.00365668125752385\\
-13223.2938591137	0.00365971681014061\\
-13217.4337687706	0.00361359169415607\\
-13211.5736784275	0.00351936209084715\\
-13205.7135880845	0.00337922067474385\\
-13199.8534977414	0.0031964455997176\\
-13193.9934073984	0.002975323823287\\
-13188.1333170553	0.00272105056194462\\
-13182.2732267122	0.00243960723559769\\
-13176.4131363692	0.00213762076854656\\
-13170.5530460261	0.00182220755591233\\
-13164.6929556831	0.0015008057697849\\
-13158.83286534	0.00118099995578919\\
-13152.9727749969	0.000870342056633228\\
-13147.1126846539	0.000576173085354545\\
-13141.2525943108	0.00030544965918062\\
-13135.3925039678	6.45794925091613e-05\\
-13129.5324136247	-0.000140730260949095\\
-13123.6723232817	-0.000305608227740763\\
-13117.8122329386	-0.000426134392418144\\
-13111.9521425955	-0.000499433291546825\\
-13106.0920522525	-0.000523742608611341\\
-13100.2319619094	-0.000498455533806195\\
-13094.3718715664	-0.000424135870874628\\
-13088.5117812233	-0.000302505515995168\\
-13082.6516908802	-0.000136404585615859\\
-13076.7916005372	7.02748838355391e-05\\
-13070.9315101941	0.000312680116960369\\
-13065.0714198511	0.000585111525206799\\
-13059.211329508	0.000881156703204051\\
-13053.3512391649	0.00119384134555545\\
-13047.4911488219	0.00151579353264106\\
-13041.6310584788	0.00183941751565272\\
-13035.7709681358	0.00215707290440395\\
-13029.9108777927	0.00246125503114236\\
-13024.0507874496	0.00274477223260003\\
-13018.1906971066	0.00300091586296916\\
-13012.3306067635	0.00322361901866571\\
-13006.4705164205	0.00340760021986583\\
-13000.6104260774	0.00354848864624062\\
-12994.7503357343	0.00364292795769067\\
-12988.8902453913	0.00368865623383579\\
-12983.0301550482	0.0036845601292744\\
-12977.1700647052	0.00363070194760178\\
-12971.3099743621	0.00352831897704251\\
-12965.449884019	0.00337979508384935\\
-12959.589793676	0.00318860521565528\\
-12953.7297033329	0.00295923410579319\\
-12947.8696129899	0.00269707108066291\\
-12942.0095226468	0.00240828343697946\\
-12936.1494323037	0.00209967136398446\\
-12930.2893419607	0.00177850782262507\\
-12924.4292516176	0.00145236715229779\\
-12918.5691612746	0.00112894644337695\\
-12912.7090709315	0.000815883888288386\\
-12906.8489805884	0.00052057839741305\\
-12900.9888902454	0.000250014739631456\\
-12895.1287999023	1.05983396940968e-05\\
-12889.2687095593	-0.000191996360520902\\
-12883.4086192162	-0.000352960387325561\\
-12877.5485288732	-0.000468464380026777\\
-12871.6884385301	-0.000535749686200485\\
-12865.828348187	-0.000553194240633414\\
-12859.968257844	-0.000520351655020747\\
-12854.1081675009	-0.00043796257693283\\
-12848.2480771579	-0.000307938030400445\\
-12842.3879868148	-0.00013331511204746\\
-12836.5278964717	8.18139306052683e-05\\
-12830.6678061287	0.000332397582802178\\
-12824.8077157856	0.000612543661835639\\
-12818.9476254426	0.000915657847748411\\
-12813.0875350995	0.00123459888875648\\
-12807.2274447564	0.0015618468286643\\
-12801.3673544134	0.00188968029365477\\
-12795.5072640703	0.00221035865913803\\
-12789.6471737273	0.0025163047989812\\
-12783.7870833842	0.00280028410320546\\
-12777.9269930411	0.00305557553454391\\
-12772.0669026981	0.00327613068007132\\
-12766.206812355	0.00345671703384009\\
-12760.346722012	0.00359304211695406\\
-12754.4866316689	0.00368185549068407\\
-12748.6265413258	0.00372102623895638\\
-12742.7664509828	0.00370959407306639\\
-12736.9063606397	0.00364779283380925\\
-12731.0462702967	0.00353704581610459\\
-12725.1861799536	0.00337993300728251\\
-12719.3260896105	0.00318013099173745\\
-12713.4659992675	0.00294232692201863\\
-12707.6059089244	0.00267210856891523\\
-12701.7458185814	0.002375833029362\\
-12695.8857282383	0.00206047717666717\\
-12690.0256378953	0.00173347337133613\\
-12684.1655475522	0.00140253430026165\\
-12678.3054572091	0.00107547107263958\\
-12672.4453668661	0.000760008862021156\\
-12666.585276523	0.000463604445762696\\
-12660.72518618	0.000193269950974337\\
-12654.8650958369	-4.45930272548612e-05\\
-12649.0050054938	-0.000244345011436138\\
-12643.1449151508	-0.000401242341946952\\
-12637.2848248077	-0.000511549822465491\\
-12631.4247344647	-0.000572629635952271\\
-12625.5646441216	-0.000583004414815551\\
-12619.7045537785	-0.000542392957633614\\
-12613.8444634355	-0.000451717729446316\\
-12607.9843730924	-0.000313083948290628\\
-12602.1242827494	-0.000129730731236164\\
-12596.2641924063	9.40445667776325e-05\\
-12590.4041020632	0.000352987054251422\\
-12584.5440117202	0.000641007685132349\\
-12578.6839213771	0.000951326424858319\\
-12572.8238310341	0.00127663183456044\\
-12566.963740691	0.00160925331756385\\
-12561.1036503479	0.00194134197063009\\
-12555.2435600049	0.00226505577614567\\
-12549.3834696618	0.002572744766284\\
-12543.5233793188	0.00285713178704507\\
-12537.6632889757	0.00311148459138351\\
-12531.8031986326	0.00332977519202992\\
-12525.9431082896	0.00350682270205226\\
-12520.0830179465	0.00363841627921462\\
-12514.2229276035	0.00372141525634715\\
-12508.3628372604	0.00375382407735921\\
-12502.5027469173	0.00373484025126078\\
-12496.6426565743	0.00366487417201057\\
-12490.7825662312	0.00354554031603154\\
-12484.9224758882	0.00337962000402058\\
-12479.0623855451	0.00317099658537821\\
-12473.2022952021	0.00292456455585278\\
-12467.342204859	0.00264611473427888\\
-12461.4821145159	0.00234219819242049\\
-12455.6220241729	0.00201997213471457\\
-12449.7619338298	0.00168703135342849\\
-12443.9018434868	0.00135122922800426\\
-12438.0417531437	0.0010204924872384\\
-12432.1816628006	0.000702634103045424\\
-12426.3215724576	0.00040516873201525\\
-12420.4614821145	0.00013513506430627\\
-12414.6013917715	-0.000101070720897505\\
-12408.7413014284	-0.00029784645285955\\
-12402.8812110853	-0.000450516831741952\\
-12397.0211207423	-0.000555444505329712\\
-12391.1610303992	-0.000610116721304849\\
-12385.3009400562	-0.000613205491169877\\
-12379.4408497131	-0.00056459982558779\\
-12373.58075937	-0.000465409259387132\\
-12367.720669027	-0.00031793856165308\\
-12361.8605786839	-0.000125634206525346\\
-12356.0004883409	0.000106996152381589\\
-12350.1403979978	0.000374489322910161\\
-12344.2803076547	0.000670554855121089\\
-12338.4202173117	0.000988222950008418\\
-12332.5601269686	0.00132000851873012\\
-12326.7000366256	0.00165808753087376\\
-12320.8399462825	0.0019944814973436\\
-12314.9798559394	0.00232124573828104\\
-12309.1197655964	0.00263065699378987\\
-12303.2596752533	0.00291539594744337\\
-12297.3995849103	0.00316872034916426\\
-12291.5394945672	0.00338462464340911\\
-12285.6794042241	0.00355798232289191\\
-12279.8193138811	0.00368466763461515\\
-12273.959223538	0.00376165374866157\\
-12268.099133195	0.00378708505443428\\
-12262.2390428519	0.00376032185772527\\
-12256.3789525089	0.00368195640269436\\
-12250.5188621658	0.00355379981892677\\
-12244.6587718227	0.00337884027913275\\
-12238.7986814797	0.00316117333453156\\
-12232.9385911366	0.00290590605113968\\
-12227.0785007936	0.00261903719041244\\
-12221.2184104505	0.00230731624555055\\
-12215.3583201074	0.00197808464517036\\
-12209.4982297644	0.00163910286021765\\
-12203.6381394213	0.00129836748561333\\
-12197.7780490783	0.000963922607587565\\
-12191.9179587352	0.000643669906154229\\
-12186.0578683921	0.000345181975279903\\
-12180.1977780491	7.55232708467938e-05\\
-12174.337687706	-0.000158917080374117\\
-12168.477597363	-0.000352576654140911\\
-12162.6175070199	-0.000500851698535456\\
-12156.7574166768	-0.000600206569420244\\
-12150.8973263338	-0.000648258030506057\\
-12145.0372359907	-0.000643832408403683\\
-12139.1771456477	-0.000586994232414466\\
-12133.3170553046	-0.000479045661057906\\
-12127.4569649615	-0.000322496686323777\\
-12121.5968746185	-0.000121006796832772\\
-12115.7367842754	0.0001207005447353\\
-12109.8766939324	0.000396948611615346\\
-12104.0166035893	0.000701240715803483\\
-12098.1565132462	0.00102641297382349\\
-12092.2964229032	0.00136480294354482\\
-12086.4363325601	0.00170843016272789\\
-12080.5762422171	0.00204918433577047\\
-12074.716151874	0.00237901673159098\\
-12068.8560615309	0.0026901302757762\\
-12062.9959711879	0.00297516384823449\\
-12057.1358808448	0.00322736643019985\\
-12051.2757905018	0.00344075698087566\\
-12045.4157001587	0.00361026625855431\\
-12039.5556098157	0.00373185722264122\\
-12033.6955194726	0.00380262115837028\\
-12027.8354291295	0.0038208472346446\\
-12021.9753387865	0.00378606383305424\\
-12016.1152484434	0.00369905064929623\\
-12010.2551581004	0.00356182125941683\\
-12004.3950677573	0.00337757653881428\\
-11998.5349774142	0.0031506300112117\\
-11992.6748870712	0.00288630686836549\\
-11986.8147967281	0.00259081902417382\\
-11980.9547063851	0.00227111913406627\\
-11975.094616042	0.00193473701038974\\
-11969.2345256989	0.00158960228214142\\
-11963.3744353559	0.00124385747522906\\
-11957.5143450128	0.00090566591928914\\
-11951.6542546698	0.000583019012334543\\
-11945.7941643267	0.000283547393260629\\
-11939.9340739836	1.43404838178694e-05\\
-11934.0739836406	-0.000218221332815659\\
-11928.2138932975	-0.000408617924373255\\
-11922.3538029545	-0.000552320429514992\\
-11916.4937126114	-0.000645898976044536\\
-11910.6336222683	-0.000687104535664512\\
-11904.7735319253	-0.000674922962167231\\
-11898.9134415822	-0.000609599915083926\\
-11893.0533512392	-0.000492636057467425\\
-11887.1932608961	-0.000326752617424923\\
-11881.333170553	-0.000115828103150199\\
-11875.47308021	0.000135192357265997\\
-11869.6129898669	0.000420412932435065\\
-11863.7528995239	0.00073312554389767\\
-11857.8928091808	0.00106596761078129\\
-11852.0327188377	0.0014110953738765\\
-11846.1726284947	0.00176036871911771\\
-11840.3125381516	0.00210554314583713\\
-11834.4524478086	0.00243846435305652\\
-11828.5923574655	0.00275126085144999\\
-11822.7322671225	0.00303653005164627\\
-11816.8721767794	0.00328751343019484\\
-11811.0120864363	0.00349825662833911\\
-11805.1519960933	0.00366375069161708\\
-11799.2919057502	0.00378005110029099\\
-11793.4318154072	0.00384437176269776\\
-11787.5717250641	0.00385515173155192\\
-11781.711634721	0.00381209304655986\\
-11775.851544378	0.00371616878650527\\
-11769.9914540349	0.00356960111660585\\
-11764.1313636919	0.00337580982533867\\
-11758.2712733488	0.0031393325419885\\
-11752.4111830057	0.00286571849612794\\
-11746.5510926627	0.00256139830547651\\
-11740.6910023196	0.00223353284780547\\
-11734.8309119766	0.0018898447679716\\
-11728.9708216335	0.00153843658425448\\
-11723.1107312904	0.00118759967797026\\
-11717.2506409474	0.000845618668812081\\
-11711.3905506043	0.000520575791086038\\
-11705.5304602613	0.000220159889687517\\
-11699.6703699182	-4.85154505816385e-05\\
-11693.8102795751	-0.000279080340854924\\
-11687.9501892321	-0.000466059601298056\\
-11682.090098889	-0.00060500277171346\\
-11676.230008546	-0.000692590032835402\\
-11670.3699182029	-0.000726711517823789\\
-11664.5098278598	-0.000706518119476864\\
-11658.6497375168	-0.000632442570444803\\
-11652.7896471737	-0.000506190274248848\\
-11646.9295568307	-0.000330700078967078\\
-11641.0694664876	-0.000110075895037949\\
-11635.2093761446	0.000150509252026209\\
-11629.3492858015	0.000444934491304871\\
-11623.4891954584	0.000766274855940216\\
-11617.6291051154	0.0011069641368053\\
-11611.7690147723	0.00145897301096747\\
-11605.9089244293	0.00181399825166154\\
-11600.0488340862	0.00216365856146674\\
-11594.1887437431	0.00249969241031521\\
-11588.3286534001	0.00281415320880766\\
-11582.468563057	0.00309959720636045\\
-11576.608472714	0.00334925967178131\\
-11570.7483823709	0.00355721518658443\\
-11564.8882920278	0.00371851825338741\\
-11559.0282016848	0.00382932088312443\\
-11553.1681113417	0.00388696436550349\\
-11547.3080209987	0.00389004303458434\\
-11541.4479306556	0.00383843850080378\\
-11535.5878403125	0.00373332351644226\\
-11529.7277499695	0.00357713535795049\\
-11523.8676596264	0.00337351932864017\\
-11518.0075692834	0.00312724369047676\\
-11512.1474789403	0.00284408800978299\\
-11506.2873885972	0.00253070753122748\\
-11500.4272982542	0.00219447676199529\\
-11494.5672079111	0.00184331594227765\\
-11488.7071175681	0.00148550448483533\\
-11482.847027225	0.00112948577771722\\
-11476.9869368819	0.000783667951793509\\
-11471.1268465389	0.000456225313723368\\
-11465.2667561958	0.000154905133447802\\
-11459.4066658528	-0.00011315564734382\\
-11453.5465755097	-0.0003415994497413\\
-11447.6864851666	-0.000524998831970757\\
-11441.8263948236	-0.000658985428308061\\
-11435.9663044805	-0.000740353989706235\\
-11430.1062141375	-0.000767139048648679\\
-11424.2461237944	-0.000738662375043392\\
-11418.3860334513	-0.00065555007787772\\
-11412.5259431083	-0.000519718923417286\\
-11406.6658527652	-0.000334332166770206\\
-11400.8057624222	-0.000103725914057236\\
-11394.9456720791	0.000166692271139188\\
-11389.0855817361	0.000470570146491156\\
-11383.225491393	0.000800759982938834\\
-11377.3654010499	0.00114948666634536\\
-11371.5053107069	0.00150853075579784\\
-11365.6452203638	0.00186942218896967\\
-11359.7851300208	0.00222364006997352\\
-11353.9250396777	0.0025628138275367\\
-11348.0649493346	0.00287892099522269\\
-11342.2048589916	0.00316447694066062\\
-11336.3447686485	0.00341271205754877\\
-11330.4846783055	0.00361773222557966\\
-11324.6245879624	0.00377465873547897\\
-11318.7644976193	0.0038797443577932\\
-11312.9044072763	0.00393046279449555\\
-11307.0443169332	0.00392556937828703\\
-11301.1842265902	0.00386513156220703\\
-11295.3241362471	0.00375052845314002\\
-11289.464045904	0.00358441937418049\\
-11283.603955561	0.0033706821712818\\
-11277.7438652179	0.00311432269489252\\
-11271.8837748749	0.00282135756896189\\
-11266.0236845318	0.00249867299264071\\
-11260.1635941887	0.00215386288732921\\
-11254.3035038457	0.00179505019347766\\
-11248.4434135026	0.00143069552166349\\
-11242.5833231596	0.00106939766507679\\
-11236.7232328165	0.000719690676834046\\
-11230.8631424734	0.000389842300133102\\
-11225.0030521304	8.76585120602249e-05\\
-11219.1429617873	-0.000179701198804234\\
-11213.2828714443	-0.000405893449987297\\
-11207.4227811012	-0.000585541459919407\\
-11201.5626907581	-0.000714362849918624\\
-11195.7026004151	-0.000789271715960848\\
-11189.842510072	-0.000808452537875103\\
-11183.982419729	-0.000771404156191048\\
-11178.1223293859	-0.000678952752354025\\
-11172.2622390429	-0.00053323349861615\\
-11166.4021486998	-0.00033764128368209\\
-11160.5420583567	-9.6751650451197e-05\\
-11154.6819680137	0.000183786213358322\\
-11148.8218776706	0.000497381929508431\\
-11142.9617873276	0.000836658723243469\\
-11137.1016969845	0.00119362692170963\\
-11131.2416066414	0.00155987207647026\\
-11125.3815162984	0.00192675328124091\\
-11119.5214259553	0.00228560701108759\\
-11113.6613356123	0.00262795167483101\\
-11107.8012452692	0.00294568805187241\\
-11101.9411549261	0.00323129087659242\\
-11096.0810645831	0.00347798703931719\\
-11090.22097424	0.00367991618436731\\
-11084.360883897	0.0038322698974219\\
-11078.5007935539	0.00393140617704841\\
-11072.6407032108	0.00397493646565578\\
-11066.7806128678	0.0039617831605952\\
-11060.9205225247	0.00389220622127304\\
-11055.0604321817	0.00376779821679077\\
-11049.2003418386	0.00359144790304411\\
-11043.3402514955	0.00336727316123264\\
-11037.4801611525	0.0031005248539906\\
-11031.6200708094	0.00279746384381757\\
-11025.7599804664	0.00246521405312791\\
-11019.8998901233	0.00211159501458328\\
-11014.0397997802	0.00174493784698165\\
-11008.1797094372	0.00137388898730985\\
-11002.3196190941	0.00100720630188527\\
-10996.4595287511	0.000653552384313878\\
-10990.599438408	0.000321289918300479\\
-10984.7393480649	1.82839388482019e-05\\
-10978.8792577219	-0.000248284332806725\\
-10973.0191673788	-0.000472087674542846\\
-10967.1590770358	-0.00064780303624405\\
-10961.2989866927	-0.000771238136418995\\
-10955.4388963497	-0.000839431472165179\\
-10949.5788060066	-0.000850723357193363\\
-10943.7187156635	-0.000804796284483161\\
-10937.8586253205	-0.000702683632908234\\
-10931.9985349774	-0.000546746483764736\\
-10926.1384446344	-0.000340619065828483\\
-10920.2783542913	-8.91240887661348e-05\\
-10914.4182639482	0.000201840063112207\\
-10908.5581736052	0.000525437638647681\\
-10902.6980832621	0.000874056086533502\\
-10896.8379929191	0.00123948510972467\\
-10890.977902576	0.00161310999657833\\
-10885.1178122329	0.0019861146766046\\
-10879.2577218899	0.00234968971526866\\
-10873.3976315468	0.00269524034099343\\
-10867.5375412038	0.00301458959249164\\
-10861.6774508607	0.00330017178559698\\
-10855.8173605176	0.00354521172185579\\
-10849.9572701746	0.00374388539552924\\
-10844.0971798315	0.00389145838628969\\
-10838.2370894885	0.00398439865071868\\
-10832.3769991454	0.00402046102484192\\
-10826.5169088023	0.00399874141766316\\
-10820.6568184593	0.00391969938738359\\
-10814.7967281162	0.0037851485389755\\
-10808.9366377732	0.00359821494007103\\
-10803.0765474301	0.00336326450764631\\
-10797.216457087	0.00308580105198491\\
-10791.356366744	0.00277233735769858\\
-10785.4962764009	0.00243024232186667\\
-10779.6361860579	0.0020675677359539\\
-10773.7760957148	0.00169285878333361\\
-10767.9160053717	0.00131495271158743\\
-10762.0559150287	0.000942770422780636\\
-10756.1958246856	0.000585105895872639\\
-10750.3357343426	0.000250418411753181\\
-10744.4756439995	-5.3367510771315e-05\\
-10738.6155536565	-0.000319049736759018\\
-10732.7554633134	-0.000540319253106635\\
-10726.8953729703	-0.000711909975547922\\
-10721.0352826273	-0.000829724068009946\\
-10715.1751922842	-0.000890929792516086\\
-10709.3151019412	-0.000894029553685494\\
-10703.4550115981	-0.000838896503484505\\
-10697.594921255	-0.000726778812581885\\
-10691.734830912	-0.000560271477435844\\
-10685.8747405689	-0.000343256298508176\\
-10680.0146502259	-8.08114172161412e-05\\
-10674.1545598828	0.000220907480782324\\
-10668.2944695397	0.000554811517343699\\
-10662.4343791967	0.000913045144044303\\
-10656.5742888536	0.00128717092367804\\
-10650.7141985106	0.00166836822485437\\
-10644.8541081675	0.00204764115135044\\
-10638.9940178244	0.00241603080492403\\
-10633.1339274814	0.00276482687384703\\
-10627.2738371383	0.00308577355024298\\
-10621.4137467953	0.00337126490894524\\
-10615.5536564522	0.00361452512434587\\
-10609.6935661091	0.00380976925589593\\
-10603.8334757661	0.00395234078702719\\
-10597.973385423	0.00403882264862621\\
-10592.11329508	0.00406711907964583\\
-10586.2532047369	0.0040365063650835\\
-10580.3931143938	0.00394765122377145\\
-10574.5330240508	0.00380259638053253\\
-10568.6729337077	0.00360471363352503\\
-10562.8128433647	0.00335862549278282\\
-10556.9527530216	0.00307009721166655\\
-10551.0926626786	0.00274590173169188\\
-10545.2325723355	0.0023936607045229\\
-10539.3724819924	0.0020216653216966\\
-10533.5123916494	0.00163868116430806\\
-10527.6523013063	0.00125374166453445\\
-10521.7922109633	0.000875935045694689\\
-10515.9321206202	0.000514189765212761\\
-10510.0720302771	0.000177063525259744\\
-10504.2119399341	-0.000127459163385799\\
-10498.351849591	-0.000392156076208997\\
-10492.491759248	-0.000610738550696399\\
-10486.6316689049	-0.000778000881692425\\
-10480.7715785618	-0.000889944287736941\\
-10474.9114882188	-0.000943872496941892\\
-10469.0513978757	-0.00093845666803155\\
-10463.1913075327	-0.000873768084241219\\
-10457.3312171896	-0.00075127781698924\\
-10451.4711268465	-0.000573823335669707\\
-10445.6110365035	-0.000345542819937063\\
-10439.7509461604	-7.17786946231689e-05\\
-10433.8908558174	0.000241047364741797\\
-10428.0307654743	0.000585585031828312\\
-10422.1706751312	0.000953728003514114\\
-10416.3105847882	0.00133680469212515\\
-10410.4504944451	0.00172578245042778\\
-10404.5904041021	0.00211148052044146\\
-10398.730313759	0.00248478668600041\\
-10392.8702234159	0.00283687251706336\\
-10387.0101330729	0.00315940212184735\\
-10381.1500427298	0.00344472947134819\\
-10375.2899523868	0.00368607962577793\\
-10369.4298620437	0.0038777095678259\\
-10363.5697717006	0.00401504482546074\\
-10357.7096813576	0.00409478863462869\\
-10351.8495910145	0.00411500103698702\\
-10345.9895006715	0.00407514601630697\\
-10340.1294103284	0.00397610552871888\\
-10334.2693199853	0.00382016006354836\\
-10328.4092296423	0.00361093616065831\\
-10322.5491392992	0.00335332209138283\\
-10316.6890489562	0.00305335366302287\\
-10310.8289586131	0.00271807281358105\\
-10304.9688682701	0.00235536230955139\\
-10299.108777927	0.00197376042610718\\
-10293.2486875839	0.00158225996608609\\
-10287.3885972409	0.00119009634824351\\
-10281.5285068978	0.000806529757394639\\
-10275.6684165548	0.000440626495162082\\
-10269.8083262117	0.000101044693077294\\
-10263.9482358686	-0.000204170547546042\\
-10258.0881455256	-0.000467777742389184\\
-10252.2280551825	-0.000683510823126175\\
-10246.3679648395	-0.000846228073398956\\
-10240.5078744964	-0.000952034662309488\\
-10234.6477841533	-0.000998375855852838\\
-10228.7876938103	-0.000984098676312138\\
-10222.9276034672	-0.0009094805221939\\
-10217.0675131242	-0.000776224040201815\\
-10211.2074227811	-0.000587418336644094\\
-10205.347332438	-0.00034746741081545\\
-10199.487242095	-6.1987467481811e-05\\
-10193.6271517519	0.000262324497739244\\
-10187.7670614089	0.000617847765708381\\
-10181.9069710658	0.000996216930582045\\
-10176.0468807227	0.00138851870049675\\
-10170.1867903797	0.00178550183289332\\
-10164.3267000366	0.0021777952602686\\
-10158.4666096936	0.0025561292641822\\
-10152.6065193505	0.0029115544778712\\
-10146.7464290074	0.00323565354394054\\
-10140.8863386644	0.00352074042202343\\
-10135.0262483213	0.00376004262754251\\
-10129.1661579783	0.00394786208149846\\
-10123.3060676352	0.00407971075127527\\
-10117.4459772922	0.00415241785433444\\
-10111.5858869491	0.00416420606459005\\
-10105.725796606	0.00411473489116484\\
-10099.865706263	0.00400511017125079\\
-10094.0056159199	0.00383785941951596\\
-10088.1455255769	0.00361687358081613\\
-10082.2854352338	0.00334731652848656\\
-10076.4253448907	0.00303550441144205\\
-10070.5652545477	0.0026887576695916\\
-10064.7051642046	0.00231522918344578\\
-10058.8450738616	0.00192371259194056\\
-10052.9849835185	0.00152343528432678\\
-10047.1248931754	0.00112384093899665\\
-10041.2648028324	0.000734366733043322\\
-10035.4047124893	0.000364220478364466\\
-10029.5446221463	2.21629464914135e-05\\
-10023.6845318032	-0.00028369947249052\\
-10017.8244414601	-0.00054610687034786\\
-10011.9643511171	-0.000758818128777706\\
-10006.104260774	-0.000916759346183609\\
-10000.244170431	-0.00101614485376583\\
-9994.3840800879	-0.00105456793537535\\
-9988.52398974484	-0.00103105907773339\\
-9982.66389940178	-0.000946110342293003\\
-9976.80380905872	-0.000801665248406276\\
-9970.94371871567	-0.000601074371164269\\
-9965.08362837261	-0.000349017667190359\\
-9959.22353802955	-5.13953281799235e-05\\
-9953.36344768649	0.000284810292908342\\
-9947.50335734343	0.000651698452519214\\
-9941.64326700037	0.00104063564336161\\
-9935.78317665731	0.0014424587168023\\
-9929.92308631425	0.00184769072254565\\
-9924.06299597119	0.00224676438194026\\
-9918.20290562813	0.00263024792603116\\
-9912.34281528507	0.00298906796786863\\
-9906.48272494201	0.00331472414342831\\
-9900.62263459895	0.00359949044392328\\
-9894.76254425589	0.00383659847186479\\
-9888.90245391283	0.00402039827435218\\
-9883.04236356977	0.00414649293305666\\
-9877.18227322671	0.0042118437045222\\
-9871.32218288365	0.00421484319840697\\
-9865.46209254059	0.00415535483135437\\
-9859.60200219753	0.00403471759031168\\
-9853.74191185447	0.00385571595604088\\
-9847.88182151142	0.00362251566100787\\
-9842.02173116836	0.00334056676256645\\
-9836.1616408253	0.00301647628608772\\
-9830.30155048224	0.00265785341320656\\
-9824.44146013918	0.00227313084248649\\
-9818.58136979612	0.00187136651527814\\
-9812.72127945306	0.00146203036832846\\
-9806.86118911	0.00105478113286438\\
-9801.00109876694	0.000659238440118995\\
-9795.14100842388	0.000284755609934703\\
-9789.28091808082	-5.9801511773406e-05\\
-9783.42082773776	-0.000366264437741306\\
-9777.5607373947	-0.00062735567827998\\
-9771.70064705164	-0.000836861544688236\\
-9765.84055670858	-0.000989780014612611\\
-9759.98046636552	-0.00108244014129332\\
-9754.12037602246	-0.001112590156819\\
-9748.2602856794	-0.00107945215577781\\
-9742.40019533634	-0.000983742032508692\\
-9736.54010499328	-0.000827654164184469\\
-9730.68001465022	-0.000614811163982807\\
-9724.81992430716	-0.000350179853627983\\
-9718.95983396411	-3.99554034799944e-05\\
-9713.09974362105	0.000308583657897991\\
-9707.23965327799	0.000687246171984016\\
-9701.37956293493	0.00108712081242176\\
-9695.51947259187	0.00149878575938505\\
-9689.65938224881	0.00191253065364958\\
-9683.79929190575	0.00231858560203689\\
-9677.93920156269	0.00270735183443493\\
-9672.07911121963	0.0030696285673527\\
-9666.21902087657	0.00339683071277788\\
-9660.35893053351	0.00368119228174126\\
-9654.49884019045	0.00391595066399643\\
-9648.63874984739	0.00409550741163851\\
-9642.77865950433	0.00421556170466336\\
-9636.91856916128	0.00427321331770823\\
-9631.05847881822	0.00426703262295962\\
-9625.19838847516	0.0041970959411474\\
-9619.3382981321	0.00406498536853612\\
-9613.47820778904	0.00387375304460541\\
-9607.61811744598	0.0036278506670992\\
-9601.75802710292	0.00333302587986098\\
-9595.89793675986	0.00299618794527391\\
-9590.0378464168	0.00262524583909361\\
-9584.17775607374	0.00222892256176408\\
-9578.31766573068	0.00181655002431203\\
-9572.45757538762	0.0013978493320181\\
-9566.59748504456	0.000982701637709513\\
-9560.7373947015	0.000580914965836019\\
-9554.87730435844	0.000201992508947934\\
-9549.01721401538	-0.000145092130400868\\
-9543.15712367232	-0.000452107437206596\\
-9537.29703332927	-0.000711759189798616\\
-9531.43694298621	-0.000917863745331579\\
-9525.57685264315	-0.00106549528902926\\
-9519.71676230009	-0.00115110354139142\\
-9513.85667195703	-0.00117259911164546\\
-9507.99658161397	-0.0011294044460698\\
-9502.13649127091	-0.00102246913060506\\
-9496.27640092785	-0.000854249146834765\\
-9490.41631058479	-0.000628650531872853\\
-9484.55622024173	-0.000350938732998813\\
-9478.69612989867	-2.76157599881354e-05\\
-9472.83603955561	0.000333731999683831\\
-9466.97594921255	0.000724611741319333\\
-9461.11585886949	0.00113582380540982\\
-9455.25576852643	0.00155767815331197\\
-9449.39567818337	0.00198022266306069\\
-9443.53558784031	0.00239347786758938\\
-9437.67549749725	0.00278767259901116\\
-9431.81540715419	0.00315347497583804\\
-9425.95531681113	0.00348221327314416\\
-9420.09522646807	0.00376608144899409\\
-9414.23513612501	0.00399832445619156\\
-9408.37504578196	0.00417339894181847\\
-9402.5149554389	0.00428710551121622\\
-9396.65486509584	0.00433668940223972\\
-9390.79477475278	0.0043209071568294\\
-9384.93468440972	0.00424005767722932\\
-9379.07459406666	0.00409597689430078\\
-9373.2145037236	0.00389199613237378\\
-9367.35441338054	0.00363286511371622\\
-9361.49432303748	0.00332464138060481\\
-9355.63423269442	0.00297454870915472\\
-9349.77414235136	0.00259080782287016\\
-9343.9140520083	0.00218244337278847\\
-9338.05396166524	0.00175907171526883\\
-9332.19387132218	0.00133067447743953\\
-9326.33378097912	0.000907363241330561\\
-9320.47369063606	0.000499140893397119\\
-9314.613600293	0.000115665271268403\\
-9308.75350994994	-0.000233979308234936\\
-9302.89341960689	-0.000541497235881272\\
-9297.03332926383	-0.000799578414947775\\
-9291.17323892077	-0.00100207201605785\\
-9285.31314857771	-0.001144133052739\\
-9279.45305823465	-0.00122233828547741\\
-9273.59296789159	-0.00123476868250341\\
-9267.73287754853	-0.00118105645194892\\
-9261.87278720547	-0.00106239549452392\\
-9256.01269686241	-0.000881514988088023\\
-9250.15260651935	-0.000642616685997277\\
-9244.29251617629	-0.000351277368403233\\
-9238.43242583323	-1.43187103329364e-05\\
-9232.57233549017	0.000360352397762063\\
-9226.71224514711	0.000763929339960075\\
-9220.85215480405	0.00118691272472488\\
-9214.992064461	0.0016193339321104\\
-9209.13197411794	0.00205098999858324\\
-9203.27188377488	0.00247168430556852\\
-9197.41179343182	0.00287146739490105\\
-9191.55170308876	0.00324087222426609\\
-9185.6916127457	0.00357113830113054\\
-9179.83152240264	0.00385441938930239\\
-9173.97143205958	0.00408396986436188\\
-9168.11134171652	0.00425430529276466\\
-9162.25125137346	0.00436133341242939\\
-9156.3911610304	0.00440245238819283\\
-9150.53107068734	0.00437661398349029\\
-9144.67098034428	0.0042843501160307\\
-9138.81089000122	0.00412776212804796\\
-9132.95079965816	0.00391047298120032\\
-9127.0907093151	0.00363754346196718\\
-9121.23061897204	0.00331535433450035\\
-9115.37052862898	0.00295145718423983\\
-9109.51043828592	0.00255439743838982\\
-9103.65034794286	0.00213351370972789\\
-9097.79025759981	0.00169871817493735\\
-9091.93016725675	0.00126026315146303\\
-9086.07007691369	0.000828499369137581\\
-9080.20998657063	0.000413631634918769\\
-9074.34989622757	2.54776579995897e-05\\
-9068.48980588451	-0.000326764266075394\\
-9062.62971554145	-0.000634733215633085\\
-9056.76962519839	-0.000891104083604343\\
-9050.90953485533	-0.00108976178970174\\
-9045.04944451227	-0.00122594712163562\\
-9039.18935416921	-0.00129637072791189\\
-9033.32926382615	-0.0012992925329808\\
-9027.46917348309	-0.00123456465848646\\
-9021.60908314003	-0.00110363679407522\\
-9015.74899279697	-0.000909523846688235\\
-9009.88890245391	-0.000656736587672686\\
-9004.02881211085	-0.000351176891762859\\
nan	nan\\
-8992.30863142473	0.000388552979863155\\
-8986.44854108167	0.000805348415206443\\
-8980.58845073861	0.00124057479759203\\
-8974.72836039555	0.00168397365532782\\
-8968.8682700525	0.00212508129596976\\
-8963.00817970944	0.00255347568281955\\
-8957.14808936638	0.00295902262171462\\
-8951.28799902332	0.00333211544331347\\
-8945.42790868026	0.00366390251348384\\
-8939.5678183372	0.00394649718408987\\
-8933.70772799414	0.0041731652057437\\
-8927.84763765108	0.00433848514944135\\
-8921.98754730802	0.00443847801676287\\
-8916.12745696496	0.00447070294047248\\
-8910.2673666219	0.00443431667582854\\
-8904.40727627884	0.00433009543330857\\
-8898.54718593578	0.00416041849243725\\
-8892.68709559272	0.00392921393786518\\
-8886.82700524966	0.00364186775339929\\
-8880.96691490661	0.0033050983764149\\
-8875.10682456355	0.00292679963415491\\
-8869.24673422049	0.00251585573192684\\
-8863.38664387743	0.00208193262934456\\
-8857.52655353437	0.00163525070234832\\
-8851.66646319131	0.00118634403701513\\
-8845.80637284825	0.000745812023475895\\
-8839.94628250519	0.000324069106560004\\
-8834.08619216213	-6.89013992747874e-05\\
-8828.22610181907	-0.00042378361213721\\
-8822.36601147601	-0.000732149909718607\\
-8816.50592113295	-0.0009866610474207\\
-8810.64583078989	-0.00118124081662268\\
-8804.78574044683	-0.00131122108748717\\
-8798.92565010377	-0.00137345377493222\\
-8793.06555976071	-0.0013663870435062\\
-8787.20546941765	-0.00129010390772904\\
-8781.3453790746	-0.00114632227046114\\
-8775.48528873154	-0.000938356351398894\\
-8769.62519838848	-0.000671040369009993\\
-8763.76510804542	-0.000350616232252363\\
-8757.90501770236	1.54121500532384e-05\\
-8752.0449273593	0.000418454542535058\\
-8746.18483701624	0.000849035927562462\\
-8740.32474667318	0.00129701919275337\\
-8734.46465633012	0.00175184372891886\\
-8728.60456598706	0.00220277432339206\\
-8722.744475644	0.00263915448377542\\
-8716.88438530094	0.00305065821529802\\
-8711.02429495788	0.00342753430388004\\
-8705.16420461482	0.00376083732708757\\
-8699.30411427176	0.00404263992075257\\
-8693.4440239287	0.00426622126576131\\
-8687.58393358564	0.00442622731398859\\
-8681.72384324258	0.00451879893457501\\
-8675.86375289952	0.0045416649148009\\
-8670.00366255646	0.00449419757514533\\
-8664.1435722134	0.00437742963913729\\
-8658.28348187034	0.00419403191096226\\
-8652.42339152729	0.00394825223982305\\
-8646.56330118423	0.00364581716300942\\
-8640.70321084117	0.00329379850529249\\
-8634.84312049811	0.0029004480401003\\
-8628.98303015505	0.00247500407744168\\
-8623.12293981199	0.00202747451125621\\
-8617.26284946893	0.00156840141982746\\
-8611.40275912587	0.00110861275528065\\
-8605.54266878281	0.000658966970117599\\
-8599.68257843975	0.000230096602570953\\
-8593.82248809669	-0.000167843125083536\\
-8587.96239775363	-0.000525414751988675\\
-8582.10230741057	-0.000834122375845357\\
-8576.24221706751	-0.0010866134956365\\
-8570.38212672445	-0.00127685410565536\\
-8564.52203638139	-0.00140027287203389\\
-8558.66194603833	-0.00145387094796795\\
-8552.80185569527	-0.00143629479067399\\
-8546.94176535222	-0.00134787021387826\\
-8541.08167500916	-0.00119059682223355\\
-8535.2215846661	-0.000968102909160292\\
-8529.36149432304	-0.000685561832787853\\
-8523.50140397998	-0.000349571796148656\\
-8517.64131363692	3.19981760311385e-05\\
-8511.78122329386	0.000450192469494161\\
-8505.9211329508	0.000895179009245003\\
-8500.06104260774	0.00135648035571344\\
-8494.20095226468	0.00182322033744776\\
-8488.34086192162	0.00228438041581017\\
-8482.48077157856	0.00272905973982109\\
-8476.6206812355	0.00314673275453141\\
-8470.76059089244	0.00352749827598055\\
-8464.90050054939	0.00386231414000165\\
-8459.04041020633	0.00414321186451714\\
-8453.18031986327	0.00436348623070412\\
-8447.32022952021	0.00451785527312759\\
-8441.46013917715	0.00460258686299975\\
-8435.60004883409	0.00461558885176763\\
-8429.73995849103	0.00455646059946184\\
-8423.87986814797	0.00442650462099318\\
-8418.01977780491	0.0042286980239978\\
-8412.15968746185	0.00396762436058506\\
-8406.29959711879	0.00364936745141265\\
-8400.43950677573	0.00328136964013129\\
-8394.57941643267	0.00287225777954352\\
-8388.71932608961	0.00243164101740384\\
-8382.85923574655	0.00196988512036785\\
-8376.99914540349	0.00149786863542219\\
-8371.13905506043	0.00102672662364745\\
-8365.27896471737	0.000567588002048909\\
-8359.41887437431	0.000131312688125549\\
-8353.55878403125	-0.000271765245471788\\
-8347.69869368819	-0.000632082332671406\\
-8341.83860334514	-0.000941072596011324\\
-8335.97851300208	-0.00119137116750191\\
-8330.11842265902	-0.00137698980918325\\
-8324.25833231596	-0.0014934601513483\\
-8318.3982419729	-0.00153794122356749\\
-8312.53815162984	-0.00150928869192557\\
-8306.67806128678	-0.00140808411730959\\
-8300.81797094372	-0.00123662349096209\\
-8294.95788060066	-0.000998865264702988\\
-8289.0977902576	-0.000700339049596867\\
-8283.23769991454	-0.000348017087531137\\
-8277.37760957148	4.98485181683334e-05\\
-8271.51751922842	0.000483919014221943\\
-8265.65742888536	0.000943988128342759\\
-8259.7973385423	0.00141922197891081\\
-8253.93724819924	0.00189841412506393\\
-8248.07715785618	0.002370249753967\\
-8242.21706751312	0.00282357277835864\\
-8236.35697717006	0.00324764954019057\\
-8230.496886827	0.00363242288834444\\
-8224.63679648394	0.00396875061699897\\
-8218.77670614088	0.00424862261316535\\
-8212.91661579783	0.00446535155618922\\
-8207.05652545477	0.00461373263109226\\
-8201.19643511171	0.00469016844152466\\
-8195.33634476865	0.00469275612697605\\
-8189.47625442559	0.00462133457518301\\
-8183.61616408253	0.00447749056206788\\
-8177.75607373947	0.0042645236188018\\
-8171.89598339641	0.00398737040045396\\
-8166.03589305335	0.00365249028792505\\
-8160.17580271029	0.00326771487287228\\
-8154.31571236723	0.00284206483124809\\
-8148.45562202417	0.00238553846709102\\
-8142.59553168111	0.00190887688143375\\
-8136.73544133805	0.00142331128130615\\
-8130.87535099499	0.000940298371978892\\
-8125.01526065193	0.000471250065677629\\
-8119.15517030888	2.72638824057015e-05\\
-8113.29507996582	-0.000381139589259708\\
-8107.43498962276	-0.000744265961507545\\
-8101.5748992797	-0.00105347714186791\\
-8095.71480893664	-0.00130139679299883\\
-8089.85471859358	-0.00148208627178479\\
-8083.99462825052	-0.00159118685275138\\
-8078.13453790746	-0.00162602483077163\\
-8072.2744475644	-0.00158567697055843\\
-8066.41435722134	-0.00147099470286946\\
-8060.55426687828	-0.00128458643998159\\
-8054.69417653522	-0.001030758370533\\
-8048.83408619216	-0.000715415075374337\\
-8042.9739958491	-0.000345922256493886\\
-8037.11390550604	6.90652311472127e-05\\
-8031.25381516299	0.000519806030933051\\
-8025.39372481993	0.000995700875414336\\
-8019.53363447687	0.00148554175362422\\
-8013.67354413381	0.00197777579864329\\
-8007.81345379075	0.00246077768362429\\
-8001.95336344769	0.0029231241047975\\
-7996.09327310463	0.0033538638718244\\
-7990.23318276157	0.00374277722110838\\
-7984.37309241851	0.00408061821249624\\
-7978.51300207545	0.00435933446178328\\
-7972.65291173239	0.00457225898802852\\
-7966.79282138933	0.00471426960627653\\
-7960.93273104627	0.00478191205601213\\
-7955.07264070321	0.00477348390764992\\
-7949.21255036015	0.00468907721033001\\
-7943.35246001709	0.00453057881694765\\
-7937.49236967403	0.00430162831903393\\
-7931.63227933097	0.00400753452702635\\
-7925.77218898791	0.00365515240923434\\
-7919.91209864485	0.00325272334162619\\
-7914.05200830179	0.00280968238947026\\
-7908.19191795873	0.00233643712631276\\
-7902.33182761568	0.001844123173884\\
-7896.47173727262	0.00134434220401038\\
-7890.61164692956	0.000848888565200893\\
-7884.7515565865	0.000369470974443799\\
-7878.89146624344	-8.25641599395829e-05\\
-7873.03137590038	-0.000496501247896817\\
-7867.17128555732	-0.000862509506677961\\
-7861.31119521426	-0.0011718764103387\\
-7855.4511048712	-0.00141721505745585\\
-7849.59101452814	-0.0015926405225492\\
-7843.73092418508	-0.00169391098271643\\
-7837.87083384202	-0.00171853023641277\\
-7832.01074349896	-0.00166580913852832\\
-7826.1506531559	-0.00153688444269864\\
-7820.29056281284	-0.00133469454520824\\
-7814.43047246978	-0.00106391264226396\\
-7808.57038212672	-0.000730838818666499\\
-7802.71029178366	-0.000343253558008612\\
-7796.8502014406	8.97639209317597e-05\\
-7790.99011109754	0.000558048261289596\\
-7785.13002075449	0.00105058652195334\\
-7779.26993041143	0.00155577709121441\\
-7773.40984006837	0.0020617028745361\\
-7767.54974972531	0.00255641232483692\\
-7761.68965938225	0.00302820168944862\\
-7755.82956903919	0.00346589180955359\\
-7749.96947869613	0.0038590929269255\\
-7744.10938835307	0.00419845122673438\\
-7738.24929801001	0.00447587126749133\\
-7732.38920766695	0.00468470901040696\\
-7726.52911732389	0.00481993084626727\\
-7720.66902698083	0.00487823481579208\\
-7714.80893663777	0.00485813110556342\\
-7708.94884629471	0.00475997985931153\\
-7703.08875595166	0.00458598535043495\\
-7697.2286656086	0.00434014658960261\\
-7691.36857526554	0.00402816547112825\\
-7685.50848492248	0.00365731456558242\\
-7679.64839457942	0.00323626762295892\\
-7673.78830423636	0.00277489673474384\\
-7667.9282138933	0.00228404089697095\\
-7662.06812355024	0.0017752513990113\\
-7656.20803320718	0.00126052001710392\\
-7650.34794286412	0.000751996407223845\\
-7644.48785252106	0.000261701355847398\\
-7638.627762178	-0.000198757344909064\\
-7632.76767183494	-0.000618459683814279\\
-7626.90758149188	-0.000987432375046988\\
-7621.04749114882	-0.00129688582194969\\
-7615.18740080576	-0.00153942347109906\\
-7609.3273104627	-0.00170921857177884\\
-7603.46722011964	-0.00180215411821402\\
-7597.60712977658	-0.0018159226151982\\
-7591.74703943353	-0.00175008325178003\\
-7585.88694909047	-0.00160607506982482\\
-7580.02685874741	-0.00138718575129767\\
-7574.16676840435	-0.00109847669622303\\
-7568.30667806129	-0.000746666095729551\\
-7562.44658771823	-0.000339972699795591\\
-7556.58649737517	0.000112076088802196\\
-7550.72640703211	0.000598867314853788\\
-7544.86631668905	0.00110895154258507\\
-7539.00622634599	0.00163031205512012\\
-7533.14613600293	0.00215064785350499\\
-7527.28604565987	0.00265766379220044\\
-7521.42595531681	0.00313936100899797\\
-7515.56586497375	0.00358432079081758\\
-7509.70577463069	0.00398197516092144\\
-7503.84568428763	0.00432285777721249\\
-7497.98559394457	0.00459882918638366\\
-7492.12550360151	0.00480327107548022\\
-7486.26541325845	0.00493124488621574\\
-7480.40532291539	0.00497961099312772\\
-7474.54523257233	0.00494710557064382\\
-7468.68514222927	0.00483437327082675\\
-7462.82505188622	0.00464395487261931\\
-7456.96496154316	0.00438023012724407\\
-7451.1048712001	0.00404931708139347\\
-7445.24478085704	0.0036589301905883\\
-7439.38469051398	0.00321820051139937\\
-7433.52460017092	0.00273746216191537\\
-7427.66450982786	0.00222801004266241\\
-7421.8044194848	0.00170183449662479\\
-7415.94432914174	0.00117133914015726\\
-7410.08423879868	0.000649048503737084\\
-7404.22414845562	0.000147312371606068\\
-7398.36405811256	-0.000321986201267342\\
-7392.5039677695	-0.000747712293557661\\
-7386.64387742644	-0.0011197432806885\\
-7380.78378708338	-0.00142920949135318\\
-7374.92369674032	-0.00166870563765352\\
-7369.06360639727	-0.00183246798045045\\
-7363.20351605421	-0.00191651299321907\\
-7357.34342571115	-0.00191873419030468\\
-7351.48333536809	-0.0018389547684456\\
-7345.62324502503	-0.00167893475050007\\
-7339.76315468197	-0.00144233239245914\\
-7333.90306433891	-0.00113462069513754\\
-7328.04297399585	-0.000762960923031298\\
-7322.18288365279	-0.000336036051262834\\
-7316.32279330973	0.000136151987313901\\
-7310.46270296667	0.000642516522328329\\
-7304.60261262361	0.00117114635007881\\
-7298.74252228055	0.00170958581632914\\
-7292.88243193749	0.00224512819346433\\
-7287.02234159443	0.00276511544347561\\
-7281.16225125137	0.0032572372966923\\
-7275.30216090832	0.00370982258164542\\
-7269.44207056526	0.00411211591289026\\
-7263.5819802222	0.00445453317931075\\
-7257.72188987914	0.00472888976538742\\
-7251.86179953608	0.0049285960728339\\
-7246.00170919302	0.00504881567328607\\
-7240.14161884996	0.00508658229843938\\
-7234.2815285069	0.00504087284005571\\
-7228.42143816384	0.0049126345663638\\
-7222.56134782078	0.00470476584106593\\
-7216.70125747772	0.0044220507260332\\
-7210.84116713466	0.00407104894102709\\
-7204.9810767916	0.00365994370832216\\
-7199.12098644854	0.00319835101011038\\
-7193.26089610548	0.00269709470247947\\
-7187.40080576242	0.00216795274406222\\
-7181.54071541936	0.00162338048697094\\
-7175.6806250763	0.00107621752968296\\
-7169.82053473324	0.000539385029281543\\
-7163.96044439018	2.55806072418099e-05\\
-7158.10035404712	-0.000453021949212453\\
-7152.24026370407	-0.000885061089799094\\
-7146.38017336101	-0.00126025717732863\\
-7140.52008301795	-0.00156965703806855\\
-7134.65999267489	-0.00180584757024491\\
-7128.79990233183	-0.00196313331618809\\
-7122.93981198877	-0.00203767374653313\\
-7117.07972164571	-0.00202757694959323\\
-7111.21963130265	-0.00193294744323167\\
-7105.35954095959	-0.00175588690663076\\
-7099.49945061653	-0.00150044773905863\\
-7093.63936027347	-0.0011725404670072\\
-7087.77926993041	-0.000779797112215492\\
-7081.91917958735	-0.000331393677184145\\
-7076.05908924429	0.000162164125179634\\
-7070.19899890123	0.000689286895977135\\
-7064.33890855817	0.00123757356929216\\
-7058.47881821511	0.00179410304284405\\
-7052.61872787205	0.00234573856389811\\
-7046.75863752899	0.00287943770187893\\
-7040.89854718593	0.00338256059547219\\
-7035.03845684288	0.00384316919069466\\
-7029.17836649982	0.0042503103878975\\
-7023.31827615676	0.00459427638360516\\
-7017.4581858137	0.00486683602111729\\
-7011.59809547064	0.00506143163847697\\
-7005.73800512758	0.00517333670794662\\
-6999.87791478452	0.00519977047984608\\
-6994.01782444146	0.0051399668529639\\
-6988.1577340984	0.00499519576923716\\
-6982.29764375534	0.00476873655094517\\
-6976.43755341228	0.00446580372984344\\
-6970.57746306922	0.00409342704370376\\
-6964.71737272616	0.00366028835835199\\
-6958.8572823831	0.00317651929661852\\
-6952.99719204004	0.00265346428939738\\
-6947.13710169699	0.00210341458817649\\
-6941.27701135393	0.0015393194740748\\
-6935.41692101087	0.0009744814463379\\
-6929.55683066781	0.00042224256351902\\
-6923.69674032475	-0.000104330668892562\\
-6917.83664998169	-0.000592756424064951\\
-6911.97655963863	-0.00103143338130802\\
-6906.11646929557	-0.00140991624614854\\
-6900.25637895251	-0.00171916442158967\\
-6894.39628860945	-0.00195175791239929\\
-6888.53619826639	-0.00210207530847515\\
-6882.67610792333	-0.00216642957996259\\
-6876.81601758027	-0.00214315840660015\\
-6870.95592723721	-0.00203266683163263\\
-6865.09583689415	-0.00183742115323662\\
-6859.23574655109	-0.00156189411605015\\
-6853.37565620803	-0.0012124626159133\\
-6847.51556586497	-0.000797260254116888\\
-6841.65547552191	-0.00032598814825652\\
-6835.79538517886	0.000190311602124636\\
-6829.9352948358	0.000739514507128819\\
-6824.07520449274	0.00130869828155682\\
-6818.21511414968	0.00188444676516167\\
-6812.35502380662	0.00245316602226645\\
-6806.49493346356	0.00300140517511289\\
-6800.6348431205	0.00351617440088875\\
-6794.77475277744	0.00398525257689921\\
-6788.91466243438	0.00439747728922471\\
-6783.05457209132	0.00474301032587694\\
-6777.19448174826	0.00501357234097803\\
-6771.3343914052	0.00520264109569365\\
-6765.47430106214	0.00530560853042484\\
-6759.61421071908	0.00531989288959067\\
-6753.75412037602	0.00524500317214631\\
-6747.89403003296	0.00508255430570568\\
-6742.0339396899	0.00483623260068631\\
-6736.17384934684	0.00451171221529959\\
-6730.31375900378	0.00411652452141069\\
-6724.45366866072	0.00365988337488168\\
-6718.59357831766	0.00315247034271061\\
-6712.7334879746	0.00260618489054492\\
-6706.87339763155	0.00203386537113897\\
-6701.01330728849	0.0014489873534096\\
-6695.15321694543	0.000865346378826853\\
-6689.29312660237	0.000296732611241846\\
-6683.43303625931	-0.000243394948793349\\
-6677.57294591625	-0.000742226991104452\\
-6671.71285557319	-0.00118790762803002\\
-6665.85276523013	-0.00156981613176464\\
-6659.99267488707	-0.00187881998418719\\
-6654.13258454401	-0.0021074932120618\\
-6648.27249420095	-0.00225029479024497\\
-6642.41240385789	-0.00230370282935873\\
-6636.55231351483	-0.0022663013013571\\
-6630.69222317177	-0.00213881717160385\\
-6624.83213282871	-0.00192410697380834\\
-6618.97204248565	-0.0016270930563418\\
-6613.1119521426	-0.00125465091792605\\
-6607.25186179954	-0.000815450207793016\\
-6601.39177145648	-0.000319753064928165\\
-6595.53168111342	0.000220825515518639\\
-6589.67159077036	0.000793589695060228\\
-6583.8115004273	0.00138506081597554\\
-6577.95141008424	0.00198129444312221\\
-6572.09131974118	0.00256820896834099\\
-6566.23122939812	0.00313191803605909\\
-6560.37113905506	0.00365905894498741\\
-6554.511048712	0.00413710926157759\\
-6548.65095836894	0.00455468414590147\\
-6542.79086802588	0.00490180733291645\\
-6536.93077768282	0.00517014932109206\\
-6531.07068733976	0.00535322708461923\\
-6525.21059699671	0.00544656052279753\\
-6519.35050665365	0.005447781874524\\
-6513.49041631059	0.005356695429094\\
-6507.63032596753	0.00517528603443037\\
-6501.77023562447	0.00490767611017306\\
-6495.91014528141	0.00456003208852832\\
-6490.05005493835	0.00414042240254238\\
-6484.18996459529	0.00365863028905291\\
-6478.32987425223	0.00312592574678473\\
-6472.46978390917	0.00255480196276792\\
-6466.60969356611	0.00195868236896549\\
-6460.74960322305	0.00135160519540629\\
-6454.88951287999	0.000747892930735089\\
-6449.02942253693	0.000161814470056469\\
-6443.16933219387	-0.000392752082520558\\
-6437.30924185081	-0.000902647990680954\\
-6431.44915150775	-0.00135574606517833\\
-6425.58906116469	-0.00174123904191137\\
-6419.72897082163	-0.00204989730375426\\
-6413.86888047857	-0.0022742898028516\\
-6408.00879013551	-0.00240896289796897\\
-6402.14869979245	-0.00245057280581024\\
-6396.2886094494	-0.00239796845315791\\
-6390.42851910634	-0.0022522226819947\\
-6384.56842876328	-0.00201661097626424\\
-6378.70833842022	-0.00169653811643794\\
-6372.84824807716	-0.00129941439902993\\
-6366.9881577341	-0.000834484252292827\\
-6361.12806739104	-0.000312611208892719\\
-6355.26797704798	0.000253975765548383\\
-6349.40788670492	0.000851968667284953\\
-6343.54779636186	0.00146729286793934\\
-6337.6877060188	0.00208543821521955\\
-6331.82761567574	0.00269180103443843\\
-6325.96752533268	0.00327202897234065\\
-6320.10743498962	0.00381236054331838\\
-6314.24734464656	0.00429995134713795\\
-6308.3872543035	0.00472317922852225\\
-6302.52716396044	0.00507192113012198\\
-6296.66707361738	0.00533779504811387\\
-6290.80698327432	0.00551436130853336\\
-6284.94689293126	0.00559727833517958\\
-6279.08680258821	0.005584409144124\\
-6273.22671224515	0.00547587595770156\\
-6267.36662190209	0.00527406154989258\\
-6261.50653155903	0.00498355719262695\\
-6255.64644121597	0.00461105833162487\\
-6249.78635087291	0.00416521035809779\\
-6243.92626052985	0.00365640802571516\\
-6238.06617018679	0.00309655316352792\\
-6232.20607984373	0.00249877632929415\\
-6226.34598950067	0.0018771289102097\\
-6220.48589915761	0.00124625288840901\\
-6214.62580881455	0.000621036029345132\\
-6208.76571847149	1.62606105484162e-05\\
-6202.90562812843	-0.000553746023853492\\
-6197.04553778538	-0.00107545082509682\\
-6191.18544744232	-0.00153643628271601\\
-6185.32535709926	-0.00192569592569518\\
-6179.4652667562	-0.00223389705270557\\
-6173.60517641314	-0.00245360442278085\\
-6167.74508607008	-0.00257945954782799\\
-6161.88499572702	-0.0026083112671582\\
-6156.02490538396	-0.00253929442639755\\
-6150.1648150409	-0.00237385470245888\\
-6144.30472469784	-0.00211571888554609\\
-6138.44463435478	-0.00177081121470925\\
-6132.58454401172	-0.00134711764011586\\
-6126.72445366866	-0.000854501117769777\\
-6120.8643633256	-0.000304472205703852\\
-6115.00427298254	0.00029007970600882\\
-6109.14418263948	0.000915188258319774\\
-6103.28409229642	0.00155613801297503\\
-6097.42400195336	0.0021978106739798\\
-6091.5639116103	0.00282504149783154\\
-6085.70382126725	0.00342297749337938\\
-6079.84373092419	0.00397742895308915\\
-6073.98364058113	0.00447520599931261\\
-6068.12355023807	0.00490443216626425\\
-6062.26345989501	0.00525482756537569\\
-6056.40336955195	0.00551795488594006\\
-6050.54327920889	0.00568742234715295\\
-6044.68318886583	0.00575903872375827\\
-6038.82309852277	0.00573091668791684\\
-6032.96300817971	0.00560352192679455\\
-6027.10291783665	0.0053796667672068\\
-6021.24282749359	0.00506444834994351\\
-6015.38273715053	0.00466513270465152\\
-6009.52264680747	0.00419098735696594\\
-6003.66255646441	0.00365306632065881\\
-5997.80246612135	0.00306395245963145\\
-5991.94237577829	0.00243746322122374\\
-5986.08228543523	0.00178832661968941\\
-5980.22219509217	0.00113183506484267\\
-5974.36210474911	0.000483485169084496\\
-5968.50201440605	-0.000141387986402107\\
-5962.64192406299	-0.000727973497765935\\
-5956.78183371994	-0.00126233562435079\\
-5950.92174337688	-0.00173174479983086\\
-5945.06165303382	-0.00212498080952864\\
-5939.20156269076	-0.00243260091635069\\
-5933.3414723477	-0.00264716653143206\\
-5927.48138200464	-0.00276342299202658\\
-5921.62129166158	-0.00277842810518147\\
-5915.76120131852	-0.00269162631846643\\
-5909.90111097546	-0.00250486665543998\\
-5904.0410206324	-0.00222236387948641\\
-5898.18093028934	-0.00185060368842271\\
-5892.32083994628	-0.0013981940668296\\
-5886.46074960322	-0.000875666198067985\\
-5880.60065926016	-0.000295229537481451\\
-5874.7405689171	0.000329513260952879\\
-5868.88047857404	0.000983884911402183\\
-5863.02038823099	0.00165247809928863\\
-5857.16029788793	0.00231951803323882\\
-5851.30020754487	0.00296923441796199\\
-5845.44011720181	0.00358623408019082\\
-5839.58002685875	0.00415586544681112\\
-5833.71993651569	0.00466456624824702\\
-5827.85984617263	0.00510018619935562\\
-5821.99975582957	0.00545227698349191\\
-5816.13966548651	0.00571234262291438\\
-5810.27957514345	0.00587404423898748\\
-5804.41948480039	0.00593335427217524\\
-5798.55939445733	0.00588865641368789\\
-5792.69930411427	0.00574078877660408\\
-5786.83921377121	0.00549302916919449\\
-5780.97912342815	0.00515102269908503\\
-5775.11903308509	0.00472265329803579\\
-5769.25894274204	0.00421786208702458\\
-5763.39885239898	0.00364841676164796\\
-5757.53876205592	0.00302763734357482\\
-5751.67867171286	0.00237008468625006\\
-5745.8185813698	0.00169121901555685\\
-5739.95849102674	0.00100703650901828\\
-5734.09840068368	0.000333692453135872\\
-5728.23831034062	-0.000312880147372273\\
-5722.37821999756	-0.000917347507992998\\
-5716.5181296545	-0.00146533863688936\\
-5710.65803931144	-0.00194378692458454\\
-5704.79794896838	-0.00234124163505816\\
-5698.93785862532	-0.00264814187668898\\
-5693.07776828226	-0.00285704650173869\\
-5687.2176779392	-0.00296281441032375\\
-5681.35758759614	-0.00296273089486637\\
-5675.49749725308	-0.00285657692616216\\
-5669.63740691002	-0.002646639623084\\
-5663.77731656696	-0.00233766353442861\\
-5657.9172262239	-0.00193674375738302\\
-5652.05713588084	-0.00145316329447726\\
-5646.19704553779	-0.000898178371887775\\
-5640.33695519473	-0.000284756680071274\\
-5634.47686485167	0.000372725380382555\\
-5628.61677450861	0.00105881938305717\\
-5622.75668416555	0.00175736760918476\\
-5616.89659382249	0.00245188331715586\\
-5611.03650347943	0.00312593960227707\\
-5605.17641313637	0.00376355767896498\\
-5599.31632279331	0.00434958541164809\\
-5593.45623245025	0.00487005713112554\\
-5587.59614210719	0.00531252619536609\\
-5581.73605176413	0.00566636238044875\\
-5575.87596142107	0.00592300699925457\\
-5570.01587107801	0.00607617962991959\\
-5564.15578073495	0.00612203146568702\\
-5558.29569039189	0.00605924154656835\\
-5552.43560004883	0.00588905347689357\\
-5546.57550970577	0.00561525162976472\\
-5540.71541936271	0.00524407727078112\\
-5534.85532901965	0.00478408644935242\\
-5528.9952386766	0.00424595288912246\\
-5523.13514833354	0.00364222041250649\\
-5517.27505799048	0.00298701063816723\\
-5511.41496764742	0.00229569275805176\\
-5505.55487730436	0.00158452311235249\\
-5499.6947869613	0.000870263010159422\\
-5493.83469661824	0.000169783776807478\\
-5487.97460627518	-0.000500331670064663\\
-5482.11451593212	-0.00112418131098558\\
-5476.25442558906	-0.00168692129635067\\
-5470.394335246	-0.00217511905284908\\
-5464.53424490294	-0.00257707382942487\\
-5458.67415455988	-0.00288309703758909\\
-5452.81406421682	-0.00308574567399403\\
-5446.95397387376	-0.00318000320655496\\
-5441.09388353071	-0.00316340353342987\\
-5435.23379318765	-0.00303609495944454\\
-5429.37370284459	-0.00280084254465944\\
-5423.51361250153	-0.00246296863114866\\
-5417.65352215847	-0.00203023281513701\\
-5411.79343181541	-0.0015126540634109\\
-5405.93334147235	-0.000922279046914138\\
-5400.07325112929	-0.000272902043125599\\
-5394.21316078623	0.000420257084672758\\
-5388.35307044317	0.00114090931835258\\
-5382.49298010011	0.00187207898338601\\
-5376.63288975705	0.00259650333742868\\
-5370.77279941399	0.0032970398478728\\
-5364.91270907093	0.00395707155384661\\
-5359.05261872787	0.00456090093196473\\
-5353.19252838481	0.0050941229348565\\
-5347.33243804175	0.00554396834214876\\
-5341.47234769869	0.00589960924439551\\
-5335.61225735563	0.0061524193568112\\
-5329.75216701258	0.00629618290975367\\
-5323.89207666952	0.00632724706264815\\
-5318.03198632646	0.00624461411186288\\
-5312.1718959834	0.00604997117492418\\
-5306.31180564034	0.00574765650713192\\
-5300.45171529728	0.00534456309915285\\
-5294.59162495422	0.00484998168923948\\
-5288.73153461116	0.0042753867587451\\
-5282.8714442681	0.00363417043458561\\
-5277.01135392504	0.00294133046499977\\
-5271.15126358198	0.00221311953307835\\
-5265.29117323892	0.0014666641034942\\
-5259.43108289586	0.000719561736366034\\
-5253.5709925528	-1.05336688027332e-05\\
-5247.71090220974	-0.000706328928454045\\
-5241.85081186668	-0.00135130108202774\\
-5235.99072152362	-0.00193008965858885\\
-5230.13063118056	-0.00242886239529102\\
-5224.2705408375	-0.00283564570531639\\
-5218.41045049444	-0.00314061200521067\\
-5212.55036015138	-0.00333631701190556\\
-5206.69026980832	-0.00341788128565944\\
-5200.83017946527	-0.0033831115981668\\
-5194.97008912221	-0.00323255911665472\\
-5189.10999877915	-0.00296951287963104\\
-5183.24990843609	-0.0025999285654072\\
-5177.38981809303	-0.00213229408257217\\
-5171.52972774997	-0.00157743500799598\\
-5165.66963740691	-0.000948264325738051\\
-5159.80954706385	-0.00025948224664403\\
-5153.94945672079	0.000472766918436309\\
-5148.08936637773	0.00123127282810813\\
-5142.22927603467	0.00199816327395481\\
-5136.36918569161	0.00275532495227472\\
-5130.50909534855	0.00348483094289926\\
-5124.64900500549	0.00416936481180091\\
-5118.78891466243	0.00479263130979069\\
-5112.92882431937	0.00533974393228034\\
-5107.06873397632	0.00579758012743732\\
-5101.20864363326	0.00615509568318416\\
-5095.3485532902	0.00640359076642127\\
-5089.48846294714	0.0065369212118717\\
-5083.62837260408	0.00655164993633903\\
-5077.76828226102	0.00644713475472385\\
-5071.90819191796	0.00622555036950828\\
-5066.0481015749	0.00589184385245099\\
-5060.18801123184	0.00545362450936383\\
-5054.32792088878	0.00492099056791236\\
-5048.46783054572	0.00430629663020967\\
-5042.60774020266	0.00362386723746691\\
-5036.7476498596	0.00288966318301387\\
-5030.88755951654	0.00212090834009687\\
-5025.02746917348	0.00133568572432813\\
-5019.16737883043	0.000552512258170917\\
-5013.30728848737	-0.000210097768708227\\
-5007.44719814431	-0.00093407025742763\\
-5001.58710780125	-0.00160219954101694\\
-4995.72701745819	-0.00219855716803276\\
-4989.86692711513	-0.00270887152233677\\
-4984.00683677207	-0.00312086923776121\\
-4978.14674642901	-0.00342457024591402\\
-4972.28665608595	-0.00361252937091565\\
-4966.42656574289	-0.00368001862940849\\
-4960.56647539983	-0.00362514578034095\\
-4954.70638505677	-0.00344890616422852\\
-4948.84629471371	-0.00315516643918719\\
-4942.98620437065	-0.00275058042713254\\
-4937.12611402759	-0.00224443888751755\\
-4931.26602368453	-0.00164845660097083\\
-4925.40593334147	-0.00097650163525037\\
-4919.54584299841	-0.000244273043748826\\
-4913.68575265535	0.000531065519587303\\
-4907.82566231229	0.00133128772401368\\
-4901.96557196923	0.0021375326189662\\
-4896.10548162617	0.00293074877724391\\
-4890.24539128312	0.00369214407177429\\
-4884.38530094006	0.00440363047408418\\
-4878.525210597	0.00504825335207591\\
-4872.66512025394	0.00561059508553304\\
-4866.80502991088	0.00607714339672218\\
-4860.94493956782	0.00643661560344464\\
-4855.08484922476	0.0066802310206236\\
-4849.2247588817	0.00680192493989265\\
-4843.36466853864	0.00679849898248883\\
-4837.50457819558	0.00666970410702392\\
-4831.64448785252	0.00641825413822676\\
-4825.78439750946	0.00604976931606611\\
-4819.9243071664	0.00557265101788421\\
-4814.06421682334	0.004997890433962\\
-4808.20412648028	0.00433881554615089\\
-4802.34403613722	0.00361078222517441\\
-4796.48394579416	0.00283081660068235\\
-4790.6238554511	0.00201721702516986\\
-4784.76376510804	0.00118912493071111\\
-4778.90367476498	0.000366074636237159\\
-4773.04358442193	-0.000432467313435233\\
-4767.18349407887	-0.00118756242026539\\
-4761.32340373581	-0.00188124924158634\\
-4755.46331339275	-0.00249697049601303\\
-4749.60322304969	-0.00301996811788801\\
-4743.74313270663	-0.00343763684092504\\
-4737.88304236357	-0.00373982784746673\\
-4732.02295202051	-0.00391909517608863\\
-4726.16286167745	-0.003970878913277\\
-4720.30277133439	-0.00389362067287524\\
-4714.44268099133	-0.00368880845362635\\
-4708.58259064827	-0.00336094962661905\\
-4702.72250030521	-0.0029174724982079\\
-4696.86240996215	-0.00236855858275609\\
-4691.0023196191	-0.0017269093620529\\
-4685.14222927604	-0.00100745286604061\\
-4679.28213893298	-0.000226996845361649\\
-4673.42204858992	0.000596163413792659\\
-4667.56195824686	0.00144267348875718\\
-4661.7018679038	0.00229257440960022\\
-4655.84177756074	0.00312577276856715\\
-4649.98168721768	0.00392251528471928\\
-4644.12159687462	0.00466385662267876\\
-4638.26150653156	0.00533210939437569\\
-4632.4014161885	0.00591126566192369\\
-4626.54132584544	0.00638737990567538\\
-4620.68123550238	0.00674890430226135\\
-4614.82114515932	0.00698696826112283\\
-4608.96105481626	0.00709559546077547\\
-4603.1009644732	0.00707185308507725\\
-4597.24087413014	0.00691592954726432\\
-4591.38078378708	0.00663113866645557\\
-4585.52069344402	0.00622384999610994\\
-4579.66060310097	0.00570334674537006\\
-4573.80051275791	0.00508161444939937\\
-4567.94042241485	0.00437306518987947\\
-4562.08033207179	0.00359420370226787\\
-4556.22024172873	0.00276324309694081\\
-4550.36015138567	0.00189967913315761\\
-4544.50006104261	0.00102383298922104\\
-4538.63997069955	0.000156373243617816\\
-4532.77988035649	-0.000682171696352505\\
-4526.91979001343	-0.00147189922510477\\
-4521.05969967037	-0.00219400474851233\\
-4515.19960932731	-0.00283122927761571\\
-4509.33951898425	-0.00336827176109309\\
-4503.47942864119	-0.0037921563139937\\
-4497.61933829813	-0.00409254553806755\\
-4491.75924795507	-0.00426199237767411\\
-4485.89915761201	-0.00429612438516661\\
-4480.03906726895	-0.00419375584983331\\
-4474.17897692589	-0.00395692493507415\\
-4468.31888658283	-0.00359085473168586\\
-4462.45879623977	-0.00310383893029356\\
-4456.59870589671	-0.00250705459753157\\
-4450.73861555366	-0.0018143062706842\\
-4444.8785252106	-0.00104170721982999\\
-4439.01843486754	-0.000207305227401601\\
-4433.15834452448	0.000669338432613838\\
-4427.29825418142	0.00156760700324504\\
-4421.43816383836	0.00246631252594408\\
-4415.5780734953	0.00334419502831241\\
-4409.71798315224	0.00418042486645277\\
-4403.85789280918	0.00495509636141761\\
-4397.99780246612	0.00564970104074225\\
-4392.13771212306	0.00624756924310367\\
-4386.27762178	0.00673426955943134\\
-4380.41753143694	0.0070979565481617\\
-4374.55744109388	0.00732965835699008\\
-4368.69735075082	0.00742349727841564\\
-4362.83726040776	0.00737683782939752\\
-4356.97717006471	0.00719035864244292\\
-4351.11707972165	0.00686804624474679\\
-4345.25698937859	0.00641711063969909\\
-4339.39689903553	0.00584782445449976\\
-4333.53680869247	0.00517328922701587\\
-4327.67671834941	0.0044091341367855\\
-4321.81662800635	0.00357315409801373\\
-4315.95653766329	0.00268489558418496\\
-4310.09644732023	0.0017651998147762\\
-4304.23635697717	0.000835713969562203\\
-4298.37626663411	-8.16181154118465e-05\\
-4292.51617629105	-0.000965073798040115\\
-4286.65608594799	-0.00179366581587415\\
-4280.79599560493	-0.00254764064671001\\
-4274.93590526187	-0.00320894922013131\\
-4269.07581491882	-0.00376167877409589\\
-4263.21572457576	-0.00419243553368495\\
-4257.3556342327	-0.00449066902062229\\
-4251.49554388964	-0.00464893015194464\\
-4245.63545354658	-0.00466305682746218\\
-4239.77536320352	-0.00453228239845683\\
-4233.91527286046	-0.00425926421897166\\
-4228.0551825174	-0.00385003135867849\\
-4222.19509217434	-0.0033138524653146\\
-4216.33500183128	-0.00266302665218088\\
-4210.47491148822	-0.00191260211466732\\
-4204.61482114516	-0.00108002890107251\\
-4198.7547308021	-0.000184753838597746\\
-4192.89464045904	0.000752232993288131\\
-4187.03455011598	0.00170888967644035\\
-4181.17445977292	0.00266264026045302\\
-4175.31436942986	0.00359090680916101\\
-4169.4542790868	0.00447164315926095\\
-4163.59418874374	0.00528385778461535\\
-4157.73409840068	0.00600811337621788\\
-4151.87400805762	0.00662699126065033\\
-4146.01391771456	0.00712550957154266\\
-4140.15382737151	0.00749148514694538\\
-4134.29373702845	0.00771583042410058\\
-4128.43364668539	0.00779277810868319\\
-4122.57355634233	0.0077200280816226\\
-4116.71346599927	0.00749881282498909\\
-4110.85337565621	0.00713387956008516\\
-4104.99328531315	0.00663338925592373\\
-4099.13319497009	0.00600873462693156\\
-4093.27310462703	0.00527428116240368\\
-4087.41301428397	0.00444703705929061\\
-4081.55292394091	0.00354625962961642\\
-4075.69283359785	0.00259300727804757\\
-4069.83274325479	0.00160964746120418\\
-4063.97265291173	0.000619332112477259\\
-4058.11256256867	-0.000354547176565682\\
-4052.25247222561	-0.00128891141044684\\
-4046.39238188255	-0.00216154085385285\\
-4040.53229153949	-0.00295160316964667\\
-4034.67220119643	-0.00364015049060143\\
-4028.81211085337	-0.00421057357706133\\
-4022.95202051031	-0.00464900218843632\\
-4017.09193016726	-0.00494464203082911\\
-4011.2318398242	-0.00509004011016201\\
-4005.37174948114	-0.00508127198444811\\
-3999.51165913808	-0.00491804623215694\\
-3993.65156879502	-0.00460372339460904\\
-3987.79147845196	-0.00414524866070209\\
-3981.9313881089	-0.00355299959823261\\
-3976.07129776584	-0.00284055224740289\\
-3970.21120742278	-0.00202437083091371\\
-3964.35111707972	-0.0011234281582107\\
-3958.49102673666	-0.000158765461476173\\
-3952.6309363936	0.000846998138636516\\
-3946.77084605054	0.00187019511530769\\
-3940.91075570748	0.00288666558511696\\
-3935.05066536443	0.00387232687513467\\
-3929.19057502137	0.00480374319374668\\
-3923.33048467831	0.00565868194429951\\
-3917.47039433525	0.00641664349010055\\
-3911.61030399219	0.00705935176191942\\
-3905.75021364913	0.00757119398092691\\
-3899.89012330607	0.00793959893294186\\
-3894.03003296301	0.00815534464361101\\
-3888.16994261995	0.00821278794353416\\
-3882.30985227689	0.00811001022850894\\
-3876.44976193383	0.00784887568288021\\
-3870.58967159077	0.00743500028746927\\
-3864.72958124771	0.00687763203637733\\
-3858.86949090465	0.00618944488518528\\
-3853.00940056159	0.00538625100051817\\
-3847.14931021853	0.00448663782634899\\
-3841.28921987547	0.00351153828073021\\
-3835.42912953241	0.00248374400730378\\
-3829.56903918935	0.00142737298294409\\
-3823.7089488463	0.000367303903315773\\
-3817.84885850324	-0.00067140940488208\\
-3811.98876816018	-0.00166412977740159\\
-3806.12867781712	-0.00258722074885845\\
-3800.26858747406	-0.00341860900343866\\
-3794.408497131	-0.00413831187208375\\
-3788.54840678794	-0.00472891728670607\\
-3782.68831644488	-0.0051760046804829\\
-3776.82822610182	-0.00546849667736869\\
-3770.96813575876	-0.00559893301236882\\
-3765.1080454157	-0.00556365993117846\\
-3759.24795507264	-0.00536293029028585\\
-3753.38786472958	-0.00500091166995849\\
-3747.52777438652	-0.00448560197798168\\
-3741.66768404346	-0.00382865420313579\\
-3735.8075937004	-0.00304511413168779\\
-3729.94750335734	-0.00215307690828768\\
-3724.08741301428	-0.00117327026290317\\
-3718.22732267122	-0.000128573984568956\\
-3712.36723232816	0.000956513234835723\\
-3706.5071419851	0.00205644718068208\\
-3700.64705164204	0.00314523831719472\\
-3694.78696129899	0.00419706472580414\\
-3688.92687095593	0.00518688334262058\\
-3683.06678061287	0.00609102504420597\\
-3677.20669026981	0.00688775946002173\\
-3671.34659992675	0.00755781605084114\\
-3665.48650958369	0.00808484897791946\\
-3659.62641924063	0.00845583456773837\\
-3653.76632889757	0.00866139172788757\\
-3647.90623855451	0.00869601745434875\\
-3642.04614821145	0.00855823154619699\\
-3636.18605786839	0.00825062676708455\\
-3630.32596752533	0.00777982291194471\\
-3624.46587718227	0.00715632550136256\\
-3618.60578683921	0.00639429208248266\\
-3612.74569649615	0.00551121130918611\\
-3606.88560615309	0.00452750205278483\\
-3601.02551581004	0.00346604171332176\\
-3595.16542546698	0.00235163460776858\\
-3589.30533512392	0.00121043276809243\\
-3583.44524478086	6.93226548304334e-05\\
-3577.5851544378	-0.00104470785231902\\
-3571.72506409474	-0.00210520730384014\\
-3565.86497375168	-0.00308688831490519\\
-3560.00488340862	-0.00396623011362101\\
-3554.14479306556	-0.00472204166663773\\
-3548.2847027225	-0.00533597192109993\\
-3542.42461237944	-0.00579295489945733\\
-3536.56452203638	-0.00608157887400178\\
-3530.70443169332	-0.00619437060029849\\
-3524.84434135026	-0.00612798755876339\\
-3518.9842510072	-0.00588331330029163\\
-3513.12416066415	-0.00546545326007842\\
-3507.26407032109	-0.00488363074642171\\
-3501.40397997803	-0.0041509851657094\\
-3495.54388963497	-0.00328427686532124\\
-3489.68379929191	-0.00230350519651384\\
-3483.82370894885	-0.00123144847820071\\
-3477.96361860579	-9.3136419914711e-05\\
-3472.10352826273	0.00108473279853733\\
-3466.24343791967	0.00227441723806101\\
-3460.38334757661	0.00344778310696529\\
-3454.52325723355	0.00457696856755867\\
-3448.66316689049	0.00563504380392908\\
-3442.80307654743	0.00659665173717525\\
-3436.94298620437	0.0074386141681885\\
-3431.08289586131	0.00814048888322185\\
-3425.22280551825	0.00868506435612603\\
-3419.36271517519	0.00905878010104059\\
-3413.50262483213	0.00925206243771255\\
-3407.64253448907	0.00925956738492283\\
-3401.78244414601	0.00908032456041693\\
-3395.92235380295	0.00871777827683565\\
-3390.06226345989	0.00817972443606576\\
-3384.20217311684	0.00747814428085077\\
-3378.34208277378	0.00662893850202409\\
-3372.48199243072	0.00565156756964951\\
-3366.62190208766	0.00456860639406005\\
-3360.7618117446	0.00340522348184922\\
-3354.90172140154	0.00218859657776847\\
-3349.04163105848	0.000947278333604621\\
-3343.18154071542	-0.000289473216962629\\
-3337.32145037236	-0.00149238370437206\\
-3331.4613600293	-0.00263285479000728\\
-3325.60126968624	-0.00368364350610682\\
-3319.74117934318	-0.00461951245901207\\
-3313.88108900012	-0.00541783540592309\\
-3308.02099865706	-0.00605914370243496\\
-3302.160908314	-0.00652760045285251\\
-3296.30081797094	-0.00681139084674552\\
-3290.44072762788	-0.00690301909406241\\
-3284.58063728482	-0.00679950453741863\\
-3278.72054694176	-0.00650247186787117\\
-3272.8604565987	-0.00601813285091013\\
-3267.00036625564	-0.00535715951998051\\
-3261.14027591259	-0.00453445135847831\\
-3255.28018556953	-0.00356880150643534\\
-3249.42009522647	-0.0024824694344331\\
-3243.56000488341	-0.00130066977059606\\
-3237.69991454035	-5.09889885340821e-05\\
-3231.83982419729	0.0012372565793532\\
-3225.97973385423	0.00253370649345193\\
-3220.11964351117	0.00380766971341746\\
-3214.25955316811	0.00502884912276843\\
-3208.39946282505	0.00616805999901185\\
-3202.53937248199	0.00719792542440139\\
-3196.67928213893	0.00809353209863417\\
-3190.81919179587	0.00883303087731186\\
-3184.95910145281	0.00939816759370712\\
-3179.09901110976	0.00977473130189391\\
-3173.2389207667	0.00995290897301492\\
-3167.37883042364	0.00992753783331191\\
-3161.51874008058	0.00969824891316643\\
-3155.65864973752	0.00926949791224047\\
-3149.79855939446	0.00865048213370397\\
-3143.9384690514	0.00785494492196435\\
-3138.07837870834	0.00690087170401437\\
-3132.21828836528	0.00581008431329038\\
-3126.35819802222	0.00460774270801342\\
-3120.49810767916	0.00332176542560803\\
-3114.6380173361	0.00198218208787227\\
-3108.77792699304	0.00062043293757984\\
-3102.91783664998	-0.00073136829003325\\
-3097.05774630692	-0.00204119091566028\\
-3091.19765596386	-0.00327784549821102\\
-3085.3375656208	-0.00441172574216919\\
-3079.47747527774	-0.00541551652062163\\
-3073.61738493469	-0.0062648511204916\\
-3067.75729459163	-0.00693890193505245\\
-3061.89720424857	-0.0074208903251861\\
-3056.03711390551	-0.00769850321069726\\
-3050.17702356245	-0.007764206095785\\
-3044.31693321939	-0.00761544462933128\\
-3038.45684287633	-0.00725472939405607\\
-3032.59675253327	-0.00668960135077654\\
-3026.73666219021	-0.00593247817007214\\
-3020.87657184715	-0.0050003844974638\\
-3015.01648150409	-0.00391457195439474\\
-3009.15639116103	-0.00270003730960112\\
-3003.29630081797	-0.00138494970234729\\
nan	nan\\
-2991.57612013185	0.00142231272316089\\
-2985.71602978879	0.00284844103973382\\
-2979.85593944573	0.00424457763248576\\
-2973.99584910267	0.0055774538797864\\
-2968.13575875961	0.00681512974368829\\
-2962.27566841655	0.00792775625113928\\
-2956.41557807349	0.00888829241280049\\
-2950.55548773043	0.00967315940618472\\
-2944.69539738738	0.0102628162442474\\
-2938.83530704432	0.0106422429257624\\
-2932.97521670126	0.0108013191729026\\
-2927.1151263582	0.0107350892715334\\
-2921.25503601514	0.0104439061639191\\
-2915.39494567208	0.00993345076560541\\
-2909.53485532902	0.00921462539855497\\
-2903.67476498596	0.00830332320286377\\
-2897.8146746429	0.00722007832548872\\
-2891.95458429984	0.0059896045260749\\
-2886.09449395678	0.00464023251354548\\
-2880.23440361372	0.00320325877352022\\
-2874.37431327066	0.00171222080351519\\
-2868.5142229276	0.00020211548993306\\
-2862.65413258454	-0.00129142120849967\\
-2856.79404224148	-0.00273295409464437\\
-2850.93395189842	-0.00408808834539056\\
-2845.07386155537	-0.00532428906249935\\
-2839.21377121231	-0.00641166128931641\\
-2833.35368086925	-0.00732367184756998\\
-2827.49359052619	-0.00803779562351115\\
-2821.63350018313	-0.00853607061840782\\
-2815.77340984007	-0.00880554815098393\\
-2809.91331949701	-0.00883862699654855\\
-2804.05322915395	-0.00863326292849645\\
-2798.19313881089	-0.00819304802201049\\
-2792.33304846783	-0.00752715712211438\\
-2786.47295812477	-0.00665016200192275\\
-2780.61286778171	-0.00558171686171607\\
-2774.75277743865	-0.00434612187831032\\
-2768.89268709559	-0.00297177442992492\\
-2763.03259675254	-0.00149052032857914\\
-2757.17250640948	6.30801788299924e-05\\
-2751.31241606642	0.00165255478330148\\
-2745.45232572336	0.00324037129351002\\
-2739.5922353803	0.00478881977361752\\
-2733.73214503724	0.00626090397022931\\
-2727.87205469418	0.0076212210206767\\
-2722.01196435112	0.0088368085913962\\
-2716.15187400806	0.00987793924262647\\
-2710.291783665	0.0107188429453168\\
-2704.43169332194	0.0113383402598657\\
-2698.57160297888	0.0117203706928083\\
-2692.71151263582	0.0118544031335403\\
-2686.85142229276	0.0117357179755324\\
-2680.9913319497	0.0113655534972052\\
-2675.13124160664	0.0107511122353016\\
-2669.27115126358	0.00990542636467931\\
-2663.41106092052	0.00884708441849526\\
-2657.55097057746	0.00759982497022164\\
-2651.6908802344	0.00619200607336325\\
-2645.83078989134	0.00465596223862386\\
-2639.97069954828	0.00302726345532345\\
-2634.11060920523	0.00134389316210531\\
-2628.25051886217	-0.000354635910199807\\
-2622.39042851911	-0.00202820751994197\\
-2616.53033817605	-0.00363704770804635\\
-2610.67024783299	-0.00514266838524258\\
-2604.81015748993	-0.00650878641024377\\
-2598.95006714687	-0.00770219629457858\\
-2593.08997680381	-0.00869357562946508\\
-2587.22988646075	-0.00945820378437724\\
-2581.36979611769	-0.00997657635428583\\
-2575.50970577463	-0.010234900178849\\
-2569.64961543157	-0.0102254564807398\\
-2563.78952508851	-0.00994682270303402\\
-2557.92943474545	-0.00940394689773573\\
-2552.06934440239	-0.00860807195946099\\
-2546.20925405933	-0.00757651052092272\\
-2540.34916371627	-0.00633227485948068\\
-2534.48907337321	-0.00490356960841335\\
-2528.62898303015	-0.00332315835780078\\
-2522.76889268709	-0.00162761827576789\\
-2516.90880234403	0.000143500384176332\\
-2511.04871200098	0.00194859066912146\\
-2505.18862165792	0.00374496006777452\\
-2499.32853131486	0.00548983383634214\\
-2493.4684409718	0.0071413670297236\\
-2487.60835062874	0.00865964139479158\\
-2481.74826028568	0.0100076234812093\\
-2475.88816994262	0.0111520610785373\\
-2470.02807959956	0.0120642963894758\\
-2464.1679892565	0.0127209761646408\\
-2458.30789891344	0.0131046413218598\\
-2452.44780857038	0.013204181292042\\
-2446.58771822732	0.0130151414254016\\
-2440.72762788426	0.0125398751761778\\
-2434.8675375412	0.0117875363843739\\
-2429.00744719815	0.0107739107154888\\
-2423.14735685509	0.00952108910289104\\
-2417.28726651203	0.00805698978402546\\
-2411.42717616897	0.00641473913918165\\
-2405.56708582591	0.00463192494249867\\
-2399.70699548285	0.0027497387414227\\
-2393.84690513979	0.000812026814088753\\
-2387.98681479673	-0.00113572855030081\\
-2382.12672445367	-0.00304747390285287\\
-2376.26663411061	-0.00487767081481468\\
-2370.40654376755	-0.00658237789869062\\
-2364.54645342449	-0.00812030309476438\\
-2358.68636308143	-0.00945380116082388\\
-2352.82627273837	-0.0105497924085143\\
-2346.96618239531	-0.0113805804089319\\
-2341.10609205225	-0.0119245486059466\\
-2335.24600170919	-0.0121667184788795\\
-2329.38591136613	-0.012099155029404\\
-2323.52582102307	-0.0117212088592692\\
-2317.66573068002	-0.0110395878732599\\
-2311.80564033696	-0.0100682556059625\\
-2305.9455499939	-0.00882815723345431\\
-2300.08545965084	-0.00734677839859645\\
-2294.22536930778	-0.00565754595920392\\
-2288.36527896472	-0.00379908356424504\\
-2282.50518862166	-0.00181433848336023\\
-2276.6450982786	0.00025040072141383\\
-2270.78500793554	0.00234657336730572\\
-2264.92491759248	0.00442447812832633\\
-2259.06482724942	0.00643444061830099\\
-2253.20473690636	0.00832798981830225\\
-2247.3446465633	0.0100590156610343\\
-2241.48455622024	0.0115848803034256\\
-2235.62446587718	0.0128674564878283\\
-2229.76437553412	0.0138740678917618\\
-2223.90428519106	0.0145783084693991\\
-2218.044194848	0.014960720446091\\
-2212.18410450494	0.0150093137914393\\
-2206.32401416188	0.0147199135818754\\
-2200.46392381882	0.0140963256054732\\
-2194.60383347576	0.0131503147588297\\
-2188.74374313271	0.0119013951481239\\
-2182.88365278965	0.0103764352298235\\
-2177.02356244659	0.00860908570863427\\
-2171.16347210353	0.00663904214923177\\
-2165.30338176047	0.00451115825297398\\
-2159.44329141741	0.0022744294019185\\
-2153.58320107435	-1.91307012518548e-05\\
-2147.72311073129	-0.00231569475419591\\
-2141.86302038823	-0.00456088025260645\\
-2136.00293004517	-0.00670103237302143\\
-2130.14283970211	-0.00868450131249065\\
-2124.28274935905	-0.010462883917712\\
-2118.42265901599	-0.0119922000052796\\
-2112.56256867293	-0.0132339750428316\\
-2106.70247832987	-0.0141562028041475\\
-2100.84238798681	-0.0147341641942396\\
-2094.98229764376	-0.0149510816021221\\
-2089.1222073007	-0.0147985918175227\\
-2083.26211695764	-0.0142770246534892\\
-2077.40202661458	-0.013395478866648\\
-2071.54193627152	-0.0121716916531481\\
-2065.68184592846	-0.0106317028143412\\
-2059.8217555854	-0.00880931952503118\\
-2053.96166524234	-0.00674539237829642\\
-2048.10157489928	-0.00448691792266446\\
-2042.24148455622	-0.00208598712897101\\
-2036.38139421316	0.000401396964019029\\
-2030.5213038701	0.0029166058741285\\
-2024.66121352704	0.00539974730823386\\
-2018.80112318398	0.00779107048717147\\
-2012.94103284092	0.0100323825085114\\
-2007.08094249787	0.0120684425051182\\
-2001.22085215481	0.0138483005413317\\
-1995.36076181175	0.0153265491510133\\
-1989.50067146869	0.0164644571434375\\
-1983.64058112563	0.01723095775178\\
-1977.78049078257	0.017603466330637\\
-1971.92040043951	0.0175685065489551\\
-1966.06031009645	0.0171221282966627\\
-1960.20021975339	0.0162701052353912\\
-1954.34012941033	0.0150279049613457\\
-1948.48003906727	0.0134204300100195\\
-1942.61994872421	0.0114815332838823\\
-1936.75985838115	0.00925331681943426\\
-1930.89976803809	0.00678522797837604\\
-1925.03967769503	0.00413297206022388\\
-1919.17958735197	0.00135726484147846\\
-1913.31949700891	-0.00147754743827254\\
-1907.45940666585	-0.00430496970760822\\
-1901.59931632279	-0.00705790764695253\\
-1895.73922597973	-0.00967024958476721\\
-1889.87913563667	-0.0120784439005949\\
-1884.01904529361	-0.0142230348116931\\
-1878.15895495056	-0.0160501199614507\\
-1872.2988646075	-0.0175126946320237\\
-1866.43877426444	-0.0185718496407358\\
-1860.57868392138	-0.0191977930156998\\
-1854.71859357832	-0.019370669308774\\
-1848.85850323526	-0.0190811548232647\\
-1842.9984128922	-0.0183308120121746\\
-1837.13832254914	-0.017132191727909\\
-1831.27823220608	-0.015508677757939\\
-1825.41814186302	-0.0134940740375226\\
-1819.55805151996	-0.0111319409451808\\
-1813.6979611769	-0.00847469302921748\\
-1807.83787083384	-0.00558247624155257\\
-1801.97778049078	-0.0025218481337946\\
-1796.11769014772	0.000635710629293177\\
-1790.25759980466	0.00381542077874518\\
-1784.3975094616	0.0069409287138406\\
-1778.53741911854	0.00993609147400074\\
-1772.67732877548	0.0127267814135153\\
-1766.81723843242	0.0152426685735685\\
-1760.95714808936	0.0174189387001174\\
-1755.09705774631	0.019197905785455\\
-1749.23696740325	0.0205304799005163\\
-1743.37687706019	0.0213774539147755\\
-1737.51678671713	0.0217105764099564\\
-1731.65669637407	0.0215133826033089\\
-1725.79660603101	0.0207817603317231\\
-1719.93651568795	0.0195242339738226\\
-1714.07642534489	0.017761955499055\\
-1708.21633500183	0.0155283984775505\\
-1702.35624465877	0.0128687577205047\\
-1696.49615431571	0.0098390640935434\\
-1690.63606397265	0.00650503079562319\\
-1684.77597362959	0.00294065386667799\\
-1678.91588328653	-0.000773404269295005\\
-1673.05579294348	-0.00455161395270145\\
-1667.19570260042	-0.00830552109696161\\
-1661.33561225736	-0.0119457906926107\\
-1655.4755219143	-0.0153843050391008\\
-1649.61543157124	-0.0185362687434548\\
-1643.75534122818	-0.0213222716012969\\
-1637.89525088512	-0.0236702606628685\\
-1632.03516054206	-0.0255173740877348\\
-1626.175070199	-0.0268115918039416\\
-1620.31497985594	-0.0275131614658803\\
-1614.45488951288	-0.0275957626874273\\
-1608.59479916982	-0.0270473779280049\\
-1602.73470882676	-0.0258708446201176\\
-1596.8746184837	-0.0240840700213118\\
-1591.01452814064	-0.0217198976977746\\
-1585.15443779758	-0.0188256223530538\\
-1579.29434745452	-0.0154621577261232\\
-1573.43425711146	-0.0117028703150683\\
-1567.57416676841	-0.00763209957447714\\
-1561.71407642535	-0.00334339277894496\\
-1555.85398608229	0.00106251020646209\\
-1549.99389573923	0.00547990064286727\\
-1544.13380539617	0.00980048597372192\\
-1538.27371505311	0.013915884500524\\
-1532.41362471005	0.0177201744695551\\
-1526.55353436699	0.0211124400539121\\
-1520.69344402393	0.0239992556573947\\
-1514.83335368087	0.0262970502124335\\
-1508.97326333781	0.027934294692799\\
-1503.11317299475	0.0288534589218024\\
-1497.25308265169	0.0290126878692258\\
-1491.39299230863	0.028387152936888\\
-1485.53290196557	0.0269700401290352\\
-1479.67281162251	0.0247731443769706\\
-1473.81272127945	0.0218270474814445\\
-1467.95263093639	0.0181808659890328\\
-1462.09254059333	0.0139015646557745\\
-1456.23245025027	0.00907284075713785\\
-1450.37235990721	0.00379359419896103\\
-1444.51226956415	-0.00182399207114651\\
-1438.6521792211	-0.00765672763855815\\
-1432.79208887804	-0.013573003818232\\
-1426.93199853498	-0.019435646200381\\
-1421.07190819192	-0.0251049733031036\\
-1415.21181784886	-0.0304419982165962\\
-1409.3517275058	-0.0353117055066393\\
-1403.49163716274	-0.0395863323882404\\
-1397.63154681968	-0.0431485813924892\\
-1391.77145647662	-0.0458946914965311\\
-1385.91136613356	-0.0477372959797141\\
-1380.0512757905	-0.0486079981452268\\
-1374.19118544744	-0.048459600428606\\
-1368.33109510438	-0.0472679282845822\\
-1362.47100476132	-0.0450331974637927\\
-1356.61091441826	-0.0417808817478559\\
-1350.7508240752	-0.0375620477701387\\
-1344.89073373214	-0.032453133982438\\
-1339.03064338909	-0.0265551619885406\\
-1333.17055304603	-0.0199923800795998\\
-1327.31046270297	-0.0129103506877623\\
-1321.45037235991	-0.00547350534665286\\
-1315.59028201685	0.00213779761317895\\
-1309.73019167379	0.0097306662554505\\
-1303.87010133073	0.0171034623045789\\
-1298.01001098767	0.0240499161933233\\
-1292.14992064461	0.0303634952677014\\
-1286.28983030155	0.0358419435443458\\
-1280.42973995849	0.0402919059350923\\
-1274.56964961543	0.0435335460266654\\
-1268.70955927237	0.0454050644083306\\
-1262.84946892931	0.0457670242753695\\
-1256.98937858626	0.0445063926070338\\
-1251.1292882432	0.0415402086416198\\
-1245.26919790014	0.0368187965883548\\
-1239.40910755708	0.030328446479454\\
-1233.54901721402	0.0220934956158966\\
-1227.68892687096	0.012177753102854\\
-1221.8288365279	0.000685221285856634\\
-1215.96874618484	-0.0122399196904163\\
-1210.10865584178	-0.0264140847141357\\
-1204.24856549872	-0.0416158205683189\\
-1198.38847515566	-0.0575880992416865\\
-1192.5283848126	-0.0740413457986789\\
-1186.66829446954	-0.0906571648315894\\
-1180.80820412648	-0.107092715865975\\
-1174.94811378342	-0.122985675223848\\
-1169.08802344036	-0.137959709919137\\
-1163.2279330973	-0.15163037851408\\
-1157.36784275424	-0.163611364615539\\
-1151.50775241118	-0.173520941108077\\
-1145.64766206812	-0.180988557399433\\
-1139.78757172506	-0.185661438070636\\
-1133.927481382	-0.18721107940182\\
-1128.06739103895	-0.185339530454364\\
-1122.20730069589	-0.179785347520509\\
-1116.34721035283	-0.170329115104188\\
-1110.48712000977	-0.156798432711866\\
-1104.62702966671	-0.139072274887151\\
-1098.76693932365	-0.117084641646392\\
-1092.90684898059	-0.0908274278657117\\
-1087.04675863753	-0.0603524528168039\\
-1081.18666829447	-0.0257726048238335\\
-1075.32657795141	0.0127379292904945\\
-1069.46648760835	0.0549443651140472\\
-1063.60639726529	0.100552951453853\\
-1057.74630692223	0.149213515023878\\
-1051.88621657917	0.200522887640001\\
-1046.02612623611	0.254029169754871\\
-1040.16603589305	0.309236770140319\\
-1034.30594554999	0.365612148604064\\
-1028.44585520693	0.422590176891381\\
-1022.58576486387	0.479581022814801\\
-1016.72567452081	0.535977454104738\\
-1010.86558417775	0.591162451709148\\
-1005.00549383469	0.644517017647452\\
-999.145403491639	0.695428059563599\\
-993.285313148579	0.743296233500197\\
-987.425222805519	0.787543627687799\\
-981.56513246246	0.827621173336601\\
-975.7050421194	0.863015673755593\\
-969.84495177634	0.893256349960191\\
-963.98486143328	0.917920809728259\\
-958.124771090221	0.936640357141518\\
-952.264680747161	0.94910457105228\\
-946.404590404101	0.95506509343728\\
-940.544500061042	0.954338581889728\\
-934.684409717982	0.946808794458036\\
-928.824319374922	0.932427789082648\\
-922.964229031862	0.91121623424658\\
-917.104138688806	0.883262841547149\\
-911.244048345747	0.848722944261556\\
-905.383958002687	0.807816259181336\\
-899.523867659627	0.760823880633252\\
-893.663777316568	0.708084566821275\\
-887.803686973508	0.649990387873447\\
-881.943596630448	0.58698181345213\\
-876.083506287388	0.519542324044213\\
-870.223415944329	0.448192635362775\\
-864.363325601269	0.373484628305135\\
-858.503235258209	0.29599507864572\\
-852.64314491515	0.216319280481339\\
-846.78305457209	0.135064655748203\\
-840.92296422903	0.0528444387668077\\
-835.062873885974	-0.0297284798989344\\
-829.202783542914	-0.112047472118835\\
-823.342693199855	-0.193517782451845\\
-817.482602856795	-0.273561767257345\\
-811.622512513735	-0.35162370894472\\
-805.762422170676	-0.427174161292262\\
-799.902331827616	-0.499713791710958\\
-794.042241484556	-0.568776696596996\\
-788.182151141496	-0.633933176002452\\
-782.322060798437	-0.694791963845459\\
-776.461970455377	-0.751001919424186\\
-770.601880112317	-0.802253194759983\\
-764.741789769258	-0.84827790079681\\
-758.881699426198	-0.888850302444771\\
-753.021609083142	-0.923786578908758\\
-747.161518740082	-0.952944191032698\\
-741.301428397022	-0.976220901207955\\
-735.441338053963	-0.993553494367116\\
-729.581247710903	-1.00491625035163\\
-723.721157367843	-1.01031921814018\\
-717.861067024784	-1.00980634184636\\
-712.000976681724	-1.00345348679411\\
-706.140886338664	-0.991366410834391\\
-700.280795995604	-0.973678722777293\\
-694.420705652545	-0.950549865111324\\
-688.560615309485	-0.922163153296808\\
-682.700524966425	-0.888723898274073\\
-676.840434623366	-0.85045763296809\\
-670.980344280306	-0.807608457584923\\
-665.12025393725	-0.760437512269963\\
-659.26016359419	-0.709221579942353\\
-653.40007325113	-0.654251816316046\\
-647.539982908071	-0.595832598875509\\
-641.679892565011	-0.534280481878652\\
-635.819802221951	-0.469923240314452\\
-629.959711878892	-0.403098982382879\\
-624.099621535832	-0.334155307436597\\
-618.239531192772	-0.263448484581125\\
-612.379440849712	-0.191342626164042\\
-606.519350506653	-0.118208830417588\\
-600.659260163593	-0.0444242682066619\\
-594.799169820533	0.0296288094771863\\
-588.939079477474	0.103564164336577\\
-583.078989134417	0.176992781905603\\
-577.218898791358	0.24952421539293\\
-571.358808448298	0.320768072493506\\
-565.498718105238	0.390335650935024\\
-559.638627762179	0.457841723775397\\
};
\addplot [color=mycolor2, forget plot]
  table[row sep=crcr]{%
-559.638627762179	0.457841723775397\\
-553.778537419119	0.522906470529288\\
-547.918447076059	0.585157545364451\\
-542.058356733	0.644232268641514\\
-536.19826638994	0.699779923750387\\
-530.33817604688	0.751464136600138\\
-524.47808570382	0.798965311660523\\
-518.617995360761	0.841983094946846\\
-512.757905017701	0.8802388317818\\
-506.897814674641	0.913477985140558\\
-501.037724331585	0.941472479198873\\
-495.177633988525	0.96402293195531\\
-489.317543645466	0.980960741453607\\
-483.457453302406	0.992149990584131\\
-477.597362959346	0.99748913773928\\
-471.737272616287	0.996912462498718\\
-465.877182273227	0.990391238982957\\
-460.017091930167	0.977934612900703\\
-454.157001587108	0.959590162529187\\
-448.296911244048	0.935444128356586\\
-442.436820900988	0.905621300974736\\
-436.576730557928	0.870284561832645\\
-430.716640214869	0.829634076516639\\
-424.856549871809	0.783906145453829\\
-418.996459528749	0.733371721814328\\
-413.136369185693	0.678334611335258\\
-407.276278842633	0.619129373073812\\
-401.416188499574	0.556118944399546\\
-395.556098156514	0.489692017017881\\
-389.696007813454	0.420260194002688\\
-383.835917470395	0.348254960325842\\
-377.975827127335	0.274124501354156\\
-372.115736784275	0.198330404927245\\
-366.255646441216	0.121344283487174\\
-360.395556098156	0.0436443525862737\\
-354.535465755096	-0.0342879983924338\\
-348.675375412036	-0.111971608682172\\
-342.815285068977	-0.188928888795259\\
-336.955194725917	-0.264689072635429\\
-331.095104382861	-0.338791350891377\\
-325.235014039801	-0.41078786043124\\
-319.374923696741	-0.480246507654551\\
-313.514833353682	-0.546753607071468\\
-307.654743010622	-0.609916319920112\\
-301.794652667562	-0.669364881091267\\
-295.934562324503	-0.724754606017306\\
-290.074471981443	-0.775767672645932\\
-284.214381638383	-0.822114676567233\\
-278.354291295323	-0.863535960352843\\
-272.494200952264	-0.899802720626777\\
-266.634110609204	-0.930717898753082\\
-260.774020266144	-0.956116862678604\\
-254.913929923085	-0.975867889088106\\
-249.053839580029	-0.989872456064077\\
-243.193749236969	-0.998065356918254\\
-237.333658893909	-1.00041464654236\\
-231.473568550849	-0.996921431105025\\
-225.61347820779	-0.987619511824232\\
-219.75338786473	-0.972574892848291\\
-213.89329752167	-0.951885162271285\\
-208.033207178611	-0.925678754301471\\
-202.173116835551	-0.894114099366228\\
-196.313026492491	-0.85737866777532\\
-190.452936149431	-0.815687911153543\\
-184.592845806372	-0.769284104693474\\
-178.732755463312	-0.718435092217464\\
-172.872665120252	-0.663432935071271\\
-167.012574777196	-0.604592464941082\\
-161.152484434137	-0.542249740482189\\
-155.292394091077	-0.476760407068243\\
-149.432303748017	-0.408497959180072\\
-143.572213404957	-0.337851905193855\\
-137.712123061898	-0.265225834903588\\
-131.852032718838	-0.19103539109517\\
-125.991942375778	-0.115706147506575\\
-120.131852032719	-0.0396713969892807\\
-114.271761689659	0.0366301447516272\\
-108.411671346599	0.11275671271717\\
-102.551581003539	0.18826593447527\\
-96.6914906604798	0.262717329394353\\
-90.83140031742	0.335674852387291\\
-84.9713099743603	0.406709467386166\\
-79.1112196313043	0.475401734021133\\
-73.2511292882446	0.541344389310977\\
-67.3910389451848	0.604144904758838\\
-61.5309486021251	0.663427998087187\\
-55.6708582590654	0.718838078007769\\
-49.8107679160057	0.770041599858514\\
-43.950677572946	0.816729309733349\\
-38.0905872298863	0.858618355104467\\
-32.2304968868266	0.895454240302984\\
-26.3704065437669	0.927012606394028\\
-20.5103162007072	0.953100816365631\\
-14.6502258576475	0.973559328006949\\
-8.79013551458775	0.988262839354347\\
-2.93004517152804	0.997121193545325\\
2.93004517152804	1.00008003273341\\
8.79013551458775	0.997121193545325\\
14.6502258576475	0.988262839354347\\
20.5103162007072	0.973559328006949\\
26.3704065437669	0.953100816365631\\
32.2304968868266	0.927012606394028\\
38.0905872298863	0.895454240302984\\
43.950677572946	0.858618355104467\\
49.8107679160057	0.816729309733349\\
55.6708582590654	0.770041599858514\\
61.5309486021251	0.718838078007769\\
67.3910389451848	0.663427998087187\\
73.2511292882446	0.604144904758838\\
79.1112196313043	0.541344389310977\\
84.9713099743603	0.475401734021133\\
90.83140031742	0.406709467386166\\
96.6914906604798	0.335674852387291\\
102.551581003539	0.262717329394353\\
108.411671346599	0.18826593447527\\
114.271761689659	0.11275671271717\\
120.131852032719	0.0366301447516272\\
125.991942375778	-0.0396713969892807\\
131.852032718838	-0.115706147506575\\
137.712123061898	-0.19103539109517\\
143.572213404957	-0.265225834903588\\
149.432303748017	-0.337851905193855\\
155.292394091077	-0.408497959180072\\
161.152484434137	-0.476760407068243\\
167.012574777196	-0.542249740482189\\
172.872665120252	-0.604592464941082\\
178.732755463312	-0.663432935071271\\
184.592845806372	-0.718435092217464\\
190.452936149431	-0.769284104693474\\
196.313026492491	-0.815687911153543\\
202.173116835551	-0.85737866777532\\
208.033207178611	-0.894114099366228\\
213.89329752167	-0.925678754301471\\
219.75338786473	-0.951885162271285\\
225.61347820779	-0.972574892848291\\
231.473568550849	-0.987619511824232\\
237.333658893909	-0.996921431105025\\
243.193749236969	-1.00041464654236\\
249.053839580029	-0.998065356918254\\
254.913929923085	-0.989872456064077\\
260.774020266144	-0.975867889088106\\
266.634110609204	-0.956116862678604\\
272.494200952264	-0.930717898753082\\
278.354291295323	-0.899802720626777\\
284.214381638383	-0.863535960352843\\
290.074471981443	-0.822114676567233\\
295.934562324503	-0.775767672645932\\
301.794652667562	-0.724754606017306\\
307.654743010622	-0.669364881091267\\
313.514833353682	-0.609916319920112\\
319.374923696741	-0.546753607071468\\
325.235014039801	-0.480246507654551\\
331.095104382861	-0.41078786043124\\
336.955194725917	-0.338791350891377\\
342.815285068977	-0.264689072635429\\
348.675375412036	-0.188928888795259\\
354.535465755096	-0.111971608682172\\
360.395556098156	-0.0342879983924338\\
366.255646441216	0.0436443525862737\\
372.115736784275	0.121344283487174\\
377.975827127335	0.198330404927245\\
383.835917470395	0.274124501354156\\
389.696007813454	0.348254960325842\\
395.556098156514	0.420260194002688\\
401.416188499574	0.489692017017881\\
407.276278842633	0.556118944399546\\
413.136369185693	0.619129373073812\\
418.996459528749	0.678334611335258\\
424.856549871809	0.733371721814328\\
430.716640214869	0.783906145453829\\
436.576730557928	0.829634076516639\\
442.436820900988	0.870284561832645\\
448.296911244048	0.905621300974736\\
454.157001587108	0.935444128356586\\
460.017091930167	0.959590162529187\\
465.877182273227	0.977934612900703\\
471.737272616287	0.990391238982957\\
477.597362959346	0.996912462498718\\
483.457453302406	0.99748913773928\\
489.317543645466	0.992149990584131\\
495.177633988525	0.980960741453607\\
501.037724331585	0.96402293195531\\
506.897814674641	0.941472479198873\\
512.757905017701	0.913477985140558\\
518.617995360761	0.8802388317818\\
524.47808570382	0.841983094946846\\
530.33817604688	0.798965311660523\\
536.19826638994	0.751464136600138\\
542.058356733	0.699779923750387\\
547.918447076059	0.644232268641514\\
553.778537419119	0.585157545364451\\
559.638627762179	0.522906470529288\\
565.498718105238	0.457841723775397\\
571.358808448298	0.390335650935024\\
577.218898791358	0.320768072493506\\
583.078989134417	0.24952421539293\\
588.939079477474	0.176992781905603\\
594.799169820533	0.103564164336577\\
600.659260163593	0.0296288094771863\\
606.519350506653	-0.0444242682066619\\
612.379440849712	-0.118208830417588\\
618.239531192772	-0.191342626164042\\
624.099621535832	-0.263448484581125\\
629.959711878892	-0.334155307436597\\
635.819802221951	-0.403098982382879\\
641.679892565011	-0.469923240314452\\
647.539982908071	-0.534280481878652\\
653.40007325113	-0.595832598875509\\
659.26016359419	-0.654251816316046\\
665.12025393725	-0.709221579942353\\
670.980344280306	-0.760437512269963\\
676.840434623366	-0.807608457584923\\
682.700524966425	-0.85045763296809\\
688.560615309485	-0.888723898274073\\
694.420705652545	-0.922163153296808\\
700.280795995604	-0.950549865111324\\
706.140886338664	-0.973678722777293\\
712.000976681724	-0.991366410834391\\
717.861067024784	-1.00345348679411\\
723.721157367843	-1.00980634184636\\
729.581247710903	-1.01031921814018\\
735.441338053963	-1.00491625035163\\
741.301428397022	-0.993553494367116\\
747.161518740082	-0.976220901207955\\
753.021609083142	-0.952944191032698\\
758.881699426198	-0.923786578908758\\
764.741789769258	-0.888850302444771\\
770.601880112317	-0.84827790079681\\
776.461970455377	-0.802253194759983\\
782.322060798437	-0.751001919424186\\
788.182151141496	-0.694791963845459\\
794.042241484556	-0.633933176002452\\
799.902331827616	-0.568776696596996\\
805.762422170676	-0.499713791710958\\
811.622512513735	-0.427174161292262\\
817.482602856795	-0.35162370894472\\
823.342693199855	-0.273561767257345\\
829.202783542914	-0.193517782451845\\
835.062873885974	-0.112047472118835\\
840.92296422903	-0.0297284798989344\\
846.78305457209	0.0528444387668077\\
852.64314491515	0.135064655748203\\
858.503235258209	0.216319280481339\\
864.363325601269	0.29599507864572\\
870.223415944329	0.373484628305135\\
876.083506287388	0.448192635362775\\
881.943596630448	0.519542324044213\\
887.803686973508	0.58698181345213\\
893.663777316568	0.649990387873447\\
899.523867659627	0.708084566821275\\
905.383958002687	0.760823880633252\\
911.244048345747	0.807816259181336\\
917.104138688806	0.848722944261556\\
922.964229031862	0.883262841547149\\
928.824319374922	0.91121623424658\\
934.684409717982	0.932427789082648\\
940.544500061042	0.946808794458036\\
946.404590404101	0.954338581889728\\
952.264680747161	0.95506509343728\\
958.124771090221	0.94910457105228\\
963.98486143328	0.936640357141518\\
969.84495177634	0.917920809728259\\
975.7050421194	0.893256349960191\\
981.56513246246	0.863015673755593\\
987.425222805519	0.827621173336601\\
993.285313148579	0.787543627687799\\
999.145403491639	0.743296233500197\\
1005.00549383469	0.695428059563599\\
1010.86558417775	0.644517017647452\\
1016.72567452081	0.591162451709148\\
1022.58576486387	0.535977454104738\\
1028.44585520693	0.479581022814801\\
1034.30594554999	0.422590176891381\\
1040.16603589305	0.365612148604064\\
1046.02612623611	0.309236770140319\\
1051.88621657917	0.254029169754871\\
1057.74630692223	0.200522887640001\\
1063.60639726529	0.149213515023878\\
1069.46648760835	0.100552951453853\\
1075.32657795141	0.0549443651140472\\
1081.18666829447	0.0127379292904945\\
1087.04675863753	-0.0257726048238335\\
1092.90684898059	-0.0603524528168039\\
1098.76693932365	-0.0908274278657117\\
1104.62702966671	-0.117084641646392\\
1110.48712000977	-0.139072274887151\\
1116.34721035283	-0.156798432711866\\
1122.20730069589	-0.170329115104188\\
1128.06739103895	-0.179785347520509\\
1133.927481382	-0.185339530454364\\
1139.78757172506	-0.18721107940182\\
1145.64766206812	-0.185661438070636\\
1151.50775241118	-0.180988557399433\\
1157.36784275424	-0.173520941108077\\
1163.2279330973	-0.163611364615539\\
1169.08802344036	-0.15163037851408\\
1174.94811378342	-0.137959709919137\\
1180.80820412648	-0.122985675223848\\
1186.66829446954	-0.107092715865975\\
1192.5283848126	-0.0906571648315894\\
1198.38847515566	-0.0740413457986789\\
1204.24856549872	-0.0575880992416865\\
1210.10865584178	-0.0416158205683189\\
1215.96874618484	-0.0264140847141357\\
1221.8288365279	-0.0122399196904163\\
1227.68892687096	0.000685221285856634\\
1233.54901721402	0.012177753102854\\
1239.40910755708	0.0220934956158966\\
1245.26919790014	0.030328446479454\\
1251.1292882432	0.0368187965883548\\
1256.98937858626	0.0415402086416198\\
1262.84946892931	0.0445063926070338\\
1268.70955927237	0.0457670242753695\\
1274.56964961543	0.0454050644083306\\
1280.42973995849	0.0435335460266654\\
1286.28983030155	0.0402919059350923\\
1292.14992064461	0.0358419435443458\\
1298.01001098767	0.0303634952677014\\
1303.87010133073	0.0240499161933233\\
1309.73019167379	0.0171034623045789\\
1315.59028201685	0.0097306662554505\\
1321.45037235991	0.00213779761317895\\
1327.31046270297	-0.00547350534665286\\
1333.17055304603	-0.0129103506877623\\
1339.03064338909	-0.0199923800795998\\
1344.89073373214	-0.0265551619885406\\
1350.7508240752	-0.032453133982438\\
1356.61091441826	-0.0375620477701387\\
1362.47100476132	-0.0417808817478559\\
1368.33109510438	-0.0450331974637927\\
1374.19118544744	-0.0472679282845822\\
1380.0512757905	-0.048459600428606\\
1385.91136613356	-0.0486079981452268\\
1391.77145647662	-0.0477372959797141\\
1397.63154681968	-0.0458946914965311\\
1403.49163716274	-0.0431485813924892\\
1409.3517275058	-0.0395863323882404\\
1415.21181784886	-0.0353117055066393\\
1421.07190819192	-0.0304419982165962\\
1426.93199853498	-0.0251049733031036\\
1432.79208887804	-0.019435646200381\\
1438.6521792211	-0.013573003818232\\
1444.51226956415	-0.00765672763855815\\
1450.37235990721	-0.00182399207114651\\
1456.23245025027	0.00379359419896103\\
1462.09254059333	0.00907284075713785\\
1467.95263093639	0.0139015646557745\\
1473.81272127945	0.0181808659890328\\
1479.67281162251	0.0218270474814445\\
1485.53290196557	0.0247731443769706\\
1491.39299230863	0.0269700401290352\\
1497.25308265169	0.028387152936888\\
1503.11317299475	0.0290126878692258\\
1508.97326333781	0.0288534589218024\\
1514.83335368087	0.027934294692799\\
1520.69344402393	0.0262970502124335\\
1526.55353436699	0.0239992556573947\\
1532.41362471005	0.0211124400539121\\
1538.27371505311	0.0177201744695551\\
1544.13380539617	0.013915884500524\\
1549.99389573923	0.00980048597372192\\
1555.85398608229	0.00547990064286727\\
1561.71407642535	0.00106251020646209\\
1567.57416676841	-0.00334339277894496\\
1573.43425711146	-0.00763209957447714\\
1579.29434745452	-0.0117028703150683\\
1585.15443779758	-0.0154621577261232\\
1591.01452814064	-0.0188256223530538\\
1596.8746184837	-0.0217198976977746\\
1602.73470882676	-0.0240840700213118\\
1608.59479916982	-0.0258708446201176\\
1614.45488951288	-0.0270473779280049\\
1620.31497985594	-0.0275957626874273\\
1626.175070199	-0.0275131614658803\\
1632.03516054206	-0.0268115918039416\\
1637.89525088512	-0.0255173740877348\\
1643.75534122818	-0.0236702606628685\\
1649.61543157124	-0.0213222716012969\\
1655.4755219143	-0.0185362687434548\\
1661.33561225736	-0.0153843050391008\\
1667.19570260042	-0.0119457906926107\\
1673.05579294348	-0.00830552109696161\\
1678.91588328653	-0.00455161395270145\\
1684.77597362959	-0.000773404269295005\\
1690.63606397265	0.00294065386667799\\
1696.49615431571	0.00650503079562319\\
1702.35624465877	0.0098390640935434\\
1708.21633500183	0.0128687577205047\\
1714.07642534489	0.0155283984775505\\
1719.93651568795	0.017761955499055\\
1725.79660603101	0.0195242339738226\\
1731.65669637407	0.0207817603317231\\
1737.51678671713	0.0215133826033089\\
1743.37687706019	0.0217105764099564\\
1749.23696740325	0.0213774539147755\\
1755.09705774631	0.0205304799005163\\
1760.95714808936	0.019197905785455\\
1766.81723843242	0.0174189387001174\\
1772.67732877548	0.0152426685735685\\
1778.53741911854	0.0127267814135153\\
1784.3975094616	0.00993609147400074\\
1790.25759980466	0.0069409287138406\\
1796.11769014772	0.00381542077874518\\
1801.97778049078	0.000635710629293177\\
1807.83787083384	-0.0025218481337946\\
1813.6979611769	-0.00558247624155257\\
1819.55805151996	-0.00847469302921748\\
1825.41814186302	-0.0111319409451808\\
1831.27823220608	-0.0134940740375226\\
1837.13832254914	-0.015508677757939\\
1842.9984128922	-0.017132191727909\\
1848.85850323526	-0.0183308120121746\\
1854.71859357832	-0.0190811548232647\\
1860.57868392138	-0.019370669308774\\
1866.43877426444	-0.0191977930156998\\
1872.2988646075	-0.0185718496407358\\
1878.15895495056	-0.0175126946320237\\
1884.01904529361	-0.0160501199614507\\
1889.87913563667	-0.0142230348116931\\
1895.73922597973	-0.0120784439005949\\
1901.59931632279	-0.00967024958476721\\
1907.45940666585	-0.00705790764695253\\
1913.31949700891	-0.00430496970760822\\
1919.17958735197	-0.00147754743827254\\
1925.03967769503	0.00135726484147846\\
1930.89976803809	0.00413297206022388\\
1936.75985838115	0.00678522797837604\\
1942.61994872421	0.00925331681943426\\
1948.48003906727	0.0114815332838823\\
1954.34012941033	0.0134204300100195\\
1960.20021975339	0.0150279049613457\\
1966.06031009645	0.0162701052353912\\
1971.92040043951	0.0171221282966627\\
1977.78049078257	0.0175685065489551\\
1983.64058112563	0.017603466330637\\
1989.50067146869	0.01723095775178\\
1995.36076181175	0.0164644571434375\\
2001.22085215481	0.0153265491510133\\
2007.08094249787	0.0138483005413317\\
2012.94103284092	0.0120684425051182\\
2018.80112318398	0.0100323825085114\\
2024.66121352704	0.00779107048717147\\
2030.5213038701	0.00539974730823386\\
2036.38139421316	0.0029166058741285\\
2042.24148455622	0.000401396964019029\\
2048.10157489928	-0.00208598712897101\\
2053.96166524234	-0.00448691792266446\\
2059.8217555854	-0.00674539237829642\\
2065.68184592846	-0.00880931952503118\\
2071.54193627152	-0.0106317028143412\\
2077.40202661458	-0.0121716916531481\\
2083.26211695764	-0.013395478866648\\
2089.1222073007	-0.0142770246534892\\
2094.98229764376	-0.0147985918175227\\
2100.84238798681	-0.0149510816021221\\
2106.70247832987	-0.0147341641942396\\
2112.56256867293	-0.0141562028041475\\
2118.42265901599	-0.0132339750428316\\
2124.28274935905	-0.0119922000052796\\
2130.14283970211	-0.010462883917712\\
2136.00293004517	-0.00868450131249065\\
2141.86302038823	-0.00670103237302143\\
2147.72311073129	-0.00456088025260645\\
2153.58320107435	-0.00231569475419591\\
2159.44329141741	-1.91307012518548e-05\\
2165.30338176047	0.0022744294019185\\
2171.16347210353	0.00451115825297398\\
2177.02356244659	0.00663904214923177\\
2182.88365278965	0.00860908570863427\\
2188.74374313271	0.0103764352298235\\
2194.60383347576	0.0119013951481239\\
2200.46392381882	0.0131503147588297\\
2206.32401416188	0.0140963256054732\\
2212.18410450494	0.0147199135818754\\
2218.044194848	0.0150093137914393\\
2223.90428519106	0.014960720446091\\
2229.76437553412	0.0145783084693991\\
2235.62446587718	0.0138740678917618\\
2241.48455622024	0.0128674564878283\\
2247.3446465633	0.0115848803034256\\
2253.20473690636	0.0100590156610343\\
2259.06482724942	0.00832798981830225\\
2264.92491759248	0.00643444061830099\\
2270.78500793554	0.00442447812832633\\
2276.6450982786	0.00234657336730572\\
2282.50518862166	0.00025040072141383\\
2288.36527896472	-0.00181433848336023\\
2294.22536930778	-0.00379908356424504\\
2300.08545965084	-0.00565754595920392\\
2305.9455499939	-0.00734677839859645\\
2311.80564033696	-0.00882815723345431\\
2317.66573068002	-0.0100682556059625\\
2323.52582102307	-0.0110395878732599\\
2329.38591136613	-0.0117212088592692\\
2335.24600170919	-0.012099155029404\\
2341.10609205225	-0.0121667184788795\\
2346.96618239531	-0.0119245486059466\\
2352.82627273837	-0.0113805804089319\\
2358.68636308143	-0.0105497924085143\\
2364.54645342449	-0.00945380116082388\\
2370.40654376755	-0.00812030309476438\\
2376.26663411061	-0.00658237789869062\\
2382.12672445367	-0.00487767081481468\\
2387.98681479673	-0.00304747390285287\\
2393.84690513979	-0.00113572855030081\\
2399.70699548285	0.000812026814088753\\
2405.56708582591	0.0027497387414227\\
2411.42717616897	0.00463192494249867\\
2417.28726651203	0.00641473913918165\\
2423.14735685509	0.00805698978402546\\
2429.00744719815	0.00952108910289104\\
2434.8675375412	0.0107739107154888\\
2440.72762788426	0.0117875363843739\\
2446.58771822732	0.0125398751761778\\
2452.44780857038	0.0130151414254016\\
2458.30789891344	0.013204181292042\\
2464.1679892565	0.0131046413218598\\
2470.02807959956	0.0127209761646408\\
2475.88816994262	0.0120642963894758\\
2481.74826028568	0.0111520610785373\\
2487.60835062874	0.0100076234812093\\
2493.4684409718	0.00865964139479158\\
2499.32853131486	0.0071413670297236\\
2505.18862165792	0.00548983383634214\\
2511.04871200098	0.00374496006777452\\
2516.90880234403	0.00194859066912146\\
2522.76889268709	0.000143500384176332\\
2528.62898303015	-0.00162761827576789\\
2534.48907337321	-0.00332315835780078\\
2540.34916371627	-0.00490356960841335\\
2546.20925405933	-0.00633227485948068\\
2552.06934440239	-0.00757651052092272\\
2557.92943474545	-0.00860807195946099\\
2563.78952508851	-0.00940394689773573\\
2569.64961543157	-0.00994682270303402\\
2575.50970577463	-0.0102254564807398\\
2581.36979611769	-0.010234900178849\\
2587.22988646075	-0.00997657635428583\\
2593.08997680381	-0.00945820378437724\\
2598.95006714687	-0.00869357562946508\\
2604.81015748993	-0.00770219629457858\\
2610.67024783299	-0.00650878641024377\\
2616.53033817605	-0.00514266838524258\\
2622.39042851911	-0.00363704770804635\\
2628.25051886217	-0.00202820751994197\\
2634.11060920523	-0.000354635910199807\\
2639.97069954828	0.00134389316210531\\
2645.83078989134	0.00302726345532345\\
2651.6908802344	0.00465596223862386\\
2657.55097057746	0.00619200607336325\\
2663.41106092052	0.00759982497022164\\
2669.27115126358	0.00884708441849526\\
2675.13124160664	0.00990542636467931\\
2680.9913319497	0.0107511122353016\\
2686.85142229276	0.0113655534972052\\
2692.71151263582	0.0117357179755324\\
2698.57160297888	0.0118544031335403\\
2704.43169332194	0.0117203706928083\\
2710.291783665	0.0113383402598657\\
2716.15187400806	0.0107188429453168\\
2722.01196435112	0.00987793924262647\\
2727.87205469418	0.0088368085913962\\
2733.73214503724	0.0076212210206767\\
2739.5922353803	0.00626090397022931\\
2745.45232572336	0.00478881977361752\\
2751.31241606642	0.00324037129351002\\
2757.17250640948	0.00165255478330148\\
2763.03259675254	6.30801788299924e-05\\
2768.89268709559	-0.00149052032857914\\
2774.75277743865	-0.00297177442992492\\
2780.61286778171	-0.00434612187831032\\
2786.47295812477	-0.00558171686171607\\
2792.33304846783	-0.00665016200192275\\
2798.19313881089	-0.00752715712211438\\
2804.05322915395	-0.00819304802201049\\
2809.91331949701	-0.00863326292849645\\
2815.77340984007	-0.00883862699654855\\
2821.63350018313	-0.00880554815098393\\
2827.49359052619	-0.00853607061840782\\
2833.35368086925	-0.00803779562351115\\
2839.21377121231	-0.00732367184756998\\
2845.07386155537	-0.00641166128931641\\
2850.93395189842	-0.00532428906249935\\
2856.79404224148	-0.00408808834539056\\
2862.65413258454	-0.00273295409464437\\
2868.5142229276	-0.00129142120849967\\
2874.37431327066	0.00020211548993306\\
2880.23440361372	0.00171222080351519\\
2886.09449395678	0.00320325877352022\\
2891.95458429984	0.00464023251354548\\
2897.8146746429	0.0059896045260749\\
2903.67476498596	0.00722007832548872\\
2909.53485532902	0.00830332320286377\\
2915.39494567208	0.00921462539855497\\
2921.25503601514	0.00993345076560541\\
2927.1151263582	0.0104439061639191\\
2932.97521670126	0.0107350892715334\\
2938.83530704432	0.0108013191729026\\
2944.69539738738	0.0106422429257624\\
2950.55548773043	0.0102628162442474\\
2956.41557807349	0.00967315940618472\\
2962.27566841655	0.00888829241280049\\
2968.13575875961	0.00792775625113928\\
2973.99584910267	0.00681512974368829\\
2979.85593944573	0.0055774538797864\\
2985.71602978879	0.00424457763248576\\
2991.57612013185	0.00284844103973382\\
2997.43621047491	0.00142231272316089\\
nan	nan\\
3009.15639116103	-0.00138494970234729\\
3015.01648150409	-0.00270003730960112\\
3020.87657184715	-0.00391457195439474\\
3026.73666219021	-0.0050003844974638\\
3032.59675253327	-0.00593247817007214\\
3038.45684287633	-0.00668960135077654\\
3044.31693321939	-0.00725472939405607\\
3050.17702356245	-0.00761544462933128\\
3056.03711390551	-0.007764206095785\\
3061.89720424857	-0.00769850321069726\\
3067.75729459163	-0.0074208903251861\\
3073.61738493469	-0.00693890193505245\\
3079.47747527774	-0.0062648511204916\\
3085.3375656208	-0.00541551652062163\\
3091.19765596386	-0.00441172574216919\\
3097.05774630692	-0.00327784549821102\\
3102.91783664998	-0.00204119091566028\\
3108.77792699304	-0.00073136829003325\\
3114.6380173361	0.00062043293757984\\
3120.49810767916	0.00198218208787227\\
3126.35819802222	0.00332176542560803\\
3132.21828836528	0.00460774270801342\\
3138.07837870834	0.00581008431329038\\
3143.9384690514	0.00690087170401437\\
3149.79855939446	0.00785494492196435\\
3155.65864973752	0.00865048213370397\\
3161.51874008058	0.00926949791224047\\
3167.37883042364	0.00969824891316643\\
3173.2389207667	0.00992753783331191\\
3179.09901110976	0.00995290897301492\\
3184.95910145281	0.00977473130189391\\
3190.81919179587	0.00939816759370712\\
3196.67928213893	0.00883303087731186\\
3202.53937248199	0.00809353209863417\\
3208.39946282505	0.00719792542440139\\
3214.25955316811	0.00616805999901185\\
3220.11964351117	0.00502884912276843\\
3225.97973385423	0.00380766971341746\\
3231.83982419729	0.00253370649345193\\
3237.69991454035	0.0012372565793532\\
3243.56000488341	-5.09889885340821e-05\\
3249.42009522647	-0.00130066977059606\\
3255.28018556953	-0.0024824694344331\\
3261.14027591259	-0.00356880150643534\\
3267.00036625564	-0.00453445135847831\\
3272.8604565987	-0.00535715951998051\\
3278.72054694176	-0.00601813285091013\\
3284.58063728482	-0.00650247186787117\\
3290.44072762788	-0.00679950453741863\\
3296.30081797094	-0.00690301909406241\\
3302.160908314	-0.00681139084674552\\
3308.02099865706	-0.00652760045285251\\
3313.88108900012	-0.00605914370243496\\
3319.74117934318	-0.00541783540592309\\
3325.60126968624	-0.00461951245901207\\
3331.4613600293	-0.00368364350610682\\
3337.32145037236	-0.00263285479000728\\
3343.18154071542	-0.00149238370437206\\
3349.04163105848	-0.000289473216962629\\
3354.90172140154	0.000947278333604621\\
3360.7618117446	0.00218859657776847\\
3366.62190208766	0.00340522348184922\\
3372.48199243072	0.00456860639406005\\
3378.34208277378	0.00565156756964951\\
3384.20217311684	0.00662893850202409\\
3390.06226345989	0.00747814428085077\\
3395.92235380295	0.00817972443606576\\
3401.78244414601	0.00871777827683565\\
3407.64253448907	0.00908032456041693\\
3413.50262483213	0.00925956738492283\\
3419.36271517519	0.00925206243771255\\
3425.22280551825	0.00905878010104059\\
3431.08289586131	0.00868506435612603\\
3436.94298620437	0.00814048888322185\\
3442.80307654743	0.0074386141681885\\
3448.66316689049	0.00659665173717525\\
3454.52325723355	0.00563504380392908\\
3460.38334757661	0.00457696856755867\\
3466.24343791967	0.00344778310696529\\
3472.10352826273	0.00227441723806101\\
3477.96361860579	0.00108473279853733\\
3483.82370894885	-9.3136419914711e-05\\
3489.68379929191	-0.00123144847820071\\
3495.54388963497	-0.00230350519651384\\
3501.40397997803	-0.00328427686532124\\
3507.26407032109	-0.0041509851657094\\
3513.12416066415	-0.00488363074642171\\
3518.9842510072	-0.00546545326007842\\
3524.84434135026	-0.00588331330029163\\
3530.70443169332	-0.00612798755876339\\
3536.56452203638	-0.00619437060029849\\
3542.42461237944	-0.00608157887400178\\
3548.2847027225	-0.00579295489945733\\
3554.14479306556	-0.00533597192109993\\
3560.00488340862	-0.00472204166663773\\
3565.86497375168	-0.00396623011362101\\
3571.72506409474	-0.00308688831490519\\
3577.5851544378	-0.00210520730384014\\
3583.44524478086	-0.00104470785231902\\
3589.30533512392	6.93226548304334e-05\\
3595.16542546698	0.00121043276809243\\
3601.02551581004	0.00235163460776858\\
3606.88560615309	0.00346604171332176\\
3612.74569649615	0.00452750205278483\\
3618.60578683921	0.00551121130918611\\
3624.46587718227	0.00639429208248266\\
3630.32596752533	0.00715632550136256\\
3636.18605786839	0.00777982291194471\\
3642.04614821145	0.00825062676708455\\
3647.90623855451	0.00855823154619699\\
3653.76632889757	0.00869601745434875\\
3659.62641924063	0.00866139172788757\\
3665.48650958369	0.00845583456773837\\
3671.34659992675	0.00808484897791946\\
3677.20669026981	0.00755781605084114\\
3683.06678061287	0.00688775946002173\\
3688.92687095593	0.00609102504420597\\
3694.78696129899	0.00518688334262058\\
3700.64705164204	0.00419706472580414\\
3706.5071419851	0.00314523831719472\\
3712.36723232816	0.00205644718068208\\
3718.22732267122	0.000956513234835723\\
3724.08741301428	-0.000128573984568956\\
3729.94750335734	-0.00117327026290317\\
3735.8075937004	-0.00215307690828768\\
3741.66768404346	-0.00304511413168779\\
3747.52777438652	-0.00382865420313579\\
3753.38786472958	-0.00448560197798168\\
3759.24795507264	-0.00500091166995849\\
3765.1080454157	-0.00536293029028585\\
3770.96813575876	-0.00556365993117846\\
3776.82822610182	-0.00559893301236882\\
3782.68831644488	-0.00546849667736869\\
3788.54840678794	-0.0051760046804829\\
3794.408497131	-0.00472891728670607\\
3800.26858747406	-0.00413831187208375\\
3806.12867781712	-0.00341860900343866\\
3811.98876816018	-0.00258722074885845\\
3817.84885850324	-0.00166412977740159\\
3823.7089488463	-0.00067140940488208\\
3829.56903918935	0.000367303903315773\\
3835.42912953241	0.00142737298294409\\
3841.28921987547	0.00248374400730378\\
3847.14931021853	0.00351153828073021\\
3853.00940056159	0.00448663782634899\\
3858.86949090465	0.00538625100051817\\
3864.72958124771	0.00618944488518528\\
3870.58967159077	0.00687763203637733\\
3876.44976193383	0.00743500028746927\\
3882.30985227689	0.00784887568288021\\
3888.16994261995	0.00811001022850894\\
3894.03003296301	0.00821278794353416\\
3899.89012330607	0.00815534464361101\\
3905.75021364913	0.00793959893294186\\
3911.61030399219	0.00757119398092691\\
3917.47039433525	0.00705935176191942\\
3923.33048467831	0.00641664349010055\\
3929.19057502137	0.00565868194429951\\
3935.05066536443	0.00480374319374668\\
3940.91075570748	0.00387232687513467\\
3946.77084605054	0.00288666558511696\\
3952.6309363936	0.00187019511530769\\
3958.49102673666	0.000846998138636516\\
3964.35111707972	-0.000158765461476173\\
3970.21120742278	-0.0011234281582107\\
3976.07129776584	-0.00202437083091371\\
3981.9313881089	-0.00284055224740289\\
3987.79147845196	-0.00355299959823261\\
3993.65156879502	-0.00414524866070209\\
3999.51165913808	-0.00460372339460904\\
4005.37174948114	-0.00491804623215694\\
4011.2318398242	-0.00508127198444811\\
4017.09193016726	-0.00509004011016201\\
4022.95202051031	-0.00494464203082911\\
4028.81211085337	-0.00464900218843632\\
4034.67220119643	-0.00421057357706133\\
4040.53229153949	-0.00364015049060143\\
4046.39238188255	-0.00295160316964667\\
4052.25247222561	-0.00216154085385285\\
4058.11256256867	-0.00128891141044684\\
4063.97265291173	-0.000354547176565682\\
4069.83274325479	0.000619332112477259\\
4075.69283359785	0.00160964746120418\\
4081.55292394091	0.00259300727804757\\
4087.41301428397	0.00354625962961642\\
4093.27310462703	0.00444703705929061\\
4099.13319497009	0.00527428116240368\\
4104.99328531315	0.00600873462693156\\
4110.85337565621	0.00663338925592373\\
4116.71346599927	0.00713387956008516\\
4122.57355634233	0.00749881282498909\\
4128.43364668539	0.0077200280816226\\
4134.29373702845	0.00779277810868319\\
4140.15382737151	0.00771583042410058\\
4146.01391771456	0.00749148514694538\\
4151.87400805762	0.00712550957154266\\
4157.73409840068	0.00662699126065033\\
4163.59418874374	0.00600811337621788\\
4169.4542790868	0.00528385778461535\\
4175.31436942986	0.00447164315926095\\
4181.17445977292	0.00359090680916101\\
4187.03455011598	0.00266264026045302\\
4192.89464045904	0.00170888967644035\\
4198.7547308021	0.000752232993288131\\
4204.61482114516	-0.000184753838597746\\
4210.47491148822	-0.00108002890107251\\
4216.33500183128	-0.00191260211466732\\
4222.19509217434	-0.00266302665218088\\
4228.0551825174	-0.0033138524653146\\
4233.91527286046	-0.00385003135867849\\
4239.77536320352	-0.00425926421897166\\
4245.63545354658	-0.00453228239845683\\
4251.49554388964	-0.00466305682746218\\
4257.3556342327	-0.00464893015194464\\
4263.21572457576	-0.00449066902062229\\
4269.07581491882	-0.00419243553368495\\
4274.93590526187	-0.00376167877409589\\
4280.79599560493	-0.00320894922013131\\
4286.65608594799	-0.00254764064671001\\
4292.51617629105	-0.00179366581587415\\
4298.37626663411	-0.000965073798040115\\
4304.23635697717	-8.16181154118465e-05\\
4310.09644732023	0.000835713969562203\\
4315.95653766329	0.0017651998147762\\
4321.81662800635	0.00268489558418496\\
4327.67671834941	0.00357315409801373\\
4333.53680869247	0.0044091341367855\\
4339.39689903553	0.00517328922701587\\
4345.25698937859	0.00584782445449976\\
4351.11707972165	0.00641711063969909\\
4356.97717006471	0.00686804624474679\\
4362.83726040776	0.00719035864244292\\
4368.69735075082	0.00737683782939752\\
4374.55744109388	0.00742349727841564\\
4380.41753143694	0.00732965835699008\\
4386.27762178	0.0070979565481617\\
4392.13771212306	0.00673426955943134\\
4397.99780246612	0.00624756924310367\\
4403.85789280918	0.00564970104074225\\
4409.71798315224	0.00495509636141761\\
4415.5780734953	0.00418042486645277\\
4421.43816383836	0.00334419502831241\\
4427.29825418142	0.00246631252594408\\
4433.15834452448	0.00156760700324504\\
4439.01843486754	0.000669338432613838\\
4444.8785252106	-0.000207305227401601\\
4450.73861555366	-0.00104170721982999\\
4456.59870589671	-0.0018143062706842\\
4462.45879623977	-0.00250705459753157\\
4468.31888658283	-0.00310383893029356\\
4474.17897692589	-0.00359085473168586\\
4480.03906726895	-0.00395692493507415\\
4485.89915761201	-0.00419375584983331\\
4491.75924795507	-0.00429612438516661\\
4497.61933829813	-0.00426199237767411\\
4503.47942864119	-0.00409254553806755\\
4509.33951898425	-0.0037921563139937\\
4515.19960932731	-0.00336827176109309\\
4521.05969967037	-0.00283122927761571\\
4526.91979001343	-0.00219400474851233\\
4532.77988035649	-0.00147189922510477\\
4538.63997069955	-0.000682171696352505\\
4544.50006104261	0.000156373243617816\\
4550.36015138567	0.00102383298922104\\
4556.22024172873	0.00189967913315761\\
4562.08033207179	0.00276324309694081\\
4567.94042241485	0.00359420370226787\\
4573.80051275791	0.00437306518987947\\
4579.66060310097	0.00508161444939937\\
4585.52069344402	0.00570334674537006\\
4591.38078378708	0.00622384999610994\\
4597.24087413014	0.00663113866645557\\
4603.1009644732	0.00691592954726432\\
4608.96105481626	0.00707185308507725\\
4614.82114515932	0.00709559546077547\\
4620.68123550238	0.00698696826112283\\
4626.54132584544	0.00674890430226135\\
4632.4014161885	0.00638737990567538\\
4638.26150653156	0.00591126566192369\\
4644.12159687462	0.00533210939437569\\
4649.98168721768	0.00466385662267876\\
4655.84177756074	0.00392251528471928\\
4661.7018679038	0.00312577276856715\\
4667.56195824686	0.00229257440960022\\
4673.42204858992	0.00144267348875718\\
4679.28213893298	0.000596163413792659\\
4685.14222927604	-0.000226996845361649\\
4691.0023196191	-0.00100745286604061\\
4696.86240996215	-0.0017269093620529\\
4702.72250030521	-0.00236855858275609\\
4708.58259064827	-0.0029174724982079\\
4714.44268099133	-0.00336094962661905\\
4720.30277133439	-0.00368880845362635\\
4726.16286167745	-0.00389362067287524\\
4732.02295202051	-0.003970878913277\\
4737.88304236357	-0.00391909517608863\\
4743.74313270663	-0.00373982784746673\\
4749.60322304969	-0.00343763684092504\\
4755.46331339275	-0.00301996811788801\\
4761.32340373581	-0.00249697049601303\\
4767.18349407887	-0.00188124924158634\\
4773.04358442193	-0.00118756242026539\\
4778.90367476498	-0.000432467313435233\\
4784.76376510804	0.000366074636237159\\
4790.6238554511	0.00118912493071111\\
4796.48394579416	0.00201721702516986\\
4802.34403613722	0.00283081660068235\\
4808.20412648028	0.00361078222517441\\
4814.06421682334	0.00433881554615089\\
4819.9243071664	0.004997890433962\\
4825.78439750946	0.00557265101788421\\
4831.64448785252	0.00604976931606611\\
4837.50457819558	0.00641825413822676\\
4843.36466853864	0.00666970410702392\\
4849.2247588817	0.00679849898248883\\
4855.08484922476	0.00680192493989265\\
4860.94493956782	0.0066802310206236\\
4866.80502991088	0.00643661560344464\\
4872.66512025394	0.00607714339672218\\
4878.525210597	0.00561059508553304\\
4884.38530094006	0.00504825335207591\\
4890.24539128312	0.00440363047408418\\
4896.10548162617	0.00369214407177429\\
4901.96557196923	0.00293074877724391\\
4907.82566231229	0.0021375326189662\\
4913.68575265535	0.00133128772401368\\
4919.54584299841	0.000531065519587303\\
4925.40593334147	-0.000244273043748826\\
4931.26602368453	-0.00097650163525037\\
4937.12611402759	-0.00164845660097083\\
4942.98620437065	-0.00224443888751755\\
4948.84629471371	-0.00275058042713254\\
4954.70638505677	-0.00315516643918719\\
4960.56647539983	-0.00344890616422852\\
4966.42656574289	-0.00362514578034095\\
4972.28665608595	-0.00368001862940849\\
4978.14674642901	-0.00361252937091565\\
4984.00683677207	-0.00342457024591402\\
4989.86692711513	-0.00312086923776121\\
4995.72701745819	-0.00270887152233677\\
5001.58710780125	-0.00219855716803276\\
5007.44719814431	-0.00160219954101694\\
5013.30728848737	-0.00093407025742763\\
5019.16737883043	-0.000210097768708227\\
5025.02746917348	0.000552512258170917\\
5030.88755951654	0.00133568572432813\\
5036.7476498596	0.00212090834009687\\
5042.60774020266	0.00288966318301387\\
5048.46783054572	0.00362386723746691\\
5054.32792088878	0.00430629663020967\\
5060.18801123184	0.00492099056791236\\
5066.0481015749	0.00545362450936383\\
5071.90819191796	0.00589184385245099\\
5077.76828226102	0.00622555036950828\\
5083.62837260408	0.00644713475472385\\
5089.48846294714	0.00655164993633903\\
5095.3485532902	0.0065369212118717\\
5101.20864363326	0.00640359076642127\\
5107.06873397632	0.00615509568318416\\
5112.92882431937	0.00579758012743732\\
5118.78891466243	0.00533974393228034\\
5124.64900500549	0.00479263130979069\\
5130.50909534855	0.00416936481180091\\
5136.36918569161	0.00348483094289926\\
5142.22927603467	0.00275532495227472\\
5148.08936637773	0.00199816327395481\\
5153.94945672079	0.00123127282810813\\
5159.80954706385	0.000472766918436309\\
5165.66963740691	-0.00025948224664403\\
5171.52972774997	-0.000948264325738051\\
5177.38981809303	-0.00157743500799598\\
5183.24990843609	-0.00213229408257217\\
5189.10999877915	-0.0025999285654072\\
5194.97008912221	-0.00296951287963104\\
5200.83017946527	-0.00323255911665472\\
5206.69026980832	-0.0033831115981668\\
5212.55036015138	-0.00341788128565944\\
5218.41045049444	-0.00333631701190556\\
5224.2705408375	-0.00314061200521067\\
5230.13063118056	-0.00283564570531639\\
5235.99072152362	-0.00242886239529102\\
5241.85081186668	-0.00193008965858885\\
5247.71090220974	-0.00135130108202774\\
5253.5709925528	-0.000706328928454045\\
5259.43108289586	-1.05336688027332e-05\\
5265.29117323892	0.000719561736366034\\
5271.15126358198	0.0014666641034942\\
5277.01135392504	0.00221311953307835\\
5282.8714442681	0.00294133046499977\\
5288.73153461116	0.00363417043458561\\
5294.59162495422	0.0042753867587451\\
5300.45171529728	0.00484998168923948\\
5306.31180564034	0.00534456309915285\\
5312.1718959834	0.00574765650713192\\
5318.03198632646	0.00604997117492418\\
5323.89207666952	0.00624461411186288\\
5329.75216701258	0.00632724706264815\\
5335.61225735563	0.00629618290975367\\
5341.47234769869	0.0061524193568112\\
5347.33243804175	0.00589960924439551\\
5353.19252838481	0.00554396834214876\\
5359.05261872787	0.0050941229348565\\
5364.91270907093	0.00456090093196473\\
5370.77279941399	0.00395707155384661\\
5376.63288975705	0.0032970398478728\\
5382.49298010011	0.00259650333742868\\
5388.35307044317	0.00187207898338601\\
5394.21316078623	0.00114090931835258\\
5400.07325112929	0.000420257084672758\\
5405.93334147235	-0.000272902043125599\\
5411.79343181541	-0.000922279046914138\\
5417.65352215847	-0.0015126540634109\\
5423.51361250153	-0.00203023281513701\\
5429.37370284459	-0.00246296863114866\\
5435.23379318765	-0.00280084254465944\\
5441.09388353071	-0.00303609495944454\\
5446.95397387376	-0.00316340353342987\\
5452.81406421682	-0.00318000320655496\\
5458.67415455988	-0.00308574567399403\\
5464.53424490294	-0.00288309703758909\\
5470.394335246	-0.00257707382942487\\
5476.25442558906	-0.00217511905284908\\
5482.11451593212	-0.00168692129635067\\
5487.97460627518	-0.00112418131098558\\
5493.83469661824	-0.000500331670064663\\
5499.6947869613	0.000169783776807478\\
5505.55487730436	0.000870263010159422\\
5511.41496764742	0.00158452311235249\\
5517.27505799048	0.00229569275805176\\
5523.13514833354	0.00298701063816723\\
5528.9952386766	0.00364222041250649\\
5534.85532901965	0.00424595288912246\\
5540.71541936271	0.00478408644935242\\
5546.57550970577	0.00524407727078112\\
5552.43560004883	0.00561525162976472\\
5558.29569039189	0.00588905347689357\\
5564.15578073495	0.00605924154656835\\
5570.01587107801	0.00612203146568702\\
5575.87596142107	0.00607617962991959\\
5581.73605176413	0.00592300699925457\\
5587.59614210719	0.00566636238044875\\
5593.45623245025	0.00531252619536609\\
5599.31632279331	0.00487005713112554\\
5605.17641313637	0.00434958541164809\\
5611.03650347943	0.00376355767896498\\
5616.89659382249	0.00312593960227707\\
5622.75668416555	0.00245188331715586\\
5628.61677450861	0.00175736760918476\\
5634.47686485167	0.00105881938305717\\
5640.33695519473	0.000372725380382555\\
5646.19704553779	-0.000284756680071274\\
5652.05713588084	-0.000898178371887775\\
5657.9172262239	-0.00145316329447726\\
5663.77731656696	-0.00193674375738302\\
5669.63740691002	-0.00233766353442861\\
5675.49749725308	-0.002646639623084\\
5681.35758759614	-0.00285657692616216\\
5687.2176779392	-0.00296273089486637\\
5693.07776828226	-0.00296281441032375\\
5698.93785862532	-0.00285704650173869\\
5704.79794896838	-0.00264814187668898\\
5710.65803931144	-0.00234124163505816\\
5716.5181296545	-0.00194378692458454\\
5722.37821999756	-0.00146533863688936\\
5728.23831034062	-0.000917347507992998\\
5734.09840068368	-0.000312880147372273\\
5739.95849102674	0.000333692453135872\\
5745.8185813698	0.00100703650901828\\
5751.67867171286	0.00169121901555685\\
5757.53876205592	0.00237008468625006\\
5763.39885239898	0.00302763734357482\\
5769.25894274204	0.00364841676164796\\
5775.11903308509	0.00421786208702458\\
5780.97912342815	0.00472265329803579\\
5786.83921377121	0.00515102269908503\\
5792.69930411427	0.00549302916919449\\
5798.55939445733	0.00574078877660408\\
5804.41948480039	0.00588865641368789\\
5810.27957514345	0.00593335427217524\\
5816.13966548651	0.00587404423898748\\
5821.99975582957	0.00571234262291438\\
5827.85984617263	0.00545227698349191\\
5833.71993651569	0.00510018619935562\\
5839.58002685875	0.00466456624824702\\
5845.44011720181	0.00415586544681112\\
5851.30020754487	0.00358623408019082\\
5857.16029788793	0.00296923441796199\\
5863.02038823099	0.00231951803323882\\
5868.88047857404	0.00165247809928863\\
5874.7405689171	0.000983884911402183\\
5880.60065926016	0.000329513260952879\\
5886.46074960322	-0.000295229537481451\\
5892.32083994628	-0.000875666198067985\\
5898.18093028934	-0.0013981940668296\\
5904.0410206324	-0.00185060368842271\\
5909.90111097546	-0.00222236387948641\\
5915.76120131852	-0.00250486665543998\\
5921.62129166158	-0.00269162631846643\\
5927.48138200464	-0.00277842810518147\\
5933.3414723477	-0.00276342299202658\\
5939.20156269076	-0.00264716653143206\\
5945.06165303382	-0.00243260091635069\\
5950.92174337688	-0.00212498080952864\\
5956.78183371994	-0.00173174479983086\\
5962.64192406299	-0.00126233562435079\\
5968.50201440605	-0.000727973497765935\\
5974.36210474911	-0.000141387986402107\\
5980.22219509217	0.000483485169084496\\
5986.08228543523	0.00113183506484267\\
5991.94237577829	0.00178832661968941\\
5997.80246612135	0.00243746322122374\\
6003.66255646441	0.00306395245963145\\
6009.52264680747	0.00365306632065881\\
6015.38273715053	0.00419098735696594\\
6021.24282749359	0.00466513270465152\\
6027.10291783665	0.00506444834994351\\
6032.96300817971	0.0053796667672068\\
6038.82309852277	0.00560352192679455\\
6044.68318886583	0.00573091668791684\\
6050.54327920889	0.00575903872375827\\
6056.40336955195	0.00568742234715295\\
6062.26345989501	0.00551795488594006\\
6068.12355023807	0.00525482756537569\\
6073.98364058113	0.00490443216626425\\
6079.84373092419	0.00447520599931261\\
6085.70382126725	0.00397742895308915\\
6091.5639116103	0.00342297749337938\\
6097.42400195336	0.00282504149783154\\
6103.28409229642	0.0021978106739798\\
6109.14418263948	0.00155613801297503\\
6115.00427298254	0.000915188258319774\\
6120.8643633256	0.00029007970600882\\
6126.72445366866	-0.000304472205703852\\
6132.58454401172	-0.000854501117769777\\
6138.44463435478	-0.00134711764011586\\
6144.30472469784	-0.00177081121470925\\
6150.1648150409	-0.00211571888554609\\
6156.02490538396	-0.00237385470245888\\
6161.88499572702	-0.00253929442639755\\
6167.74508607008	-0.0026083112671582\\
6173.60517641314	-0.00257945954782799\\
6179.4652667562	-0.00245360442278085\\
6185.32535709926	-0.00223389705270557\\
6191.18544744232	-0.00192569592569518\\
6197.04553778538	-0.00153643628271601\\
6202.90562812843	-0.00107545082509682\\
6208.76571847149	-0.000553746023853492\\
6214.62580881455	1.62606105484162e-05\\
6220.48589915761	0.000621036029345132\\
6226.34598950067	0.00124625288840901\\
6232.20607984373	0.0018771289102097\\
6238.06617018679	0.00249877632929415\\
6243.92626052985	0.00309655316352792\\
6249.78635087291	0.00365640802571516\\
6255.64644121597	0.00416521035809779\\
6261.50653155903	0.00461105833162487\\
6267.36662190209	0.00498355719262695\\
6273.22671224515	0.00527406154989258\\
6279.08680258821	0.00547587595770156\\
6284.94689293126	0.005584409144124\\
6290.80698327432	0.00559727833517958\\
6296.66707361738	0.00551436130853336\\
6302.52716396044	0.00533779504811387\\
6308.3872543035	0.00507192113012198\\
6314.24734464656	0.00472317922852225\\
6320.10743498962	0.00429995134713795\\
6325.96752533268	0.00381236054331838\\
6331.82761567574	0.00327202897234065\\
6337.6877060188	0.00269180103443843\\
6343.54779636186	0.00208543821521955\\
6349.40788670492	0.00146729286793934\\
6355.26797704798	0.000851968667284953\\
6361.12806739104	0.000253975765548383\\
6366.9881577341	-0.000312611208892719\\
6372.84824807716	-0.000834484252292827\\
6378.70833842022	-0.00129941439902993\\
6384.56842876328	-0.00169653811643794\\
6390.42851910634	-0.00201661097626424\\
6396.2886094494	-0.0022522226819947\\
6402.14869979245	-0.00239796845315791\\
6408.00879013551	-0.00245057280581024\\
6413.86888047857	-0.00240896289796897\\
6419.72897082163	-0.0022742898028516\\
6425.58906116469	-0.00204989730375426\\
6431.44915150775	-0.00174123904191137\\
6437.30924185081	-0.00135574606517833\\
6443.16933219387	-0.000902647990680954\\
6449.02942253693	-0.000392752082520558\\
6454.88951287999	0.000161814470056469\\
6460.74960322305	0.000747892930735089\\
6466.60969356611	0.00135160519540629\\
6472.46978390917	0.00195868236896549\\
6478.32987425223	0.00255480196276792\\
6484.18996459529	0.00312592574678473\\
6490.05005493835	0.00365863028905291\\
6495.91014528141	0.00414042240254238\\
6501.77023562447	0.00456003208852832\\
6507.63032596753	0.00490767611017306\\
6513.49041631059	0.00517528603443037\\
6519.35050665365	0.005356695429094\\
6525.21059699671	0.005447781874524\\
6531.07068733976	0.00544656052279753\\
6536.93077768282	0.00535322708461923\\
6542.79086802588	0.00517014932109206\\
6548.65095836894	0.00490180733291645\\
6554.511048712	0.00455468414590147\\
6560.37113905506	0.00413710926157759\\
6566.23122939812	0.00365905894498741\\
6572.09131974118	0.00313191803605909\\
6577.95141008424	0.00256820896834099\\
6583.8115004273	0.00198129444312221\\
6589.67159077036	0.00138506081597554\\
6595.53168111342	0.000793589695060228\\
6601.39177145648	0.000220825515518639\\
6607.25186179954	-0.000319753064928165\\
6613.1119521426	-0.000815450207793016\\
6618.97204248565	-0.00125465091792605\\
6624.83213282871	-0.0016270930563418\\
6630.69222317177	-0.00192410697380834\\
6636.55231351483	-0.00213881717160385\\
6642.41240385789	-0.0022663013013571\\
6648.27249420095	-0.00230370282935873\\
6654.13258454401	-0.00225029479024497\\
6659.99267488707	-0.0021074932120618\\
6665.85276523013	-0.00187881998418719\\
6671.71285557319	-0.00156981613176464\\
6677.57294591625	-0.00118790762803002\\
6683.43303625931	-0.000742226991104452\\
6689.29312660237	-0.000243394948793349\\
6695.15321694543	0.000296732611241846\\
6701.01330728849	0.000865346378826853\\
6706.87339763155	0.0014489873534096\\
6712.7334879746	0.00203386537113897\\
6718.59357831766	0.00260618489054492\\
6724.45366866072	0.00315247034271061\\
6730.31375900378	0.00365988337488168\\
6736.17384934684	0.00411652452141069\\
6742.0339396899	0.00451171221529959\\
6747.89403003296	0.00483623260068631\\
6753.75412037602	0.00508255430570568\\
6759.61421071908	0.00524500317214631\\
6765.47430106214	0.00531989288959067\\
6771.3343914052	0.00530560853042484\\
6777.19448174826	0.00520264109569365\\
6783.05457209132	0.00501357234097803\\
6788.91466243438	0.00474301032587694\\
6794.77475277744	0.00439747728922471\\
6800.6348431205	0.00398525257689921\\
6806.49493346356	0.00351617440088875\\
6812.35502380662	0.00300140517511289\\
6818.21511414968	0.00245316602226645\\
6824.07520449274	0.00188444676516167\\
6829.9352948358	0.00130869828155682\\
6835.79538517886	0.000739514507128819\\
6841.65547552191	0.000190311602124636\\
6847.51556586497	-0.00032598814825652\\
6853.37565620803	-0.000797260254116888\\
6859.23574655109	-0.0012124626159133\\
6865.09583689415	-0.00156189411605015\\
6870.95592723721	-0.00183742115323662\\
6876.81601758027	-0.00203266683163263\\
6882.67610792333	-0.00214315840660015\\
6888.53619826639	-0.00216642957996259\\
6894.39628860945	-0.00210207530847515\\
6900.25637895251	-0.00195175791239929\\
6906.11646929557	-0.00171916442158967\\
6911.97655963863	-0.00140991624614854\\
6917.83664998169	-0.00103143338130802\\
6923.69674032475	-0.000592756424064951\\
6929.55683066781	-0.000104330668892562\\
6935.41692101087	0.00042224256351902\\
6941.27701135393	0.0009744814463379\\
6947.13710169699	0.0015393194740748\\
6952.99719204004	0.00210341458817649\\
6958.8572823831	0.00265346428939738\\
6964.71737272616	0.00317651929661852\\
6970.57746306922	0.00366028835835199\\
6976.43755341228	0.00409342704370376\\
6982.29764375534	0.00446580372984344\\
6988.1577340984	0.00476873655094517\\
6994.01782444146	0.00499519576923716\\
6999.87791478452	0.0051399668529639\\
7005.73800512758	0.00519977047984608\\
7011.59809547064	0.00517333670794662\\
7017.4581858137	0.00506143163847697\\
7023.31827615676	0.00486683602111729\\
7029.17836649982	0.00459427638360516\\
7035.03845684288	0.0042503103878975\\
7040.89854718593	0.00384316919069466\\
7046.75863752899	0.00338256059547219\\
7052.61872787205	0.00287943770187893\\
7058.47881821511	0.00234573856389811\\
7064.33890855817	0.00179410304284405\\
7070.19899890123	0.00123757356929216\\
7076.05908924429	0.000689286895977135\\
7081.91917958735	0.000162164125179634\\
7087.77926993041	-0.000331393677184145\\
7093.63936027347	-0.000779797112215492\\
7099.49945061653	-0.0011725404670072\\
7105.35954095959	-0.00150044773905863\\
7111.21963130265	-0.00175588690663076\\
7117.07972164571	-0.00193294744323167\\
7122.93981198877	-0.00202757694959323\\
7128.79990233183	-0.00203767374653313\\
7134.65999267489	-0.00196313331618809\\
7140.52008301795	-0.00180584757024491\\
7146.38017336101	-0.00156965703806855\\
7152.24026370407	-0.00126025717732863\\
7158.10035404712	-0.000885061089799094\\
7163.96044439018	-0.000453021949212453\\
7169.82053473324	2.55806072418099e-05\\
7175.6806250763	0.000539385029281543\\
7181.54071541936	0.00107621752968296\\
7187.40080576242	0.00162338048697094\\
7193.26089610548	0.00216795274406222\\
7199.12098644854	0.00269709470247947\\
7204.9810767916	0.00319835101011038\\
7210.84116713466	0.00365994370832216\\
7216.70125747772	0.00407104894102709\\
7222.56134782078	0.0044220507260332\\
7228.42143816384	0.00470476584106593\\
7234.2815285069	0.0049126345663638\\
7240.14161884996	0.00504087284005571\\
7246.00170919302	0.00508658229843938\\
7251.86179953608	0.00504881567328607\\
7257.72188987914	0.0049285960728339\\
7263.5819802222	0.00472888976538742\\
7269.44207056526	0.00445453317931075\\
7275.30216090832	0.00411211591289026\\
7281.16225125137	0.00370982258164542\\
7287.02234159443	0.0032572372966923\\
7292.88243193749	0.00276511544347561\\
7298.74252228055	0.00224512819346433\\
7304.60261262361	0.00170958581632914\\
7310.46270296667	0.00117114635007881\\
7316.32279330973	0.000642516522328329\\
7322.18288365279	0.000136151987313901\\
7328.04297399585	-0.000336036051262834\\
7333.90306433891	-0.000762960923031298\\
7339.76315468197	-0.00113462069513754\\
7345.62324502503	-0.00144233239245914\\
7351.48333536809	-0.00167893475050007\\
7357.34342571115	-0.0018389547684456\\
7363.20351605421	-0.00191873419030468\\
7369.06360639727	-0.00191651299321907\\
7374.92369674032	-0.00183246798045045\\
7380.78378708338	-0.00166870563765352\\
7386.64387742644	-0.00142920949135318\\
7392.5039677695	-0.0011197432806885\\
7398.36405811256	-0.000747712293557661\\
7404.22414845562	-0.000321986201267342\\
7410.08423879868	0.000147312371606068\\
7415.94432914174	0.000649048503737084\\
7421.8044194848	0.00117133914015726\\
7427.66450982786	0.00170183449662479\\
7433.52460017092	0.00222801004266241\\
7439.38469051398	0.00273746216191537\\
7445.24478085704	0.00321820051139937\\
7451.1048712001	0.0036589301905883\\
7456.96496154316	0.00404931708139347\\
7462.82505188622	0.00438023012724407\\
7468.68514222927	0.00464395487261931\\
7474.54523257233	0.00483437327082675\\
7480.40532291539	0.00494710557064382\\
7486.26541325845	0.00497961099312772\\
7492.12550360151	0.00493124488621574\\
7497.98559394457	0.00480327107548022\\
7503.84568428763	0.00459882918638366\\
7509.70577463069	0.00432285777721249\\
7515.56586497375	0.00398197516092144\\
7521.42595531681	0.00358432079081758\\
7527.28604565987	0.00313936100899797\\
7533.14613600293	0.00265766379220044\\
7539.00622634599	0.00215064785350499\\
7544.86631668905	0.00163031205512012\\
7550.72640703211	0.00110895154258507\\
7556.58649737517	0.000598867314853788\\
7562.44658771823	0.000112076088802196\\
7568.30667806129	-0.000339972699795591\\
7574.16676840435	-0.000746666095729551\\
7580.02685874741	-0.00109847669622303\\
7585.88694909047	-0.00138718575129767\\
7591.74703943353	-0.00160607506982482\\
7597.60712977658	-0.00175008325178003\\
7603.46722011964	-0.0018159226151982\\
7609.3273104627	-0.00180215411821402\\
7615.18740080576	-0.00170921857177884\\
7621.04749114882	-0.00153942347109906\\
7626.90758149188	-0.00129688582194969\\
7632.76767183494	-0.000987432375046988\\
7638.627762178	-0.000618459683814279\\
7644.48785252106	-0.000198757344909064\\
7650.34794286412	0.000261701355847398\\
7656.20803320718	0.000751996407223845\\
7662.06812355024	0.00126052001710392\\
7667.9282138933	0.0017752513990113\\
7673.78830423636	0.00228404089697095\\
7679.64839457942	0.00277489673474384\\
7685.50848492248	0.00323626762295892\\
7691.36857526554	0.00365731456558242\\
7697.2286656086	0.00402816547112825\\
7703.08875595166	0.00434014658960261\\
7708.94884629471	0.00458598535043495\\
7714.80893663777	0.00475997985931153\\
7720.66902698083	0.00485813110556342\\
7726.52911732389	0.00487823481579208\\
7732.38920766695	0.00481993084626727\\
7738.24929801001	0.00468470901040696\\
7744.10938835307	0.00447587126749133\\
7749.96947869613	0.00419845122673438\\
7755.82956903919	0.0038590929269255\\
7761.68965938225	0.00346589180955359\\
7767.54974972531	0.00302820168944862\\
7773.40984006837	0.00255641232483692\\
7779.26993041143	0.0020617028745361\\
7785.13002075449	0.00155577709121441\\
7790.99011109754	0.00105058652195334\\
7796.8502014406	0.000558048261289596\\
7802.71029178366	8.97639209317597e-05\\
7808.57038212672	-0.000343253558008612\\
7814.43047246978	-0.000730838818666499\\
7820.29056281284	-0.00106391264226396\\
7826.1506531559	-0.00133469454520824\\
7832.01074349896	-0.00153688444269864\\
7837.87083384202	-0.00166580913852832\\
7843.73092418508	-0.00171853023641277\\
7849.59101452814	-0.00169391098271643\\
7855.4511048712	-0.0015926405225492\\
7861.31119521426	-0.00141721505745585\\
7867.17128555732	-0.0011718764103387\\
7873.03137590038	-0.000862509506677961\\
7878.89146624344	-0.000496501247896817\\
7884.7515565865	-8.25641599395829e-05\\
7890.61164692956	0.000369470974443799\\
7896.47173727262	0.000848888565200893\\
7902.33182761568	0.00134434220401038\\
7908.19191795873	0.001844123173884\\
7914.05200830179	0.00233643712631276\\
7919.91209864485	0.00280968238947026\\
7925.77218898791	0.00325272334162619\\
7931.63227933097	0.00365515240923434\\
7937.49236967403	0.00400753452702635\\
7943.35246001709	0.00430162831903393\\
7949.21255036015	0.00453057881694765\\
7955.07264070321	0.00468907721033001\\
7960.93273104627	0.00477348390764992\\
7966.79282138933	0.00478191205601213\\
7972.65291173239	0.00471426960627653\\
7978.51300207545	0.00457225898802852\\
7984.37309241851	0.00435933446178328\\
7990.23318276157	0.00408061821249624\\
7996.09327310463	0.00374277722110838\\
8001.95336344769	0.0033538638718244\\
8007.81345379075	0.0029231241047975\\
8013.67354413381	0.00246077768362429\\
8019.53363447687	0.00197777579864329\\
8025.39372481993	0.00148554175362422\\
8031.25381516299	0.000995700875414336\\
8037.11390550604	0.000519806030933051\\
8042.9739958491	6.90652311472127e-05\\
8048.83408619216	-0.000345922256493886\\
8054.69417653522	-0.000715415075374337\\
8060.55426687828	-0.001030758370533\\
8066.41435722134	-0.00128458643998159\\
8072.2744475644	-0.00147099470286946\\
8078.13453790746	-0.00158567697055843\\
8083.99462825052	-0.00162602483077163\\
8089.85471859358	-0.00159118685275138\\
8095.71480893664	-0.00148208627178479\\
8101.5748992797	-0.00130139679299883\\
8107.43498962276	-0.00105347714186791\\
8113.29507996582	-0.000744265961507545\\
8119.15517030888	-0.000381139589259708\\
8125.01526065193	2.72638824057015e-05\\
8130.87535099499	0.000471250065677629\\
8136.73544133805	0.000940298371978892\\
8142.59553168111	0.00142331128130615\\
8148.45562202417	0.00190887688143375\\
8154.31571236723	0.00238553846709102\\
8160.17580271029	0.00284206483124809\\
8166.03589305335	0.00326771487287228\\
8171.89598339641	0.00365249028792505\\
8177.75607373947	0.00398737040045396\\
8183.61616408253	0.0042645236188018\\
8189.47625442559	0.00447749056206788\\
8195.33634476865	0.00462133457518301\\
8201.19643511171	0.00469275612697605\\
8207.05652545477	0.00469016844152466\\
8212.91661579783	0.00461373263109226\\
8218.77670614088	0.00446535155618922\\
8224.63679648394	0.00424862261316535\\
8230.496886827	0.00396875061699897\\
8236.35697717006	0.00363242288834444\\
8242.21706751312	0.00324764954019057\\
8248.07715785618	0.00282357277835864\\
8253.93724819924	0.002370249753967\\
8259.7973385423	0.00189841412506393\\
8265.65742888536	0.00141922197891081\\
8271.51751922842	0.000943988128342759\\
8277.37760957148	0.000483919014221943\\
8283.23769991454	4.98485181683334e-05\\
8289.0977902576	-0.000348017087531137\\
8294.95788060066	-0.000700339049596867\\
8300.81797094372	-0.000998865264702988\\
8306.67806128678	-0.00123662349096209\\
8312.53815162984	-0.00140808411730959\\
8318.3982419729	-0.00150928869192557\\
8324.25833231596	-0.00153794122356749\\
8330.11842265902	-0.0014934601513483\\
8335.97851300208	-0.00137698980918325\\
8341.83860334514	-0.00119137116750191\\
8347.69869368819	-0.000941072596011324\\
8353.55878403125	-0.000632082332671406\\
8359.41887437431	-0.000271765245471788\\
8365.27896471737	0.000131312688125549\\
8371.13905506043	0.000567588002048909\\
8376.99914540349	0.00102672662364745\\
8382.85923574655	0.00149786863542219\\
8388.71932608961	0.00196988512036785\\
8394.57941643267	0.00243164101740384\\
8400.43950677573	0.00287225777954352\\
8406.29959711879	0.00328136964013129\\
8412.15968746185	0.00364936745141265\\
8418.01977780491	0.00396762436058506\\
8423.87986814797	0.0042286980239978\\
8429.73995849103	0.00442650462099318\\
8435.60004883409	0.00455646059946184\\
8441.46013917715	0.00461558885176763\\
8447.32022952021	0.00460258686299975\\
8453.18031986327	0.00451785527312759\\
8459.04041020633	0.00436348623070412\\
8464.90050054939	0.00414321186451714\\
8470.76059089244	0.00386231414000165\\
8476.6206812355	0.00352749827598055\\
8482.48077157856	0.00314673275453141\\
8488.34086192162	0.00272905973982109\\
8494.20095226468	0.00228438041581017\\
8500.06104260774	0.00182322033744776\\
8505.9211329508	0.00135648035571344\\
8511.78122329386	0.000895179009245003\\
8517.64131363692	0.000450192469494161\\
8523.50140397998	3.19981760311385e-05\\
8529.36149432304	-0.000349571796148656\\
8535.2215846661	-0.000685561832787853\\
8541.08167500916	-0.000968102909160292\\
8546.94176535222	-0.00119059682223355\\
8552.80185569527	-0.00134787021387826\\
8558.66194603833	-0.00143629479067399\\
8564.52203638139	-0.00145387094796795\\
8570.38212672445	-0.00140027287203389\\
8576.24221706751	-0.00127685410565536\\
8582.10230741057	-0.0010866134956365\\
8587.96239775363	-0.000834122375845357\\
8593.82248809669	-0.000525414751988675\\
8599.68257843975	-0.000167843125083536\\
8605.54266878281	0.000230096602570953\\
8611.40275912587	0.000658966970117599\\
8617.26284946893	0.00110861275528065\\
8623.12293981199	0.00156840141982746\\
8628.98303015505	0.00202747451125621\\
8634.84312049811	0.00247500407744168\\
8640.70321084117	0.0029004480401003\\
8646.56330118423	0.00329379850529249\\
8652.42339152729	0.00364581716300942\\
8658.28348187034	0.00394825223982305\\
8664.1435722134	0.00419403191096226\\
8670.00366255646	0.00437742963913729\\
8675.86375289952	0.00449419757514533\\
8681.72384324258	0.0045416649148009\\
8687.58393358564	0.00451879893457501\\
8693.4440239287	0.00442622731398859\\
8699.30411427176	0.00426622126576131\\
8705.16420461482	0.00404263992075257\\
8711.02429495788	0.00376083732708757\\
8716.88438530094	0.00342753430388004\\
8722.744475644	0.00305065821529802\\
8728.60456598706	0.00263915448377542\\
8734.46465633012	0.00220277432339206\\
8740.32474667318	0.00175184372891886\\
8746.18483701624	0.00129701919275337\\
8752.0449273593	0.000849035927562462\\
8757.90501770236	0.000418454542535058\\
8763.76510804542	1.54121500532384e-05\\
8769.62519838848	-0.000350616232252363\\
8775.48528873154	-0.000671040369009993\\
8781.3453790746	-0.000938356351398894\\
8787.20546941765	-0.00114632227046114\\
8793.06555976071	-0.00129010390772904\\
8798.92565010377	-0.0013663870435062\\
8804.78574044683	-0.00137345377493222\\
8810.64583078989	-0.00131122108748717\\
8816.50592113295	-0.00118124081662268\\
8822.36601147601	-0.0009866610474207\\
8828.22610181907	-0.000732149909718607\\
8834.08619216213	-0.00042378361213721\\
8839.94628250519	-6.89013992747874e-05\\
8845.80637284825	0.000324069106560004\\
8851.66646319131	0.000745812023475895\\
8857.52655353437	0.00118634403701513\\
8863.38664387743	0.00163525070234832\\
8869.24673422049	0.00208193262934456\\
8875.10682456355	0.00251585573192684\\
8880.96691490661	0.00292679963415491\\
8886.82700524966	0.0033050983764149\\
8892.68709559272	0.00364186775339929\\
8898.54718593578	0.00392921393786518\\
8904.40727627884	0.00416041849243725\\
8910.2673666219	0.00433009543330857\\
8916.12745696496	0.00443431667582854\\
8921.98754730802	0.00447070294047248\\
8927.84763765108	0.00443847801676287\\
8933.70772799414	0.00433848514944135\\
8939.5678183372	0.0041731652057437\\
8945.42790868026	0.00394649718408987\\
8951.28799902332	0.00366390251348384\\
8957.14808936638	0.00333211544331347\\
8963.00817970944	0.00295902262171462\\
8968.8682700525	0.00255347568281955\\
8974.72836039555	0.00212508129596976\\
8980.58845073861	0.00168397365532782\\
8986.44854108167	0.00124057479759203\\
8992.30863142473	0.000805348415206443\\
8998.16872176779	0.000388552979863155\\
nan	nan\\
9009.88890245391	-0.000351176891762859\\
9015.74899279697	-0.000656736587672686\\
9021.60908314003	-0.000909523846688235\\
9027.46917348309	-0.00110363679407522\\
9033.32926382615	-0.00123456465848646\\
9039.18935416921	-0.0012992925329808\\
9045.04944451227	-0.00129637072791189\\
9050.90953485533	-0.00122594712163562\\
9056.76962519839	-0.00108976178970174\\
9062.62971554145	-0.000891104083604343\\
9068.48980588451	-0.000634733215633085\\
9074.34989622757	-0.000326764266075394\\
9080.20998657063	2.54776579995897e-05\\
9086.07007691369	0.000413631634918769\\
9091.93016725675	0.000828499369137581\\
9097.79025759981	0.00126026315146303\\
9103.65034794286	0.00169871817493735\\
9109.51043828592	0.00213351370972789\\
9115.37052862898	0.00255439743838982\\
9121.23061897204	0.00295145718423983\\
9127.0907093151	0.00331535433450035\\
9132.95079965816	0.00363754346196718\\
9138.81089000122	0.00391047298120032\\
9144.67098034428	0.00412776212804796\\
9150.53107068734	0.0042843501160307\\
9156.3911610304	0.00437661398349029\\
9162.25125137346	0.00440245238819283\\
9168.11134171652	0.00436133341242939\\
9173.97143205958	0.00425430529276466\\
9179.83152240264	0.00408396986436188\\
9185.6916127457	0.00385441938930239\\
9191.55170308876	0.00357113830113054\\
9197.41179343182	0.00324087222426609\\
9203.27188377488	0.00287146739490105\\
9209.13197411794	0.00247168430556852\\
9214.992064461	0.00205098999858324\\
9220.85215480405	0.0016193339321104\\
9226.71224514711	0.00118691272472488\\
9232.57233549017	0.000763929339960075\\
9238.43242583323	0.000360352397762063\\
9244.29251617629	-1.43187103329364e-05\\
9250.15260651935	-0.000351277368403233\\
9256.01269686241	-0.000642616685997277\\
9261.87278720547	-0.000881514988088023\\
9267.73287754853	-0.00106239549452392\\
9273.59296789159	-0.00118105645194892\\
9279.45305823465	-0.00123476868250341\\
9285.31314857771	-0.00122233828547741\\
9291.17323892077	-0.001144133052739\\
9297.03332926383	-0.00100207201605785\\
9302.89341960689	-0.000799578414947775\\
9308.75350994994	-0.000541497235881272\\
9314.613600293	-0.000233979308234936\\
9320.47369063606	0.000115665271268403\\
9326.33378097912	0.000499140893397119\\
9332.19387132218	0.000907363241330561\\
9338.05396166524	0.00133067447743953\\
9343.9140520083	0.00175907171526883\\
9349.77414235136	0.00218244337278847\\
9355.63423269442	0.00259080782287016\\
9361.49432303748	0.00297454870915472\\
9367.35441338054	0.00332464138060481\\
9373.2145037236	0.00363286511371622\\
9379.07459406666	0.00389199613237378\\
9384.93468440972	0.00409597689430078\\
9390.79477475278	0.00424005767722932\\
9396.65486509584	0.0043209071568294\\
9402.5149554389	0.00433668940223972\\
9408.37504578196	0.00428710551121622\\
9414.23513612501	0.00417339894181847\\
9420.09522646807	0.00399832445619156\\
9425.95531681113	0.00376608144899409\\
9431.81540715419	0.00348221327314416\\
9437.67549749725	0.00315347497583804\\
9443.53558784031	0.00278767259901116\\
9449.39567818337	0.00239347786758938\\
9455.25576852643	0.00198022266306069\\
9461.11585886949	0.00155767815331197\\
9466.97594921255	0.00113582380540982\\
9472.83603955561	0.000724611741319333\\
9478.69612989867	0.000333731999683831\\
9484.55622024173	-2.76157599881354e-05\\
9490.41631058479	-0.000350938732998813\\
9496.27640092785	-0.000628650531872853\\
9502.13649127091	-0.000854249146834765\\
9507.99658161397	-0.00102246913060506\\
9513.85667195703	-0.0011294044460698\\
9519.71676230009	-0.00117259911164546\\
9525.57685264315	-0.00115110354139142\\
9531.43694298621	-0.00106549528902926\\
9537.29703332927	-0.000917863745331579\\
9543.15712367232	-0.000711759189798616\\
9549.01721401538	-0.000452107437206596\\
9554.87730435844	-0.000145092130400868\\
9560.7373947015	0.000201992508947934\\
9566.59748504456	0.000580914965836019\\
9572.45757538762	0.000982701637709513\\
9578.31766573068	0.0013978493320181\\
9584.17775607374	0.00181655002431203\\
9590.0378464168	0.00222892256176408\\
9595.89793675986	0.00262524583909361\\
9601.75802710292	0.00299618794527391\\
9607.61811744598	0.00333302587986098\\
9613.47820778904	0.0036278506670992\\
9619.3382981321	0.00387375304460541\\
9625.19838847516	0.00406498536853612\\
9631.05847881822	0.0041970959411474\\
9636.91856916128	0.00426703262295962\\
9642.77865950433	0.00427321331770823\\
9648.63874984739	0.00421556170466336\\
9654.49884019045	0.00409550741163851\\
9660.35893053351	0.00391595066399643\\
9666.21902087657	0.00368119228174126\\
9672.07911121963	0.00339683071277788\\
9677.93920156269	0.0030696285673527\\
9683.79929190575	0.00270735183443493\\
9689.65938224881	0.00231858560203689\\
9695.51947259187	0.00191253065364958\\
9701.37956293493	0.00149878575938505\\
9707.23965327799	0.00108712081242176\\
9713.09974362105	0.000687246171984016\\
9718.95983396411	0.000308583657897991\\
9724.81992430716	-3.99554034799944e-05\\
9730.68001465022	-0.000350179853627983\\
9736.54010499328	-0.000614811163982807\\
9742.40019533634	-0.000827654164184469\\
9748.2602856794	-0.000983742032508692\\
9754.12037602246	-0.00107945215577781\\
9759.98046636552	-0.001112590156819\\
9765.84055670858	-0.00108244014129332\\
9771.70064705164	-0.000989780014612611\\
9777.5607373947	-0.000836861544688236\\
9783.42082773776	-0.00062735567827998\\
9789.28091808082	-0.000366264437741306\\
9795.14100842388	-5.9801511773406e-05\\
9801.00109876694	0.000284755609934703\\
9806.86118911	0.000659238440118995\\
9812.72127945306	0.00105478113286438\\
9818.58136979612	0.00146203036832846\\
9824.44146013918	0.00187136651527814\\
9830.30155048224	0.00227313084248649\\
9836.1616408253	0.00265785341320656\\
9842.02173116836	0.00301647628608772\\
9847.88182151142	0.00334056676256645\\
9853.74191185447	0.00362251566100787\\
9859.60200219753	0.00385571595604088\\
9865.46209254059	0.00403471759031168\\
9871.32218288365	0.00415535483135437\\
9877.18227322671	0.00421484319840697\\
9883.04236356977	0.0042118437045222\\
9888.90245391283	0.00414649293305666\\
9894.76254425589	0.00402039827435218\\
9900.62263459895	0.00383659847186479\\
9906.48272494201	0.00359949044392328\\
9912.34281528507	0.00331472414342831\\
9918.20290562813	0.00298906796786863\\
9924.06299597119	0.00263024792603116\\
9929.92308631425	0.00224676438194026\\
9935.78317665731	0.00184769072254565\\
9941.64326700037	0.0014424587168023\\
9947.50335734343	0.00104063564336161\\
9953.36344768649	0.000651698452519214\\
9959.22353802955	0.000284810292908342\\
9965.08362837261	-5.13953281799235e-05\\
9970.94371871567	-0.000349017667190359\\
9976.80380905872	-0.000601074371164269\\
9982.66389940178	-0.000801665248406276\\
9988.52398974484	-0.000946110342293003\\
9994.3840800879	-0.00103105907773339\\
10000.244170431	-0.00105456793537535\\
10006.104260774	-0.00101614485376583\\
10011.9643511171	-0.000916759346183609\\
10017.8244414601	-0.000758818128777706\\
10023.6845318032	-0.00054610687034786\\
10029.5446221463	-0.00028369947249052\\
10035.4047124893	2.21629464914135e-05\\
10041.2648028324	0.000364220478364466\\
10047.1248931754	0.000734366733043322\\
10052.9849835185	0.00112384093899665\\
10058.8450738616	0.00152343528432678\\
10064.7051642046	0.00192371259194056\\
10070.5652545477	0.00231522918344578\\
10076.4253448907	0.0026887576695916\\
10082.2854352338	0.00303550441144205\\
10088.1455255769	0.00334731652848656\\
10094.0056159199	0.00361687358081613\\
10099.865706263	0.00383785941951596\\
10105.725796606	0.00400511017125079\\
10111.5858869491	0.00411473489116484\\
10117.4459772922	0.00416420606459005\\
10123.3060676352	0.00415241785433444\\
10129.1661579783	0.00407971075127527\\
10135.0262483213	0.00394786208149846\\
10140.8863386644	0.00376004262754251\\
10146.7464290074	0.00352074042202343\\
10152.6065193505	0.00323565354394054\\
10158.4666096936	0.0029115544778712\\
10164.3267000366	0.0025561292641822\\
10170.1867903797	0.0021777952602686\\
10176.0468807227	0.00178550183289332\\
10181.9069710658	0.00138851870049675\\
10187.7670614089	0.000996216930582045\\
10193.6271517519	0.000617847765708381\\
10199.487242095	0.000262324497739244\\
10205.347332438	-6.1987467481811e-05\\
10211.2074227811	-0.00034746741081545\\
10217.0675131242	-0.000587418336644094\\
10222.9276034672	-0.000776224040201815\\
10228.7876938103	-0.0009094805221939\\
10234.6477841533	-0.000984098676312138\\
10240.5078744964	-0.000998375855852838\\
10246.3679648395	-0.000952034662309488\\
10252.2280551825	-0.000846228073398956\\
10258.0881455256	-0.000683510823126175\\
10263.9482358686	-0.000467777742389184\\
10269.8083262117	-0.000204170547546042\\
10275.6684165548	0.000101044693077294\\
10281.5285068978	0.000440626495162082\\
10287.3885972409	0.000806529757394639\\
10293.2486875839	0.00119009634824351\\
10299.108777927	0.00158225996608609\\
10304.9688682701	0.00197376042610718\\
10310.8289586131	0.00235536230955139\\
10316.6890489562	0.00271807281358105\\
10322.5491392992	0.00305335366302287\\
10328.4092296423	0.00335332209138283\\
10334.2693199853	0.00361093616065831\\
10340.1294103284	0.00382016006354836\\
10345.9895006715	0.00397610552871888\\
10351.8495910145	0.00407514601630697\\
10357.7096813576	0.00411500103698702\\
10363.5697717006	0.00409478863462869\\
10369.4298620437	0.00401504482546074\\
10375.2899523868	0.0038777095678259\\
10381.1500427298	0.00368607962577793\\
10387.0101330729	0.00344472947134819\\
10392.8702234159	0.00315940212184735\\
10398.730313759	0.00283687251706336\\
10404.5904041021	0.00248478668600041\\
10410.4504944451	0.00211148052044146\\
10416.3105847882	0.00172578245042778\\
10422.1706751312	0.00133680469212515\\
10428.0307654743	0.000953728003514114\\
10433.8908558174	0.000585585031828312\\
10439.7509461604	0.000241047364741797\\
10445.6110365035	-7.17786946231689e-05\\
10451.4711268465	-0.000345542819937063\\
10457.3312171896	-0.000573823335669707\\
10463.1913075327	-0.00075127781698924\\
10469.0513978757	-0.000873768084241219\\
10474.9114882188	-0.00093845666803155\\
10480.7715785618	-0.000943872496941892\\
10486.6316689049	-0.000889944287736941\\
10492.491759248	-0.000778000881692425\\
10498.351849591	-0.000610738550696399\\
10504.2119399341	-0.000392156076208997\\
10510.0720302771	-0.000127459163385799\\
10515.9321206202	0.000177063525259744\\
10521.7922109633	0.000514189765212761\\
10527.6523013063	0.000875935045694689\\
10533.5123916494	0.00125374166453445\\
10539.3724819924	0.00163868116430806\\
10545.2325723355	0.0020216653216966\\
10551.0926626786	0.0023936607045229\\
10556.9527530216	0.00274590173169188\\
10562.8128433647	0.00307009721166655\\
10568.6729337077	0.00335862549278282\\
10574.5330240508	0.00360471363352503\\
10580.3931143938	0.00380259638053253\\
10586.2532047369	0.00394765122377145\\
10592.11329508	0.0040365063650835\\
10597.973385423	0.00406711907964583\\
10603.8334757661	0.00403882264862621\\
10609.6935661091	0.00395234078702719\\
10615.5536564522	0.00380976925589593\\
10621.4137467953	0.00361452512434587\\
10627.2738371383	0.00337126490894524\\
10633.1339274814	0.00308577355024298\\
10638.9940178244	0.00276482687384703\\
10644.8541081675	0.00241603080492403\\
10650.7141985106	0.00204764115135044\\
10656.5742888536	0.00166836822485437\\
10662.4343791967	0.00128717092367804\\
10668.2944695397	0.000913045144044303\\
10674.1545598828	0.000554811517343699\\
10680.0146502259	0.000220907480782324\\
10685.8747405689	-8.08114172161412e-05\\
10691.734830912	-0.000343256298508176\\
10697.594921255	-0.000560271477435844\\
10703.4550115981	-0.000726778812581885\\
10709.3151019412	-0.000838896503484505\\
10715.1751922842	-0.000894029553685494\\
10721.0352826273	-0.000890929792516086\\
10726.8953729703	-0.000829724068009946\\
10732.7554633134	-0.000711909975547922\\
10738.6155536565	-0.000540319253106635\\
10744.4756439995	-0.000319049736759018\\
10750.3357343426	-5.3367510771315e-05\\
10756.1958246856	0.000250418411753181\\
10762.0559150287	0.000585105895872639\\
10767.9160053717	0.000942770422780636\\
10773.7760957148	0.00131495271158743\\
10779.6361860579	0.00169285878333361\\
10785.4962764009	0.0020675677359539\\
10791.356366744	0.00243024232186667\\
10797.216457087	0.00277233735769858\\
10803.0765474301	0.00308580105198491\\
10808.9366377732	0.00336326450764631\\
10814.7967281162	0.00359821494007103\\
10820.6568184593	0.0037851485389755\\
10826.5169088023	0.00391969938738359\\
10832.3769991454	0.00399874141766316\\
10838.2370894885	0.00402046102484192\\
10844.0971798315	0.00398439865071868\\
10849.9572701746	0.00389145838628969\\
10855.8173605176	0.00374388539552924\\
10861.6774508607	0.00354521172185579\\
10867.5375412038	0.00330017178559698\\
10873.3976315468	0.00301458959249164\\
10879.2577218899	0.00269524034099343\\
10885.1178122329	0.00234968971526866\\
10890.977902576	0.0019861146766046\\
10896.8379929191	0.00161310999657833\\
10902.6980832621	0.00123948510972467\\
10908.5581736052	0.000874056086533502\\
10914.4182639482	0.000525437638647681\\
10920.2783542913	0.000201840063112207\\
10926.1384446344	-8.91240887661348e-05\\
10931.9985349774	-0.000340619065828483\\
10937.8586253205	-0.000546746483764736\\
10943.7187156635	-0.000702683632908234\\
10949.5788060066	-0.000804796284483161\\
10955.4388963497	-0.000850723357193363\\
10961.2989866927	-0.000839431472165179\\
10967.1590770358	-0.000771238136418995\\
10973.0191673788	-0.00064780303624405\\
10978.8792577219	-0.000472087674542846\\
10984.7393480649	-0.000248284332806725\\
10990.599438408	1.82839388482019e-05\\
10996.4595287511	0.000321289918300479\\
11002.3196190941	0.000653552384313878\\
11008.1797094372	0.00100720630188527\\
11014.0397997802	0.00137388898730985\\
11019.8998901233	0.00174493784698165\\
11025.7599804664	0.00211159501458328\\
11031.6200708094	0.00246521405312791\\
11037.4801611525	0.00279746384381757\\
11043.3402514955	0.0031005248539906\\
11049.2003418386	0.00336727316123264\\
11055.0604321817	0.00359144790304411\\
11060.9205225247	0.00376779821679077\\
11066.7806128678	0.00389220622127304\\
11072.6407032108	0.0039617831605952\\
11078.5007935539	0.00397493646565578\\
11084.360883897	0.00393140617704841\\
11090.22097424	0.0038322698974219\\
11096.0810645831	0.00367991618436731\\
11101.9411549261	0.00347798703931719\\
11107.8012452692	0.00323129087659242\\
11113.6613356123	0.00294568805187241\\
11119.5214259553	0.00262795167483101\\
11125.3815162984	0.00228560701108759\\
11131.2416066414	0.00192675328124091\\
11137.1016969845	0.00155987207647026\\
11142.9617873276	0.00119362692170963\\
11148.8218776706	0.000836658723243469\\
11154.6819680137	0.000497381929508431\\
11160.5420583567	0.000183786213358322\\
11166.4021486998	-9.6751650451197e-05\\
11172.2622390429	-0.00033764128368209\\
11178.1223293859	-0.00053323349861615\\
11183.982419729	-0.000678952752354025\\
11189.842510072	-0.000771404156191048\\
11195.7026004151	-0.000808452537875103\\
11201.5626907581	-0.000789271715960848\\
11207.4227811012	-0.000714362849918624\\
11213.2828714443	-0.000585541459919407\\
11219.1429617873	-0.000405893449987297\\
11225.0030521304	-0.000179701198804234\\
11230.8631424734	8.76585120602249e-05\\
11236.7232328165	0.000389842300133102\\
11242.5833231596	0.000719690676834046\\
11248.4434135026	0.00106939766507679\\
11254.3035038457	0.00143069552166349\\
11260.1635941887	0.00179505019347766\\
11266.0236845318	0.00215386288732921\\
11271.8837748749	0.00249867299264071\\
11277.7438652179	0.00282135756896189\\
11283.603955561	0.00311432269489252\\
11289.464045904	0.0033706821712818\\
11295.3241362471	0.00358441937418049\\
11301.1842265902	0.00375052845314002\\
11307.0443169332	0.00386513156220703\\
11312.9044072763	0.00392556937828703\\
11318.7644976193	0.00393046279449555\\
11324.6245879624	0.0038797443577932\\
11330.4846783055	0.00377465873547897\\
11336.3447686485	0.00361773222557966\\
11342.2048589916	0.00341271205754877\\
11348.0649493346	0.00316447694066062\\
11353.9250396777	0.00287892099522269\\
11359.7851300208	0.0025628138275367\\
11365.6452203638	0.00222364006997352\\
11371.5053107069	0.00186942218896967\\
11377.3654010499	0.00150853075579784\\
11383.225491393	0.00114948666634536\\
11389.0855817361	0.000800759982938834\\
11394.9456720791	0.000470570146491156\\
11400.8057624222	0.000166692271139188\\
11406.6658527652	-0.000103725914057236\\
11412.5259431083	-0.000334332166770206\\
11418.3860334513	-0.000519718923417286\\
11424.2461237944	-0.00065555007787772\\
11430.1062141375	-0.000738662375043392\\
11435.9663044805	-0.000767139048648679\\
11441.8263948236	-0.000740353989706235\\
11447.6864851666	-0.000658985428308061\\
11453.5465755097	-0.000524998831970757\\
11459.4066658528	-0.0003415994497413\\
11465.2667561958	-0.00011315564734382\\
11471.1268465389	0.000154905133447802\\
11476.9869368819	0.000456225313723368\\
11482.847027225	0.000783667951793509\\
11488.7071175681	0.00112948577771722\\
11494.5672079111	0.00148550448483533\\
11500.4272982542	0.00184331594227765\\
11506.2873885972	0.00219447676199529\\
11512.1474789403	0.00253070753122748\\
11518.0075692834	0.00284408800978299\\
11523.8676596264	0.00312724369047676\\
11529.7277499695	0.00337351932864017\\
11535.5878403125	0.00357713535795049\\
11541.4479306556	0.00373332351644226\\
11547.3080209987	0.00383843850080378\\
11553.1681113417	0.00389004303458434\\
11559.0282016848	0.00388696436550349\\
11564.8882920278	0.00382932088312443\\
11570.7483823709	0.00371851825338741\\
11576.608472714	0.00355721518658443\\
11582.468563057	0.00334925967178131\\
11588.3286534001	0.00309959720636045\\
11594.1887437431	0.00281415320880766\\
11600.0488340862	0.00249969241031521\\
11605.9089244293	0.00216365856146674\\
11611.7690147723	0.00181399825166154\\
11617.6291051154	0.00145897301096747\\
11623.4891954584	0.0011069641368053\\
11629.3492858015	0.000766274855940216\\
11635.2093761446	0.000444934491304871\\
11641.0694664876	0.000150509252026209\\
11646.9295568307	-0.000110075895037949\\
11652.7896471737	-0.000330700078967078\\
11658.6497375168	-0.000506190274248848\\
11664.5098278598	-0.000632442570444803\\
11670.3699182029	-0.000706518119476864\\
11676.230008546	-0.000726711517823789\\
11682.090098889	-0.000692590032835402\\
11687.9501892321	-0.00060500277171346\\
11693.8102795751	-0.000466059601298056\\
11699.6703699182	-0.000279080340854924\\
11705.5304602613	-4.85154505816385e-05\\
11711.3905506043	0.000220159889687517\\
11717.2506409474	0.000520575791086038\\
11723.1107312904	0.000845618668812081\\
11728.9708216335	0.00118759967797026\\
11734.8309119766	0.00153843658425448\\
11740.6910023196	0.0018898447679716\\
11746.5510926627	0.00223353284780547\\
11752.4111830057	0.00256139830547651\\
11758.2712733488	0.00286571849612794\\
11764.1313636919	0.0031393325419885\\
11769.9914540349	0.00337580982533867\\
11775.851544378	0.00356960111660585\\
11781.711634721	0.00371616878650527\\
11787.5717250641	0.00381209304655986\\
11793.4318154072	0.00385515173155192\\
11799.2919057502	0.00384437176269776\\
11805.1519960933	0.00378005110029099\\
11811.0120864363	0.00366375069161708\\
11816.8721767794	0.00349825662833911\\
11822.7322671225	0.00328751343019484\\
11828.5923574655	0.00303653005164627\\
11834.4524478086	0.00275126085144999\\
11840.3125381516	0.00243846435305652\\
11846.1726284947	0.00210554314583713\\
11852.0327188377	0.00176036871911771\\
11857.8928091808	0.0014110953738765\\
11863.7528995239	0.00106596761078129\\
11869.6129898669	0.00073312554389767\\
11875.47308021	0.000420412932435065\\
11881.333170553	0.000135192357265997\\
11887.1932608961	-0.000115828103150199\\
11893.0533512392	-0.000326752617424923\\
11898.9134415822	-0.000492636057467425\\
11904.7735319253	-0.000609599915083926\\
11910.6336222683	-0.000674922962167231\\
11916.4937126114	-0.000687104535664512\\
11922.3538029545	-0.000645898976044536\\
11928.2138932975	-0.000552320429514992\\
11934.0739836406	-0.000408617924373255\\
11939.9340739836	-0.000218221332815659\\
11945.7941643267	1.43404838178694e-05\\
11951.6542546698	0.000283547393260629\\
11957.5143450128	0.000583019012334543\\
11963.3744353559	0.00090566591928914\\
11969.2345256989	0.00124385747522906\\
11975.094616042	0.00158960228214142\\
11980.9547063851	0.00193473701038974\\
11986.8147967281	0.00227111913406627\\
11992.6748870712	0.00259081902417382\\
11998.5349774142	0.00288630686836549\\
12004.3950677573	0.0031506300112117\\
12010.2551581004	0.00337757653881428\\
12016.1152484434	0.00356182125941683\\
12021.9753387865	0.00369905064929623\\
12027.8354291295	0.00378606383305424\\
12033.6955194726	0.0038208472346446\\
12039.5556098157	0.00380262115837028\\
12045.4157001587	0.00373185722264122\\
12051.2757905018	0.00361026625855431\\
12057.1358808448	0.00344075698087566\\
12062.9959711879	0.00322736643019985\\
12068.8560615309	0.00297516384823449\\
12074.716151874	0.0026901302757762\\
12080.5762422171	0.00237901673159098\\
12086.4363325601	0.00204918433577047\\
12092.2964229032	0.00170843016272789\\
12098.1565132462	0.00136480294354482\\
12104.0166035893	0.00102641297382349\\
12109.8766939324	0.000701240715803483\\
12115.7367842754	0.000396948611615346\\
12121.5968746185	0.0001207005447353\\
12127.4569649615	-0.000121006796832772\\
12133.3170553046	-0.000322496686323777\\
12139.1771456477	-0.000479045661057906\\
12145.0372359907	-0.000586994232414466\\
12150.8973263338	-0.000643832408403683\\
12156.7574166768	-0.000648258030506057\\
12162.6175070199	-0.000600206569420244\\
12168.477597363	-0.000500851698535456\\
12174.337687706	-0.000352576654140911\\
12180.1977780491	-0.000158917080374117\\
12186.0578683921	7.55232708467938e-05\\
12191.9179587352	0.000345181975279903\\
12197.7780490783	0.000643669906154229\\
12203.6381394213	0.000963922607587565\\
12209.4982297644	0.00129836748561333\\
12215.3583201074	0.00163910286021765\\
12221.2184104505	0.00197808464517036\\
12227.0785007936	0.00230731624555055\\
12232.9385911366	0.00261903719041244\\
12238.7986814797	0.00290590605113968\\
12244.6587718227	0.00316117333453156\\
12250.5188621658	0.00337884027913275\\
12256.3789525089	0.00355379981892677\\
12262.2390428519	0.00368195640269436\\
12268.099133195	0.00376032185772527\\
12273.959223538	0.00378708505443428\\
12279.8193138811	0.00376165374866157\\
12285.6794042241	0.00368466763461515\\
12291.5394945672	0.00355798232289191\\
12297.3995849103	0.00338462464340911\\
12303.2596752533	0.00316872034916426\\
12309.1197655964	0.00291539594744337\\
12314.9798559394	0.00263065699378987\\
12320.8399462825	0.00232124573828104\\
12326.7000366256	0.0019944814973436\\
12332.5601269686	0.00165808753087376\\
12338.4202173117	0.00132000851873012\\
12344.2803076547	0.000988222950008418\\
12350.1403979978	0.000670554855121089\\
12356.0004883409	0.000374489322910161\\
12361.8605786839	0.000106996152381589\\
12367.720669027	-0.000125634206525346\\
12373.58075937	-0.00031793856165308\\
12379.4408497131	-0.000465409259387132\\
12385.3009400562	-0.00056459982558779\\
12391.1610303992	-0.000613205491169877\\
12397.0211207423	-0.000610116721304849\\
12402.8812110853	-0.000555444505329712\\
12408.7413014284	-0.000450516831741952\\
12414.6013917715	-0.00029784645285955\\
12420.4614821145	-0.000101070720897505\\
12426.3215724576	0.00013513506430627\\
12432.1816628006	0.00040516873201525\\
12438.0417531437	0.000702634103045424\\
12443.9018434868	0.0010204924872384\\
12449.7619338298	0.00135122922800426\\
12455.6220241729	0.00168703135342849\\
12461.4821145159	0.00201997213471457\\
12467.342204859	0.00234219819242049\\
12473.2022952021	0.00264611473427888\\
12479.0623855451	0.00292456455585278\\
12484.9224758882	0.00317099658537821\\
12490.7825662312	0.00337962000402058\\
12496.6426565743	0.00354554031603154\\
12502.5027469173	0.00366487417201057\\
12508.3628372604	0.00373484025126078\\
12514.2229276035	0.00375382407735921\\
12520.0830179465	0.00372141525634715\\
12525.9431082896	0.00363841627921462\\
12531.8031986326	0.00350682270205226\\
12537.6632889757	0.00332977519202992\\
12543.5233793188	0.00311148459138351\\
12549.3834696618	0.00285713178704507\\
12555.2435600049	0.002572744766284\\
12561.1036503479	0.00226505577614567\\
12566.963740691	0.00194134197063009\\
12572.8238310341	0.00160925331756385\\
12578.6839213771	0.00127663183456044\\
12584.5440117202	0.000951326424858319\\
12590.4041020632	0.000641007685132349\\
12596.2641924063	0.000352987054251422\\
12602.1242827494	9.40445667776325e-05\\
12607.9843730924	-0.000129730731236164\\
12613.8444634355	-0.000313083948290628\\
12619.7045537785	-0.000451717729446316\\
12625.5646441216	-0.000542392957633614\\
12631.4247344647	-0.000583004414815551\\
12637.2848248077	-0.000572629635952271\\
12643.1449151508	-0.000511549822465491\\
12649.0050054938	-0.000401242341946952\\
12654.8650958369	-0.000244345011436138\\
12660.72518618	-4.45930272548612e-05\\
12666.585276523	0.000193269950974337\\
12672.4453668661	0.000463604445762696\\
12678.3054572091	0.000760008862021156\\
12684.1655475522	0.00107547107263958\\
12690.0256378953	0.00140253430026165\\
12695.8857282383	0.00173347337133613\\
12701.7458185814	0.00206047717666717\\
12707.6059089244	0.002375833029362\\
12713.4659992675	0.00267210856891523\\
12719.3260896105	0.00294232692201863\\
12725.1861799536	0.00318013099173745\\
12731.0462702967	0.00337993300728251\\
12736.9063606397	0.00353704581610459\\
12742.7664509828	0.00364779283380925\\
12748.6265413258	0.00370959407306639\\
12754.4866316689	0.00372102623895638\\
12760.346722012	0.00368185549068407\\
12766.206812355	0.00359304211695406\\
12772.0669026981	0.00345671703384009\\
12777.9269930411	0.00327613068007132\\
12783.7870833842	0.00305557553454391\\
12789.6471737273	0.00280028410320546\\
12795.5072640703	0.0025163047989812\\
12801.3673544134	0.00221035865913803\\
12807.2274447564	0.00188968029365477\\
12813.0875350995	0.0015618468286643\\
12818.9476254426	0.00123459888875648\\
12824.8077157856	0.000915657847748411\\
12830.6678061287	0.000612543661835639\\
12836.5278964717	0.000332397582802178\\
12842.3879868148	8.18139306052683e-05\\
12848.2480771579	-0.00013331511204746\\
12854.1081675009	-0.000307938030400445\\
12859.968257844	-0.00043796257693283\\
12865.828348187	-0.000520351655020747\\
12871.6884385301	-0.000553194240633414\\
12877.5485288732	-0.000535749686200485\\
12883.4086192162	-0.000468464380026777\\
12889.2687095593	-0.000352960387325561\\
12895.1287999023	-0.000191996360520902\\
12900.9888902454	1.05983396940968e-05\\
12906.8489805884	0.000250014739631456\\
12912.7090709315	0.00052057839741305\\
12918.5691612746	0.000815883888288386\\
12924.4292516176	0.00112894644337695\\
12930.2893419607	0.00145236715229779\\
12936.1494323037	0.00177850782262507\\
12942.0095226468	0.00209967136398446\\
12947.8696129899	0.00240828343697946\\
12953.7297033329	0.00269707108066291\\
12959.589793676	0.00295923410579319\\
12965.449884019	0.00318860521565528\\
12971.3099743621	0.00337979508384935\\
12977.1700647052	0.00352831897704251\\
12983.0301550482	0.00363070194760178\\
12988.8902453913	0.0036845601292744\\
12994.7503357343	0.00368865623383579\\
13000.6104260774	0.00364292795769067\\
13006.4705164205	0.00354848864624062\\
13012.3306067635	0.00340760021986583\\
13018.1906971066	0.00322361901866571\\
13024.0507874496	0.00300091586296916\\
13029.9108777927	0.00274477223260003\\
13035.7709681358	0.00246125503114236\\
13041.6310584788	0.00215707290440395\\
13047.4911488219	0.00183941751565272\\
13053.3512391649	0.00151579353264106\\
13059.211329508	0.00119384134555545\\
13065.0714198511	0.000881156703204051\\
13070.9315101941	0.000585111525206799\\
13076.7916005372	0.000312680116960369\\
13082.6516908802	7.02748838355391e-05\\
13088.5117812233	-0.000136404585615859\\
13094.3718715664	-0.000302505515995168\\
13100.2319619094	-0.000424135870874628\\
13106.0920522525	-0.000498455533806195\\
13111.9521425955	-0.000523742608611341\\
13117.8122329386	-0.000499433291546825\\
13123.6723232817	-0.000426134392418144\\
13129.5324136247	-0.000305608227740763\\
13135.3925039678	-0.000140730260949095\\
13141.2525943108	6.45794925091613e-05\\
13147.1126846539	0.00030544965918062\\
13152.9727749969	0.000576173085354545\\
13158.83286534	0.000870342056633228\\
13164.6929556831	0.00118099995578919\\
13170.5530460261	0.0015008057697849\\
13176.4131363692	0.00182220755591233\\
13182.2732267122	0.00213762076854656\\
13188.1333170553	0.00243960723559769\\
13193.9934073984	0.00272105056194462\\
13199.8534977414	0.002975323823287\\
13205.7135880845	0.0031964455997176\\
13211.5736784275	0.00337922067474385\\
13217.4337687706	0.00351936209084715\\
13223.2938591137	0.00361359169415607\\
13229.1539494567	0.00365971681014061\\
13235.0140397998	0.00365668125752385\\
13240.8741301428	0.00360458951335667\\
13246.7342204859	0.00350470347765947\\
13252.594310829	0.00335941193135101\\
13258.454401172	0.00317217342681173\\
13264.3144915151	0.00294743397577846\\
13270.1745818581	0.002690521493395\\
13276.0346722012	0.00240751950441071\\
13281.8947625443	0.00210512310452079\\
13287.7548528873	0.00179048058761845\\
13293.6149432304	0.00147102448452132\\
13299.4750335734	0.00115429600674359\\
13305.3351239165	0.00084776704176819\\
13311.1952142596	0.000558663901163979\\
13317.0553046026	0.000293796978764833\\
13322.9153949457	5.94003341502795e-05\\
13328.7754852887	-0.000139015020327091\\
13334.6355756318	-0.000296790676461904\\
13340.4956659748	-0.000410230185557357\\
13346.3557563179	-0.000476685645360349\\
13352.215846661	-0.000494619490605733\\
13358.075937004	-0.000463640045495784\\
13363.9360273471	-0.000384510016431121\\
13369.7961176901	-0.000259127742386496\\
13375.6562080332	-9.04816629876709e-05\\
13381.5162983763	0.000117420903896955\\
13387.3763887193	0.000359648977902956\\
13393.2364790624	0.000630464863146462\\
13399.0965694054	0.000923460053034822\\
13404.9566597485	0.00123170687408411\\
13410.8167500916	0.00154792228059124\\
13416.6768404346	0.0018646399278886\\
13422.5369307777	0.00217438645866533\\
13428.3970211207	0.00246985784069783\\
13434.2571114638	0.00274409159504284\\
13440.1172018069	0.00299063085426809\\
13445.9772921499	0.00320367638486224\\
13451.837382493	0.00337822299508318\\
13457.697472836	0.00351017711966958\\
13463.5575631791	0.00359645281948817\\
13469.4176535222	0.00363504394500396\\
13475.2777438652	0.00362507077624197\\
13481.1378342083	0.00356680005502438\\
13486.9979245513	0.0034616379546464\\
13492.8580148944	0.00331209616943637\\
13498.7181052375	0.00312173194208246\\
13504.5781955805	0.00289506346023967\\
13510.4382859236	0.00263746263442819\\
13516.2983762666	0.00235502780130478\\
13522.1584666097	0.00205443936895238\\
13528.0185569528	0.00174280182061814\\
13533.8786472958	0.00142747581368767\\
13539.7387376389	0.00111590434142437\\
13545.5988279819	0.000815437062982916\\
13551.458918325	0.000533156947610412\\
13557.3190086681	0.000275713321537797\\
13563.1790990111	4.9165252847688e-05\\
13569.0391893542	-0.000141161037301188\\
13574.8992796972	-0.00029079738163158\\
13580.7593700403	-0.000396238548846174\\
13586.6194603833	-0.00045502433916357\\
13592.4795507264	-0.000465796970757441\\
13598.3396410695	-0.000428332418795748\\
13604.1997314125	-0.000343544984132847\\
13610.0598217556	-0.000213465000840088\\
13615.9199120986	-4.11902254466342e-05\\
13621.7800024417	0.000169187928476629\\
13627.6400927848	0.000412681554150447\\
13633.5001831278	0.000683524506567394\\
13639.3602734709	0.000975308945195455\\
13645.2203638139	0.00128113693133899\\
13651.080454157	0.00159378349377976\\
13656.9405445001	0.00190586730827287\\
13662.8006348431	0.00221002495888296\\
13668.6607251862	0.002499084667487\\
13674.5208155292	0.00276623539278202\\
13680.3809058723	0.00300518731189572\\
13686.2409962154	0.00321031990360969\\
13692.1010865584	0.00337681414709833\\
13697.9611769015	0.0035007657267381\\
13703.8212672445	0.00357927658441986\\
13709.6813575876	0.00361052267204781\\
13715.5414479307	0.00359379632114949\\
13721.4015382737	0.00352952224543625\\
13727.2616286168	0.0034192468150607\\
13733.1217189598	0.00326560087258786\\
13738.9818093029	0.00307223698416493\\
13744.841899646	0.00284374262236742\\
13750.701989989	0.00258553134405858\\
13756.5620803321	0.00230371454466459\\
13762.4221706751	0.00200495682623152\\
13768.2822610182	0.0016963184025553\\
13774.1423513612	0.00138508826722323\\
13780.0024417043	0.0010786120667953\\
13785.8625320474	0.00078411874361199\\
13791.7226223904	0.000508550039020138\\
13797.5827127335	0.000258396878264007\\
13803.4428030765	3.9546492979655e-05\\
13809.3028934196	-0.000142856118191634\\
13815.1629837627	-0.000284529130962334\\
13821.0230741057	-0.000382154396046635\\
13826.8831644488	-0.000433455140392547\\
13832.7432547918	-0.000437249049686191\\
13838.6033451349	-0.000393475494789959\\
13844.463435478	-0.000303196275538598\\
13850.323525821	-0.000168569880621201\\
13856.1836161641	7.20011301820675e-06\\
13862.0437065071	0.000219941267876601\\
13867.9037968502	0.000464611327181464\\
13873.7638871933	0.000735417718974825\\
13879.6239775363	0.00102595469092932\\
13885.4840678794	0.00132935482955634\\
13891.3441582224	0.00163845137996679\\
13897.2042485655	0.00194594752989359\\
13903.0643389086	0.00224458865951919\\
13908.9244292516	0.00252733349123196\\
13914.7845195947	0.00278752010166614\\
13920.6446099377	0.00301902288247949\\
13926.5047002808	0.00321639675151773\\
13932.3647906239	0.00337500522006096\\
13938.2248809669	0.00349112930359026\\
13944.08497131	0.00356205471872728\\
13949.945061653	0.00358613532134955\\
13955.8051519961	0.00356283130393929\\
13961.6652423392	0.00349272126603635\\
13967.5253326822	0.00337748788853582\\
13973.3854230253	0.00321987756595162\\
13979.2455133683	0.00302363496483255\\
13985.1056037114	0.0027934140670323\\
13990.9656940544	0.00253466781147139\\
13996.8257843975	0.0022535189504021\\
14002.6858747406	0.00195661517922525\\
14008.5459650836	0.00165097196672921\\
14014.4060554267	0.00134380680201195\\
14020.2661457697	0.00104236877370208\\
14026.1262361128	0.000753767505555156\\
14031.9863264559	0.000484805484690877\\
14037.8464167989	0.000241817736857821\\
14043.706507142	3.05226269723719e-05\\
14049.566597485	-0.000144112701387935\\
14055.4266878281	-0.000277989081314912\\
14061.2867781712	-0.000367971528561726\\
14067.1468685142	-0.000411962640454991\\
14073.0069588573	-0.000408951469478573\\
14078.8670492003	-0.000359036732912303\\
14084.7271395434	-0.000263423826120395\\
14090.5872298865	-0.000124395725499868\\
14096.4473202295	5.47415169383387e-05\\
14102.3074105726	0.000269737386289097\\
14108.1675009156	0.000515497755686738\\
14114.0275912587	0.000786205582809815\\
14119.8876816018	0.00107545859238555\\
14125.7477719448	0.0013764206862526\\
14131.6078622879	0.00168198350024101\\
14137.4679526309	0.00198493429025324\\
14143.328042974	0.00227812618235113\\
14149.1881333171	0.00255464676876176\\
14155.0482236601	0.00280798107254935\\
14160.9083140032	0.00303216503936423\\
14166.7684043462	0.00322192594002392\\
14172.6284946893	0.00337280637915283\\
14178.4885850324	0.00348126899325907\\
14184.3486753754	0.00354477937927039\\
14190.2087657185	0.00356186530974095\\
14196.0688560615	0.00353215085120681\\
14201.9289464046	0.00345636459554653\\
14207.7890367477	0.00333632182494585\\
14213.6491270907	0.00317488104669863\\
14219.5092174338	0.00297587593821246\\
14225.3693077768	0.00274402432222415\\
14231.2293981199	0.00248481633338656\\
14237.0894884629	0.0022043844268812\\
14242.949578806	0.00190935830668996\\
14248.8096691491	0.00160670820536757\\
14254.6697594921	0.00130358021957292\\
14260.5298498352	0.00100712759141242\\
14266.3899401782	0.000724341918675397\\
14272.2500305213	0.000461888276574824\\
14278.1101208644	0.000225948139063632\\
14283.9702112074	2.20738013458808e-05\\
14289.8303015505	-0.000144942268043085\\
14295.6903918935	-0.000271180071694326\\
14301.5504822366	-0.000353684076786514\\
14307.4105725797	-0.000390532398900328\\
14313.2706629227	-0.000380881556884313\\
14319.1307532658	-0.000324985756845948\\
14324.9908436088	-0.000224190264832648\\
14330.8509339519	-8.08990394676799e-05\\
14336.711024295	0.000101482597034424\\
14342.571114638	0.000318628882536111\\
14348.4312049811	0.00056539621042337\\
14354.2912953241	0.000835944963785444\\
14360.1513856672	0.00112387770259079\\
14366.0114760103	0.00142239043416452\\
14371.8715663533	0.00172443339015354\\
14377.7316566964	0.00202287751120831\\
14383.5917470394	0.00231068270777798\\
14389.4518373825	0.00258106392615789\\
14395.3119277256	0.00282765110239682\\
14401.1720180686	0.00304463923383036\\
14407.0321084117	0.00322692503256452\\
14412.8921987547	0.00337022694442108\\
14418.7522890978	0.0034711857108244\\
14424.6123794408	0.00352744311202012\\
14430.4724697839	0.0035376970457357\\
14436.332560127	0.00350173165535322\\
14442.19265047	0.00342042181080717\\
14448.0527408131	0.00329571185067635\\
14453.9128311561	0.00313056910176388\\
14459.7729214992	0.00292891328709512\\
14465.6330118423	0.00269552350139673\\
14471.4931021853	0.00243592496150328\\
14477.3531925284	0.00215625821548399\\
14483.2132828714	0.0018631339060022\\
14489.0733732145	0.0015634765229513\\
14494.9334635576	0.00126436083839635\\
14500.7935539006	0.000972844886872625\\
14506.6536442437	0.000695803434123047\\
14512.5137345867	0.00043976586314819\\
14518.3738249298	0.000210762300133952\\
14524.2339152729	1.4181606485196e-05\\
14530.0940056159	-0.000145355419131996\\
14535.954095959	-0.000264104645316144\\
14541.814186302	-0.000339286466673613\\
14547.6742766451	-0.000369150855255968\\
14553.5343669882	-0.000353018082501412\\
14559.3944573312	-0.000291294164323109\\
14565.2545476743	-0.000185460678968508\\
14571.1146380173	-3.80392120851743e-05\\
14576.9747283604	0.000147468718030301\\
14582.8348187035	0.000366664823968551\\
14588.6949090465	0.000614358326495787\\
14594.5549993896	0.00088468887352642\\
14600.4150897326	0.00117126519018534\\
14606.2751800757	0.00146731617983447\\
14612.1352704188	0.00176585090394756\\
14617.9953607618	0.00205982366122552\\
14623.8554511049	0.002342300268085\\
14629.7155414479	0.00260662161640171\\
14635.575631791	0.00284656065096442\\
14641.4357221341	0.00305646906617973\\
14647.2958124771	0.00323141026656279\\
14653.1559028202	0.00336727546096407\\
14659.0159931632	0.00346088016106414\\
14664.8760835063	0.00351003881731542\\
14670.7361738493	0.00351361584370634\\
14676.5962641924	0.00347155184018167\\
14682.4563545355	0.00338486440746752\\
14688.3164448785	0.00325562354887474\\
14694.1765352216	0.00308690225354372\\
14700.0366255646	0.00288270344049986\\
14705.8967159077	0.00264786499999256\\
14711.7568062508	0.00238794518504862\\
14717.6168965938	0.00210909106740961\\
14723.4769869369	0.00181789317173546\\
14729.3370772799	0.00152122972457019\\
14735.197167623	0.00122610419944735\\
14741.0572579661	0.00093947999445889\\
14746.9173483091	0.000668116144800087\\
14752.7774386522	0.000418407946166413\\
14758.6375289952	0.000196236246832203\\
14764.4976193383	6.82895969455063e-06\\
14770.3577096814	-0.000145361944579666\\
14776.2178000244	-0.000256765069308663\\
14782.0778903675	-0.000324773389527042\\
14787.9379807105	-0.000347805249640168\\
14793.7980710536	-0.000325341134040087\\
14799.6581613967	-0.000257935356010565\\
14805.5182517397	-0.000147202402636761\\
14811.3783420828	4.22172853864268e-06\\
14817.2384324258	0.000192742275460329\\
14823.0985227689	0.000413891046680078\\
14828.958613112	0.000662432320091626\\
14834.818703455	0.00093248679559564\\
14840.6787937981	0.00121767066731926\\
14846.5388841411	0.00151124652591381\\
14852.3989744842	0.0018062825240459\\
14858.2590648272	0.00209581604510282\\
14864.1191551703	0.00237301801082623\\
14869.9792455134	0.00263135395085735\\
14875.8393358564	0.00286473803519307\\
14881.6994261995	0.00306767643896439\\
14887.5595165425	0.00323539666232488\\
14893.4196068856	0.00336395976184966\\
14899.2796972287	0.00345035285387688\\
14905.1397875717	0.00349255971797196\\
14910.9998779148	0.00348960784607493\\
14916.8599682578	0.00344159083945087\\
14922.7200586009	0.00334966563817136\\
14928.580148944	0.00321602466163508\\
14934.440239287	0.00304384353095849\\
14940.3003296301	0.00283720561916546\\
14946.1604199731	0.00260100522197361\\
14952.0205103162	0.00234083164499458\\
14957.8806006593	0.00206283695243711\\
14963.7406910023	0.00177359050624812\\
14969.6007813454	0.0014799237351317\\
14975.4608716884	0.00118876880093558\\
14981.3209620315	0.000906994972593629\\
14987.1810523746	0.00064124656956867\\
14993.0411427176	0.000397786297766314\\
14998.9012330607	0.000182347671641106\\
nan	nan\\
15010.6214137468	-0.000144970885391611\\
15016.4815040899	-0.000249163352292922\\
15022.3415944329	-0.0003101397746323\\
15028.201684776	-0.000326483551076439\\
15034.061775119	-0.000297832002020913\\
15039.9218654621	-0.000224884381242437\\
15045.7819558052	-0.000109384826114604\\
15051.6420461482	4.59193383597596e-05\\
15057.5021364913	0.000237342945100516\\
15063.3622268343	0.000460350426003838\\
15069.2223171774	0.000709663273851925\\
15075.0824075204	0.000979384979148323\\
15080.9424978635	0.00126314048489069\\
15086.8025882066	0.00155422686134492\\
15092.6626785496	0.00184577163966613\\
15098.5227688927	0.00213089506466031\\
15104.3828592357	0.00240287243599037\\
15110.2429495788	0.00265529270753584\\
15116.1030399219	0.00288220960418514\\
15121.9631302649	0.00307828169475268\\
15127.823220608	0.00323889812103693\\
15133.683310951	0.00336028702397862\\
15139.5434012941	0.00343960411759045\\
15145.4034916372	0.00347499933022859\\
15151.2635819802	0.00346565995275887\\
15157.1236723233	0.00341182928680324\\
15162.9837626663	0.00331480036596573\\
15168.8438530094	0.00317688491076186\\
15174.7039433525	0.00300135826212091\\
15180.5640336955	0.00279238160564078\\
15186.4241240386	0.00255490333257139\\
15192.2842143816	0.00229454187596013\\
15198.1443047247	0.00201745279600168\\
15204.0043950678	0.00173018325839858\\
15209.8644854108	0.00143951734607849\\
15215.7245757539	0.00115231585894787\\
15221.5846660969	0.000875354384118947\\
15227.44475644	0.000615163458716024\\
15233.3048467831	0.000377874595476299\\
15239.1649371261	0.000169075801332674\\
15245.0250274692	-6.32000663148055e-06\\
15250.8851178122	-0.000144190589545165\\
15256.7452081553	-0.000241301260081821\\
15262.6052984984	-0.00029538076439031\\
15268.4653888414	-0.000305174392639081\\
15274.3254791845	-0.000270473076442667\\
15280.1855695275	-0.000192117798626628\\
15286.0456598706	-7.19792236303063e-05\\
15291.9057502136	8.7086958430274e-05\\
15297.7658405567	0.000281307908173864\\
15303.6259308998	0.000506083120695707\\
15309.4860212428	0.000756093394389925\\
15315.3461115859	0.00102542670394675\\
15321.2062019289	0.00130771799906446\\
15327.066292272	0.00159629962321549\\
15332.9263826151	0.00188435879806841\\
15338.7864729581	0.00216509845392739\\
15344.6465633012	0.00243189760977451\\
15350.5066536442	0.00267846751867429\\
15356.3667439873	0.00289899989653999\\
15362.2268343304	0.00308830374023165\\
15368.0869246734	0.00324192751253394\\
15373.9470150165	0.00335626381843553\\
15379.8071053595	0.00342863411070367\\
15385.6671957026	0.00345735143676508\\
15391.5272860457	0.00344175975666725\\
15397.3873763887	0.00338224891582324\\
15403.2474667318	0.00328024493180674\\
15409.1075570748	0.00313817583574399\\
15414.9676474179	0.00295941388710672\\
15420.827737761	0.0027481955362655\\
15426.687828104	0.0025095210349352\\
15432.5479184471	0.00224903607266765\\
15438.4080087901	0.00197289824158901\\
15444.2680991332	0.00168763148727556\\
15450.1281894763	0.00139997198631395\\
15455.9882798193	0.00111670909117421\\
15461.8483701624	0.000844525097901928\\
15467.7084605054	0.000589837618094773\\
15473.5685508485	0.000358648273218747\\
15479.4286411916	0.000156401278303213\\
15485.2887315346	-1.21447528169376e-05\\
15491.1488218777	-0.000143028762349835\\
15497.0089122207	-0.000233180329688468\\
15502.8690025638	-0.000280491691646587\\
15508.7290929068	-0.000283867012633818\\
15514.5891832499	-0.000243247753188978\\
15520.449273593	-0.000159613549862391\\
15526.309363936	-3.49585975311358e-05\\
15532.1694542791	0.000127755896776441\\
15538.0295446221	0.000324672055058073\\
15543.8896349652	0.000551126793784171\\
15549.7497253083	0.000801762245226182\\
15555.6098156513	0.00107065252003196\\
15561.4699059944	0.00135144381209565\\
15567.3299963374	0.00163750453345362\\
15573.1900866805	0.00192208193174272\\
15579.0501770236	0.00219846149160917\\
15584.9102673666	0.00246012535751445\\
15590.7703577097	0.00270090604026635\\
15596.6304480527	0.00291513178258332\\
15602.4905383958	0.0030977601577299\\
15608.3506287389	0.0032444967545569\\
15614.2107190819	0.00335189615523155\\
15620.070809425	0.00341744283179781\\
15625.930899768	0.00343961006209267\\
15631.7909901111	0.00341789548531318\\
15637.6510804542	0.00335283246943439\\
15643.5111707972	0.00324597703373892\\
15649.3712611403	0.00309987064662115\\
15655.2313514833	0.00291797978783883\\
15661.0914418264	0.00270461371255668\\
15666.9515321695	0.0024648223675902\\
15672.8116225125	0.00220427687848571\\
15678.6717128556	0.00192913543522503\\
15684.5318031986	0.00164589774830985\\
15690.3918935417	0.00136125151483173\\
15696.2519838848	0.00108191452100489\\
15702.1120742278	0.000814476109168386\\
15707.9721645709	0.000565241750036063\\
15713.8322549139	0.000340084386486619\\
15719.692345257	0.000144306053242435\\
15725.5524356	-1.74869670384477e-05\\
15731.4125259431	-0.000141492511864545\\
15737.2726162862	-0.000224801881802328\\
15743.1327066292	-0.000265468058943033\\
15748.9927969723	-0.000262551201128199\\
15754.8528873153	-0.00021614034906893\\
15760.7129776584	-0.000127350845271362\\
15766.5730680015	1.70246302675327e-06\\
15772.4331583445	0.000167955593274787\\
15778.2932486876	0.000367468169909713\\
15784.1533390306	0.000595516812678381\\
15790.0134293737	0.000846706957782417\\
15795.8735197168	0.00111510046472146\\
15801.7336100598	0.00139435599053467\\
15807.5937004029	0.00167787881306794\\
15813.4537907459	0.00195897656392919\\
15819.313881089	0.00223101719329131\\
15825.1739714321	0.00248758543855506\\
15831.0340617751	0.00272263410518938\\
15836.8941521182	0.00293062659267823\\
15842.7542424612	0.00310666730637738\\
15848.6143328043	0.00324661688391024\\
15854.4744231474	0.00334718952380148\\
15860.3345134904	0.00340603012836392\\
15866.1946038335	0.0034217694496319\\
15872.0546941765	0.00339405594718373\\
15877.9147845196	0.00332356361703518\\
15883.7748748627	0.00321197561753751\\
15889.6349652057	0.0030619440901945\\
15895.4950555488	0.00287702713371492\\
15901.3551458918	0.00266160442981069\\
15907.2152362349	0.00242077352038266\\
15913.0753265779	0.00216022919334625\\
15918.935416921	0.00188612883015765\\
15924.7955072641	0.0016049468990927\\
15930.6555976071	0.00132332203250974\\
15936.5156879502	0.00104790029991056\\
15942.3757782932	0.000785178376656384\\
15948.2358686363	0.000541350309089475\\
15954.0959589794	0.000322161490170081\\
15959.9560493224	0.000132773287949975\\
15965.8161396655	-2.23584835662776e-05\\
15971.6762300085	-0.000139588389904397\\
15977.5363203516	-0.000216167031927556\\
15983.3964106947	-0.000250305519499976\\
15989.2565010377	-0.000241217251211042\\
15995.1165913808	-0.000189136024533497\\
16000.9766817238	-9.53100597599009e-05\\
16006.8367720669	3.80279104543498e-05\\
16012.69686241	0.000207713769378205\\
16018.556952753	0.000409727098285845\\
16024.4170430961	0.000639286430811113\\
16030.2771334391	0.000890962422844709\\
16036.1372237822	0.00115880625959995\\
16041.9973141253	0.00143649026307191\\
16047.8574044683	0.00171745737655641\\
16053.7174948114	0.00199507599508508\\
16059.5775851544	0.00226279648579603\\
16065.4376754975	0.00251430570430218\\
16071.2977658406	0.00274367586196693\\
16077.1578561836	0.00294550423353621\\
16083.0179465267	0.00311504041315398\\
16088.8780368697	0.00324829812088709\\
16094.7381272128	0.00334214892908698\\
16100.5982175559	0.00339439570411013\\
16106.4583078989	0.0034038240405048\\
16112.318398242	0.00337023048262889\\
16118.178488585	0.00329442687901623\\
16124.0385789281	0.0031782207765661\\
16129.8986692712	0.00302437232816796\\
16135.7587596142	0.00283652874096806\\
16141.6188499573	0.00261913782143336\\
16147.4789403003	0.00237734266739566\\
16153.3390306434	0.00211685999924417\\
16159.1991209864	0.00184384500913134\\
16165.0592113295	0.00156474592294067\\
16170.9193016726	0.00128615171126053\\
16176.7793920156	0.0010146365460701\\
16182.6394823587	0.000756604674649038\\
16188.4995727017	0.000518139371084741\\
16194.3596630448	0.000304859527714513\\
16200.2197533879	0.000121787267190864\\
16206.0798437309	-2.67703058012217e-05\\
16211.939934074	-0.000137322429085251\\
16217.800024417	-0.000207276700249463\\
16223.6601147601	-0.000234999859707407\\
16229.5202051032	-0.000219855914514827\\
16235.3802954462	-0.000162220713228812\\
16241.2403857893	-6.34726380712965e-05\\
16247.1004761323	7.40403555731246e-05\\
16252.9605664754	0.000247056564347153\\
16258.8206568185	0.000451477899609607\\
16264.6807471615	0.000682466952817851\\
16270.5408375046	0.000934561464678312\\
16276.4009278476	0.00120180348947253\\
16282.2610181907	0.00147788020042007\\
16288.1211085338	0.00175627300844687\\
16293.9811988768	0.0020304114721701\\
16299.8412892199	0.00229382836536761\\
16305.7013795629	0.00254031224206561\\
16311.561469906	0.0027640539006836\\
16317.4215602491	0.00295978329357303\\
16323.2816505921	0.00312289365547601\\
16329.1417409352	0.00324954992734402\\
16335.0018312782	0.00333677892398197\\
16340.8619216213	0.00338253912540536\\
16346.7220119643	0.00338576845392555\\
16352.5821023074	0.0033464089189049\\
16358.4421926505	0.00326540755775076\\
16364.3022829935	0.00314469366049456\\
16370.1623733366	0.00298713282620357\\
16376.0224636796	0.00279645894443526\\
16381.8825540227	0.00257718571678646\\
16387.7426443658	0.00233449981442621\\
16393.6027347088	0.00207413820114466\\
16399.4628250519	0.00180225252264397\\
16405.3229153949	0.0015252637683092\\
16411.183005738	0.00124971063835898\\
16417.0430960811	0.000982095197569767\\
16422.9031864241	0.000728729458506226\\
16428.7632767672	0.000495586514094176\\
16434.6233671102	0.000288159730377807\\
16440.4834574533	0.000111333318655286\\
16446.3435477964	-3.07326637172032e-05\\
16452.2036381394	-0.000134700176321897\\
16458.0637284825	-0.000198131620406222\\
16463.9238188255	-0.000219546982932227\\
16469.7839091686	-0.000198458360456623\\
16475.6439995117	-0.000135381057646192\\
16481.5040898547	-3.18210083328037e-05\\
16487.3641801978	0.000109761174163993\\
16493.2242705408	0.000286008659417088\\
16499.0843608839	0.000492747986061979\\
16504.944451227	0.000725087884969616\\
16510.80454157	0.000977534999517035\\
16516.6646319131	0.0012441237653532\\
16522.5247222561	0.0015185573789269\\
16528.3848125992	0.00179435652390311\\
16534.2449029423	0.00206501234264672\\
16540.1049932853	0.00232414004163137\\
16545.9650836284	0.00256562950545911\\
16551.8251739714	0.00278378936732268\\
16557.6852643145	0.00297348113850494\\
16563.5453546575	0.00313024023616807\\
16569.4054450006	0.00325038105919412\\
16575.2655353437	0.00333108363872912\\
16581.1256256867	0.00337045982670993\\
16586.9857160298	0.00336759746885296\\
16592.8458063728	0.00332258152877746\\
16598.7058967159	0.00323649167421322\\
16604.565987059	0.00311137639197449\\
16610.426077402	0.00295020425211911\\
16616.2861677451	0.00275679348066917\\
16622.1462580881	0.00253572151148781\\
16628.0063484312	0.00229221665998899\\
16633.8664387743	0.00203203448196694\\
16639.7265291173	0.0017613217414424\\
16645.5866194604	0.00148647120241221\\
16651.4467098034	0.00121397067475159\\
16657.3068001465	0.000950249878674834\\
16663.1668904896	0.000701528742090925\\
16669.0269808326	0.000473670710044766\\
16674.8870711757	0.000272044525574439\\
16680.7471615187	0.000101397740207191\\
16686.6072518618	-3.42550660060854e-05\\
16692.4673422049	-0.000131726723126663\\
16698.3274325479	-0.000188732347219367\\
16704.187522891	-0.00020394289452982\\
16710.047613234	-0.000177016138859087\\
16715.9077035771	-0.000108604350224984\\
16721.7677939202	-3.38503015062493e-07\\
16727.6278842632	0.000145210604301558\\
16733.4879746063	0.000324593391174183\\
16739.3480649493	0.00053356324939121\\
16745.2081552924	0.000767177072449169\\
16751.0682456355	0.00101991218018234\\
16756.9283359785	0.00128579687304424\\
16762.7884263216	0.00155855152981105\\
16768.6485166646	0.00183173691520315\\
16774.5086070077	0.00209890619477049\\
16780.3686973508	0.00235375706848653\\
16786.2287876938	0.00259028043329555\\
16792.0888780369	0.00280290206770192\\
16797.9489683799	0.00298661399816999\\
16803.809058723	0.00313709245131778\\
16809.669149066	0.00325079961408492\\
16815.5292394091	0.00332506680697236\\
16821.3893297522	0.00335815711499183\\
16827.2494200952	0.00334930600689151\\
16833.1095104383	0.00329873899231712\\
16838.9696007813	0.00320766590989548\\
16844.8296911244	0.00307825198979473\\
16850.6897814675	0.00291356638295905\\
16856.5498718105	0.0027175093805363\\
16862.4099621536	0.00249472004890933\\
16868.2700524966	0.00225046646811791\\
16874.1301428397	0.00199052117006053\\
16879.9902331828	0.00172102472175951\\
16885.8503235258	0.0014483406774797\\
16891.7104138689	0.0011789053255574\\
16897.5705042119	0.000919075777729024\\
16903.430594555	0.000674979986037294\\
16909.2906848981	0.000452372225917957\\
16915.1507752411	0.000256497453378197\\
16921.0108655842	9.19677336562851e-05\\
16926.8709559272	-3.73463474530113e-05\\
16932.7310462703	-0.000128406733021716\\
16938.5911366134	-0.000179079263506846\\
16944.4512269564	-0.000188183687877569\\
16950.3113172995	-0.000155521145550106\\
16956.1714076425	-8.18784793205741e-05\\
16962.0314979856	3.09907134717422e-05\\
16967.8915883287	0.000180407835448466\\
16973.7516786717	0.000362832855259908\\
16979.6117690148	0.000573948176819946\\
16985.4718593578	0.000808760824820732\\
16991.3319497009	0.0010617205281706\\
16997.1920400439	0.00132685090883612\\
17003.052130387	0.00159789067534705\\
17008.9122207301	0.00186844148536767\\
17014.7723110731	0.0021321189855739\\
17020.6324014162	0.00238270346375212\\
17026.4924917592	0.00261428655787328\\
17032.3525821023	0.00282141056221078\\
17038.2126724454	0.00299919704571256\\
17044.0727627884	0.00314346175192286\\
17049.9328531315	0.00325081307430903\\
17055.7929434745	0.00331873178943838\\
17061.6530338176	0.00334563017360636\\
17067.5131241607	0.0033308891162053\\
17073.3732145037	0.0032748723616779\\
17079.2333048468	0.00317891755331364\\
17085.0933951898	0.00304530429881759\\
17090.9534855329	0.00287720001974668\\
17096.813575876	0.00267858487124884\\
17102.673666219	0.00245415751154037\\
17108.5337565621	0.00220922395216201\\
17114.3938469051	0.00194957211792032\\
17120.2539372482	0.00168133508208871\\
17126.1140275913	0.00141084620853153\\
17131.9741179343	0.00114448962183809\\
17137.8342082774	0.000888549535702041\\
17143.6942986204	0.000649061995683265\\
17149.5543889635	0.000431672533579973\\
17155.4144793066	0.000241503090366643\\
17161.2745696496	8.30313444355445e-05\\
17167.1346599927	-4.00147120079053e-05\\
17172.9947503357	-0.000124744466343105\\
17178.8548406788	-0.000169172586033585\\
17184.7149310219	-0.00017226553134028\\
17190.5750213649	-0.000133965590637644\\
17196.435111708	-5.51918795326886e-05\\
17202.295202051	6.21817112289388e-05\\
17208.1552923941	0.000215371090139906\\
17214.0153827372	0.000400748001395787\\
17219.8754730802	0.000613925957206676\\
17225.7355634233	0.000849864030840129\\
17231.5956537663	0.00110298605450711\\
17237.4557441094	0.00136731240367987\\
17243.3158344524	0.00163660125346903\\
17249.1759247955	0.00190449597017936\\
17255.0360151386	0.00216467515785807\\
17260.8961054816	0.00241100181814975\\
17266.7561958247	0.00263766810384271\\
17272.6162861677	0.00283933225223779\\
17278.4763765108	0.00301124446952437\\
17284.3364668539	0.00314935880011472\\
17290.1965571969	0.00325042834598547\\
17296.05664754	0.00331208159513595\\
17301.916737883	0.0033328780652403\\
17307.7768282261	0.00331234195655322\\
17313.6369185692	0.00325097302838986\\
17319.4970089122	0.00315023445060586\\
17325.3570992553	0.00301251792512637\\
17331.2171895983	0.0028410869089278\\
17337.0772799414	0.00263999928608769\\
17342.9373702845	0.00241401132129086\\
17348.7974606275	0.00216846516790967\\
17354.6575509706	0.00190916259096835\\
17360.5176413136	0.00164222789021971\\
17366.3777316567	0.00137396326148658\\
17372.2378219998	0.00111070001227798\\
17378.0979123428	0.000858649144152956\\
17383.9580026859	0.000623754827763278\\
17389.8180930289	0.000411554227368581\\
17395.678183372	0.000227046980089\\
17401.5382737151	7.45774065927705e-05\\
17407.3983640581	-4.22677719732014e-05\\
17413.2584544012	-0.00012074380267877\\
17419.1185447442	-0.000159012370673364\\
17424.9786350873	-0.000156184656030153\\
17430.8387254303	-0.00011234196916861\\
17436.6988157734	-2.85334859471857e-05\\
17442.5589061165	9.32488559845947e-05\\
17448.4189964595	0.000250117699026392\\
17454.2790868026	0.00043835872060944\\
17460.1391771456	0.000653518578438651\\
17465.9992674887	0.000890510263788304\\
17471.8593578318	0.00114373337056256\\
17477.7194481748	0.00140720643686414\\
17483.5795385179	0.00167470823173893\\
17489.4396288609	0.0019399246500727\\
17495.299719204	0.0021965977470681\\
17501.1598095471	0.00243867339507389\\
17507.0198998901	0.0026604440784261\\
17512.8799902332	0.00285668345889799\\
17518.7400805762	0.00302276953889241\\
17524.6001709193	0.00315479352014854\\
17530.4602612624	0.00324965179432829\\
17536.3203516054	0.00330511890044406\\
17542.1804419485	0.00331989973423141\\
17548.0405322915	0.00329365978488761\\
17553.9006226346	0.00322703269307582\\
17559.7607129777	0.00312160496001904\\
17565.6208033207	0.00297987817644177\\
17571.4808936638	0.0028052096702421\\
17577.3409840068	0.00260173298152088\\
17583.2010743499	0.00237426004794859\\
17589.061164693	0.00212816741562344\\
17594.921255036	0.00186926916557996\\
17600.7813453791	0.0016036795596945\\
17606.6414357221	0.00133766865039903\\
17612.5015260652	0.00107751426396015\\
17618.3616164083	0.000829353851739215\\
17624.2217067513	0.00059903970494476\\
17630.0817970944	0.000392000948702619\\
17635.9418874374	0.000213115569519962\\
17641.8019777805	6.65954925906442e-05\\
17647.6620681235	-4.41125836659883e-05\\
17653.5221584666	-0.000116408261157179\\
17659.3822488097	-0.000148598516845465\\
17665.2423391527	-0.000139937344294651\\
17671.1024294958	-9.06430339274512e-05\\
17676.9625198388	-1.89269188114981e-06\\
17682.8226101819	0.000124205865406549\\
17688.682700525	0.000284664169936385\\
17694.542790868	0.000475683925448585\\
17700.4028812111	0.000692746916920503\\
17706.2629715541	0.00093072187814373\\
17712.1230618972	0.00118398578974072\\
17717.9831522403	0.00144655674036633\\
17723.8432425833	0.00171223521203553\\
17729.7033329264	0.00197475045254338\\
17735.5634232694	0.00222790847940802\\
17741.4235136125	0.00246573822198747\\
17747.2836039556	0.00268263235409453\\
17753.1436942986	0.00287347949522077\\
17759.0037846417	0.00303378466396183\\
17764.8638749847	0.0031597751448728\\
17770.7239653278	0.0032484892758537\\
17776.5840556709	0.00329784606602108\\
17782.4441460139	0.00330669400845668\\
17788.304236357	0.00327483794213662\\
17794.1643267	0.00320304333721315\\
17800.0244170431	0.00309301790968213\\
17805.8845073862	0.0029473710069704\\
17811.7445977292	0.00276955173020013\\
17817.6046880723	0.00256376726082426\\
17823.4647784153	0.00233488332481833\\
17829.3248687584	0.00208830914989589\\
17835.1849591015	0.0018298696352209\\
17841.0450494445	0.00156566775460915\\
17846.9051397876	0.00130194044328263\\
17852.7652301306	0.00104491137126863\\
17858.6253204737	0.000800644078546268\\
17864.4854108167	0.000574898937514028\\
17870.3455011598	0.000372997317038402\\
17876.2055915029	0.000199696150929326\\
17882.0656818459	5.90758674657438e-05\\
17887.925772189	-4.55556798860067e-05\\
17893.785862532	-0.000111741018777535\\
17899.6459528751	-0.000137930771273336\\
17905.5060432182	-0.000123519918815147\\
17911.3661335612	-6.88617701286397e-05\\
17917.2262239043	2.47406902264884e-05\\
17923.0863142473	0.000155065861107603\\
17928.9464045904	0.000319026251571868\\
17934.8064949335	0.000512741623865245\\
17940.6665852765	0.000731630819955942\\
17946.5266756196	0.000970520098580096\\
17952.3867659626	0.00122376542092549\\
17958.2468563057	0.00148538579471838\\
17964.1069466488	0.00174920452650565\\
17969.9670369918	0.00200899504622712\\
17975.8271273349	0.00225862786163019\\
17981.6872176779	0.0024922151740684\\
17987.547308021	0.00270424974416834\\
17993.4073983641	0.00288973473190572\\
17999.2674887071	0.00304430145036805\\
18005.1275790502	0.00316431225816866\\
18010.9876693932	0.00324694616736504\\
18016.8477597363	0.00329026515224825\\
18022.7078500794	0.00329325960037216\\
18028.5679404224	0.00325587184018219\\
18034.4280307655	0.00317899719677166\\
18040.2881211085	0.00306446255839037\\
18046.1482114516	0.00291498296643236\\
18052.0083017947	0.00273409726046745\\
18057.8683921377	0.00252608430355498\\
18063.7284824808	0.00229586177103383\\
18069.5885728238	0.00204886989662629\\
18075.4486631669	0.00179094292414862\\
18081.3087535099	0.00152817130194673\\
18087.168843853	0.00126675787536547\\
18093.0289341961	0.00101287147215145\\
18098.8890245391	0.000772501337282419\\
18104.7491148822	0.000551315851514117\\
18110.6092052252	0.000354528866579199\\
18116.4692955683	0.000186776808663174\\
18122.3293859114	5.20094469453597e-05\\
18128.1894762544	-4.6603099474696e-05\\
18134.0495665975	-0.000106744926945176\\
18139.9096569405	-0.000127008731112874\\
18145.7697472836	-0.000106928732249591\\
18151.6298376267	-4.69913718307363e-05\\
18157.4899279697	5.13764650934753e-05\\
18163.3500183128	0.00018584141676491\\
18169.2101086558	0.000353218992327754\\
18175.0701989989	0.000549548987395443\\
18180.930289342	0.000770189181675641\\
18186.790379685	0.00100992510187001\\
18192.6504700281	0.00126309325460047\\
18198.5105603711	0.00152371491722304\\
18204.3706507142	0.00178563732594442\\
18210.2307410573	0.00204267892724433\\
18216.0908314003	0.00228877526361677\\
18221.9509217434	0.00251812205101473\\
18227.8110120864	0.00272531207219984\\
18233.6711024295	0.00290546265778688\\
18239.5311927726	0.00305433074932621\\
18245.3912831156	0.00316841283337592\\
18251.2513734587	0.00324502739241868\\
18257.1114638017	0.00328237793273344\\
18262.9715541448	0.00327959510792965\\
18268.8316444879	0.00323675694993573\\
18274.6917348309	0.00315488673757002\\
18280.551825174	0.00303592855912138\\
18286.411915517	0.00288270115277156\\
18292.2720058601	0.00269883112095715\\
18298.1320962031	0.00248866710038391\\
18303.9921865462	0.00225717691971971\\
18309.8522768893	0.00200983017628503\\
18315.7123672323	0.00175246900762923\\
18321.5724575754	0.00149117011098425\\
18327.4325479184	0.00123210126927024\\
18333.2926382615	0.000981375771129694\\
18339.1527286046	0.000744908160966237\\
18345.0128189476	0.000528274722729892\\
18350.8729092907	0.000336581988190763\\
18356.7329996337	0.000174346370379179\\
18362.5930899768	4.53877591612407e-05\\
18368.4531803199	-4.72604142253817e-05\\
18374.3132706629	-0.000101422526372739\\
18380.173361006	-0.000115831846490102\\
18386.033451349	-9.01601573534626e-05\\
18391.8935416921	-2.50252198610244e-05\\
18397.7536320352	7.80240883375843e-05\\
18403.6137223782	0.000216544602743797\\
18409.4738127213	0.000387256794770521\\
18415.3339030643	0.000586122414200106\\
18421.1939934074	0.000808440013231233\\
18427.0540837505	0.00104895609258157\\
18432.9141740935	0.00130198924221325\\
18438.7742644366	0.00156156434323884\\
18444.6343547796	0.00182155366113274\\
18450.4944451227	0.00207582149875784\\
18456.3545354658	0.00231836899430911\\
18462.2146258088	0.00254347564764448\\
18468.0747161519	0.00274583423561655\\
18473.9348064949	0.00292067593503878\\
18479.794896838	0.00306388270339354\\
18485.654987181	0.00317208426875\\
18491.5150775241	0.00324273744517434\\
18497.3751678672	0.00327418590659551\\
18503.2352582102	0.00326569901448781\\
18509.0953485533	0.00321748878943333\\
18514.9554388963	0.00313070463210016\\
18520.8155292394	0.00300740592507665\\
18526.6756195825	0.00285051316814547\\
18532.5357099255	0.00266373880694001\\
18538.3958002686	0.00245149939270799\\
18544.2558906116	0.00221881115160648\\
18550.1159809547	0.0019711714330792\\
18555.9760712978	0.00171442883853689\\
18561.8361616408	0.00145464509873329\\
18567.6962519839	0.00119795196145568\\
18573.5563423269	0.000950406468306214\\
18579.41643267	0.000717848036250807\\
18585.2765230131	0.000505760715971023\\
18591.1366133561	0.000319143876072086\\
18596.9967036992	0.000162394362353604\\
18602.8567940422	3.92029096404471e-05\\
18608.7168843853	-4.75327533708477e-05\\
18614.5769747284	-9.57760604676606e-05\\
18620.4370650714	-0.000104399422478881\\
18626.2971554145	-7.32105775005867e-05\\
18632.1572457575	-2.956861103588e-06\\
18638.0173361006	0.000104692697937652\\
18643.8774264437	0.000247187027568191\\
18649.7375167867	0.000421153466158602\\
18655.5976071298	0.000622477587468703\\
18661.4576974728	0.000846400507645332\\
18667.3177878159	0.00108763137293946\\
18673.177878159	0.00134047236945581\\
18679.037968502	0.00159895330114962\\
18684.8980588451	0.00185697255779655\\
18690.7581491881	0.00210844114424735\\
18696.6182395312	0.002347426371543\\
18702.4783298743	0.00256829181876632\\
18708.3384202173	0.00276583026410804\\
18714.1985105604	0.00293538644997722\\
18720.0586009034	0.00307296678813353\\
18725.9186912465	0.00317533341923403\\
18731.7787815895	0.0032400804121195\\
18737.6388719326	0.00326569030914807\\
18743.4989622757	0.0032515696888833\\
18749.3590526187	0.00319806291285394\\
18755.2191429618	0.00310644373782046\\
18761.0792333048	0.00297888499786085\\
18766.9393236479	0.00281840707790009\\
18772.799413991	0.0026288064000839\\
18778.659504334	0.00241456561663699\\
18784.5195946771	0.00218074763341678\\
18790.3796850201	0.00193287596938492\\
18796.2397753632	0.0016768042792222\\
18802.0998657063	0.00141857812115274\\
18807.9599560493	0.00116429223417216\\
18813.8200463924	0.000919946693924104\\
18819.6801367354	0.000691305342053608\\
18825.5402270785	0.000483759829222655\\
18831.4003174216	0.000302202478736994\\
18837.2604077646	0.000150910968479387\\
18843.1204981077	3.3447549295294e-05\\
18848.9805884507	-4.74248258504103e-05\\
18854.8406787938	-8.98074873347542e-05\\
18860.7007691369	-9.27106205544781e-05\\
18866.5608594799	-5.60763775477899e-05\\
18872.420949823	1.92200110015848e-05\\
18878.281040166	0.00013139114364504\\
18884.1411305091	0.00027777987656005\\
18890.0012208522	0.000454922265381188\\
18895.8613111952	0.000658629529578646\\
18901.7214015383	0.00088408709999427\\
18907.5814918813	0.00112596840758451\\
18913.4415822244	0.00137856072409014\\
18919.3016725674	0.00163590008177391\\
18925.1617629105	0.00189191208585699\\
18931.0218532536	0.00214055529512317\\
18936.8819435966	0.0023759637864148\\
18942.7420339397	0.00259258553886001\\
18948.6021242827	0.00278531337337056\\
18954.4622146258	0.00294960535948395\\
18960.3223049689	0.00308159185063536\\
18966.1823953119	0.00317816662590684\\
18972.042485655	0.00323705999149077\\
18977.902575998	0.00325689212143291\\
18983.7626663411	0.00323720538448551\\
18989.6227566842	0.00317847489959295\\
18995.4828470272	0.00308209707643156\\
19001.3429373703	0.00295035641776533\\
19007.2030277133	0.00278637137247749\\
19013.0631180564	0.0025940205224798\\
19018.9232083995	0.00237785085088631\\
19024.7832987425	0.00214297026078808\\
19030.6433890856	0.00189492688498233\\
19036.5034794286	0.00163957803855868\\
19042.3635697717	0.00138295190946436\\
19048.2236601148	0.00113110525273996\\
19054.0837504578	0.000889980447925988\\
19059.9438408009	0.000665265292906495\\
19065.8039311439	0.00046225884215683\\
19071.664021487	0.000285746453934804\\
19077.5241118301	0.000139886992642582\\
19083.3842021731	2.81148451499311e-05\\
19089.2442925162	-4.69409405442995e-05\\
19095.1043828592	-8.35184904910721e-05\\
19100.9644732023	-8.07644595356537e-05\\
19106.8245635454	-3.87539349948621e-05\\
19112.6846538884	4.15115748822309e-05\\
19118.5447442315	0.000158128014557549\\
19124.4048345745	0.000308333947919791\\
19130.2649249176	0.000488575946729301\\
19136.1250152607	0.000694592652493112\\
19141.9851056037	0.00092151552327798\\
19147.8451959468	0.00116398388359977\\
19153.7052862898	0.0014162715587859\\
19159.5653766329	0.00167242210244262\\
19165.4254669759	0.0019263894235084\\
19171.285557319	0.00217218049308533\\
19177.1456476621	0.00240399676273911\\
19183.0057380051	0.00261637095717588\\
19188.8658283482	0.00280429601460806\\
19194.7259186912	0.00296334313373728\\
19200.5860090343	0.00308976614442579\\
19206.4460993774	0.00318058974261593\\
19212.3061897204	0.00323367951106862\\
19218.1662800635	0.00324779207856016\\
19224.0263704065	0.00322260423823502\\
19229.8864607496	0.00315872034386454\\
19235.7465510927	0.00305765781455272\\
19241.6066414357	0.00292181109561528\\
19247.4667317788	0.00275439493145204\\
19253.3268221218	0.00255936829414784\\
19259.1869124649	0.00234134076831363\\
19265.047002808	0.00210546360503902\\
19270.907093151	0.00185730802041086\\
19276.7671834941	0.00160273361346353\\
19282.6272738371	0.00134775001116313\\
19288.4873641802	0.0010983750073472\\
19294.3474545233	0.00086049254402621\\
19300.2075448663	0.000639713886660138\\
19306.0676352094	0.000441245268687923\\
19311.9277255524	0.000269765127162093\\
19317.7878158955	0.000129313824193651\\
19323.6479062386	2.3198453580295e-05\\
19329.5079965816	-4.60850246296429e-05\\
19335.3680869247	-7.69104883892955e-05\\
19341.2281772677	-6.85598160528999e-05\\
19347.0882676108	-2.12396113792094e-05\\
19352.9483579539	6.39238983051875e-05\\
19358.8084482969	0.000184911665046173\\
19364.66853864	0.000338859686494413\\
19370.528628983	0.000522126800690162\\
19376.3887193261	0.000730380804708862\\
19382.2488096691	0.000958700860340368\\
19388.1089000122	0.00120169376626511\\
19393.9689903553	0.00145362134940467\\
19399.8290806983	0.0017085359665373\\
19405.6891710414	0.00196042091644812\\
19411.5492613844	0.00220333244790632\\
19417.4093517275	0.0024315400118711\\
19423.2694420706	0.002639661448472\\
19429.1295324136	0.00282278991991434\\
19434.9896227567	0.00297660959547147\\
19440.8497130997	0.00309749736159997\\
19446.7098034428	0.00318260816004539\\
19452.5698937859	0.00322994194408079\\
19458.4299841289	0.00323839067706191\\
19464.290074472	0.00320776426904855\\
19470.150164815	0.00313879484447176\\
19476.0102551581	0.00303311924507743\\
19481.8703455012	0.00289324018648735\\
19487.7304358442	0.00272246699002114\\
19493.5905261873	0.00252483729308094\\
19499.4506165303	0.00230502159057735\\
19505.3107068734	0.00206821286369505\\
19511.1707972165	0.00182000390460429\\
19517.0308875595	0.00156625523458023\\
19522.8909779026	0.00131295673521743\\
19528.7510682456	0.00106608625917946\\
19534.6111585887	0.000831468558037372\\
19540.4712489318	0.000614637856179421\\
19546.3313392748	0.000420707313197848\\
19552.1914296179	0.00025424845348466\\
19558.0515199609	0.000119183406251231\\
19563.911610304	1.869249586499e-05\\
19569.7717006471	-4.48606402078331e-05\\
19575.6317909901	-6.99846428289845e-05\\
19581.4918813332	-5.60954245490624e-05\\
19587.3519716762	-3.5297438650051e-06\\
19593.2120620193	8.64629550001872e-05\\
19599.0721523623	0.000211750239203305\\
19604.9322427054	0.00036936721547464\\
19610.7923330485	0.000555586692197041\\
19616.6524233915	0.000766007315066848\\
19622.5125137346	0.000995657592319269\\
19628.3726040776	0.00123911335097204\\
19634.2326944207	0.0014906258491366\\
19640.0927847638	0.00174425751862979\\
19645.9528751068	0.00199402213284423\\
19651.8129654499	0.00223402609081966\\
19657.6730557929	0.00245860748341024\\
19663.533146136	0.00266246965982387\\
19669.3932364791	0.00284080614424192\\
19675.2533268221	0.00298941395599161\\
19681.1134171652	0.0031047926621266\\
19686.9735075082	0.00318422682797309\\
19692.8335978513	0.00322584992333993\\
19698.6936881944	0.00322868818089843\\
19704.5537785374	0.0031926833759882\\
19710.4138688805	0.00311869399495326\\
19716.2739592235	0.00300847476949458\\
19722.1340495666	0.00286463506467592\\
19727.9941399097	0.00269057710737819\\
19733.8542302527	0.00249041551766442\\
19739.7143205958	0.00226888004578474\\
19745.5744109388	0.00203120381427884\\
19751.4345012819	0.00178299970573255\\
19757.294591625	0.00153012781571654\\
19763.154681968	0.0012785571011974\\
19769.0147723111	0.00103422449039854\\
19774.8748626541	0.000802894779875644\\
19780.7349529972	0.000590024624599525\\
19786.5950433403	0.000400633830089154\\
19792.4551336833	0.000239186982342586\\
19798.3152240264	0.000109488206609079\\
19804.1753143694	1.45915358610645e-05\\
19810.0354047125	-4.32709993287928e-05\\
19815.8954950555	-6.27418663285009e-05\\
19821.7555853986	-4.33698768218519e-05\\
19827.6156757417	1.43793630239751e-05\\
19833.4757660847	0.000109134640544155\\
19839.3358564278	0.000238651693966943\\
19845.1959467708	0.000399866366243662\\
19851.0560371139	0.000588967096438991\\
19856.916127457	0.000801485033775356\\
19862.7762178	0.00103239964388067\\
19868.6363081431	0.00127625731164266\\
19874.4963984861	0.00152730013897931\\
19880.3564888292	0.00177960189582683\\
19886.2165791723	0.00202720791431624\\
19892.0766695153	0.00226427562413767\\
19897.9367598584	0.00248521241217128\\
19903.7968502014	0.00268480755391253\\
19909.6569405445	0.00285835510397872\\
19915.5170308876	0.00300176484823732\\
19921.3771212306	0.00311165870084367\\
19927.2372115737	0.00318545027524024\\
19933.0973019167	0.00322140575420832\\
19938.9573922598	0.00321868462698279\\
19944.8174826029	0.00317735933591647\\
19950.6775729459	0.00309841337376431\\
19956.537663289	0.00298371788102505\\
19962.397753632	0.00283598730003151\\
19968.2578439751	0.00265871513697177\\
19974.1179343182	0.00245609135167552\\
19979.9780246612	0.00223290332879933\\
19985.8381150043	0.00199442277095076\\
19991.6982053473	0.00174628118556276\\
19997.5582956904	0.0014943369067839\\
20003.4183860334	0.00124453679189974\\
20009.2784763765	0.00100277585766498\\
20015.1385667196	0.000774758169158772\\
20020.9986570626	0.000565862263930092\\
20026.8587474057	0.000381014286463652\\
20032.7188377487	0.000224571825151499\\
20038.5789280918	0.000100221191034979\\
20044.4390184349	1.08905596460874e-05\\
20050.2991087779	-4.1318977528854e-05\\
20056.159199121	-5.51828285233295e-05\\
20062.019289464	-3.03816211365579e-05\\
20067.8793798071	3.24914455642226e-05\\
20073.7394701502	0.000131944787599777\\
20079.5995604932	0.000265623821086051\\
20085.4596508363	0.000430366706511684\\
20091.3197411793	0.000622279132573751\\
20097.1798315224	0.000836826370871963\\
20103.0399218655	0.00106894042550905\\
20108.9000122085	0.00131313974585914\\
20114.7601025516	0.00156365867478947\\
20120.6201928946	0.00181458357556377\\
20126.4802832377	0.00205999242345\\
20132.3403735808	0.00229409456716184\\
20138.2004639238	0.00251136736170224\\
20144.0605542669	0.0027066864489546\\
20149.9206446099	0.00287544661260889\\
20155.780734953	0.00301367035727035\\
20161.6408252961	0.00311810165184447\\
20167.5009156391	0.00318628262782528\\
20173.3610059822	0.00321661142590622\\
20179.2210963252	0.00320837982946083\\
20185.0811866683	0.00316178980087591\\
20190.9412770114	0.00307794853466199\\
20196.8013673544	0.00295884214821608\\
20202.6614576975	0.00280728863553673\\
20208.5215480405	0.00262687119822895\\
20214.3816383836	0.00242185353095053\\
20220.2417287267	0.00219707906397015\\
20226.1018190697	0.0019578565441099\\
20231.9619094128	0.00170983465669336\\
20237.8219997558	0.00145886864972079\\
20243.6820900989	0.00121088210922227\\
20249.5421804419	0.000971727148917599\\
20255.402270785	0.000747046313836898\\
20261.2623611281	0.000542139456705485\\
20267.1224514711	0.000361838727574037\\
20272.9825418142	0.000210394625440384\\
20278.8426321572	9.13757987774044e-05\\
20284.7027225003	7.58495697702066e-06\\
20290.5628128434	-3.90071259829308e-05\\
20296.4229031864	-4.73079616430469e-05\\
20302.2829935295	-1.71289608970521e-05\\
20308.1430838725	5.0810288928693e-05\\
20314.0031742156	0.000154899180587666\\
20319.8632645587	0.000292674268021953\\
20325.7233549017	0.000460877566986869\\
20331.5834452448	0.000655533595495984\\
20337.4435355878	0.000872043332353873\\
20343.3036259309	0.00110529287324029\\
20349.163716274	0.00134977421724108\\
20355.023806617	0.00159971533113736\\
20360.8838969601	0.00184921642009313\\
20366.7439873031	0.00209238918780621\\
20372.6040776462	0.00232349579893282\\
20378.4641679893	0.00253708426446659\\
20384.3242583323	0.00272811705584226\\
20390.1843486754	0.00289208991347358\\
20396.0444390184	0.00302513804812941\\
20401.9045293615	0.00312412723097415\\
20407.7646197046	0.00318672762535123\\
20413.6247100476	0.00321146862150076\\
20419.4848003907	0.00319777338318522\\
20425.3448907337	0.0031459722950326\\
20431.2049810768	0.00305729499728184\\
20437.0650714198	0.00293384119935575\\
20442.9251617629	0.00277853096561077\\
20448.785252106	0.00259503564985561\\
20454.645342449	0.00238769111223189\\
20460.5054327921	0.00216139526995276\\
20466.3655231351	0.0019214924022144\\
20472.2256134782	0.00167364694233901\\
20478.0857038213	0.00142370973740285\\
20483.9457941643	0.00117757993293203\\
20489.8058845074	0.000941065743037567\\
20495.6659748504	0.000719747391844812\\
20501.5260651935	0.000518845460369104\\
20507.3861555366	0.000343097744896139\\
20513.2462458796	0.000196647531327036\\
20519.1063362227	8.2945920120351e-05\\
20524.9664265657	4.67050443822302e-06\\
20530.8265169088	-3.63376823612654e-05\\
20536.6866072519	-3.91174651187153e-05\\
20542.5466975949	-3.61005289553254e-06\\
20548.406787938	6.93397352561368e-05\\
20554.266878281	0.000178003569884647\\
20560.1269686241	0.000319810557931723\\
20565.9870589672	0.000491408066685149\\
20571.8471493102	0.00068874098589843\\
20577.7072396533	0.000907147554284746\\
20583.5673299963	0.00114146948591178\\
20589.4274203394	0.0013861737950614\\
20595.2875106825	0.00163548344248802\\
20601.1476010255	0.00188351371818666\\
20607.0076913686	0.00212441114117562\\
20612.8677817116	0.00235249159792822\\
20618.7278720547	0.00256237445937183\\
20624.5879623978	0.0027491095121929\\
20630.4480527408	0.00290829371006961\\
20636.3081430839	0.00303617499155613\\
20642.1682334269	0.00312974071630488\\
20648.02832377	0.00318678863591547\\
20653.8884141131	0.003205978726969\\
20659.7485044561	0.00318686466657632\\
20665.6085947992	0.00312990421121617\\
20671.4686851422	0.00303644823754358\\
20677.3287754853	0.00290870870720147\\
20683.1888658283	0.00274970631572453\\
20689.0489561714	0.00256319906432044\\
20694.9090465145	0.00235359344330247\\
20700.7691368575	0.00212584032652471\\
20706.6292272006	0.00188531803599131\\
20712.4893175436	0.00163770533880346\\
20718.3494078867	0.00138884737503024\\
20724.2094982298	0.00114461768209432\\
20730.0695885728	0.000910779572327836\\
20735.9296789159	0.000692850135347828\\
20741.7897692589	0.000495970074259087\\
20747.649859602	0.000324782446600355\\
20753.5099499451	0.000183323170174404\\
20759.3700402881	7.49258758179725e-05\\
20765.2301306312	2.14335015914561e-06\\
20771.0902209742	-3.33125804714552e-05\\
20776.9503113173	-3.0611309356979e-05\\
20782.8104016604	1.01770948662542e-05\\
20788.6704920034	8.80836917304688e-05\\
20794.5305823465	0.000201263685613291\\
20800.3906726895	0.000347040108862511\\
20806.2507630326	0.000521967137027511\\
20812.1108533757	0.000721911538721629\\
20817.9709437187	0.000942150334932957\\
20823.8310340618	0.00117748236034001\\
20829.6911244048	0.0014223510915718\\
20835.5512147479	0.00167097584175276\\
20841.411305091	0.00191748822394495\\
20847.271395434	0.00215607066185803\\
20853.1314857771	0.00238109367907378\\
20858.9915761201	0.00258724872640782\\
20864.8516664632	0.00276967341411233\\
20870.7117568062	0.00292406619401881\\
20876.5718471493	0.00304678778741879\\
20882.4319374924	0.00313494696671604\\
20888.2920278354	0.00318646866930303\\
20894.1521181785	0.00320014283863114\\
20900.0122085215	0.00317565284344941\\
20905.8722988646	0.00311358280714593\\
20911.7323892077	0.00301540367833292\\
20917.5924795507	0.00288343837409154\\
20923.4525698938	0.00272080682239459\\
20929.3126602368	0.00253135220365635\\
20935.1727505799	0.00231955013498935\\
20941.032840923	0.00209040294326255\\
20946.892931266	0.00184932152459245\\
20952.7530216091	0.00160199757994134\\
20958.6131119521	0.00135426924396446\\
20964.4732022952	0.00111198327904755\\
20970.3332926383	0.000880857087363718\\
20976.1933829813	0.000666343797465135\\
20982.0534733244	0.000473503608921444\\
20987.9135636674	0.000306884430186792\\
20993.7736540105	0.000170414625237528\\
20999.6337443536	6.7310398293246e-05\\
nan	nan\\
21011.3539250397	-2.99334587582342e-05\\
21017.2140153827	-2.1789238725292e-05\\
21023.0741057258	2.4234625778998e-05\\
21028.9341960689	0.000107046138705955\\
21034.7942864119	0.000224685251098001\\
21040.654376755	0.000374370252199059\\
21046.514467098	0.000552563544909791\\
21052.3745574411	0.000755055250273588\\
21058.2346477842	0.000977062665328763\\
21064.0947381272	0.00121334322453697\\
21069.9548284703	0.00145831829705226\\
21075.8149188133	0.00170620489645645\\
21081.6750091564	0.00195115219337911\\
21087.5350994994	0.00218737960872977\\
21093.3951898425	0.00240931322825304\\
21099.2552801856	0.00261171731907342\\
21105.1153705286	0.00278981784564513\\
21110.9754608717	0.00293941507080181\\
21116.8355512147	0.00305698258643913\\
21122.6956415578	0.00313975043883375\\
21128.5557319009	0.00318577038886933\\
21134.4158222439	0.00319396176997816\\
21140.275912587	0.00316413686430596\\
21146.13600293	0.00309700520125194\\
21151.9960932731	0.00299415667996184\\
21157.8561836162	0.00285802391758802\\
21163.7162739592	0.00269182471422838\\
21169.5763643023	0.00249948599611892\\
21175.4364546453	0.00228555103410204\\
21181.2965449884	0.00205507212966843\\
21187.1566353315	0.00181349130351671\\
21193.0167256745	0.00156651180374478\\
21198.8768160176	0.00131996346766029\\
21204.7369063606	0.00107966511550895\\
21210.5969967037	0.000851287224124466\\
21216.4570870468	0.00064021812117586\\
21222.3171773898	0.000451436857593549\\
21228.1772677329	0.000289395757199251\\
21234.0373580759	0.000157915414167216\\
21239.897448419	6.00946144661507e-05\\
21245.7575387621	-1.76269488871369e-06\\
21251.6176291051	-2.62016677201703e-05\\
21257.4777194482	-1.26507737700991e-05\\
21263.3378097912	3.85648376552169e-05\\
21269.1979001343	0.000126231137913003\\
21275.0579904774	0.000248273996051639\\
21280.9180808204	0.000401808250545425\\
21286.7781711635	0.000583205914801907\\
21292.6382615065	0.000788181904068069\\
21298.4983518496	0.00101189525817956\\
21304.3584421926	0.00124906346904916\\
21310.2185325357	0.00149408721301988\\
21316.0786228788	0.00174118254285544\\
21321.9387132218	0.00198451741860817\\
21327.7988035649	0.00221834935470449\\
21333.6588939079	0.00243716093445656\\
21339.518984251	0.00263578999428017\\
21345.3790745941	0.00280955140584624\\
21351.2391649371	0.00295434758344741\\
21357.0992552802	0.00306676510985293\\
21362.9593456232	0.00314415520234353\\
21368.8194359663	0.00318469612222795\\
21374.6795263094	0.00318743605714714\\
21380.5396166524	0.00315231546681139\\
21386.3997069955	0.00308016836798435\\
21392.2597973385	0.00297270253069246\\
21398.1198876816	0.00283245905618398\\
21403.9799780247	0.00266275229325185\\
21409.8400683677	0.00246759151377172\\
21415.7001587108	0.00225158619786335\\
21421.5602490538	0.00201983716683753\\
21427.4203393969	0.0017778161343922\\
21433.28042974	0.00153123652077801\\
21439.140520083	0.00128591857954535\\
21445.0006104261	0.00104765202075956\\
21450.8607007691	0.000822059373266662\\
21456.7207911122	0.000614463310369162\\
21462.5808814553	0.000429761069651542\\
21468.4409717983	0.000272308929783336\\
21474.3010621414	0.000145819469240684\\
21480.1611524844	5.32740301088044e-05\\
21486.0212428275	-3.1475492157126e-06\\
21491.8813331706	-2.21182763003273e-05\\
21497.7414235136	-3.19521269947748e-06\\
21503.6015138567	5.31701861906142e-05\\
21509.4616041997	0.000145642840780946\\
21515.3216945428	0.000272035669565267\\
21521.1817848858	0.00042936131505069\\
21527.0418752289	0.000613902750050097\\
21532.901965572	0.000821301095556858\\
21538.762055915	0.00104665857552332\\
21544.6221462581	0.00128465417689415\\
21550.4822366011	0.00152966928354407\\
21556.3423269442	0.00177592031809374\\
21562.2024172873	0.00201759526007206\\
21568.0625076303	0.00224899081828803\\
21573.9225979734	0.00246464701992132\\
21579.7826883164	0.00265947604038275\\
21585.6427786595	0.00282888223403618\\
21591.5028690026	0.00296887053453593\\
21597.3629593456	0.00307614066740391\\
21603.2230496887	0.0031481649537589\\
21609.0831400317	0.00318324787040729\\
21614.9432303748	0.00318056596340672\\
21620.8033207179	0.00314018717540569\\
21626.6634110609	0.00306306913297706\\
21632.523501404	0.00295103643708406\\
21638.383591747	0.00280673749533747\\
21644.2436820901	0.00263358191700662\\
21650.1037724332	0.00243565995089277\\
21655.9638627762	0.00221764586926778\\
21661.8239531193	0.0019846875803692\\
21667.6840434623	0.0017422850759966\\
21673.5441338054	0.00149616058417645\\
21679.4042241485	0.00125212349272012\\
21685.2643144915	0.00101593323165793\\
21691.1244048346	0.000793163351206407\\
21696.9844951776	0.000589070002689916\\
21702.8445855207	0.000408467925958908\\
21708.7046758638	0.00025561686900393\\
21714.5647662068	0.000134121119197111\\
21720.4248565499	4.68445156296738e-05\\
21726.2849468929	-4.15705347971521e-06\\
21732.145037236	-1.76840772969702e-05\\
21738.0051275791	6.57836785429514e-06\\
21743.8652179221	6.8053288864956e-05\\
21749.7253082652	0.000165285496909142\\
21755.5853986082	0.000295976052946832\\
21761.4454889513	0.000457036622334211\\
21767.3055792943	0.000644662453455298\\
21773.1656696374	0.000854422255905918\\
21779.0257599805	0.0010813628551111\\
21784.8858503235	0.0013201261518781\\
21790.7459406666	0.00156507562502382\\
21796.6060310096	0.00181042939062635\\
21802.4661213527	0.00205039667687175\\
21808.3262116958	0.00227931449321152\\
21814.1863020388	0.00249178126828024\\
21820.0463923819	0.00268278430328859\\
21825.9064827249	0.00284781803319105\\
21831.766573068	0.00298299030619131\\
21837.6266634111	0.00308511417400602\\
21843.4867537541	0.00315178302898263\\
21849.3468440972	0.00318142731597603\\
21855.2069344402	0.0031733514830158\\
21861.0670247833	0.00312775030068174\\
21866.9271151264	0.00304570416747892\\
21872.7872054694	0.00292915351416795\\
21878.6472958125	0.00278085291357972\\
21884.5073861555	0.00260430598086248\\
21890.3674764986	0.00240368260310614\\
21896.2275668417	0.00218372045324131\\
21902.0876571847	0.00194961311443229\\
21907.9477475278	0.00170688745674643\\
21913.8078378708	0.00146127316119429\\
21919.6679282139	0.00121856747127369\\
21925.5280185569	0.000984498364458058\\
21931.3881089	0.000764589372995192\\
21937.2481992431	0.000564029244210892\\
21943.1082895861	0.000387549515836936\\
21948.9683799292	0.000239312894783654\\
21954.8284702722	0.000122815072568025\\
21960.6885606153	4.08022931965327e-05\\
21966.5486509584	-4.79338274669962e-06\\
21972.4087413014	-1.28995918351125e-05\\
21978.2688316445	1.66711127725148e-05\\
21984.1289219875	8.32169292912925e-05\\
21989.9890123306	0.000185163462725372\\
21995.8491026737	0.000320100972490042\\
22001.7091930167	0.000484841331038173\\
22007.5692833598	0.000675493347235683\\
22013.4293737028	0.000887554674900925\\
22019.2894640459	0.00111601813571697\\
22025.149554389	0.00135548994578969\\
22031.009644732	0.00160031705445061\\
22036.8697350751	0.00184472058897597\\
22042.7298254181	0.00208293225547603\\
22048.5899157612	0.00230933047621894\\
22054.4500061043	0.00251857305087442\\
22060.3100964473	0.00270572321065645\\
22066.1701867904	0.00286636609161833\\
22072.0302771334	0.00299671287874913\\
22077.8903674765	0.003093690164508\\
22083.7504578196	0.00315501241427375\\
22089.6105481626	0.00317923583013728\\
22095.4706385057	0.00316579234372959\\
22101.3307288487	0.00311500293750803\\
22107.1908191918	0.00302806998259028\\
22113.0509095349	0.00290704877566157\\
22118.9109998779	0.00275479894871231\\
22124.771090221	0.00257491690069267\\
22130.631180564	0.00237165084697883\\
22136.4912709071	0.00214980049387174\\
22142.3513612502	0.00191460370676974\\
22148.2114515932	0.001671612848364\\
22154.0715419363	0.00142656370606698\\
22159.9316322793	0.0011852401029903\\
22165.7917226224	0.000953337388016618\\
22171.6518129654	0.000736328026782848\\
22177.5119033085	0.000539332465649766\\
22183.3719936516	0.000366998315679557\\
22189.2320839946	0.00022339070732694\\
22195.0921743377	0.000111896402403045\\
22200.9522646807	3.51439251248846e-05\\
22206.8123550238	-5.05840432979276e-06\\
22212.6724453669	-7.76507293368395e-06\\
22218.5325357099	2.7084389079859e-05\\
22224.392626053	9.86640620266708e-05\\
22230.252716396	0.000205281210359837\\
22236.1128067391	0.000344416312183844\\
22241.9728970822	0.000512782598102907\\
22247.8329874252	0.000706403692488533\\
22253.6930777683	0.000920707523120472\\
22259.5531681113	0.00115063428147733\\
22265.4132584544	0.00139075588434417\\
22271.2733487975	0.00163540411647749\\
22277.1334391405	0.00187880442921946\\
22282.9935294836	0.00211521223675625\\
22288.8536198266	0.00233904849334019\\
22294.7137101697	0.0025450313515061\\
22300.5738005128	0.00272830079464606\\
22306.4338908558	0.00288453330298151\\
22312.2939811989	0.00301004384776905\\
22318.1540715419	0.00310187280745876\\
22324.014161885	0.00315785575616815\\
22329.8742522281	0.00317667447853186\\
22335.7343425711	0.00315788800870175\\
22341.5944329142	0.00310194296288637\\
22347.4545232572	0.00301016292294484\\
22353.3146136003	0.00288471712361161\\
22359.1747039434	0.00272856918415062\\
22365.0347942864	0.00254540709595185\\
22370.8948846295	0.00233955612011781\\
22376.7549749725	0.00211587665201912\\
22382.6150653156	0.00187964946473694\\
22388.4751556586	0.00163645104061069\\
22394.3352460017	0.00139202193415246\\
22400.1953363448	0.00115213127346555\\
22406.0554266878	0.000922440598519648\\
22411.9155170309	0.000708370249733787\\
22417.7756073739	0.000514971460086131\\
22423.635697717	0.000346807168967215\\
22429.4957880601	0.000207844369965308\\
22435.3558784031	0.00010136053232084\\
22441.2159687462	2.98663034621566e-05\\
22447.0760590892	-4.95368442179901e-06\\
22452.9361494323	-2.28050819703234e-06\\
22458.7962397754	3.78197872575473e-05\\
22464.6563301184	0.000114397817855907\\
22470.5164204615	0.000225643336778932\\
22476.3765108045	0.00036892802650677\\
22482.2366011476	0.000540867594888604\\
22488.0966914907	0.000737401708231853\\
22493.9567818337	0.000953889873499597\\
22499.8168721768	0.00118522100549985\\
22505.6769625198	0.00142593409216422\\
22511.5370528629	0.00167034710918789\\
22517.397143206	0.00191269114091845\\
22523.257233549	0.0021472465417992\\
22529.1173238921	0.0023684779246202\\
22534.9774142351	0.00257116478977409\\
22540.8375045782	0.00275052471302078\\
22546.6975949213	0.00290232618492463\\
22552.5576852643	0.00302298843953884\\
22558.4177756074	0.00310966591729985\\
22564.2778659504	0.00316031537020093\\
22570.1379562935	0.00317374402608486\\
22575.9980466366	0.00314963767746706\\
22581.8581369796	0.00308856803303492\\
22587.7182273227	0.00299197915987841\\
22593.5783176657	0.00286215333808297\\
22599.4384080088	0.00270215713492345\\
22605.2984983518	0.00251576897278676\\
22611.1585886949	0.00230738990181894\\
22617.018679038	0.00208193968385947\\
22622.878769381	0.00184474064196413\\
22628.7388597241	0.00160139201688716\\
22634.5989500671	0.00135763779703005\\
22640.4590404102	0.00111923114136709\\
22646.3191307533	0.000891798595481356\\
22652.1792210963	0.000680707305269361\\
22658.0393114394	0.000490938362023845\\
22663.8994017824	0.000326969267657327\\
22669.7594921255	0.000192668293299023\\
22675.6195824686	9.12032237855432e-05\\
22681.4796728116	2.49666407026223e-05\\
22687.3397631547	-4.4804937216856e-06\\
22693.1998534977	3.55437834762091e-06\\
22699.0599438408	4.8879123236168e-05\\
22704.9200341839	0.000130421509569822\\
22710.7801245269	0.00024625457320355\\
22716.64021487	0.000393642153240611\\
22722.500305213	0.000569103523113763\\
22728.3603955561	0.00076849559011972\\
22734.2204858992	0.000987110722275153\\
22740.0805762422	0.00121978789258141\\
22745.9406665853	0.00146103451687346\\
22751.8007569283	0.00170515610887546\\
22757.6608472714	0.00194639069217638\\
22763.5209376145	0.00217904479632731\\
22769.3810279575	0.00239762782755336\\
22775.2411183006	0.0025969816427352\\
22781.1012086436	0.00277240226895486\\
22786.9612989867	0.00291975089634746\\
22792.8213893298	0.00303555152548288\\
22798.6814796728	0.00311707296543532\\
22804.5415700159	0.0031623932482611\\
22810.4016603589	0.00317044494088894\\
22816.261750702	0.00314104028593801\\
22822.121841045	0.00307487557957131\\
22827.9819313881	0.00297351468418514\\
22833.8420217312	0.00283935206639484\\
22839.7021120742	0.00267555623391138\\
22845.5622024173	0.00248599490723884\\
22851.4222927603	0.00227514369367563\\
22857.2823831034	0.00204798041953301\\
22863.1424734465	0.00180986761577869\\
22869.0025637895	0.00156642593077006\\
22874.8626541326	0.00132340145869683\\
22880.7227444756	0.00108653011481503\\
};
\addplot [color=mycolor2, forget plot]
  table[row sep=crcr]{%
22880.7227444756	0.00108653011481503\\
22886.5828348187	0.000861402258853052\\
22892.4429251618	0.000653330761546587\\
22898.3030155048	0.000467225627731655\\
22904.1631058479	0.000307478134835005\\
22910.0231961909	0.000177857220571269\\
22915.883286534	8.14205645561398e-05\\
22921.7433768771	2.04424615879587e-05\\
22927.6034672201	-3.63981209199257e-06\\
22933.4635575632	9.74012524708163e-06\\
22939.3236479062	6.02644410705445e-05\\
22945.1837382493	0.000146738638253502\\
22951.0438285924	0.000267119794869069\\
22956.9039189354	0.000418564826472744\\
22962.7640092785	0.000597497630761689\\
22968.6240996215	0.000799693528888538\\
22974.4841899646	0.00102037900962845\\
22980.3442803077	0.0012543444214817\\
22986.2043706507	0.00149606695238417\\
22992.0644609938	0.00173984099396714\\
22997.9245513368	0.00197991281340175\\
23003.7846416799	0.00221061635395005\\
23009.6447320229	0.00242650695928725\\
23015.504822366	0.00262248986564705\\
23021.3649127091	0.00279394042944539\\
23027.2250030521	0.00293681325323594\\
23033.0850933952	0.00304773763518869\\
23038.9451837382	0.00312409708999906\\
23044.8052740813	0.00316409106521263\\
23050.6653644244	0.0031667773971503\\
23056.5254547674	0.00313209450585511\\
23062.3855451105	0.00306086280545275\\
23068.2456354535	0.00295476529839557\\
23074.1057257966	0.00281630781211539\\
23079.9658161397	0.00264875981775918\\
23085.8259064827	0.0024560772285584\\
23091.6859968258	0.00224280900082403\\
23097.5460871688	0.00201398974254761\\
23103.4061775119	0.00177502086512199\\
23109.266267855	0.00153154308315336\\
23115.126358198	0.00128930327248653\\
23120.9864485411	0.00105401882863958\\
23126.8465388841	0.000831242727223982\\
23132.7066292272	0.000626232471060041\\
23138.5667195703	0.000443826016700346\\
23144.4268099133	0.000288327608511191\\
23150.2869002564	0.000163406214180456\\
23156.1469905994	7.20089582365438e-05\\
23162.0070809425	1.62915959366599e-05\\
23167.8671712856	-2.43233227679195e-06\\
23173.7272616286	1.62775334424545e-05\\
23179.5873519717	7.19780163018575e-05\\
23185.4474423147	0.000163352900106792\\
23191.3075326578	0.000288244031139553\\
23197.1676230009	0.000443702289782391\\
23203.0277133439	0.000626057228032108\\
23208.887803687	0.000831003728648045\\
23214.74789403	0.00105370363988343\\
23220.6079843731	0.00128889998659323\\
23226.4680747162	0.00153104106156609\\
23232.3281650592	0.0017744114680305\\
23238.1882554023	0.00201326702029599\\
23244.0483457453	0.00224197031848113\\
23249.9084360884	0.00245512379765688\\
23255.7685264314	0.00264769711123416\\
23261.6286167745	0.00281514584262769\\
23267.4887071176	0.00295351874348592\\
23273.3487974606	0.00305955096821521\\
23279.2088878037	0.00313074110472752\\
23285.0689781467	0.00316541018410151\\
23290.9290684898	0.00316274127696465\\
23296.7891588329	0.00312279874297584\\
23302.6492491759	0.00304652667970685\\
23308.509339519	0.00293572660830513\\
23314.369429862	0.00279301492349569\\
23320.2295202051	0.00262176111254601\\
23326.0896105482	0.00242600820243765\\
23331.9497008912	0.00221037731292248\\
23337.8097912343	0.001979958569241\\
23343.6698815773	0.0017401909490252\\
23349.5299719204	0.00149673390007424\\
23355.3900622635	0.00125533375879847\\
23361.2501526065	0.0010216881225975\\
23367.1102429496	0.000801311376971325\\
23372.9703332926	0.000599404551240569\\
23378.8304236357	0.00042073257420086\\
23384.6905139788	0.000269511826509594\\
23390.5506043218	0.000149310643285781\\
23396.4106946649	6.29651148781004e-05\\
23402.2707850079	1.25121724722545e-05\\
23408.130875351	-8.5846270381279e-07\\
23413.9909656941	2.31676669029033e-05\\
23419.8510560371	8.40223600124816e-05\\
23425.7111463802	0.00018026819382562\\
23431.5712367232	0.000309632476041659\\
23437.4313270663	0.000469060909647021\\
23443.2914174093	0.000654789703332076\\
23449.1515077524	0.000862434425106863\\
23455.0115980955	0.00108709350147891\\
23460.8716884385	0.00132346391919193\\
23466.7317787816	0.00156596639829582\\
23472.5918691246	0.00180887708212871\\
23478.4519594677	0.00204646263590623\\
23484.3120498108	0.00227311556530636\\
23490.1721401538	0.00248348656134311\\
23496.0322304969	0.00267261074815455\\
23501.8923208399	0.00283602485397446\\
23507.752411183	0.00296987254054473\\
23513.6125015261	0.00307099540487415\\
23519.4725918691	0.00313700750641909\\
23525.3326822122	0.00316635166067952\\
23531.1927725552	0.00315833617138655\\
23537.0528628983	0.00311315113484939\\
23542.9129532414	0.00303186393183529\\
23548.7730435844	0.00291639401417057\\
23554.6331339275	0.00276946758167119\\
23560.4932242705	0.00259455321928503\\
23566.3533146136	0.00239578001412959\\
23572.2134049567	0.00217784008551419\\
23578.0734952997	0.00194587782846629\\
23583.9335856428	0.00170536848525696\\
23589.7936759858	0.00146198891095097\\
23595.6537663289	0.00122148358345847\\
23601.513856672	0.000989529020273607\\
23607.373947015	0.000771599802240223\\
23613.2340373581	0.00057283936599741\\
23619.0941277011	0.000397938614805616\\
23624.9542180442	0.000251025212382475\\
23630.8143083873	0.000135566172420187\\
23636.6743987303	5.42860425817925e-05\\
23642.5344890734	9.10261360959128e-06\\
23648.3945794164	1.08167060916587e-06\\
23654.2546697595	3.04118540724698e-05\\
23660.1147601026	9.64002235887186e-05\\
23665.9748504456	0.000197488628571264\\
23671.8349407887	0.000331290499238187\\
23677.6950311317	0.000494647189191627\\
23683.5551214748	0.000683702539442639\\
23689.4152118178	0.000893993903777696\\
23695.2753021609	0.00112055748681614\\
23701.135392504	0.00135804550845667\\
23706.995482847	0.00160085242911215\\
23712.8555731901	0.00184324725662905\\
23718.7156635331	0.00207950881207306\\
23724.5757538762	0.00230406076177345\\
23730.4358442193	0.00251160322893876\\
23736.2959345623	0.00269723787816496\\
23742.1560249054	0.00285658352129508\\
23748.0161152484	0.00298587951579517\\
23753.8762055915	0.00308207451594186\\
23759.7362959346	0.0031428984816753\\
23765.5963862776	0.00316691624660542\\
23771.4564766207	0.0031535613804648\\
23777.3165669637	0.00310314954747763\\
23783.1766573068	0.00301687104530908\\
23789.0367476499	0.00289676270111265\\
23794.8968379929	0.00274565978831946\\
23800.756928336	0.00256712909893085\\
23806.617018679	0.00236538475137398\\
23812.4771090221	0.00214518872106238\\
23818.3371993652	0.00191173844163099\\
23824.1972897082	0.00167054412948655\\
23830.0573800513	0.00142729872741364\\
23835.9174703943	0.0011877435365919\\
23841.7775607374	0.000957532708655188\\
23847.6376510805	0.000742099795682522\\
23853.4977414235	0.000546529508093835\\
23859.3578317666	0.000375437706815771\\
23865.2179221096	0.000232862462252826\\
23871.0780124527	0.00012216875107912\\
23876.9381027957	4.59690400613531e-05\\
23882.7981931388	6.06163117441586e-06\\
23888.6582834819	3.38822303215212e-06\\
23894.5183738249	3.80116901284416e-05\\
23900.378464168	0.000109114604210599\\
23906.238554511	0.000215018532550082\\
23912.0986448541	0.000353223657520941\\
23917.9587351972	0.000520467782298221\\
23923.8188255402	0.000712803329900807\\
23929.6789158833	0.000925690518250051\\
23935.5390062263	0.00115410451200649\\
23941.3990965694	0.00139265402219157\\
23947.2591869125	0.00163570855442351\\
23953.1192772555	0.00187753130238685\\
23958.9793675986	0.0021124145500993\\
23964.8394579416	0.00233481438709966\\
23970.6995482847	0.0025394815572651\\
23976.5596386278	0.00272158535269575\\
23982.4197289708	0.00287682762893246\\
23988.2798193139	0.00300154425033711\\
23994.1399096569	0.00309279157124961\\
24000	0.00314841591240077\\
};
\end{axis}
\end{tikzpicture}%

% This file was created by matlab2tikz.
%
%The latest updates can be retrieved from
%  http://www.mathworks.com/matlabcentral/fileexchange/22022-matlab2tikz-matlab2tikz
%where you can also make suggestions and rate matlab2tikz.
%
\definecolor{mycolor1}{rgb}{0.00000,0.44700,0.74100}%
\definecolor{mycolor2}{rgb}{0.85000,0.32500,0.09800}%
%
\begin{tikzpicture}

\begin{axis}[%
width=4.521in,
height=3.548in,
at={(0.758in,0.499in)},
scale only axis,
unbounded coords=jump,
xmin=-25000,
xmax=25000,
ymode=log,
ymin=1e-06,
ymax=2.01031921814129,
yminorticks=true,
axis background/.style={fill=white}
]
\addplot [color=mycolor1, forget plot]
  table[row sep=crcr]{%
-24000	0.996832895608896\\
-22118.9109998779	0.997425083103077\\
-22001.7091930167	0.999324506656028\\
-21890.3674764986	0.997816279546436\\
-21773.1656696374	0.998918637145748\\
-21655.9638627762	0.998015312415495\\
-21538.762055915	0.998715345822663\\
-21421.5602490538	0.998222183864211\\
-21304.3584421926	0.998505912788735\\
-21181.2965449884	0.998186508694978\\
-21064.0947381272	0.998541681705627\\
-20946.892931266	0.998398002424056\\
-20823.8310340618	0.998577648912566\\
-20706.6292272006	0.99836229466118\\
-20583.5673299963	0.998613826209011\\
-20466.3655231351	0.998326353057042\\
-20349.163716274	0.998400284672398\\
-20226.1018190697	0.998290165344215\\
-20108.9000122085	0.998436341329032\\
-19985.8381150043	0.998253718814034\\
-19868.6363081431	0.998472699863488\\
-19751.4345012819	0.9984698721876\\
-19628.3726040776	0.998509374151514\\
-19511.1707972165	0.998433744768065\\
-19388.1089000122	0.998546378648339\\
-19270.907093151	0.998397266383833\\
-19153.7052862898	0.998327577897565\\
-19030.6433890856	0.998360421961488\\
-18913.4415822244	0.998364099916366\\
-18790.3796850201	0.998323195719407\\
-18673.177878159	0.998401046699811\\
-18555.9760712978	0.998545354899659\\
-18432.9141740935	0.998438435657947\\
-18315.7123672323	0.998508829888668\\
-18192.6504700281	0.998476285082214\\
-18075.4486631669	0.998471828698683\\
-17958.2468563057	0.998250795477041\\
-17835.1849591015	0.998434332246732\\
-17717.9831522403	0.9982877647853\\
-17594.921255036	0.998396320441471\\
-17477.7194481748	0.998325291767864\\
-17360.5176413136	0.998626036742342\\
-17237.4557441094	0.99836339874995\\
-17120.2539372482	0.998589153788874\\
-16997.1920400439	0.998402109324027\\
-16879.9902331828	0.998551659321487\\
-16762.7884263216	0.998168263083267\\
-16639.7265291173	0.998513528801385\\
-16522.5247222561	0.998205643474438\\
-16405.3229153949	0.998750289360982\\
-16282.2610181907	0.998243726991221\\
-16165.0592113295	0.998713848285666\\
-16041.9973141253	0.998282542622693\\
-15924.7955072641	0.998676677965681\\
-15807.5937004029	0.998041023433576\\
-15684.5318031986	0.998638748484395\\
-15567.3299963374	0.998077918071155\\
-15444.2680991332	0.998600028015826\\
-15327.066292272	0.998115641198052\\
-15209.8644854108	0.998847684143466\\
-15086.8025882066	0.998154228359988\\
-14969.6007813454	0.998811231197942\\
-14846.5388841411	0.998193717476693\\
-14729.3370772799	0.998773895797101\\
-14612.1352704188	0.997940176338702\\
-14489.0733732145	0.998735639162434\\
-14371.8715663533	0.997977122489888\\
-14254.6697594921	0.998992872409103\\
-14131.6078622879	0.998015065708074\\
-14014.4060554267	0.998957631226353\\
-13891.3441582224	0.99805405247396\\
-13774.1423513612	0.998921387929662\\
-13656.9405445001	0.997789975040251\\
-13533.8786472958	0.998884095661933\\
-13416.6768404346	0.997825613544543\\
-13293.6149432304	0.998845703995176\\
-13176.4131363692	0.997862379233059\\
-13059.211329508	0.999118843297217\\
-12936.1494323037	0.99790032863796\\
-12818.9476254426	0.999084342150723\\
-12701.7458185814	0.997624166970605\\
-12578.6839213771	0.999048673577544\\
-12461.4821145159	0.997657801810784\\
-12338.4202173117	0.999011777048074\\
-12221.2184104505	0.997692683751328\\
-12104.0166035893	0.999298759283758\\
-11980.9547063851	0.997728880862959\\
-11863.7528995239	0.999266874459693\\
-11740.6910023196	0.997766467148784\\
-11623.4891954584	0.99923372514095\\
-11506.2873885972	0.997469292468153\\
-11383.225491393	0.99919924001508\\
-11266.0236845318	0.997501327008615\\
-11148.8218776706	0.999502618074376\\
-11025.7599804664	0.997534785947494\\
-10908.5581736052	0.999474562365066\\
-10785.4962764009	0.997569757679972\\
-10668.2944695397	0.999445188483724\\
-10551.0926626786	0.997254098269628\\
-10428.0307654743	0.999414414971301\\
-10310.8289586131	0.997281927184534\\
-10193.6271517519	0.999737675502979\\
-10070.5652545477	0.997311242332677\\
-9953.36344768649	0.999715189706602\\
-9830.30155048224	0.997342146588606\\
-9713.09974362105	0.99969141634205\\
-9595.89793675986	0.997003812058187\\
-9472.83603955561	0.999666268001692\\
-9355.63423269442	0.9970254512871\\
-9232.57233549017	0.999639647603724\\
-9115.37052862898	0.997048542818365\\
-8998.16872176779	1\\
-8875.10682456355	0.997073200368364\\
-8757.90501770236	0.999984587851479\\
-8640.70321084117	0.996706201494268\\
-8517.64131363692	0.999968001824271\\
-8400.43950677573	0.99671863036335\\
-8277.37760957148	0.999950151478832\\
-8160.17580271029	0.996732285126175\\
-8042.9739958491	1.00034592225815\\
-7919.91209864485	0.996747276656142\\
-7802.71029178366	1.00034325355995\\
-7679.64839457942	0.996763732379059\\
-7568.30667806129	1.0007466660933\\
-7439.38469051398	0.996781799490862\\
-7328.04297399585	1.0007629609204\\
-7199.12098644854	0.996801648986212\\
-7093.63936027347	1.00117254046307\\
-6958.8572823831	0.996823480707068\\
-6853.37565620803	1.0012124626192\\
-6718.59357831766	0.996847529653874\\
-6613.1119521426	1.00125465091802\\
-6484.18996459529	0.996341369711935\\
-6372.84824807716	1.00129941439487\\
-6243.92626052985	0.996343591974253\\
-6138.44463435478	1.00177081121673\\
-5997.80246612135	0.996936047542885\\
-5904.0410206324	1.00222236387663\\
-5751.67867171286	0.997629915316042\\
-5663.77731656696	1.00233766353038\\
-5517.27505799048	0.997012989360955\\
-5429.37370284459	1.00280084254137\\
-5265.29117323892	0.998533335895457\\
-5189.10999877915	1.00296951287786\\
-5030.88755951654	0.997879091663719\\
-4954.70638505677	1.00344890616027\\
-4778.90367476498	0.999633925359806\\
-4714.44268099133	1.00368880845461\\
-4544.50006104261	0.998976167006839\\
-4474.17897692589	1.0039569249358\\
-4310.09644732023	0.998234800181342\\
-4239.77536320352	1.00453228240155\\
-4175.31436942986	0.996409093187677\\
-4116.71346599927	0.992501187171489\\
-4046.39238188255	1.00216154085782\\
-3993.65156879502	1.00460372339531\\
-3835.42912953241	0.99751625599657\\
-3765.1080454157	1.00556365993178\\
-3712.36723232816	0.999043486767625\\
-3647.90623855451	0.991303982546033\\
-3595.16542546698	0.997648365395583\\
-3524.84434135026	1.00612798755571\\
-3472.10352826273	0.998915267203532\\
-3407.64253448907	0.990740432618655\\
-3354.90172140154	0.997811403423119\\
-3290.44072762788	1.00690301909345\\
-3243.56000488341	1.00130066977383\\
-3167.37883042364	0.990072462165594\\
-3120.49810767916	0.996678234573682\\
-3050.17702356245	1.00776420609403\\
-3003.29630081797	1.00138494970508\\
-2932.97521670126	0.989198680830231\\
-2891.95458429984	0.994010395470691\\
-2809.91331949701	1.00883862699568\\
-2768.89268709559	1.00297177443135\\
-2692.71151263582	0.988145596868668\\
-2651.6908802344	0.993807993924732\\
-2569.64961543157	1.01022545647664\\
-2528.62898303015	1.0033231583576\\
-2452.44780857038	0.986795818711662\\
-2411.42717616897	0.993585260858148\\
-2335.24600170919	1.01216671848221\\
-2300.08545965084	1.00734677840065\\
-2206.32401416188	0.985280086419699\\
-2171.16347210353	0.993360957849265\\
-2100.84238798681	1.01473416419364\\
-2065.68184592846	1.01063170281041\\
-1960.20021975339	0.983729894764392\\
-1925.03967769503	0.995867027936065\\
-1866.43877426444	1.01857184964234\\
-1837.13832254914	1.01713219173129\\
-1801.97778049078	1.00252184813393\\
-1749.23696740325	0.979469520097654\\
-1719.93651568795	0.980475766028825\\
-1690.63606397265	0.993494969204586\\
-1632.03516054206	1.02551737408785\\
-1608.59479916982	1.0270473779307\\
-1585.15443779758	1.01882562234953\\
-1544.13380539617	0.99019951403007\\
-1514.83335368087	0.9737029497909\\
-1491.39299230863	0.971612847063762\\
-1467.95263093639	0.981819134009605\\
-1438.6521792211	1.00765672763801\\
-1403.49163716274	1.03958633238676\\
-1380.0512757905	1.04860799814564\\
-1362.47100476132	1.04503319746053\\
-1339.03064338909	1.02655516198561\\
-1303.87010133073	0.982896537691829\\
-1280.42973995849	0.959708094065395\\
-1262.84946892931	0.954232975724163\\
-1245.26919790014	0.963181203408916\\
-1227.68892687096	0.98782224689618\\
-1210.10865584178	1.02641408471288\\
-1163.2279330973	1.15163037851228\\
-1151.50775241118	1.17352094110361\\
-1139.78757172506	1.18566143807447\\
-1128.06739103895	1.18533953045042\\
-1116.34721035283	1.17032911510502\\
-1104.62702966671	1.13907227488671\\
-1092.90684898059	1.09082742786468\\
-1081.18666829447	1.02577260482068\\
-1069.46648760835	0.94505563488776\\
-1063.60639726529	0.899447048543343\\
-1057.74630692223	0.850786484979452\\
-1051.88621657917	0.799477112357801\\
-1046.02612623611	0.745970830243797\\
-1040.16603589305	0.690763229858802\\
-1034.30594554999	0.634387851397492\\
-1028.44585520693	0.577409823109672\\
-1022.58576486387	0.520418977184212\\
-1016.72567452081	0.464022545893965\\
-1010.86558417775	0.408837548291054\\
-1005.00549383469	0.35548298235287\\
-999.145403491639	0.3045719404386\\
-993.285313148579	0.256703766500901\\
-987.425222805519	0.212456372314503\\
-981.56513246246	0.172378826660861\\
-975.7050421194	0.136984326241274\\
-969.84495177634	0.106743650040611\\
-958.124771090221	0.0633596428597587\\
-952.264680747161	0.0508954289468671\\
-946.404590404101	0.0449349065649443\\
-940.544500061042	0.0456614181101872\\
-934.684409717982	0.0531912055387559\\
-928.824319374922	0.0675722109178138\\
-917.104138688806	0.1167371584567\\
-911.244048345747	0.15127705573661\\
-905.383958002687	0.192183740818182\\
-899.523867659627	0.239176119363421\\
-893.663777316568	0.29191543318068\\
-887.803686973508	0.350009612126327\\
-881.943596630448	0.413018186548377\\
-876.083506287388	0.480457675957719\\
-870.223415944329	0.551807364638034\\
-864.363325601269	0.626515371693959\\
-858.503235258209	0.70400492135538\\
-852.64314491515	0.783680719516259\\
-846.78305457209	0.864935344250767\\
-840.92296422903	0.947155561231066\\
-835.062873885974	1.02972847989858\\
-829.202783542914	1.11204747212259\\
-823.342693199855	1.19351778244862\\
-811.622512513735	1.35162370894618\\
-799.902331827616	1.49971379171229\\
-788.182151141496	1.63393317599829\\
-776.461970455377	1.7510019194178\\
-764.741789769258	1.84827790080604\\
-753.021609083142	1.92378657891012\\
-741.301428397022	1.97622090119845\\
-729.581247710903	2.00491625036143\\
-717.861067024784	2.00980634185377\\
-706.140886338664	1.99136641083882\\
-694.420705652545	1.950549865115\\
-682.700524966425	1.88872389827024\\
-670.980344280306	1.80760845759489\\
-659.26016359419	1.70922157994595\\
-647.539982908071	1.5958325988802\\
-635.819802221951	1.46992324031516\\
-624.099621535832	1.33415530743284\\
-612.379440849712	1.19134262616218\\
-600.659260163593	1.04442426820744\\
-594.799169820533	0.970371190522173\\
-588.939079477474	0.896435835667216\\
-583.078989134417	0.82300721809515\\
-577.218898791358	0.750475784610004\\
-571.358808448298	0.679231927509071\\
-565.498718105238	0.609664349062848\\
-559.638627762179	0.542158276223431\\
-553.778537419119	0.477093529468452\\
-547.918447076059	0.414842454637032\\
-542.058356733	0.355767731357972\\
-536.19826638994	0.300220076248053\\
-530.33817604688	0.248535863398309\\
-524.47808570382	0.201034688337709\\
-518.617995360761	0.158016905053816\\
-512.757905017701	0.119761168221692\\
-506.897814674641	0.0865220148568635\\
-501.037724331585	0.0585275208026193\\
-495.177633988525	0.0359770680418801\\
-489.317543645466	0.0190392585477494\\
-483.457453302406	0.00785000941386517\\
-477.597362959346	0.00251086225912748\\
-471.737272616287	0.00308753750039357\\
-465.877182273227	0.00960876101515315\\
-460.017091930167	0.0220653870951406\\
-454.157001587108	0.0404098374684634\\
-448.296911244048	0.0645558716463799\\
-442.436820900988	0.0943786990291977\\
-436.576730557928	0.129715438170306\\
-430.716640214869	0.170365923483219\\
-424.856549871809	0.216093854548109\\
-418.996459528749	0.266628278185201\\
-413.136369185693	0.321665388666851\\
-407.276278842633	0.38087062692543\\
-401.416188499574	0.443881055599092\\
-395.556098156514	0.510307982980246\\
-389.696007813454	0.579739806000109\\
-383.835917470395	0.651745039672601\\
-377.975827127335	0.725875498643981\\
-372.115736784275	0.801669595071956\\
-366.255646441216	0.878655716514344\\
-360.395556098156	0.956355647412887\\
-354.535465755096	1.03428799839619\\
-342.815285068977	1.18892888879279\\
-331.095104382861	1.33879135089252\\
-319.374923696741	1.4802465076536\\
-307.654743010622	1.60991631992812\\
-295.934562324503	1.72475460601704\\
-284.214381638383	1.82211467656764\\
-272.494200952264	1.8998027206274\\
-260.774020266144	1.95611686268043\\
-249.053839580029	1.98987245605357\\
-237.333658893909	2.00041464654201\\
-225.61347820779	1.98761951181921\\
-213.89329752167	1.95188516227118\\
-202.173116835551	1.89411409936246\\
-190.452936149431	1.81568791115953\\
-178.732755463312	1.71843509222644\\
-167.012574777196	1.60459246494166\\
-155.292394091077	1.4767604070695\\
-143.572213404957	1.33785190519818\\
-131.852032718838	1.19103539109236\\
-120.131852032719	1.03967139699291\\
-114.271761689659	0.963369855247249\\
-108.411671346599	0.887243287282018\\
-102.551581003539	0.811734065523141\\
-96.6914906604798	0.737282670606966\\
-90.83140031742	0.66432514761198\\
-84.9713099743603	0.593290532616441\\
-79.1112196313043	0.524598265977831\\
-73.2511292882446	0.458655610687447\\
-67.3910389451848	0.395855095241961\\
-61.5309486021251	0.336572001914722\\
-55.6708582590654	0.28116192199057\\
-49.8107679160057	0.229958400141359\\
-43.950677572946	0.183270690263628\\
-38.0905872298863	0.141381644892136\\
-32.2304968868266	0.104545759699547\\
-26.3704065437669	0.0729873936031107\\
-20.5103162007072	0.0468991836360684\\
-14.6502258576475	0.0264406719969562\\
-8.79013551458775	0.0117371606487721\\
-2.93004517152804	0.00287880645765348\\
2.93004517152804	-8.00327355025226e-05\\
8.79013551458775	0.00287880645765348\\
14.6502258576475	0.0117371606487721\\
20.5103162007072	0.0264406719969562\\
26.3704065437669	0.0468991836360684\\
32.2304968868266	0.0729873936031107\\
38.0905872298863	0.104545759699547\\
43.950677572946	0.141381644892136\\
49.8107679160057	0.183270690263628\\
55.6708582590654	0.229958400141359\\
61.5309486021251	0.28116192199057\\
67.3910389451848	0.336572001914722\\
73.2511292882446	0.395855095241961\\
79.1112196313043	0.458655610687447\\
84.9713099743603	0.524598265977831\\
90.83140031742	0.593290532616441\\
96.6914906604798	0.66432514761198\\
102.551581003539	0.737282670606966\\
108.411671346599	0.811734065523141\\
114.271761689659	0.887243287282018\\
120.131852032719	0.963369855247249\\
125.991942375778	1.03967139699291\\
137.712123061898	1.19103539109236\\
149.432303748017	1.33785190519818\\
161.152484434137	1.4767604070695\\
172.872665120252	1.60459246494166\\
184.592845806372	1.71843509222644\\
196.313026492491	1.81568791115953\\
208.033207178611	1.89411409936246\\
219.75338786473	1.95188516227118\\
231.473568550849	1.98761951181921\\
243.193749236969	2.00041464654201\\
254.913929923085	1.98987245605357\\
266.634110609204	1.95611686268043\\
278.354291295323	1.8998027206274\\
290.074471981443	1.82211467656764\\
301.794652667562	1.72475460601704\\
313.514833353682	1.60991631992812\\
325.235014039801	1.4802465076536\\
336.955194725917	1.33879135089252\\
348.675375412036	1.18892888879279\\
360.395556098156	1.03428799839619\\
366.255646441216	0.956355647412887\\
372.115736784275	0.878655716514344\\
377.975827127335	0.801669595071956\\
383.835917470395	0.725875498643981\\
389.696007813454	0.651745039672601\\
395.556098156514	0.579739806000109\\
401.416188499574	0.510307982980246\\
407.276278842633	0.443881055599092\\
413.136369185693	0.38087062692543\\
418.996459528749	0.321665388666851\\
424.856549871809	0.266628278185201\\
430.716640214869	0.216093854548109\\
436.576730557928	0.170365923483219\\
442.436820900988	0.129715438170306\\
448.296911244048	0.0943786990291977\\
454.157001587108	0.0645558716463799\\
460.017091930167	0.0404098374684634\\
465.877182273227	0.0220653870951406\\
471.737272616287	0.00960876101515315\\
477.597362959346	0.00308753750039357\\
483.457453302406	0.00251086225912748\\
489.317543645466	0.00785000941386517\\
495.177633988525	0.0190392585477494\\
501.037724331585	0.0359770680418801\\
506.897814674641	0.0585275208026193\\
512.757905017701	0.0865220148568635\\
518.617995360761	0.119761168221692\\
524.47808570382	0.158016905053816\\
530.33817604688	0.201034688337709\\
536.19826638994	0.248535863398309\\
542.058356733	0.300220076248053\\
547.918447076059	0.355767731357972\\
553.778537419119	0.414842454637032\\
559.638627762179	0.477093529468452\\
565.498718105238	0.542158276223431\\
571.358808448298	0.609664349062848\\
577.218898791358	0.679231927509071\\
583.078989134417	0.750475784610004\\
588.939079477474	0.82300721809515\\
594.799169820533	0.896435835667216\\
600.659260163593	0.970371190522173\\
612.379440849712	1.11820883041428\\
624.099621535832	1.26344848457997\\
635.819802221951	1.40309898238801\\
647.539982908071	1.5342804818781\\
659.26016359419	1.65425181630979\\
670.980344280306	1.76043751227084\\
682.700524966425	1.85045763297059\\
694.420705652545	1.92216315329097\\
706.140886338664	1.97367872277151\\
717.861067024784	2.00345348679547\\
729.581247710903	2.01031921814129\\
741.301428397022	1.9935534943581\\
753.021609083142	1.95294419102921\\
764.741789769258	1.888850302438\\
776.461970455377	1.80225319476383\\
788.182151141496	1.69479196384835\\
799.902331827616	1.5687766965916\\
811.622512513735	1.42717416129172\\
823.342693199855	1.27356176725305\\
829.202783542914	1.19351778244862\\
835.062873885974	1.11204747212259\\
840.92296422903	1.02972847989858\\
846.78305457209	0.947155561231066\\
852.64314491515	0.864935344250767\\
858.503235258209	0.783680719516259\\
864.363325601269	0.70400492135538\\
870.223415944329	0.626515371693959\\
876.083506287388	0.551807364638034\\
881.943596630448	0.480457675957719\\
887.803686973508	0.413018186548377\\
893.663777316568	0.350009612126327\\
899.523867659627	0.29191543318068\\
905.383958002687	0.239176119363421\\
911.244048345747	0.192183740818182\\
917.104138688806	0.15127705573661\\
922.964229031862	0.1167371584567\\
934.684409717982	0.0675722109178138\\
940.544500061042	0.0531912055387559\\
946.404590404101	0.0456614181101872\\
952.264680747161	0.0449349065649443\\
958.124771090221	0.0508954289468671\\
963.98486143328	0.0633596428597587\\
975.7050421194	0.106743650040611\\
981.56513246246	0.136984326241274\\
987.425222805519	0.172378826660861\\
993.285313148579	0.212456372314503\\
999.145403491639	0.256703766500901\\
1005.00549383469	0.3045719404386\\
1010.86558417775	0.35548298235287\\
1016.72567452081	0.408837548291054\\
1022.58576486387	0.464022545893965\\
1028.44585520693	0.520418977184212\\
1034.30594554999	0.577409823109672\\
1040.16603589305	0.634387851397492\\
1046.02612623611	0.690763229858802\\
1051.88621657917	0.745970830243797\\
1057.74630692223	0.799477112357801\\
1063.60639726529	0.850786484979452\\
1069.46648760835	0.899447048543343\\
1075.32657795141	0.94505563488776\\
1087.04675863753	1.02577260482068\\
1098.76693932365	1.09082742786468\\
1110.48712000977	1.13907227488671\\
1122.20730069589	1.17032911510502\\
1133.927481382	1.18533953045042\\
1145.64766206812	1.18566143807447\\
1157.36784275424	1.17352094110361\\
1174.94811378342	1.13795970991585\\
1198.38847515566	1.0740413458022\\
1227.68892687096	0.999314778718142\\
1245.26919790014	0.96967155352373\\
1262.84946892931	0.955493607395847\\
1280.42973995849	0.956466453974541\\
1298.01001098767	0.969636504731472\\
1368.33109510438	1.04503319746053\\
1385.91136613356	1.04860799814564\\
1409.3517275058	1.03958633238676\\
1438.6521792211	1.01357300381582\\
1479.67281162251	0.978172952518464\\
1503.11317299475	0.970987312130006\\
1526.55353436699	0.976000744338978\\
1561.71407642535	0.998937489796227\\
1596.8746184837	1.0217198976992\\
1620.31497985594	1.02759576268758\\
1643.75534122818	1.02367026066286\\
1678.91588328653	1.00455161395467\\
1719.93651568795	0.982238044497972\\
1749.23696740325	0.978622546087365\\
1778.53741911854	0.987273218588935\\
1854.71859357832	1.01908115482726\\
1884.01904529361	1.01605011995891\\
1925.03967769503	0.998642735160218\\
1966.06031009645	0.983729894764392\\
1995.36076181175	0.983535542852743\\
2030.5213038701	0.994600252688535\\
2089.1222073007	1.01427702465734\\
2124.28274935905	1.01199220000341\\
2182.88365278965	0.991390914292411\\
2218.044194848	0.984990686209556\\
2253.20473690636	0.989940984337306\\
2341.10609205225	1.01216671848221\\
2376.26663411061	1.00658237790266\\
2458.30789891344	0.986795818711662\\
2499.32853131486	0.992858632969638\\
2575.50970577463	1.01022545647664\\
2616.53033817605	1.00514266838457\\
2698.57160297888	0.988145596868668\\
2739.5922353803	0.993739096028116\\
2815.77340984007	1.00883862699568\\
2856.79404224148	1.00408808834625\\
2944.69539738738	0.9893577570768\\
2991.57612013185	0.997151558963844\\
3056.03711390551	1.00776420609403\\
3102.91783664998	1.00204119091642\\
3179.09901110976	0.99004709103107\\
3225.97973385423	0.996192330290711\\
3296.30081797094	1.00690301909345\\
3343.18154071542	1.00149238370662\\
3419.36271517519	0.990747937563676\\
3472.10352826273	0.997725582765613\\
3536.56452203638	1.00619437060048\\
3589.30533512392	0.999930677343879\\
3659.62641924063	0.991338608268868\\
3712.36723232816	0.997943552822156\\
3776.82822610182	1.0055989330132\\
3829.56903918935	0.999632696094606\\
3899.89012330607	0.991844655360532\\
3958.49102673666	0.999153001859629\\
4022.95202051031	1.0049446420332\\
4087.41301428397	0.996453740372192\\
4146.01391771456	0.992508514855878\\
4222.19509217434	1.00266302664932\\
4274.93590526187	1.00376167877472\\
4415.5780734953	0.995819575132754\\
4509.33951898425	1.00379215631079\\
4667.56195824686	0.997707425594251\\
4743.74313270663	1.00373982784518\\
4919.54584299841	0.999468934484346\\
4984.00683677207	1.00342457024295\\
5159.80954706385	0.99952723307841\\
5230.13063118056	1.00283564570942\\
5382.49298010011	0.997403496660058\\
5464.53424490294	1.00288309704148\\
5634.47686485167	0.998941180613018\\
5710.65803931144	1.00234124163613\\
5857.16029788793	0.99703076558458\\
5950.92174337688	1.00212498080733\\
6097.42400195336	0.997174958504069\\
6191.18544744232	1.00192569592976\\
6337.6877060188	0.997308198965534\\
6431.44915150775	1.00174123904595\\
6577.95141008424	0.997431791028484\\
6671.71285557319	1.00156981612853\\
6818.21511414968	0.997546833978176\\
6911.97655963863	1.00140991625016\\
7052.61872787205	0.997120562294703\\
7152.24026370407	1.001260257175\\
7292.88243193749	0.997234884553152\\
7392.5039677695	1.00111974328217\\
7533.14613600293	0.99734233621052\\
7632.76767183494	1.00098743237546\\
7773.40984006837	0.997443587672973\\
7873.03137590038	1.00086250950372\\
8013.67354413381	0.997539222313923\\
8113.29507996582	1.00074426596485\\
8248.07715785618	0.997176427220803\\
8353.55878403125	1.00063208233352\\
8488.34086192162	0.997270940256605\\
8593.82248809669	1.00052541474962\\
8728.60456598706	0.997360845519608\\
8834.08619216213	1.00042378361499\\
8968.8682700525	0.997446524320757\\
9074.34989622757	1.00032676426611\\
9209.13197411794	0.997528315694111\\
9314.613600293	1.00023397930964\\
9449.39567818337	0.997606522130844\\
9554.87730435844	1.00014509213173\\
9689.65938224881	0.997681414400053\\
9795.14100842388	1.00005980151245\\
9924.06299597119	0.997369752073881\\
10041.2648028324	0.999635779522416\\
10158.4666096936	0.997088445525501\\
10281.5285068978	0.999559373508491\\
10398.730313759	0.997163127481571\\
10521.7922109633	0.999485810234261\\
10638.9940178244	0.997235173128067\\
10762.0559150287	0.99941489410079\\
10879.2577218899	0.997304759661315\\
11002.3196190941	0.999346447618324\\
11119.5214259553	0.997372048321964\\
11242.5833231596	0.999280309319404\\
11359.7851300208	0.997437186173558\\
11482.847027225	0.99921633205038\\
11600.0488340862	0.997500307590131\\
11723.1107312904	0.999154381331314\\
11840.3125381516	0.997561535644733\\
11963.3744353559	0.999094334077006\\
12080.5762422171	0.997620983265017\\
12203.6381394213	0.999036077394327\\
12320.8399462825	0.997678754264332\\
12438.0417531437	0.999297365892883\\
12561.1036503479	0.997734944223111\\
12678.3054572091	0.999239991136187\\
12801.3673544134	0.997789641337458\\
12918.5691612746	0.999184116114039\\
13041.6310584788	0.997842927092744\\
13158.83286534	0.999129657943485\\
13281.8947625443	0.997894876895867\\
13399.0965694054	0.999076539950561\\
13522.1584666097	0.99794556063267\\
13639.3602734709	0.999024691056645\\
13762.4221706751	0.997995043175541\\
13879.6239775363	0.998974045307422\\
14002.6858747406	0.998043384824429\\
14119.8876816018	0.998924541410523\\
14242.949578806	0.998090641689595\\
14360.1513856672	0.998876122298525\\
14477.3531925284	0.997843741780814\\
14600.4150897326	0.998828734809326\\
14717.6168965938	0.997890908929009\\
14840.6787937981	0.998782329333246\\
14957.8806006593	0.997937163044036\\
15080.9424978635	0.998736859518867\\
15198.1443047247	0.997982547202462\\
15321.2062019289	0.998692282004134\\
15438.4080087901	0.998027101757273\\
15561.4699059944	0.998648556189408\\
15678.6717128556	0.998070864562638\\
15801.7336100598	0.998605644010643\\
15918.935416921	0.998113871173709\\
16041.9973141253	0.99856350973776\\
16159.1991209864	0.998156154987978\\
16282.2610181907	0.998522119798214\\
16399.4628250519	0.99819774747869\\
16516.6646319131	0.99875587623264\\
16639.7265291173	0.998238678261105\\
16756.9283359785	0.998714203128856\\
16879.9902331828	0.99827897527588\\
16997.1920400439	0.998673149089442\\
17120.2539372482	0.998318664913985\\
17237.4557441094	0.998632687595478\\
17360.5176413136	0.998357772108214\\
17477.7194481748	0.998592793563823\\
17600.7813453791	0.998396320441471\\
17717.9831522403	0.998553443263037\\
17841.0450494445	0.998434332246732\\
17958.2468563057	0.998514614204243\\
18081.3087535099	0.998471828698683\\
18198.5105603711	0.998476285082214\\
18315.7123672323	0.998247530992583\\
18438.7742644366	0.998438435657947\\
18555.9760712978	0.998285571159586\\
18679.037968502	0.998401046699811\\
18796.2397753632	0.998323195719407\\
18919.3016725674	0.998364099916366\\
19036.5034794286	0.998360421961488\\
19159.5653766329	0.998327577897565\\
19276.7671834941	0.998397266383833\\
19399.8290806983	0.998291464030889\\
19517.0308875595	0.998433744768065\\
19640.0927847638	0.998255742484413\\
19757.294591625	0.9984698721876\\
19874.4963984861	0.998472699863488\\
19997.5582956904	0.998505663091104\\
20114.7601025516	0.998436341329032\\
20237.8219997558	0.998541131352518\\
20355.023806617	0.998400284672398\\
20478.0857038213	0.998576290262543\\
20595.2875106825	0.998364516555352\\
20718.3494078867	0.998611152628874\\
20835.5512147479	0.99832902416032\\
20958.6131119521	0.99864573075935\\
21075.8149188133	0.998293795106627\\
21193.0167256745	0.998433488196848\\
21316.0786228788	0.998258817458754\\
21433.28042974	0.998468763481287\\
21556.3423269442	0.998224079684414\\
21673.5441338054	0.998503839412849\\
21796.6060310096	0.998189570612652\\
21913.8078378708	0.998538726835942\\
22036.8697350751	0.998155279408667\\
22154.0715419363	0.99857343629547\\
22277.1334391405	0.998121195573732\\
22394.3352460017	0.998607978061857\\
22511.5370528629	0.998329652889421\\
22634.5989500671	0.99864236220627\\
22751.8007569283	0.998294843893876\\
22874.8626541326	0.998676598542008\\
22992.0644609938	0.998260159002638\\
23115.126358198	0.998710696724831\\
23232.3281650592	0.998225588528144\\
23349.5299719204	0.998503266104091\\
23472.5918691246	0.998191122918003\\
23589.7936759858	0.998538011093011\\
23712.8555731901	0.998156752746594\\
23830.0573800513	0.99857270127189\\
23953.1192772555	0.998122468698315\\
24000	0.996851584090231\\
};
\addplot [color=mycolor2, forget plot]
  table[row sep=crcr]{%
-24000	0.00316710439196338\\
-23994.1399096569	0.00314841591240077\\
-23988.2798193139	0.00309279157124961\\
-23982.4197289708	0.00300154425033711\\
-23976.5596386278	0.00287682762893246\\
-23970.6995482847	0.00272158535269575\\
-23964.8394579416	0.0025394815572651\\
-23958.9793675986	0.00233481438709966\\
-23953.1192772555	0.0021124145500993\\
-23947.2591869125	0.00187753130238685\\
-23941.3990965694	0.00163570855442351\\
-23935.5390062263	0.00139265402219157\\
-23929.6789158833	0.00115410451200649\\
-23923.8188255402	0.000925690518250051\\
-23917.9587351972	0.000712803329900807\\
-23912.0986448541	0.000520467782298221\\
-23906.238554511	0.000353223657520941\\
-23900.378464168	0.000215018532550082\\
-23894.5183738249	0.000109114604210599\\
-23888.6582834819	3.80116901284416e-05\\
-23882.7981931388	3.38822303215212e-06\\
-23876.9381027957	6.06163117441586e-06\\
-23871.0780124527	4.59690400613531e-05\\
-23865.2179221096	0.00012216875107912\\
-23859.3578317666	0.000232862462252826\\
-23853.4977414235	0.000375437706815771\\
-23847.6376510805	0.000546529508093835\\
-23841.7775607374	0.000742099795682522\\
-23835.9174703943	0.000957532708655188\\
-23830.0573800513	0.0011877435365919\\
-23824.1972897082	0.00142729872741364\\
-23818.3371993652	0.00167054412948655\\
-23812.4771090221	0.00191173844163099\\
-23806.617018679	0.00214518872106238\\
-23800.756928336	0.00236538475137398\\
-23794.8968379929	0.00256712909893085\\
-23789.0367476499	0.00274565978831946\\
-23783.1766573068	0.00289676270111265\\
-23777.3165669637	0.00301687104530908\\
-23771.4564766207	0.00310314954747763\\
-23765.5963862776	0.0031535613804648\\
-23759.7362959346	0.00316691624660542\\
-23753.8762055915	0.0031428984816753\\
-23748.0161152484	0.00308207451594186\\
-23742.1560249054	0.00298587951579517\\
-23736.2959345623	0.00285658352129508\\
-23730.4358442193	0.00269723787816496\\
-23724.5757538762	0.00251160322893876\\
-23718.7156635331	0.00230406076177345\\
-23712.8555731901	0.00207950881207306\\
-23706.995482847	0.00184324725662905\\
-23701.135392504	0.00160085242911215\\
-23695.2753021609	0.00135804550845667\\
-23689.4152118178	0.00112055748681614\\
-23683.5551214748	0.000893993903777696\\
-23677.6950311317	0.000683702539442639\\
-23671.8349407887	0.000494647189191627\\
-23665.9748504456	0.000331290499238187\\
-23660.1147601026	0.000197488628571264\\
-23654.2546697595	9.64002235887186e-05\\
-23648.3945794164	3.04118540724698e-05\\
-23642.5344890734	1.08167060916587e-06\\
-23636.6743987303	9.10261360959128e-06\\
-23630.8143083873	5.42860425817925e-05\\
-23624.9542180442	0.000135566172420187\\
-23619.0941277011	0.000251025212382475\\
-23613.2340373581	0.000397938614805616\\
-23607.373947015	0.00057283936599741\\
-23601.513856672	0.000771599802240223\\
-23595.6537663289	0.000989529020273607\\
-23589.7936759858	0.00122148358345847\\
-23583.9335856428	0.00146198891095097\\
-23578.0734952997	0.00170536848525696\\
-23572.2134049567	0.00194587782846629\\
-23566.3533146136	0.00217784008551419\\
-23560.4932242705	0.00239578001412959\\
-23554.6331339275	0.00259455321928503\\
-23548.7730435844	0.00276946758167119\\
-23542.9129532414	0.00291639401417057\\
-23537.0528628983	0.00303186393183529\\
-23531.1927725552	0.00311315113484939\\
-23525.3326822122	0.00315833617138655\\
-23519.4725918691	0.00316635166067952\\
-23513.6125015261	0.00313700750641909\\
-23507.752411183	0.00307099540487415\\
-23501.8923208399	0.00296987254054473\\
-23496.0322304969	0.00283602485397446\\
-23490.1721401538	0.00267261074815455\\
-23484.3120498108	0.00248348656134311\\
-23478.4519594677	0.00227311556530636\\
-23472.5918691246	0.00204646263590623\\
-23466.7317787816	0.00180887708212871\\
-23460.8716884385	0.00156596639829582\\
-23455.0115980955	0.00132346391919193\\
-23449.1515077524	0.00108709350147891\\
-23443.2914174093	0.000862434425106863\\
-23437.4313270663	0.000654789703332076\\
-23431.5712367232	0.000469060909647021\\
-23425.7111463802	0.000309632476041659\\
-23419.8510560371	0.00018026819382562\\
-23413.9909656941	8.40223600124816e-05\\
-23408.130875351	2.31676669029033e-05\\
-23402.2707850079	-8.5846270381279e-07\\
-23396.4106946649	1.25121724722545e-05\\
-23390.5506043218	6.29651148781004e-05\\
-23384.6905139788	0.000149310643285781\\
-23378.8304236357	0.000269511826509594\\
-23372.9703332926	0.00042073257420086\\
-23367.1102429496	0.000599404551240569\\
-23361.2501526065	0.000801311376971325\\
-23355.3900622635	0.0010216881225975\\
-23349.5299719204	0.00125533375879847\\
-23343.6698815773	0.00149673390007424\\
-23337.8097912343	0.0017401909490252\\
-23331.9497008912	0.001979958569241\\
-23326.0896105482	0.00221037731292248\\
-23320.2295202051	0.00242600820243765\\
-23314.369429862	0.00262176111254601\\
-23308.509339519	0.00279301492349569\\
-23302.6492491759	0.00293572660830513\\
-23296.7891588329	0.00304652667970685\\
-23290.9290684898	0.00312279874297584\\
-23285.0689781467	0.00316274127696465\\
-23279.2088878037	0.00316541018410151\\
-23273.3487974606	0.00313074110472752\\
-23267.4887071176	0.00305955096821521\\
-23261.6286167745	0.00295351874348592\\
-23255.7685264314	0.00281514584262769\\
-23249.9084360884	0.00264769711123416\\
-23244.0483457453	0.00245512379765688\\
-23238.1882554023	0.00224197031848113\\
-23232.3281650592	0.00201326702029599\\
-23226.4680747162	0.0017744114680305\\
-23220.6079843731	0.00153104106156609\\
-23214.74789403	0.00128889998659323\\
-23208.887803687	0.00105370363988343\\
-23203.0277133439	0.000831003728648045\\
-23197.1676230009	0.000626057228032108\\
-23191.3075326578	0.000443702289782391\\
-23185.4474423147	0.000288244031139553\\
-23179.5873519717	0.000163352900106792\\
-23173.7272616286	7.19780163018575e-05\\
-23167.8671712856	1.62775334424545e-05\\
-23162.0070809425	-2.43233227679195e-06\\
-23156.1469905994	1.62915959366599e-05\\
-23150.2869002564	7.20089582365438e-05\\
-23144.4268099133	0.000163406214180456\\
-23138.5667195703	0.000288327608511191\\
-23132.7066292272	0.000443826016700346\\
-23126.8465388841	0.000626232471060041\\
-23120.9864485411	0.000831242727223982\\
-23115.126358198	0.00105401882863958\\
-23109.266267855	0.00128930327248653\\
-23103.4061775119	0.00153154308315336\\
-23097.5460871688	0.00177502086512199\\
-23091.6859968258	0.00201398974254761\\
-23085.8259064827	0.00224280900082403\\
-23079.9658161397	0.0024560772285584\\
-23074.1057257966	0.00264875981775918\\
-23068.2456354535	0.00281630781211539\\
-23062.3855451105	0.00295476529839557\\
-23056.5254547674	0.00306086280545275\\
-23050.6653644244	0.00313209450585511\\
-23044.8052740813	0.0031667773971503\\
-23038.9451837382	0.00316409106521263\\
-23033.0850933952	0.00312409708999906\\
-23027.2250030521	0.00304773763518869\\
-23021.3649127091	0.00293681325323594\\
-23015.504822366	0.00279394042944539\\
-23009.6447320229	0.00262248986564705\\
-23003.7846416799	0.00242650695928725\\
-22997.9245513368	0.00221061635395005\\
-22992.0644609938	0.00197991281340175\\
-22986.2043706507	0.00173984099396714\\
-22980.3442803077	0.00149606695238417\\
-22974.4841899646	0.0012543444214817\\
-22968.6240996215	0.00102037900962845\\
-22962.7640092785	0.000799693528888538\\
-22956.9039189354	0.000597497630761689\\
-22951.0438285924	0.000418564826472744\\
-22945.1837382493	0.000267119794869069\\
-22939.3236479062	0.000146738638253502\\
-22933.4635575632	6.02644410705445e-05\\
-22927.6034672201	9.74012524708163e-06\\
-22921.7433768771	-3.63981209199257e-06\\
-22915.883286534	2.04424615879587e-05\\
-22910.0231961909	8.14205645561398e-05\\
-22904.1631058479	0.000177857220571269\\
-22898.3030155048	0.000307478134835005\\
-22892.4429251618	0.000467225627731655\\
-22886.5828348187	0.000653330761546587\\
-22880.7227444756	0.000861402258853052\\
-22874.8626541326	0.00108653011481503\\
-22869.0025637895	0.00132340145869683\\
-22863.1424734465	0.00156642593077006\\
-22857.2823831034	0.00180986761577869\\
-22851.4222927603	0.00204798041953301\\
-22845.5622024173	0.00227514369367563\\
-22839.7021120742	0.00248599490723884\\
-22833.8420217312	0.00267555623391138\\
-22827.9819313881	0.00283935206639484\\
-22822.121841045	0.00297351468418514\\
-22816.261750702	0.00307487557957131\\
-22810.4016603589	0.00314104028593801\\
-22804.5415700159	0.00317044494088894\\
-22798.6814796728	0.0031623932482611\\
-22792.8213893298	0.00311707296543532\\
-22786.9612989867	0.00303555152548288\\
-22781.1012086436	0.00291975089634746\\
-22775.2411183006	0.00277240226895486\\
-22769.3810279575	0.0025969816427352\\
-22763.5209376145	0.00239762782755336\\
-22757.6608472714	0.00217904479632731\\
-22751.8007569283	0.00194639069217638\\
-22745.9406665853	0.00170515610887546\\
-22740.0805762422	0.00146103451687346\\
-22734.2204858992	0.00121978789258141\\
-22728.3603955561	0.000987110722275153\\
-22722.500305213	0.00076849559011972\\
-22716.64021487	0.000569103523113763\\
-22710.7801245269	0.000393642153240611\\
-22704.9200341839	0.00024625457320355\\
-22699.0599438408	0.000130421509569822\\
-22693.1998534977	4.8879123236168e-05\\
-22687.3397631547	3.55437834762091e-06\\
-22681.4796728116	-4.4804937216856e-06\\
-22675.6195824686	2.49666407026223e-05\\
-22669.7594921255	9.12032237855432e-05\\
-22663.8994017824	0.000192668293299023\\
-22658.0393114394	0.000326969267657327\\
-22652.1792210963	0.000490938362023845\\
-22646.3191307533	0.000680707305269361\\
-22640.4590404102	0.000891798595481356\\
-22634.5989500671	0.00111923114136709\\
-22628.7388597241	0.00135763779703005\\
-22622.878769381	0.00160139201688716\\
-22617.018679038	0.00184474064196413\\
-22611.1585886949	0.00208193968385947\\
-22605.2984983518	0.00230738990181894\\
-22599.4384080088	0.00251576897278676\\
-22593.5783176657	0.00270215713492345\\
-22587.7182273227	0.00286215333808297\\
-22581.8581369796	0.00299197915987841\\
-22575.9980466366	0.00308856803303492\\
-22570.1379562935	0.00314963767746706\\
-22564.2778659504	0.00317374402608486\\
-22558.4177756074	0.00316031537020093\\
-22552.5576852643	0.00310966591729985\\
-22546.6975949213	0.00302298843953884\\
-22540.8375045782	0.00290232618492463\\
-22534.9774142351	0.00275052471302078\\
-22529.1173238921	0.00257116478977409\\
-22523.257233549	0.0023684779246202\\
-22517.397143206	0.0021472465417992\\
-22511.5370528629	0.00191269114091845\\
-22505.6769625198	0.00167034710918789\\
-22499.8168721768	0.00142593409216422\\
-22493.9567818337	0.00118522100549985\\
-22488.0966914907	0.000953889873499597\\
-22482.2366011476	0.000737401708231853\\
-22476.3765108045	0.000540867594888604\\
-22470.5164204615	0.00036892802650677\\
-22464.6563301184	0.000225643336778932\\
-22458.7962397754	0.000114397817855907\\
-22452.9361494323	3.78197872575473e-05\\
-22447.0760590892	-2.28050819703234e-06\\
-22441.2159687462	-4.95368442179901e-06\\
-22435.3558784031	2.98663034621566e-05\\
-22429.4957880601	0.00010136053232084\\
-22423.635697717	0.000207844369965308\\
-22417.7756073739	0.000346807168967215\\
-22411.9155170309	0.000514971460086131\\
-22406.0554266878	0.000708370249733787\\
-22400.1953363448	0.000922440598519648\\
-22394.3352460017	0.00115213127346555\\
-22388.4751556586	0.00139202193415246\\
-22382.6150653156	0.00163645104061069\\
-22376.7549749725	0.00187964946473694\\
-22370.8948846295	0.00211587665201912\\
-22365.0347942864	0.00233955612011781\\
-22359.1747039434	0.00254540709595185\\
-22353.3146136003	0.00272856918415062\\
-22347.4545232572	0.00288471712361161\\
-22341.5944329142	0.00301016292294484\\
-22335.7343425711	0.00310194296288637\\
-22329.8742522281	0.00315788800870175\\
-22324.014161885	0.00317667447853186\\
-22318.1540715419	0.00315785575616815\\
-22312.2939811989	0.00310187280745876\\
-22306.4338908558	0.00301004384776905\\
-22300.5738005128	0.00288453330298151\\
-22294.7137101697	0.00272830079464606\\
-22288.8536198266	0.0025450313515061\\
-22282.9935294836	0.00233904849334019\\
-22277.1334391405	0.00211521223675625\\
-22271.2733487975	0.00187880442921946\\
-22265.4132584544	0.00163540411647749\\
-22259.5531681113	0.00139075588434417\\
-22253.6930777683	0.00115063428147733\\
-22247.8329874252	0.000920707523120472\\
-22241.9728970822	0.000706403692488533\\
-22236.1128067391	0.000512782598102907\\
-22230.252716396	0.000344416312183844\\
-22224.392626053	0.000205281210359837\\
-22218.5325357099	9.86640620266708e-05\\
-22212.6724453669	2.7084389079859e-05\\
-22206.8123550238	-7.76507293368395e-06\\
-22200.9522646807	-5.05840432979276e-06\\
-22195.0921743377	3.51439251248846e-05\\
-22189.2320839946	0.000111896402403045\\
-22183.3719936516	0.00022339070732694\\
-22177.5119033085	0.000366998315679557\\
-22171.6518129654	0.000539332465649766\\
-22165.7917226224	0.000736328026782848\\
-22159.9316322793	0.000953337388016618\\
-22154.0715419363	0.0011852401029903\\
-22148.2114515932	0.00142656370606698\\
-22142.3513612502	0.001671612848364\\
-22136.4912709071	0.00191460370676974\\
-22130.631180564	0.00214980049387174\\
-22124.771090221	0.00237165084697883\\
-22118.9109998779	0.00257491690069267\\
-22113.0509095349	0.00275479894871231\\
-22107.1908191918	0.00290704877566157\\
-22101.3307288487	0.00302806998259028\\
-22095.4706385057	0.00311500293750803\\
-22089.6105481626	0.00316579234372959\\
-22083.7504578196	0.00317923583013728\\
-22077.8903674765	0.00315501241427375\\
-22072.0302771334	0.003093690164508\\
-22066.1701867904	0.00299671287874913\\
-22060.3100964473	0.00286636609161833\\
-22054.4500061043	0.00270572321065645\\
-22048.5899157612	0.00251857305087442\\
-22042.7298254181	0.00230933047621894\\
-22036.8697350751	0.00208293225547603\\
-22031.009644732	0.00184472058897597\\
-22025.149554389	0.00160031705445061\\
-22019.2894640459	0.00135548994578969\\
-22013.4293737028	0.00111601813571697\\
-22007.5692833598	0.000887554674900925\\
-22001.7091930167	0.000675493347235683\\
-21995.8491026737	0.000484841331038173\\
-21989.9890123306	0.000320100972490042\\
-21984.1289219875	0.000185163462725372\\
-21978.2688316445	8.32169292912925e-05\\
-21972.4087413014	1.66711127725148e-05\\
-21966.5486509584	-1.28995918351125e-05\\
-21960.6885606153	-4.79338274669962e-06\\
-21954.8284702722	4.08022931965327e-05\\
-21948.9683799292	0.000122815072568025\\
-21943.1082895861	0.000239312894783654\\
-21937.2481992431	0.000387549515836936\\
-21931.3881089	0.000564029244210892\\
-21925.5280185569	0.000764589372995192\\
-21919.6679282139	0.000984498364458058\\
-21913.8078378708	0.00121856747127369\\
-21907.9477475278	0.00146127316119429\\
-21902.0876571847	0.00170688745674643\\
-21896.2275668417	0.00194961311443229\\
-21890.3674764986	0.00218372045324131\\
-21884.5073861555	0.00240368260310614\\
-21878.6472958125	0.00260430598086248\\
-21872.7872054694	0.00278085291357972\\
-21866.9271151264	0.00292915351416795\\
-21861.0670247833	0.00304570416747892\\
-21855.2069344402	0.00312775030068174\\
-21849.3468440972	0.0031733514830158\\
-21843.4867537541	0.00318142731597603\\
-21837.6266634111	0.00315178302898263\\
-21831.766573068	0.00308511417400602\\
-21825.9064827249	0.00298299030619131\\
-21820.0463923819	0.00284781803319105\\
-21814.1863020388	0.00268278430328859\\
-21808.3262116958	0.00249178126828024\\
-21802.4661213527	0.00227931449321152\\
-21796.6060310096	0.00205039667687175\\
-21790.7459406666	0.00181042939062635\\
-21784.8858503235	0.00156507562502382\\
-21779.0257599805	0.0013201261518781\\
-21773.1656696374	0.0010813628551111\\
-21767.3055792943	0.000854422255905918\\
-21761.4454889513	0.000644662453455298\\
-21755.5853986082	0.000457036622334211\\
-21749.7253082652	0.000295976052946832\\
-21743.8652179221	0.000165285496909142\\
-21738.0051275791	6.8053288864956e-05\\
-21732.145037236	6.57836785429514e-06\\
-21726.2849468929	-1.76840772969702e-05\\
-21720.4248565499	-4.15705347971521e-06\\
-21714.5647662068	4.68445156296738e-05\\
-21708.7046758638	0.000134121119197111\\
-21702.8445855207	0.00025561686900393\\
-21696.9844951776	0.000408467925958908\\
-21691.1244048346	0.000589070002689916\\
-21685.2643144915	0.000793163351206407\\
-21679.4042241485	0.00101593323165793\\
-21673.5441338054	0.00125212349272012\\
-21667.6840434623	0.00149616058417645\\
-21661.8239531193	0.0017422850759966\\
-21655.9638627762	0.0019846875803692\\
-21650.1037724332	0.00221764586926778\\
-21644.2436820901	0.00243565995089277\\
-21638.383591747	0.00263358191700662\\
-21632.523501404	0.00280673749533747\\
-21626.6634110609	0.00295103643708406\\
-21620.8033207179	0.00306306913297706\\
-21614.9432303748	0.00314018717540569\\
-21609.0831400317	0.00318056596340672\\
-21603.2230496887	0.00318324787040729\\
-21597.3629593456	0.0031481649537589\\
-21591.5028690026	0.00307614066740391\\
-21585.6427786595	0.00296887053453593\\
-21579.7826883164	0.00282888223403618\\
-21573.9225979734	0.00265947604038275\\
-21568.0625076303	0.00246464701992132\\
-21562.2024172873	0.00224899081828803\\
-21556.3423269442	0.00201759526007206\\
-21550.4822366011	0.00177592031809374\\
-21544.6221462581	0.00152966928354407\\
-21538.762055915	0.00128465417689415\\
-21532.901965572	0.00104665857552332\\
-21527.0418752289	0.000821301095556858\\
-21521.1817848858	0.000613902750050097\\
-21515.3216945428	0.00042936131505069\\
-21509.4616041997	0.000272035669565267\\
-21503.6015138567	0.000145642840780946\\
-21497.7414235136	5.31701861906142e-05\\
-21491.8813331706	-3.19521269947748e-06\\
-21486.0212428275	-2.21182763003273e-05\\
-21480.1611524844	-3.1475492157126e-06\\
-21474.3010621414	5.32740301088044e-05\\
-21468.4409717983	0.000145819469240684\\
-21462.5808814553	0.000272308929783336\\
-21456.7207911122	0.000429761069651542\\
-21450.8607007691	0.000614463310369162\\
-21445.0006104261	0.000822059373266662\\
-21439.140520083	0.00104765202075956\\
-21433.28042974	0.00128591857954535\\
-21427.4203393969	0.00153123652077801\\
-21421.5602490538	0.0017778161343922\\
-21415.7001587108	0.00201983716683753\\
-21409.8400683677	0.00225158619786335\\
-21403.9799780247	0.00246759151377172\\
-21398.1198876816	0.00266275229325185\\
-21392.2597973385	0.00283245905618398\\
-21386.3997069955	0.00297270253069246\\
-21380.5396166524	0.00308016836798435\\
-21374.6795263094	0.00315231546681139\\
-21368.8194359663	0.00318743605714714\\
-21362.9593456232	0.00318469612222795\\
-21357.0992552802	0.00314415520234353\\
-21351.2391649371	0.00306676510985293\\
-21345.3790745941	0.00295434758344741\\
-21339.518984251	0.00280955140584624\\
-21333.6588939079	0.00263578999428017\\
-21327.7988035649	0.00243716093445656\\
-21321.9387132218	0.00221834935470449\\
-21316.0786228788	0.00198451741860817\\
-21310.2185325357	0.00174118254285544\\
-21304.3584421926	0.00149408721301988\\
-21298.4983518496	0.00124906346904916\\
-21292.6382615065	0.00101189525817956\\
-21286.7781711635	0.000788181904068069\\
-21280.9180808204	0.000583205914801907\\
-21275.0579904774	0.000401808250545425\\
-21269.1979001343	0.000248273996051639\\
-21263.3378097912	0.000126231137913003\\
-21257.4777194482	3.85648376552169e-05\\
-21251.6176291051	-1.26507737700991e-05\\
-21245.7575387621	-2.62016677201703e-05\\
-21239.897448419	-1.76269488871369e-06\\
-21234.0373580759	6.00946144661507e-05\\
-21228.1772677329	0.000157915414167216\\
-21222.3171773898	0.000289395757199251\\
-21216.4570870468	0.000451436857593549\\
-21210.5969967037	0.00064021812117586\\
-21204.7369063606	0.000851287224124466\\
-21198.8768160176	0.00107966511550895\\
-21193.0167256745	0.00131996346766029\\
-21187.1566353315	0.00156651180374478\\
-21181.2965449884	0.00181349130351671\\
-21175.4364546453	0.00205507212966843\\
-21169.5763643023	0.00228555103410204\\
-21163.7162739592	0.00249948599611892\\
-21157.8561836162	0.00269182471422838\\
-21151.9960932731	0.00285802391758802\\
-21146.13600293	0.00299415667996184\\
-21140.275912587	0.00309700520125194\\
-21134.4158222439	0.00316413686430596\\
-21128.5557319009	0.00319396176997816\\
-21122.6956415578	0.00318577038886933\\
-21116.8355512147	0.00313975043883375\\
-21110.9754608717	0.00305698258643913\\
-21105.1153705286	0.00293941507080181\\
-21099.2552801856	0.00278981784564513\\
-21093.3951898425	0.00261171731907342\\
-21087.5350994994	0.00240931322825304\\
-21081.6750091564	0.00218737960872977\\
-21075.8149188133	0.00195115219337911\\
-21069.9548284703	0.00170620489645645\\
-21064.0947381272	0.00145831829705226\\
-21058.2346477842	0.00121334322453697\\
-21052.3745574411	0.000977062665328763\\
-21046.514467098	0.000755055250273588\\
-21040.654376755	0.000552563544909791\\
-21034.7942864119	0.000374370252199059\\
-21028.9341960689	0.000224685251098001\\
-21023.0741057258	0.000107046138705955\\
-21017.2140153827	2.4234625778998e-05\\
-21011.3539250397	-2.1789238725292e-05\\
-21005.4938346966	-2.99334587582342e-05\\
nan	nan\\
-20993.7736540105	6.7310398293246e-05\\
-20987.9135636674	0.000170414625237528\\
-20982.0534733244	0.000306884430186792\\
-20976.1933829813	0.000473503608921444\\
-20970.3332926383	0.000666343797465135\\
-20964.4732022952	0.000880857087363718\\
-20958.6131119521	0.00111198327904755\\
-20952.7530216091	0.00135426924396446\\
-20946.892931266	0.00160199757994134\\
-20941.032840923	0.00184932152459245\\
-20935.1727505799	0.00209040294326255\\
-20929.3126602368	0.00231955013498935\\
-20923.4525698938	0.00253135220365635\\
-20917.5924795507	0.00272080682239459\\
-20911.7323892077	0.00288343837409154\\
-20905.8722988646	0.00301540367833292\\
-20900.0122085215	0.00311358280714593\\
-20894.1521181785	0.00317565284344941\\
-20888.2920278354	0.00320014283863114\\
-20882.4319374924	0.00318646866930303\\
-20876.5718471493	0.00313494696671604\\
-20870.7117568062	0.00304678778741879\\
-20864.8516664632	0.00292406619401881\\
-20858.9915761201	0.00276967341411233\\
-20853.1314857771	0.00258724872640782\\
-20847.271395434	0.00238109367907378\\
-20841.411305091	0.00215607066185803\\
-20835.5512147479	0.00191748822394495\\
-20829.6911244048	0.00167097584175276\\
-20823.8310340618	0.0014223510915718\\
-20817.9709437187	0.00117748236034001\\
-20812.1108533757	0.000942150334932957\\
-20806.2507630326	0.000721911538721629\\
-20800.3906726895	0.000521967137027511\\
-20794.5305823465	0.000347040108862511\\
-20788.6704920034	0.000201263685613291\\
-20782.8104016604	8.80836917304688e-05\\
-20776.9503113173	1.01770948662542e-05\\
-20771.0902209742	-3.0611309356979e-05\\
-20765.2301306312	-3.33125804714552e-05\\
-20759.3700402881	2.14335015914561e-06\\
-20753.5099499451	7.49258758179725e-05\\
-20747.649859602	0.000183323170174404\\
-20741.7897692589	0.000324782446600355\\
-20735.9296789159	0.000495970074259087\\
-20730.0695885728	0.000692850135347828\\
-20724.2094982298	0.000910779572327836\\
-20718.3494078867	0.00114461768209432\\
-20712.4893175436	0.00138884737503024\\
-20706.6292272006	0.00163770533880346\\
-20700.7691368575	0.00188531803599131\\
-20694.9090465145	0.00212584032652471\\
-20689.0489561714	0.00235359344330247\\
-20683.1888658283	0.00256319906432044\\
-20677.3287754853	0.00274970631572453\\
-20671.4686851422	0.00290870870720147\\
-20665.6085947992	0.00303644823754358\\
-20659.7485044561	0.00312990421121617\\
-20653.8884141131	0.00318686466657632\\
-20648.02832377	0.003205978726969\\
-20642.1682334269	0.00318678863591547\\
-20636.3081430839	0.00312974071630488\\
-20630.4480527408	0.00303617499155613\\
-20624.5879623978	0.00290829371006961\\
-20618.7278720547	0.0027491095121929\\
-20612.8677817116	0.00256237445937183\\
-20607.0076913686	0.00235249159792822\\
-20601.1476010255	0.00212441114117562\\
-20595.2875106825	0.00188351371818666\\
-20589.4274203394	0.00163548344248802\\
-20583.5673299963	0.0013861737950614\\
-20577.7072396533	0.00114146948591178\\
-20571.8471493102	0.000907147554284746\\
-20565.9870589672	0.00068874098589843\\
-20560.1269686241	0.000491408066685149\\
-20554.266878281	0.000319810557931723\\
-20548.406787938	0.000178003569884647\\
-20542.5466975949	6.93397352561368e-05\\
-20536.6866072519	-3.61005289553254e-06\\
-20530.8265169088	-3.91174651187153e-05\\
-20524.9664265657	-3.63376823612654e-05\\
-20519.1063362227	4.67050443822302e-06\\
-20513.2462458796	8.2945920120351e-05\\
-20507.3861555366	0.000196647531327036\\
-20501.5260651935	0.000343097744896139\\
-20495.6659748504	0.000518845460369104\\
-20489.8058845074	0.000719747391844812\\
-20483.9457941643	0.000941065743037567\\
-20478.0857038213	0.00117757993293203\\
-20472.2256134782	0.00142370973740285\\
-20466.3655231351	0.00167364694233901\\
-20460.5054327921	0.0019214924022144\\
-20454.645342449	0.00216139526995276\\
-20448.785252106	0.00238769111223189\\
-20442.9251617629	0.00259503564985561\\
-20437.0650714198	0.00277853096561077\\
-20431.2049810768	0.00293384119935575\\
-20425.3448907337	0.00305729499728184\\
-20419.4848003907	0.0031459722950326\\
-20413.6247100476	0.00319777338318522\\
-20407.7646197046	0.00321146862150076\\
-20401.9045293615	0.00318672762535123\\
-20396.0444390184	0.00312412723097415\\
-20390.1843486754	0.00302513804812941\\
-20384.3242583323	0.00289208991347358\\
-20378.4641679893	0.00272811705584226\\
-20372.6040776462	0.00253708426446659\\
-20366.7439873031	0.00232349579893282\\
-20360.8838969601	0.00209238918780621\\
-20355.023806617	0.00184921642009313\\
-20349.163716274	0.00159971533113736\\
-20343.3036259309	0.00134977421724108\\
-20337.4435355878	0.00110529287324029\\
-20331.5834452448	0.000872043332353873\\
-20325.7233549017	0.000655533595495984\\
-20319.8632645587	0.000460877566986869\\
-20314.0031742156	0.000292674268021953\\
-20308.1430838725	0.000154899180587666\\
-20302.2829935295	5.0810288928693e-05\\
-20296.4229031864	-1.71289608970521e-05\\
-20290.5628128434	-4.73079616430469e-05\\
-20284.7027225003	-3.90071259829308e-05\\
-20278.8426321572	7.58495697702066e-06\\
-20272.9825418142	9.13757987774044e-05\\
-20267.1224514711	0.000210394625440384\\
-20261.2623611281	0.000361838727574037\\
-20255.402270785	0.000542139456705485\\
-20249.5421804419	0.000747046313836898\\
-20243.6820900989	0.000971727148917599\\
-20237.8219997558	0.00121088210922227\\
-20231.9619094128	0.00145886864972079\\
-20226.1018190697	0.00170983465669336\\
-20220.2417287267	0.0019578565441099\\
-20214.3816383836	0.00219707906397015\\
-20208.5215480405	0.00242185353095053\\
-20202.6614576975	0.00262687119822895\\
-20196.8013673544	0.00280728863553673\\
-20190.9412770114	0.00295884214821608\\
-20185.0811866683	0.00307794853466199\\
-20179.2210963252	0.00316178980087591\\
-20173.3610059822	0.00320837982946083\\
-20167.5009156391	0.00321661142590622\\
-20161.6408252961	0.00318628262782528\\
-20155.780734953	0.00311810165184447\\
-20149.9206446099	0.00301367035727035\\
-20144.0605542669	0.00287544661260889\\
-20138.2004639238	0.0027066864489546\\
-20132.3403735808	0.00251136736170224\\
-20126.4802832377	0.00229409456716184\\
-20120.6201928946	0.00205999242345\\
-20114.7601025516	0.00181458357556377\\
-20108.9000122085	0.00156365867478947\\
-20103.0399218655	0.00131313974585914\\
-20097.1798315224	0.00106894042550905\\
-20091.3197411793	0.000836826370871963\\
-20085.4596508363	0.000622279132573751\\
-20079.5995604932	0.000430366706511684\\
-20073.7394701502	0.000265623821086051\\
-20067.8793798071	0.000131944787599777\\
-20062.019289464	3.24914455642226e-05\\
-20056.159199121	-3.03816211365579e-05\\
-20050.2991087779	-5.51828285233295e-05\\
-20044.4390184349	-4.1318977528854e-05\\
-20038.5789280918	1.08905596460874e-05\\
-20032.7188377487	0.000100221191034979\\
-20026.8587474057	0.000224571825151499\\
-20020.9986570626	0.000381014286463652\\
-20015.1385667196	0.000565862263930092\\
-20009.2784763765	0.000774758169158772\\
-20003.4183860334	0.00100277585766498\\
-19997.5582956904	0.00124453679189974\\
-19991.6982053473	0.0014943369067839\\
-19985.8381150043	0.00174628118556276\\
-19979.9780246612	0.00199442277095076\\
-19974.1179343182	0.00223290332879933\\
-19968.2578439751	0.00245609135167552\\
-19962.397753632	0.00265871513697177\\
-19956.537663289	0.00283598730003151\\
-19950.6775729459	0.00298371788102505\\
-19944.8174826029	0.00309841337376431\\
-19938.9573922598	0.00317735933591647\\
-19933.0973019167	0.00321868462698279\\
-19927.2372115737	0.00322140575420832\\
-19921.3771212306	0.00318545027524024\\
-19915.5170308876	0.00311165870084367\\
-19909.6569405445	0.00300176484823732\\
-19903.7968502014	0.00285835510397872\\
-19897.9367598584	0.00268480755391253\\
-19892.0766695153	0.00248521241217128\\
-19886.2165791723	0.00226427562413767\\
-19880.3564888292	0.00202720791431624\\
-19874.4963984861	0.00177960189582683\\
-19868.6363081431	0.00152730013897931\\
-19862.7762178	0.00127625731164266\\
-19856.916127457	0.00103239964388067\\
-19851.0560371139	0.000801485033775356\\
-19845.1959467708	0.000588967096438991\\
-19839.3358564278	0.000399866366243662\\
-19833.4757660847	0.000238651693966943\\
-19827.6156757417	0.000109134640544155\\
-19821.7555853986	1.43793630239751e-05\\
-19815.8954950555	-4.33698768218519e-05\\
-19810.0354047125	-6.27418663285009e-05\\
-19804.1753143694	-4.32709993287928e-05\\
-19798.3152240264	1.45915358610645e-05\\
-19792.4551336833	0.000109488206609079\\
-19786.5950433403	0.000239186982342586\\
-19780.7349529972	0.000400633830089154\\
-19774.8748626541	0.000590024624599525\\
-19769.0147723111	0.000802894779875644\\
-19763.154681968	0.00103422449039854\\
-19757.294591625	0.0012785571011974\\
-19751.4345012819	0.00153012781571654\\
-19745.5744109388	0.00178299970573255\\
-19739.7143205958	0.00203120381427884\\
-19733.8542302527	0.00226888004578474\\
-19727.9941399097	0.00249041551766442\\
-19722.1340495666	0.00269057710737819\\
-19716.2739592235	0.00286463506467592\\
-19710.4138688805	0.00300847476949458\\
-19704.5537785374	0.00311869399495326\\
-19698.6936881944	0.0031926833759882\\
-19692.8335978513	0.00322868818089843\\
-19686.9735075082	0.00322584992333993\\
-19681.1134171652	0.00318422682797309\\
-19675.2533268221	0.0031047926621266\\
-19669.3932364791	0.00298941395599161\\
-19663.533146136	0.00284080614424192\\
-19657.6730557929	0.00266246965982387\\
-19651.8129654499	0.00245860748341024\\
-19645.9528751068	0.00223402609081966\\
-19640.0927847638	0.00199402213284423\\
-19634.2326944207	0.00174425751862979\\
-19628.3726040776	0.0014906258491366\\
-19622.5125137346	0.00123911335097204\\
-19616.6524233915	0.000995657592319269\\
-19610.7923330485	0.000766007315066848\\
-19604.9322427054	0.000555586692197041\\
-19599.0721523623	0.00036936721547464\\
-19593.2120620193	0.000211750239203305\\
-19587.3519716762	8.64629550001872e-05\\
-19581.4918813332	-3.5297438650051e-06\\
-19575.6317909901	-5.60954245490624e-05\\
-19569.7717006471	-6.99846428289845e-05\\
-19563.911610304	-4.48606402078331e-05\\
-19558.0515199609	1.869249586499e-05\\
-19552.1914296179	0.000119183406251231\\
-19546.3313392748	0.00025424845348466\\
-19540.4712489318	0.000420707313197848\\
-19534.6111585887	0.000614637856179421\\
-19528.7510682456	0.000831468558037372\\
-19522.8909779026	0.00106608625917946\\
-19517.0308875595	0.00131295673521743\\
-19511.1707972165	0.00156625523458023\\
-19505.3107068734	0.00182000390460429\\
-19499.4506165303	0.00206821286369505\\
-19493.5905261873	0.00230502159057735\\
-19487.7304358442	0.00252483729308094\\
-19481.8703455012	0.00272246699002114\\
-19476.0102551581	0.00289324018648735\\
-19470.150164815	0.00303311924507743\\
-19464.290074472	0.00313879484447176\\
-19458.4299841289	0.00320776426904855\\
-19452.5698937859	0.00323839067706191\\
-19446.7098034428	0.00322994194408079\\
-19440.8497130997	0.00318260816004539\\
-19434.9896227567	0.00309749736159997\\
-19429.1295324136	0.00297660959547147\\
-19423.2694420706	0.00282278991991434\\
-19417.4093517275	0.002639661448472\\
-19411.5492613844	0.0024315400118711\\
-19405.6891710414	0.00220333244790632\\
-19399.8290806983	0.00196042091644812\\
-19393.9689903553	0.0017085359665373\\
-19388.1089000122	0.00145362134940467\\
-19382.2488096691	0.00120169376626511\\
-19376.3887193261	0.000958700860340368\\
-19370.528628983	0.000730380804708862\\
-19364.66853864	0.000522126800690162\\
-19358.8084482969	0.000338859686494413\\
-19352.9483579539	0.000184911665046173\\
-19347.0882676108	6.39238983051875e-05\\
-19341.2281772677	-2.12396113792094e-05\\
-19335.3680869247	-6.85598160528999e-05\\
-19329.5079965816	-7.69104883892955e-05\\
-19323.6479062386	-4.60850246296429e-05\\
-19317.7878158955	2.3198453580295e-05\\
-19311.9277255524	0.000129313824193651\\
-19306.0676352094	0.000269765127162093\\
-19300.2075448663	0.000441245268687923\\
-19294.3474545233	0.000639713886660138\\
-19288.4873641802	0.00086049254402621\\
-19282.6272738371	0.0010983750073472\\
-19276.7671834941	0.00134775001116313\\
-19270.907093151	0.00160273361346353\\
-19265.047002808	0.00185730802041086\\
-19259.1869124649	0.00210546360503902\\
-19253.3268221218	0.00234134076831363\\
-19247.4667317788	0.00255936829414784\\
-19241.6066414357	0.00275439493145204\\
-19235.7465510927	0.00292181109561528\\
-19229.8864607496	0.00305765781455272\\
-19224.0263704065	0.00315872034386454\\
-19218.1662800635	0.00322260423823502\\
-19212.3061897204	0.00324779207856016\\
-19206.4460993774	0.00323367951106862\\
-19200.5860090343	0.00318058974261593\\
-19194.7259186912	0.00308976614442579\\
-19188.8658283482	0.00296334313373728\\
-19183.0057380051	0.00280429601460806\\
-19177.1456476621	0.00261637095717588\\
-19171.285557319	0.00240399676273911\\
-19165.4254669759	0.00217218049308533\\
-19159.5653766329	0.0019263894235084\\
-19153.7052862898	0.00167242210244262\\
-19147.8451959468	0.0014162715587859\\
-19141.9851056037	0.00116398388359977\\
-19136.1250152607	0.00092151552327798\\
-19130.2649249176	0.000694592652493112\\
-19124.4048345745	0.000488575946729301\\
-19118.5447442315	0.000308333947919791\\
-19112.6846538884	0.000158128014557549\\
-19106.8245635454	4.15115748822309e-05\\
-19100.9644732023	-3.87539349948621e-05\\
-19095.1043828592	-8.07644595356537e-05\\
-19089.2442925162	-8.35184904910721e-05\\
-19083.3842021731	-4.69409405442995e-05\\
-19077.5241118301	2.81148451499311e-05\\
-19071.664021487	0.000139886992642582\\
-19065.8039311439	0.000285746453934804\\
-19059.9438408009	0.00046225884215683\\
-19054.0837504578	0.000665265292906495\\
-19048.2236601148	0.000889980447925988\\
-19042.3635697717	0.00113110525273996\\
-19036.5034794286	0.00138295190946436\\
-19030.6433890856	0.00163957803855868\\
-19024.7832987425	0.00189492688498233\\
-19018.9232083995	0.00214297026078808\\
-19013.0631180564	0.00237785085088631\\
-19007.2030277133	0.0025940205224798\\
-19001.3429373703	0.00278637137247749\\
-18995.4828470272	0.00295035641776533\\
-18989.6227566842	0.00308209707643156\\
-18983.7626663411	0.00317847489959295\\
-18977.902575998	0.00323720538448551\\
-18972.042485655	0.00325689212143291\\
-18966.1823953119	0.00323705999149077\\
-18960.3223049689	0.00317816662590684\\
-18954.4622146258	0.00308159185063536\\
-18948.6021242827	0.00294960535948395\\
-18942.7420339397	0.00278531337337056\\
-18936.8819435966	0.00259258553886001\\
-18931.0218532536	0.0023759637864148\\
-18925.1617629105	0.00214055529512317\\
-18919.3016725674	0.00189191208585699\\
-18913.4415822244	0.00163590008177391\\
-18907.5814918813	0.00137856072409014\\
-18901.7214015383	0.00112596840758451\\
-18895.8613111952	0.00088408709999427\\
-18890.0012208522	0.000658629529578646\\
-18884.1411305091	0.000454922265381188\\
-18878.281040166	0.00027777987656005\\
-18872.420949823	0.00013139114364504\\
-18866.5608594799	1.92200110015848e-05\\
-18860.7007691369	-5.60763775477899e-05\\
-18854.8406787938	-9.27106205544781e-05\\
-18848.9805884507	-8.98074873347542e-05\\
-18843.1204981077	-4.74248258504103e-05\\
-18837.2604077646	3.3447549295294e-05\\
-18831.4003174216	0.000150910968479387\\
-18825.5402270785	0.000302202478736994\\
-18819.6801367354	0.000483759829222655\\
-18813.8200463924	0.000691305342053608\\
-18807.9599560493	0.000919946693924104\\
-18802.0998657063	0.00116429223417216\\
-18796.2397753632	0.00141857812115274\\
-18790.3796850201	0.0016768042792222\\
-18784.5195946771	0.00193287596938492\\
-18778.659504334	0.00218074763341678\\
-18772.799413991	0.00241456561663699\\
-18766.9393236479	0.0026288064000839\\
-18761.0792333048	0.00281840707790009\\
-18755.2191429618	0.00297888499786085\\
-18749.3590526187	0.00310644373782046\\
-18743.4989622757	0.00319806291285394\\
-18737.6388719326	0.0032515696888833\\
-18731.7787815895	0.00326569030914807\\
-18725.9186912465	0.0032400804121195\\
-18720.0586009034	0.00317533341923403\\
-18714.1985105604	0.00307296678813353\\
-18708.3384202173	0.00293538644997722\\
-18702.4783298743	0.00276583026410804\\
-18696.6182395312	0.00256829181876632\\
-18690.7581491881	0.002347426371543\\
-18684.8980588451	0.00210844114424735\\
-18679.037968502	0.00185697255779655\\
-18673.177878159	0.00159895330114962\\
-18667.3177878159	0.00134047236945581\\
-18661.4576974728	0.00108763137293946\\
-18655.5976071298	0.000846400507645332\\
-18649.7375167867	0.000622477587468703\\
-18643.8774264437	0.000421153466158602\\
-18638.0173361006	0.000247187027568191\\
-18632.1572457575	0.000104692697937652\\
-18626.2971554145	-2.956861103588e-06\\
-18620.4370650714	-7.32105775005867e-05\\
-18614.5769747284	-0.000104399422478881\\
-18608.7168843853	-9.57760604676606e-05\\
-18602.8567940422	-4.75327533708477e-05\\
-18596.9967036992	3.92029096404471e-05\\
-18591.1366133561	0.000162394362353604\\
-18585.2765230131	0.000319143876072086\\
-18579.41643267	0.000505760715971023\\
-18573.5563423269	0.000717848036250807\\
-18567.6962519839	0.000950406468306214\\
-18561.8361616408	0.00119795196145568\\
-18555.9760712978	0.00145464509873329\\
-18550.1159809547	0.00171442883853689\\
-18544.2558906116	0.0019711714330792\\
-18538.3958002686	0.00221881115160648\\
-18532.5357099255	0.00245149939270799\\
-18526.6756195825	0.00266373880694001\\
-18520.8155292394	0.00285051316814547\\
-18514.9554388963	0.00300740592507665\\
-18509.0953485533	0.00313070463210016\\
-18503.2352582102	0.00321748878943333\\
-18497.3751678672	0.00326569901448781\\
-18491.5150775241	0.00327418590659551\\
-18485.654987181	0.00324273744517434\\
-18479.794896838	0.00317208426875\\
-18473.9348064949	0.00306388270339354\\
-18468.0747161519	0.00292067593503878\\
-18462.2146258088	0.00274583423561655\\
-18456.3545354658	0.00254347564764448\\
-18450.4944451227	0.00231836899430911\\
-18444.6343547796	0.00207582149875784\\
-18438.7742644366	0.00182155366113274\\
-18432.9141740935	0.00156156434323884\\
-18427.0540837505	0.00130198924221325\\
-18421.1939934074	0.00104895609258157\\
-18415.3339030643	0.000808440013231233\\
-18409.4738127213	0.000586122414200106\\
-18403.6137223782	0.000387256794770521\\
-18397.7536320352	0.000216544602743797\\
-18391.8935416921	7.80240883375843e-05\\
-18386.033451349	-2.50252198610244e-05\\
-18380.173361006	-9.01601573534626e-05\\
-18374.3132706629	-0.000115831846490102\\
-18368.4531803199	-0.000101422526372739\\
-18362.5930899768	-4.72604142253817e-05\\
-18356.7329996337	4.53877591612407e-05\\
-18350.8729092907	0.000174346370379179\\
-18345.0128189476	0.000336581988190763\\
-18339.1527286046	0.000528274722729892\\
-18333.2926382615	0.000744908160966237\\
-18327.4325479184	0.000981375771129694\\
-18321.5724575754	0.00123210126927024\\
-18315.7123672323	0.00149117011098425\\
-18309.8522768893	0.00175246900762923\\
-18303.9921865462	0.00200983017628503\\
-18298.1320962031	0.00225717691971971\\
-18292.2720058601	0.00248866710038391\\
-18286.411915517	0.00269883112095715\\
-18280.551825174	0.00288270115277156\\
-18274.6917348309	0.00303592855912138\\
-18268.8316444879	0.00315488673757002\\
-18262.9715541448	0.00323675694993573\\
-18257.1114638017	0.00327959510792965\\
-18251.2513734587	0.00328237793273344\\
-18245.3912831156	0.00324502739241868\\
-18239.5311927726	0.00316841283337592\\
-18233.6711024295	0.00305433074932621\\
-18227.8110120864	0.00290546265778688\\
-18221.9509217434	0.00272531207219984\\
-18216.0908314003	0.00251812205101473\\
-18210.2307410573	0.00228877526361677\\
-18204.3706507142	0.00204267892724433\\
-18198.5105603711	0.00178563732594442\\
-18192.6504700281	0.00152371491722304\\
-18186.790379685	0.00126309325460047\\
-18180.930289342	0.00100992510187001\\
-18175.0701989989	0.000770189181675641\\
-18169.2101086558	0.000549548987395443\\
-18163.3500183128	0.000353218992327754\\
-18157.4899279697	0.00018584141676491\\
-18151.6298376267	5.13764650934753e-05\\
-18145.7697472836	-4.69913718307363e-05\\
-18139.9096569405	-0.000106928732249591\\
-18134.0495665975	-0.000127008731112874\\
-18128.1894762544	-0.000106744926945176\\
-18122.3293859114	-4.6603099474696e-05\\
-18116.4692955683	5.20094469453597e-05\\
-18110.6092052252	0.000186776808663174\\
-18104.7491148822	0.000354528866579199\\
-18098.8890245391	0.000551315851514117\\
-18093.0289341961	0.000772501337282419\\
-18087.168843853	0.00101287147215145\\
-18081.3087535099	0.00126675787536547\\
-18075.4486631669	0.00152817130194673\\
-18069.5885728238	0.00179094292414862\\
-18063.7284824808	0.00204886989662629\\
-18057.8683921377	0.00229586177103383\\
-18052.0083017947	0.00252608430355498\\
-18046.1482114516	0.00273409726046745\\
-18040.2881211085	0.00291498296643236\\
-18034.4280307655	0.00306446255839037\\
-18028.5679404224	0.00317899719677166\\
-18022.7078500794	0.00325587184018219\\
-18016.8477597363	0.00329325960037216\\
-18010.9876693932	0.00329026515224825\\
-18005.1275790502	0.00324694616736504\\
-17999.2674887071	0.00316431225816866\\
-17993.4073983641	0.00304430145036805\\
-17987.547308021	0.00288973473190572\\
-17981.6872176779	0.00270424974416834\\
-17975.8271273349	0.0024922151740684\\
-17969.9670369918	0.00225862786163019\\
-17964.1069466488	0.00200899504622712\\
-17958.2468563057	0.00174920452650565\\
-17952.3867659626	0.00148538579471838\\
-17946.5266756196	0.00122376542092549\\
-17940.6665852765	0.000970520098580096\\
-17934.8064949335	0.000731630819955942\\
-17928.9464045904	0.000512741623865245\\
-17923.0863142473	0.000319026251571868\\
-17917.2262239043	0.000155065861107603\\
-17911.3661335612	2.47406902264884e-05\\
-17905.5060432182	-6.88617701286397e-05\\
-17899.6459528751	-0.000123519918815147\\
-17893.785862532	-0.000137930771273336\\
-17887.925772189	-0.000111741018777535\\
-17882.0656818459	-4.55556798860067e-05\\
-17876.2055915029	5.90758674657438e-05\\
-17870.3455011598	0.000199696150929326\\
-17864.4854108167	0.000372997317038402\\
-17858.6253204737	0.000574898937514028\\
-17852.7652301306	0.000800644078546268\\
-17846.9051397876	0.00104491137126863\\
-17841.0450494445	0.00130194044328263\\
-17835.1849591015	0.00156566775460915\\
-17829.3248687584	0.0018298696352209\\
-17823.4647784153	0.00208830914989589\\
-17817.6046880723	0.00233488332481833\\
-17811.7445977292	0.00256376726082426\\
-17805.8845073862	0.00276955173020013\\
-17800.0244170431	0.0029473710069704\\
-17794.1643267	0.00309301790968213\\
-17788.304236357	0.00320304333721315\\
-17782.4441460139	0.00327483794213662\\
-17776.5840556709	0.00330669400845668\\
-17770.7239653278	0.00329784606602108\\
-17764.8638749847	0.0032484892758537\\
-17759.0037846417	0.0031597751448728\\
-17753.1436942986	0.00303378466396183\\
-17747.2836039556	0.00287347949522077\\
-17741.4235136125	0.00268263235409453\\
-17735.5634232694	0.00246573822198747\\
-17729.7033329264	0.00222790847940802\\
-17723.8432425833	0.00197475045254338\\
-17717.9831522403	0.00171223521203553\\
-17712.1230618972	0.00144655674036633\\
-17706.2629715541	0.00118398578974072\\
-17700.4028812111	0.00093072187814373\\
-17694.542790868	0.000692746916920503\\
-17688.682700525	0.000475683925448585\\
-17682.8226101819	0.000284664169936385\\
-17676.9625198388	0.000124205865406549\\
-17671.1024294958	-1.89269188114981e-06\\
-17665.2423391527	-9.06430339274512e-05\\
-17659.3822488097	-0.000139937344294651\\
-17653.5221584666	-0.000148598516845465\\
-17647.6620681235	-0.000116408261157179\\
-17641.8019777805	-4.41125836659883e-05\\
-17635.9418874374	6.65954925906442e-05\\
-17630.0817970944	0.000213115569519962\\
-17624.2217067513	0.000392000948702619\\
-17618.3616164083	0.00059903970494476\\
-17612.5015260652	0.000829353851739215\\
-17606.6414357221	0.00107751426396015\\
-17600.7813453791	0.00133766865039903\\
-17594.921255036	0.0016036795596945\\
-17589.061164693	0.00186926916557996\\
-17583.2010743499	0.00212816741562344\\
-17577.3409840068	0.00237426004794859\\
-17571.4808936638	0.00260173298152088\\
-17565.6208033207	0.0028052096702421\\
-17559.7607129777	0.00297987817644177\\
-17553.9006226346	0.00312160496001904\\
-17548.0405322915	0.00322703269307582\\
-17542.1804419485	0.00329365978488761\\
-17536.3203516054	0.00331989973423141\\
-17530.4602612624	0.00330511890044406\\
-17524.6001709193	0.00324965179432829\\
-17518.7400805762	0.00315479352014854\\
-17512.8799902332	0.00302276953889241\\
-17507.0198998901	0.00285668345889799\\
-17501.1598095471	0.0026604440784261\\
-17495.299719204	0.00243867339507389\\
-17489.4396288609	0.0021965977470681\\
-17483.5795385179	0.0019399246500727\\
-17477.7194481748	0.00167470823173893\\
-17471.8593578318	0.00140720643686414\\
-17465.9992674887	0.00114373337056256\\
-17460.1391771456	0.000890510263788304\\
-17454.2790868026	0.000653518578438651\\
-17448.4189964595	0.00043835872060944\\
-17442.5589061165	0.000250117699026392\\
-17436.6988157734	9.32488559845947e-05\\
-17430.8387254303	-2.85334859471857e-05\\
-17424.9786350873	-0.00011234196916861\\
-17419.1185447442	-0.000156184656030153\\
-17413.2584544012	-0.000159012370673364\\
-17407.3983640581	-0.00012074380267877\\
-17401.5382737151	-4.22677719732014e-05\\
-17395.678183372	7.45774065927705e-05\\
-17389.8180930289	0.000227046980089\\
-17383.9580026859	0.000411554227368581\\
-17378.0979123428	0.000623754827763278\\
-17372.2378219998	0.000858649144152956\\
-17366.3777316567	0.00111070001227798\\
-17360.5176413136	0.00137396326148658\\
-17354.6575509706	0.00164222789021971\\
-17348.7974606275	0.00190916259096835\\
-17342.9373702845	0.00216846516790967\\
-17337.0772799414	0.00241401132129086\\
-17331.2171895983	0.00263999928608769\\
-17325.3570992553	0.0028410869089278\\
-17319.4970089122	0.00301251792512637\\
-17313.6369185692	0.00315023445060586\\
-17307.7768282261	0.00325097302838986\\
-17301.916737883	0.00331234195655322\\
-17296.05664754	0.0033328780652403\\
-17290.1965571969	0.00331208159513595\\
-17284.3364668539	0.00325042834598547\\
-17278.4763765108	0.00314935880011472\\
-17272.6162861677	0.00301124446952437\\
-17266.7561958247	0.00283933225223779\\
-17260.8961054816	0.00263766810384271\\
-17255.0360151386	0.00241100181814975\\
-17249.1759247955	0.00216467515785807\\
-17243.3158344524	0.00190449597017936\\
-17237.4557441094	0.00163660125346903\\
-17231.5956537663	0.00136731240367987\\
-17225.7355634233	0.00110298605450711\\
-17219.8754730802	0.000849864030840129\\
-17214.0153827372	0.000613925957206676\\
-17208.1552923941	0.000400748001395787\\
-17202.295202051	0.000215371090139906\\
-17196.435111708	6.21817112289388e-05\\
-17190.5750213649	-5.51918795326886e-05\\
-17184.7149310219	-0.000133965590637644\\
-17178.8548406788	-0.00017226553134028\\
-17172.9947503357	-0.000169172586033585\\
-17167.1346599927	-0.000124744466343105\\
-17161.2745696496	-4.00147120079053e-05\\
-17155.4144793066	8.30313444355445e-05\\
-17149.5543889635	0.000241503090366643\\
-17143.6942986204	0.000431672533579973\\
-17137.8342082774	0.000649061995683265\\
-17131.9741179343	0.000888549535702041\\
-17126.1140275913	0.00114448962183809\\
-17120.2539372482	0.00141084620853153\\
-17114.3938469051	0.00168133508208871\\
-17108.5337565621	0.00194957211792032\\
-17102.673666219	0.00220922395216201\\
-17096.813575876	0.00245415751154037\\
-17090.9534855329	0.00267858487124884\\
-17085.0933951898	0.00287720001974668\\
-17079.2333048468	0.00304530429881759\\
-17073.3732145037	0.00317891755331364\\
-17067.5131241607	0.0032748723616779\\
-17061.6530338176	0.0033308891162053\\
-17055.7929434745	0.00334563017360636\\
-17049.9328531315	0.00331873178943838\\
-17044.0727627884	0.00325081307430903\\
-17038.2126724454	0.00314346175192286\\
-17032.3525821023	0.00299919704571256\\
-17026.4924917592	0.00282141056221078\\
-17020.6324014162	0.00261428655787328\\
-17014.7723110731	0.00238270346375212\\
-17008.9122207301	0.0021321189855739\\
-17003.052130387	0.00186844148536767\\
-16997.1920400439	0.00159789067534705\\
-16991.3319497009	0.00132685090883612\\
-16985.4718593578	0.0010617205281706\\
-16979.6117690148	0.000808760824820732\\
-16973.7516786717	0.000573948176819946\\
-16967.8915883287	0.000362832855259908\\
-16962.0314979856	0.000180407835448466\\
-16956.1714076425	3.09907134717422e-05\\
-16950.3113172995	-8.18784793205741e-05\\
-16944.4512269564	-0.000155521145550106\\
-16938.5911366134	-0.000188183687877569\\
-16932.7310462703	-0.000179079263506846\\
-16926.8709559272	-0.000128406733021716\\
-16921.0108655842	-3.73463474530113e-05\\
-16915.1507752411	9.19677336562851e-05\\
-16909.2906848981	0.000256497453378197\\
-16903.430594555	0.000452372225917957\\
-16897.5705042119	0.000674979986037294\\
-16891.7104138689	0.000919075777729024\\
-16885.8503235258	0.0011789053255574\\
-16879.9902331828	0.0014483406774797\\
-16874.1301428397	0.00172102472175951\\
-16868.2700524966	0.00199052117006053\\
-16862.4099621536	0.00225046646811791\\
-16856.5498718105	0.00249472004890933\\
-16850.6897814675	0.0027175093805363\\
-16844.8296911244	0.00291356638295905\\
-16838.9696007813	0.00307825198979473\\
-16833.1095104383	0.00320766590989548\\
-16827.2494200952	0.00329873899231712\\
-16821.3893297522	0.00334930600689151\\
-16815.5292394091	0.00335815711499183\\
-16809.669149066	0.00332506680697236\\
-16803.809058723	0.00325079961408492\\
-16797.9489683799	0.00313709245131778\\
-16792.0888780369	0.00298661399816999\\
-16786.2287876938	0.00280290206770192\\
-16780.3686973508	0.00259028043329555\\
-16774.5086070077	0.00235375706848653\\
-16768.6485166646	0.00209890619477049\\
-16762.7884263216	0.00183173691520315\\
-16756.9283359785	0.00155855152981105\\
-16751.0682456355	0.00128579687304424\\
-16745.2081552924	0.00101991218018234\\
-16739.3480649493	0.000767177072449169\\
-16733.4879746063	0.00053356324939121\\
-16727.6278842632	0.000324593391174183\\
-16721.7677939202	0.000145210604301558\\
-16715.9077035771	-3.38503015062493e-07\\
-16710.047613234	-0.000108604350224984\\
-16704.187522891	-0.000177016138859087\\
-16698.3274325479	-0.00020394289452982\\
-16692.4673422049	-0.000188732347219367\\
-16686.6072518618	-0.000131726723126663\\
-16680.7471615187	-3.42550660060854e-05\\
-16674.8870711757	0.000101397740207191\\
-16669.0269808326	0.000272044525574439\\
-16663.1668904896	0.000473670710044766\\
-16657.3068001465	0.000701528742090925\\
-16651.4467098034	0.000950249878674834\\
-16645.5866194604	0.00121397067475159\\
-16639.7265291173	0.00148647120241221\\
-16633.8664387743	0.0017613217414424\\
-16628.0063484312	0.00203203448196694\\
-16622.1462580881	0.00229221665998899\\
-16616.2861677451	0.00253572151148781\\
-16610.426077402	0.00275679348066917\\
-16604.565987059	0.00295020425211911\\
-16598.7058967159	0.00311137639197449\\
-16592.8458063728	0.00323649167421322\\
-16586.9857160298	0.00332258152877746\\
-16581.1256256867	0.00336759746885296\\
-16575.2655353437	0.00337045982670993\\
-16569.4054450006	0.00333108363872912\\
-16563.5453546575	0.00325038105919412\\
-16557.6852643145	0.00313024023616807\\
-16551.8251739714	0.00297348113850494\\
-16545.9650836284	0.00278378936732268\\
-16540.1049932853	0.00256562950545911\\
-16534.2449029423	0.00232414004163137\\
-16528.3848125992	0.00206501234264672\\
-16522.5247222561	0.00179435652390311\\
-16516.6646319131	0.0015185573789269\\
-16510.80454157	0.0012441237653532\\
-16504.944451227	0.000977534999517035\\
-16499.0843608839	0.000725087884969616\\
-16493.2242705408	0.000492747986061979\\
-16487.3641801978	0.000286008659417088\\
-16481.5040898547	0.000109761174163993\\
-16475.6439995117	-3.18210083328037e-05\\
-16469.7839091686	-0.000135381057646192\\
-16463.9238188255	-0.000198458360456623\\
-16458.0637284825	-0.000219546982932227\\
-16452.2036381394	-0.000198131620406222\\
-16446.3435477964	-0.000134700176321897\\
-16440.4834574533	-3.07326637172032e-05\\
-16434.6233671102	0.000111333318655286\\
-16428.7632767672	0.000288159730377807\\
-16422.9031864241	0.000495586514094176\\
-16417.0430960811	0.000728729458506226\\
-16411.183005738	0.000982095197569767\\
-16405.3229153949	0.00124971063835898\\
-16399.4628250519	0.0015252637683092\\
-16393.6027347088	0.00180225252264397\\
-16387.7426443658	0.00207413820114466\\
-16381.8825540227	0.00233449981442621\\
-16376.0224636796	0.00257718571678646\\
-16370.1623733366	0.00279645894443526\\
-16364.3022829935	0.00298713282620357\\
-16358.4421926505	0.00314469366049456\\
-16352.5821023074	0.00326540755775076\\
-16346.7220119643	0.0033464089189049\\
-16340.8619216213	0.00338576845392555\\
-16335.0018312782	0.00338253912540536\\
-16329.1417409352	0.00333677892398197\\
-16323.2816505921	0.00324954992734402\\
-16317.4215602491	0.00312289365547601\\
-16311.561469906	0.00295978329357303\\
-16305.7013795629	0.0027640539006836\\
-16299.8412892199	0.00254031224206561\\
-16293.9811988768	0.00229382836536761\\
-16288.1211085338	0.0020304114721701\\
-16282.2610181907	0.00175627300844687\\
-16276.4009278476	0.00147788020042007\\
-16270.5408375046	0.00120180348947253\\
-16264.6807471615	0.000934561464678312\\
-16258.8206568185	0.000682466952817851\\
-16252.9605664754	0.000451477899609607\\
-16247.1004761323	0.000247056564347153\\
-16241.2403857893	7.40403555731246e-05\\
-16235.3802954462	-6.34726380712965e-05\\
-16229.5202051032	-0.000162220713228812\\
-16223.6601147601	-0.000219855914514827\\
-16217.800024417	-0.000234999859707407\\
-16211.939934074	-0.000207276700249463\\
-16206.0798437309	-0.000137322429085251\\
-16200.2197533879	-2.67703058012217e-05\\
-16194.3596630448	0.000121787267190864\\
-16188.4995727017	0.000304859527714513\\
-16182.6394823587	0.000518139371084741\\
-16176.7793920156	0.000756604674649038\\
-16170.9193016726	0.0010146365460701\\
-16165.0592113295	0.00128615171126053\\
-16159.1991209864	0.00156474592294067\\
-16153.3390306434	0.00184384500913134\\
-16147.4789403003	0.00211685999924417\\
-16141.6188499573	0.00237734266739566\\
-16135.7587596142	0.00261913782143336\\
-16129.8986692712	0.00283652874096806\\
-16124.0385789281	0.00302437232816796\\
-16118.178488585	0.0031782207765661\\
-16112.318398242	0.00329442687901623\\
-16106.4583078989	0.00337023048262889\\
-16100.5982175559	0.0034038240405048\\
-16094.7381272128	0.00339439570411013\\
-16088.8780368697	0.00334214892908698\\
-16083.0179465267	0.00324829812088709\\
-16077.1578561836	0.00311504041315398\\
-16071.2977658406	0.00294550423353621\\
-16065.4376754975	0.00274367586196693\\
-16059.5775851544	0.00251430570430218\\
-16053.7174948114	0.00226279648579603\\
-16047.8574044683	0.00199507599508508\\
-16041.9973141253	0.00171745737655641\\
-16036.1372237822	0.00143649026307191\\
-16030.2771334391	0.00115880625959995\\
-16024.4170430961	0.000890962422844709\\
-16018.556952753	0.000639286430811113\\
-16012.69686241	0.000409727098285845\\
-16006.8367720669	0.000207713769378205\\
-16000.9766817238	3.80279104543498e-05\\
-15995.1165913808	-9.53100597599009e-05\\
-15989.2565010377	-0.000189136024533497\\
-15983.3964106947	-0.000241217251211042\\
-15977.5363203516	-0.000250305519499976\\
-15971.6762300085	-0.000216167031927556\\
-15965.8161396655	-0.000139588389904397\\
-15959.9560493224	-2.23584835662776e-05\\
-15954.0959589794	0.000132773287949975\\
-15948.2358686363	0.000322161490170081\\
-15942.3757782932	0.000541350309089475\\
-15936.5156879502	0.000785178376656384\\
-15930.6555976071	0.00104790029991056\\
-15924.7955072641	0.00132332203250974\\
-15918.935416921	0.0016049468990927\\
-15913.0753265779	0.00188612883015765\\
-15907.2152362349	0.00216022919334625\\
-15901.3551458918	0.00242077352038266\\
-15895.4950555488	0.00266160442981069\\
-15889.6349652057	0.00287702713371492\\
-15883.7748748627	0.0030619440901945\\
-15877.9147845196	0.00321197561753751\\
-15872.0546941765	0.00332356361703518\\
-15866.1946038335	0.00339405594718373\\
-15860.3345134904	0.0034217694496319\\
-15854.4744231474	0.00340603012836392\\
-15848.6143328043	0.00334718952380148\\
-15842.7542424612	0.00324661688391024\\
-15836.8941521182	0.00310666730637738\\
-15831.0340617751	0.00293062659267823\\
-15825.1739714321	0.00272263410518938\\
-15819.313881089	0.00248758543855506\\
-15813.4537907459	0.00223101719329131\\
-15807.5937004029	0.00195897656392919\\
-15801.7336100598	0.00167787881306794\\
-15795.8735197168	0.00139435599053467\\
-15790.0134293737	0.00111510046472146\\
-15784.1533390306	0.000846706957782417\\
-15778.2932486876	0.000595516812678381\\
-15772.4331583445	0.000367468169909713\\
-15766.5730680015	0.000167955593274787\\
-15760.7129776584	1.70246302675327e-06\\
-15754.8528873153	-0.000127350845271362\\
-15748.9927969723	-0.00021614034906893\\
-15743.1327066292	-0.000262551201128199\\
-15737.2726162862	-0.000265468058943033\\
-15731.4125259431	-0.000224801881802328\\
-15725.5524356	-0.000141492511864545\\
-15719.692345257	-1.74869670384477e-05\\
-15713.8322549139	0.000144306053242435\\
-15707.9721645709	0.000340084386486619\\
-15702.1120742278	0.000565241750036063\\
-15696.2519838848	0.000814476109168386\\
-15690.3918935417	0.00108191452100489\\
-15684.5318031986	0.00136125151483173\\
-15678.6717128556	0.00164589774830985\\
-15672.8116225125	0.00192913543522503\\
-15666.9515321695	0.00220427687848571\\
-15661.0914418264	0.0024648223675902\\
-15655.2313514833	0.00270461371255668\\
-15649.3712611403	0.00291797978783883\\
-15643.5111707972	0.00309987064662115\\
-15637.6510804542	0.00324597703373892\\
-15631.7909901111	0.00335283246943439\\
-15625.930899768	0.00341789548531318\\
-15620.070809425	0.00343961006209267\\
-15614.2107190819	0.00341744283179781\\
-15608.3506287389	0.00335189615523155\\
-15602.4905383958	0.0032444967545569\\
-15596.6304480527	0.0030977601577299\\
-15590.7703577097	0.00291513178258332\\
-15584.9102673666	0.00270090604026635\\
-15579.0501770236	0.00246012535751445\\
-15573.1900866805	0.00219846149160917\\
-15567.3299963374	0.00192208193174272\\
-15561.4699059944	0.00163750453345362\\
-15555.6098156513	0.00135144381209565\\
-15549.7497253083	0.00107065252003196\\
-15543.8896349652	0.000801762245226182\\
-15538.0295446221	0.000551126793784171\\
-15532.1694542791	0.000324672055058073\\
-15526.309363936	0.000127755896776441\\
-15520.449273593	-3.49585975311358e-05\\
-15514.5891832499	-0.000159613549862391\\
-15508.7290929068	-0.000243247753188978\\
-15502.8690025638	-0.000283867012633818\\
-15497.0089122207	-0.000280491691646587\\
-15491.1488218777	-0.000233180329688468\\
-15485.2887315346	-0.000143028762349835\\
-15479.4286411916	-1.21447528169376e-05\\
-15473.5685508485	0.000156401278303213\\
-15467.7084605054	0.000358648273218747\\
-15461.8483701624	0.000589837618094773\\
-15455.9882798193	0.000844525097901928\\
-15450.1281894763	0.00111670909117421\\
-15444.2680991332	0.00139997198631395\\
-15438.4080087901	0.00168763148727556\\
-15432.5479184471	0.00197289824158901\\
-15426.687828104	0.00224903607266765\\
-15420.827737761	0.0025095210349352\\
-15414.9676474179	0.0027481955362655\\
-15409.1075570748	0.00295941388710672\\
-15403.2474667318	0.00313817583574399\\
-15397.3873763887	0.00328024493180674\\
-15391.5272860457	0.00338224891582324\\
-15385.6671957026	0.00344175975666725\\
-15379.8071053595	0.00345735143676508\\
-15373.9470150165	0.00342863411070367\\
-15368.0869246734	0.00335626381843553\\
-15362.2268343304	0.00324192751253394\\
-15356.3667439873	0.00308830374023165\\
-15350.5066536442	0.00289899989653999\\
-15344.6465633012	0.00267846751867429\\
-15338.7864729581	0.00243189760977451\\
-15332.9263826151	0.00216509845392739\\
-15327.066292272	0.00188435879806841\\
-15321.2062019289	0.00159629962321549\\
-15315.3461115859	0.00130771799906446\\
-15309.4860212428	0.00102542670394675\\
-15303.6259308998	0.000756093394389925\\
-15297.7658405567	0.000506083120695707\\
-15291.9057502136	0.000281307908173864\\
-15286.0456598706	8.7086958430274e-05\\
-15280.1855695275	-7.19792236303063e-05\\
-15274.3254791845	-0.000192117798626628\\
-15268.4653888414	-0.000270473076442667\\
-15262.6052984984	-0.000305174392639081\\
-15256.7452081553	-0.00029538076439031\\
-15250.8851178122	-0.000241301260081821\\
-15245.0250274692	-0.000144190589545165\\
-15239.1649371261	-6.32000663148055e-06\\
-15233.3048467831	0.000169075801332674\\
-15227.44475644	0.000377874595476299\\
-15221.5846660969	0.000615163458716024\\
-15215.7245757539	0.000875354384118947\\
-15209.8644854108	0.00115231585894787\\
-15204.0043950678	0.00143951734607849\\
-15198.1443047247	0.00173018325839858\\
-15192.2842143816	0.00201745279600168\\
-15186.4241240386	0.00229454187596013\\
-15180.5640336955	0.00255490333257139\\
-15174.7039433525	0.00279238160564078\\
-15168.8438530094	0.00300135826212091\\
-15162.9837626663	0.00317688491076186\\
-15157.1236723233	0.00331480036596573\\
-15151.2635819802	0.00341182928680324\\
-15145.4034916372	0.00346565995275887\\
-15139.5434012941	0.00347499933022859\\
-15133.683310951	0.00343960411759045\\
-15127.823220608	0.00336028702397862\\
-15121.9631302649	0.00323889812103693\\
-15116.1030399219	0.00307828169475268\\
-15110.2429495788	0.00288220960418514\\
-15104.3828592357	0.00265529270753584\\
-15098.5227688927	0.00240287243599037\\
-15092.6626785496	0.00213089506466031\\
-15086.8025882066	0.00184577163966613\\
-15080.9424978635	0.00155422686134492\\
-15075.0824075204	0.00126314048489069\\
-15069.2223171774	0.000979384979148323\\
-15063.3622268343	0.000709663273851925\\
-15057.5021364913	0.000460350426003838\\
-15051.6420461482	0.000237342945100516\\
-15045.7819558052	4.59193383597596e-05\\
-15039.9218654621	-0.000109384826114604\\
-15034.061775119	-0.000224884381242437\\
-15028.201684776	-0.000297832002020913\\
-15022.3415944329	-0.000326483551076439\\
-15016.4815040899	-0.0003101397746323\\
-15010.6214137468	-0.000249163352292922\\
-15004.7613234037	-0.000144970885391611\\
nan	nan\\
-14993.0411427176	0.000182347671641106\\
-14987.1810523746	0.000397786297766314\\
-14981.3209620315	0.00064124656956867\\
-14975.4608716884	0.000906994972593629\\
-14969.6007813454	0.00118876880093558\\
-14963.7406910023	0.0014799237351317\\
-14957.8806006593	0.00177359050624812\\
-14952.0205103162	0.00206283695243711\\
-14946.1604199731	0.00234083164499458\\
-14940.3003296301	0.00260100522197361\\
-14934.440239287	0.00283720561916546\\
-14928.580148944	0.00304384353095849\\
-14922.7200586009	0.00321602466163508\\
-14916.8599682578	0.00334966563817136\\
-14910.9998779148	0.00344159083945087\\
-14905.1397875717	0.00348960784607493\\
-14899.2796972287	0.00349255971797196\\
-14893.4196068856	0.00345035285387688\\
-14887.5595165425	0.00336395976184966\\
-14881.6994261995	0.00323539666232488\\
-14875.8393358564	0.00306767643896439\\
-14869.9792455134	0.00286473803519307\\
-14864.1191551703	0.00263135395085735\\
-14858.2590648272	0.00237301801082623\\
-14852.3989744842	0.00209581604510282\\
-14846.5388841411	0.0018062825240459\\
-14840.6787937981	0.00151124652591381\\
-14834.818703455	0.00121767066731926\\
-14828.958613112	0.00093248679559564\\
-14823.0985227689	0.000662432320091626\\
-14817.2384324258	0.000413891046680078\\
-14811.3783420828	0.000192742275460329\\
-14805.5182517397	4.22172853864268e-06\\
-14799.6581613967	-0.000147202402636761\\
-14793.7980710536	-0.000257935356010565\\
-14787.9379807105	-0.000325341134040087\\
-14782.0778903675	-0.000347805249640168\\
-14776.2178000244	-0.000324773389527042\\
-14770.3577096814	-0.000256765069308663\\
-14764.4976193383	-0.000145361944579666\\
-14758.6375289952	6.82895969455063e-06\\
-14752.7774386522	0.000196236246832203\\
-14746.9173483091	0.000418407946166413\\
-14741.0572579661	0.000668116144800087\\
-14735.197167623	0.00093947999445889\\
-14729.3370772799	0.00122610419944735\\
-14723.4769869369	0.00152122972457019\\
-14717.6168965938	0.00181789317173546\\
-14711.7568062508	0.00210909106740961\\
-14705.8967159077	0.00238794518504862\\
-14700.0366255646	0.00264786499999256\\
-14694.1765352216	0.00288270344049986\\
-14688.3164448785	0.00308690225354372\\
-14682.4563545355	0.00325562354887474\\
-14676.5962641924	0.00338486440746752\\
-14670.7361738493	0.00347155184018167\\
-14664.8760835063	0.00351361584370634\\
-14659.0159931632	0.00351003881731542\\
-14653.1559028202	0.00346088016106414\\
-14647.2958124771	0.00336727546096407\\
-14641.4357221341	0.00323141026656279\\
-14635.575631791	0.00305646906617973\\
-14629.7155414479	0.00284656065096442\\
-14623.8554511049	0.00260662161640171\\
-14617.9953607618	0.002342300268085\\
-14612.1352704188	0.00205982366122552\\
-14606.2751800757	0.00176585090394756\\
-14600.4150897326	0.00146731617983447\\
-14594.5549993896	0.00117126519018534\\
-14588.6949090465	0.00088468887352642\\
-14582.8348187035	0.000614358326495787\\
-14576.9747283604	0.000366664823968551\\
-14571.1146380173	0.000147468718030301\\
-14565.2545476743	-3.80392120851743e-05\\
-14559.3944573312	-0.000185460678968508\\
-14553.5343669882	-0.000291294164323109\\
-14547.6742766451	-0.000353018082501412\\
-14541.814186302	-0.000369150855255968\\
-14535.954095959	-0.000339286466673613\\
-14530.0940056159	-0.000264104645316144\\
-14524.2339152729	-0.000145355419131996\\
-14518.3738249298	1.4181606485196e-05\\
-14512.5137345867	0.000210762300133952\\
-14506.6536442437	0.00043976586314819\\
-14500.7935539006	0.000695803434123047\\
-14494.9334635576	0.000972844886872625\\
-14489.0733732145	0.00126436083839635\\
-14483.2132828714	0.0015634765229513\\
-14477.3531925284	0.0018631339060022\\
-14471.4931021853	0.00215625821548399\\
-14465.6330118423	0.00243592496150328\\
-14459.7729214992	0.00269552350139673\\
-14453.9128311561	0.00292891328709512\\
-14448.0527408131	0.00313056910176388\\
-14442.19265047	0.00329571185067635\\
-14436.332560127	0.00342042181080717\\
-14430.4724697839	0.00350173165535322\\
-14424.6123794408	0.0035376970457357\\
-14418.7522890978	0.00352744311202012\\
-14412.8921987547	0.0034711857108244\\
-14407.0321084117	0.00337022694442108\\
-14401.1720180686	0.00322692503256452\\
-14395.3119277256	0.00304463923383036\\
-14389.4518373825	0.00282765110239682\\
-14383.5917470394	0.00258106392615789\\
-14377.7316566964	0.00231068270777798\\
-14371.8715663533	0.00202287751120831\\
-14366.0114760103	0.00172443339015354\\
-14360.1513856672	0.00142239043416452\\
-14354.2912953241	0.00112387770259079\\
-14348.4312049811	0.000835944963785444\\
-14342.571114638	0.00056539621042337\\
-14336.711024295	0.000318628882536111\\
-14330.8509339519	0.000101482597034424\\
-14324.9908436088	-8.08990394676799e-05\\
-14319.1307532658	-0.000224190264832648\\
-14313.2706629227	-0.000324985756845948\\
-14307.4105725797	-0.000380881556884313\\
-14301.5504822366	-0.000390532398900328\\
-14295.6903918935	-0.000353684076786514\\
-14289.8303015505	-0.000271180071694326\\
-14283.9702112074	-0.000144942268043085\\
-14278.1101208644	2.20738013458808e-05\\
-14272.2500305213	0.000225948139063632\\
-14266.3899401782	0.000461888276574824\\
-14260.5298498352	0.000724341918675397\\
-14254.6697594921	0.00100712759141242\\
-14248.8096691491	0.00130358021957292\\
-14242.949578806	0.00160670820536757\\
-14237.0894884629	0.00190935830668996\\
-14231.2293981199	0.0022043844268812\\
-14225.3693077768	0.00248481633338656\\
-14219.5092174338	0.00274402432222415\\
-14213.6491270907	0.00297587593821246\\
-14207.7890367477	0.00317488104669863\\
-14201.9289464046	0.00333632182494585\\
-14196.0688560615	0.00345636459554653\\
-14190.2087657185	0.00353215085120681\\
-14184.3486753754	0.00356186530974095\\
-14178.4885850324	0.00354477937927039\\
-14172.6284946893	0.00348126899325907\\
-14166.7684043462	0.00337280637915283\\
-14160.9083140032	0.00322192594002392\\
-14155.0482236601	0.00303216503936423\\
-14149.1881333171	0.00280798107254935\\
-14143.328042974	0.00255464676876176\\
-14137.4679526309	0.00227812618235113\\
-14131.6078622879	0.00198493429025324\\
-14125.7477719448	0.00168198350024101\\
-14119.8876816018	0.0013764206862526\\
-14114.0275912587	0.00107545859238555\\
-14108.1675009156	0.000786205582809815\\
-14102.3074105726	0.000515497755686738\\
-14096.4473202295	0.000269737386289097\\
-14090.5872298865	5.47415169383387e-05\\
-14084.7271395434	-0.000124395725499868\\
-14078.8670492003	-0.000263423826120395\\
-14073.0069588573	-0.000359036732912303\\
-14067.1468685142	-0.000408951469478573\\
-14061.2867781712	-0.000411962640454991\\
-14055.4266878281	-0.000367971528561726\\
-14049.566597485	-0.000277989081314912\\
-14043.706507142	-0.000144112701387935\\
-14037.8464167989	3.05226269723719e-05\\
-14031.9863264559	0.000241817736857821\\
-14026.1262361128	0.000484805484690877\\
-14020.2661457697	0.000753767505555156\\
-14014.4060554267	0.00104236877370208\\
-14008.5459650836	0.00134380680201195\\
-14002.6858747406	0.00165097196672921\\
-13996.8257843975	0.00195661517922525\\
-13990.9656940544	0.0022535189504021\\
-13985.1056037114	0.00253466781147139\\
-13979.2455133683	0.0027934140670323\\
-13973.3854230253	0.00302363496483255\\
-13967.5253326822	0.00321987756595162\\
-13961.6652423392	0.00337748788853582\\
-13955.8051519961	0.00349272126603635\\
-13949.945061653	0.00356283130393929\\
-13944.08497131	0.00358613532134955\\
-13938.2248809669	0.00356205471872728\\
-13932.3647906239	0.00349112930359026\\
-13926.5047002808	0.00337500522006096\\
-13920.6446099377	0.00321639675151773\\
-13914.7845195947	0.00301902288247949\\
-13908.9244292516	0.00278752010166614\\
-13903.0643389086	0.00252733349123196\\
-13897.2042485655	0.00224458865951919\\
-13891.3441582224	0.00194594752989359\\
-13885.4840678794	0.00163845137996679\\
-13879.6239775363	0.00132935482955634\\
-13873.7638871933	0.00102595469092932\\
-13867.9037968502	0.000735417718974825\\
-13862.0437065071	0.000464611327181464\\
-13856.1836161641	0.000219941267876601\\
-13850.323525821	7.20011301820675e-06\\
-13844.463435478	-0.000168569880621201\\
-13838.6033451349	-0.000303196275538598\\
-13832.7432547918	-0.000393475494789959\\
-13826.8831644488	-0.000437249049686191\\
-13821.0230741057	-0.000433455140392547\\
-13815.1629837627	-0.000382154396046635\\
-13809.3028934196	-0.000284529130962334\\
-13803.4428030765	-0.000142856118191634\\
-13797.5827127335	3.9546492979655e-05\\
-13791.7226223904	0.000258396878264007\\
-13785.8625320474	0.000508550039020138\\
-13780.0024417043	0.00078411874361199\\
-13774.1423513612	0.0010786120667953\\
-13768.2822610182	0.00138508826722323\\
-13762.4221706751	0.0016963184025553\\
-13756.5620803321	0.00200495682623152\\
-13750.701989989	0.00230371454466459\\
-13744.841899646	0.00258553134405858\\
-13738.9818093029	0.00284374262236742\\
-13733.1217189598	0.00307223698416493\\
-13727.2616286168	0.00326560087258786\\
-13721.4015382737	0.0034192468150607\\
-13715.5414479307	0.00352952224543625\\
-13709.6813575876	0.00359379632114949\\
-13703.8212672445	0.00361052267204781\\
-13697.9611769015	0.00357927658441986\\
-13692.1010865584	0.0035007657267381\\
-13686.2409962154	0.00337681414709833\\
-13680.3809058723	0.00321031990360969\\
-13674.5208155292	0.00300518731189572\\
-13668.6607251862	0.00276623539278202\\
-13662.8006348431	0.002499084667487\\
-13656.9405445001	0.00221002495888296\\
-13651.080454157	0.00190586730827287\\
-13645.2203638139	0.00159378349377976\\
-13639.3602734709	0.00128113693133899\\
-13633.5001831278	0.000975308945195455\\
-13627.6400927848	0.000683524506567394\\
-13621.7800024417	0.000412681554150447\\
-13615.9199120986	0.000169187928476629\\
-13610.0598217556	-4.11902254466342e-05\\
-13604.1997314125	-0.000213465000840088\\
-13598.3396410695	-0.000343544984132847\\
-13592.4795507264	-0.000428332418795748\\
-13586.6194603833	-0.000465796970757441\\
-13580.7593700403	-0.00045502433916357\\
-13574.8992796972	-0.000396238548846174\\
-13569.0391893542	-0.00029079738163158\\
-13563.1790990111	-0.000141161037301188\\
-13557.3190086681	4.9165252847688e-05\\
-13551.458918325	0.000275713321537797\\
-13545.5988279819	0.000533156947610412\\
-13539.7387376389	0.000815437062982916\\
-13533.8786472958	0.00111590434142437\\
-13528.0185569528	0.00142747581368767\\
-13522.1584666097	0.00174280182061814\\
-13516.2983762666	0.00205443936895238\\
-13510.4382859236	0.00235502780130478\\
-13504.5781955805	0.00263746263442819\\
-13498.7181052375	0.00289506346023967\\
-13492.8580148944	0.00312173194208246\\
-13486.9979245513	0.00331209616943637\\
-13481.1378342083	0.0034616379546464\\
-13475.2777438652	0.00356680005502438\\
-13469.4176535222	0.00362507077624197\\
-13463.5575631791	0.00363504394500396\\
-13457.697472836	0.00359645281948817\\
-13451.837382493	0.00351017711966958\\
-13445.9772921499	0.00337822299508318\\
-13440.1172018069	0.00320367638486224\\
-13434.2571114638	0.00299063085426809\\
-13428.3970211207	0.00274409159504284\\
-13422.5369307777	0.00246985784069783\\
-13416.6768404346	0.00217438645866533\\
-13410.8167500916	0.0018646399278886\\
-13404.9566597485	0.00154792228059124\\
-13399.0965694054	0.00123170687408411\\
-13393.2364790624	0.000923460053034822\\
-13387.3763887193	0.000630464863146462\\
-13381.5162983763	0.000359648977902956\\
-13375.6562080332	0.000117420903896955\\
-13369.7961176901	-9.04816629876709e-05\\
-13363.9360273471	-0.000259127742386496\\
-13358.075937004	-0.000384510016431121\\
-13352.215846661	-0.000463640045495784\\
-13346.3557563179	-0.000494619490605733\\
-13340.4956659748	-0.000476685645360349\\
-13334.6355756318	-0.000410230185557357\\
-13328.7754852887	-0.000296790676461904\\
-13322.9153949457	-0.000139015020327091\\
-13317.0553046026	5.94003341502795e-05\\
-13311.1952142596	0.000293796978764833\\
-13305.3351239165	0.000558663901163979\\
-13299.4750335734	0.00084776704176819\\
-13293.6149432304	0.00115429600674359\\
-13287.7548528873	0.00147102448452132\\
-13281.8947625443	0.00179048058761845\\
-13276.0346722012	0.00210512310452079\\
-13270.1745818581	0.00240751950441071\\
-13264.3144915151	0.002690521493395\\
-13258.454401172	0.00294743397577846\\
-13252.594310829	0.00317217342681173\\
-13246.7342204859	0.00335941193135101\\
-13240.8741301428	0.00350470347765947\\
-13235.0140397998	0.00360458951335667\\
-13229.1539494567	0.00365668125752385\\
-13223.2938591137	0.00365971681014061\\
-13217.4337687706	0.00361359169415607\\
-13211.5736784275	0.00351936209084715\\
-13205.7135880845	0.00337922067474385\\
-13199.8534977414	0.0031964455997176\\
-13193.9934073984	0.002975323823287\\
-13188.1333170553	0.00272105056194462\\
-13182.2732267122	0.00243960723559769\\
-13176.4131363692	0.00213762076854656\\
-13170.5530460261	0.00182220755591233\\
-13164.6929556831	0.0015008057697849\\
-13158.83286534	0.00118099995578919\\
-13152.9727749969	0.000870342056633228\\
-13147.1126846539	0.000576173085354545\\
-13141.2525943108	0.00030544965918062\\
-13135.3925039678	6.45794925091613e-05\\
-13129.5324136247	-0.000140730260949095\\
-13123.6723232817	-0.000305608227740763\\
-13117.8122329386	-0.000426134392418144\\
-13111.9521425955	-0.000499433291546825\\
-13106.0920522525	-0.000523742608611341\\
-13100.2319619094	-0.000498455533806195\\
-13094.3718715664	-0.000424135870874628\\
-13088.5117812233	-0.000302505515995168\\
-13082.6516908802	-0.000136404585615859\\
-13076.7916005372	7.02748838355391e-05\\
-13070.9315101941	0.000312680116960369\\
-13065.0714198511	0.000585111525206799\\
-13059.211329508	0.000881156703204051\\
-13053.3512391649	0.00119384134555545\\
-13047.4911488219	0.00151579353264106\\
-13041.6310584788	0.00183941751565272\\
-13035.7709681358	0.00215707290440395\\
-13029.9108777927	0.00246125503114236\\
-13024.0507874496	0.00274477223260003\\
-13018.1906971066	0.00300091586296916\\
-13012.3306067635	0.00322361901866571\\
-13006.4705164205	0.00340760021986583\\
-13000.6104260774	0.00354848864624062\\
-12994.7503357343	0.00364292795769067\\
-12988.8902453913	0.00368865623383579\\
-12983.0301550482	0.0036845601292744\\
-12977.1700647052	0.00363070194760178\\
-12971.3099743621	0.00352831897704251\\
-12965.449884019	0.00337979508384935\\
-12959.589793676	0.00318860521565528\\
-12953.7297033329	0.00295923410579319\\
-12947.8696129899	0.00269707108066291\\
-12942.0095226468	0.00240828343697946\\
-12936.1494323037	0.00209967136398446\\
-12930.2893419607	0.00177850782262507\\
-12924.4292516176	0.00145236715229779\\
-12918.5691612746	0.00112894644337695\\
-12912.7090709315	0.000815883888288386\\
-12906.8489805884	0.00052057839741305\\
-12900.9888902454	0.000250014739631456\\
-12895.1287999023	1.05983396940968e-05\\
-12889.2687095593	-0.000191996360520902\\
-12883.4086192162	-0.000352960387325561\\
-12877.5485288732	-0.000468464380026777\\
-12871.6884385301	-0.000535749686200485\\
-12865.828348187	-0.000553194240633414\\
-12859.968257844	-0.000520351655020747\\
-12854.1081675009	-0.00043796257693283\\
-12848.2480771579	-0.000307938030400445\\
-12842.3879868148	-0.00013331511204746\\
-12836.5278964717	8.18139306052683e-05\\
-12830.6678061287	0.000332397582802178\\
-12824.8077157856	0.000612543661835639\\
-12818.9476254426	0.000915657847748411\\
-12813.0875350995	0.00123459888875648\\
-12807.2274447564	0.0015618468286643\\
-12801.3673544134	0.00188968029365477\\
-12795.5072640703	0.00221035865913803\\
-12789.6471737273	0.0025163047989812\\
-12783.7870833842	0.00280028410320546\\
-12777.9269930411	0.00305557553454391\\
-12772.0669026981	0.00327613068007132\\
-12766.206812355	0.00345671703384009\\
-12760.346722012	0.00359304211695406\\
-12754.4866316689	0.00368185549068407\\
-12748.6265413258	0.00372102623895638\\
-12742.7664509828	0.00370959407306639\\
-12736.9063606397	0.00364779283380925\\
-12731.0462702967	0.00353704581610459\\
-12725.1861799536	0.00337993300728251\\
-12719.3260896105	0.00318013099173745\\
-12713.4659992675	0.00294232692201863\\
-12707.6059089244	0.00267210856891523\\
-12701.7458185814	0.002375833029362\\
-12695.8857282383	0.00206047717666717\\
-12690.0256378953	0.00173347337133613\\
-12684.1655475522	0.00140253430026165\\
-12678.3054572091	0.00107547107263958\\
-12672.4453668661	0.000760008862021156\\
-12666.585276523	0.000463604445762696\\
-12660.72518618	0.000193269950974337\\
-12654.8650958369	-4.45930272548612e-05\\
-12649.0050054938	-0.000244345011436138\\
-12643.1449151508	-0.000401242341946952\\
-12637.2848248077	-0.000511549822465491\\
-12631.4247344647	-0.000572629635952271\\
-12625.5646441216	-0.000583004414815551\\
-12619.7045537785	-0.000542392957633614\\
-12613.8444634355	-0.000451717729446316\\
-12607.9843730924	-0.000313083948290628\\
-12602.1242827494	-0.000129730731236164\\
-12596.2641924063	9.40445667776325e-05\\
-12590.4041020632	0.000352987054251422\\
-12584.5440117202	0.000641007685132349\\
-12578.6839213771	0.000951326424858319\\
-12572.8238310341	0.00127663183456044\\
-12566.963740691	0.00160925331756385\\
-12561.1036503479	0.00194134197063009\\
-12555.2435600049	0.00226505577614567\\
-12549.3834696618	0.002572744766284\\
-12543.5233793188	0.00285713178704507\\
-12537.6632889757	0.00311148459138351\\
-12531.8031986326	0.00332977519202992\\
-12525.9431082896	0.00350682270205226\\
-12520.0830179465	0.00363841627921462\\
-12514.2229276035	0.00372141525634715\\
-12508.3628372604	0.00375382407735921\\
-12502.5027469173	0.00373484025126078\\
-12496.6426565743	0.00366487417201057\\
-12490.7825662312	0.00354554031603154\\
-12484.9224758882	0.00337962000402058\\
-12479.0623855451	0.00317099658537821\\
-12473.2022952021	0.00292456455585278\\
-12467.342204859	0.00264611473427888\\
-12461.4821145159	0.00234219819242049\\
-12455.6220241729	0.00201997213471457\\
-12449.7619338298	0.00168703135342849\\
-12443.9018434868	0.00135122922800426\\
-12438.0417531437	0.0010204924872384\\
-12432.1816628006	0.000702634103045424\\
-12426.3215724576	0.00040516873201525\\
-12420.4614821145	0.00013513506430627\\
-12414.6013917715	-0.000101070720897505\\
-12408.7413014284	-0.00029784645285955\\
-12402.8812110853	-0.000450516831741952\\
-12397.0211207423	-0.000555444505329712\\
-12391.1610303992	-0.000610116721304849\\
-12385.3009400562	-0.000613205491169877\\
-12379.4408497131	-0.00056459982558779\\
-12373.58075937	-0.000465409259387132\\
-12367.720669027	-0.00031793856165308\\
-12361.8605786839	-0.000125634206525346\\
-12356.0004883409	0.000106996152381589\\
-12350.1403979978	0.000374489322910161\\
-12344.2803076547	0.000670554855121089\\
-12338.4202173117	0.000988222950008418\\
-12332.5601269686	0.00132000851873012\\
-12326.7000366256	0.00165808753087376\\
-12320.8399462825	0.0019944814973436\\
-12314.9798559394	0.00232124573828104\\
-12309.1197655964	0.00263065699378987\\
-12303.2596752533	0.00291539594744337\\
-12297.3995849103	0.00316872034916426\\
-12291.5394945672	0.00338462464340911\\
-12285.6794042241	0.00355798232289191\\
-12279.8193138811	0.00368466763461515\\
-12273.959223538	0.00376165374866157\\
-12268.099133195	0.00378708505443428\\
-12262.2390428519	0.00376032185772527\\
-12256.3789525089	0.00368195640269436\\
-12250.5188621658	0.00355379981892677\\
-12244.6587718227	0.00337884027913275\\
-12238.7986814797	0.00316117333453156\\
-12232.9385911366	0.00290590605113968\\
-12227.0785007936	0.00261903719041244\\
-12221.2184104505	0.00230731624555055\\
-12215.3583201074	0.00197808464517036\\
-12209.4982297644	0.00163910286021765\\
-12203.6381394213	0.00129836748561333\\
-12197.7780490783	0.000963922607587565\\
-12191.9179587352	0.000643669906154229\\
-12186.0578683921	0.000345181975279903\\
-12180.1977780491	7.55232708467938e-05\\
-12174.337687706	-0.000158917080374117\\
-12168.477597363	-0.000352576654140911\\
-12162.6175070199	-0.000500851698535456\\
-12156.7574166768	-0.000600206569420244\\
-12150.8973263338	-0.000648258030506057\\
-12145.0372359907	-0.000643832408403683\\
-12139.1771456477	-0.000586994232414466\\
-12133.3170553046	-0.000479045661057906\\
-12127.4569649615	-0.000322496686323777\\
-12121.5968746185	-0.000121006796832772\\
-12115.7367842754	0.0001207005447353\\
-12109.8766939324	0.000396948611615346\\
-12104.0166035893	0.000701240715803483\\
-12098.1565132462	0.00102641297382349\\
-12092.2964229032	0.00136480294354482\\
-12086.4363325601	0.00170843016272789\\
-12080.5762422171	0.00204918433577047\\
-12074.716151874	0.00237901673159098\\
-12068.8560615309	0.0026901302757762\\
-12062.9959711879	0.00297516384823449\\
-12057.1358808448	0.00322736643019985\\
-12051.2757905018	0.00344075698087566\\
-12045.4157001587	0.00361026625855431\\
-12039.5556098157	0.00373185722264122\\
-12033.6955194726	0.00380262115837028\\
-12027.8354291295	0.0038208472346446\\
-12021.9753387865	0.00378606383305424\\
-12016.1152484434	0.00369905064929623\\
-12010.2551581004	0.00356182125941683\\
-12004.3950677573	0.00337757653881428\\
-11998.5349774142	0.0031506300112117\\
-11992.6748870712	0.00288630686836549\\
-11986.8147967281	0.00259081902417382\\
-11980.9547063851	0.00227111913406627\\
-11975.094616042	0.00193473701038974\\
-11969.2345256989	0.00158960228214142\\
-11963.3744353559	0.00124385747522906\\
-11957.5143450128	0.00090566591928914\\
-11951.6542546698	0.000583019012334543\\
-11945.7941643267	0.000283547393260629\\
-11939.9340739836	1.43404838178694e-05\\
-11934.0739836406	-0.000218221332815659\\
-11928.2138932975	-0.000408617924373255\\
-11922.3538029545	-0.000552320429514992\\
-11916.4937126114	-0.000645898976044536\\
-11910.6336222683	-0.000687104535664512\\
-11904.7735319253	-0.000674922962167231\\
-11898.9134415822	-0.000609599915083926\\
-11893.0533512392	-0.000492636057467425\\
-11887.1932608961	-0.000326752617424923\\
-11881.333170553	-0.000115828103150199\\
-11875.47308021	0.000135192357265997\\
-11869.6129898669	0.000420412932435065\\
-11863.7528995239	0.00073312554389767\\
-11857.8928091808	0.00106596761078129\\
-11852.0327188377	0.0014110953738765\\
-11846.1726284947	0.00176036871911771\\
-11840.3125381516	0.00210554314583713\\
-11834.4524478086	0.00243846435305652\\
-11828.5923574655	0.00275126085144999\\
-11822.7322671225	0.00303653005164627\\
-11816.8721767794	0.00328751343019484\\
-11811.0120864363	0.00349825662833911\\
-11805.1519960933	0.00366375069161708\\
-11799.2919057502	0.00378005110029099\\
-11793.4318154072	0.00384437176269776\\
-11787.5717250641	0.00385515173155192\\
-11781.711634721	0.00381209304655986\\
-11775.851544378	0.00371616878650527\\
-11769.9914540349	0.00356960111660585\\
-11764.1313636919	0.00337580982533867\\
-11758.2712733488	0.0031393325419885\\
-11752.4111830057	0.00286571849612794\\
-11746.5510926627	0.00256139830547651\\
-11740.6910023196	0.00223353284780547\\
-11734.8309119766	0.0018898447679716\\
-11728.9708216335	0.00153843658425448\\
-11723.1107312904	0.00118759967797026\\
-11717.2506409474	0.000845618668812081\\
-11711.3905506043	0.000520575791086038\\
-11705.5304602613	0.000220159889687517\\
-11699.6703699182	-4.85154505816385e-05\\
-11693.8102795751	-0.000279080340854924\\
-11687.9501892321	-0.000466059601298056\\
-11682.090098889	-0.00060500277171346\\
-11676.230008546	-0.000692590032835402\\
-11670.3699182029	-0.000726711517823789\\
-11664.5098278598	-0.000706518119476864\\
-11658.6497375168	-0.000632442570444803\\
-11652.7896471737	-0.000506190274248848\\
-11646.9295568307	-0.000330700078967078\\
-11641.0694664876	-0.000110075895037949\\
-11635.2093761446	0.000150509252026209\\
-11629.3492858015	0.000444934491304871\\
-11623.4891954584	0.000766274855940216\\
-11617.6291051154	0.0011069641368053\\
-11611.7690147723	0.00145897301096747\\
-11605.9089244293	0.00181399825166154\\
-11600.0488340862	0.00216365856146674\\
-11594.1887437431	0.00249969241031521\\
-11588.3286534001	0.00281415320880766\\
-11582.468563057	0.00309959720636045\\
-11576.608472714	0.00334925967178131\\
-11570.7483823709	0.00355721518658443\\
-11564.8882920278	0.00371851825338741\\
-11559.0282016848	0.00382932088312443\\
-11553.1681113417	0.00388696436550349\\
-11547.3080209987	0.00389004303458434\\
-11541.4479306556	0.00383843850080378\\
-11535.5878403125	0.00373332351644226\\
-11529.7277499695	0.00357713535795049\\
-11523.8676596264	0.00337351932864017\\
-11518.0075692834	0.00312724369047676\\
-11512.1474789403	0.00284408800978299\\
-11506.2873885972	0.00253070753122748\\
-11500.4272982542	0.00219447676199529\\
-11494.5672079111	0.00184331594227765\\
-11488.7071175681	0.00148550448483533\\
-11482.847027225	0.00112948577771722\\
-11476.9869368819	0.000783667951793509\\
-11471.1268465389	0.000456225313723368\\
-11465.2667561958	0.000154905133447802\\
-11459.4066658528	-0.00011315564734382\\
-11453.5465755097	-0.0003415994497413\\
-11447.6864851666	-0.000524998831970757\\
-11441.8263948236	-0.000658985428308061\\
-11435.9663044805	-0.000740353989706235\\
-11430.1062141375	-0.000767139048648679\\
-11424.2461237944	-0.000738662375043392\\
-11418.3860334513	-0.00065555007787772\\
-11412.5259431083	-0.000519718923417286\\
-11406.6658527652	-0.000334332166770206\\
-11400.8057624222	-0.000103725914057236\\
-11394.9456720791	0.000166692271139188\\
-11389.0855817361	0.000470570146491156\\
-11383.225491393	0.000800759982938834\\
-11377.3654010499	0.00114948666634536\\
-11371.5053107069	0.00150853075579784\\
-11365.6452203638	0.00186942218896967\\
-11359.7851300208	0.00222364006997352\\
-11353.9250396777	0.0025628138275367\\
-11348.0649493346	0.00287892099522269\\
-11342.2048589916	0.00316447694066062\\
-11336.3447686485	0.00341271205754877\\
-11330.4846783055	0.00361773222557966\\
-11324.6245879624	0.00377465873547897\\
-11318.7644976193	0.0038797443577932\\
-11312.9044072763	0.00393046279449555\\
-11307.0443169332	0.00392556937828703\\
-11301.1842265902	0.00386513156220703\\
-11295.3241362471	0.00375052845314002\\
-11289.464045904	0.00358441937418049\\
-11283.603955561	0.0033706821712818\\
-11277.7438652179	0.00311432269489252\\
-11271.8837748749	0.00282135756896189\\
-11266.0236845318	0.00249867299264071\\
-11260.1635941887	0.00215386288732921\\
-11254.3035038457	0.00179505019347766\\
-11248.4434135026	0.00143069552166349\\
-11242.5833231596	0.00106939766507679\\
-11236.7232328165	0.000719690676834046\\
-11230.8631424734	0.000389842300133102\\
-11225.0030521304	8.76585120602249e-05\\
-11219.1429617873	-0.000179701198804234\\
-11213.2828714443	-0.000405893449987297\\
-11207.4227811012	-0.000585541459919407\\
-11201.5626907581	-0.000714362849918624\\
-11195.7026004151	-0.000789271715960848\\
-11189.842510072	-0.000808452537875103\\
-11183.982419729	-0.000771404156191048\\
-11178.1223293859	-0.000678952752354025\\
-11172.2622390429	-0.00053323349861615\\
-11166.4021486998	-0.00033764128368209\\
-11160.5420583567	-9.6751650451197e-05\\
-11154.6819680137	0.000183786213358322\\
-11148.8218776706	0.000497381929508431\\
-11142.9617873276	0.000836658723243469\\
-11137.1016969845	0.00119362692170963\\
-11131.2416066414	0.00155987207647026\\
-11125.3815162984	0.00192675328124091\\
-11119.5214259553	0.00228560701108759\\
-11113.6613356123	0.00262795167483101\\
-11107.8012452692	0.00294568805187241\\
-11101.9411549261	0.00323129087659242\\
-11096.0810645831	0.00347798703931719\\
-11090.22097424	0.00367991618436731\\
-11084.360883897	0.0038322698974219\\
-11078.5007935539	0.00393140617704841\\
-11072.6407032108	0.00397493646565578\\
-11066.7806128678	0.0039617831605952\\
-11060.9205225247	0.00389220622127304\\
-11055.0604321817	0.00376779821679077\\
-11049.2003418386	0.00359144790304411\\
-11043.3402514955	0.00336727316123264\\
-11037.4801611525	0.0031005248539906\\
-11031.6200708094	0.00279746384381757\\
-11025.7599804664	0.00246521405312791\\
-11019.8998901233	0.00211159501458328\\
-11014.0397997802	0.00174493784698165\\
-11008.1797094372	0.00137388898730985\\
-11002.3196190941	0.00100720630188527\\
-10996.4595287511	0.000653552384313878\\
-10990.599438408	0.000321289918300479\\
-10984.7393480649	1.82839388482019e-05\\
-10978.8792577219	-0.000248284332806725\\
-10973.0191673788	-0.000472087674542846\\
-10967.1590770358	-0.00064780303624405\\
-10961.2989866927	-0.000771238136418995\\
-10955.4388963497	-0.000839431472165179\\
-10949.5788060066	-0.000850723357193363\\
-10943.7187156635	-0.000804796284483161\\
-10937.8586253205	-0.000702683632908234\\
-10931.9985349774	-0.000546746483764736\\
-10926.1384446344	-0.000340619065828483\\
-10920.2783542913	-8.91240887661348e-05\\
-10914.4182639482	0.000201840063112207\\
-10908.5581736052	0.000525437638647681\\
-10902.6980832621	0.000874056086533502\\
-10896.8379929191	0.00123948510972467\\
-10890.977902576	0.00161310999657833\\
-10885.1178122329	0.0019861146766046\\
-10879.2577218899	0.00234968971526866\\
-10873.3976315468	0.00269524034099343\\
-10867.5375412038	0.00301458959249164\\
-10861.6774508607	0.00330017178559698\\
-10855.8173605176	0.00354521172185579\\
-10849.9572701746	0.00374388539552924\\
-10844.0971798315	0.00389145838628969\\
-10838.2370894885	0.00398439865071868\\
-10832.3769991454	0.00402046102484192\\
-10826.5169088023	0.00399874141766316\\
-10820.6568184593	0.00391969938738359\\
-10814.7967281162	0.0037851485389755\\
-10808.9366377732	0.00359821494007103\\
-10803.0765474301	0.00336326450764631\\
-10797.216457087	0.00308580105198491\\
-10791.356366744	0.00277233735769858\\
-10785.4962764009	0.00243024232186667\\
-10779.6361860579	0.0020675677359539\\
-10773.7760957148	0.00169285878333361\\
-10767.9160053717	0.00131495271158743\\
-10762.0559150287	0.000942770422780636\\
-10756.1958246856	0.000585105895872639\\
-10750.3357343426	0.000250418411753181\\
-10744.4756439995	-5.3367510771315e-05\\
-10738.6155536565	-0.000319049736759018\\
-10732.7554633134	-0.000540319253106635\\
-10726.8953729703	-0.000711909975547922\\
-10721.0352826273	-0.000829724068009946\\
-10715.1751922842	-0.000890929792516086\\
-10709.3151019412	-0.000894029553685494\\
-10703.4550115981	-0.000838896503484505\\
-10697.594921255	-0.000726778812581885\\
-10691.734830912	-0.000560271477435844\\
-10685.8747405689	-0.000343256298508176\\
-10680.0146502259	-8.08114172161412e-05\\
-10674.1545598828	0.000220907480782324\\
-10668.2944695397	0.000554811517343699\\
-10662.4343791967	0.000913045144044303\\
-10656.5742888536	0.00128717092367804\\
-10650.7141985106	0.00166836822485437\\
-10644.8541081675	0.00204764115135044\\
-10638.9940178244	0.00241603080492403\\
-10633.1339274814	0.00276482687384703\\
-10627.2738371383	0.00308577355024298\\
-10621.4137467953	0.00337126490894524\\
-10615.5536564522	0.00361452512434587\\
-10609.6935661091	0.00380976925589593\\
-10603.8334757661	0.00395234078702719\\
-10597.973385423	0.00403882264862621\\
-10592.11329508	0.00406711907964583\\
-10586.2532047369	0.0040365063650835\\
-10580.3931143938	0.00394765122377145\\
-10574.5330240508	0.00380259638053253\\
-10568.6729337077	0.00360471363352503\\
-10562.8128433647	0.00335862549278282\\
-10556.9527530216	0.00307009721166655\\
-10551.0926626786	0.00274590173169188\\
-10545.2325723355	0.0023936607045229\\
-10539.3724819924	0.0020216653216966\\
-10533.5123916494	0.00163868116430806\\
-10527.6523013063	0.00125374166453445\\
-10521.7922109633	0.000875935045694689\\
-10515.9321206202	0.000514189765212761\\
-10510.0720302771	0.000177063525259744\\
-10504.2119399341	-0.000127459163385799\\
-10498.351849591	-0.000392156076208997\\
-10492.491759248	-0.000610738550696399\\
-10486.6316689049	-0.000778000881692425\\
-10480.7715785618	-0.000889944287736941\\
-10474.9114882188	-0.000943872496941892\\
-10469.0513978757	-0.00093845666803155\\
-10463.1913075327	-0.000873768084241219\\
-10457.3312171896	-0.00075127781698924\\
-10451.4711268465	-0.000573823335669707\\
-10445.6110365035	-0.000345542819937063\\
-10439.7509461604	-7.17786946231689e-05\\
-10433.8908558174	0.000241047364741797\\
-10428.0307654743	0.000585585031828312\\
-10422.1706751312	0.000953728003514114\\
-10416.3105847882	0.00133680469212515\\
-10410.4504944451	0.00172578245042778\\
-10404.5904041021	0.00211148052044146\\
-10398.730313759	0.00248478668600041\\
-10392.8702234159	0.00283687251706336\\
-10387.0101330729	0.00315940212184735\\
-10381.1500427298	0.00344472947134819\\
-10375.2899523868	0.00368607962577793\\
-10369.4298620437	0.0038777095678259\\
-10363.5697717006	0.00401504482546074\\
-10357.7096813576	0.00409478863462869\\
-10351.8495910145	0.00411500103698702\\
-10345.9895006715	0.00407514601630697\\
-10340.1294103284	0.00397610552871888\\
-10334.2693199853	0.00382016006354836\\
-10328.4092296423	0.00361093616065831\\
-10322.5491392992	0.00335332209138283\\
-10316.6890489562	0.00305335366302287\\
-10310.8289586131	0.00271807281358105\\
-10304.9688682701	0.00235536230955139\\
-10299.108777927	0.00197376042610718\\
-10293.2486875839	0.00158225996608609\\
-10287.3885972409	0.00119009634824351\\
-10281.5285068978	0.000806529757394639\\
-10275.6684165548	0.000440626495162082\\
-10269.8083262117	0.000101044693077294\\
-10263.9482358686	-0.000204170547546042\\
-10258.0881455256	-0.000467777742389184\\
-10252.2280551825	-0.000683510823126175\\
-10246.3679648395	-0.000846228073398956\\
-10240.5078744964	-0.000952034662309488\\
-10234.6477841533	-0.000998375855852838\\
-10228.7876938103	-0.000984098676312138\\
-10222.9276034672	-0.0009094805221939\\
-10217.0675131242	-0.000776224040201815\\
-10211.2074227811	-0.000587418336644094\\
-10205.347332438	-0.00034746741081545\\
-10199.487242095	-6.1987467481811e-05\\
-10193.6271517519	0.000262324497739244\\
-10187.7670614089	0.000617847765708381\\
-10181.9069710658	0.000996216930582045\\
-10176.0468807227	0.00138851870049675\\
-10170.1867903797	0.00178550183289332\\
-10164.3267000366	0.0021777952602686\\
-10158.4666096936	0.0025561292641822\\
-10152.6065193505	0.0029115544778712\\
-10146.7464290074	0.00323565354394054\\
-10140.8863386644	0.00352074042202343\\
-10135.0262483213	0.00376004262754251\\
-10129.1661579783	0.00394786208149846\\
-10123.3060676352	0.00407971075127527\\
-10117.4459772922	0.00415241785433444\\
-10111.5858869491	0.00416420606459005\\
-10105.725796606	0.00411473489116484\\
-10099.865706263	0.00400511017125079\\
-10094.0056159199	0.00383785941951596\\
-10088.1455255769	0.00361687358081613\\
-10082.2854352338	0.00334731652848656\\
-10076.4253448907	0.00303550441144205\\
-10070.5652545477	0.0026887576695916\\
-10064.7051642046	0.00231522918344578\\
-10058.8450738616	0.00192371259194056\\
-10052.9849835185	0.00152343528432678\\
-10047.1248931754	0.00112384093899665\\
-10041.2648028324	0.000734366733043322\\
-10035.4047124893	0.000364220478364466\\
-10029.5446221463	2.21629464914135e-05\\
-10023.6845318032	-0.00028369947249052\\
-10017.8244414601	-0.00054610687034786\\
-10011.9643511171	-0.000758818128777706\\
-10006.104260774	-0.000916759346183609\\
-10000.244170431	-0.00101614485376583\\
-9994.3840800879	-0.00105456793537535\\
-9988.52398974484	-0.00103105907773339\\
-9982.66389940178	-0.000946110342293003\\
-9976.80380905872	-0.000801665248406276\\
-9970.94371871567	-0.000601074371164269\\
-9965.08362837261	-0.000349017667190359\\
-9959.22353802955	-5.13953281799235e-05\\
-9953.36344768649	0.000284810292908342\\
-9947.50335734343	0.000651698452519214\\
-9941.64326700037	0.00104063564336161\\
-9935.78317665731	0.0014424587168023\\
-9929.92308631425	0.00184769072254565\\
-9924.06299597119	0.00224676438194026\\
-9918.20290562813	0.00263024792603116\\
-9912.34281528507	0.00298906796786863\\
-9906.48272494201	0.00331472414342831\\
-9900.62263459895	0.00359949044392328\\
-9894.76254425589	0.00383659847186479\\
-9888.90245391283	0.00402039827435218\\
-9883.04236356977	0.00414649293305666\\
-9877.18227322671	0.0042118437045222\\
-9871.32218288365	0.00421484319840697\\
-9865.46209254059	0.00415535483135437\\
-9859.60200219753	0.00403471759031168\\
-9853.74191185447	0.00385571595604088\\
-9847.88182151142	0.00362251566100787\\
-9842.02173116836	0.00334056676256645\\
-9836.1616408253	0.00301647628608772\\
-9830.30155048224	0.00265785341320656\\
-9824.44146013918	0.00227313084248649\\
-9818.58136979612	0.00187136651527814\\
-9812.72127945306	0.00146203036832846\\
-9806.86118911	0.00105478113286438\\
-9801.00109876694	0.000659238440118995\\
-9795.14100842388	0.000284755609934703\\
-9789.28091808082	-5.9801511773406e-05\\
-9783.42082773776	-0.000366264437741306\\
-9777.5607373947	-0.00062735567827998\\
-9771.70064705164	-0.000836861544688236\\
-9765.84055670858	-0.000989780014612611\\
-9759.98046636552	-0.00108244014129332\\
-9754.12037602246	-0.001112590156819\\
-9748.2602856794	-0.00107945215577781\\
-9742.40019533634	-0.000983742032508692\\
-9736.54010499328	-0.000827654164184469\\
-9730.68001465022	-0.000614811163982807\\
-9724.81992430716	-0.000350179853627983\\
-9718.95983396411	-3.99554034799944e-05\\
-9713.09974362105	0.000308583657897991\\
-9707.23965327799	0.000687246171984016\\
-9701.37956293493	0.00108712081242176\\
-9695.51947259187	0.00149878575938505\\
-9689.65938224881	0.00191253065364958\\
-9683.79929190575	0.00231858560203689\\
-9677.93920156269	0.00270735183443493\\
-9672.07911121963	0.0030696285673527\\
-9666.21902087657	0.00339683071277788\\
-9660.35893053351	0.00368119228174126\\
-9654.49884019045	0.00391595066399643\\
-9648.63874984739	0.00409550741163851\\
-9642.77865950433	0.00421556170466336\\
-9636.91856916128	0.00427321331770823\\
-9631.05847881822	0.00426703262295962\\
-9625.19838847516	0.0041970959411474\\
-9619.3382981321	0.00406498536853612\\
-9613.47820778904	0.00387375304460541\\
-9607.61811744598	0.0036278506670992\\
-9601.75802710292	0.00333302587986098\\
-9595.89793675986	0.00299618794527391\\
-9590.0378464168	0.00262524583909361\\
-9584.17775607374	0.00222892256176408\\
-9578.31766573068	0.00181655002431203\\
-9572.45757538762	0.0013978493320181\\
-9566.59748504456	0.000982701637709513\\
-9560.7373947015	0.000580914965836019\\
-9554.87730435844	0.000201992508947934\\
-9549.01721401538	-0.000145092130400868\\
-9543.15712367232	-0.000452107437206596\\
-9537.29703332927	-0.000711759189798616\\
-9531.43694298621	-0.000917863745331579\\
-9525.57685264315	-0.00106549528902926\\
-9519.71676230009	-0.00115110354139142\\
-9513.85667195703	-0.00117259911164546\\
-9507.99658161397	-0.0011294044460698\\
-9502.13649127091	-0.00102246913060506\\
-9496.27640092785	-0.000854249146834765\\
-9490.41631058479	-0.000628650531872853\\
-9484.55622024173	-0.000350938732998813\\
-9478.69612989867	-2.76157599881354e-05\\
-9472.83603955561	0.000333731999683831\\
-9466.97594921255	0.000724611741319333\\
-9461.11585886949	0.00113582380540982\\
-9455.25576852643	0.00155767815331197\\
-9449.39567818337	0.00198022266306069\\
-9443.53558784031	0.00239347786758938\\
-9437.67549749725	0.00278767259901116\\
-9431.81540715419	0.00315347497583804\\
-9425.95531681113	0.00348221327314416\\
-9420.09522646807	0.00376608144899409\\
-9414.23513612501	0.00399832445619156\\
-9408.37504578196	0.00417339894181847\\
-9402.5149554389	0.00428710551121622\\
-9396.65486509584	0.00433668940223972\\
-9390.79477475278	0.0043209071568294\\
-9384.93468440972	0.00424005767722932\\
-9379.07459406666	0.00409597689430078\\
-9373.2145037236	0.00389199613237378\\
-9367.35441338054	0.00363286511371622\\
-9361.49432303748	0.00332464138060481\\
-9355.63423269442	0.00297454870915472\\
-9349.77414235136	0.00259080782287016\\
-9343.9140520083	0.00218244337278847\\
-9338.05396166524	0.00175907171526883\\
-9332.19387132218	0.00133067447743953\\
-9326.33378097912	0.000907363241330561\\
-9320.47369063606	0.000499140893397119\\
-9314.613600293	0.000115665271268403\\
-9308.75350994994	-0.000233979308234936\\
-9302.89341960689	-0.000541497235881272\\
-9297.03332926383	-0.000799578414947775\\
-9291.17323892077	-0.00100207201605785\\
-9285.31314857771	-0.001144133052739\\
-9279.45305823465	-0.00122233828547741\\
-9273.59296789159	-0.00123476868250341\\
-9267.73287754853	-0.00118105645194892\\
-9261.87278720547	-0.00106239549452392\\
-9256.01269686241	-0.000881514988088023\\
-9250.15260651935	-0.000642616685997277\\
-9244.29251617629	-0.000351277368403233\\
-9238.43242583323	-1.43187103329364e-05\\
-9232.57233549017	0.000360352397762063\\
-9226.71224514711	0.000763929339960075\\
-9220.85215480405	0.00118691272472488\\
-9214.992064461	0.0016193339321104\\
-9209.13197411794	0.00205098999858324\\
-9203.27188377488	0.00247168430556852\\
-9197.41179343182	0.00287146739490105\\
-9191.55170308876	0.00324087222426609\\
-9185.6916127457	0.00357113830113054\\
-9179.83152240264	0.00385441938930239\\
-9173.97143205958	0.00408396986436188\\
-9168.11134171652	0.00425430529276466\\
-9162.25125137346	0.00436133341242939\\
-9156.3911610304	0.00440245238819283\\
-9150.53107068734	0.00437661398349029\\
-9144.67098034428	0.0042843501160307\\
-9138.81089000122	0.00412776212804796\\
-9132.95079965816	0.00391047298120032\\
-9127.0907093151	0.00363754346196718\\
-9121.23061897204	0.00331535433450035\\
-9115.37052862898	0.00295145718423983\\
-9109.51043828592	0.00255439743838982\\
-9103.65034794286	0.00213351370972789\\
-9097.79025759981	0.00169871817493735\\
-9091.93016725675	0.00126026315146303\\
-9086.07007691369	0.000828499369137581\\
-9080.20998657063	0.000413631634918769\\
-9074.34989622757	2.54776579995897e-05\\
-9068.48980588451	-0.000326764266075394\\
-9062.62971554145	-0.000634733215633085\\
-9056.76962519839	-0.000891104083604343\\
-9050.90953485533	-0.00108976178970174\\
-9045.04944451227	-0.00122594712163562\\
-9039.18935416921	-0.00129637072791189\\
-9033.32926382615	-0.0012992925329808\\
-9027.46917348309	-0.00123456465848646\\
-9021.60908314003	-0.00110363679407522\\
-9015.74899279697	-0.000909523846688235\\
-9009.88890245391	-0.000656736587672686\\
-9004.02881211085	-0.000351176891762859\\
nan	nan\\
-8992.30863142473	0.000388552979863155\\
-8986.44854108167	0.000805348415206443\\
-8980.58845073861	0.00124057479759203\\
-8974.72836039555	0.00168397365532782\\
-8968.8682700525	0.00212508129596976\\
-8963.00817970944	0.00255347568281955\\
-8957.14808936638	0.00295902262171462\\
-8951.28799902332	0.00333211544331347\\
-8945.42790868026	0.00366390251348384\\
-8939.5678183372	0.00394649718408987\\
-8933.70772799414	0.0041731652057437\\
-8927.84763765108	0.00433848514944135\\
-8921.98754730802	0.00443847801676287\\
-8916.12745696496	0.00447070294047248\\
-8910.2673666219	0.00443431667582854\\
-8904.40727627884	0.00433009543330857\\
-8898.54718593578	0.00416041849243725\\
-8892.68709559272	0.00392921393786518\\
-8886.82700524966	0.00364186775339929\\
-8880.96691490661	0.0033050983764149\\
-8875.10682456355	0.00292679963415491\\
-8869.24673422049	0.00251585573192684\\
-8863.38664387743	0.00208193262934456\\
-8857.52655353437	0.00163525070234832\\
-8851.66646319131	0.00118634403701513\\
-8845.80637284825	0.000745812023475895\\
-8839.94628250519	0.000324069106560004\\
-8834.08619216213	-6.89013992747874e-05\\
-8828.22610181907	-0.00042378361213721\\
-8822.36601147601	-0.000732149909718607\\
-8816.50592113295	-0.0009866610474207\\
-8810.64583078989	-0.00118124081662268\\
-8804.78574044683	-0.00131122108748717\\
-8798.92565010377	-0.00137345377493222\\
-8793.06555976071	-0.0013663870435062\\
-8787.20546941765	-0.00129010390772904\\
-8781.3453790746	-0.00114632227046114\\
-8775.48528873154	-0.000938356351398894\\
-8769.62519838848	-0.000671040369009993\\
-8763.76510804542	-0.000350616232252363\\
-8757.90501770236	1.54121500532384e-05\\
-8752.0449273593	0.000418454542535058\\
-8746.18483701624	0.000849035927562462\\
-8740.32474667318	0.00129701919275337\\
-8734.46465633012	0.00175184372891886\\
-8728.60456598706	0.00220277432339206\\
-8722.744475644	0.00263915448377542\\
-8716.88438530094	0.00305065821529802\\
-8711.02429495788	0.00342753430388004\\
-8705.16420461482	0.00376083732708757\\
-8699.30411427176	0.00404263992075257\\
-8693.4440239287	0.00426622126576131\\
-8687.58393358564	0.00442622731398859\\
-8681.72384324258	0.00451879893457501\\
-8675.86375289952	0.0045416649148009\\
-8670.00366255646	0.00449419757514533\\
-8664.1435722134	0.00437742963913729\\
-8658.28348187034	0.00419403191096226\\
-8652.42339152729	0.00394825223982305\\
-8646.56330118423	0.00364581716300942\\
-8640.70321084117	0.00329379850529249\\
-8634.84312049811	0.0029004480401003\\
-8628.98303015505	0.00247500407744168\\
-8623.12293981199	0.00202747451125621\\
-8617.26284946893	0.00156840141982746\\
-8611.40275912587	0.00110861275528065\\
-8605.54266878281	0.000658966970117599\\
-8599.68257843975	0.000230096602570953\\
-8593.82248809669	-0.000167843125083536\\
-8587.96239775363	-0.000525414751988675\\
-8582.10230741057	-0.000834122375845357\\
-8576.24221706751	-0.0010866134956365\\
-8570.38212672445	-0.00127685410565536\\
-8564.52203638139	-0.00140027287203389\\
-8558.66194603833	-0.00145387094796795\\
-8552.80185569527	-0.00143629479067399\\
-8546.94176535222	-0.00134787021387826\\
-8541.08167500916	-0.00119059682223355\\
-8535.2215846661	-0.000968102909160292\\
-8529.36149432304	-0.000685561832787853\\
-8523.50140397998	-0.000349571796148656\\
-8517.64131363692	3.19981760311385e-05\\
-8511.78122329386	0.000450192469494161\\
-8505.9211329508	0.000895179009245003\\
-8500.06104260774	0.00135648035571344\\
-8494.20095226468	0.00182322033744776\\
-8488.34086192162	0.00228438041581017\\
-8482.48077157856	0.00272905973982109\\
-8476.6206812355	0.00314673275453141\\
-8470.76059089244	0.00352749827598055\\
-8464.90050054939	0.00386231414000165\\
-8459.04041020633	0.00414321186451714\\
-8453.18031986327	0.00436348623070412\\
-8447.32022952021	0.00451785527312759\\
-8441.46013917715	0.00460258686299975\\
-8435.60004883409	0.00461558885176763\\
-8429.73995849103	0.00455646059946184\\
-8423.87986814797	0.00442650462099318\\
-8418.01977780491	0.0042286980239978\\
-8412.15968746185	0.00396762436058506\\
-8406.29959711879	0.00364936745141265\\
-8400.43950677573	0.00328136964013129\\
-8394.57941643267	0.00287225777954352\\
-8388.71932608961	0.00243164101740384\\
-8382.85923574655	0.00196988512036785\\
-8376.99914540349	0.00149786863542219\\
-8371.13905506043	0.00102672662364745\\
-8365.27896471737	0.000567588002048909\\
-8359.41887437431	0.000131312688125549\\
-8353.55878403125	-0.000271765245471788\\
-8347.69869368819	-0.000632082332671406\\
-8341.83860334514	-0.000941072596011324\\
-8335.97851300208	-0.00119137116750191\\
-8330.11842265902	-0.00137698980918325\\
-8324.25833231596	-0.0014934601513483\\
-8318.3982419729	-0.00153794122356749\\
-8312.53815162984	-0.00150928869192557\\
-8306.67806128678	-0.00140808411730959\\
-8300.81797094372	-0.00123662349096209\\
-8294.95788060066	-0.000998865264702988\\
-8289.0977902576	-0.000700339049596867\\
-8283.23769991454	-0.000348017087531137\\
-8277.37760957148	4.98485181683334e-05\\
-8271.51751922842	0.000483919014221943\\
-8265.65742888536	0.000943988128342759\\
-8259.7973385423	0.00141922197891081\\
-8253.93724819924	0.00189841412506393\\
-8248.07715785618	0.002370249753967\\
-8242.21706751312	0.00282357277835864\\
-8236.35697717006	0.00324764954019057\\
-8230.496886827	0.00363242288834444\\
-8224.63679648394	0.00396875061699897\\
-8218.77670614088	0.00424862261316535\\
-8212.91661579783	0.00446535155618922\\
-8207.05652545477	0.00461373263109226\\
-8201.19643511171	0.00469016844152466\\
-8195.33634476865	0.00469275612697605\\
-8189.47625442559	0.00462133457518301\\
-8183.61616408253	0.00447749056206788\\
-8177.75607373947	0.0042645236188018\\
-8171.89598339641	0.00398737040045396\\
-8166.03589305335	0.00365249028792505\\
-8160.17580271029	0.00326771487287228\\
-8154.31571236723	0.00284206483124809\\
-8148.45562202417	0.00238553846709102\\
-8142.59553168111	0.00190887688143375\\
-8136.73544133805	0.00142331128130615\\
-8130.87535099499	0.000940298371978892\\
-8125.01526065193	0.000471250065677629\\
-8119.15517030888	2.72638824057015e-05\\
-8113.29507996582	-0.000381139589259708\\
-8107.43498962276	-0.000744265961507545\\
-8101.5748992797	-0.00105347714186791\\
-8095.71480893664	-0.00130139679299883\\
-8089.85471859358	-0.00148208627178479\\
-8083.99462825052	-0.00159118685275138\\
-8078.13453790746	-0.00162602483077163\\
-8072.2744475644	-0.00158567697055843\\
-8066.41435722134	-0.00147099470286946\\
-8060.55426687828	-0.00128458643998159\\
-8054.69417653522	-0.001030758370533\\
-8048.83408619216	-0.000715415075374337\\
-8042.9739958491	-0.000345922256493886\\
-8037.11390550604	6.90652311472127e-05\\
-8031.25381516299	0.000519806030933051\\
-8025.39372481993	0.000995700875414336\\
-8019.53363447687	0.00148554175362422\\
-8013.67354413381	0.00197777579864329\\
-8007.81345379075	0.00246077768362429\\
-8001.95336344769	0.0029231241047975\\
-7996.09327310463	0.0033538638718244\\
-7990.23318276157	0.00374277722110838\\
-7984.37309241851	0.00408061821249624\\
-7978.51300207545	0.00435933446178328\\
-7972.65291173239	0.00457225898802852\\
-7966.79282138933	0.00471426960627653\\
-7960.93273104627	0.00478191205601213\\
-7955.07264070321	0.00477348390764992\\
-7949.21255036015	0.00468907721033001\\
-7943.35246001709	0.00453057881694765\\
-7937.49236967403	0.00430162831903393\\
-7931.63227933097	0.00400753452702635\\
-7925.77218898791	0.00365515240923434\\
-7919.91209864485	0.00325272334162619\\
-7914.05200830179	0.00280968238947026\\
-7908.19191795873	0.00233643712631276\\
-7902.33182761568	0.001844123173884\\
-7896.47173727262	0.00134434220401038\\
-7890.61164692956	0.000848888565200893\\
-7884.7515565865	0.000369470974443799\\
-7878.89146624344	-8.25641599395829e-05\\
-7873.03137590038	-0.000496501247896817\\
-7867.17128555732	-0.000862509506677961\\
-7861.31119521426	-0.0011718764103387\\
-7855.4511048712	-0.00141721505745585\\
-7849.59101452814	-0.0015926405225492\\
-7843.73092418508	-0.00169391098271643\\
-7837.87083384202	-0.00171853023641277\\
-7832.01074349896	-0.00166580913852832\\
-7826.1506531559	-0.00153688444269864\\
-7820.29056281284	-0.00133469454520824\\
-7814.43047246978	-0.00106391264226396\\
-7808.57038212672	-0.000730838818666499\\
-7802.71029178366	-0.000343253558008612\\
-7796.8502014406	8.97639209317597e-05\\
-7790.99011109754	0.000558048261289596\\
-7785.13002075449	0.00105058652195334\\
-7779.26993041143	0.00155577709121441\\
-7773.40984006837	0.0020617028745361\\
-7767.54974972531	0.00255641232483692\\
-7761.68965938225	0.00302820168944862\\
-7755.82956903919	0.00346589180955359\\
-7749.96947869613	0.0038590929269255\\
-7744.10938835307	0.00419845122673438\\
-7738.24929801001	0.00447587126749133\\
-7732.38920766695	0.00468470901040696\\
-7726.52911732389	0.00481993084626727\\
-7720.66902698083	0.00487823481579208\\
-7714.80893663777	0.00485813110556342\\
-7708.94884629471	0.00475997985931153\\
-7703.08875595166	0.00458598535043495\\
-7697.2286656086	0.00434014658960261\\
-7691.36857526554	0.00402816547112825\\
-7685.50848492248	0.00365731456558242\\
-7679.64839457942	0.00323626762295892\\
-7673.78830423636	0.00277489673474384\\
-7667.9282138933	0.00228404089697095\\
-7662.06812355024	0.0017752513990113\\
-7656.20803320718	0.00126052001710392\\
-7650.34794286412	0.000751996407223845\\
-7644.48785252106	0.000261701355847398\\
-7638.627762178	-0.000198757344909064\\
-7632.76767183494	-0.000618459683814279\\
-7626.90758149188	-0.000987432375046988\\
-7621.04749114882	-0.00129688582194969\\
-7615.18740080576	-0.00153942347109906\\
-7609.3273104627	-0.00170921857177884\\
-7603.46722011964	-0.00180215411821402\\
-7597.60712977658	-0.0018159226151982\\
-7591.74703943353	-0.00175008325178003\\
-7585.88694909047	-0.00160607506982482\\
-7580.02685874741	-0.00138718575129767\\
-7574.16676840435	-0.00109847669622303\\
-7568.30667806129	-0.000746666095729551\\
-7562.44658771823	-0.000339972699795591\\
-7556.58649737517	0.000112076088802196\\
-7550.72640703211	0.000598867314853788\\
-7544.86631668905	0.00110895154258507\\
-7539.00622634599	0.00163031205512012\\
-7533.14613600293	0.00215064785350499\\
-7527.28604565987	0.00265766379220044\\
-7521.42595531681	0.00313936100899797\\
-7515.56586497375	0.00358432079081758\\
-7509.70577463069	0.00398197516092144\\
-7503.84568428763	0.00432285777721249\\
-7497.98559394457	0.00459882918638366\\
-7492.12550360151	0.00480327107548022\\
-7486.26541325845	0.00493124488621574\\
-7480.40532291539	0.00497961099312772\\
-7474.54523257233	0.00494710557064382\\
-7468.68514222927	0.00483437327082675\\
-7462.82505188622	0.00464395487261931\\
-7456.96496154316	0.00438023012724407\\
-7451.1048712001	0.00404931708139347\\
-7445.24478085704	0.0036589301905883\\
-7439.38469051398	0.00321820051139937\\
-7433.52460017092	0.00273746216191537\\
-7427.66450982786	0.00222801004266241\\
-7421.8044194848	0.00170183449662479\\
-7415.94432914174	0.00117133914015726\\
-7410.08423879868	0.000649048503737084\\
-7404.22414845562	0.000147312371606068\\
-7398.36405811256	-0.000321986201267342\\
-7392.5039677695	-0.000747712293557661\\
-7386.64387742644	-0.0011197432806885\\
-7380.78378708338	-0.00142920949135318\\
-7374.92369674032	-0.00166870563765352\\
-7369.06360639727	-0.00183246798045045\\
-7363.20351605421	-0.00191651299321907\\
-7357.34342571115	-0.00191873419030468\\
-7351.48333536809	-0.0018389547684456\\
-7345.62324502503	-0.00167893475050007\\
-7339.76315468197	-0.00144233239245914\\
-7333.90306433891	-0.00113462069513754\\
-7328.04297399585	-0.000762960923031298\\
-7322.18288365279	-0.000336036051262834\\
-7316.32279330973	0.000136151987313901\\
-7310.46270296667	0.000642516522328329\\
-7304.60261262361	0.00117114635007881\\
-7298.74252228055	0.00170958581632914\\
-7292.88243193749	0.00224512819346433\\
-7287.02234159443	0.00276511544347561\\
-7281.16225125137	0.0032572372966923\\
-7275.30216090832	0.00370982258164542\\
-7269.44207056526	0.00411211591289026\\
-7263.5819802222	0.00445453317931075\\
-7257.72188987914	0.00472888976538742\\
-7251.86179953608	0.0049285960728339\\
-7246.00170919302	0.00504881567328607\\
-7240.14161884996	0.00508658229843938\\
-7234.2815285069	0.00504087284005571\\
-7228.42143816384	0.0049126345663638\\
-7222.56134782078	0.00470476584106593\\
-7216.70125747772	0.0044220507260332\\
-7210.84116713466	0.00407104894102709\\
-7204.9810767916	0.00365994370832216\\
-7199.12098644854	0.00319835101011038\\
-7193.26089610548	0.00269709470247947\\
-7187.40080576242	0.00216795274406222\\
-7181.54071541936	0.00162338048697094\\
-7175.6806250763	0.00107621752968296\\
-7169.82053473324	0.000539385029281543\\
-7163.96044439018	2.55806072418099e-05\\
-7158.10035404712	-0.000453021949212453\\
-7152.24026370407	-0.000885061089799094\\
-7146.38017336101	-0.00126025717732863\\
-7140.52008301795	-0.00156965703806855\\
-7134.65999267489	-0.00180584757024491\\
-7128.79990233183	-0.00196313331618809\\
-7122.93981198877	-0.00203767374653313\\
-7117.07972164571	-0.00202757694959323\\
-7111.21963130265	-0.00193294744323167\\
-7105.35954095959	-0.00175588690663076\\
-7099.49945061653	-0.00150044773905863\\
-7093.63936027347	-0.0011725404670072\\
-7087.77926993041	-0.000779797112215492\\
-7081.91917958735	-0.000331393677184145\\
-7076.05908924429	0.000162164125179634\\
-7070.19899890123	0.000689286895977135\\
-7064.33890855817	0.00123757356929216\\
-7058.47881821511	0.00179410304284405\\
-7052.61872787205	0.00234573856389811\\
-7046.75863752899	0.00287943770187893\\
-7040.89854718593	0.00338256059547219\\
-7035.03845684288	0.00384316919069466\\
-7029.17836649982	0.0042503103878975\\
-7023.31827615676	0.00459427638360516\\
-7017.4581858137	0.00486683602111729\\
-7011.59809547064	0.00506143163847697\\
-7005.73800512758	0.00517333670794662\\
-6999.87791478452	0.00519977047984608\\
-6994.01782444146	0.0051399668529639\\
-6988.1577340984	0.00499519576923716\\
-6982.29764375534	0.00476873655094517\\
-6976.43755341228	0.00446580372984344\\
-6970.57746306922	0.00409342704370376\\
-6964.71737272616	0.00366028835835199\\
-6958.8572823831	0.00317651929661852\\
-6952.99719204004	0.00265346428939738\\
-6947.13710169699	0.00210341458817649\\
-6941.27701135393	0.0015393194740748\\
-6935.41692101087	0.0009744814463379\\
-6929.55683066781	0.00042224256351902\\
-6923.69674032475	-0.000104330668892562\\
-6917.83664998169	-0.000592756424064951\\
-6911.97655963863	-0.00103143338130802\\
-6906.11646929557	-0.00140991624614854\\
-6900.25637895251	-0.00171916442158967\\
-6894.39628860945	-0.00195175791239929\\
-6888.53619826639	-0.00210207530847515\\
-6882.67610792333	-0.00216642957996259\\
-6876.81601758027	-0.00214315840660015\\
-6870.95592723721	-0.00203266683163263\\
-6865.09583689415	-0.00183742115323662\\
-6859.23574655109	-0.00156189411605015\\
-6853.37565620803	-0.0012124626159133\\
-6847.51556586497	-0.000797260254116888\\
-6841.65547552191	-0.00032598814825652\\
-6835.79538517886	0.000190311602124636\\
-6829.9352948358	0.000739514507128819\\
-6824.07520449274	0.00130869828155682\\
-6818.21511414968	0.00188444676516167\\
-6812.35502380662	0.00245316602226645\\
-6806.49493346356	0.00300140517511289\\
-6800.6348431205	0.00351617440088875\\
-6794.77475277744	0.00398525257689921\\
-6788.91466243438	0.00439747728922471\\
-6783.05457209132	0.00474301032587694\\
-6777.19448174826	0.00501357234097803\\
-6771.3343914052	0.00520264109569365\\
-6765.47430106214	0.00530560853042484\\
-6759.61421071908	0.00531989288959067\\
-6753.75412037602	0.00524500317214631\\
-6747.89403003296	0.00508255430570568\\
-6742.0339396899	0.00483623260068631\\
-6736.17384934684	0.00451171221529959\\
-6730.31375900378	0.00411652452141069\\
-6724.45366866072	0.00365988337488168\\
-6718.59357831766	0.00315247034271061\\
-6712.7334879746	0.00260618489054492\\
-6706.87339763155	0.00203386537113897\\
-6701.01330728849	0.0014489873534096\\
-6695.15321694543	0.000865346378826853\\
-6689.29312660237	0.000296732611241846\\
-6683.43303625931	-0.000243394948793349\\
-6677.57294591625	-0.000742226991104452\\
-6671.71285557319	-0.00118790762803002\\
-6665.85276523013	-0.00156981613176464\\
-6659.99267488707	-0.00187881998418719\\
-6654.13258454401	-0.0021074932120618\\
-6648.27249420095	-0.00225029479024497\\
-6642.41240385789	-0.00230370282935873\\
-6636.55231351483	-0.0022663013013571\\
-6630.69222317177	-0.00213881717160385\\
-6624.83213282871	-0.00192410697380834\\
-6618.97204248565	-0.0016270930563418\\
-6613.1119521426	-0.00125465091792605\\
-6607.25186179954	-0.000815450207793016\\
-6601.39177145648	-0.000319753064928165\\
-6595.53168111342	0.000220825515518639\\
-6589.67159077036	0.000793589695060228\\
-6583.8115004273	0.00138506081597554\\
-6577.95141008424	0.00198129444312221\\
-6572.09131974118	0.00256820896834099\\
-6566.23122939812	0.00313191803605909\\
-6560.37113905506	0.00365905894498741\\
-6554.511048712	0.00413710926157759\\
-6548.65095836894	0.00455468414590147\\
-6542.79086802588	0.00490180733291645\\
-6536.93077768282	0.00517014932109206\\
-6531.07068733976	0.00535322708461923\\
-6525.21059699671	0.00544656052279753\\
-6519.35050665365	0.005447781874524\\
-6513.49041631059	0.005356695429094\\
-6507.63032596753	0.00517528603443037\\
-6501.77023562447	0.00490767611017306\\
-6495.91014528141	0.00456003208852832\\
-6490.05005493835	0.00414042240254238\\
-6484.18996459529	0.00365863028905291\\
-6478.32987425223	0.00312592574678473\\
-6472.46978390917	0.00255480196276792\\
-6466.60969356611	0.00195868236896549\\
-6460.74960322305	0.00135160519540629\\
-6454.88951287999	0.000747892930735089\\
-6449.02942253693	0.000161814470056469\\
-6443.16933219387	-0.000392752082520558\\
-6437.30924185081	-0.000902647990680954\\
-6431.44915150775	-0.00135574606517833\\
-6425.58906116469	-0.00174123904191137\\
-6419.72897082163	-0.00204989730375426\\
-6413.86888047857	-0.0022742898028516\\
-6408.00879013551	-0.00240896289796897\\
-6402.14869979245	-0.00245057280581024\\
-6396.2886094494	-0.00239796845315791\\
-6390.42851910634	-0.0022522226819947\\
-6384.56842876328	-0.00201661097626424\\
-6378.70833842022	-0.00169653811643794\\
-6372.84824807716	-0.00129941439902993\\
-6366.9881577341	-0.000834484252292827\\
-6361.12806739104	-0.000312611208892719\\
-6355.26797704798	0.000253975765548383\\
-6349.40788670492	0.000851968667284953\\
-6343.54779636186	0.00146729286793934\\
-6337.6877060188	0.00208543821521955\\
-6331.82761567574	0.00269180103443843\\
-6325.96752533268	0.00327202897234065\\
-6320.10743498962	0.00381236054331838\\
-6314.24734464656	0.00429995134713795\\
-6308.3872543035	0.00472317922852225\\
-6302.52716396044	0.00507192113012198\\
-6296.66707361738	0.00533779504811387\\
-6290.80698327432	0.00551436130853336\\
-6284.94689293126	0.00559727833517958\\
-6279.08680258821	0.005584409144124\\
-6273.22671224515	0.00547587595770156\\
-6267.36662190209	0.00527406154989258\\
-6261.50653155903	0.00498355719262695\\
-6255.64644121597	0.00461105833162487\\
-6249.78635087291	0.00416521035809779\\
-6243.92626052985	0.00365640802571516\\
-6238.06617018679	0.00309655316352792\\
-6232.20607984373	0.00249877632929415\\
-6226.34598950067	0.0018771289102097\\
-6220.48589915761	0.00124625288840901\\
-6214.62580881455	0.000621036029345132\\
-6208.76571847149	1.62606105484162e-05\\
-6202.90562812843	-0.000553746023853492\\
-6197.04553778538	-0.00107545082509682\\
-6191.18544744232	-0.00153643628271601\\
-6185.32535709926	-0.00192569592569518\\
-6179.4652667562	-0.00223389705270557\\
-6173.60517641314	-0.00245360442278085\\
-6167.74508607008	-0.00257945954782799\\
-6161.88499572702	-0.0026083112671582\\
-6156.02490538396	-0.00253929442639755\\
-6150.1648150409	-0.00237385470245888\\
-6144.30472469784	-0.00211571888554609\\
-6138.44463435478	-0.00177081121470925\\
-6132.58454401172	-0.00134711764011586\\
-6126.72445366866	-0.000854501117769777\\
-6120.8643633256	-0.000304472205703852\\
-6115.00427298254	0.00029007970600882\\
-6109.14418263948	0.000915188258319774\\
-6103.28409229642	0.00155613801297503\\
-6097.42400195336	0.0021978106739798\\
-6091.5639116103	0.00282504149783154\\
-6085.70382126725	0.00342297749337938\\
-6079.84373092419	0.00397742895308915\\
-6073.98364058113	0.00447520599931261\\
-6068.12355023807	0.00490443216626425\\
-6062.26345989501	0.00525482756537569\\
-6056.40336955195	0.00551795488594006\\
-6050.54327920889	0.00568742234715295\\
-6044.68318886583	0.00575903872375827\\
-6038.82309852277	0.00573091668791684\\
-6032.96300817971	0.00560352192679455\\
-6027.10291783665	0.0053796667672068\\
-6021.24282749359	0.00506444834994351\\
-6015.38273715053	0.00466513270465152\\
-6009.52264680747	0.00419098735696594\\
-6003.66255646441	0.00365306632065881\\
-5997.80246612135	0.00306395245963145\\
-5991.94237577829	0.00243746322122374\\
-5986.08228543523	0.00178832661968941\\
-5980.22219509217	0.00113183506484267\\
-5974.36210474911	0.000483485169084496\\
-5968.50201440605	-0.000141387986402107\\
-5962.64192406299	-0.000727973497765935\\
-5956.78183371994	-0.00126233562435079\\
-5950.92174337688	-0.00173174479983086\\
-5945.06165303382	-0.00212498080952864\\
-5939.20156269076	-0.00243260091635069\\
-5933.3414723477	-0.00264716653143206\\
-5927.48138200464	-0.00276342299202658\\
-5921.62129166158	-0.00277842810518147\\
-5915.76120131852	-0.00269162631846643\\
-5909.90111097546	-0.00250486665543998\\
-5904.0410206324	-0.00222236387948641\\
-5898.18093028934	-0.00185060368842271\\
-5892.32083994628	-0.0013981940668296\\
-5886.46074960322	-0.000875666198067985\\
-5880.60065926016	-0.000295229537481451\\
-5874.7405689171	0.000329513260952879\\
-5868.88047857404	0.000983884911402183\\
-5863.02038823099	0.00165247809928863\\
-5857.16029788793	0.00231951803323882\\
-5851.30020754487	0.00296923441796199\\
-5845.44011720181	0.00358623408019082\\
-5839.58002685875	0.00415586544681112\\
-5833.71993651569	0.00466456624824702\\
-5827.85984617263	0.00510018619935562\\
-5821.99975582957	0.00545227698349191\\
-5816.13966548651	0.00571234262291438\\
-5810.27957514345	0.00587404423898748\\
-5804.41948480039	0.00593335427217524\\
-5798.55939445733	0.00588865641368789\\
-5792.69930411427	0.00574078877660408\\
-5786.83921377121	0.00549302916919449\\
-5780.97912342815	0.00515102269908503\\
-5775.11903308509	0.00472265329803579\\
-5769.25894274204	0.00421786208702458\\
-5763.39885239898	0.00364841676164796\\
-5757.53876205592	0.00302763734357482\\
-5751.67867171286	0.00237008468625006\\
-5745.8185813698	0.00169121901555685\\
-5739.95849102674	0.00100703650901828\\
-5734.09840068368	0.000333692453135872\\
-5728.23831034062	-0.000312880147372273\\
-5722.37821999756	-0.000917347507992998\\
-5716.5181296545	-0.00146533863688936\\
-5710.65803931144	-0.00194378692458454\\
-5704.79794896838	-0.00234124163505816\\
-5698.93785862532	-0.00264814187668898\\
-5693.07776828226	-0.00285704650173869\\
-5687.2176779392	-0.00296281441032375\\
-5681.35758759614	-0.00296273089486637\\
-5675.49749725308	-0.00285657692616216\\
-5669.63740691002	-0.002646639623084\\
-5663.77731656696	-0.00233766353442861\\
-5657.9172262239	-0.00193674375738302\\
-5652.05713588084	-0.00145316329447726\\
-5646.19704553779	-0.000898178371887775\\
-5640.33695519473	-0.000284756680071274\\
-5634.47686485167	0.000372725380382555\\
-5628.61677450861	0.00105881938305717\\
-5622.75668416555	0.00175736760918476\\
-5616.89659382249	0.00245188331715586\\
-5611.03650347943	0.00312593960227707\\
-5605.17641313637	0.00376355767896498\\
-5599.31632279331	0.00434958541164809\\
-5593.45623245025	0.00487005713112554\\
-5587.59614210719	0.00531252619536609\\
-5581.73605176413	0.00566636238044875\\
-5575.87596142107	0.00592300699925457\\
-5570.01587107801	0.00607617962991959\\
-5564.15578073495	0.00612203146568702\\
-5558.29569039189	0.00605924154656835\\
-5552.43560004883	0.00588905347689357\\
-5546.57550970577	0.00561525162976472\\
-5540.71541936271	0.00524407727078112\\
-5534.85532901965	0.00478408644935242\\
-5528.9952386766	0.00424595288912246\\
-5523.13514833354	0.00364222041250649\\
-5517.27505799048	0.00298701063816723\\
-5511.41496764742	0.00229569275805176\\
-5505.55487730436	0.00158452311235249\\
-5499.6947869613	0.000870263010159422\\
-5493.83469661824	0.000169783776807478\\
-5487.97460627518	-0.000500331670064663\\
-5482.11451593212	-0.00112418131098558\\
-5476.25442558906	-0.00168692129635067\\
-5470.394335246	-0.00217511905284908\\
-5464.53424490294	-0.00257707382942487\\
-5458.67415455988	-0.00288309703758909\\
-5452.81406421682	-0.00308574567399403\\
-5446.95397387376	-0.00318000320655496\\
-5441.09388353071	-0.00316340353342987\\
-5435.23379318765	-0.00303609495944454\\
-5429.37370284459	-0.00280084254465944\\
-5423.51361250153	-0.00246296863114866\\
-5417.65352215847	-0.00203023281513701\\
-5411.79343181541	-0.0015126540634109\\
-5405.93334147235	-0.000922279046914138\\
-5400.07325112929	-0.000272902043125599\\
-5394.21316078623	0.000420257084672758\\
-5388.35307044317	0.00114090931835258\\
-5382.49298010011	0.00187207898338601\\
-5376.63288975705	0.00259650333742868\\
-5370.77279941399	0.0032970398478728\\
-5364.91270907093	0.00395707155384661\\
-5359.05261872787	0.00456090093196473\\
-5353.19252838481	0.0050941229348565\\
-5347.33243804175	0.00554396834214876\\
-5341.47234769869	0.00589960924439551\\
-5335.61225735563	0.0061524193568112\\
-5329.75216701258	0.00629618290975367\\
-5323.89207666952	0.00632724706264815\\
-5318.03198632646	0.00624461411186288\\
-5312.1718959834	0.00604997117492418\\
-5306.31180564034	0.00574765650713192\\
-5300.45171529728	0.00534456309915285\\
-5294.59162495422	0.00484998168923948\\
-5288.73153461116	0.0042753867587451\\
-5282.8714442681	0.00363417043458561\\
-5277.01135392504	0.00294133046499977\\
-5271.15126358198	0.00221311953307835\\
-5265.29117323892	0.0014666641034942\\
-5259.43108289586	0.000719561736366034\\
-5253.5709925528	-1.05336688027332e-05\\
-5247.71090220974	-0.000706328928454045\\
-5241.85081186668	-0.00135130108202774\\
-5235.99072152362	-0.00193008965858885\\
-5230.13063118056	-0.00242886239529102\\
-5224.2705408375	-0.00283564570531639\\
-5218.41045049444	-0.00314061200521067\\
-5212.55036015138	-0.00333631701190556\\
-5206.69026980832	-0.00341788128565944\\
-5200.83017946527	-0.0033831115981668\\
-5194.97008912221	-0.00323255911665472\\
-5189.10999877915	-0.00296951287963104\\
-5183.24990843609	-0.0025999285654072\\
-5177.38981809303	-0.00213229408257217\\
-5171.52972774997	-0.00157743500799598\\
-5165.66963740691	-0.000948264325738051\\
-5159.80954706385	-0.00025948224664403\\
-5153.94945672079	0.000472766918436309\\
-5148.08936637773	0.00123127282810813\\
-5142.22927603467	0.00199816327395481\\
-5136.36918569161	0.00275532495227472\\
-5130.50909534855	0.00348483094289926\\
-5124.64900500549	0.00416936481180091\\
-5118.78891466243	0.00479263130979069\\
-5112.92882431937	0.00533974393228034\\
-5107.06873397632	0.00579758012743732\\
-5101.20864363326	0.00615509568318416\\
-5095.3485532902	0.00640359076642127\\
-5089.48846294714	0.0065369212118717\\
-5083.62837260408	0.00655164993633903\\
-5077.76828226102	0.00644713475472385\\
-5071.90819191796	0.00622555036950828\\
-5066.0481015749	0.00589184385245099\\
-5060.18801123184	0.00545362450936383\\
-5054.32792088878	0.00492099056791236\\
-5048.46783054572	0.00430629663020967\\
-5042.60774020266	0.00362386723746691\\
-5036.7476498596	0.00288966318301387\\
-5030.88755951654	0.00212090834009687\\
-5025.02746917348	0.00133568572432813\\
-5019.16737883043	0.000552512258170917\\
-5013.30728848737	-0.000210097768708227\\
-5007.44719814431	-0.00093407025742763\\
-5001.58710780125	-0.00160219954101694\\
-4995.72701745819	-0.00219855716803276\\
-4989.86692711513	-0.00270887152233677\\
-4984.00683677207	-0.00312086923776121\\
-4978.14674642901	-0.00342457024591402\\
-4972.28665608595	-0.00361252937091565\\
-4966.42656574289	-0.00368001862940849\\
-4960.56647539983	-0.00362514578034095\\
-4954.70638505677	-0.00344890616422852\\
-4948.84629471371	-0.00315516643918719\\
-4942.98620437065	-0.00275058042713254\\
-4937.12611402759	-0.00224443888751755\\
-4931.26602368453	-0.00164845660097083\\
-4925.40593334147	-0.00097650163525037\\
-4919.54584299841	-0.000244273043748826\\
-4913.68575265535	0.000531065519587303\\
-4907.82566231229	0.00133128772401368\\
-4901.96557196923	0.0021375326189662\\
-4896.10548162617	0.00293074877724391\\
-4890.24539128312	0.00369214407177429\\
-4884.38530094006	0.00440363047408418\\
-4878.525210597	0.00504825335207591\\
-4872.66512025394	0.00561059508553304\\
-4866.80502991088	0.00607714339672218\\
-4860.94493956782	0.00643661560344464\\
-4855.08484922476	0.0066802310206236\\
-4849.2247588817	0.00680192493989265\\
-4843.36466853864	0.00679849898248883\\
-4837.50457819558	0.00666970410702392\\
-4831.64448785252	0.00641825413822676\\
-4825.78439750946	0.00604976931606611\\
-4819.9243071664	0.00557265101788421\\
-4814.06421682334	0.004997890433962\\
-4808.20412648028	0.00433881554615089\\
-4802.34403613722	0.00361078222517441\\
-4796.48394579416	0.00283081660068235\\
-4790.6238554511	0.00201721702516986\\
-4784.76376510804	0.00118912493071111\\
-4778.90367476498	0.000366074636237159\\
-4773.04358442193	-0.000432467313435233\\
-4767.18349407887	-0.00118756242026539\\
-4761.32340373581	-0.00188124924158634\\
-4755.46331339275	-0.00249697049601303\\
-4749.60322304969	-0.00301996811788801\\
-4743.74313270663	-0.00343763684092504\\
-4737.88304236357	-0.00373982784746673\\
-4732.02295202051	-0.00391909517608863\\
-4726.16286167745	-0.003970878913277\\
-4720.30277133439	-0.00389362067287524\\
-4714.44268099133	-0.00368880845362635\\
-4708.58259064827	-0.00336094962661905\\
-4702.72250030521	-0.0029174724982079\\
-4696.86240996215	-0.00236855858275609\\
-4691.0023196191	-0.0017269093620529\\
-4685.14222927604	-0.00100745286604061\\
-4679.28213893298	-0.000226996845361649\\
-4673.42204858992	0.000596163413792659\\
-4667.56195824686	0.00144267348875718\\
-4661.7018679038	0.00229257440960022\\
-4655.84177756074	0.00312577276856715\\
-4649.98168721768	0.00392251528471928\\
-4644.12159687462	0.00466385662267876\\
-4638.26150653156	0.00533210939437569\\
-4632.4014161885	0.00591126566192369\\
-4626.54132584544	0.00638737990567538\\
-4620.68123550238	0.00674890430226135\\
-4614.82114515932	0.00698696826112283\\
-4608.96105481626	0.00709559546077547\\
-4603.1009644732	0.00707185308507725\\
-4597.24087413014	0.00691592954726432\\
-4591.38078378708	0.00663113866645557\\
-4585.52069344402	0.00622384999610994\\
-4579.66060310097	0.00570334674537006\\
-4573.80051275791	0.00508161444939937\\
-4567.94042241485	0.00437306518987947\\
-4562.08033207179	0.00359420370226787\\
-4556.22024172873	0.00276324309694081\\
-4550.36015138567	0.00189967913315761\\
-4544.50006104261	0.00102383298922104\\
-4538.63997069955	0.000156373243617816\\
-4532.77988035649	-0.000682171696352505\\
-4526.91979001343	-0.00147189922510477\\
-4521.05969967037	-0.00219400474851233\\
-4515.19960932731	-0.00283122927761571\\
-4509.33951898425	-0.00336827176109309\\
-4503.47942864119	-0.0037921563139937\\
-4497.61933829813	-0.00409254553806755\\
-4491.75924795507	-0.00426199237767411\\
-4485.89915761201	-0.00429612438516661\\
-4480.03906726895	-0.00419375584983331\\
-4474.17897692589	-0.00395692493507415\\
-4468.31888658283	-0.00359085473168586\\
-4462.45879623977	-0.00310383893029356\\
-4456.59870589671	-0.00250705459753157\\
-4450.73861555366	-0.0018143062706842\\
-4444.8785252106	-0.00104170721982999\\
-4439.01843486754	-0.000207305227401601\\
-4433.15834452448	0.000669338432613838\\
-4427.29825418142	0.00156760700324504\\
-4421.43816383836	0.00246631252594408\\
-4415.5780734953	0.00334419502831241\\
-4409.71798315224	0.00418042486645277\\
-4403.85789280918	0.00495509636141761\\
-4397.99780246612	0.00564970104074225\\
-4392.13771212306	0.00624756924310367\\
-4386.27762178	0.00673426955943134\\
-4380.41753143694	0.0070979565481617\\
-4374.55744109388	0.00732965835699008\\
-4368.69735075082	0.00742349727841564\\
-4362.83726040776	0.00737683782939752\\
-4356.97717006471	0.00719035864244292\\
-4351.11707972165	0.00686804624474679\\
-4345.25698937859	0.00641711063969909\\
-4339.39689903553	0.00584782445449976\\
-4333.53680869247	0.00517328922701587\\
-4327.67671834941	0.0044091341367855\\
-4321.81662800635	0.00357315409801373\\
-4315.95653766329	0.00268489558418496\\
-4310.09644732023	0.0017651998147762\\
-4304.23635697717	0.000835713969562203\\
-4298.37626663411	-8.16181154118465e-05\\
-4292.51617629105	-0.000965073798040115\\
-4286.65608594799	-0.00179366581587415\\
-4280.79599560493	-0.00254764064671001\\
-4274.93590526187	-0.00320894922013131\\
-4269.07581491882	-0.00376167877409589\\
-4263.21572457576	-0.00419243553368495\\
-4257.3556342327	-0.00449066902062229\\
-4251.49554388964	-0.00464893015194464\\
-4245.63545354658	-0.00466305682746218\\
-4239.77536320352	-0.00453228239845683\\
-4233.91527286046	-0.00425926421897166\\
-4228.0551825174	-0.00385003135867849\\
-4222.19509217434	-0.0033138524653146\\
-4216.33500183128	-0.00266302665218088\\
-4210.47491148822	-0.00191260211466732\\
-4204.61482114516	-0.00108002890107251\\
-4198.7547308021	-0.000184753838597746\\
-4192.89464045904	0.000752232993288131\\
-4187.03455011598	0.00170888967644035\\
-4181.17445977292	0.00266264026045302\\
-4175.31436942986	0.00359090680916101\\
-4169.4542790868	0.00447164315926095\\
-4163.59418874374	0.00528385778461535\\
-4157.73409840068	0.00600811337621788\\
-4151.87400805762	0.00662699126065033\\
-4146.01391771456	0.00712550957154266\\
-4140.15382737151	0.00749148514694538\\
-4134.29373702845	0.00771583042410058\\
-4128.43364668539	0.00779277810868319\\
-4122.57355634233	0.0077200280816226\\
-4116.71346599927	0.00749881282498909\\
-4110.85337565621	0.00713387956008516\\
-4104.99328531315	0.00663338925592373\\
-4099.13319497009	0.00600873462693156\\
-4093.27310462703	0.00527428116240368\\
-4087.41301428397	0.00444703705929061\\
-4081.55292394091	0.00354625962961642\\
-4075.69283359785	0.00259300727804757\\
-4069.83274325479	0.00160964746120418\\
-4063.97265291173	0.000619332112477259\\
-4058.11256256867	-0.000354547176565682\\
-4052.25247222561	-0.00128891141044684\\
-4046.39238188255	-0.00216154085385285\\
-4040.53229153949	-0.00295160316964667\\
-4034.67220119643	-0.00364015049060143\\
-4028.81211085337	-0.00421057357706133\\
-4022.95202051031	-0.00464900218843632\\
-4017.09193016726	-0.00494464203082911\\
-4011.2318398242	-0.00509004011016201\\
-4005.37174948114	-0.00508127198444811\\
-3999.51165913808	-0.00491804623215694\\
-3993.65156879502	-0.00460372339460904\\
-3987.79147845196	-0.00414524866070209\\
-3981.9313881089	-0.00355299959823261\\
-3976.07129776584	-0.00284055224740289\\
-3970.21120742278	-0.00202437083091371\\
-3964.35111707972	-0.0011234281582107\\
-3958.49102673666	-0.000158765461476173\\
-3952.6309363936	0.000846998138636516\\
-3946.77084605054	0.00187019511530769\\
-3940.91075570748	0.00288666558511696\\
-3935.05066536443	0.00387232687513467\\
-3929.19057502137	0.00480374319374668\\
-3923.33048467831	0.00565868194429951\\
-3917.47039433525	0.00641664349010055\\
-3911.61030399219	0.00705935176191942\\
-3905.75021364913	0.00757119398092691\\
-3899.89012330607	0.00793959893294186\\
-3894.03003296301	0.00815534464361101\\
-3888.16994261995	0.00821278794353416\\
-3882.30985227689	0.00811001022850894\\
-3876.44976193383	0.00784887568288021\\
-3870.58967159077	0.00743500028746927\\
-3864.72958124771	0.00687763203637733\\
-3858.86949090465	0.00618944488518528\\
-3853.00940056159	0.00538625100051817\\
-3847.14931021853	0.00448663782634899\\
-3841.28921987547	0.00351153828073021\\
-3835.42912953241	0.00248374400730378\\
-3829.56903918935	0.00142737298294409\\
-3823.7089488463	0.000367303903315773\\
-3817.84885850324	-0.00067140940488208\\
-3811.98876816018	-0.00166412977740159\\
-3806.12867781712	-0.00258722074885845\\
-3800.26858747406	-0.00341860900343866\\
-3794.408497131	-0.00413831187208375\\
-3788.54840678794	-0.00472891728670607\\
-3782.68831644488	-0.0051760046804829\\
-3776.82822610182	-0.00546849667736869\\
-3770.96813575876	-0.00559893301236882\\
-3765.1080454157	-0.00556365993117846\\
-3759.24795507264	-0.00536293029028585\\
-3753.38786472958	-0.00500091166995849\\
-3747.52777438652	-0.00448560197798168\\
-3741.66768404346	-0.00382865420313579\\
-3735.8075937004	-0.00304511413168779\\
-3729.94750335734	-0.00215307690828768\\
-3724.08741301428	-0.00117327026290317\\
-3718.22732267122	-0.000128573984568956\\
-3712.36723232816	0.000956513234835723\\
-3706.5071419851	0.00205644718068208\\
-3700.64705164204	0.00314523831719472\\
-3694.78696129899	0.00419706472580414\\
-3688.92687095593	0.00518688334262058\\
-3683.06678061287	0.00609102504420597\\
-3677.20669026981	0.00688775946002173\\
-3671.34659992675	0.00755781605084114\\
-3665.48650958369	0.00808484897791946\\
-3659.62641924063	0.00845583456773837\\
-3653.76632889757	0.00866139172788757\\
-3647.90623855451	0.00869601745434875\\
-3642.04614821145	0.00855823154619699\\
-3636.18605786839	0.00825062676708455\\
-3630.32596752533	0.00777982291194471\\
-3624.46587718227	0.00715632550136256\\
-3618.60578683921	0.00639429208248266\\
-3612.74569649615	0.00551121130918611\\
-3606.88560615309	0.00452750205278483\\
-3601.02551581004	0.00346604171332176\\
-3595.16542546698	0.00235163460776858\\
-3589.30533512392	0.00121043276809243\\
-3583.44524478086	6.93226548304334e-05\\
-3577.5851544378	-0.00104470785231902\\
-3571.72506409474	-0.00210520730384014\\
-3565.86497375168	-0.00308688831490519\\
-3560.00488340862	-0.00396623011362101\\
-3554.14479306556	-0.00472204166663773\\
-3548.2847027225	-0.00533597192109993\\
-3542.42461237944	-0.00579295489945733\\
-3536.56452203638	-0.00608157887400178\\
-3530.70443169332	-0.00619437060029849\\
-3524.84434135026	-0.00612798755876339\\
-3518.9842510072	-0.00588331330029163\\
-3513.12416066415	-0.00546545326007842\\
-3507.26407032109	-0.00488363074642171\\
-3501.40397997803	-0.0041509851657094\\
-3495.54388963497	-0.00328427686532124\\
-3489.68379929191	-0.00230350519651384\\
-3483.82370894885	-0.00123144847820071\\
-3477.96361860579	-9.3136419914711e-05\\
-3472.10352826273	0.00108473279853733\\
-3466.24343791967	0.00227441723806101\\
-3460.38334757661	0.00344778310696529\\
-3454.52325723355	0.00457696856755867\\
-3448.66316689049	0.00563504380392908\\
-3442.80307654743	0.00659665173717525\\
-3436.94298620437	0.0074386141681885\\
-3431.08289586131	0.00814048888322185\\
-3425.22280551825	0.00868506435612603\\
-3419.36271517519	0.00905878010104059\\
-3413.50262483213	0.00925206243771255\\
-3407.64253448907	0.00925956738492283\\
-3401.78244414601	0.00908032456041693\\
-3395.92235380295	0.00871777827683565\\
-3390.06226345989	0.00817972443606576\\
-3384.20217311684	0.00747814428085077\\
-3378.34208277378	0.00662893850202409\\
-3372.48199243072	0.00565156756964951\\
-3366.62190208766	0.00456860639406005\\
-3360.7618117446	0.00340522348184922\\
-3354.90172140154	0.00218859657776847\\
-3349.04163105848	0.000947278333604621\\
-3343.18154071542	-0.000289473216962629\\
-3337.32145037236	-0.00149238370437206\\
-3331.4613600293	-0.00263285479000728\\
-3325.60126968624	-0.00368364350610682\\
-3319.74117934318	-0.00461951245901207\\
-3313.88108900012	-0.00541783540592309\\
-3308.02099865706	-0.00605914370243496\\
-3302.160908314	-0.00652760045285251\\
-3296.30081797094	-0.00681139084674552\\
-3290.44072762788	-0.00690301909406241\\
-3284.58063728482	-0.00679950453741863\\
-3278.72054694176	-0.00650247186787117\\
-3272.8604565987	-0.00601813285091013\\
-3267.00036625564	-0.00535715951998051\\
-3261.14027591259	-0.00453445135847831\\
-3255.28018556953	-0.00356880150643534\\
-3249.42009522647	-0.0024824694344331\\
-3243.56000488341	-0.00130066977059606\\
-3237.69991454035	-5.09889885340821e-05\\
-3231.83982419729	0.0012372565793532\\
-3225.97973385423	0.00253370649345193\\
-3220.11964351117	0.00380766971341746\\
-3214.25955316811	0.00502884912276843\\
-3208.39946282505	0.00616805999901185\\
-3202.53937248199	0.00719792542440139\\
-3196.67928213893	0.00809353209863417\\
-3190.81919179587	0.00883303087731186\\
-3184.95910145281	0.00939816759370712\\
-3179.09901110976	0.00977473130189391\\
-3173.2389207667	0.00995290897301492\\
-3167.37883042364	0.00992753783331191\\
-3161.51874008058	0.00969824891316643\\
-3155.65864973752	0.00926949791224047\\
-3149.79855939446	0.00865048213370397\\
-3143.9384690514	0.00785494492196435\\
-3138.07837870834	0.00690087170401437\\
-3132.21828836528	0.00581008431329038\\
-3126.35819802222	0.00460774270801342\\
-3120.49810767916	0.00332176542560803\\
-3114.6380173361	0.00198218208787227\\
-3108.77792699304	0.00062043293757984\\
-3102.91783664998	-0.00073136829003325\\
-3097.05774630692	-0.00204119091566028\\
-3091.19765596386	-0.00327784549821102\\
-3085.3375656208	-0.00441172574216919\\
-3079.47747527774	-0.00541551652062163\\
-3073.61738493469	-0.0062648511204916\\
-3067.75729459163	-0.00693890193505245\\
-3061.89720424857	-0.0074208903251861\\
-3056.03711390551	-0.00769850321069726\\
-3050.17702356245	-0.007764206095785\\
-3044.31693321939	-0.00761544462933128\\
-3038.45684287633	-0.00725472939405607\\
-3032.59675253327	-0.00668960135077654\\
-3026.73666219021	-0.00593247817007214\\
-3020.87657184715	-0.0050003844974638\\
-3015.01648150409	-0.00391457195439474\\
-3009.15639116103	-0.00270003730960112\\
-3003.29630081797	-0.00138494970234729\\
nan	nan\\
-2991.57612013185	0.00142231272316089\\
-2985.71602978879	0.00284844103973382\\
-2979.85593944573	0.00424457763248576\\
-2973.99584910267	0.0055774538797864\\
-2968.13575875961	0.00681512974368829\\
-2962.27566841655	0.00792775625113928\\
-2956.41557807349	0.00888829241280049\\
-2950.55548773043	0.00967315940618472\\
-2944.69539738738	0.0102628162442474\\
-2938.83530704432	0.0106422429257624\\
-2932.97521670126	0.0108013191729026\\
-2927.1151263582	0.0107350892715334\\
-2921.25503601514	0.0104439061639191\\
-2915.39494567208	0.00993345076560541\\
-2909.53485532902	0.00921462539855497\\
-2903.67476498596	0.00830332320286377\\
-2897.8146746429	0.00722007832548872\\
-2891.95458429984	0.0059896045260749\\
-2886.09449395678	0.00464023251354548\\
-2880.23440361372	0.00320325877352022\\
-2874.37431327066	0.00171222080351519\\
-2868.5142229276	0.00020211548993306\\
-2862.65413258454	-0.00129142120849967\\
-2856.79404224148	-0.00273295409464437\\
-2850.93395189842	-0.00408808834539056\\
-2845.07386155537	-0.00532428906249935\\
-2839.21377121231	-0.00641166128931641\\
-2833.35368086925	-0.00732367184756998\\
-2827.49359052619	-0.00803779562351115\\
-2821.63350018313	-0.00853607061840782\\
-2815.77340984007	-0.00880554815098393\\
-2809.91331949701	-0.00883862699654855\\
-2804.05322915395	-0.00863326292849645\\
-2798.19313881089	-0.00819304802201049\\
-2792.33304846783	-0.00752715712211438\\
-2786.47295812477	-0.00665016200192275\\
-2780.61286778171	-0.00558171686171607\\
-2774.75277743865	-0.00434612187831032\\
-2768.89268709559	-0.00297177442992492\\
-2763.03259675254	-0.00149052032857914\\
-2757.17250640948	6.30801788299924e-05\\
-2751.31241606642	0.00165255478330148\\
-2745.45232572336	0.00324037129351002\\
-2739.5922353803	0.00478881977361752\\
-2733.73214503724	0.00626090397022931\\
-2727.87205469418	0.0076212210206767\\
-2722.01196435112	0.0088368085913962\\
-2716.15187400806	0.00987793924262647\\
-2710.291783665	0.0107188429453168\\
-2704.43169332194	0.0113383402598657\\
-2698.57160297888	0.0117203706928083\\
-2692.71151263582	0.0118544031335403\\
-2686.85142229276	0.0117357179755324\\
-2680.9913319497	0.0113655534972052\\
-2675.13124160664	0.0107511122353016\\
-2669.27115126358	0.00990542636467931\\
-2663.41106092052	0.00884708441849526\\
-2657.55097057746	0.00759982497022164\\
-2651.6908802344	0.00619200607336325\\
-2645.83078989134	0.00465596223862386\\
-2639.97069954828	0.00302726345532345\\
-2634.11060920523	0.00134389316210531\\
-2628.25051886217	-0.000354635910199807\\
-2622.39042851911	-0.00202820751994197\\
-2616.53033817605	-0.00363704770804635\\
-2610.67024783299	-0.00514266838524258\\
-2604.81015748993	-0.00650878641024377\\
-2598.95006714687	-0.00770219629457858\\
-2593.08997680381	-0.00869357562946508\\
-2587.22988646075	-0.00945820378437724\\
-2581.36979611769	-0.00997657635428583\\
-2575.50970577463	-0.010234900178849\\
-2569.64961543157	-0.0102254564807398\\
-2563.78952508851	-0.00994682270303402\\
-2557.92943474545	-0.00940394689773573\\
-2552.06934440239	-0.00860807195946099\\
-2546.20925405933	-0.00757651052092272\\
-2540.34916371627	-0.00633227485948068\\
-2534.48907337321	-0.00490356960841335\\
-2528.62898303015	-0.00332315835780078\\
-2522.76889268709	-0.00162761827576789\\
-2516.90880234403	0.000143500384176332\\
-2511.04871200098	0.00194859066912146\\
-2505.18862165792	0.00374496006777452\\
-2499.32853131486	0.00548983383634214\\
-2493.4684409718	0.0071413670297236\\
-2487.60835062874	0.00865964139479158\\
-2481.74826028568	0.0100076234812093\\
-2475.88816994262	0.0111520610785373\\
-2470.02807959956	0.0120642963894758\\
-2464.1679892565	0.0127209761646408\\
-2458.30789891344	0.0131046413218598\\
-2452.44780857038	0.013204181292042\\
-2446.58771822732	0.0130151414254016\\
-2440.72762788426	0.0125398751761778\\
-2434.8675375412	0.0117875363843739\\
-2429.00744719815	0.0107739107154888\\
-2423.14735685509	0.00952108910289104\\
-2417.28726651203	0.00805698978402546\\
-2411.42717616897	0.00641473913918165\\
-2405.56708582591	0.00463192494249867\\
-2399.70699548285	0.0027497387414227\\
-2393.84690513979	0.000812026814088753\\
-2387.98681479673	-0.00113572855030081\\
-2382.12672445367	-0.00304747390285287\\
-2376.26663411061	-0.00487767081481468\\
-2370.40654376755	-0.00658237789869062\\
-2364.54645342449	-0.00812030309476438\\
-2358.68636308143	-0.00945380116082388\\
-2352.82627273837	-0.0105497924085143\\
-2346.96618239531	-0.0113805804089319\\
-2341.10609205225	-0.0119245486059466\\
-2335.24600170919	-0.0121667184788795\\
-2329.38591136613	-0.012099155029404\\
-2323.52582102307	-0.0117212088592692\\
-2317.66573068002	-0.0110395878732599\\
-2311.80564033696	-0.0100682556059625\\
-2305.9455499939	-0.00882815723345431\\
-2300.08545965084	-0.00734677839859645\\
-2294.22536930778	-0.00565754595920392\\
-2288.36527896472	-0.00379908356424504\\
-2282.50518862166	-0.00181433848336023\\
-2276.6450982786	0.00025040072141383\\
-2270.78500793554	0.00234657336730572\\
-2264.92491759248	0.00442447812832633\\
-2259.06482724942	0.00643444061830099\\
-2253.20473690636	0.00832798981830225\\
-2247.3446465633	0.0100590156610343\\
-2241.48455622024	0.0115848803034256\\
-2235.62446587718	0.0128674564878283\\
-2229.76437553412	0.0138740678917618\\
-2223.90428519106	0.0145783084693991\\
-2218.044194848	0.014960720446091\\
-2212.18410450494	0.0150093137914393\\
-2206.32401416188	0.0147199135818754\\
-2200.46392381882	0.0140963256054732\\
-2194.60383347576	0.0131503147588297\\
-2188.74374313271	0.0119013951481239\\
-2182.88365278965	0.0103764352298235\\
-2177.02356244659	0.00860908570863427\\
-2171.16347210353	0.00663904214923177\\
-2165.30338176047	0.00451115825297398\\
-2159.44329141741	0.0022744294019185\\
-2153.58320107435	-1.91307012518548e-05\\
-2147.72311073129	-0.00231569475419591\\
-2141.86302038823	-0.00456088025260645\\
-2136.00293004517	-0.00670103237302143\\
-2130.14283970211	-0.00868450131249065\\
-2124.28274935905	-0.010462883917712\\
-2118.42265901599	-0.0119922000052796\\
-2112.56256867293	-0.0132339750428316\\
-2106.70247832987	-0.0141562028041475\\
-2100.84238798681	-0.0147341641942396\\
-2094.98229764376	-0.0149510816021221\\
-2089.1222073007	-0.0147985918175227\\
-2083.26211695764	-0.0142770246534892\\
-2077.40202661458	-0.013395478866648\\
-2071.54193627152	-0.0121716916531481\\
-2065.68184592846	-0.0106317028143412\\
-2059.8217555854	-0.00880931952503118\\
-2053.96166524234	-0.00674539237829642\\
-2048.10157489928	-0.00448691792266446\\
-2042.24148455622	-0.00208598712897101\\
-2036.38139421316	0.000401396964019029\\
-2030.5213038701	0.0029166058741285\\
-2024.66121352704	0.00539974730823386\\
-2018.80112318398	0.00779107048717147\\
-2012.94103284092	0.0100323825085114\\
-2007.08094249787	0.0120684425051182\\
-2001.22085215481	0.0138483005413317\\
-1995.36076181175	0.0153265491510133\\
-1989.50067146869	0.0164644571434375\\
-1983.64058112563	0.01723095775178\\
-1977.78049078257	0.017603466330637\\
-1971.92040043951	0.0175685065489551\\
-1966.06031009645	0.0171221282966627\\
-1960.20021975339	0.0162701052353912\\
-1954.34012941033	0.0150279049613457\\
-1948.48003906727	0.0134204300100195\\
-1942.61994872421	0.0114815332838823\\
-1936.75985838115	0.00925331681943426\\
-1930.89976803809	0.00678522797837604\\
-1925.03967769503	0.00413297206022388\\
-1919.17958735197	0.00135726484147846\\
-1913.31949700891	-0.00147754743827254\\
-1907.45940666585	-0.00430496970760822\\
-1901.59931632279	-0.00705790764695253\\
-1895.73922597973	-0.00967024958476721\\
-1889.87913563667	-0.0120784439005949\\
-1884.01904529361	-0.0142230348116931\\
-1878.15895495056	-0.0160501199614507\\
-1872.2988646075	-0.0175126946320237\\
-1866.43877426444	-0.0185718496407358\\
-1860.57868392138	-0.0191977930156998\\
-1854.71859357832	-0.019370669308774\\
-1848.85850323526	-0.0190811548232647\\
-1842.9984128922	-0.0183308120121746\\
-1837.13832254914	-0.017132191727909\\
-1831.27823220608	-0.015508677757939\\
-1825.41814186302	-0.0134940740375226\\
-1819.55805151996	-0.0111319409451808\\
-1813.6979611769	-0.00847469302921748\\
-1807.83787083384	-0.00558247624155257\\
-1801.97778049078	-0.0025218481337946\\
-1796.11769014772	0.000635710629293177\\
-1790.25759980466	0.00381542077874518\\
-1784.3975094616	0.0069409287138406\\
-1778.53741911854	0.00993609147400074\\
-1772.67732877548	0.0127267814135153\\
-1766.81723843242	0.0152426685735685\\
-1760.95714808936	0.0174189387001174\\
-1755.09705774631	0.019197905785455\\
-1749.23696740325	0.0205304799005163\\
-1743.37687706019	0.0213774539147755\\
-1737.51678671713	0.0217105764099564\\
-1731.65669637407	0.0215133826033089\\
-1725.79660603101	0.0207817603317231\\
-1719.93651568795	0.0195242339738226\\
-1714.07642534489	0.017761955499055\\
-1708.21633500183	0.0155283984775505\\
-1702.35624465877	0.0128687577205047\\
-1696.49615431571	0.0098390640935434\\
-1690.63606397265	0.00650503079562319\\
-1684.77597362959	0.00294065386667799\\
-1678.91588328653	-0.000773404269295005\\
-1673.05579294348	-0.00455161395270145\\
-1667.19570260042	-0.00830552109696161\\
-1661.33561225736	-0.0119457906926107\\
-1655.4755219143	-0.0153843050391008\\
-1649.61543157124	-0.0185362687434548\\
-1643.75534122818	-0.0213222716012969\\
-1637.89525088512	-0.0236702606628685\\
-1632.03516054206	-0.0255173740877348\\
-1626.175070199	-0.0268115918039416\\
-1620.31497985594	-0.0275131614658803\\
-1614.45488951288	-0.0275957626874273\\
-1608.59479916982	-0.0270473779280049\\
-1602.73470882676	-0.0258708446201176\\
-1596.8746184837	-0.0240840700213118\\
-1591.01452814064	-0.0217198976977746\\
-1585.15443779758	-0.0188256223530538\\
-1579.29434745452	-0.0154621577261232\\
-1573.43425711146	-0.0117028703150683\\
-1567.57416676841	-0.00763209957447714\\
-1561.71407642535	-0.00334339277894496\\
-1555.85398608229	0.00106251020646209\\
-1549.99389573923	0.00547990064286727\\
-1544.13380539617	0.00980048597372192\\
-1538.27371505311	0.013915884500524\\
-1532.41362471005	0.0177201744695551\\
-1526.55353436699	0.0211124400539121\\
-1520.69344402393	0.0239992556573947\\
-1514.83335368087	0.0262970502124335\\
-1508.97326333781	0.027934294692799\\
-1503.11317299475	0.0288534589218024\\
-1497.25308265169	0.0290126878692258\\
-1491.39299230863	0.028387152936888\\
-1485.53290196557	0.0269700401290352\\
-1479.67281162251	0.0247731443769706\\
-1473.81272127945	0.0218270474814445\\
-1467.95263093639	0.0181808659890328\\
-1462.09254059333	0.0139015646557745\\
-1456.23245025027	0.00907284075713785\\
-1450.37235990721	0.00379359419896103\\
-1444.51226956415	-0.00182399207114651\\
-1438.6521792211	-0.00765672763855815\\
-1432.79208887804	-0.013573003818232\\
-1426.93199853498	-0.019435646200381\\
-1421.07190819192	-0.0251049733031036\\
-1415.21181784886	-0.0304419982165962\\
-1409.3517275058	-0.0353117055066393\\
-1403.49163716274	-0.0395863323882404\\
-1397.63154681968	-0.0431485813924892\\
-1391.77145647662	-0.0458946914965311\\
-1385.91136613356	-0.0477372959797141\\
-1380.0512757905	-0.0486079981452268\\
-1374.19118544744	-0.048459600428606\\
-1368.33109510438	-0.0472679282845822\\
-1362.47100476132	-0.0450331974637927\\
-1356.61091441826	-0.0417808817478559\\
-1350.7508240752	-0.0375620477701387\\
-1344.89073373214	-0.032453133982438\\
-1339.03064338909	-0.0265551619885406\\
-1333.17055304603	-0.0199923800795998\\
-1327.31046270297	-0.0129103506877623\\
-1321.45037235991	-0.00547350534665286\\
-1315.59028201685	0.00213779761317895\\
-1309.73019167379	0.0097306662554505\\
-1303.87010133073	0.0171034623045789\\
-1298.01001098767	0.0240499161933233\\
-1292.14992064461	0.0303634952677014\\
-1286.28983030155	0.0358419435443458\\
-1280.42973995849	0.0402919059350923\\
-1274.56964961543	0.0435335460266654\\
-1268.70955927237	0.0454050644083306\\
-1262.84946892931	0.0457670242753695\\
-1256.98937858626	0.0445063926070338\\
-1251.1292882432	0.0415402086416198\\
-1245.26919790014	0.0368187965883548\\
-1239.40910755708	0.030328446479454\\
-1233.54901721402	0.0220934956158966\\
-1227.68892687096	0.012177753102854\\
-1221.8288365279	0.000685221285856634\\
-1215.96874618484	-0.0122399196904163\\
-1210.10865584178	-0.0264140847141357\\
-1204.24856549872	-0.0416158205683189\\
-1198.38847515566	-0.0575880992416865\\
-1192.5283848126	-0.0740413457986789\\
-1186.66829446954	-0.0906571648315894\\
-1180.80820412648	-0.107092715865975\\
-1174.94811378342	-0.122985675223848\\
-1169.08802344036	-0.137959709919137\\
-1163.2279330973	-0.15163037851408\\
-1157.36784275424	-0.163611364615539\\
-1151.50775241118	-0.173520941108077\\
-1145.64766206812	-0.180988557399433\\
-1139.78757172506	-0.185661438070636\\
-1133.927481382	-0.18721107940182\\
-1128.06739103895	-0.185339530454364\\
-1122.20730069589	-0.179785347520509\\
-1116.34721035283	-0.170329115104188\\
-1110.48712000977	-0.156798432711866\\
-1104.62702966671	-0.139072274887151\\
-1098.76693932365	-0.117084641646392\\
-1092.90684898059	-0.0908274278657117\\
-1087.04675863753	-0.0603524528168039\\
-1081.18666829447	-0.0257726048238335\\
-1075.32657795141	0.0127379292904945\\
-1069.46648760835	0.0549443651140472\\
-1063.60639726529	0.100552951453853\\
-1057.74630692223	0.149213515023878\\
-1051.88621657917	0.200522887640001\\
-1046.02612623611	0.254029169754871\\
-1040.16603589305	0.309236770140319\\
-1034.30594554999	0.365612148604064\\
-1028.44585520693	0.422590176891381\\
-1022.58576486387	0.479581022814801\\
-1016.72567452081	0.535977454104738\\
-1010.86558417775	0.591162451709148\\
-1005.00549383469	0.644517017647452\\
-999.145403491639	0.695428059563599\\
-993.285313148579	0.743296233500197\\
-987.425222805519	0.787543627687799\\
-981.56513246246	0.827621173336601\\
-975.7050421194	0.863015673755593\\
-969.84495177634	0.893256349960191\\
-963.98486143328	0.917920809728259\\
-958.124771090221	0.936640357141518\\
-952.264680747161	0.94910457105228\\
-946.404590404101	0.95506509343728\\
-940.544500061042	0.954338581889728\\
-934.684409717982	0.946808794458036\\
-928.824319374922	0.932427789082648\\
-922.964229031862	0.91121623424658\\
-917.104138688806	0.883262841547149\\
-911.244048345747	0.848722944261556\\
-905.383958002687	0.807816259181336\\
-899.523867659627	0.760823880633252\\
-893.663777316568	0.708084566821275\\
-887.803686973508	0.649990387873447\\
-881.943596630448	0.58698181345213\\
-876.083506287388	0.519542324044213\\
-870.223415944329	0.448192635362775\\
-864.363325601269	0.373484628305135\\
-858.503235258209	0.29599507864572\\
-852.64314491515	0.216319280481339\\
-846.78305457209	0.135064655748203\\
-840.92296422903	0.0528444387668077\\
-835.062873885974	-0.0297284798989344\\
-829.202783542914	-0.112047472118835\\
-823.342693199855	-0.193517782451845\\
-817.482602856795	-0.273561767257345\\
-811.622512513735	-0.35162370894472\\
-805.762422170676	-0.427174161292262\\
-799.902331827616	-0.499713791710958\\
-794.042241484556	-0.568776696596996\\
-788.182151141496	-0.633933176002452\\
-782.322060798437	-0.694791963845459\\
-776.461970455377	-0.751001919424186\\
-770.601880112317	-0.802253194759983\\
-764.741789769258	-0.84827790079681\\
-758.881699426198	-0.888850302444771\\
-753.021609083142	-0.923786578908758\\
-747.161518740082	-0.952944191032698\\
-741.301428397022	-0.976220901207955\\
-735.441338053963	-0.993553494367116\\
-729.581247710903	-1.00491625035163\\
-723.721157367843	-1.01031921814018\\
-717.861067024784	-1.00980634184636\\
-712.000976681724	-1.00345348679411\\
-706.140886338664	-0.991366410834391\\
-700.280795995604	-0.973678722777293\\
-694.420705652545	-0.950549865111324\\
-688.560615309485	-0.922163153296808\\
-682.700524966425	-0.888723898274073\\
-676.840434623366	-0.85045763296809\\
-670.980344280306	-0.807608457584923\\
-665.12025393725	-0.760437512269963\\
-659.26016359419	-0.709221579942353\\
-653.40007325113	-0.654251816316046\\
-647.539982908071	-0.595832598875509\\
-641.679892565011	-0.534280481878652\\
-635.819802221951	-0.469923240314452\\
-629.959711878892	-0.403098982382879\\
-624.099621535832	-0.334155307436597\\
-618.239531192772	-0.263448484581125\\
-612.379440849712	-0.191342626164042\\
-606.519350506653	-0.118208830417588\\
-600.659260163593	-0.0444242682066619\\
-594.799169820533	0.0296288094771863\\
-588.939079477474	0.103564164336577\\
-583.078989134417	0.176992781905603\\
-577.218898791358	0.24952421539293\\
-571.358808448298	0.320768072493506\\
-565.498718105238	0.390335650935024\\
-559.638627762179	0.457841723775397\\
};
\addplot [color=mycolor2, forget plot]
  table[row sep=crcr]{%
-559.638627762179	0.457841723775397\\
-553.778537419119	0.522906470529288\\
-547.918447076059	0.585157545364451\\
-542.058356733	0.644232268641514\\
-536.19826638994	0.699779923750387\\
-530.33817604688	0.751464136600138\\
-524.47808570382	0.798965311660523\\
-518.617995360761	0.841983094946846\\
-512.757905017701	0.8802388317818\\
-506.897814674641	0.913477985140558\\
-501.037724331585	0.941472479198873\\
-495.177633988525	0.96402293195531\\
-489.317543645466	0.980960741453607\\
-483.457453302406	0.992149990584131\\
-477.597362959346	0.99748913773928\\
-471.737272616287	0.996912462498718\\
-465.877182273227	0.990391238982957\\
-460.017091930167	0.977934612900703\\
-454.157001587108	0.959590162529187\\
-448.296911244048	0.935444128356586\\
-442.436820900988	0.905621300974736\\
-436.576730557928	0.870284561832645\\
-430.716640214869	0.829634076516639\\
-424.856549871809	0.783906145453829\\
-418.996459528749	0.733371721814328\\
-413.136369185693	0.678334611335258\\
-407.276278842633	0.619129373073812\\
-401.416188499574	0.556118944399546\\
-395.556098156514	0.489692017017881\\
-389.696007813454	0.420260194002688\\
-383.835917470395	0.348254960325842\\
-377.975827127335	0.274124501354156\\
-372.115736784275	0.198330404927245\\
-366.255646441216	0.121344283487174\\
-360.395556098156	0.0436443525862737\\
-354.535465755096	-0.0342879983924338\\
-348.675375412036	-0.111971608682172\\
-342.815285068977	-0.188928888795259\\
-336.955194725917	-0.264689072635429\\
-331.095104382861	-0.338791350891377\\
-325.235014039801	-0.41078786043124\\
-319.374923696741	-0.480246507654551\\
-313.514833353682	-0.546753607071468\\
-307.654743010622	-0.609916319920112\\
-301.794652667562	-0.669364881091267\\
-295.934562324503	-0.724754606017306\\
-290.074471981443	-0.775767672645932\\
-284.214381638383	-0.822114676567233\\
-278.354291295323	-0.863535960352843\\
-272.494200952264	-0.899802720626777\\
-266.634110609204	-0.930717898753082\\
-260.774020266144	-0.956116862678604\\
-254.913929923085	-0.975867889088106\\
-249.053839580029	-0.989872456064077\\
-243.193749236969	-0.998065356918254\\
-237.333658893909	-1.00041464654236\\
-231.473568550849	-0.996921431105025\\
-225.61347820779	-0.987619511824232\\
-219.75338786473	-0.972574892848291\\
-213.89329752167	-0.951885162271285\\
-208.033207178611	-0.925678754301471\\
-202.173116835551	-0.894114099366228\\
-196.313026492491	-0.85737866777532\\
-190.452936149431	-0.815687911153543\\
-184.592845806372	-0.769284104693474\\
-178.732755463312	-0.718435092217464\\
-172.872665120252	-0.663432935071271\\
-167.012574777196	-0.604592464941082\\
-161.152484434137	-0.542249740482189\\
-155.292394091077	-0.476760407068243\\
-149.432303748017	-0.408497959180072\\
-143.572213404957	-0.337851905193855\\
-137.712123061898	-0.265225834903588\\
-131.852032718838	-0.19103539109517\\
-125.991942375778	-0.115706147506575\\
-120.131852032719	-0.0396713969892807\\
-114.271761689659	0.0366301447516272\\
-108.411671346599	0.11275671271717\\
-102.551581003539	0.18826593447527\\
-96.6914906604798	0.262717329394353\\
-90.83140031742	0.335674852387291\\
-84.9713099743603	0.406709467386166\\
-79.1112196313043	0.475401734021133\\
-73.2511292882446	0.541344389310977\\
-67.3910389451848	0.604144904758838\\
-61.5309486021251	0.663427998087187\\
-55.6708582590654	0.718838078007769\\
-49.8107679160057	0.770041599858514\\
-43.950677572946	0.816729309733349\\
-38.0905872298863	0.858618355104467\\
-32.2304968868266	0.895454240302984\\
-26.3704065437669	0.927012606394028\\
-20.5103162007072	0.953100816365631\\
-14.6502258576475	0.973559328006949\\
-8.79013551458775	0.988262839354347\\
-2.93004517152804	0.997121193545325\\
2.93004517152804	1.00008003273341\\
8.79013551458775	0.997121193545325\\
14.6502258576475	0.988262839354347\\
20.5103162007072	0.973559328006949\\
26.3704065437669	0.953100816365631\\
32.2304968868266	0.927012606394028\\
38.0905872298863	0.895454240302984\\
43.950677572946	0.858618355104467\\
49.8107679160057	0.816729309733349\\
55.6708582590654	0.770041599858514\\
61.5309486021251	0.718838078007769\\
67.3910389451848	0.663427998087187\\
73.2511292882446	0.604144904758838\\
79.1112196313043	0.541344389310977\\
84.9713099743603	0.475401734021133\\
90.83140031742	0.406709467386166\\
96.6914906604798	0.335674852387291\\
102.551581003539	0.262717329394353\\
108.411671346599	0.18826593447527\\
114.271761689659	0.11275671271717\\
120.131852032719	0.0366301447516272\\
125.991942375778	-0.0396713969892807\\
131.852032718838	-0.115706147506575\\
137.712123061898	-0.19103539109517\\
143.572213404957	-0.265225834903588\\
149.432303748017	-0.337851905193855\\
155.292394091077	-0.408497959180072\\
161.152484434137	-0.476760407068243\\
167.012574777196	-0.542249740482189\\
172.872665120252	-0.604592464941082\\
178.732755463312	-0.663432935071271\\
184.592845806372	-0.718435092217464\\
190.452936149431	-0.769284104693474\\
196.313026492491	-0.815687911153543\\
202.173116835551	-0.85737866777532\\
208.033207178611	-0.894114099366228\\
213.89329752167	-0.925678754301471\\
219.75338786473	-0.951885162271285\\
225.61347820779	-0.972574892848291\\
231.473568550849	-0.987619511824232\\
237.333658893909	-0.996921431105025\\
243.193749236969	-1.00041464654236\\
249.053839580029	-0.998065356918254\\
254.913929923085	-0.989872456064077\\
260.774020266144	-0.975867889088106\\
266.634110609204	-0.956116862678604\\
272.494200952264	-0.930717898753082\\
278.354291295323	-0.899802720626777\\
284.214381638383	-0.863535960352843\\
290.074471981443	-0.822114676567233\\
295.934562324503	-0.775767672645932\\
301.794652667562	-0.724754606017306\\
307.654743010622	-0.669364881091267\\
313.514833353682	-0.609916319920112\\
319.374923696741	-0.546753607071468\\
325.235014039801	-0.480246507654551\\
331.095104382861	-0.41078786043124\\
336.955194725917	-0.338791350891377\\
342.815285068977	-0.264689072635429\\
348.675375412036	-0.188928888795259\\
354.535465755096	-0.111971608682172\\
360.395556098156	-0.0342879983924338\\
366.255646441216	0.0436443525862737\\
372.115736784275	0.121344283487174\\
377.975827127335	0.198330404927245\\
383.835917470395	0.274124501354156\\
389.696007813454	0.348254960325842\\
395.556098156514	0.420260194002688\\
401.416188499574	0.489692017017881\\
407.276278842633	0.556118944399546\\
413.136369185693	0.619129373073812\\
418.996459528749	0.678334611335258\\
424.856549871809	0.733371721814328\\
430.716640214869	0.783906145453829\\
436.576730557928	0.829634076516639\\
442.436820900988	0.870284561832645\\
448.296911244048	0.905621300974736\\
454.157001587108	0.935444128356586\\
460.017091930167	0.959590162529187\\
465.877182273227	0.977934612900703\\
471.737272616287	0.990391238982957\\
477.597362959346	0.996912462498718\\
483.457453302406	0.99748913773928\\
489.317543645466	0.992149990584131\\
495.177633988525	0.980960741453607\\
501.037724331585	0.96402293195531\\
506.897814674641	0.941472479198873\\
512.757905017701	0.913477985140558\\
518.617995360761	0.8802388317818\\
524.47808570382	0.841983094946846\\
530.33817604688	0.798965311660523\\
536.19826638994	0.751464136600138\\
542.058356733	0.699779923750387\\
547.918447076059	0.644232268641514\\
553.778537419119	0.585157545364451\\
559.638627762179	0.522906470529288\\
565.498718105238	0.457841723775397\\
571.358808448298	0.390335650935024\\
577.218898791358	0.320768072493506\\
583.078989134417	0.24952421539293\\
588.939079477474	0.176992781905603\\
594.799169820533	0.103564164336577\\
600.659260163593	0.0296288094771863\\
606.519350506653	-0.0444242682066619\\
612.379440849712	-0.118208830417588\\
618.239531192772	-0.191342626164042\\
624.099621535832	-0.263448484581125\\
629.959711878892	-0.334155307436597\\
635.819802221951	-0.403098982382879\\
641.679892565011	-0.469923240314452\\
647.539982908071	-0.534280481878652\\
653.40007325113	-0.595832598875509\\
659.26016359419	-0.654251816316046\\
665.12025393725	-0.709221579942353\\
670.980344280306	-0.760437512269963\\
676.840434623366	-0.807608457584923\\
682.700524966425	-0.85045763296809\\
688.560615309485	-0.888723898274073\\
694.420705652545	-0.922163153296808\\
700.280795995604	-0.950549865111324\\
706.140886338664	-0.973678722777293\\
712.000976681724	-0.991366410834391\\
717.861067024784	-1.00345348679411\\
723.721157367843	-1.00980634184636\\
729.581247710903	-1.01031921814018\\
735.441338053963	-1.00491625035163\\
741.301428397022	-0.993553494367116\\
747.161518740082	-0.976220901207955\\
753.021609083142	-0.952944191032698\\
758.881699426198	-0.923786578908758\\
764.741789769258	-0.888850302444771\\
770.601880112317	-0.84827790079681\\
776.461970455377	-0.802253194759983\\
782.322060798437	-0.751001919424186\\
788.182151141496	-0.694791963845459\\
794.042241484556	-0.633933176002452\\
799.902331827616	-0.568776696596996\\
805.762422170676	-0.499713791710958\\
811.622512513735	-0.427174161292262\\
817.482602856795	-0.35162370894472\\
823.342693199855	-0.273561767257345\\
829.202783542914	-0.193517782451845\\
835.062873885974	-0.112047472118835\\
840.92296422903	-0.0297284798989344\\
846.78305457209	0.0528444387668077\\
852.64314491515	0.135064655748203\\
858.503235258209	0.216319280481339\\
864.363325601269	0.29599507864572\\
870.223415944329	0.373484628305135\\
876.083506287388	0.448192635362775\\
881.943596630448	0.519542324044213\\
887.803686973508	0.58698181345213\\
893.663777316568	0.649990387873447\\
899.523867659627	0.708084566821275\\
905.383958002687	0.760823880633252\\
911.244048345747	0.807816259181336\\
917.104138688806	0.848722944261556\\
922.964229031862	0.883262841547149\\
928.824319374922	0.91121623424658\\
934.684409717982	0.932427789082648\\
940.544500061042	0.946808794458036\\
946.404590404101	0.954338581889728\\
952.264680747161	0.95506509343728\\
958.124771090221	0.94910457105228\\
963.98486143328	0.936640357141518\\
969.84495177634	0.917920809728259\\
975.7050421194	0.893256349960191\\
981.56513246246	0.863015673755593\\
987.425222805519	0.827621173336601\\
993.285313148579	0.787543627687799\\
999.145403491639	0.743296233500197\\
1005.00549383469	0.695428059563599\\
1010.86558417775	0.644517017647452\\
1016.72567452081	0.591162451709148\\
1022.58576486387	0.535977454104738\\
1028.44585520693	0.479581022814801\\
1034.30594554999	0.422590176891381\\
1040.16603589305	0.365612148604064\\
1046.02612623611	0.309236770140319\\
1051.88621657917	0.254029169754871\\
1057.74630692223	0.200522887640001\\
1063.60639726529	0.149213515023878\\
1069.46648760835	0.100552951453853\\
1075.32657795141	0.0549443651140472\\
1081.18666829447	0.0127379292904945\\
1087.04675863753	-0.0257726048238335\\
1092.90684898059	-0.0603524528168039\\
1098.76693932365	-0.0908274278657117\\
1104.62702966671	-0.117084641646392\\
1110.48712000977	-0.139072274887151\\
1116.34721035283	-0.156798432711866\\
1122.20730069589	-0.170329115104188\\
1128.06739103895	-0.179785347520509\\
1133.927481382	-0.185339530454364\\
1139.78757172506	-0.18721107940182\\
1145.64766206812	-0.185661438070636\\
1151.50775241118	-0.180988557399433\\
1157.36784275424	-0.173520941108077\\
1163.2279330973	-0.163611364615539\\
1169.08802344036	-0.15163037851408\\
1174.94811378342	-0.137959709919137\\
1180.80820412648	-0.122985675223848\\
1186.66829446954	-0.107092715865975\\
1192.5283848126	-0.0906571648315894\\
1198.38847515566	-0.0740413457986789\\
1204.24856549872	-0.0575880992416865\\
1210.10865584178	-0.0416158205683189\\
1215.96874618484	-0.0264140847141357\\
1221.8288365279	-0.0122399196904163\\
1227.68892687096	0.000685221285856634\\
1233.54901721402	0.012177753102854\\
1239.40910755708	0.0220934956158966\\
1245.26919790014	0.030328446479454\\
1251.1292882432	0.0368187965883548\\
1256.98937858626	0.0415402086416198\\
1262.84946892931	0.0445063926070338\\
1268.70955927237	0.0457670242753695\\
1274.56964961543	0.0454050644083306\\
1280.42973995849	0.0435335460266654\\
1286.28983030155	0.0402919059350923\\
1292.14992064461	0.0358419435443458\\
1298.01001098767	0.0303634952677014\\
1303.87010133073	0.0240499161933233\\
1309.73019167379	0.0171034623045789\\
1315.59028201685	0.0097306662554505\\
1321.45037235991	0.00213779761317895\\
1327.31046270297	-0.00547350534665286\\
1333.17055304603	-0.0129103506877623\\
1339.03064338909	-0.0199923800795998\\
1344.89073373214	-0.0265551619885406\\
1350.7508240752	-0.032453133982438\\
1356.61091441826	-0.0375620477701387\\
1362.47100476132	-0.0417808817478559\\
1368.33109510438	-0.0450331974637927\\
1374.19118544744	-0.0472679282845822\\
1380.0512757905	-0.048459600428606\\
1385.91136613356	-0.0486079981452268\\
1391.77145647662	-0.0477372959797141\\
1397.63154681968	-0.0458946914965311\\
1403.49163716274	-0.0431485813924892\\
1409.3517275058	-0.0395863323882404\\
1415.21181784886	-0.0353117055066393\\
1421.07190819192	-0.0304419982165962\\
1426.93199853498	-0.0251049733031036\\
1432.79208887804	-0.019435646200381\\
1438.6521792211	-0.013573003818232\\
1444.51226956415	-0.00765672763855815\\
1450.37235990721	-0.00182399207114651\\
1456.23245025027	0.00379359419896103\\
1462.09254059333	0.00907284075713785\\
1467.95263093639	0.0139015646557745\\
1473.81272127945	0.0181808659890328\\
1479.67281162251	0.0218270474814445\\
1485.53290196557	0.0247731443769706\\
1491.39299230863	0.0269700401290352\\
1497.25308265169	0.028387152936888\\
1503.11317299475	0.0290126878692258\\
1508.97326333781	0.0288534589218024\\
1514.83335368087	0.027934294692799\\
1520.69344402393	0.0262970502124335\\
1526.55353436699	0.0239992556573947\\
1532.41362471005	0.0211124400539121\\
1538.27371505311	0.0177201744695551\\
1544.13380539617	0.013915884500524\\
1549.99389573923	0.00980048597372192\\
1555.85398608229	0.00547990064286727\\
1561.71407642535	0.00106251020646209\\
1567.57416676841	-0.00334339277894496\\
1573.43425711146	-0.00763209957447714\\
1579.29434745452	-0.0117028703150683\\
1585.15443779758	-0.0154621577261232\\
1591.01452814064	-0.0188256223530538\\
1596.8746184837	-0.0217198976977746\\
1602.73470882676	-0.0240840700213118\\
1608.59479916982	-0.0258708446201176\\
1614.45488951288	-0.0270473779280049\\
1620.31497985594	-0.0275957626874273\\
1626.175070199	-0.0275131614658803\\
1632.03516054206	-0.0268115918039416\\
1637.89525088512	-0.0255173740877348\\
1643.75534122818	-0.0236702606628685\\
1649.61543157124	-0.0213222716012969\\
1655.4755219143	-0.0185362687434548\\
1661.33561225736	-0.0153843050391008\\
1667.19570260042	-0.0119457906926107\\
1673.05579294348	-0.00830552109696161\\
1678.91588328653	-0.00455161395270145\\
1684.77597362959	-0.000773404269295005\\
1690.63606397265	0.00294065386667799\\
1696.49615431571	0.00650503079562319\\
1702.35624465877	0.0098390640935434\\
1708.21633500183	0.0128687577205047\\
1714.07642534489	0.0155283984775505\\
1719.93651568795	0.017761955499055\\
1725.79660603101	0.0195242339738226\\
1731.65669637407	0.0207817603317231\\
1737.51678671713	0.0215133826033089\\
1743.37687706019	0.0217105764099564\\
1749.23696740325	0.0213774539147755\\
1755.09705774631	0.0205304799005163\\
1760.95714808936	0.019197905785455\\
1766.81723843242	0.0174189387001174\\
1772.67732877548	0.0152426685735685\\
1778.53741911854	0.0127267814135153\\
1784.3975094616	0.00993609147400074\\
1790.25759980466	0.0069409287138406\\
1796.11769014772	0.00381542077874518\\
1801.97778049078	0.000635710629293177\\
1807.83787083384	-0.0025218481337946\\
1813.6979611769	-0.00558247624155257\\
1819.55805151996	-0.00847469302921748\\
1825.41814186302	-0.0111319409451808\\
1831.27823220608	-0.0134940740375226\\
1837.13832254914	-0.015508677757939\\
1842.9984128922	-0.017132191727909\\
1848.85850323526	-0.0183308120121746\\
1854.71859357832	-0.0190811548232647\\
1860.57868392138	-0.019370669308774\\
1866.43877426444	-0.0191977930156998\\
1872.2988646075	-0.0185718496407358\\
1878.15895495056	-0.0175126946320237\\
1884.01904529361	-0.0160501199614507\\
1889.87913563667	-0.0142230348116931\\
1895.73922597973	-0.0120784439005949\\
1901.59931632279	-0.00967024958476721\\
1907.45940666585	-0.00705790764695253\\
1913.31949700891	-0.00430496970760822\\
1919.17958735197	-0.00147754743827254\\
1925.03967769503	0.00135726484147846\\
1930.89976803809	0.00413297206022388\\
1936.75985838115	0.00678522797837604\\
1942.61994872421	0.00925331681943426\\
1948.48003906727	0.0114815332838823\\
1954.34012941033	0.0134204300100195\\
1960.20021975339	0.0150279049613457\\
1966.06031009645	0.0162701052353912\\
1971.92040043951	0.0171221282966627\\
1977.78049078257	0.0175685065489551\\
1983.64058112563	0.017603466330637\\
1989.50067146869	0.01723095775178\\
1995.36076181175	0.0164644571434375\\
2001.22085215481	0.0153265491510133\\
2007.08094249787	0.0138483005413317\\
2012.94103284092	0.0120684425051182\\
2018.80112318398	0.0100323825085114\\
2024.66121352704	0.00779107048717147\\
2030.5213038701	0.00539974730823386\\
2036.38139421316	0.0029166058741285\\
2042.24148455622	0.000401396964019029\\
2048.10157489928	-0.00208598712897101\\
2053.96166524234	-0.00448691792266446\\
2059.8217555854	-0.00674539237829642\\
2065.68184592846	-0.00880931952503118\\
2071.54193627152	-0.0106317028143412\\
2077.40202661458	-0.0121716916531481\\
2083.26211695764	-0.013395478866648\\
2089.1222073007	-0.0142770246534892\\
2094.98229764376	-0.0147985918175227\\
2100.84238798681	-0.0149510816021221\\
2106.70247832987	-0.0147341641942396\\
2112.56256867293	-0.0141562028041475\\
2118.42265901599	-0.0132339750428316\\
2124.28274935905	-0.0119922000052796\\
2130.14283970211	-0.010462883917712\\
2136.00293004517	-0.00868450131249065\\
2141.86302038823	-0.00670103237302143\\
2147.72311073129	-0.00456088025260645\\
2153.58320107435	-0.00231569475419591\\
2159.44329141741	-1.91307012518548e-05\\
2165.30338176047	0.0022744294019185\\
2171.16347210353	0.00451115825297398\\
2177.02356244659	0.00663904214923177\\
2182.88365278965	0.00860908570863427\\
2188.74374313271	0.0103764352298235\\
2194.60383347576	0.0119013951481239\\
2200.46392381882	0.0131503147588297\\
2206.32401416188	0.0140963256054732\\
2212.18410450494	0.0147199135818754\\
2218.044194848	0.0150093137914393\\
2223.90428519106	0.014960720446091\\
2229.76437553412	0.0145783084693991\\
2235.62446587718	0.0138740678917618\\
2241.48455622024	0.0128674564878283\\
2247.3446465633	0.0115848803034256\\
2253.20473690636	0.0100590156610343\\
2259.06482724942	0.00832798981830225\\
2264.92491759248	0.00643444061830099\\
2270.78500793554	0.00442447812832633\\
2276.6450982786	0.00234657336730572\\
2282.50518862166	0.00025040072141383\\
2288.36527896472	-0.00181433848336023\\
2294.22536930778	-0.00379908356424504\\
2300.08545965084	-0.00565754595920392\\
2305.9455499939	-0.00734677839859645\\
2311.80564033696	-0.00882815723345431\\
2317.66573068002	-0.0100682556059625\\
2323.52582102307	-0.0110395878732599\\
2329.38591136613	-0.0117212088592692\\
2335.24600170919	-0.012099155029404\\
2341.10609205225	-0.0121667184788795\\
2346.96618239531	-0.0119245486059466\\
2352.82627273837	-0.0113805804089319\\
2358.68636308143	-0.0105497924085143\\
2364.54645342449	-0.00945380116082388\\
2370.40654376755	-0.00812030309476438\\
2376.26663411061	-0.00658237789869062\\
2382.12672445367	-0.00487767081481468\\
2387.98681479673	-0.00304747390285287\\
2393.84690513979	-0.00113572855030081\\
2399.70699548285	0.000812026814088753\\
2405.56708582591	0.0027497387414227\\
2411.42717616897	0.00463192494249867\\
2417.28726651203	0.00641473913918165\\
2423.14735685509	0.00805698978402546\\
2429.00744719815	0.00952108910289104\\
2434.8675375412	0.0107739107154888\\
2440.72762788426	0.0117875363843739\\
2446.58771822732	0.0125398751761778\\
2452.44780857038	0.0130151414254016\\
2458.30789891344	0.013204181292042\\
2464.1679892565	0.0131046413218598\\
2470.02807959956	0.0127209761646408\\
2475.88816994262	0.0120642963894758\\
2481.74826028568	0.0111520610785373\\
2487.60835062874	0.0100076234812093\\
2493.4684409718	0.00865964139479158\\
2499.32853131486	0.0071413670297236\\
2505.18862165792	0.00548983383634214\\
2511.04871200098	0.00374496006777452\\
2516.90880234403	0.00194859066912146\\
2522.76889268709	0.000143500384176332\\
2528.62898303015	-0.00162761827576789\\
2534.48907337321	-0.00332315835780078\\
2540.34916371627	-0.00490356960841335\\
2546.20925405933	-0.00633227485948068\\
2552.06934440239	-0.00757651052092272\\
2557.92943474545	-0.00860807195946099\\
2563.78952508851	-0.00940394689773573\\
2569.64961543157	-0.00994682270303402\\
2575.50970577463	-0.0102254564807398\\
2581.36979611769	-0.010234900178849\\
2587.22988646075	-0.00997657635428583\\
2593.08997680381	-0.00945820378437724\\
2598.95006714687	-0.00869357562946508\\
2604.81015748993	-0.00770219629457858\\
2610.67024783299	-0.00650878641024377\\
2616.53033817605	-0.00514266838524258\\
2622.39042851911	-0.00363704770804635\\
2628.25051886217	-0.00202820751994197\\
2634.11060920523	-0.000354635910199807\\
2639.97069954828	0.00134389316210531\\
2645.83078989134	0.00302726345532345\\
2651.6908802344	0.00465596223862386\\
2657.55097057746	0.00619200607336325\\
2663.41106092052	0.00759982497022164\\
2669.27115126358	0.00884708441849526\\
2675.13124160664	0.00990542636467931\\
2680.9913319497	0.0107511122353016\\
2686.85142229276	0.0113655534972052\\
2692.71151263582	0.0117357179755324\\
2698.57160297888	0.0118544031335403\\
2704.43169332194	0.0117203706928083\\
2710.291783665	0.0113383402598657\\
2716.15187400806	0.0107188429453168\\
2722.01196435112	0.00987793924262647\\
2727.87205469418	0.0088368085913962\\
2733.73214503724	0.0076212210206767\\
2739.5922353803	0.00626090397022931\\
2745.45232572336	0.00478881977361752\\
2751.31241606642	0.00324037129351002\\
2757.17250640948	0.00165255478330148\\
2763.03259675254	6.30801788299924e-05\\
2768.89268709559	-0.00149052032857914\\
2774.75277743865	-0.00297177442992492\\
2780.61286778171	-0.00434612187831032\\
2786.47295812477	-0.00558171686171607\\
2792.33304846783	-0.00665016200192275\\
2798.19313881089	-0.00752715712211438\\
2804.05322915395	-0.00819304802201049\\
2809.91331949701	-0.00863326292849645\\
2815.77340984007	-0.00883862699654855\\
2821.63350018313	-0.00880554815098393\\
2827.49359052619	-0.00853607061840782\\
2833.35368086925	-0.00803779562351115\\
2839.21377121231	-0.00732367184756998\\
2845.07386155537	-0.00641166128931641\\
2850.93395189842	-0.00532428906249935\\
2856.79404224148	-0.00408808834539056\\
2862.65413258454	-0.00273295409464437\\
2868.5142229276	-0.00129142120849967\\
2874.37431327066	0.00020211548993306\\
2880.23440361372	0.00171222080351519\\
2886.09449395678	0.00320325877352022\\
2891.95458429984	0.00464023251354548\\
2897.8146746429	0.0059896045260749\\
2903.67476498596	0.00722007832548872\\
2909.53485532902	0.00830332320286377\\
2915.39494567208	0.00921462539855497\\
2921.25503601514	0.00993345076560541\\
2927.1151263582	0.0104439061639191\\
2932.97521670126	0.0107350892715334\\
2938.83530704432	0.0108013191729026\\
2944.69539738738	0.0106422429257624\\
2950.55548773043	0.0102628162442474\\
2956.41557807349	0.00967315940618472\\
2962.27566841655	0.00888829241280049\\
2968.13575875961	0.00792775625113928\\
2973.99584910267	0.00681512974368829\\
2979.85593944573	0.0055774538797864\\
2985.71602978879	0.00424457763248576\\
2991.57612013185	0.00284844103973382\\
2997.43621047491	0.00142231272316089\\
nan	nan\\
3009.15639116103	-0.00138494970234729\\
3015.01648150409	-0.00270003730960112\\
3020.87657184715	-0.00391457195439474\\
3026.73666219021	-0.0050003844974638\\
3032.59675253327	-0.00593247817007214\\
3038.45684287633	-0.00668960135077654\\
3044.31693321939	-0.00725472939405607\\
3050.17702356245	-0.00761544462933128\\
3056.03711390551	-0.007764206095785\\
3061.89720424857	-0.00769850321069726\\
3067.75729459163	-0.0074208903251861\\
3073.61738493469	-0.00693890193505245\\
3079.47747527774	-0.0062648511204916\\
3085.3375656208	-0.00541551652062163\\
3091.19765596386	-0.00441172574216919\\
3097.05774630692	-0.00327784549821102\\
3102.91783664998	-0.00204119091566028\\
3108.77792699304	-0.00073136829003325\\
3114.6380173361	0.00062043293757984\\
3120.49810767916	0.00198218208787227\\
3126.35819802222	0.00332176542560803\\
3132.21828836528	0.00460774270801342\\
3138.07837870834	0.00581008431329038\\
3143.9384690514	0.00690087170401437\\
3149.79855939446	0.00785494492196435\\
3155.65864973752	0.00865048213370397\\
3161.51874008058	0.00926949791224047\\
3167.37883042364	0.00969824891316643\\
3173.2389207667	0.00992753783331191\\
3179.09901110976	0.00995290897301492\\
3184.95910145281	0.00977473130189391\\
3190.81919179587	0.00939816759370712\\
3196.67928213893	0.00883303087731186\\
3202.53937248199	0.00809353209863417\\
3208.39946282505	0.00719792542440139\\
3214.25955316811	0.00616805999901185\\
3220.11964351117	0.00502884912276843\\
3225.97973385423	0.00380766971341746\\
3231.83982419729	0.00253370649345193\\
3237.69991454035	0.0012372565793532\\
3243.56000488341	-5.09889885340821e-05\\
3249.42009522647	-0.00130066977059606\\
3255.28018556953	-0.0024824694344331\\
3261.14027591259	-0.00356880150643534\\
3267.00036625564	-0.00453445135847831\\
3272.8604565987	-0.00535715951998051\\
3278.72054694176	-0.00601813285091013\\
3284.58063728482	-0.00650247186787117\\
3290.44072762788	-0.00679950453741863\\
3296.30081797094	-0.00690301909406241\\
3302.160908314	-0.00681139084674552\\
3308.02099865706	-0.00652760045285251\\
3313.88108900012	-0.00605914370243496\\
3319.74117934318	-0.00541783540592309\\
3325.60126968624	-0.00461951245901207\\
3331.4613600293	-0.00368364350610682\\
3337.32145037236	-0.00263285479000728\\
3343.18154071542	-0.00149238370437206\\
3349.04163105848	-0.000289473216962629\\
3354.90172140154	0.000947278333604621\\
3360.7618117446	0.00218859657776847\\
3366.62190208766	0.00340522348184922\\
3372.48199243072	0.00456860639406005\\
3378.34208277378	0.00565156756964951\\
3384.20217311684	0.00662893850202409\\
3390.06226345989	0.00747814428085077\\
3395.92235380295	0.00817972443606576\\
3401.78244414601	0.00871777827683565\\
3407.64253448907	0.00908032456041693\\
3413.50262483213	0.00925956738492283\\
3419.36271517519	0.00925206243771255\\
3425.22280551825	0.00905878010104059\\
3431.08289586131	0.00868506435612603\\
3436.94298620437	0.00814048888322185\\
3442.80307654743	0.0074386141681885\\
3448.66316689049	0.00659665173717525\\
3454.52325723355	0.00563504380392908\\
3460.38334757661	0.00457696856755867\\
3466.24343791967	0.00344778310696529\\
3472.10352826273	0.00227441723806101\\
3477.96361860579	0.00108473279853733\\
3483.82370894885	-9.3136419914711e-05\\
3489.68379929191	-0.00123144847820071\\
3495.54388963497	-0.00230350519651384\\
3501.40397997803	-0.00328427686532124\\
3507.26407032109	-0.0041509851657094\\
3513.12416066415	-0.00488363074642171\\
3518.9842510072	-0.00546545326007842\\
3524.84434135026	-0.00588331330029163\\
3530.70443169332	-0.00612798755876339\\
3536.56452203638	-0.00619437060029849\\
3542.42461237944	-0.00608157887400178\\
3548.2847027225	-0.00579295489945733\\
3554.14479306556	-0.00533597192109993\\
3560.00488340862	-0.00472204166663773\\
3565.86497375168	-0.00396623011362101\\
3571.72506409474	-0.00308688831490519\\
3577.5851544378	-0.00210520730384014\\
3583.44524478086	-0.00104470785231902\\
3589.30533512392	6.93226548304334e-05\\
3595.16542546698	0.00121043276809243\\
3601.02551581004	0.00235163460776858\\
3606.88560615309	0.00346604171332176\\
3612.74569649615	0.00452750205278483\\
3618.60578683921	0.00551121130918611\\
3624.46587718227	0.00639429208248266\\
3630.32596752533	0.00715632550136256\\
3636.18605786839	0.00777982291194471\\
3642.04614821145	0.00825062676708455\\
3647.90623855451	0.00855823154619699\\
3653.76632889757	0.00869601745434875\\
3659.62641924063	0.00866139172788757\\
3665.48650958369	0.00845583456773837\\
3671.34659992675	0.00808484897791946\\
3677.20669026981	0.00755781605084114\\
3683.06678061287	0.00688775946002173\\
3688.92687095593	0.00609102504420597\\
3694.78696129899	0.00518688334262058\\
3700.64705164204	0.00419706472580414\\
3706.5071419851	0.00314523831719472\\
3712.36723232816	0.00205644718068208\\
3718.22732267122	0.000956513234835723\\
3724.08741301428	-0.000128573984568956\\
3729.94750335734	-0.00117327026290317\\
3735.8075937004	-0.00215307690828768\\
3741.66768404346	-0.00304511413168779\\
3747.52777438652	-0.00382865420313579\\
3753.38786472958	-0.00448560197798168\\
3759.24795507264	-0.00500091166995849\\
3765.1080454157	-0.00536293029028585\\
3770.96813575876	-0.00556365993117846\\
3776.82822610182	-0.00559893301236882\\
3782.68831644488	-0.00546849667736869\\
3788.54840678794	-0.0051760046804829\\
3794.408497131	-0.00472891728670607\\
3800.26858747406	-0.00413831187208375\\
3806.12867781712	-0.00341860900343866\\
3811.98876816018	-0.00258722074885845\\
3817.84885850324	-0.00166412977740159\\
3823.7089488463	-0.00067140940488208\\
3829.56903918935	0.000367303903315773\\
3835.42912953241	0.00142737298294409\\
3841.28921987547	0.00248374400730378\\
3847.14931021853	0.00351153828073021\\
3853.00940056159	0.00448663782634899\\
3858.86949090465	0.00538625100051817\\
3864.72958124771	0.00618944488518528\\
3870.58967159077	0.00687763203637733\\
3876.44976193383	0.00743500028746927\\
3882.30985227689	0.00784887568288021\\
3888.16994261995	0.00811001022850894\\
3894.03003296301	0.00821278794353416\\
3899.89012330607	0.00815534464361101\\
3905.75021364913	0.00793959893294186\\
3911.61030399219	0.00757119398092691\\
3917.47039433525	0.00705935176191942\\
3923.33048467831	0.00641664349010055\\
3929.19057502137	0.00565868194429951\\
3935.05066536443	0.00480374319374668\\
3940.91075570748	0.00387232687513467\\
3946.77084605054	0.00288666558511696\\
3952.6309363936	0.00187019511530769\\
3958.49102673666	0.000846998138636516\\
3964.35111707972	-0.000158765461476173\\
3970.21120742278	-0.0011234281582107\\
3976.07129776584	-0.00202437083091371\\
3981.9313881089	-0.00284055224740289\\
3987.79147845196	-0.00355299959823261\\
3993.65156879502	-0.00414524866070209\\
3999.51165913808	-0.00460372339460904\\
4005.37174948114	-0.00491804623215694\\
4011.2318398242	-0.00508127198444811\\
4017.09193016726	-0.00509004011016201\\
4022.95202051031	-0.00494464203082911\\
4028.81211085337	-0.00464900218843632\\
4034.67220119643	-0.00421057357706133\\
4040.53229153949	-0.00364015049060143\\
4046.39238188255	-0.00295160316964667\\
4052.25247222561	-0.00216154085385285\\
4058.11256256867	-0.00128891141044684\\
4063.97265291173	-0.000354547176565682\\
4069.83274325479	0.000619332112477259\\
4075.69283359785	0.00160964746120418\\
4081.55292394091	0.00259300727804757\\
4087.41301428397	0.00354625962961642\\
4093.27310462703	0.00444703705929061\\
4099.13319497009	0.00527428116240368\\
4104.99328531315	0.00600873462693156\\
4110.85337565621	0.00663338925592373\\
4116.71346599927	0.00713387956008516\\
4122.57355634233	0.00749881282498909\\
4128.43364668539	0.0077200280816226\\
4134.29373702845	0.00779277810868319\\
4140.15382737151	0.00771583042410058\\
4146.01391771456	0.00749148514694538\\
4151.87400805762	0.00712550957154266\\
4157.73409840068	0.00662699126065033\\
4163.59418874374	0.00600811337621788\\
4169.4542790868	0.00528385778461535\\
4175.31436942986	0.00447164315926095\\
4181.17445977292	0.00359090680916101\\
4187.03455011598	0.00266264026045302\\
4192.89464045904	0.00170888967644035\\
4198.7547308021	0.000752232993288131\\
4204.61482114516	-0.000184753838597746\\
4210.47491148822	-0.00108002890107251\\
4216.33500183128	-0.00191260211466732\\
4222.19509217434	-0.00266302665218088\\
4228.0551825174	-0.0033138524653146\\
4233.91527286046	-0.00385003135867849\\
4239.77536320352	-0.00425926421897166\\
4245.63545354658	-0.00453228239845683\\
4251.49554388964	-0.00466305682746218\\
4257.3556342327	-0.00464893015194464\\
4263.21572457576	-0.00449066902062229\\
4269.07581491882	-0.00419243553368495\\
4274.93590526187	-0.00376167877409589\\
4280.79599560493	-0.00320894922013131\\
4286.65608594799	-0.00254764064671001\\
4292.51617629105	-0.00179366581587415\\
4298.37626663411	-0.000965073798040115\\
4304.23635697717	-8.16181154118465e-05\\
4310.09644732023	0.000835713969562203\\
4315.95653766329	0.0017651998147762\\
4321.81662800635	0.00268489558418496\\
4327.67671834941	0.00357315409801373\\
4333.53680869247	0.0044091341367855\\
4339.39689903553	0.00517328922701587\\
4345.25698937859	0.00584782445449976\\
4351.11707972165	0.00641711063969909\\
4356.97717006471	0.00686804624474679\\
4362.83726040776	0.00719035864244292\\
4368.69735075082	0.00737683782939752\\
4374.55744109388	0.00742349727841564\\
4380.41753143694	0.00732965835699008\\
4386.27762178	0.0070979565481617\\
4392.13771212306	0.00673426955943134\\
4397.99780246612	0.00624756924310367\\
4403.85789280918	0.00564970104074225\\
4409.71798315224	0.00495509636141761\\
4415.5780734953	0.00418042486645277\\
4421.43816383836	0.00334419502831241\\
4427.29825418142	0.00246631252594408\\
4433.15834452448	0.00156760700324504\\
4439.01843486754	0.000669338432613838\\
4444.8785252106	-0.000207305227401601\\
4450.73861555366	-0.00104170721982999\\
4456.59870589671	-0.0018143062706842\\
4462.45879623977	-0.00250705459753157\\
4468.31888658283	-0.00310383893029356\\
4474.17897692589	-0.00359085473168586\\
4480.03906726895	-0.00395692493507415\\
4485.89915761201	-0.00419375584983331\\
4491.75924795507	-0.00429612438516661\\
4497.61933829813	-0.00426199237767411\\
4503.47942864119	-0.00409254553806755\\
4509.33951898425	-0.0037921563139937\\
4515.19960932731	-0.00336827176109309\\
4521.05969967037	-0.00283122927761571\\
4526.91979001343	-0.00219400474851233\\
4532.77988035649	-0.00147189922510477\\
4538.63997069955	-0.000682171696352505\\
4544.50006104261	0.000156373243617816\\
4550.36015138567	0.00102383298922104\\
4556.22024172873	0.00189967913315761\\
4562.08033207179	0.00276324309694081\\
4567.94042241485	0.00359420370226787\\
4573.80051275791	0.00437306518987947\\
4579.66060310097	0.00508161444939937\\
4585.52069344402	0.00570334674537006\\
4591.38078378708	0.00622384999610994\\
4597.24087413014	0.00663113866645557\\
4603.1009644732	0.00691592954726432\\
4608.96105481626	0.00707185308507725\\
4614.82114515932	0.00709559546077547\\
4620.68123550238	0.00698696826112283\\
4626.54132584544	0.00674890430226135\\
4632.4014161885	0.00638737990567538\\
4638.26150653156	0.00591126566192369\\
4644.12159687462	0.00533210939437569\\
4649.98168721768	0.00466385662267876\\
4655.84177756074	0.00392251528471928\\
4661.7018679038	0.00312577276856715\\
4667.56195824686	0.00229257440960022\\
4673.42204858992	0.00144267348875718\\
4679.28213893298	0.000596163413792659\\
4685.14222927604	-0.000226996845361649\\
4691.0023196191	-0.00100745286604061\\
4696.86240996215	-0.0017269093620529\\
4702.72250030521	-0.00236855858275609\\
4708.58259064827	-0.0029174724982079\\
4714.44268099133	-0.00336094962661905\\
4720.30277133439	-0.00368880845362635\\
4726.16286167745	-0.00389362067287524\\
4732.02295202051	-0.003970878913277\\
4737.88304236357	-0.00391909517608863\\
4743.74313270663	-0.00373982784746673\\
4749.60322304969	-0.00343763684092504\\
4755.46331339275	-0.00301996811788801\\
4761.32340373581	-0.00249697049601303\\
4767.18349407887	-0.00188124924158634\\
4773.04358442193	-0.00118756242026539\\
4778.90367476498	-0.000432467313435233\\
4784.76376510804	0.000366074636237159\\
4790.6238554511	0.00118912493071111\\
4796.48394579416	0.00201721702516986\\
4802.34403613722	0.00283081660068235\\
4808.20412648028	0.00361078222517441\\
4814.06421682334	0.00433881554615089\\
4819.9243071664	0.004997890433962\\
4825.78439750946	0.00557265101788421\\
4831.64448785252	0.00604976931606611\\
4837.50457819558	0.00641825413822676\\
4843.36466853864	0.00666970410702392\\
4849.2247588817	0.00679849898248883\\
4855.08484922476	0.00680192493989265\\
4860.94493956782	0.0066802310206236\\
4866.80502991088	0.00643661560344464\\
4872.66512025394	0.00607714339672218\\
4878.525210597	0.00561059508553304\\
4884.38530094006	0.00504825335207591\\
4890.24539128312	0.00440363047408418\\
4896.10548162617	0.00369214407177429\\
4901.96557196923	0.00293074877724391\\
4907.82566231229	0.0021375326189662\\
4913.68575265535	0.00133128772401368\\
4919.54584299841	0.000531065519587303\\
4925.40593334147	-0.000244273043748826\\
4931.26602368453	-0.00097650163525037\\
4937.12611402759	-0.00164845660097083\\
4942.98620437065	-0.00224443888751755\\
4948.84629471371	-0.00275058042713254\\
4954.70638505677	-0.00315516643918719\\
4960.56647539983	-0.00344890616422852\\
4966.42656574289	-0.00362514578034095\\
4972.28665608595	-0.00368001862940849\\
4978.14674642901	-0.00361252937091565\\
4984.00683677207	-0.00342457024591402\\
4989.86692711513	-0.00312086923776121\\
4995.72701745819	-0.00270887152233677\\
5001.58710780125	-0.00219855716803276\\
5007.44719814431	-0.00160219954101694\\
5013.30728848737	-0.00093407025742763\\
5019.16737883043	-0.000210097768708227\\
5025.02746917348	0.000552512258170917\\
5030.88755951654	0.00133568572432813\\
5036.7476498596	0.00212090834009687\\
5042.60774020266	0.00288966318301387\\
5048.46783054572	0.00362386723746691\\
5054.32792088878	0.00430629663020967\\
5060.18801123184	0.00492099056791236\\
5066.0481015749	0.00545362450936383\\
5071.90819191796	0.00589184385245099\\
5077.76828226102	0.00622555036950828\\
5083.62837260408	0.00644713475472385\\
5089.48846294714	0.00655164993633903\\
5095.3485532902	0.0065369212118717\\
5101.20864363326	0.00640359076642127\\
5107.06873397632	0.00615509568318416\\
5112.92882431937	0.00579758012743732\\
5118.78891466243	0.00533974393228034\\
5124.64900500549	0.00479263130979069\\
5130.50909534855	0.00416936481180091\\
5136.36918569161	0.00348483094289926\\
5142.22927603467	0.00275532495227472\\
5148.08936637773	0.00199816327395481\\
5153.94945672079	0.00123127282810813\\
5159.80954706385	0.000472766918436309\\
5165.66963740691	-0.00025948224664403\\
5171.52972774997	-0.000948264325738051\\
5177.38981809303	-0.00157743500799598\\
5183.24990843609	-0.00213229408257217\\
5189.10999877915	-0.0025999285654072\\
5194.97008912221	-0.00296951287963104\\
5200.83017946527	-0.00323255911665472\\
5206.69026980832	-0.0033831115981668\\
5212.55036015138	-0.00341788128565944\\
5218.41045049444	-0.00333631701190556\\
5224.2705408375	-0.00314061200521067\\
5230.13063118056	-0.00283564570531639\\
5235.99072152362	-0.00242886239529102\\
5241.85081186668	-0.00193008965858885\\
5247.71090220974	-0.00135130108202774\\
5253.5709925528	-0.000706328928454045\\
5259.43108289586	-1.05336688027332e-05\\
5265.29117323892	0.000719561736366034\\
5271.15126358198	0.0014666641034942\\
5277.01135392504	0.00221311953307835\\
5282.8714442681	0.00294133046499977\\
5288.73153461116	0.00363417043458561\\
5294.59162495422	0.0042753867587451\\
5300.45171529728	0.00484998168923948\\
5306.31180564034	0.00534456309915285\\
5312.1718959834	0.00574765650713192\\
5318.03198632646	0.00604997117492418\\
5323.89207666952	0.00624461411186288\\
5329.75216701258	0.00632724706264815\\
5335.61225735563	0.00629618290975367\\
5341.47234769869	0.0061524193568112\\
5347.33243804175	0.00589960924439551\\
5353.19252838481	0.00554396834214876\\
5359.05261872787	0.0050941229348565\\
5364.91270907093	0.00456090093196473\\
5370.77279941399	0.00395707155384661\\
5376.63288975705	0.0032970398478728\\
5382.49298010011	0.00259650333742868\\
5388.35307044317	0.00187207898338601\\
5394.21316078623	0.00114090931835258\\
5400.07325112929	0.000420257084672758\\
5405.93334147235	-0.000272902043125599\\
5411.79343181541	-0.000922279046914138\\
5417.65352215847	-0.0015126540634109\\
5423.51361250153	-0.00203023281513701\\
5429.37370284459	-0.00246296863114866\\
5435.23379318765	-0.00280084254465944\\
5441.09388353071	-0.00303609495944454\\
5446.95397387376	-0.00316340353342987\\
5452.81406421682	-0.00318000320655496\\
5458.67415455988	-0.00308574567399403\\
5464.53424490294	-0.00288309703758909\\
5470.394335246	-0.00257707382942487\\
5476.25442558906	-0.00217511905284908\\
5482.11451593212	-0.00168692129635067\\
5487.97460627518	-0.00112418131098558\\
5493.83469661824	-0.000500331670064663\\
5499.6947869613	0.000169783776807478\\
5505.55487730436	0.000870263010159422\\
5511.41496764742	0.00158452311235249\\
5517.27505799048	0.00229569275805176\\
5523.13514833354	0.00298701063816723\\
5528.9952386766	0.00364222041250649\\
5534.85532901965	0.00424595288912246\\
5540.71541936271	0.00478408644935242\\
5546.57550970577	0.00524407727078112\\
5552.43560004883	0.00561525162976472\\
5558.29569039189	0.00588905347689357\\
5564.15578073495	0.00605924154656835\\
5570.01587107801	0.00612203146568702\\
5575.87596142107	0.00607617962991959\\
5581.73605176413	0.00592300699925457\\
5587.59614210719	0.00566636238044875\\
5593.45623245025	0.00531252619536609\\
5599.31632279331	0.00487005713112554\\
5605.17641313637	0.00434958541164809\\
5611.03650347943	0.00376355767896498\\
5616.89659382249	0.00312593960227707\\
5622.75668416555	0.00245188331715586\\
5628.61677450861	0.00175736760918476\\
5634.47686485167	0.00105881938305717\\
5640.33695519473	0.000372725380382555\\
5646.19704553779	-0.000284756680071274\\
5652.05713588084	-0.000898178371887775\\
5657.9172262239	-0.00145316329447726\\
5663.77731656696	-0.00193674375738302\\
5669.63740691002	-0.00233766353442861\\
5675.49749725308	-0.002646639623084\\
5681.35758759614	-0.00285657692616216\\
5687.2176779392	-0.00296273089486637\\
5693.07776828226	-0.00296281441032375\\
5698.93785862532	-0.00285704650173869\\
5704.79794896838	-0.00264814187668898\\
5710.65803931144	-0.00234124163505816\\
5716.5181296545	-0.00194378692458454\\
5722.37821999756	-0.00146533863688936\\
5728.23831034062	-0.000917347507992998\\
5734.09840068368	-0.000312880147372273\\
5739.95849102674	0.000333692453135872\\
5745.8185813698	0.00100703650901828\\
5751.67867171286	0.00169121901555685\\
5757.53876205592	0.00237008468625006\\
5763.39885239898	0.00302763734357482\\
5769.25894274204	0.00364841676164796\\
5775.11903308509	0.00421786208702458\\
5780.97912342815	0.00472265329803579\\
5786.83921377121	0.00515102269908503\\
5792.69930411427	0.00549302916919449\\
5798.55939445733	0.00574078877660408\\
5804.41948480039	0.00588865641368789\\
5810.27957514345	0.00593335427217524\\
5816.13966548651	0.00587404423898748\\
5821.99975582957	0.00571234262291438\\
5827.85984617263	0.00545227698349191\\
5833.71993651569	0.00510018619935562\\
5839.58002685875	0.00466456624824702\\
5845.44011720181	0.00415586544681112\\
5851.30020754487	0.00358623408019082\\
5857.16029788793	0.00296923441796199\\
5863.02038823099	0.00231951803323882\\
5868.88047857404	0.00165247809928863\\
5874.7405689171	0.000983884911402183\\
5880.60065926016	0.000329513260952879\\
5886.46074960322	-0.000295229537481451\\
5892.32083994628	-0.000875666198067985\\
5898.18093028934	-0.0013981940668296\\
5904.0410206324	-0.00185060368842271\\
5909.90111097546	-0.00222236387948641\\
5915.76120131852	-0.00250486665543998\\
5921.62129166158	-0.00269162631846643\\
5927.48138200464	-0.00277842810518147\\
5933.3414723477	-0.00276342299202658\\
5939.20156269076	-0.00264716653143206\\
5945.06165303382	-0.00243260091635069\\
5950.92174337688	-0.00212498080952864\\
5956.78183371994	-0.00173174479983086\\
5962.64192406299	-0.00126233562435079\\
5968.50201440605	-0.000727973497765935\\
5974.36210474911	-0.000141387986402107\\
5980.22219509217	0.000483485169084496\\
5986.08228543523	0.00113183506484267\\
5991.94237577829	0.00178832661968941\\
5997.80246612135	0.00243746322122374\\
6003.66255646441	0.00306395245963145\\
6009.52264680747	0.00365306632065881\\
6015.38273715053	0.00419098735696594\\
6021.24282749359	0.00466513270465152\\
6027.10291783665	0.00506444834994351\\
6032.96300817971	0.0053796667672068\\
6038.82309852277	0.00560352192679455\\
6044.68318886583	0.00573091668791684\\
6050.54327920889	0.00575903872375827\\
6056.40336955195	0.00568742234715295\\
6062.26345989501	0.00551795488594006\\
6068.12355023807	0.00525482756537569\\
6073.98364058113	0.00490443216626425\\
6079.84373092419	0.00447520599931261\\
6085.70382126725	0.00397742895308915\\
6091.5639116103	0.00342297749337938\\
6097.42400195336	0.00282504149783154\\
6103.28409229642	0.0021978106739798\\
6109.14418263948	0.00155613801297503\\
6115.00427298254	0.000915188258319774\\
6120.8643633256	0.00029007970600882\\
6126.72445366866	-0.000304472205703852\\
6132.58454401172	-0.000854501117769777\\
6138.44463435478	-0.00134711764011586\\
6144.30472469784	-0.00177081121470925\\
6150.1648150409	-0.00211571888554609\\
6156.02490538396	-0.00237385470245888\\
6161.88499572702	-0.00253929442639755\\
6167.74508607008	-0.0026083112671582\\
6173.60517641314	-0.00257945954782799\\
6179.4652667562	-0.00245360442278085\\
6185.32535709926	-0.00223389705270557\\
6191.18544744232	-0.00192569592569518\\
6197.04553778538	-0.00153643628271601\\
6202.90562812843	-0.00107545082509682\\
6208.76571847149	-0.000553746023853492\\
6214.62580881455	1.62606105484162e-05\\
6220.48589915761	0.000621036029345132\\
6226.34598950067	0.00124625288840901\\
6232.20607984373	0.0018771289102097\\
6238.06617018679	0.00249877632929415\\
6243.92626052985	0.00309655316352792\\
6249.78635087291	0.00365640802571516\\
6255.64644121597	0.00416521035809779\\
6261.50653155903	0.00461105833162487\\
6267.36662190209	0.00498355719262695\\
6273.22671224515	0.00527406154989258\\
6279.08680258821	0.00547587595770156\\
6284.94689293126	0.005584409144124\\
6290.80698327432	0.00559727833517958\\
6296.66707361738	0.00551436130853336\\
6302.52716396044	0.00533779504811387\\
6308.3872543035	0.00507192113012198\\
6314.24734464656	0.00472317922852225\\
6320.10743498962	0.00429995134713795\\
6325.96752533268	0.00381236054331838\\
6331.82761567574	0.00327202897234065\\
6337.6877060188	0.00269180103443843\\
6343.54779636186	0.00208543821521955\\
6349.40788670492	0.00146729286793934\\
6355.26797704798	0.000851968667284953\\
6361.12806739104	0.000253975765548383\\
6366.9881577341	-0.000312611208892719\\
6372.84824807716	-0.000834484252292827\\
6378.70833842022	-0.00129941439902993\\
6384.56842876328	-0.00169653811643794\\
6390.42851910634	-0.00201661097626424\\
6396.2886094494	-0.0022522226819947\\
6402.14869979245	-0.00239796845315791\\
6408.00879013551	-0.00245057280581024\\
6413.86888047857	-0.00240896289796897\\
6419.72897082163	-0.0022742898028516\\
6425.58906116469	-0.00204989730375426\\
6431.44915150775	-0.00174123904191137\\
6437.30924185081	-0.00135574606517833\\
6443.16933219387	-0.000902647990680954\\
6449.02942253693	-0.000392752082520558\\
6454.88951287999	0.000161814470056469\\
6460.74960322305	0.000747892930735089\\
6466.60969356611	0.00135160519540629\\
6472.46978390917	0.00195868236896549\\
6478.32987425223	0.00255480196276792\\
6484.18996459529	0.00312592574678473\\
6490.05005493835	0.00365863028905291\\
6495.91014528141	0.00414042240254238\\
6501.77023562447	0.00456003208852832\\
6507.63032596753	0.00490767611017306\\
6513.49041631059	0.00517528603443037\\
6519.35050665365	0.005356695429094\\
6525.21059699671	0.005447781874524\\
6531.07068733976	0.00544656052279753\\
6536.93077768282	0.00535322708461923\\
6542.79086802588	0.00517014932109206\\
6548.65095836894	0.00490180733291645\\
6554.511048712	0.00455468414590147\\
6560.37113905506	0.00413710926157759\\
6566.23122939812	0.00365905894498741\\
6572.09131974118	0.00313191803605909\\
6577.95141008424	0.00256820896834099\\
6583.8115004273	0.00198129444312221\\
6589.67159077036	0.00138506081597554\\
6595.53168111342	0.000793589695060228\\
6601.39177145648	0.000220825515518639\\
6607.25186179954	-0.000319753064928165\\
6613.1119521426	-0.000815450207793016\\
6618.97204248565	-0.00125465091792605\\
6624.83213282871	-0.0016270930563418\\
6630.69222317177	-0.00192410697380834\\
6636.55231351483	-0.00213881717160385\\
6642.41240385789	-0.0022663013013571\\
6648.27249420095	-0.00230370282935873\\
6654.13258454401	-0.00225029479024497\\
6659.99267488707	-0.0021074932120618\\
6665.85276523013	-0.00187881998418719\\
6671.71285557319	-0.00156981613176464\\
6677.57294591625	-0.00118790762803002\\
6683.43303625931	-0.000742226991104452\\
6689.29312660237	-0.000243394948793349\\
6695.15321694543	0.000296732611241846\\
6701.01330728849	0.000865346378826853\\
6706.87339763155	0.0014489873534096\\
6712.7334879746	0.00203386537113897\\
6718.59357831766	0.00260618489054492\\
6724.45366866072	0.00315247034271061\\
6730.31375900378	0.00365988337488168\\
6736.17384934684	0.00411652452141069\\
6742.0339396899	0.00451171221529959\\
6747.89403003296	0.00483623260068631\\
6753.75412037602	0.00508255430570568\\
6759.61421071908	0.00524500317214631\\
6765.47430106214	0.00531989288959067\\
6771.3343914052	0.00530560853042484\\
6777.19448174826	0.00520264109569365\\
6783.05457209132	0.00501357234097803\\
6788.91466243438	0.00474301032587694\\
6794.77475277744	0.00439747728922471\\
6800.6348431205	0.00398525257689921\\
6806.49493346356	0.00351617440088875\\
6812.35502380662	0.00300140517511289\\
6818.21511414968	0.00245316602226645\\
6824.07520449274	0.00188444676516167\\
6829.9352948358	0.00130869828155682\\
6835.79538517886	0.000739514507128819\\
6841.65547552191	0.000190311602124636\\
6847.51556586497	-0.00032598814825652\\
6853.37565620803	-0.000797260254116888\\
6859.23574655109	-0.0012124626159133\\
6865.09583689415	-0.00156189411605015\\
6870.95592723721	-0.00183742115323662\\
6876.81601758027	-0.00203266683163263\\
6882.67610792333	-0.00214315840660015\\
6888.53619826639	-0.00216642957996259\\
6894.39628860945	-0.00210207530847515\\
6900.25637895251	-0.00195175791239929\\
6906.11646929557	-0.00171916442158967\\
6911.97655963863	-0.00140991624614854\\
6917.83664998169	-0.00103143338130802\\
6923.69674032475	-0.000592756424064951\\
6929.55683066781	-0.000104330668892562\\
6935.41692101087	0.00042224256351902\\
6941.27701135393	0.0009744814463379\\
6947.13710169699	0.0015393194740748\\
6952.99719204004	0.00210341458817649\\
6958.8572823831	0.00265346428939738\\
6964.71737272616	0.00317651929661852\\
6970.57746306922	0.00366028835835199\\
6976.43755341228	0.00409342704370376\\
6982.29764375534	0.00446580372984344\\
6988.1577340984	0.00476873655094517\\
6994.01782444146	0.00499519576923716\\
6999.87791478452	0.0051399668529639\\
7005.73800512758	0.00519977047984608\\
7011.59809547064	0.00517333670794662\\
7017.4581858137	0.00506143163847697\\
7023.31827615676	0.00486683602111729\\
7029.17836649982	0.00459427638360516\\
7035.03845684288	0.0042503103878975\\
7040.89854718593	0.00384316919069466\\
7046.75863752899	0.00338256059547219\\
7052.61872787205	0.00287943770187893\\
7058.47881821511	0.00234573856389811\\
7064.33890855817	0.00179410304284405\\
7070.19899890123	0.00123757356929216\\
7076.05908924429	0.000689286895977135\\
7081.91917958735	0.000162164125179634\\
7087.77926993041	-0.000331393677184145\\
7093.63936027347	-0.000779797112215492\\
7099.49945061653	-0.0011725404670072\\
7105.35954095959	-0.00150044773905863\\
7111.21963130265	-0.00175588690663076\\
7117.07972164571	-0.00193294744323167\\
7122.93981198877	-0.00202757694959323\\
7128.79990233183	-0.00203767374653313\\
7134.65999267489	-0.00196313331618809\\
7140.52008301795	-0.00180584757024491\\
7146.38017336101	-0.00156965703806855\\
7152.24026370407	-0.00126025717732863\\
7158.10035404712	-0.000885061089799094\\
7163.96044439018	-0.000453021949212453\\
7169.82053473324	2.55806072418099e-05\\
7175.6806250763	0.000539385029281543\\
7181.54071541936	0.00107621752968296\\
7187.40080576242	0.00162338048697094\\
7193.26089610548	0.00216795274406222\\
7199.12098644854	0.00269709470247947\\
7204.9810767916	0.00319835101011038\\
7210.84116713466	0.00365994370832216\\
7216.70125747772	0.00407104894102709\\
7222.56134782078	0.0044220507260332\\
7228.42143816384	0.00470476584106593\\
7234.2815285069	0.0049126345663638\\
7240.14161884996	0.00504087284005571\\
7246.00170919302	0.00508658229843938\\
7251.86179953608	0.00504881567328607\\
7257.72188987914	0.0049285960728339\\
7263.5819802222	0.00472888976538742\\
7269.44207056526	0.00445453317931075\\
7275.30216090832	0.00411211591289026\\
7281.16225125137	0.00370982258164542\\
7287.02234159443	0.0032572372966923\\
7292.88243193749	0.00276511544347561\\
7298.74252228055	0.00224512819346433\\
7304.60261262361	0.00170958581632914\\
7310.46270296667	0.00117114635007881\\
7316.32279330973	0.000642516522328329\\
7322.18288365279	0.000136151987313901\\
7328.04297399585	-0.000336036051262834\\
7333.90306433891	-0.000762960923031298\\
7339.76315468197	-0.00113462069513754\\
7345.62324502503	-0.00144233239245914\\
7351.48333536809	-0.00167893475050007\\
7357.34342571115	-0.0018389547684456\\
7363.20351605421	-0.00191873419030468\\
7369.06360639727	-0.00191651299321907\\
7374.92369674032	-0.00183246798045045\\
7380.78378708338	-0.00166870563765352\\
7386.64387742644	-0.00142920949135318\\
7392.5039677695	-0.0011197432806885\\
7398.36405811256	-0.000747712293557661\\
7404.22414845562	-0.000321986201267342\\
7410.08423879868	0.000147312371606068\\
7415.94432914174	0.000649048503737084\\
7421.8044194848	0.00117133914015726\\
7427.66450982786	0.00170183449662479\\
7433.52460017092	0.00222801004266241\\
7439.38469051398	0.00273746216191537\\
7445.24478085704	0.00321820051139937\\
7451.1048712001	0.0036589301905883\\
7456.96496154316	0.00404931708139347\\
7462.82505188622	0.00438023012724407\\
7468.68514222927	0.00464395487261931\\
7474.54523257233	0.00483437327082675\\
7480.40532291539	0.00494710557064382\\
7486.26541325845	0.00497961099312772\\
7492.12550360151	0.00493124488621574\\
7497.98559394457	0.00480327107548022\\
7503.84568428763	0.00459882918638366\\
7509.70577463069	0.00432285777721249\\
7515.56586497375	0.00398197516092144\\
7521.42595531681	0.00358432079081758\\
7527.28604565987	0.00313936100899797\\
7533.14613600293	0.00265766379220044\\
7539.00622634599	0.00215064785350499\\
7544.86631668905	0.00163031205512012\\
7550.72640703211	0.00110895154258507\\
7556.58649737517	0.000598867314853788\\
7562.44658771823	0.000112076088802196\\
7568.30667806129	-0.000339972699795591\\
7574.16676840435	-0.000746666095729551\\
7580.02685874741	-0.00109847669622303\\
7585.88694909047	-0.00138718575129767\\
7591.74703943353	-0.00160607506982482\\
7597.60712977658	-0.00175008325178003\\
7603.46722011964	-0.0018159226151982\\
7609.3273104627	-0.00180215411821402\\
7615.18740080576	-0.00170921857177884\\
7621.04749114882	-0.00153942347109906\\
7626.90758149188	-0.00129688582194969\\
7632.76767183494	-0.000987432375046988\\
7638.627762178	-0.000618459683814279\\
7644.48785252106	-0.000198757344909064\\
7650.34794286412	0.000261701355847398\\
7656.20803320718	0.000751996407223845\\
7662.06812355024	0.00126052001710392\\
7667.9282138933	0.0017752513990113\\
7673.78830423636	0.00228404089697095\\
7679.64839457942	0.00277489673474384\\
7685.50848492248	0.00323626762295892\\
7691.36857526554	0.00365731456558242\\
7697.2286656086	0.00402816547112825\\
7703.08875595166	0.00434014658960261\\
7708.94884629471	0.00458598535043495\\
7714.80893663777	0.00475997985931153\\
7720.66902698083	0.00485813110556342\\
7726.52911732389	0.00487823481579208\\
7732.38920766695	0.00481993084626727\\
7738.24929801001	0.00468470901040696\\
7744.10938835307	0.00447587126749133\\
7749.96947869613	0.00419845122673438\\
7755.82956903919	0.0038590929269255\\
7761.68965938225	0.00346589180955359\\
7767.54974972531	0.00302820168944862\\
7773.40984006837	0.00255641232483692\\
7779.26993041143	0.0020617028745361\\
7785.13002075449	0.00155577709121441\\
7790.99011109754	0.00105058652195334\\
7796.8502014406	0.000558048261289596\\
7802.71029178366	8.97639209317597e-05\\
7808.57038212672	-0.000343253558008612\\
7814.43047246978	-0.000730838818666499\\
7820.29056281284	-0.00106391264226396\\
7826.1506531559	-0.00133469454520824\\
7832.01074349896	-0.00153688444269864\\
7837.87083384202	-0.00166580913852832\\
7843.73092418508	-0.00171853023641277\\
7849.59101452814	-0.00169391098271643\\
7855.4511048712	-0.0015926405225492\\
7861.31119521426	-0.00141721505745585\\
7867.17128555732	-0.0011718764103387\\
7873.03137590038	-0.000862509506677961\\
7878.89146624344	-0.000496501247896817\\
7884.7515565865	-8.25641599395829e-05\\
7890.61164692956	0.000369470974443799\\
7896.47173727262	0.000848888565200893\\
7902.33182761568	0.00134434220401038\\
7908.19191795873	0.001844123173884\\
7914.05200830179	0.00233643712631276\\
7919.91209864485	0.00280968238947026\\
7925.77218898791	0.00325272334162619\\
7931.63227933097	0.00365515240923434\\
7937.49236967403	0.00400753452702635\\
7943.35246001709	0.00430162831903393\\
7949.21255036015	0.00453057881694765\\
7955.07264070321	0.00468907721033001\\
7960.93273104627	0.00477348390764992\\
7966.79282138933	0.00478191205601213\\
7972.65291173239	0.00471426960627653\\
7978.51300207545	0.00457225898802852\\
7984.37309241851	0.00435933446178328\\
7990.23318276157	0.00408061821249624\\
7996.09327310463	0.00374277722110838\\
8001.95336344769	0.0033538638718244\\
8007.81345379075	0.0029231241047975\\
8013.67354413381	0.00246077768362429\\
8019.53363447687	0.00197777579864329\\
8025.39372481993	0.00148554175362422\\
8031.25381516299	0.000995700875414336\\
8037.11390550604	0.000519806030933051\\
8042.9739958491	6.90652311472127e-05\\
8048.83408619216	-0.000345922256493886\\
8054.69417653522	-0.000715415075374337\\
8060.55426687828	-0.001030758370533\\
8066.41435722134	-0.00128458643998159\\
8072.2744475644	-0.00147099470286946\\
8078.13453790746	-0.00158567697055843\\
8083.99462825052	-0.00162602483077163\\
8089.85471859358	-0.00159118685275138\\
8095.71480893664	-0.00148208627178479\\
8101.5748992797	-0.00130139679299883\\
8107.43498962276	-0.00105347714186791\\
8113.29507996582	-0.000744265961507545\\
8119.15517030888	-0.000381139589259708\\
8125.01526065193	2.72638824057015e-05\\
8130.87535099499	0.000471250065677629\\
8136.73544133805	0.000940298371978892\\
8142.59553168111	0.00142331128130615\\
8148.45562202417	0.00190887688143375\\
8154.31571236723	0.00238553846709102\\
8160.17580271029	0.00284206483124809\\
8166.03589305335	0.00326771487287228\\
8171.89598339641	0.00365249028792505\\
8177.75607373947	0.00398737040045396\\
8183.61616408253	0.0042645236188018\\
8189.47625442559	0.00447749056206788\\
8195.33634476865	0.00462133457518301\\
8201.19643511171	0.00469275612697605\\
8207.05652545477	0.00469016844152466\\
8212.91661579783	0.00461373263109226\\
8218.77670614088	0.00446535155618922\\
8224.63679648394	0.00424862261316535\\
8230.496886827	0.00396875061699897\\
8236.35697717006	0.00363242288834444\\
8242.21706751312	0.00324764954019057\\
8248.07715785618	0.00282357277835864\\
8253.93724819924	0.002370249753967\\
8259.7973385423	0.00189841412506393\\
8265.65742888536	0.00141922197891081\\
8271.51751922842	0.000943988128342759\\
8277.37760957148	0.000483919014221943\\
8283.23769991454	4.98485181683334e-05\\
8289.0977902576	-0.000348017087531137\\
8294.95788060066	-0.000700339049596867\\
8300.81797094372	-0.000998865264702988\\
8306.67806128678	-0.00123662349096209\\
8312.53815162984	-0.00140808411730959\\
8318.3982419729	-0.00150928869192557\\
8324.25833231596	-0.00153794122356749\\
8330.11842265902	-0.0014934601513483\\
8335.97851300208	-0.00137698980918325\\
8341.83860334514	-0.00119137116750191\\
8347.69869368819	-0.000941072596011324\\
8353.55878403125	-0.000632082332671406\\
8359.41887437431	-0.000271765245471788\\
8365.27896471737	0.000131312688125549\\
8371.13905506043	0.000567588002048909\\
8376.99914540349	0.00102672662364745\\
8382.85923574655	0.00149786863542219\\
8388.71932608961	0.00196988512036785\\
8394.57941643267	0.00243164101740384\\
8400.43950677573	0.00287225777954352\\
8406.29959711879	0.00328136964013129\\
8412.15968746185	0.00364936745141265\\
8418.01977780491	0.00396762436058506\\
8423.87986814797	0.0042286980239978\\
8429.73995849103	0.00442650462099318\\
8435.60004883409	0.00455646059946184\\
8441.46013917715	0.00461558885176763\\
8447.32022952021	0.00460258686299975\\
8453.18031986327	0.00451785527312759\\
8459.04041020633	0.00436348623070412\\
8464.90050054939	0.00414321186451714\\
8470.76059089244	0.00386231414000165\\
8476.6206812355	0.00352749827598055\\
8482.48077157856	0.00314673275453141\\
8488.34086192162	0.00272905973982109\\
8494.20095226468	0.00228438041581017\\
8500.06104260774	0.00182322033744776\\
8505.9211329508	0.00135648035571344\\
8511.78122329386	0.000895179009245003\\
8517.64131363692	0.000450192469494161\\
8523.50140397998	3.19981760311385e-05\\
8529.36149432304	-0.000349571796148656\\
8535.2215846661	-0.000685561832787853\\
8541.08167500916	-0.000968102909160292\\
8546.94176535222	-0.00119059682223355\\
8552.80185569527	-0.00134787021387826\\
8558.66194603833	-0.00143629479067399\\
8564.52203638139	-0.00145387094796795\\
8570.38212672445	-0.00140027287203389\\
8576.24221706751	-0.00127685410565536\\
8582.10230741057	-0.0010866134956365\\
8587.96239775363	-0.000834122375845357\\
8593.82248809669	-0.000525414751988675\\
8599.68257843975	-0.000167843125083536\\
8605.54266878281	0.000230096602570953\\
8611.40275912587	0.000658966970117599\\
8617.26284946893	0.00110861275528065\\
8623.12293981199	0.00156840141982746\\
8628.98303015505	0.00202747451125621\\
8634.84312049811	0.00247500407744168\\
8640.70321084117	0.0029004480401003\\
8646.56330118423	0.00329379850529249\\
8652.42339152729	0.00364581716300942\\
8658.28348187034	0.00394825223982305\\
8664.1435722134	0.00419403191096226\\
8670.00366255646	0.00437742963913729\\
8675.86375289952	0.00449419757514533\\
8681.72384324258	0.0045416649148009\\
8687.58393358564	0.00451879893457501\\
8693.4440239287	0.00442622731398859\\
8699.30411427176	0.00426622126576131\\
8705.16420461482	0.00404263992075257\\
8711.02429495788	0.00376083732708757\\
8716.88438530094	0.00342753430388004\\
8722.744475644	0.00305065821529802\\
8728.60456598706	0.00263915448377542\\
8734.46465633012	0.00220277432339206\\
8740.32474667318	0.00175184372891886\\
8746.18483701624	0.00129701919275337\\
8752.0449273593	0.000849035927562462\\
8757.90501770236	0.000418454542535058\\
8763.76510804542	1.54121500532384e-05\\
8769.62519838848	-0.000350616232252363\\
8775.48528873154	-0.000671040369009993\\
8781.3453790746	-0.000938356351398894\\
8787.20546941765	-0.00114632227046114\\
8793.06555976071	-0.00129010390772904\\
8798.92565010377	-0.0013663870435062\\
8804.78574044683	-0.00137345377493222\\
8810.64583078989	-0.00131122108748717\\
8816.50592113295	-0.00118124081662268\\
8822.36601147601	-0.0009866610474207\\
8828.22610181907	-0.000732149909718607\\
8834.08619216213	-0.00042378361213721\\
8839.94628250519	-6.89013992747874e-05\\
8845.80637284825	0.000324069106560004\\
8851.66646319131	0.000745812023475895\\
8857.52655353437	0.00118634403701513\\
8863.38664387743	0.00163525070234832\\
8869.24673422049	0.00208193262934456\\
8875.10682456355	0.00251585573192684\\
8880.96691490661	0.00292679963415491\\
8886.82700524966	0.0033050983764149\\
8892.68709559272	0.00364186775339929\\
8898.54718593578	0.00392921393786518\\
8904.40727627884	0.00416041849243725\\
8910.2673666219	0.00433009543330857\\
8916.12745696496	0.00443431667582854\\
8921.98754730802	0.00447070294047248\\
8927.84763765108	0.00443847801676287\\
8933.70772799414	0.00433848514944135\\
8939.5678183372	0.0041731652057437\\
8945.42790868026	0.00394649718408987\\
8951.28799902332	0.00366390251348384\\
8957.14808936638	0.00333211544331347\\
8963.00817970944	0.00295902262171462\\
8968.8682700525	0.00255347568281955\\
8974.72836039555	0.00212508129596976\\
8980.58845073861	0.00168397365532782\\
8986.44854108167	0.00124057479759203\\
8992.30863142473	0.000805348415206443\\
8998.16872176779	0.000388552979863155\\
nan	nan\\
9009.88890245391	-0.000351176891762859\\
9015.74899279697	-0.000656736587672686\\
9021.60908314003	-0.000909523846688235\\
9027.46917348309	-0.00110363679407522\\
9033.32926382615	-0.00123456465848646\\
9039.18935416921	-0.0012992925329808\\
9045.04944451227	-0.00129637072791189\\
9050.90953485533	-0.00122594712163562\\
9056.76962519839	-0.00108976178970174\\
9062.62971554145	-0.000891104083604343\\
9068.48980588451	-0.000634733215633085\\
9074.34989622757	-0.000326764266075394\\
9080.20998657063	2.54776579995897e-05\\
9086.07007691369	0.000413631634918769\\
9091.93016725675	0.000828499369137581\\
9097.79025759981	0.00126026315146303\\
9103.65034794286	0.00169871817493735\\
9109.51043828592	0.00213351370972789\\
9115.37052862898	0.00255439743838982\\
9121.23061897204	0.00295145718423983\\
9127.0907093151	0.00331535433450035\\
9132.95079965816	0.00363754346196718\\
9138.81089000122	0.00391047298120032\\
9144.67098034428	0.00412776212804796\\
9150.53107068734	0.0042843501160307\\
9156.3911610304	0.00437661398349029\\
9162.25125137346	0.00440245238819283\\
9168.11134171652	0.00436133341242939\\
9173.97143205958	0.00425430529276466\\
9179.83152240264	0.00408396986436188\\
9185.6916127457	0.00385441938930239\\
9191.55170308876	0.00357113830113054\\
9197.41179343182	0.00324087222426609\\
9203.27188377488	0.00287146739490105\\
9209.13197411794	0.00247168430556852\\
9214.992064461	0.00205098999858324\\
9220.85215480405	0.0016193339321104\\
9226.71224514711	0.00118691272472488\\
9232.57233549017	0.000763929339960075\\
9238.43242583323	0.000360352397762063\\
9244.29251617629	-1.43187103329364e-05\\
9250.15260651935	-0.000351277368403233\\
9256.01269686241	-0.000642616685997277\\
9261.87278720547	-0.000881514988088023\\
9267.73287754853	-0.00106239549452392\\
9273.59296789159	-0.00118105645194892\\
9279.45305823465	-0.00123476868250341\\
9285.31314857771	-0.00122233828547741\\
9291.17323892077	-0.001144133052739\\
9297.03332926383	-0.00100207201605785\\
9302.89341960689	-0.000799578414947775\\
9308.75350994994	-0.000541497235881272\\
9314.613600293	-0.000233979308234936\\
9320.47369063606	0.000115665271268403\\
9326.33378097912	0.000499140893397119\\
9332.19387132218	0.000907363241330561\\
9338.05396166524	0.00133067447743953\\
9343.9140520083	0.00175907171526883\\
9349.77414235136	0.00218244337278847\\
9355.63423269442	0.00259080782287016\\
9361.49432303748	0.00297454870915472\\
9367.35441338054	0.00332464138060481\\
9373.2145037236	0.00363286511371622\\
9379.07459406666	0.00389199613237378\\
9384.93468440972	0.00409597689430078\\
9390.79477475278	0.00424005767722932\\
9396.65486509584	0.0043209071568294\\
9402.5149554389	0.00433668940223972\\
9408.37504578196	0.00428710551121622\\
9414.23513612501	0.00417339894181847\\
9420.09522646807	0.00399832445619156\\
9425.95531681113	0.00376608144899409\\
9431.81540715419	0.00348221327314416\\
9437.67549749725	0.00315347497583804\\
9443.53558784031	0.00278767259901116\\
9449.39567818337	0.00239347786758938\\
9455.25576852643	0.00198022266306069\\
9461.11585886949	0.00155767815331197\\
9466.97594921255	0.00113582380540982\\
9472.83603955561	0.000724611741319333\\
9478.69612989867	0.000333731999683831\\
9484.55622024173	-2.76157599881354e-05\\
9490.41631058479	-0.000350938732998813\\
9496.27640092785	-0.000628650531872853\\
9502.13649127091	-0.000854249146834765\\
9507.99658161397	-0.00102246913060506\\
9513.85667195703	-0.0011294044460698\\
9519.71676230009	-0.00117259911164546\\
9525.57685264315	-0.00115110354139142\\
9531.43694298621	-0.00106549528902926\\
9537.29703332927	-0.000917863745331579\\
9543.15712367232	-0.000711759189798616\\
9549.01721401538	-0.000452107437206596\\
9554.87730435844	-0.000145092130400868\\
9560.7373947015	0.000201992508947934\\
9566.59748504456	0.000580914965836019\\
9572.45757538762	0.000982701637709513\\
9578.31766573068	0.0013978493320181\\
9584.17775607374	0.00181655002431203\\
9590.0378464168	0.00222892256176408\\
9595.89793675986	0.00262524583909361\\
9601.75802710292	0.00299618794527391\\
9607.61811744598	0.00333302587986098\\
9613.47820778904	0.0036278506670992\\
9619.3382981321	0.00387375304460541\\
9625.19838847516	0.00406498536853612\\
9631.05847881822	0.0041970959411474\\
9636.91856916128	0.00426703262295962\\
9642.77865950433	0.00427321331770823\\
9648.63874984739	0.00421556170466336\\
9654.49884019045	0.00409550741163851\\
9660.35893053351	0.00391595066399643\\
9666.21902087657	0.00368119228174126\\
9672.07911121963	0.00339683071277788\\
9677.93920156269	0.0030696285673527\\
9683.79929190575	0.00270735183443493\\
9689.65938224881	0.00231858560203689\\
9695.51947259187	0.00191253065364958\\
9701.37956293493	0.00149878575938505\\
9707.23965327799	0.00108712081242176\\
9713.09974362105	0.000687246171984016\\
9718.95983396411	0.000308583657897991\\
9724.81992430716	-3.99554034799944e-05\\
9730.68001465022	-0.000350179853627983\\
9736.54010499328	-0.000614811163982807\\
9742.40019533634	-0.000827654164184469\\
9748.2602856794	-0.000983742032508692\\
9754.12037602246	-0.00107945215577781\\
9759.98046636552	-0.001112590156819\\
9765.84055670858	-0.00108244014129332\\
9771.70064705164	-0.000989780014612611\\
9777.5607373947	-0.000836861544688236\\
9783.42082773776	-0.00062735567827998\\
9789.28091808082	-0.000366264437741306\\
9795.14100842388	-5.9801511773406e-05\\
9801.00109876694	0.000284755609934703\\
9806.86118911	0.000659238440118995\\
9812.72127945306	0.00105478113286438\\
9818.58136979612	0.00146203036832846\\
9824.44146013918	0.00187136651527814\\
9830.30155048224	0.00227313084248649\\
9836.1616408253	0.00265785341320656\\
9842.02173116836	0.00301647628608772\\
9847.88182151142	0.00334056676256645\\
9853.74191185447	0.00362251566100787\\
9859.60200219753	0.00385571595604088\\
9865.46209254059	0.00403471759031168\\
9871.32218288365	0.00415535483135437\\
9877.18227322671	0.00421484319840697\\
9883.04236356977	0.0042118437045222\\
9888.90245391283	0.00414649293305666\\
9894.76254425589	0.00402039827435218\\
9900.62263459895	0.00383659847186479\\
9906.48272494201	0.00359949044392328\\
9912.34281528507	0.00331472414342831\\
9918.20290562813	0.00298906796786863\\
9924.06299597119	0.00263024792603116\\
9929.92308631425	0.00224676438194026\\
9935.78317665731	0.00184769072254565\\
9941.64326700037	0.0014424587168023\\
9947.50335734343	0.00104063564336161\\
9953.36344768649	0.000651698452519214\\
9959.22353802955	0.000284810292908342\\
9965.08362837261	-5.13953281799235e-05\\
9970.94371871567	-0.000349017667190359\\
9976.80380905872	-0.000601074371164269\\
9982.66389940178	-0.000801665248406276\\
9988.52398974484	-0.000946110342293003\\
9994.3840800879	-0.00103105907773339\\
10000.244170431	-0.00105456793537535\\
10006.104260774	-0.00101614485376583\\
10011.9643511171	-0.000916759346183609\\
10017.8244414601	-0.000758818128777706\\
10023.6845318032	-0.00054610687034786\\
10029.5446221463	-0.00028369947249052\\
10035.4047124893	2.21629464914135e-05\\
10041.2648028324	0.000364220478364466\\
10047.1248931754	0.000734366733043322\\
10052.9849835185	0.00112384093899665\\
10058.8450738616	0.00152343528432678\\
10064.7051642046	0.00192371259194056\\
10070.5652545477	0.00231522918344578\\
10076.4253448907	0.0026887576695916\\
10082.2854352338	0.00303550441144205\\
10088.1455255769	0.00334731652848656\\
10094.0056159199	0.00361687358081613\\
10099.865706263	0.00383785941951596\\
10105.725796606	0.00400511017125079\\
10111.5858869491	0.00411473489116484\\
10117.4459772922	0.00416420606459005\\
10123.3060676352	0.00415241785433444\\
10129.1661579783	0.00407971075127527\\
10135.0262483213	0.00394786208149846\\
10140.8863386644	0.00376004262754251\\
10146.7464290074	0.00352074042202343\\
10152.6065193505	0.00323565354394054\\
10158.4666096936	0.0029115544778712\\
10164.3267000366	0.0025561292641822\\
10170.1867903797	0.0021777952602686\\
10176.0468807227	0.00178550183289332\\
10181.9069710658	0.00138851870049675\\
10187.7670614089	0.000996216930582045\\
10193.6271517519	0.000617847765708381\\
10199.487242095	0.000262324497739244\\
10205.347332438	-6.1987467481811e-05\\
10211.2074227811	-0.00034746741081545\\
10217.0675131242	-0.000587418336644094\\
10222.9276034672	-0.000776224040201815\\
10228.7876938103	-0.0009094805221939\\
10234.6477841533	-0.000984098676312138\\
10240.5078744964	-0.000998375855852838\\
10246.3679648395	-0.000952034662309488\\
10252.2280551825	-0.000846228073398956\\
10258.0881455256	-0.000683510823126175\\
10263.9482358686	-0.000467777742389184\\
10269.8083262117	-0.000204170547546042\\
10275.6684165548	0.000101044693077294\\
10281.5285068978	0.000440626495162082\\
10287.3885972409	0.000806529757394639\\
10293.2486875839	0.00119009634824351\\
10299.108777927	0.00158225996608609\\
10304.9688682701	0.00197376042610718\\
10310.8289586131	0.00235536230955139\\
10316.6890489562	0.00271807281358105\\
10322.5491392992	0.00305335366302287\\
10328.4092296423	0.00335332209138283\\
10334.2693199853	0.00361093616065831\\
10340.1294103284	0.00382016006354836\\
10345.9895006715	0.00397610552871888\\
10351.8495910145	0.00407514601630697\\
10357.7096813576	0.00411500103698702\\
10363.5697717006	0.00409478863462869\\
10369.4298620437	0.00401504482546074\\
10375.2899523868	0.0038777095678259\\
10381.1500427298	0.00368607962577793\\
10387.0101330729	0.00344472947134819\\
10392.8702234159	0.00315940212184735\\
10398.730313759	0.00283687251706336\\
10404.5904041021	0.00248478668600041\\
10410.4504944451	0.00211148052044146\\
10416.3105847882	0.00172578245042778\\
10422.1706751312	0.00133680469212515\\
10428.0307654743	0.000953728003514114\\
10433.8908558174	0.000585585031828312\\
10439.7509461604	0.000241047364741797\\
10445.6110365035	-7.17786946231689e-05\\
10451.4711268465	-0.000345542819937063\\
10457.3312171896	-0.000573823335669707\\
10463.1913075327	-0.00075127781698924\\
10469.0513978757	-0.000873768084241219\\
10474.9114882188	-0.00093845666803155\\
10480.7715785618	-0.000943872496941892\\
10486.6316689049	-0.000889944287736941\\
10492.491759248	-0.000778000881692425\\
10498.351849591	-0.000610738550696399\\
10504.2119399341	-0.000392156076208997\\
10510.0720302771	-0.000127459163385799\\
10515.9321206202	0.000177063525259744\\
10521.7922109633	0.000514189765212761\\
10527.6523013063	0.000875935045694689\\
10533.5123916494	0.00125374166453445\\
10539.3724819924	0.00163868116430806\\
10545.2325723355	0.0020216653216966\\
10551.0926626786	0.0023936607045229\\
10556.9527530216	0.00274590173169188\\
10562.8128433647	0.00307009721166655\\
10568.6729337077	0.00335862549278282\\
10574.5330240508	0.00360471363352503\\
10580.3931143938	0.00380259638053253\\
10586.2532047369	0.00394765122377145\\
10592.11329508	0.0040365063650835\\
10597.973385423	0.00406711907964583\\
10603.8334757661	0.00403882264862621\\
10609.6935661091	0.00395234078702719\\
10615.5536564522	0.00380976925589593\\
10621.4137467953	0.00361452512434587\\
10627.2738371383	0.00337126490894524\\
10633.1339274814	0.00308577355024298\\
10638.9940178244	0.00276482687384703\\
10644.8541081675	0.00241603080492403\\
10650.7141985106	0.00204764115135044\\
10656.5742888536	0.00166836822485437\\
10662.4343791967	0.00128717092367804\\
10668.2944695397	0.000913045144044303\\
10674.1545598828	0.000554811517343699\\
10680.0146502259	0.000220907480782324\\
10685.8747405689	-8.08114172161412e-05\\
10691.734830912	-0.000343256298508176\\
10697.594921255	-0.000560271477435844\\
10703.4550115981	-0.000726778812581885\\
10709.3151019412	-0.000838896503484505\\
10715.1751922842	-0.000894029553685494\\
10721.0352826273	-0.000890929792516086\\
10726.8953729703	-0.000829724068009946\\
10732.7554633134	-0.000711909975547922\\
10738.6155536565	-0.000540319253106635\\
10744.4756439995	-0.000319049736759018\\
10750.3357343426	-5.3367510771315e-05\\
10756.1958246856	0.000250418411753181\\
10762.0559150287	0.000585105895872639\\
10767.9160053717	0.000942770422780636\\
10773.7760957148	0.00131495271158743\\
10779.6361860579	0.00169285878333361\\
10785.4962764009	0.0020675677359539\\
10791.356366744	0.00243024232186667\\
10797.216457087	0.00277233735769858\\
10803.0765474301	0.00308580105198491\\
10808.9366377732	0.00336326450764631\\
10814.7967281162	0.00359821494007103\\
10820.6568184593	0.0037851485389755\\
10826.5169088023	0.00391969938738359\\
10832.3769991454	0.00399874141766316\\
10838.2370894885	0.00402046102484192\\
10844.0971798315	0.00398439865071868\\
10849.9572701746	0.00389145838628969\\
10855.8173605176	0.00374388539552924\\
10861.6774508607	0.00354521172185579\\
10867.5375412038	0.00330017178559698\\
10873.3976315468	0.00301458959249164\\
10879.2577218899	0.00269524034099343\\
10885.1178122329	0.00234968971526866\\
10890.977902576	0.0019861146766046\\
10896.8379929191	0.00161310999657833\\
10902.6980832621	0.00123948510972467\\
10908.5581736052	0.000874056086533502\\
10914.4182639482	0.000525437638647681\\
10920.2783542913	0.000201840063112207\\
10926.1384446344	-8.91240887661348e-05\\
10931.9985349774	-0.000340619065828483\\
10937.8586253205	-0.000546746483764736\\
10943.7187156635	-0.000702683632908234\\
10949.5788060066	-0.000804796284483161\\
10955.4388963497	-0.000850723357193363\\
10961.2989866927	-0.000839431472165179\\
10967.1590770358	-0.000771238136418995\\
10973.0191673788	-0.00064780303624405\\
10978.8792577219	-0.000472087674542846\\
10984.7393480649	-0.000248284332806725\\
10990.599438408	1.82839388482019e-05\\
10996.4595287511	0.000321289918300479\\
11002.3196190941	0.000653552384313878\\
11008.1797094372	0.00100720630188527\\
11014.0397997802	0.00137388898730985\\
11019.8998901233	0.00174493784698165\\
11025.7599804664	0.00211159501458328\\
11031.6200708094	0.00246521405312791\\
11037.4801611525	0.00279746384381757\\
11043.3402514955	0.0031005248539906\\
11049.2003418386	0.00336727316123264\\
11055.0604321817	0.00359144790304411\\
11060.9205225247	0.00376779821679077\\
11066.7806128678	0.00389220622127304\\
11072.6407032108	0.0039617831605952\\
11078.5007935539	0.00397493646565578\\
11084.360883897	0.00393140617704841\\
11090.22097424	0.0038322698974219\\
11096.0810645831	0.00367991618436731\\
11101.9411549261	0.00347798703931719\\
11107.8012452692	0.00323129087659242\\
11113.6613356123	0.00294568805187241\\
11119.5214259553	0.00262795167483101\\
11125.3815162984	0.00228560701108759\\
11131.2416066414	0.00192675328124091\\
11137.1016969845	0.00155987207647026\\
11142.9617873276	0.00119362692170963\\
11148.8218776706	0.000836658723243469\\
11154.6819680137	0.000497381929508431\\
11160.5420583567	0.000183786213358322\\
11166.4021486998	-9.6751650451197e-05\\
11172.2622390429	-0.00033764128368209\\
11178.1223293859	-0.00053323349861615\\
11183.982419729	-0.000678952752354025\\
11189.842510072	-0.000771404156191048\\
11195.7026004151	-0.000808452537875103\\
11201.5626907581	-0.000789271715960848\\
11207.4227811012	-0.000714362849918624\\
11213.2828714443	-0.000585541459919407\\
11219.1429617873	-0.000405893449987297\\
11225.0030521304	-0.000179701198804234\\
11230.8631424734	8.76585120602249e-05\\
11236.7232328165	0.000389842300133102\\
11242.5833231596	0.000719690676834046\\
11248.4434135026	0.00106939766507679\\
11254.3035038457	0.00143069552166349\\
11260.1635941887	0.00179505019347766\\
11266.0236845318	0.00215386288732921\\
11271.8837748749	0.00249867299264071\\
11277.7438652179	0.00282135756896189\\
11283.603955561	0.00311432269489252\\
11289.464045904	0.0033706821712818\\
11295.3241362471	0.00358441937418049\\
11301.1842265902	0.00375052845314002\\
11307.0443169332	0.00386513156220703\\
11312.9044072763	0.00392556937828703\\
11318.7644976193	0.00393046279449555\\
11324.6245879624	0.0038797443577932\\
11330.4846783055	0.00377465873547897\\
11336.3447686485	0.00361773222557966\\
11342.2048589916	0.00341271205754877\\
11348.0649493346	0.00316447694066062\\
11353.9250396777	0.00287892099522269\\
11359.7851300208	0.0025628138275367\\
11365.6452203638	0.00222364006997352\\
11371.5053107069	0.00186942218896967\\
11377.3654010499	0.00150853075579784\\
11383.225491393	0.00114948666634536\\
11389.0855817361	0.000800759982938834\\
11394.9456720791	0.000470570146491156\\
11400.8057624222	0.000166692271139188\\
11406.6658527652	-0.000103725914057236\\
11412.5259431083	-0.000334332166770206\\
11418.3860334513	-0.000519718923417286\\
11424.2461237944	-0.00065555007787772\\
11430.1062141375	-0.000738662375043392\\
11435.9663044805	-0.000767139048648679\\
11441.8263948236	-0.000740353989706235\\
11447.6864851666	-0.000658985428308061\\
11453.5465755097	-0.000524998831970757\\
11459.4066658528	-0.0003415994497413\\
11465.2667561958	-0.00011315564734382\\
11471.1268465389	0.000154905133447802\\
11476.9869368819	0.000456225313723368\\
11482.847027225	0.000783667951793509\\
11488.7071175681	0.00112948577771722\\
11494.5672079111	0.00148550448483533\\
11500.4272982542	0.00184331594227765\\
11506.2873885972	0.00219447676199529\\
11512.1474789403	0.00253070753122748\\
11518.0075692834	0.00284408800978299\\
11523.8676596264	0.00312724369047676\\
11529.7277499695	0.00337351932864017\\
11535.5878403125	0.00357713535795049\\
11541.4479306556	0.00373332351644226\\
11547.3080209987	0.00383843850080378\\
11553.1681113417	0.00389004303458434\\
11559.0282016848	0.00388696436550349\\
11564.8882920278	0.00382932088312443\\
11570.7483823709	0.00371851825338741\\
11576.608472714	0.00355721518658443\\
11582.468563057	0.00334925967178131\\
11588.3286534001	0.00309959720636045\\
11594.1887437431	0.00281415320880766\\
11600.0488340862	0.00249969241031521\\
11605.9089244293	0.00216365856146674\\
11611.7690147723	0.00181399825166154\\
11617.6291051154	0.00145897301096747\\
11623.4891954584	0.0011069641368053\\
11629.3492858015	0.000766274855940216\\
11635.2093761446	0.000444934491304871\\
11641.0694664876	0.000150509252026209\\
11646.9295568307	-0.000110075895037949\\
11652.7896471737	-0.000330700078967078\\
11658.6497375168	-0.000506190274248848\\
11664.5098278598	-0.000632442570444803\\
11670.3699182029	-0.000706518119476864\\
11676.230008546	-0.000726711517823789\\
11682.090098889	-0.000692590032835402\\
11687.9501892321	-0.00060500277171346\\
11693.8102795751	-0.000466059601298056\\
11699.6703699182	-0.000279080340854924\\
11705.5304602613	-4.85154505816385e-05\\
11711.3905506043	0.000220159889687517\\
11717.2506409474	0.000520575791086038\\
11723.1107312904	0.000845618668812081\\
11728.9708216335	0.00118759967797026\\
11734.8309119766	0.00153843658425448\\
11740.6910023196	0.0018898447679716\\
11746.5510926627	0.00223353284780547\\
11752.4111830057	0.00256139830547651\\
11758.2712733488	0.00286571849612794\\
11764.1313636919	0.0031393325419885\\
11769.9914540349	0.00337580982533867\\
11775.851544378	0.00356960111660585\\
11781.711634721	0.00371616878650527\\
11787.5717250641	0.00381209304655986\\
11793.4318154072	0.00385515173155192\\
11799.2919057502	0.00384437176269776\\
11805.1519960933	0.00378005110029099\\
11811.0120864363	0.00366375069161708\\
11816.8721767794	0.00349825662833911\\
11822.7322671225	0.00328751343019484\\
11828.5923574655	0.00303653005164627\\
11834.4524478086	0.00275126085144999\\
11840.3125381516	0.00243846435305652\\
11846.1726284947	0.00210554314583713\\
11852.0327188377	0.00176036871911771\\
11857.8928091808	0.0014110953738765\\
11863.7528995239	0.00106596761078129\\
11869.6129898669	0.00073312554389767\\
11875.47308021	0.000420412932435065\\
11881.333170553	0.000135192357265997\\
11887.1932608961	-0.000115828103150199\\
11893.0533512392	-0.000326752617424923\\
11898.9134415822	-0.000492636057467425\\
11904.7735319253	-0.000609599915083926\\
11910.6336222683	-0.000674922962167231\\
11916.4937126114	-0.000687104535664512\\
11922.3538029545	-0.000645898976044536\\
11928.2138932975	-0.000552320429514992\\
11934.0739836406	-0.000408617924373255\\
11939.9340739836	-0.000218221332815659\\
11945.7941643267	1.43404838178694e-05\\
11951.6542546698	0.000283547393260629\\
11957.5143450128	0.000583019012334543\\
11963.3744353559	0.00090566591928914\\
11969.2345256989	0.00124385747522906\\
11975.094616042	0.00158960228214142\\
11980.9547063851	0.00193473701038974\\
11986.8147967281	0.00227111913406627\\
11992.6748870712	0.00259081902417382\\
11998.5349774142	0.00288630686836549\\
12004.3950677573	0.0031506300112117\\
12010.2551581004	0.00337757653881428\\
12016.1152484434	0.00356182125941683\\
12021.9753387865	0.00369905064929623\\
12027.8354291295	0.00378606383305424\\
12033.6955194726	0.0038208472346446\\
12039.5556098157	0.00380262115837028\\
12045.4157001587	0.00373185722264122\\
12051.2757905018	0.00361026625855431\\
12057.1358808448	0.00344075698087566\\
12062.9959711879	0.00322736643019985\\
12068.8560615309	0.00297516384823449\\
12074.716151874	0.0026901302757762\\
12080.5762422171	0.00237901673159098\\
12086.4363325601	0.00204918433577047\\
12092.2964229032	0.00170843016272789\\
12098.1565132462	0.00136480294354482\\
12104.0166035893	0.00102641297382349\\
12109.8766939324	0.000701240715803483\\
12115.7367842754	0.000396948611615346\\
12121.5968746185	0.0001207005447353\\
12127.4569649615	-0.000121006796832772\\
12133.3170553046	-0.000322496686323777\\
12139.1771456477	-0.000479045661057906\\
12145.0372359907	-0.000586994232414466\\
12150.8973263338	-0.000643832408403683\\
12156.7574166768	-0.000648258030506057\\
12162.6175070199	-0.000600206569420244\\
12168.477597363	-0.000500851698535456\\
12174.337687706	-0.000352576654140911\\
12180.1977780491	-0.000158917080374117\\
12186.0578683921	7.55232708467938e-05\\
12191.9179587352	0.000345181975279903\\
12197.7780490783	0.000643669906154229\\
12203.6381394213	0.000963922607587565\\
12209.4982297644	0.00129836748561333\\
12215.3583201074	0.00163910286021765\\
12221.2184104505	0.00197808464517036\\
12227.0785007936	0.00230731624555055\\
12232.9385911366	0.00261903719041244\\
12238.7986814797	0.00290590605113968\\
12244.6587718227	0.00316117333453156\\
12250.5188621658	0.00337884027913275\\
12256.3789525089	0.00355379981892677\\
12262.2390428519	0.00368195640269436\\
12268.099133195	0.00376032185772527\\
12273.959223538	0.00378708505443428\\
12279.8193138811	0.00376165374866157\\
12285.6794042241	0.00368466763461515\\
12291.5394945672	0.00355798232289191\\
12297.3995849103	0.00338462464340911\\
12303.2596752533	0.00316872034916426\\
12309.1197655964	0.00291539594744337\\
12314.9798559394	0.00263065699378987\\
12320.8399462825	0.00232124573828104\\
12326.7000366256	0.0019944814973436\\
12332.5601269686	0.00165808753087376\\
12338.4202173117	0.00132000851873012\\
12344.2803076547	0.000988222950008418\\
12350.1403979978	0.000670554855121089\\
12356.0004883409	0.000374489322910161\\
12361.8605786839	0.000106996152381589\\
12367.720669027	-0.000125634206525346\\
12373.58075937	-0.00031793856165308\\
12379.4408497131	-0.000465409259387132\\
12385.3009400562	-0.00056459982558779\\
12391.1610303992	-0.000613205491169877\\
12397.0211207423	-0.000610116721304849\\
12402.8812110853	-0.000555444505329712\\
12408.7413014284	-0.000450516831741952\\
12414.6013917715	-0.00029784645285955\\
12420.4614821145	-0.000101070720897505\\
12426.3215724576	0.00013513506430627\\
12432.1816628006	0.00040516873201525\\
12438.0417531437	0.000702634103045424\\
12443.9018434868	0.0010204924872384\\
12449.7619338298	0.00135122922800426\\
12455.6220241729	0.00168703135342849\\
12461.4821145159	0.00201997213471457\\
12467.342204859	0.00234219819242049\\
12473.2022952021	0.00264611473427888\\
12479.0623855451	0.00292456455585278\\
12484.9224758882	0.00317099658537821\\
12490.7825662312	0.00337962000402058\\
12496.6426565743	0.00354554031603154\\
12502.5027469173	0.00366487417201057\\
12508.3628372604	0.00373484025126078\\
12514.2229276035	0.00375382407735921\\
12520.0830179465	0.00372141525634715\\
12525.9431082896	0.00363841627921462\\
12531.8031986326	0.00350682270205226\\
12537.6632889757	0.00332977519202992\\
12543.5233793188	0.00311148459138351\\
12549.3834696618	0.00285713178704507\\
12555.2435600049	0.002572744766284\\
12561.1036503479	0.00226505577614567\\
12566.963740691	0.00194134197063009\\
12572.8238310341	0.00160925331756385\\
12578.6839213771	0.00127663183456044\\
12584.5440117202	0.000951326424858319\\
12590.4041020632	0.000641007685132349\\
12596.2641924063	0.000352987054251422\\
12602.1242827494	9.40445667776325e-05\\
12607.9843730924	-0.000129730731236164\\
12613.8444634355	-0.000313083948290628\\
12619.7045537785	-0.000451717729446316\\
12625.5646441216	-0.000542392957633614\\
12631.4247344647	-0.000583004414815551\\
12637.2848248077	-0.000572629635952271\\
12643.1449151508	-0.000511549822465491\\
12649.0050054938	-0.000401242341946952\\
12654.8650958369	-0.000244345011436138\\
12660.72518618	-4.45930272548612e-05\\
12666.585276523	0.000193269950974337\\
12672.4453668661	0.000463604445762696\\
12678.3054572091	0.000760008862021156\\
12684.1655475522	0.00107547107263958\\
12690.0256378953	0.00140253430026165\\
12695.8857282383	0.00173347337133613\\
12701.7458185814	0.00206047717666717\\
12707.6059089244	0.002375833029362\\
12713.4659992675	0.00267210856891523\\
12719.3260896105	0.00294232692201863\\
12725.1861799536	0.00318013099173745\\
12731.0462702967	0.00337993300728251\\
12736.9063606397	0.00353704581610459\\
12742.7664509828	0.00364779283380925\\
12748.6265413258	0.00370959407306639\\
12754.4866316689	0.00372102623895638\\
12760.346722012	0.00368185549068407\\
12766.206812355	0.00359304211695406\\
12772.0669026981	0.00345671703384009\\
12777.9269930411	0.00327613068007132\\
12783.7870833842	0.00305557553454391\\
12789.6471737273	0.00280028410320546\\
12795.5072640703	0.0025163047989812\\
12801.3673544134	0.00221035865913803\\
12807.2274447564	0.00188968029365477\\
12813.0875350995	0.0015618468286643\\
12818.9476254426	0.00123459888875648\\
12824.8077157856	0.000915657847748411\\
12830.6678061287	0.000612543661835639\\
12836.5278964717	0.000332397582802178\\
12842.3879868148	8.18139306052683e-05\\
12848.2480771579	-0.00013331511204746\\
12854.1081675009	-0.000307938030400445\\
12859.968257844	-0.00043796257693283\\
12865.828348187	-0.000520351655020747\\
12871.6884385301	-0.000553194240633414\\
12877.5485288732	-0.000535749686200485\\
12883.4086192162	-0.000468464380026777\\
12889.2687095593	-0.000352960387325561\\
12895.1287999023	-0.000191996360520902\\
12900.9888902454	1.05983396940968e-05\\
12906.8489805884	0.000250014739631456\\
12912.7090709315	0.00052057839741305\\
12918.5691612746	0.000815883888288386\\
12924.4292516176	0.00112894644337695\\
12930.2893419607	0.00145236715229779\\
12936.1494323037	0.00177850782262507\\
12942.0095226468	0.00209967136398446\\
12947.8696129899	0.00240828343697946\\
12953.7297033329	0.00269707108066291\\
12959.589793676	0.00295923410579319\\
12965.449884019	0.00318860521565528\\
12971.3099743621	0.00337979508384935\\
12977.1700647052	0.00352831897704251\\
12983.0301550482	0.00363070194760178\\
12988.8902453913	0.0036845601292744\\
12994.7503357343	0.00368865623383579\\
13000.6104260774	0.00364292795769067\\
13006.4705164205	0.00354848864624062\\
13012.3306067635	0.00340760021986583\\
13018.1906971066	0.00322361901866571\\
13024.0507874496	0.00300091586296916\\
13029.9108777927	0.00274477223260003\\
13035.7709681358	0.00246125503114236\\
13041.6310584788	0.00215707290440395\\
13047.4911488219	0.00183941751565272\\
13053.3512391649	0.00151579353264106\\
13059.211329508	0.00119384134555545\\
13065.0714198511	0.000881156703204051\\
13070.9315101941	0.000585111525206799\\
13076.7916005372	0.000312680116960369\\
13082.6516908802	7.02748838355391e-05\\
13088.5117812233	-0.000136404585615859\\
13094.3718715664	-0.000302505515995168\\
13100.2319619094	-0.000424135870874628\\
13106.0920522525	-0.000498455533806195\\
13111.9521425955	-0.000523742608611341\\
13117.8122329386	-0.000499433291546825\\
13123.6723232817	-0.000426134392418144\\
13129.5324136247	-0.000305608227740763\\
13135.3925039678	-0.000140730260949095\\
13141.2525943108	6.45794925091613e-05\\
13147.1126846539	0.00030544965918062\\
13152.9727749969	0.000576173085354545\\
13158.83286534	0.000870342056633228\\
13164.6929556831	0.00118099995578919\\
13170.5530460261	0.0015008057697849\\
13176.4131363692	0.00182220755591233\\
13182.2732267122	0.00213762076854656\\
13188.1333170553	0.00243960723559769\\
13193.9934073984	0.00272105056194462\\
13199.8534977414	0.002975323823287\\
13205.7135880845	0.0031964455997176\\
13211.5736784275	0.00337922067474385\\
13217.4337687706	0.00351936209084715\\
13223.2938591137	0.00361359169415607\\
13229.1539494567	0.00365971681014061\\
13235.0140397998	0.00365668125752385\\
13240.8741301428	0.00360458951335667\\
13246.7342204859	0.00350470347765947\\
13252.594310829	0.00335941193135101\\
13258.454401172	0.00317217342681173\\
13264.3144915151	0.00294743397577846\\
13270.1745818581	0.002690521493395\\
13276.0346722012	0.00240751950441071\\
13281.8947625443	0.00210512310452079\\
13287.7548528873	0.00179048058761845\\
13293.6149432304	0.00147102448452132\\
13299.4750335734	0.00115429600674359\\
13305.3351239165	0.00084776704176819\\
13311.1952142596	0.000558663901163979\\
13317.0553046026	0.000293796978764833\\
13322.9153949457	5.94003341502795e-05\\
13328.7754852887	-0.000139015020327091\\
13334.6355756318	-0.000296790676461904\\
13340.4956659748	-0.000410230185557357\\
13346.3557563179	-0.000476685645360349\\
13352.215846661	-0.000494619490605733\\
13358.075937004	-0.000463640045495784\\
13363.9360273471	-0.000384510016431121\\
13369.7961176901	-0.000259127742386496\\
13375.6562080332	-9.04816629876709e-05\\
13381.5162983763	0.000117420903896955\\
13387.3763887193	0.000359648977902956\\
13393.2364790624	0.000630464863146462\\
13399.0965694054	0.000923460053034822\\
13404.9566597485	0.00123170687408411\\
13410.8167500916	0.00154792228059124\\
13416.6768404346	0.0018646399278886\\
13422.5369307777	0.00217438645866533\\
13428.3970211207	0.00246985784069783\\
13434.2571114638	0.00274409159504284\\
13440.1172018069	0.00299063085426809\\
13445.9772921499	0.00320367638486224\\
13451.837382493	0.00337822299508318\\
13457.697472836	0.00351017711966958\\
13463.5575631791	0.00359645281948817\\
13469.4176535222	0.00363504394500396\\
13475.2777438652	0.00362507077624197\\
13481.1378342083	0.00356680005502438\\
13486.9979245513	0.0034616379546464\\
13492.8580148944	0.00331209616943637\\
13498.7181052375	0.00312173194208246\\
13504.5781955805	0.00289506346023967\\
13510.4382859236	0.00263746263442819\\
13516.2983762666	0.00235502780130478\\
13522.1584666097	0.00205443936895238\\
13528.0185569528	0.00174280182061814\\
13533.8786472958	0.00142747581368767\\
13539.7387376389	0.00111590434142437\\
13545.5988279819	0.000815437062982916\\
13551.458918325	0.000533156947610412\\
13557.3190086681	0.000275713321537797\\
13563.1790990111	4.9165252847688e-05\\
13569.0391893542	-0.000141161037301188\\
13574.8992796972	-0.00029079738163158\\
13580.7593700403	-0.000396238548846174\\
13586.6194603833	-0.00045502433916357\\
13592.4795507264	-0.000465796970757441\\
13598.3396410695	-0.000428332418795748\\
13604.1997314125	-0.000343544984132847\\
13610.0598217556	-0.000213465000840088\\
13615.9199120986	-4.11902254466342e-05\\
13621.7800024417	0.000169187928476629\\
13627.6400927848	0.000412681554150447\\
13633.5001831278	0.000683524506567394\\
13639.3602734709	0.000975308945195455\\
13645.2203638139	0.00128113693133899\\
13651.080454157	0.00159378349377976\\
13656.9405445001	0.00190586730827287\\
13662.8006348431	0.00221002495888296\\
13668.6607251862	0.002499084667487\\
13674.5208155292	0.00276623539278202\\
13680.3809058723	0.00300518731189572\\
13686.2409962154	0.00321031990360969\\
13692.1010865584	0.00337681414709833\\
13697.9611769015	0.0035007657267381\\
13703.8212672445	0.00357927658441986\\
13709.6813575876	0.00361052267204781\\
13715.5414479307	0.00359379632114949\\
13721.4015382737	0.00352952224543625\\
13727.2616286168	0.0034192468150607\\
13733.1217189598	0.00326560087258786\\
13738.9818093029	0.00307223698416493\\
13744.841899646	0.00284374262236742\\
13750.701989989	0.00258553134405858\\
13756.5620803321	0.00230371454466459\\
13762.4221706751	0.00200495682623152\\
13768.2822610182	0.0016963184025553\\
13774.1423513612	0.00138508826722323\\
13780.0024417043	0.0010786120667953\\
13785.8625320474	0.00078411874361199\\
13791.7226223904	0.000508550039020138\\
13797.5827127335	0.000258396878264007\\
13803.4428030765	3.9546492979655e-05\\
13809.3028934196	-0.000142856118191634\\
13815.1629837627	-0.000284529130962334\\
13821.0230741057	-0.000382154396046635\\
13826.8831644488	-0.000433455140392547\\
13832.7432547918	-0.000437249049686191\\
13838.6033451349	-0.000393475494789959\\
13844.463435478	-0.000303196275538598\\
13850.323525821	-0.000168569880621201\\
13856.1836161641	7.20011301820675e-06\\
13862.0437065071	0.000219941267876601\\
13867.9037968502	0.000464611327181464\\
13873.7638871933	0.000735417718974825\\
13879.6239775363	0.00102595469092932\\
13885.4840678794	0.00132935482955634\\
13891.3441582224	0.00163845137996679\\
13897.2042485655	0.00194594752989359\\
13903.0643389086	0.00224458865951919\\
13908.9244292516	0.00252733349123196\\
13914.7845195947	0.00278752010166614\\
13920.6446099377	0.00301902288247949\\
13926.5047002808	0.00321639675151773\\
13932.3647906239	0.00337500522006096\\
13938.2248809669	0.00349112930359026\\
13944.08497131	0.00356205471872728\\
13949.945061653	0.00358613532134955\\
13955.8051519961	0.00356283130393929\\
13961.6652423392	0.00349272126603635\\
13967.5253326822	0.00337748788853582\\
13973.3854230253	0.00321987756595162\\
13979.2455133683	0.00302363496483255\\
13985.1056037114	0.0027934140670323\\
13990.9656940544	0.00253466781147139\\
13996.8257843975	0.0022535189504021\\
14002.6858747406	0.00195661517922525\\
14008.5459650836	0.00165097196672921\\
14014.4060554267	0.00134380680201195\\
14020.2661457697	0.00104236877370208\\
14026.1262361128	0.000753767505555156\\
14031.9863264559	0.000484805484690877\\
14037.8464167989	0.000241817736857821\\
14043.706507142	3.05226269723719e-05\\
14049.566597485	-0.000144112701387935\\
14055.4266878281	-0.000277989081314912\\
14061.2867781712	-0.000367971528561726\\
14067.1468685142	-0.000411962640454991\\
14073.0069588573	-0.000408951469478573\\
14078.8670492003	-0.000359036732912303\\
14084.7271395434	-0.000263423826120395\\
14090.5872298865	-0.000124395725499868\\
14096.4473202295	5.47415169383387e-05\\
14102.3074105726	0.000269737386289097\\
14108.1675009156	0.000515497755686738\\
14114.0275912587	0.000786205582809815\\
14119.8876816018	0.00107545859238555\\
14125.7477719448	0.0013764206862526\\
14131.6078622879	0.00168198350024101\\
14137.4679526309	0.00198493429025324\\
14143.328042974	0.00227812618235113\\
14149.1881333171	0.00255464676876176\\
14155.0482236601	0.00280798107254935\\
14160.9083140032	0.00303216503936423\\
14166.7684043462	0.00322192594002392\\
14172.6284946893	0.00337280637915283\\
14178.4885850324	0.00348126899325907\\
14184.3486753754	0.00354477937927039\\
14190.2087657185	0.00356186530974095\\
14196.0688560615	0.00353215085120681\\
14201.9289464046	0.00345636459554653\\
14207.7890367477	0.00333632182494585\\
14213.6491270907	0.00317488104669863\\
14219.5092174338	0.00297587593821246\\
14225.3693077768	0.00274402432222415\\
14231.2293981199	0.00248481633338656\\
14237.0894884629	0.0022043844268812\\
14242.949578806	0.00190935830668996\\
14248.8096691491	0.00160670820536757\\
14254.6697594921	0.00130358021957292\\
14260.5298498352	0.00100712759141242\\
14266.3899401782	0.000724341918675397\\
14272.2500305213	0.000461888276574824\\
14278.1101208644	0.000225948139063632\\
14283.9702112074	2.20738013458808e-05\\
14289.8303015505	-0.000144942268043085\\
14295.6903918935	-0.000271180071694326\\
14301.5504822366	-0.000353684076786514\\
14307.4105725797	-0.000390532398900328\\
14313.2706629227	-0.000380881556884313\\
14319.1307532658	-0.000324985756845948\\
14324.9908436088	-0.000224190264832648\\
14330.8509339519	-8.08990394676799e-05\\
14336.711024295	0.000101482597034424\\
14342.571114638	0.000318628882536111\\
14348.4312049811	0.00056539621042337\\
14354.2912953241	0.000835944963785444\\
14360.1513856672	0.00112387770259079\\
14366.0114760103	0.00142239043416452\\
14371.8715663533	0.00172443339015354\\
14377.7316566964	0.00202287751120831\\
14383.5917470394	0.00231068270777798\\
14389.4518373825	0.00258106392615789\\
14395.3119277256	0.00282765110239682\\
14401.1720180686	0.00304463923383036\\
14407.0321084117	0.00322692503256452\\
14412.8921987547	0.00337022694442108\\
14418.7522890978	0.0034711857108244\\
14424.6123794408	0.00352744311202012\\
14430.4724697839	0.0035376970457357\\
14436.332560127	0.00350173165535322\\
14442.19265047	0.00342042181080717\\
14448.0527408131	0.00329571185067635\\
14453.9128311561	0.00313056910176388\\
14459.7729214992	0.00292891328709512\\
14465.6330118423	0.00269552350139673\\
14471.4931021853	0.00243592496150328\\
14477.3531925284	0.00215625821548399\\
14483.2132828714	0.0018631339060022\\
14489.0733732145	0.0015634765229513\\
14494.9334635576	0.00126436083839635\\
14500.7935539006	0.000972844886872625\\
14506.6536442437	0.000695803434123047\\
14512.5137345867	0.00043976586314819\\
14518.3738249298	0.000210762300133952\\
14524.2339152729	1.4181606485196e-05\\
14530.0940056159	-0.000145355419131996\\
14535.954095959	-0.000264104645316144\\
14541.814186302	-0.000339286466673613\\
14547.6742766451	-0.000369150855255968\\
14553.5343669882	-0.000353018082501412\\
14559.3944573312	-0.000291294164323109\\
14565.2545476743	-0.000185460678968508\\
14571.1146380173	-3.80392120851743e-05\\
14576.9747283604	0.000147468718030301\\
14582.8348187035	0.000366664823968551\\
14588.6949090465	0.000614358326495787\\
14594.5549993896	0.00088468887352642\\
14600.4150897326	0.00117126519018534\\
14606.2751800757	0.00146731617983447\\
14612.1352704188	0.00176585090394756\\
14617.9953607618	0.00205982366122552\\
14623.8554511049	0.002342300268085\\
14629.7155414479	0.00260662161640171\\
14635.575631791	0.00284656065096442\\
14641.4357221341	0.00305646906617973\\
14647.2958124771	0.00323141026656279\\
14653.1559028202	0.00336727546096407\\
14659.0159931632	0.00346088016106414\\
14664.8760835063	0.00351003881731542\\
14670.7361738493	0.00351361584370634\\
14676.5962641924	0.00347155184018167\\
14682.4563545355	0.00338486440746752\\
14688.3164448785	0.00325562354887474\\
14694.1765352216	0.00308690225354372\\
14700.0366255646	0.00288270344049986\\
14705.8967159077	0.00264786499999256\\
14711.7568062508	0.00238794518504862\\
14717.6168965938	0.00210909106740961\\
14723.4769869369	0.00181789317173546\\
14729.3370772799	0.00152122972457019\\
14735.197167623	0.00122610419944735\\
14741.0572579661	0.00093947999445889\\
14746.9173483091	0.000668116144800087\\
14752.7774386522	0.000418407946166413\\
14758.6375289952	0.000196236246832203\\
14764.4976193383	6.82895969455063e-06\\
14770.3577096814	-0.000145361944579666\\
14776.2178000244	-0.000256765069308663\\
14782.0778903675	-0.000324773389527042\\
14787.9379807105	-0.000347805249640168\\
14793.7980710536	-0.000325341134040087\\
14799.6581613967	-0.000257935356010565\\
14805.5182517397	-0.000147202402636761\\
14811.3783420828	4.22172853864268e-06\\
14817.2384324258	0.000192742275460329\\
14823.0985227689	0.000413891046680078\\
14828.958613112	0.000662432320091626\\
14834.818703455	0.00093248679559564\\
14840.6787937981	0.00121767066731926\\
14846.5388841411	0.00151124652591381\\
14852.3989744842	0.0018062825240459\\
14858.2590648272	0.00209581604510282\\
14864.1191551703	0.00237301801082623\\
14869.9792455134	0.00263135395085735\\
14875.8393358564	0.00286473803519307\\
14881.6994261995	0.00306767643896439\\
14887.5595165425	0.00323539666232488\\
14893.4196068856	0.00336395976184966\\
14899.2796972287	0.00345035285387688\\
14905.1397875717	0.00349255971797196\\
14910.9998779148	0.00348960784607493\\
14916.8599682578	0.00344159083945087\\
14922.7200586009	0.00334966563817136\\
14928.580148944	0.00321602466163508\\
14934.440239287	0.00304384353095849\\
14940.3003296301	0.00283720561916546\\
14946.1604199731	0.00260100522197361\\
14952.0205103162	0.00234083164499458\\
14957.8806006593	0.00206283695243711\\
14963.7406910023	0.00177359050624812\\
14969.6007813454	0.0014799237351317\\
14975.4608716884	0.00118876880093558\\
14981.3209620315	0.000906994972593629\\
14987.1810523746	0.00064124656956867\\
14993.0411427176	0.000397786297766314\\
14998.9012330607	0.000182347671641106\\
nan	nan\\
15010.6214137468	-0.000144970885391611\\
15016.4815040899	-0.000249163352292922\\
15022.3415944329	-0.0003101397746323\\
15028.201684776	-0.000326483551076439\\
15034.061775119	-0.000297832002020913\\
15039.9218654621	-0.000224884381242437\\
15045.7819558052	-0.000109384826114604\\
15051.6420461482	4.59193383597596e-05\\
15057.5021364913	0.000237342945100516\\
15063.3622268343	0.000460350426003838\\
15069.2223171774	0.000709663273851925\\
15075.0824075204	0.000979384979148323\\
15080.9424978635	0.00126314048489069\\
15086.8025882066	0.00155422686134492\\
15092.6626785496	0.00184577163966613\\
15098.5227688927	0.00213089506466031\\
15104.3828592357	0.00240287243599037\\
15110.2429495788	0.00265529270753584\\
15116.1030399219	0.00288220960418514\\
15121.9631302649	0.00307828169475268\\
15127.823220608	0.00323889812103693\\
15133.683310951	0.00336028702397862\\
15139.5434012941	0.00343960411759045\\
15145.4034916372	0.00347499933022859\\
15151.2635819802	0.00346565995275887\\
15157.1236723233	0.00341182928680324\\
15162.9837626663	0.00331480036596573\\
15168.8438530094	0.00317688491076186\\
15174.7039433525	0.00300135826212091\\
15180.5640336955	0.00279238160564078\\
15186.4241240386	0.00255490333257139\\
15192.2842143816	0.00229454187596013\\
15198.1443047247	0.00201745279600168\\
15204.0043950678	0.00173018325839858\\
15209.8644854108	0.00143951734607849\\
15215.7245757539	0.00115231585894787\\
15221.5846660969	0.000875354384118947\\
15227.44475644	0.000615163458716024\\
15233.3048467831	0.000377874595476299\\
15239.1649371261	0.000169075801332674\\
15245.0250274692	-6.32000663148055e-06\\
15250.8851178122	-0.000144190589545165\\
15256.7452081553	-0.000241301260081821\\
15262.6052984984	-0.00029538076439031\\
15268.4653888414	-0.000305174392639081\\
15274.3254791845	-0.000270473076442667\\
15280.1855695275	-0.000192117798626628\\
15286.0456598706	-7.19792236303063e-05\\
15291.9057502136	8.7086958430274e-05\\
15297.7658405567	0.000281307908173864\\
15303.6259308998	0.000506083120695707\\
15309.4860212428	0.000756093394389925\\
15315.3461115859	0.00102542670394675\\
15321.2062019289	0.00130771799906446\\
15327.066292272	0.00159629962321549\\
15332.9263826151	0.00188435879806841\\
15338.7864729581	0.00216509845392739\\
15344.6465633012	0.00243189760977451\\
15350.5066536442	0.00267846751867429\\
15356.3667439873	0.00289899989653999\\
15362.2268343304	0.00308830374023165\\
15368.0869246734	0.00324192751253394\\
15373.9470150165	0.00335626381843553\\
15379.8071053595	0.00342863411070367\\
15385.6671957026	0.00345735143676508\\
15391.5272860457	0.00344175975666725\\
15397.3873763887	0.00338224891582324\\
15403.2474667318	0.00328024493180674\\
15409.1075570748	0.00313817583574399\\
15414.9676474179	0.00295941388710672\\
15420.827737761	0.0027481955362655\\
15426.687828104	0.0025095210349352\\
15432.5479184471	0.00224903607266765\\
15438.4080087901	0.00197289824158901\\
15444.2680991332	0.00168763148727556\\
15450.1281894763	0.00139997198631395\\
15455.9882798193	0.00111670909117421\\
15461.8483701624	0.000844525097901928\\
15467.7084605054	0.000589837618094773\\
15473.5685508485	0.000358648273218747\\
15479.4286411916	0.000156401278303213\\
15485.2887315346	-1.21447528169376e-05\\
15491.1488218777	-0.000143028762349835\\
15497.0089122207	-0.000233180329688468\\
15502.8690025638	-0.000280491691646587\\
15508.7290929068	-0.000283867012633818\\
15514.5891832499	-0.000243247753188978\\
15520.449273593	-0.000159613549862391\\
15526.309363936	-3.49585975311358e-05\\
15532.1694542791	0.000127755896776441\\
15538.0295446221	0.000324672055058073\\
15543.8896349652	0.000551126793784171\\
15549.7497253083	0.000801762245226182\\
15555.6098156513	0.00107065252003196\\
15561.4699059944	0.00135144381209565\\
15567.3299963374	0.00163750453345362\\
15573.1900866805	0.00192208193174272\\
15579.0501770236	0.00219846149160917\\
15584.9102673666	0.00246012535751445\\
15590.7703577097	0.00270090604026635\\
15596.6304480527	0.00291513178258332\\
15602.4905383958	0.0030977601577299\\
15608.3506287389	0.0032444967545569\\
15614.2107190819	0.00335189615523155\\
15620.070809425	0.00341744283179781\\
15625.930899768	0.00343961006209267\\
15631.7909901111	0.00341789548531318\\
15637.6510804542	0.00335283246943439\\
15643.5111707972	0.00324597703373892\\
15649.3712611403	0.00309987064662115\\
15655.2313514833	0.00291797978783883\\
15661.0914418264	0.00270461371255668\\
15666.9515321695	0.0024648223675902\\
15672.8116225125	0.00220427687848571\\
15678.6717128556	0.00192913543522503\\
15684.5318031986	0.00164589774830985\\
15690.3918935417	0.00136125151483173\\
15696.2519838848	0.00108191452100489\\
15702.1120742278	0.000814476109168386\\
15707.9721645709	0.000565241750036063\\
15713.8322549139	0.000340084386486619\\
15719.692345257	0.000144306053242435\\
15725.5524356	-1.74869670384477e-05\\
15731.4125259431	-0.000141492511864545\\
15737.2726162862	-0.000224801881802328\\
15743.1327066292	-0.000265468058943033\\
15748.9927969723	-0.000262551201128199\\
15754.8528873153	-0.00021614034906893\\
15760.7129776584	-0.000127350845271362\\
15766.5730680015	1.70246302675327e-06\\
15772.4331583445	0.000167955593274787\\
15778.2932486876	0.000367468169909713\\
15784.1533390306	0.000595516812678381\\
15790.0134293737	0.000846706957782417\\
15795.8735197168	0.00111510046472146\\
15801.7336100598	0.00139435599053467\\
15807.5937004029	0.00167787881306794\\
15813.4537907459	0.00195897656392919\\
15819.313881089	0.00223101719329131\\
15825.1739714321	0.00248758543855506\\
15831.0340617751	0.00272263410518938\\
15836.8941521182	0.00293062659267823\\
15842.7542424612	0.00310666730637738\\
15848.6143328043	0.00324661688391024\\
15854.4744231474	0.00334718952380148\\
15860.3345134904	0.00340603012836392\\
15866.1946038335	0.0034217694496319\\
15872.0546941765	0.00339405594718373\\
15877.9147845196	0.00332356361703518\\
15883.7748748627	0.00321197561753751\\
15889.6349652057	0.0030619440901945\\
15895.4950555488	0.00287702713371492\\
15901.3551458918	0.00266160442981069\\
15907.2152362349	0.00242077352038266\\
15913.0753265779	0.00216022919334625\\
15918.935416921	0.00188612883015765\\
15924.7955072641	0.0016049468990927\\
15930.6555976071	0.00132332203250974\\
15936.5156879502	0.00104790029991056\\
15942.3757782932	0.000785178376656384\\
15948.2358686363	0.000541350309089475\\
15954.0959589794	0.000322161490170081\\
15959.9560493224	0.000132773287949975\\
15965.8161396655	-2.23584835662776e-05\\
15971.6762300085	-0.000139588389904397\\
15977.5363203516	-0.000216167031927556\\
15983.3964106947	-0.000250305519499976\\
15989.2565010377	-0.000241217251211042\\
15995.1165913808	-0.000189136024533497\\
16000.9766817238	-9.53100597599009e-05\\
16006.8367720669	3.80279104543498e-05\\
16012.69686241	0.000207713769378205\\
16018.556952753	0.000409727098285845\\
16024.4170430961	0.000639286430811113\\
16030.2771334391	0.000890962422844709\\
16036.1372237822	0.00115880625959995\\
16041.9973141253	0.00143649026307191\\
16047.8574044683	0.00171745737655641\\
16053.7174948114	0.00199507599508508\\
16059.5775851544	0.00226279648579603\\
16065.4376754975	0.00251430570430218\\
16071.2977658406	0.00274367586196693\\
16077.1578561836	0.00294550423353621\\
16083.0179465267	0.00311504041315398\\
16088.8780368697	0.00324829812088709\\
16094.7381272128	0.00334214892908698\\
16100.5982175559	0.00339439570411013\\
16106.4583078989	0.0034038240405048\\
16112.318398242	0.00337023048262889\\
16118.178488585	0.00329442687901623\\
16124.0385789281	0.0031782207765661\\
16129.8986692712	0.00302437232816796\\
16135.7587596142	0.00283652874096806\\
16141.6188499573	0.00261913782143336\\
16147.4789403003	0.00237734266739566\\
16153.3390306434	0.00211685999924417\\
16159.1991209864	0.00184384500913134\\
16165.0592113295	0.00156474592294067\\
16170.9193016726	0.00128615171126053\\
16176.7793920156	0.0010146365460701\\
16182.6394823587	0.000756604674649038\\
16188.4995727017	0.000518139371084741\\
16194.3596630448	0.000304859527714513\\
16200.2197533879	0.000121787267190864\\
16206.0798437309	-2.67703058012217e-05\\
16211.939934074	-0.000137322429085251\\
16217.800024417	-0.000207276700249463\\
16223.6601147601	-0.000234999859707407\\
16229.5202051032	-0.000219855914514827\\
16235.3802954462	-0.000162220713228812\\
16241.2403857893	-6.34726380712965e-05\\
16247.1004761323	7.40403555731246e-05\\
16252.9605664754	0.000247056564347153\\
16258.8206568185	0.000451477899609607\\
16264.6807471615	0.000682466952817851\\
16270.5408375046	0.000934561464678312\\
16276.4009278476	0.00120180348947253\\
16282.2610181907	0.00147788020042007\\
16288.1211085338	0.00175627300844687\\
16293.9811988768	0.0020304114721701\\
16299.8412892199	0.00229382836536761\\
16305.7013795629	0.00254031224206561\\
16311.561469906	0.0027640539006836\\
16317.4215602491	0.00295978329357303\\
16323.2816505921	0.00312289365547601\\
16329.1417409352	0.00324954992734402\\
16335.0018312782	0.00333677892398197\\
16340.8619216213	0.00338253912540536\\
16346.7220119643	0.00338576845392555\\
16352.5821023074	0.0033464089189049\\
16358.4421926505	0.00326540755775076\\
16364.3022829935	0.00314469366049456\\
16370.1623733366	0.00298713282620357\\
16376.0224636796	0.00279645894443526\\
16381.8825540227	0.00257718571678646\\
16387.7426443658	0.00233449981442621\\
16393.6027347088	0.00207413820114466\\
16399.4628250519	0.00180225252264397\\
16405.3229153949	0.0015252637683092\\
16411.183005738	0.00124971063835898\\
16417.0430960811	0.000982095197569767\\
16422.9031864241	0.000728729458506226\\
16428.7632767672	0.000495586514094176\\
16434.6233671102	0.000288159730377807\\
16440.4834574533	0.000111333318655286\\
16446.3435477964	-3.07326637172032e-05\\
16452.2036381394	-0.000134700176321897\\
16458.0637284825	-0.000198131620406222\\
16463.9238188255	-0.000219546982932227\\
16469.7839091686	-0.000198458360456623\\
16475.6439995117	-0.000135381057646192\\
16481.5040898547	-3.18210083328037e-05\\
16487.3641801978	0.000109761174163993\\
16493.2242705408	0.000286008659417088\\
16499.0843608839	0.000492747986061979\\
16504.944451227	0.000725087884969616\\
16510.80454157	0.000977534999517035\\
16516.6646319131	0.0012441237653532\\
16522.5247222561	0.0015185573789269\\
16528.3848125992	0.00179435652390311\\
16534.2449029423	0.00206501234264672\\
16540.1049932853	0.00232414004163137\\
16545.9650836284	0.00256562950545911\\
16551.8251739714	0.00278378936732268\\
16557.6852643145	0.00297348113850494\\
16563.5453546575	0.00313024023616807\\
16569.4054450006	0.00325038105919412\\
16575.2655353437	0.00333108363872912\\
16581.1256256867	0.00337045982670993\\
16586.9857160298	0.00336759746885296\\
16592.8458063728	0.00332258152877746\\
16598.7058967159	0.00323649167421322\\
16604.565987059	0.00311137639197449\\
16610.426077402	0.00295020425211911\\
16616.2861677451	0.00275679348066917\\
16622.1462580881	0.00253572151148781\\
16628.0063484312	0.00229221665998899\\
16633.8664387743	0.00203203448196694\\
16639.7265291173	0.0017613217414424\\
16645.5866194604	0.00148647120241221\\
16651.4467098034	0.00121397067475159\\
16657.3068001465	0.000950249878674834\\
16663.1668904896	0.000701528742090925\\
16669.0269808326	0.000473670710044766\\
16674.8870711757	0.000272044525574439\\
16680.7471615187	0.000101397740207191\\
16686.6072518618	-3.42550660060854e-05\\
16692.4673422049	-0.000131726723126663\\
16698.3274325479	-0.000188732347219367\\
16704.187522891	-0.00020394289452982\\
16710.047613234	-0.000177016138859087\\
16715.9077035771	-0.000108604350224984\\
16721.7677939202	-3.38503015062493e-07\\
16727.6278842632	0.000145210604301558\\
16733.4879746063	0.000324593391174183\\
16739.3480649493	0.00053356324939121\\
16745.2081552924	0.000767177072449169\\
16751.0682456355	0.00101991218018234\\
16756.9283359785	0.00128579687304424\\
16762.7884263216	0.00155855152981105\\
16768.6485166646	0.00183173691520315\\
16774.5086070077	0.00209890619477049\\
16780.3686973508	0.00235375706848653\\
16786.2287876938	0.00259028043329555\\
16792.0888780369	0.00280290206770192\\
16797.9489683799	0.00298661399816999\\
16803.809058723	0.00313709245131778\\
16809.669149066	0.00325079961408492\\
16815.5292394091	0.00332506680697236\\
16821.3893297522	0.00335815711499183\\
16827.2494200952	0.00334930600689151\\
16833.1095104383	0.00329873899231712\\
16838.9696007813	0.00320766590989548\\
16844.8296911244	0.00307825198979473\\
16850.6897814675	0.00291356638295905\\
16856.5498718105	0.0027175093805363\\
16862.4099621536	0.00249472004890933\\
16868.2700524966	0.00225046646811791\\
16874.1301428397	0.00199052117006053\\
16879.9902331828	0.00172102472175951\\
16885.8503235258	0.0014483406774797\\
16891.7104138689	0.0011789053255574\\
16897.5705042119	0.000919075777729024\\
16903.430594555	0.000674979986037294\\
16909.2906848981	0.000452372225917957\\
16915.1507752411	0.000256497453378197\\
16921.0108655842	9.19677336562851e-05\\
16926.8709559272	-3.73463474530113e-05\\
16932.7310462703	-0.000128406733021716\\
16938.5911366134	-0.000179079263506846\\
16944.4512269564	-0.000188183687877569\\
16950.3113172995	-0.000155521145550106\\
16956.1714076425	-8.18784793205741e-05\\
16962.0314979856	3.09907134717422e-05\\
16967.8915883287	0.000180407835448466\\
16973.7516786717	0.000362832855259908\\
16979.6117690148	0.000573948176819946\\
16985.4718593578	0.000808760824820732\\
16991.3319497009	0.0010617205281706\\
16997.1920400439	0.00132685090883612\\
17003.052130387	0.00159789067534705\\
17008.9122207301	0.00186844148536767\\
17014.7723110731	0.0021321189855739\\
17020.6324014162	0.00238270346375212\\
17026.4924917592	0.00261428655787328\\
17032.3525821023	0.00282141056221078\\
17038.2126724454	0.00299919704571256\\
17044.0727627884	0.00314346175192286\\
17049.9328531315	0.00325081307430903\\
17055.7929434745	0.00331873178943838\\
17061.6530338176	0.00334563017360636\\
17067.5131241607	0.0033308891162053\\
17073.3732145037	0.0032748723616779\\
17079.2333048468	0.00317891755331364\\
17085.0933951898	0.00304530429881759\\
17090.9534855329	0.00287720001974668\\
17096.813575876	0.00267858487124884\\
17102.673666219	0.00245415751154037\\
17108.5337565621	0.00220922395216201\\
17114.3938469051	0.00194957211792032\\
17120.2539372482	0.00168133508208871\\
17126.1140275913	0.00141084620853153\\
17131.9741179343	0.00114448962183809\\
17137.8342082774	0.000888549535702041\\
17143.6942986204	0.000649061995683265\\
17149.5543889635	0.000431672533579973\\
17155.4144793066	0.000241503090366643\\
17161.2745696496	8.30313444355445e-05\\
17167.1346599927	-4.00147120079053e-05\\
17172.9947503357	-0.000124744466343105\\
17178.8548406788	-0.000169172586033585\\
17184.7149310219	-0.00017226553134028\\
17190.5750213649	-0.000133965590637644\\
17196.435111708	-5.51918795326886e-05\\
17202.295202051	6.21817112289388e-05\\
17208.1552923941	0.000215371090139906\\
17214.0153827372	0.000400748001395787\\
17219.8754730802	0.000613925957206676\\
17225.7355634233	0.000849864030840129\\
17231.5956537663	0.00110298605450711\\
17237.4557441094	0.00136731240367987\\
17243.3158344524	0.00163660125346903\\
17249.1759247955	0.00190449597017936\\
17255.0360151386	0.00216467515785807\\
17260.8961054816	0.00241100181814975\\
17266.7561958247	0.00263766810384271\\
17272.6162861677	0.00283933225223779\\
17278.4763765108	0.00301124446952437\\
17284.3364668539	0.00314935880011472\\
17290.1965571969	0.00325042834598547\\
17296.05664754	0.00331208159513595\\
17301.916737883	0.0033328780652403\\
17307.7768282261	0.00331234195655322\\
17313.6369185692	0.00325097302838986\\
17319.4970089122	0.00315023445060586\\
17325.3570992553	0.00301251792512637\\
17331.2171895983	0.0028410869089278\\
17337.0772799414	0.00263999928608769\\
17342.9373702845	0.00241401132129086\\
17348.7974606275	0.00216846516790967\\
17354.6575509706	0.00190916259096835\\
17360.5176413136	0.00164222789021971\\
17366.3777316567	0.00137396326148658\\
17372.2378219998	0.00111070001227798\\
17378.0979123428	0.000858649144152956\\
17383.9580026859	0.000623754827763278\\
17389.8180930289	0.000411554227368581\\
17395.678183372	0.000227046980089\\
17401.5382737151	7.45774065927705e-05\\
17407.3983640581	-4.22677719732014e-05\\
17413.2584544012	-0.00012074380267877\\
17419.1185447442	-0.000159012370673364\\
17424.9786350873	-0.000156184656030153\\
17430.8387254303	-0.00011234196916861\\
17436.6988157734	-2.85334859471857e-05\\
17442.5589061165	9.32488559845947e-05\\
17448.4189964595	0.000250117699026392\\
17454.2790868026	0.00043835872060944\\
17460.1391771456	0.000653518578438651\\
17465.9992674887	0.000890510263788304\\
17471.8593578318	0.00114373337056256\\
17477.7194481748	0.00140720643686414\\
17483.5795385179	0.00167470823173893\\
17489.4396288609	0.0019399246500727\\
17495.299719204	0.0021965977470681\\
17501.1598095471	0.00243867339507389\\
17507.0198998901	0.0026604440784261\\
17512.8799902332	0.00285668345889799\\
17518.7400805762	0.00302276953889241\\
17524.6001709193	0.00315479352014854\\
17530.4602612624	0.00324965179432829\\
17536.3203516054	0.00330511890044406\\
17542.1804419485	0.00331989973423141\\
17548.0405322915	0.00329365978488761\\
17553.9006226346	0.00322703269307582\\
17559.7607129777	0.00312160496001904\\
17565.6208033207	0.00297987817644177\\
17571.4808936638	0.0028052096702421\\
17577.3409840068	0.00260173298152088\\
17583.2010743499	0.00237426004794859\\
17589.061164693	0.00212816741562344\\
17594.921255036	0.00186926916557996\\
17600.7813453791	0.0016036795596945\\
17606.6414357221	0.00133766865039903\\
17612.5015260652	0.00107751426396015\\
17618.3616164083	0.000829353851739215\\
17624.2217067513	0.00059903970494476\\
17630.0817970944	0.000392000948702619\\
17635.9418874374	0.000213115569519962\\
17641.8019777805	6.65954925906442e-05\\
17647.6620681235	-4.41125836659883e-05\\
17653.5221584666	-0.000116408261157179\\
17659.3822488097	-0.000148598516845465\\
17665.2423391527	-0.000139937344294651\\
17671.1024294958	-9.06430339274512e-05\\
17676.9625198388	-1.89269188114981e-06\\
17682.8226101819	0.000124205865406549\\
17688.682700525	0.000284664169936385\\
17694.542790868	0.000475683925448585\\
17700.4028812111	0.000692746916920503\\
17706.2629715541	0.00093072187814373\\
17712.1230618972	0.00118398578974072\\
17717.9831522403	0.00144655674036633\\
17723.8432425833	0.00171223521203553\\
17729.7033329264	0.00197475045254338\\
17735.5634232694	0.00222790847940802\\
17741.4235136125	0.00246573822198747\\
17747.2836039556	0.00268263235409453\\
17753.1436942986	0.00287347949522077\\
17759.0037846417	0.00303378466396183\\
17764.8638749847	0.0031597751448728\\
17770.7239653278	0.0032484892758537\\
17776.5840556709	0.00329784606602108\\
17782.4441460139	0.00330669400845668\\
17788.304236357	0.00327483794213662\\
17794.1643267	0.00320304333721315\\
17800.0244170431	0.00309301790968213\\
17805.8845073862	0.0029473710069704\\
17811.7445977292	0.00276955173020013\\
17817.6046880723	0.00256376726082426\\
17823.4647784153	0.00233488332481833\\
17829.3248687584	0.00208830914989589\\
17835.1849591015	0.0018298696352209\\
17841.0450494445	0.00156566775460915\\
17846.9051397876	0.00130194044328263\\
17852.7652301306	0.00104491137126863\\
17858.6253204737	0.000800644078546268\\
17864.4854108167	0.000574898937514028\\
17870.3455011598	0.000372997317038402\\
17876.2055915029	0.000199696150929326\\
17882.0656818459	5.90758674657438e-05\\
17887.925772189	-4.55556798860067e-05\\
17893.785862532	-0.000111741018777535\\
17899.6459528751	-0.000137930771273336\\
17905.5060432182	-0.000123519918815147\\
17911.3661335612	-6.88617701286397e-05\\
17917.2262239043	2.47406902264884e-05\\
17923.0863142473	0.000155065861107603\\
17928.9464045904	0.000319026251571868\\
17934.8064949335	0.000512741623865245\\
17940.6665852765	0.000731630819955942\\
17946.5266756196	0.000970520098580096\\
17952.3867659626	0.00122376542092549\\
17958.2468563057	0.00148538579471838\\
17964.1069466488	0.00174920452650565\\
17969.9670369918	0.00200899504622712\\
17975.8271273349	0.00225862786163019\\
17981.6872176779	0.0024922151740684\\
17987.547308021	0.00270424974416834\\
17993.4073983641	0.00288973473190572\\
17999.2674887071	0.00304430145036805\\
18005.1275790502	0.00316431225816866\\
18010.9876693932	0.00324694616736504\\
18016.8477597363	0.00329026515224825\\
18022.7078500794	0.00329325960037216\\
18028.5679404224	0.00325587184018219\\
18034.4280307655	0.00317899719677166\\
18040.2881211085	0.00306446255839037\\
18046.1482114516	0.00291498296643236\\
18052.0083017947	0.00273409726046745\\
18057.8683921377	0.00252608430355498\\
18063.7284824808	0.00229586177103383\\
18069.5885728238	0.00204886989662629\\
18075.4486631669	0.00179094292414862\\
18081.3087535099	0.00152817130194673\\
18087.168843853	0.00126675787536547\\
18093.0289341961	0.00101287147215145\\
18098.8890245391	0.000772501337282419\\
18104.7491148822	0.000551315851514117\\
18110.6092052252	0.000354528866579199\\
18116.4692955683	0.000186776808663174\\
18122.3293859114	5.20094469453597e-05\\
18128.1894762544	-4.6603099474696e-05\\
18134.0495665975	-0.000106744926945176\\
18139.9096569405	-0.000127008731112874\\
18145.7697472836	-0.000106928732249591\\
18151.6298376267	-4.69913718307363e-05\\
18157.4899279697	5.13764650934753e-05\\
18163.3500183128	0.00018584141676491\\
18169.2101086558	0.000353218992327754\\
18175.0701989989	0.000549548987395443\\
18180.930289342	0.000770189181675641\\
18186.790379685	0.00100992510187001\\
18192.6504700281	0.00126309325460047\\
18198.5105603711	0.00152371491722304\\
18204.3706507142	0.00178563732594442\\
18210.2307410573	0.00204267892724433\\
18216.0908314003	0.00228877526361677\\
18221.9509217434	0.00251812205101473\\
18227.8110120864	0.00272531207219984\\
18233.6711024295	0.00290546265778688\\
18239.5311927726	0.00305433074932621\\
18245.3912831156	0.00316841283337592\\
18251.2513734587	0.00324502739241868\\
18257.1114638017	0.00328237793273344\\
18262.9715541448	0.00327959510792965\\
18268.8316444879	0.00323675694993573\\
18274.6917348309	0.00315488673757002\\
18280.551825174	0.00303592855912138\\
18286.411915517	0.00288270115277156\\
18292.2720058601	0.00269883112095715\\
18298.1320962031	0.00248866710038391\\
18303.9921865462	0.00225717691971971\\
18309.8522768893	0.00200983017628503\\
18315.7123672323	0.00175246900762923\\
18321.5724575754	0.00149117011098425\\
18327.4325479184	0.00123210126927024\\
18333.2926382615	0.000981375771129694\\
18339.1527286046	0.000744908160966237\\
18345.0128189476	0.000528274722729892\\
18350.8729092907	0.000336581988190763\\
18356.7329996337	0.000174346370379179\\
18362.5930899768	4.53877591612407e-05\\
18368.4531803199	-4.72604142253817e-05\\
18374.3132706629	-0.000101422526372739\\
18380.173361006	-0.000115831846490102\\
18386.033451349	-9.01601573534626e-05\\
18391.8935416921	-2.50252198610244e-05\\
18397.7536320352	7.80240883375843e-05\\
18403.6137223782	0.000216544602743797\\
18409.4738127213	0.000387256794770521\\
18415.3339030643	0.000586122414200106\\
18421.1939934074	0.000808440013231233\\
18427.0540837505	0.00104895609258157\\
18432.9141740935	0.00130198924221325\\
18438.7742644366	0.00156156434323884\\
18444.6343547796	0.00182155366113274\\
18450.4944451227	0.00207582149875784\\
18456.3545354658	0.00231836899430911\\
18462.2146258088	0.00254347564764448\\
18468.0747161519	0.00274583423561655\\
18473.9348064949	0.00292067593503878\\
18479.794896838	0.00306388270339354\\
18485.654987181	0.00317208426875\\
18491.5150775241	0.00324273744517434\\
18497.3751678672	0.00327418590659551\\
18503.2352582102	0.00326569901448781\\
18509.0953485533	0.00321748878943333\\
18514.9554388963	0.00313070463210016\\
18520.8155292394	0.00300740592507665\\
18526.6756195825	0.00285051316814547\\
18532.5357099255	0.00266373880694001\\
18538.3958002686	0.00245149939270799\\
18544.2558906116	0.00221881115160648\\
18550.1159809547	0.0019711714330792\\
18555.9760712978	0.00171442883853689\\
18561.8361616408	0.00145464509873329\\
18567.6962519839	0.00119795196145568\\
18573.5563423269	0.000950406468306214\\
18579.41643267	0.000717848036250807\\
18585.2765230131	0.000505760715971023\\
18591.1366133561	0.000319143876072086\\
18596.9967036992	0.000162394362353604\\
18602.8567940422	3.92029096404471e-05\\
18608.7168843853	-4.75327533708477e-05\\
18614.5769747284	-9.57760604676606e-05\\
18620.4370650714	-0.000104399422478881\\
18626.2971554145	-7.32105775005867e-05\\
18632.1572457575	-2.956861103588e-06\\
18638.0173361006	0.000104692697937652\\
18643.8774264437	0.000247187027568191\\
18649.7375167867	0.000421153466158602\\
18655.5976071298	0.000622477587468703\\
18661.4576974728	0.000846400507645332\\
18667.3177878159	0.00108763137293946\\
18673.177878159	0.00134047236945581\\
18679.037968502	0.00159895330114962\\
18684.8980588451	0.00185697255779655\\
18690.7581491881	0.00210844114424735\\
18696.6182395312	0.002347426371543\\
18702.4783298743	0.00256829181876632\\
18708.3384202173	0.00276583026410804\\
18714.1985105604	0.00293538644997722\\
18720.0586009034	0.00307296678813353\\
18725.9186912465	0.00317533341923403\\
18731.7787815895	0.0032400804121195\\
18737.6388719326	0.00326569030914807\\
18743.4989622757	0.0032515696888833\\
18749.3590526187	0.00319806291285394\\
18755.2191429618	0.00310644373782046\\
18761.0792333048	0.00297888499786085\\
18766.9393236479	0.00281840707790009\\
18772.799413991	0.0026288064000839\\
18778.659504334	0.00241456561663699\\
18784.5195946771	0.00218074763341678\\
18790.3796850201	0.00193287596938492\\
18796.2397753632	0.0016768042792222\\
18802.0998657063	0.00141857812115274\\
18807.9599560493	0.00116429223417216\\
18813.8200463924	0.000919946693924104\\
18819.6801367354	0.000691305342053608\\
18825.5402270785	0.000483759829222655\\
18831.4003174216	0.000302202478736994\\
18837.2604077646	0.000150910968479387\\
18843.1204981077	3.3447549295294e-05\\
18848.9805884507	-4.74248258504103e-05\\
18854.8406787938	-8.98074873347542e-05\\
18860.7007691369	-9.27106205544781e-05\\
18866.5608594799	-5.60763775477899e-05\\
18872.420949823	1.92200110015848e-05\\
18878.281040166	0.00013139114364504\\
18884.1411305091	0.00027777987656005\\
18890.0012208522	0.000454922265381188\\
18895.8613111952	0.000658629529578646\\
18901.7214015383	0.00088408709999427\\
18907.5814918813	0.00112596840758451\\
18913.4415822244	0.00137856072409014\\
18919.3016725674	0.00163590008177391\\
18925.1617629105	0.00189191208585699\\
18931.0218532536	0.00214055529512317\\
18936.8819435966	0.0023759637864148\\
18942.7420339397	0.00259258553886001\\
18948.6021242827	0.00278531337337056\\
18954.4622146258	0.00294960535948395\\
18960.3223049689	0.00308159185063536\\
18966.1823953119	0.00317816662590684\\
18972.042485655	0.00323705999149077\\
18977.902575998	0.00325689212143291\\
18983.7626663411	0.00323720538448551\\
18989.6227566842	0.00317847489959295\\
18995.4828470272	0.00308209707643156\\
19001.3429373703	0.00295035641776533\\
19007.2030277133	0.00278637137247749\\
19013.0631180564	0.0025940205224798\\
19018.9232083995	0.00237785085088631\\
19024.7832987425	0.00214297026078808\\
19030.6433890856	0.00189492688498233\\
19036.5034794286	0.00163957803855868\\
19042.3635697717	0.00138295190946436\\
19048.2236601148	0.00113110525273996\\
19054.0837504578	0.000889980447925988\\
19059.9438408009	0.000665265292906495\\
19065.8039311439	0.00046225884215683\\
19071.664021487	0.000285746453934804\\
19077.5241118301	0.000139886992642582\\
19083.3842021731	2.81148451499311e-05\\
19089.2442925162	-4.69409405442995e-05\\
19095.1043828592	-8.35184904910721e-05\\
19100.9644732023	-8.07644595356537e-05\\
19106.8245635454	-3.87539349948621e-05\\
19112.6846538884	4.15115748822309e-05\\
19118.5447442315	0.000158128014557549\\
19124.4048345745	0.000308333947919791\\
19130.2649249176	0.000488575946729301\\
19136.1250152607	0.000694592652493112\\
19141.9851056037	0.00092151552327798\\
19147.8451959468	0.00116398388359977\\
19153.7052862898	0.0014162715587859\\
19159.5653766329	0.00167242210244262\\
19165.4254669759	0.0019263894235084\\
19171.285557319	0.00217218049308533\\
19177.1456476621	0.00240399676273911\\
19183.0057380051	0.00261637095717588\\
19188.8658283482	0.00280429601460806\\
19194.7259186912	0.00296334313373728\\
19200.5860090343	0.00308976614442579\\
19206.4460993774	0.00318058974261593\\
19212.3061897204	0.00323367951106862\\
19218.1662800635	0.00324779207856016\\
19224.0263704065	0.00322260423823502\\
19229.8864607496	0.00315872034386454\\
19235.7465510927	0.00305765781455272\\
19241.6066414357	0.00292181109561528\\
19247.4667317788	0.00275439493145204\\
19253.3268221218	0.00255936829414784\\
19259.1869124649	0.00234134076831363\\
19265.047002808	0.00210546360503902\\
19270.907093151	0.00185730802041086\\
19276.7671834941	0.00160273361346353\\
19282.6272738371	0.00134775001116313\\
19288.4873641802	0.0010983750073472\\
19294.3474545233	0.00086049254402621\\
19300.2075448663	0.000639713886660138\\
19306.0676352094	0.000441245268687923\\
19311.9277255524	0.000269765127162093\\
19317.7878158955	0.000129313824193651\\
19323.6479062386	2.3198453580295e-05\\
19329.5079965816	-4.60850246296429e-05\\
19335.3680869247	-7.69104883892955e-05\\
19341.2281772677	-6.85598160528999e-05\\
19347.0882676108	-2.12396113792094e-05\\
19352.9483579539	6.39238983051875e-05\\
19358.8084482969	0.000184911665046173\\
19364.66853864	0.000338859686494413\\
19370.528628983	0.000522126800690162\\
19376.3887193261	0.000730380804708862\\
19382.2488096691	0.000958700860340368\\
19388.1089000122	0.00120169376626511\\
19393.9689903553	0.00145362134940467\\
19399.8290806983	0.0017085359665373\\
19405.6891710414	0.00196042091644812\\
19411.5492613844	0.00220333244790632\\
19417.4093517275	0.0024315400118711\\
19423.2694420706	0.002639661448472\\
19429.1295324136	0.00282278991991434\\
19434.9896227567	0.00297660959547147\\
19440.8497130997	0.00309749736159997\\
19446.7098034428	0.00318260816004539\\
19452.5698937859	0.00322994194408079\\
19458.4299841289	0.00323839067706191\\
19464.290074472	0.00320776426904855\\
19470.150164815	0.00313879484447176\\
19476.0102551581	0.00303311924507743\\
19481.8703455012	0.00289324018648735\\
19487.7304358442	0.00272246699002114\\
19493.5905261873	0.00252483729308094\\
19499.4506165303	0.00230502159057735\\
19505.3107068734	0.00206821286369505\\
19511.1707972165	0.00182000390460429\\
19517.0308875595	0.00156625523458023\\
19522.8909779026	0.00131295673521743\\
19528.7510682456	0.00106608625917946\\
19534.6111585887	0.000831468558037372\\
19540.4712489318	0.000614637856179421\\
19546.3313392748	0.000420707313197848\\
19552.1914296179	0.00025424845348466\\
19558.0515199609	0.000119183406251231\\
19563.911610304	1.869249586499e-05\\
19569.7717006471	-4.48606402078331e-05\\
19575.6317909901	-6.99846428289845e-05\\
19581.4918813332	-5.60954245490624e-05\\
19587.3519716762	-3.5297438650051e-06\\
19593.2120620193	8.64629550001872e-05\\
19599.0721523623	0.000211750239203305\\
19604.9322427054	0.00036936721547464\\
19610.7923330485	0.000555586692197041\\
19616.6524233915	0.000766007315066848\\
19622.5125137346	0.000995657592319269\\
19628.3726040776	0.00123911335097204\\
19634.2326944207	0.0014906258491366\\
19640.0927847638	0.00174425751862979\\
19645.9528751068	0.00199402213284423\\
19651.8129654499	0.00223402609081966\\
19657.6730557929	0.00245860748341024\\
19663.533146136	0.00266246965982387\\
19669.3932364791	0.00284080614424192\\
19675.2533268221	0.00298941395599161\\
19681.1134171652	0.0031047926621266\\
19686.9735075082	0.00318422682797309\\
19692.8335978513	0.00322584992333993\\
19698.6936881944	0.00322868818089843\\
19704.5537785374	0.0031926833759882\\
19710.4138688805	0.00311869399495326\\
19716.2739592235	0.00300847476949458\\
19722.1340495666	0.00286463506467592\\
19727.9941399097	0.00269057710737819\\
19733.8542302527	0.00249041551766442\\
19739.7143205958	0.00226888004578474\\
19745.5744109388	0.00203120381427884\\
19751.4345012819	0.00178299970573255\\
19757.294591625	0.00153012781571654\\
19763.154681968	0.0012785571011974\\
19769.0147723111	0.00103422449039854\\
19774.8748626541	0.000802894779875644\\
19780.7349529972	0.000590024624599525\\
19786.5950433403	0.000400633830089154\\
19792.4551336833	0.000239186982342586\\
19798.3152240264	0.000109488206609079\\
19804.1753143694	1.45915358610645e-05\\
19810.0354047125	-4.32709993287928e-05\\
19815.8954950555	-6.27418663285009e-05\\
19821.7555853986	-4.33698768218519e-05\\
19827.6156757417	1.43793630239751e-05\\
19833.4757660847	0.000109134640544155\\
19839.3358564278	0.000238651693966943\\
19845.1959467708	0.000399866366243662\\
19851.0560371139	0.000588967096438991\\
19856.916127457	0.000801485033775356\\
19862.7762178	0.00103239964388067\\
19868.6363081431	0.00127625731164266\\
19874.4963984861	0.00152730013897931\\
19880.3564888292	0.00177960189582683\\
19886.2165791723	0.00202720791431624\\
19892.0766695153	0.00226427562413767\\
19897.9367598584	0.00248521241217128\\
19903.7968502014	0.00268480755391253\\
19909.6569405445	0.00285835510397872\\
19915.5170308876	0.00300176484823732\\
19921.3771212306	0.00311165870084367\\
19927.2372115737	0.00318545027524024\\
19933.0973019167	0.00322140575420832\\
19938.9573922598	0.00321868462698279\\
19944.8174826029	0.00317735933591647\\
19950.6775729459	0.00309841337376431\\
19956.537663289	0.00298371788102505\\
19962.397753632	0.00283598730003151\\
19968.2578439751	0.00265871513697177\\
19974.1179343182	0.00245609135167552\\
19979.9780246612	0.00223290332879933\\
19985.8381150043	0.00199442277095076\\
19991.6982053473	0.00174628118556276\\
19997.5582956904	0.0014943369067839\\
20003.4183860334	0.00124453679189974\\
20009.2784763765	0.00100277585766498\\
20015.1385667196	0.000774758169158772\\
20020.9986570626	0.000565862263930092\\
20026.8587474057	0.000381014286463652\\
20032.7188377487	0.000224571825151499\\
20038.5789280918	0.000100221191034979\\
20044.4390184349	1.08905596460874e-05\\
20050.2991087779	-4.1318977528854e-05\\
20056.159199121	-5.51828285233295e-05\\
20062.019289464	-3.03816211365579e-05\\
20067.8793798071	3.24914455642226e-05\\
20073.7394701502	0.000131944787599777\\
20079.5995604932	0.000265623821086051\\
20085.4596508363	0.000430366706511684\\
20091.3197411793	0.000622279132573751\\
20097.1798315224	0.000836826370871963\\
20103.0399218655	0.00106894042550905\\
20108.9000122085	0.00131313974585914\\
20114.7601025516	0.00156365867478947\\
20120.6201928946	0.00181458357556377\\
20126.4802832377	0.00205999242345\\
20132.3403735808	0.00229409456716184\\
20138.2004639238	0.00251136736170224\\
20144.0605542669	0.0027066864489546\\
20149.9206446099	0.00287544661260889\\
20155.780734953	0.00301367035727035\\
20161.6408252961	0.00311810165184447\\
20167.5009156391	0.00318628262782528\\
20173.3610059822	0.00321661142590622\\
20179.2210963252	0.00320837982946083\\
20185.0811866683	0.00316178980087591\\
20190.9412770114	0.00307794853466199\\
20196.8013673544	0.00295884214821608\\
20202.6614576975	0.00280728863553673\\
20208.5215480405	0.00262687119822895\\
20214.3816383836	0.00242185353095053\\
20220.2417287267	0.00219707906397015\\
20226.1018190697	0.0019578565441099\\
20231.9619094128	0.00170983465669336\\
20237.8219997558	0.00145886864972079\\
20243.6820900989	0.00121088210922227\\
20249.5421804419	0.000971727148917599\\
20255.402270785	0.000747046313836898\\
20261.2623611281	0.000542139456705485\\
20267.1224514711	0.000361838727574037\\
20272.9825418142	0.000210394625440384\\
20278.8426321572	9.13757987774044e-05\\
20284.7027225003	7.58495697702066e-06\\
20290.5628128434	-3.90071259829308e-05\\
20296.4229031864	-4.73079616430469e-05\\
20302.2829935295	-1.71289608970521e-05\\
20308.1430838725	5.0810288928693e-05\\
20314.0031742156	0.000154899180587666\\
20319.8632645587	0.000292674268021953\\
20325.7233549017	0.000460877566986869\\
20331.5834452448	0.000655533595495984\\
20337.4435355878	0.000872043332353873\\
20343.3036259309	0.00110529287324029\\
20349.163716274	0.00134977421724108\\
20355.023806617	0.00159971533113736\\
20360.8838969601	0.00184921642009313\\
20366.7439873031	0.00209238918780621\\
20372.6040776462	0.00232349579893282\\
20378.4641679893	0.00253708426446659\\
20384.3242583323	0.00272811705584226\\
20390.1843486754	0.00289208991347358\\
20396.0444390184	0.00302513804812941\\
20401.9045293615	0.00312412723097415\\
20407.7646197046	0.00318672762535123\\
20413.6247100476	0.00321146862150076\\
20419.4848003907	0.00319777338318522\\
20425.3448907337	0.0031459722950326\\
20431.2049810768	0.00305729499728184\\
20437.0650714198	0.00293384119935575\\
20442.9251617629	0.00277853096561077\\
20448.785252106	0.00259503564985561\\
20454.645342449	0.00238769111223189\\
20460.5054327921	0.00216139526995276\\
20466.3655231351	0.0019214924022144\\
20472.2256134782	0.00167364694233901\\
20478.0857038213	0.00142370973740285\\
20483.9457941643	0.00117757993293203\\
20489.8058845074	0.000941065743037567\\
20495.6659748504	0.000719747391844812\\
20501.5260651935	0.000518845460369104\\
20507.3861555366	0.000343097744896139\\
20513.2462458796	0.000196647531327036\\
20519.1063362227	8.2945920120351e-05\\
20524.9664265657	4.67050443822302e-06\\
20530.8265169088	-3.63376823612654e-05\\
20536.6866072519	-3.91174651187153e-05\\
20542.5466975949	-3.61005289553254e-06\\
20548.406787938	6.93397352561368e-05\\
20554.266878281	0.000178003569884647\\
20560.1269686241	0.000319810557931723\\
20565.9870589672	0.000491408066685149\\
20571.8471493102	0.00068874098589843\\
20577.7072396533	0.000907147554284746\\
20583.5673299963	0.00114146948591178\\
20589.4274203394	0.0013861737950614\\
20595.2875106825	0.00163548344248802\\
20601.1476010255	0.00188351371818666\\
20607.0076913686	0.00212441114117562\\
20612.8677817116	0.00235249159792822\\
20618.7278720547	0.00256237445937183\\
20624.5879623978	0.0027491095121929\\
20630.4480527408	0.00290829371006961\\
20636.3081430839	0.00303617499155613\\
20642.1682334269	0.00312974071630488\\
20648.02832377	0.00318678863591547\\
20653.8884141131	0.003205978726969\\
20659.7485044561	0.00318686466657632\\
20665.6085947992	0.00312990421121617\\
20671.4686851422	0.00303644823754358\\
20677.3287754853	0.00290870870720147\\
20683.1888658283	0.00274970631572453\\
20689.0489561714	0.00256319906432044\\
20694.9090465145	0.00235359344330247\\
20700.7691368575	0.00212584032652471\\
20706.6292272006	0.00188531803599131\\
20712.4893175436	0.00163770533880346\\
20718.3494078867	0.00138884737503024\\
20724.2094982298	0.00114461768209432\\
20730.0695885728	0.000910779572327836\\
20735.9296789159	0.000692850135347828\\
20741.7897692589	0.000495970074259087\\
20747.649859602	0.000324782446600355\\
20753.5099499451	0.000183323170174404\\
20759.3700402881	7.49258758179725e-05\\
20765.2301306312	2.14335015914561e-06\\
20771.0902209742	-3.33125804714552e-05\\
20776.9503113173	-3.0611309356979e-05\\
20782.8104016604	1.01770948662542e-05\\
20788.6704920034	8.80836917304688e-05\\
20794.5305823465	0.000201263685613291\\
20800.3906726895	0.000347040108862511\\
20806.2507630326	0.000521967137027511\\
20812.1108533757	0.000721911538721629\\
20817.9709437187	0.000942150334932957\\
20823.8310340618	0.00117748236034001\\
20829.6911244048	0.0014223510915718\\
20835.5512147479	0.00167097584175276\\
20841.411305091	0.00191748822394495\\
20847.271395434	0.00215607066185803\\
20853.1314857771	0.00238109367907378\\
20858.9915761201	0.00258724872640782\\
20864.8516664632	0.00276967341411233\\
20870.7117568062	0.00292406619401881\\
20876.5718471493	0.00304678778741879\\
20882.4319374924	0.00313494696671604\\
20888.2920278354	0.00318646866930303\\
20894.1521181785	0.00320014283863114\\
20900.0122085215	0.00317565284344941\\
20905.8722988646	0.00311358280714593\\
20911.7323892077	0.00301540367833292\\
20917.5924795507	0.00288343837409154\\
20923.4525698938	0.00272080682239459\\
20929.3126602368	0.00253135220365635\\
20935.1727505799	0.00231955013498935\\
20941.032840923	0.00209040294326255\\
20946.892931266	0.00184932152459245\\
20952.7530216091	0.00160199757994134\\
20958.6131119521	0.00135426924396446\\
20964.4732022952	0.00111198327904755\\
20970.3332926383	0.000880857087363718\\
20976.1933829813	0.000666343797465135\\
20982.0534733244	0.000473503608921444\\
20987.9135636674	0.000306884430186792\\
20993.7736540105	0.000170414625237528\\
20999.6337443536	6.7310398293246e-05\\
nan	nan\\
21011.3539250397	-2.99334587582342e-05\\
21017.2140153827	-2.1789238725292e-05\\
21023.0741057258	2.4234625778998e-05\\
21028.9341960689	0.000107046138705955\\
21034.7942864119	0.000224685251098001\\
21040.654376755	0.000374370252199059\\
21046.514467098	0.000552563544909791\\
21052.3745574411	0.000755055250273588\\
21058.2346477842	0.000977062665328763\\
21064.0947381272	0.00121334322453697\\
21069.9548284703	0.00145831829705226\\
21075.8149188133	0.00170620489645645\\
21081.6750091564	0.00195115219337911\\
21087.5350994994	0.00218737960872977\\
21093.3951898425	0.00240931322825304\\
21099.2552801856	0.00261171731907342\\
21105.1153705286	0.00278981784564513\\
21110.9754608717	0.00293941507080181\\
21116.8355512147	0.00305698258643913\\
21122.6956415578	0.00313975043883375\\
21128.5557319009	0.00318577038886933\\
21134.4158222439	0.00319396176997816\\
21140.275912587	0.00316413686430596\\
21146.13600293	0.00309700520125194\\
21151.9960932731	0.00299415667996184\\
21157.8561836162	0.00285802391758802\\
21163.7162739592	0.00269182471422838\\
21169.5763643023	0.00249948599611892\\
21175.4364546453	0.00228555103410204\\
21181.2965449884	0.00205507212966843\\
21187.1566353315	0.00181349130351671\\
21193.0167256745	0.00156651180374478\\
21198.8768160176	0.00131996346766029\\
21204.7369063606	0.00107966511550895\\
21210.5969967037	0.000851287224124466\\
21216.4570870468	0.00064021812117586\\
21222.3171773898	0.000451436857593549\\
21228.1772677329	0.000289395757199251\\
21234.0373580759	0.000157915414167216\\
21239.897448419	6.00946144661507e-05\\
21245.7575387621	-1.76269488871369e-06\\
21251.6176291051	-2.62016677201703e-05\\
21257.4777194482	-1.26507737700991e-05\\
21263.3378097912	3.85648376552169e-05\\
21269.1979001343	0.000126231137913003\\
21275.0579904774	0.000248273996051639\\
21280.9180808204	0.000401808250545425\\
21286.7781711635	0.000583205914801907\\
21292.6382615065	0.000788181904068069\\
21298.4983518496	0.00101189525817956\\
21304.3584421926	0.00124906346904916\\
21310.2185325357	0.00149408721301988\\
21316.0786228788	0.00174118254285544\\
21321.9387132218	0.00198451741860817\\
21327.7988035649	0.00221834935470449\\
21333.6588939079	0.00243716093445656\\
21339.518984251	0.00263578999428017\\
21345.3790745941	0.00280955140584624\\
21351.2391649371	0.00295434758344741\\
21357.0992552802	0.00306676510985293\\
21362.9593456232	0.00314415520234353\\
21368.8194359663	0.00318469612222795\\
21374.6795263094	0.00318743605714714\\
21380.5396166524	0.00315231546681139\\
21386.3997069955	0.00308016836798435\\
21392.2597973385	0.00297270253069246\\
21398.1198876816	0.00283245905618398\\
21403.9799780247	0.00266275229325185\\
21409.8400683677	0.00246759151377172\\
21415.7001587108	0.00225158619786335\\
21421.5602490538	0.00201983716683753\\
21427.4203393969	0.0017778161343922\\
21433.28042974	0.00153123652077801\\
21439.140520083	0.00128591857954535\\
21445.0006104261	0.00104765202075956\\
21450.8607007691	0.000822059373266662\\
21456.7207911122	0.000614463310369162\\
21462.5808814553	0.000429761069651542\\
21468.4409717983	0.000272308929783336\\
21474.3010621414	0.000145819469240684\\
21480.1611524844	5.32740301088044e-05\\
21486.0212428275	-3.1475492157126e-06\\
21491.8813331706	-2.21182763003273e-05\\
21497.7414235136	-3.19521269947748e-06\\
21503.6015138567	5.31701861906142e-05\\
21509.4616041997	0.000145642840780946\\
21515.3216945428	0.000272035669565267\\
21521.1817848858	0.00042936131505069\\
21527.0418752289	0.000613902750050097\\
21532.901965572	0.000821301095556858\\
21538.762055915	0.00104665857552332\\
21544.6221462581	0.00128465417689415\\
21550.4822366011	0.00152966928354407\\
21556.3423269442	0.00177592031809374\\
21562.2024172873	0.00201759526007206\\
21568.0625076303	0.00224899081828803\\
21573.9225979734	0.00246464701992132\\
21579.7826883164	0.00265947604038275\\
21585.6427786595	0.00282888223403618\\
21591.5028690026	0.00296887053453593\\
21597.3629593456	0.00307614066740391\\
21603.2230496887	0.0031481649537589\\
21609.0831400317	0.00318324787040729\\
21614.9432303748	0.00318056596340672\\
21620.8033207179	0.00314018717540569\\
21626.6634110609	0.00306306913297706\\
21632.523501404	0.00295103643708406\\
21638.383591747	0.00280673749533747\\
21644.2436820901	0.00263358191700662\\
21650.1037724332	0.00243565995089277\\
21655.9638627762	0.00221764586926778\\
21661.8239531193	0.0019846875803692\\
21667.6840434623	0.0017422850759966\\
21673.5441338054	0.00149616058417645\\
21679.4042241485	0.00125212349272012\\
21685.2643144915	0.00101593323165793\\
21691.1244048346	0.000793163351206407\\
21696.9844951776	0.000589070002689916\\
21702.8445855207	0.000408467925958908\\
21708.7046758638	0.00025561686900393\\
21714.5647662068	0.000134121119197111\\
21720.4248565499	4.68445156296738e-05\\
21726.2849468929	-4.15705347971521e-06\\
21732.145037236	-1.76840772969702e-05\\
21738.0051275791	6.57836785429514e-06\\
21743.8652179221	6.8053288864956e-05\\
21749.7253082652	0.000165285496909142\\
21755.5853986082	0.000295976052946832\\
21761.4454889513	0.000457036622334211\\
21767.3055792943	0.000644662453455298\\
21773.1656696374	0.000854422255905918\\
21779.0257599805	0.0010813628551111\\
21784.8858503235	0.0013201261518781\\
21790.7459406666	0.00156507562502382\\
21796.6060310096	0.00181042939062635\\
21802.4661213527	0.00205039667687175\\
21808.3262116958	0.00227931449321152\\
21814.1863020388	0.00249178126828024\\
21820.0463923819	0.00268278430328859\\
21825.9064827249	0.00284781803319105\\
21831.766573068	0.00298299030619131\\
21837.6266634111	0.00308511417400602\\
21843.4867537541	0.00315178302898263\\
21849.3468440972	0.00318142731597603\\
21855.2069344402	0.0031733514830158\\
21861.0670247833	0.00312775030068174\\
21866.9271151264	0.00304570416747892\\
21872.7872054694	0.00292915351416795\\
21878.6472958125	0.00278085291357972\\
21884.5073861555	0.00260430598086248\\
21890.3674764986	0.00240368260310614\\
21896.2275668417	0.00218372045324131\\
21902.0876571847	0.00194961311443229\\
21907.9477475278	0.00170688745674643\\
21913.8078378708	0.00146127316119429\\
21919.6679282139	0.00121856747127369\\
21925.5280185569	0.000984498364458058\\
21931.3881089	0.000764589372995192\\
21937.2481992431	0.000564029244210892\\
21943.1082895861	0.000387549515836936\\
21948.9683799292	0.000239312894783654\\
21954.8284702722	0.000122815072568025\\
21960.6885606153	4.08022931965327e-05\\
21966.5486509584	-4.79338274669962e-06\\
21972.4087413014	-1.28995918351125e-05\\
21978.2688316445	1.66711127725148e-05\\
21984.1289219875	8.32169292912925e-05\\
21989.9890123306	0.000185163462725372\\
21995.8491026737	0.000320100972490042\\
22001.7091930167	0.000484841331038173\\
22007.5692833598	0.000675493347235683\\
22013.4293737028	0.000887554674900925\\
22019.2894640459	0.00111601813571697\\
22025.149554389	0.00135548994578969\\
22031.009644732	0.00160031705445061\\
22036.8697350751	0.00184472058897597\\
22042.7298254181	0.00208293225547603\\
22048.5899157612	0.00230933047621894\\
22054.4500061043	0.00251857305087442\\
22060.3100964473	0.00270572321065645\\
22066.1701867904	0.00286636609161833\\
22072.0302771334	0.00299671287874913\\
22077.8903674765	0.003093690164508\\
22083.7504578196	0.00315501241427375\\
22089.6105481626	0.00317923583013728\\
22095.4706385057	0.00316579234372959\\
22101.3307288487	0.00311500293750803\\
22107.1908191918	0.00302806998259028\\
22113.0509095349	0.00290704877566157\\
22118.9109998779	0.00275479894871231\\
22124.771090221	0.00257491690069267\\
22130.631180564	0.00237165084697883\\
22136.4912709071	0.00214980049387174\\
22142.3513612502	0.00191460370676974\\
22148.2114515932	0.001671612848364\\
22154.0715419363	0.00142656370606698\\
22159.9316322793	0.0011852401029903\\
22165.7917226224	0.000953337388016618\\
22171.6518129654	0.000736328026782848\\
22177.5119033085	0.000539332465649766\\
22183.3719936516	0.000366998315679557\\
22189.2320839946	0.00022339070732694\\
22195.0921743377	0.000111896402403045\\
22200.9522646807	3.51439251248846e-05\\
22206.8123550238	-5.05840432979276e-06\\
22212.6724453669	-7.76507293368395e-06\\
22218.5325357099	2.7084389079859e-05\\
22224.392626053	9.86640620266708e-05\\
22230.252716396	0.000205281210359837\\
22236.1128067391	0.000344416312183844\\
22241.9728970822	0.000512782598102907\\
22247.8329874252	0.000706403692488533\\
22253.6930777683	0.000920707523120472\\
22259.5531681113	0.00115063428147733\\
22265.4132584544	0.00139075588434417\\
22271.2733487975	0.00163540411647749\\
22277.1334391405	0.00187880442921946\\
22282.9935294836	0.00211521223675625\\
22288.8536198266	0.00233904849334019\\
22294.7137101697	0.0025450313515061\\
22300.5738005128	0.00272830079464606\\
22306.4338908558	0.00288453330298151\\
22312.2939811989	0.00301004384776905\\
22318.1540715419	0.00310187280745876\\
22324.014161885	0.00315785575616815\\
22329.8742522281	0.00317667447853186\\
22335.7343425711	0.00315788800870175\\
22341.5944329142	0.00310194296288637\\
22347.4545232572	0.00301016292294484\\
22353.3146136003	0.00288471712361161\\
22359.1747039434	0.00272856918415062\\
22365.0347942864	0.00254540709595185\\
22370.8948846295	0.00233955612011781\\
22376.7549749725	0.00211587665201912\\
22382.6150653156	0.00187964946473694\\
22388.4751556586	0.00163645104061069\\
22394.3352460017	0.00139202193415246\\
22400.1953363448	0.00115213127346555\\
22406.0554266878	0.000922440598519648\\
22411.9155170309	0.000708370249733787\\
22417.7756073739	0.000514971460086131\\
22423.635697717	0.000346807168967215\\
22429.4957880601	0.000207844369965308\\
22435.3558784031	0.00010136053232084\\
22441.2159687462	2.98663034621566e-05\\
22447.0760590892	-4.95368442179901e-06\\
22452.9361494323	-2.28050819703234e-06\\
22458.7962397754	3.78197872575473e-05\\
22464.6563301184	0.000114397817855907\\
22470.5164204615	0.000225643336778932\\
22476.3765108045	0.00036892802650677\\
22482.2366011476	0.000540867594888604\\
22488.0966914907	0.000737401708231853\\
22493.9567818337	0.000953889873499597\\
22499.8168721768	0.00118522100549985\\
22505.6769625198	0.00142593409216422\\
22511.5370528629	0.00167034710918789\\
22517.397143206	0.00191269114091845\\
22523.257233549	0.0021472465417992\\
22529.1173238921	0.0023684779246202\\
22534.9774142351	0.00257116478977409\\
22540.8375045782	0.00275052471302078\\
22546.6975949213	0.00290232618492463\\
22552.5576852643	0.00302298843953884\\
22558.4177756074	0.00310966591729985\\
22564.2778659504	0.00316031537020093\\
22570.1379562935	0.00317374402608486\\
22575.9980466366	0.00314963767746706\\
22581.8581369796	0.00308856803303492\\
22587.7182273227	0.00299197915987841\\
22593.5783176657	0.00286215333808297\\
22599.4384080088	0.00270215713492345\\
22605.2984983518	0.00251576897278676\\
22611.1585886949	0.00230738990181894\\
22617.018679038	0.00208193968385947\\
22622.878769381	0.00184474064196413\\
22628.7388597241	0.00160139201688716\\
22634.5989500671	0.00135763779703005\\
22640.4590404102	0.00111923114136709\\
22646.3191307533	0.000891798595481356\\
22652.1792210963	0.000680707305269361\\
22658.0393114394	0.000490938362023845\\
22663.8994017824	0.000326969267657327\\
22669.7594921255	0.000192668293299023\\
22675.6195824686	9.12032237855432e-05\\
22681.4796728116	2.49666407026223e-05\\
22687.3397631547	-4.4804937216856e-06\\
22693.1998534977	3.55437834762091e-06\\
22699.0599438408	4.8879123236168e-05\\
22704.9200341839	0.000130421509569822\\
22710.7801245269	0.00024625457320355\\
22716.64021487	0.000393642153240611\\
22722.500305213	0.000569103523113763\\
22728.3603955561	0.00076849559011972\\
22734.2204858992	0.000987110722275153\\
22740.0805762422	0.00121978789258141\\
22745.9406665853	0.00146103451687346\\
22751.8007569283	0.00170515610887546\\
22757.6608472714	0.00194639069217638\\
22763.5209376145	0.00217904479632731\\
22769.3810279575	0.00239762782755336\\
22775.2411183006	0.0025969816427352\\
22781.1012086436	0.00277240226895486\\
22786.9612989867	0.00291975089634746\\
22792.8213893298	0.00303555152548288\\
22798.6814796728	0.00311707296543532\\
22804.5415700159	0.0031623932482611\\
22810.4016603589	0.00317044494088894\\
22816.261750702	0.00314104028593801\\
22822.121841045	0.00307487557957131\\
22827.9819313881	0.00297351468418514\\
22833.8420217312	0.00283935206639484\\
22839.7021120742	0.00267555623391138\\
22845.5622024173	0.00248599490723884\\
22851.4222927603	0.00227514369367563\\
22857.2823831034	0.00204798041953301\\
22863.1424734465	0.00180986761577869\\
22869.0025637895	0.00156642593077006\\
22874.8626541326	0.00132340145869683\\
22880.7227444756	0.00108653011481503\\
};
\addplot [color=mycolor2, forget plot]
  table[row sep=crcr]{%
22880.7227444756	0.00108653011481503\\
22886.5828348187	0.000861402258853052\\
22892.4429251618	0.000653330761546587\\
22898.3030155048	0.000467225627731655\\
22904.1631058479	0.000307478134835005\\
22910.0231961909	0.000177857220571269\\
22915.883286534	8.14205645561398e-05\\
22921.7433768771	2.04424615879587e-05\\
22927.6034672201	-3.63981209199257e-06\\
22933.4635575632	9.74012524708163e-06\\
22939.3236479062	6.02644410705445e-05\\
22945.1837382493	0.000146738638253502\\
22951.0438285924	0.000267119794869069\\
22956.9039189354	0.000418564826472744\\
22962.7640092785	0.000597497630761689\\
22968.6240996215	0.000799693528888538\\
22974.4841899646	0.00102037900962845\\
22980.3442803077	0.0012543444214817\\
22986.2043706507	0.00149606695238417\\
22992.0644609938	0.00173984099396714\\
22997.9245513368	0.00197991281340175\\
23003.7846416799	0.00221061635395005\\
23009.6447320229	0.00242650695928725\\
23015.504822366	0.00262248986564705\\
23021.3649127091	0.00279394042944539\\
23027.2250030521	0.00293681325323594\\
23033.0850933952	0.00304773763518869\\
23038.9451837382	0.00312409708999906\\
23044.8052740813	0.00316409106521263\\
23050.6653644244	0.0031667773971503\\
23056.5254547674	0.00313209450585511\\
23062.3855451105	0.00306086280545275\\
23068.2456354535	0.00295476529839557\\
23074.1057257966	0.00281630781211539\\
23079.9658161397	0.00264875981775918\\
23085.8259064827	0.0024560772285584\\
23091.6859968258	0.00224280900082403\\
23097.5460871688	0.00201398974254761\\
23103.4061775119	0.00177502086512199\\
23109.266267855	0.00153154308315336\\
23115.126358198	0.00128930327248653\\
23120.9864485411	0.00105401882863958\\
23126.8465388841	0.000831242727223982\\
23132.7066292272	0.000626232471060041\\
23138.5667195703	0.000443826016700346\\
23144.4268099133	0.000288327608511191\\
23150.2869002564	0.000163406214180456\\
23156.1469905994	7.20089582365438e-05\\
23162.0070809425	1.62915959366599e-05\\
23167.8671712856	-2.43233227679195e-06\\
23173.7272616286	1.62775334424545e-05\\
23179.5873519717	7.19780163018575e-05\\
23185.4474423147	0.000163352900106792\\
23191.3075326578	0.000288244031139553\\
23197.1676230009	0.000443702289782391\\
23203.0277133439	0.000626057228032108\\
23208.887803687	0.000831003728648045\\
23214.74789403	0.00105370363988343\\
23220.6079843731	0.00128889998659323\\
23226.4680747162	0.00153104106156609\\
23232.3281650592	0.0017744114680305\\
23238.1882554023	0.00201326702029599\\
23244.0483457453	0.00224197031848113\\
23249.9084360884	0.00245512379765688\\
23255.7685264314	0.00264769711123416\\
23261.6286167745	0.00281514584262769\\
23267.4887071176	0.00295351874348592\\
23273.3487974606	0.00305955096821521\\
23279.2088878037	0.00313074110472752\\
23285.0689781467	0.00316541018410151\\
23290.9290684898	0.00316274127696465\\
23296.7891588329	0.00312279874297584\\
23302.6492491759	0.00304652667970685\\
23308.509339519	0.00293572660830513\\
23314.369429862	0.00279301492349569\\
23320.2295202051	0.00262176111254601\\
23326.0896105482	0.00242600820243765\\
23331.9497008912	0.00221037731292248\\
23337.8097912343	0.001979958569241\\
23343.6698815773	0.0017401909490252\\
23349.5299719204	0.00149673390007424\\
23355.3900622635	0.00125533375879847\\
23361.2501526065	0.0010216881225975\\
23367.1102429496	0.000801311376971325\\
23372.9703332926	0.000599404551240569\\
23378.8304236357	0.00042073257420086\\
23384.6905139788	0.000269511826509594\\
23390.5506043218	0.000149310643285781\\
23396.4106946649	6.29651148781004e-05\\
23402.2707850079	1.25121724722545e-05\\
23408.130875351	-8.5846270381279e-07\\
23413.9909656941	2.31676669029033e-05\\
23419.8510560371	8.40223600124816e-05\\
23425.7111463802	0.00018026819382562\\
23431.5712367232	0.000309632476041659\\
23437.4313270663	0.000469060909647021\\
23443.2914174093	0.000654789703332076\\
23449.1515077524	0.000862434425106863\\
23455.0115980955	0.00108709350147891\\
23460.8716884385	0.00132346391919193\\
23466.7317787816	0.00156596639829582\\
23472.5918691246	0.00180887708212871\\
23478.4519594677	0.00204646263590623\\
23484.3120498108	0.00227311556530636\\
23490.1721401538	0.00248348656134311\\
23496.0322304969	0.00267261074815455\\
23501.8923208399	0.00283602485397446\\
23507.752411183	0.00296987254054473\\
23513.6125015261	0.00307099540487415\\
23519.4725918691	0.00313700750641909\\
23525.3326822122	0.00316635166067952\\
23531.1927725552	0.00315833617138655\\
23537.0528628983	0.00311315113484939\\
23542.9129532414	0.00303186393183529\\
23548.7730435844	0.00291639401417057\\
23554.6331339275	0.00276946758167119\\
23560.4932242705	0.00259455321928503\\
23566.3533146136	0.00239578001412959\\
23572.2134049567	0.00217784008551419\\
23578.0734952997	0.00194587782846629\\
23583.9335856428	0.00170536848525696\\
23589.7936759858	0.00146198891095097\\
23595.6537663289	0.00122148358345847\\
23601.513856672	0.000989529020273607\\
23607.373947015	0.000771599802240223\\
23613.2340373581	0.00057283936599741\\
23619.0941277011	0.000397938614805616\\
23624.9542180442	0.000251025212382475\\
23630.8143083873	0.000135566172420187\\
23636.6743987303	5.42860425817925e-05\\
23642.5344890734	9.10261360959128e-06\\
23648.3945794164	1.08167060916587e-06\\
23654.2546697595	3.04118540724698e-05\\
23660.1147601026	9.64002235887186e-05\\
23665.9748504456	0.000197488628571264\\
23671.8349407887	0.000331290499238187\\
23677.6950311317	0.000494647189191627\\
23683.5551214748	0.000683702539442639\\
23689.4152118178	0.000893993903777696\\
23695.2753021609	0.00112055748681614\\
23701.135392504	0.00135804550845667\\
23706.995482847	0.00160085242911215\\
23712.8555731901	0.00184324725662905\\
23718.7156635331	0.00207950881207306\\
23724.5757538762	0.00230406076177345\\
23730.4358442193	0.00251160322893876\\
23736.2959345623	0.00269723787816496\\
23742.1560249054	0.00285658352129508\\
23748.0161152484	0.00298587951579517\\
23753.8762055915	0.00308207451594186\\
23759.7362959346	0.0031428984816753\\
23765.5963862776	0.00316691624660542\\
23771.4564766207	0.0031535613804648\\
23777.3165669637	0.00310314954747763\\
23783.1766573068	0.00301687104530908\\
23789.0367476499	0.00289676270111265\\
23794.8968379929	0.00274565978831946\\
23800.756928336	0.00256712909893085\\
23806.617018679	0.00236538475137398\\
23812.4771090221	0.00214518872106238\\
23818.3371993652	0.00191173844163099\\
23824.1972897082	0.00167054412948655\\
23830.0573800513	0.00142729872741364\\
23835.9174703943	0.0011877435365919\\
23841.7775607374	0.000957532708655188\\
23847.6376510805	0.000742099795682522\\
23853.4977414235	0.000546529508093835\\
23859.3578317666	0.000375437706815771\\
23865.2179221096	0.000232862462252826\\
23871.0780124527	0.00012216875107912\\
23876.9381027957	4.59690400613531e-05\\
23882.7981931388	6.06163117441586e-06\\
23888.6582834819	3.38822303215212e-06\\
23894.5183738249	3.80116901284416e-05\\
23900.378464168	0.000109114604210599\\
23906.238554511	0.000215018532550082\\
23912.0986448541	0.000353223657520941\\
23917.9587351972	0.000520467782298221\\
23923.8188255402	0.000712803329900807\\
23929.6789158833	0.000925690518250051\\
23935.5390062263	0.00115410451200649\\
23941.3990965694	0.00139265402219157\\
23947.2591869125	0.00163570855442351\\
23953.1192772555	0.00187753130238685\\
23958.9793675986	0.0021124145500993\\
23964.8394579416	0.00233481438709966\\
23970.6995482847	0.0025394815572651\\
23976.5596386278	0.00272158535269575\\
23982.4197289708	0.00287682762893246\\
23988.2798193139	0.00300154425033711\\
23994.1399096569	0.00309279157124961\\
24000	0.00314841591240077\\
};
\end{axis}
\end{tikzpicture}%

% This file was created by matlab2tikz.
%
%The latest updates can be retrieved from
%  http://www.mathworks.com/matlabcentral/fileexchange/22022-matlab2tikz-matlab2tikz
%where you can also make suggestions and rate matlab2tikz.
%
\definecolor{mycolor1}{rgb}{0.00000,0.44700,0.74100}%
\definecolor{mycolor2}{rgb}{0.85000,0.32500,0.09800}%
%
\begin{tikzpicture}

\begin{axis}[%
width=4.521in,
height=3.548in,
at={(0.758in,0.499in)},
scale only axis,
xmin=-0.0025,
xmax=0.0025,
ymin=-0.2,
ymax=1,
axis background/.style={fill=white}
]
\addplot [color=mycolor1, forget plot]
  table[row sep=crcr]{%
-0.00208333333333333	-0.00275664447710899\\
-0.00206249999999997	-0.00321525137559386\\
-0.00204166666666672	-0.0028129025276622\\
-0.00202083333333336	-0.00164077260919482\\
-0.00197916666666664	0.00167531519044095\\
-0.00195833333333328	0.00293260050756272\\
-0.00193750000000004	0.00342268694821279\\
-0.00191666666666668	0.00299635269250975\\
-0.00189583333333332	0.00174895541859221\\
-0.00185416666666671	-0.00178825778754943\\
-0.00183333333333335	-0.00313255054216932\\
-0.00181249999999999	-0.00365873432395158\\
-0.00179166666666664	-0.0032054005547778\\
-0.00177083333333339	-0.00187241109519876\\
-0.00172916666666667	0.00191752943484214\\
-0.00170833333333331	0.00336176155744994\\
-0.00168749999999995	0.0039297516812814\\
-0.00166666666666671	0.00344580559638619\\
-0.00164583333333335	0.00201461953280879\\
-0.00160416666666663	-0.00206694731288171\\
-0.00158333333333338	-0.00362716378566974\\
-0.00156250000000002	-0.00424413181578387\\
-0.00154166666666666	-0.00372519523933645\\
-0.0015208333333333	-0.002180204699889\\
-0.0014791666666667	0.00224161891678731\\
-0.00145833333333334	0.00393806353872705\\
-0.00143749999999998	0.00461318675628686\\
-0.00141666666666662	0.00405388893692493\\
-0.00139583333333337	0.00237544691181935\\
-0.00135416666666666	-0.00244853758602914\\
-0.0013333333333333	-0.00430725699548273\\
-0.00131250000000005	-0.00505253787593318\\
-0.00129166666666669	-0.00444620076953062\\
-0.00127083333333333	-0.00260909742773596\\
-0.00122916666666661	0.00269754140833722\\
-0.00120833333333337	0.0047528353053603\\
-0.00118750000000001	0.00558438396813665\\
-0.00116666666666665	0.00492257942340879\\
-0.00114583333333329	0.00289372623803441\\
-0.00110416666666668	-0.0030029234545641\\
-0.00108333333333333	-0.00530123937905569\\
-0.00106249999999997	-0.00624137031732919\\
-0.00104166666666672	-0.00551328895421788\\
-0.00102083333333336	-0.00324806006309997\\
-0.000979166666666642	0.00338627538493397\\
-0.000958333333333283	0.0059927053850195\\
-0.000937500000000036	0.00707355302630641\\
-0.000916666666666677	0.00626510108433853\\
-0.000895833333333318	0.00370127774632312\\
-0.000854166666666711	-0.00388182788029012\\
-0.000833333333333353	-0.00689161119277237\\
-0.000812499999999994	-0.00816179195343048\\
-0.000791666666666635	-0.00725432757133937\\
-0.000770833333333387	-0.00430148494842963\\
-0.00072916666666667	0.00454728408833982\\
-0.000708333333333311	0.00810777787384986\\
-0.000687499999999952	0.00964575412678159\\
-0.000666666666666704	0.00861451399096547\\
-0.000645833333333345	0.00513403042231919\\
-0.000624999999999987	0\\
-0.000604166666666628	-0.00548810148592738\\
-0.00058333333333338	-0.00984515884681769\\
-0.000562500000000021	-0.0117892550438441\\
-0.000541666666666663	-0.0106024787581114\\
-0.000520833333333304	-0.00636619772367586\\
-0.000499999999999945	0\\
-0.000479166666666697	0.00691978013443029\\
-0.000458333333333338	0.0125302021686771\\
-0.00043749999999998	0.0151576136277995\\
-0.000416666666666621	0.0137832223855449\\
-0.000395833333333373	0.00837657595220498\\
-0.000375000000000014	1.11022302462516e-16\\
-0.000354166666666655	-0.00936205547599389\\
-0.000333333333333297	-0.017229027981931\\
-0.000312500000000049	-0.0212206590789193\\
-0.00029166666666669	-0.0196903176936354\\
-0.000270833333333331	-0.0122426879301458\\
-0.000249999999999972	0\\
-0.000229166666666614	0.0144686311901723\\
-0.000208333333333366	0.0275664447710896\\
-0.000187500000000007	0.0353677651315323\\
-0.000166666666666648	0.034458055963862\\
-0.00014583333333329	0.0227364204416993\\
-0.000125000000000042	1.11022302462516e-16\\
-0.000104166666666683	-0.0318309886183786\\
-6.24999999999654e-05	-0.106103295394597\\
-4.16666666667176e-05	-0.137832223855448\\
-2.08333333333588e-05	-0.159154943091895\\
0	0.833333333333333\\
2.08333333333588e-05	-0.159154943091895\\
4.16666666667176e-05	-0.137832223855448\\
6.24999999999654e-05	-0.106103295394597\\
0.000104166666666683	-0.0318309886183786\\
0.000125000000000042	1.11022302462516e-16\\
0.00014583333333329	0.0227364204416993\\
0.000166666666666648	0.034458055963862\\
0.000187500000000007	0.0353677651315323\\
0.000208333333333366	0.0275664447710896\\
0.000229166666666614	0.0144686311901723\\
0.000249999999999972	0\\
0.000270833333333331	-0.0122426879301458\\
0.00029166666666669	-0.0196903176936354\\
0.000312500000000049	-0.0212206590789193\\
0.000333333333333297	-0.017229027981931\\
0.000354166666666655	-0.00936205547599389\\
0.000375000000000014	1.11022302462516e-16\\
0.000395833333333373	0.00837657595220498\\
0.000416666666666621	0.0137832223855449\\
0.00043749999999998	0.0151576136277995\\
0.000458333333333338	0.0125302021686771\\
0.000479166666666697	0.00691978013443029\\
0.000499999999999945	0\\
0.000520833333333304	-0.00636619772367586\\
0.000541666666666663	-0.0106024787581114\\
0.000562500000000021	-0.0117892550438441\\
0.00058333333333338	-0.00984515884681769\\
0.000604166666666628	-0.00548810148592738\\
0.000645833333333345	0.00513403042231919\\
0.000666666666666704	0.00861451399096547\\
0.000687499999999952	0.00964575412678159\\
0.000708333333333311	0.00810777787384986\\
0.00072916666666667	0.00454728408833982\\
0.000770833333333387	-0.00430148494842963\\
0.000791666666666635	-0.00725432757133937\\
0.000812499999999994	-0.00816179195343048\\
0.000833333333333353	-0.00689161119277237\\
0.000854166666666711	-0.00388182788029012\\
0.000895833333333318	0.00370127774632312\\
0.000916666666666677	0.00626510108433853\\
0.000937500000000036	0.00707355302630641\\
0.000958333333333283	0.0059927053850195\\
0.000979166666666642	0.00338627538493397\\
0.00102083333333336	-0.00324806006309997\\
0.00104166666666672	-0.00551328895421788\\
0.00106249999999997	-0.00624137031732919\\
0.00108333333333333	-0.00530123937905569\\
0.00110416666666668	-0.0030029234545641\\
0.00114583333333329	0.00289372623803441\\
0.00116666666666665	0.00492257942340879\\
0.00118750000000001	0.00558438396813665\\
0.00120833333333337	0.0047528353053603\\
0.00122916666666661	0.00269754140833722\\
0.00127083333333333	-0.00260909742773596\\
0.00129166666666669	-0.00444620076953062\\
0.00131250000000005	-0.00505253787593318\\
0.0013333333333333	-0.00430725699548273\\
0.00135416666666666	-0.00244853758602914\\
0.00139583333333337	0.00237544691181935\\
0.00141666666666662	0.00405388893692493\\
0.00143749999999998	0.00461318675628686\\
0.00145833333333334	0.00393806353872705\\
0.0014791666666667	0.00224161891678731\\
0.0015208333333333	-0.002180204699889\\
0.00154166666666666	-0.00372519523933645\\
0.00156250000000002	-0.00424413181578387\\
0.00158333333333338	-0.00362716378566974\\
0.00160416666666663	-0.00206694731288171\\
0.00164583333333335	0.00201461953280879\\
0.00166666666666671	0.00344580559638619\\
0.00168749999999995	0.0039297516812814\\
0.00170833333333331	0.00336176155744994\\
0.00172916666666667	0.00191752943484214\\
0.00177083333333339	-0.00187241109519876\\
0.00179166666666664	-0.0032054005547778\\
0.00181249999999999	-0.00365873432395158\\
0.00183333333333335	-0.00313255054216932\\
0.00185416666666671	-0.00178825778754943\\
0.00189583333333332	0.00174895541859221\\
0.00191666666666668	0.00299635269250975\\
0.00193750000000004	0.00342268694821279\\
0.00195833333333328	0.00293260050756272\\
0.00197916666666664	0.00167531519044095\\
0.00202083333333336	-0.00164077260919482\\
0.00204166666666672	-0.0028129025276622\\
0.00206249999999997	-0.00321525137559386\\
0.00208333333333333	-0.00275664447710899\\
};
\addplot [color=mycolor2, forget plot]
  table[row sep=crcr]{%
-0.00208333333333333	0.00275664447710897\\
-0.00206249999999999	0.00321525137559384\\
-0.00204166666666666	0.0028129025276622\\
-0.00202083333333333	0.00164077260919479\\
-0.00197916666666667	-0.00167531519044101\\
-0.00195833333333334	-0.00293260050756272\\
-0.00193750000000001	-0.00342268694821279\\
-0.00191666666666668	-0.00299635269250972\\
-0.00189583333333335	-0.00174895541859224\\
-0.00185416666666666	0.0017882577875494\\
-0.00183333333333333	0.00313255054216929\\
-0.00181249999999999	0.00365873432395161\\
-0.00179166666666666	0.00320540055477786\\
-0.00177083333333333	0.00187241109519876\\
-0.00172916666666667	-0.00191752943484214\\
-0.00170833333333334	-0.00336176155744997\\
-0.00168750000000001	-0.00392975168128137\\
-0.00166666666666668	-0.00344580559638619\\
-0.00164583333333335	-0.00201461953280879\\
-0.00160416666666666	0.00206694731288176\\
-0.00158333333333333	0.00362716378566968\\
-0.00156249999999999	0.00424413181578387\\
-0.00154166666666666	0.00372519523933643\\
-0.00152083333333333	0.00218020469988894\\
-0.00147916666666667	-0.00224161891678729\\
-0.00145833333333334	-0.00393806353872708\\
-0.00143750000000001	-0.00461318675628683\\
-0.00141666666666668	-0.00405388893692493\\
-0.00139583333333335	-0.00237544691181932\\
-0.00135416666666666	0.00244853758602917\\
-0.00133333333333333	0.00430725699548276\\
-0.00131249999999999	0.00505253787593318\\
-0.00129166666666666	0.00444620076953056\\
-0.00127083333333333	0.00260909742773599\\
-0.00122916666666667	-0.00269754140833722\\
-0.00120833333333334	-0.00475283530536028\\
-0.00118750000000001	-0.00558438396813668\\
-0.00116666666666668	-0.00492257942340885\\
-0.00114583333333335	-0.00289372623803444\\
-0.00110416666666666	0.00300292345456407\\
-0.00108333333333333	0.00530123937905572\\
-0.00106249999999999	0.00624137031732924\\
-0.00104166666666666	0.00551328895421793\\
-0.00102083333333333	0.00324806006309991\\
-0.00097916666666667	-0.00338627538493394\\
-0.000958333333333339	-0.00599270538501948\\
-0.000937500000000008	-0.00707355302630647\\
-0.000916666666666677	-0.00626510108433856\\
-0.000895833333333346	-0.00370127774632314\\
-0.000854166666666656	0.00388182788029012\\
-0.000833333333333325	0.00689161119277243\\
-0.000812499999999994	0.00816179195343053\\
-0.000791666666666663	0.00725432757133937\\
-0.000770833333333332	0.00430148494842961\\
-0.00072916666666667	-0.00454728408833988\\
-0.000708333333333339	-0.00810777787384989\\
-0.000687500000000008	-0.00964575412678154\\
-0.000666666666666677	-0.00861451399096552\\
-0.000645833333333345	-0.00513403042231919\\
-0.000624999999999987	-2.77555756156289e-17\\
-0.000604166666666656	0.0054881014859274\\
-0.000583333333333325	0.00984515884681775\\
-0.000562499999999994	0.0117892550438441\\
-0.000541666666666663	0.0106024787581114\\
-0.000520833333333331	0.00636619772367583\\
-0.0005	-0\\
-0.000479166666666669	-0.00691978013443029\\
-0.000458333333333338	-0.0125302021686771\\
-0.000437500000000007	-0.0151576136277995\\
-0.000416666666666676	-0.0137832223855448\\
-0.000395833333333345	-0.00837657595220501\\
-0.000374999999999986	-5.55111512312578e-17\\
-0.000354166666666655	0.00936205547599384\\
-0.000333333333333324	0.017229027981931\\
-0.000312499999999993	0.0212206590789194\\
-0.000291666666666662	0.0196903176936354\\
-0.000270833333333331	0.0122426879301458\\
-0.00025	-0\\
-0.000229166666666669	-0.0144686311901722\\
-0.000208333333333338	-0.0275664447710896\\
-0.000187500000000007	-0.0353677651315323\\
-0.000166666666666676	-0.034458055963862\\
-0.000145833333333345	-0.0227364204416993\\
-0.000125000000000014	-1.38777878078145e-16\\
-0.000104166666666655	0.0318309886183787\\
-6.24999999999931e-05	0.106103295394597\\
-4.16666666666621e-05	0.137832223855448\\
-2.0833333333331e-05	0.159154943091895\\
0	0.166666666666667\\
2.0833333333331e-05	0.159154943091895\\
4.16666666666621e-05	0.137832223855448\\
6.24999999999931e-05	0.106103295394597\\
0.000104166666666655	0.0318309886183787\\
0.000125000000000014	-1.38777878078145e-16\\
0.000145833333333345	-0.0227364204416993\\
0.000166666666666676	-0.034458055963862\\
0.000187500000000007	-0.0353677651315323\\
0.000208333333333338	-0.0275664447710896\\
0.000229166666666669	-0.0144686311901722\\
0.00025	-0\\
0.000270833333333331	0.0122426879301458\\
0.000291666666666662	0.0196903176936354\\
0.000312499999999993	0.0212206590789194\\
0.000333333333333324	0.017229027981931\\
0.000354166666666655	0.00936205547599384\\
0.000374999999999986	-5.55111512312578e-17\\
0.000395833333333345	-0.00837657595220501\\
0.000416666666666676	-0.0137832223855448\\
0.000437500000000007	-0.0151576136277995\\
0.000458333333333338	-0.0125302021686771\\
0.000479166666666669	-0.00691978013443029\\
0.0005	-0\\
0.000520833333333331	0.00636619772367583\\
0.000541666666666663	0.0106024787581114\\
0.000562499999999994	0.0117892550438441\\
0.000583333333333325	0.00984515884681775\\
0.000604166666666656	0.0054881014859274\\
0.000645833333333345	-0.00513403042231919\\
0.000666666666666677	-0.00861451399096552\\
0.000687500000000008	-0.00964575412678154\\
0.000708333333333339	-0.00810777787384989\\
0.00072916666666667	-0.00454728408833988\\
0.000770833333333332	0.00430148494842961\\
0.000791666666666663	0.00725432757133937\\
0.000812499999999994	0.00816179195343053\\
0.000833333333333325	0.00689161119277243\\
0.000854166666666656	0.00388182788029012\\
0.000895833333333346	-0.00370127774632314\\
0.000916666666666677	-0.00626510108433856\\
0.000937500000000008	-0.00707355302630647\\
0.000958333333333339	-0.00599270538501948\\
0.00097916666666667	-0.00338627538493394\\
0.00102083333333333	0.00324806006309991\\
0.00104166666666666	0.00551328895421793\\
0.00106249999999999	0.00624137031732924\\
0.00108333333333333	0.00530123937905572\\
0.00110416666666666	0.00300292345456407\\
0.00114583333333335	-0.00289372623803444\\
0.00116666666666668	-0.00492257942340885\\
0.00118750000000001	-0.00558438396813668\\
0.00120833333333334	-0.00475283530536028\\
0.00122916666666667	-0.00269754140833722\\
0.00127083333333333	0.00260909742773599\\
0.00129166666666666	0.00444620076953056\\
0.00131249999999999	0.00505253787593318\\
0.00133333333333333	0.00430725699548276\\
0.00135416666666666	0.00244853758602917\\
0.00139583333333335	-0.00237544691181932\\
0.00141666666666668	-0.00405388893692493\\
0.00143750000000001	-0.00461318675628683\\
0.00145833333333334	-0.00393806353872708\\
0.00147916666666667	-0.00224161891678729\\
0.00152083333333333	0.00218020469988894\\
0.00154166666666666	0.00372519523933643\\
0.00156249999999999	0.00424413181578387\\
0.00158333333333333	0.00362716378566968\\
0.00160416666666666	0.00206694731288176\\
0.00164583333333335	-0.00201461953280879\\
0.00166666666666668	-0.00344580559638619\\
0.00168750000000001	-0.00392975168128137\\
0.00170833333333334	-0.00336176155744997\\
0.00172916666666667	-0.00191752943484214\\
0.00177083333333333	0.00187241109519876\\
0.00179166666666666	0.00320540055477786\\
0.00181249999999999	0.00365873432395161\\
0.00183333333333333	0.00313255054216929\\
0.00185416666666666	0.0017882577875494\\
0.00189583333333335	-0.00174895541859224\\
0.00191666666666668	-0.00299635269250972\\
0.00193750000000001	-0.00342268694821279\\
0.00195833333333334	-0.00293260050756272\\
0.00197916666666667	-0.00167531519044101\\
0.00202083333333333	0.00164077260919479\\
0.00204166666666666	0.0028129025276622\\
0.00206249999999999	0.00321525137559384\\
0.00208333333333333	0.00275664447710897\\
};
\end{axis}
\end{tikzpicture}%

\subsubsection{Modification de la fréquence de coupure}

% TODO


\section{Démodulateur de fréquence adapté à la norme V21}

\subsection{Contexte de synchronisation idéale}

Nous allons ici introduire une nouvelle méthode de démodulation plus adaptée à la norme V21.
La multiplication du signal avec le cosinus correspondant (par exemple) au bit 1 donne, en fonction du temps, une mesure de la synchronisation entre le signal et ce cosinus: plus le résultat est proche de 1, plus les signaux sont synchronisés à cet instant, et donc plus le signal est susceptible d'être un bit 1.

\begin{center}
\begin{figure}[H]
	\centering
	\begin{tikzpicture}
        \begin{axis}[
            ymax=1.1, ymin=-1.1,
            ylabel={Amplitude},
            xlabel={t [s]},
            samples=2000,
        ]
        \addplot[domain=0:100, red] {cos(2*pi*x)};
        \addplot[domain=0:100, blue] {cos(4*pi*x)};
        \addplot[domain=0:100, black, ultra thick] {cos(2*pi*x)*cos(4*pi*x)};
        \legend{cos(2t), cos(4t), produit}
        \end{axis}
        \node[] at (axis cs:55,1) {synchronises};
        \node[] at (axis cs:29,1) {desynchronises};
    \end{tikzpicture}
	\caption{Synchronisation entre deux cosinus de fréquences différentes}
	\label{fig:synchronisation}
\end{figure}
\end{center}

On fait ensuite une accumulation (une moyenne en quelque sorte) de ces mesures sur une période $T_s$ en élevant l'intégrale sur $T_s$ au carré, pour avoir une idée de la synchronisation avec le signal d'un bit 1 et d'un bit 0 à chaque instant échantilloné.

% TODO montrer l'intégrale pas juste le produit

\begin{center}
\begin{figure}[H]
	\centering
	\begin{tikzpicture}
        \begin{axis}[
            ymax=1.1, ymin=-1.1,
            ylabel={Amplitude},
            xlabel={t [s]},
            samples=1000,
        ]
        \addplot[domain=0:200, red, thick] {cos(2*pi*x) * (x<30*pi) + cos(4*pi*x) * (x>=30*pi)};
        \addplot[domain=0:200, blue, thick] {cos(4*pi*x)};
        \addplot[domain=0:200, black, ultra thick] {(cos(2*pi*x) * (x<30*pi) + cos(4*pi*x) * (x>=30*pi)) * cos(4*pi*x)};
        \legend{signal, cos(4t), produit}
        \end{axis}
    \end{tikzpicture}
	\caption{Synchronisation entre le signal obtenu et le signal théorique d'un bit}
\end{figure}
\end{center}

Finalement, pour chaque échantillon temporel (de durée $T_s$), on effectue une comparaison:

$$
\operatorname{reconstitué}(kT_s) = \begin{cases}
    1 & \text{si } \operatorname{sync}_1(kT_s) - \operatorname{sync}_0(kT_s) > 0 \\
    0 & \text{sinon}
\end{cases}
$$

En notant $\operatorname{sync}_b$ le produit pour le bit $b$.

Le signal démodulé avec cette méthode est le suivant:

% This file was created by matlab2tikz.
%
%The latest updates can be retrieved from
%  http://www.mathworks.com/matlabcentral/fileexchange/22022-matlab2tikz-matlab2tikz
%where you can also make suggestions and rate matlab2tikz.
%
\definecolor{mycolor1}{rgb}{0.00000,0.44700,0.74100}%
\definecolor{mycolor2}{rgb}{0.85000,0.32500,0.09800}%
%
\begin{tikzpicture}

\begin{axis}[%
width=4.521in,
height=3.566in,
at={(0.758in,0.481in)},
scale only axis,
xmin=0,
xmax=4000,
ymin=-0.1,
ymax=1.1,
axis background/.style={fill=white},
title style={font=\bfseries},
title={Signal démodulé avec synchronisation parfaite},
legend style={legend cell align=left, align=left, draw=white!15!black}
]
\addplot [color=mycolor1, line width=2.0pt]
  table[row sep=crcr]{%
1	0\\
160	0\\
161	1\\
640	1\\
641	0\\
800	0\\
801	1\\
960	1\\
961	0\\
1120	0\\
1121	1\\
1440	1\\
1441	0\\
1600	0\\
1601	1\\
1760	1\\
1761	0\\
2240	0\\
2241	1\\
2400	1\\
2401	0\\
2560	0\\
2561	1\\
3200	1\\
3201	0\\
3360	0\\
3361	1\\
3520	1\\
3521	0\\
3680	0\\
3681	1\\
3840	1\\
3841	0\\
4000	0\\
};
\addlegendentry{Signal d'origine}

\addplot [color=mycolor2, dashed, line width=4.0pt]
  table[row sep=crcr]{%
1	0\\
1760	0\\
1761	1\\
2240	1\\
2241	0\\
4000	0\\
};
\addlegendentry{Signal démodulé}

\end{axis}
\end{tikzpicture}%


\subsection{Gestion d'une erreur de synchronisation de phase porteuse}



% ne pas écrire en dessous

\begin{center}
    \includegraphics[width=0.125\textwidth]{frog.jpg}
\end{center}
\end{document}
