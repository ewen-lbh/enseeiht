\documentclass{article}

% Language setting
% Replace `english' with e.g. `spanish' to change the document language
\usepackage[french]{babel}
\usepackage[utf8]{inputenc}
\newcommand{\TF}{\operatorname{TF}}
\newcommand{\NRZ}{\operatorname{NRZ}}

\usepackage{float}

% Set page size and margins
% Replace `letterpaper' with `a4paper' for UK/EU standard size
\usepackage[a4paper,top=2cm,bottom=2cm,left=3cm,right=3cm,marginparwidth=1.75cm]{geometry}

% Useful packages
\usepackage{siunitx}
\usepackage{amsmath}
\usepackage{pgfplots}
\pgfplotsset{compat=newest}
\usetikzlibrary{plotmarks}
\usetikzlibrary{arrows.meta}
\usepgfplotslibrary{patchplots}
\usepackage{grffile}
\pgfplotsset{plot coordinates/math parser=false}
\newlength\figureheight
\newlength\figurewidth
  
\usepackage{graphicx}
\usepackage{hyperref}
\newcommand{\TF}{\operatorname{TF}}
\newcommand{\sinc}{\operatorname{sinc}}




\title{
\includegraphics[width=0.2\textwidth]{n7.png}
\\[1cm]
Rapport de projet -- Traitement du signal

}
\author{Florent Puy, Ewen Le Bihan}

\date{ENSEEIHT, département Sciences du Numérique}

\begin{document}

\maketitle

\tableofcontents

\section{Introduction}

Dans ce projet, nous implémentons un modem suivant les règles V21 de l'union internationnale des télécommunications (UIT) en Matlab. Nous utiliserons la méthode de la modulation en fréquence numérique.

\setcounter{section}{2}

\section{Modem en fréquence}

Nous allons tout d'abord réaliser un signal NRZ binaire à partir duquel nous construirons ensuite un signal sinusoïdal modulé fréquence $F_0=\SI{6000}{\hertz}$ pour les bits 0 et de fréquence $F_1=\SI{2000}{\hertz}$ pour les bits 1.

Nous comparerons ensuite les densités spectrales de puissance théoriques et expérimentales du signal NRZ et du signal modulé en fréquence.

\subsection{Génération d'un signal NRZ}

\subsection{Signal NRZ}


On génère tout d'abord un signal NRZ prennant deux valeurs, 0 ou 1, générées aléatoirement d'une durée $T_s=1/300 s$. On effectue cela sur $N_s$ périodes. Voici les résultats ainsi obtenus. Pour calculer le signal NRZ depuis un vecteur binaire de taille $1 \times N_\text{bits}$, fait le produit tensoriel de Kronecker entre le vecteur binaire et un vecteur comportant $N_s$ fois le bit 1.

% This file was created by matlab2tikz.
%
%The latest updates can be retrieved from
%  http://www.mathworks.com/matlabcentral/fileexchange/22022-matlab2tikz-matlab2tikz
%where you can also make suggestions and rate matlab2tikz.
%
\definecolor{mycolor1}{rgb}{0.00000,0.44700,0.74100}%
%
\begin{tikzpicture}

\begin{axis}[%
width=4.521in,
height=3.559in,
at={(0.758in,0.488in)},
scale only axis,
xmin=0,
xmax=0.09,
xlabel style={font=\color{white!15!black}},
xlabel={temps [s]},
ymin=-0.1,
ymax=1.1,
ylabel style={font=\color{white!15!black}},
ylabel={bit},
axis background/.style={fill=white},
title style={font=\bfseries},
title={Signal NRZ aléatoire}
]
\addplot [color=mycolor1, forget plot]
  table[row sep=crcr]{%
0	0\\
2.08333333333333e-05	0\\
4.16666666666667e-05	0\\
6.25e-05	0\\
8.33333333333333e-05	0\\
0.000104166666666667	0\\
0.000125	0\\
0.000145833333333333	0\\
0.000166666666666667	0\\
0.0001875	0\\
0.000208333333333333	0\\
0.000229166666666667	0\\
0.00025	0\\
0.000270833333333333	0\\
0.000291666666666667	0\\
0.0003125	0\\
0.000333333333333333	0\\
0.000354166666666667	0\\
0.000375	0\\
0.000395833333333333	0\\
0.000416666666666667	0\\
0.0004375	0\\
0.000458333333333333	0\\
0.000479166666666667	0\\
0.0005	0\\
0.000520833333333333	0\\
0.000541666666666667	0\\
0.0005625	0\\
0.000583333333333333	0\\
0.000604166666666667	0\\
0.000625	0\\
0.000645833333333333	0\\
0.000666666666666667	0\\
0.0006875	0\\
0.000708333333333333	0\\
0.000729166666666667	0\\
0.00075	0\\
0.000770833333333333	0\\
0.000791666666666667	0\\
0.0008125	0\\
0.000833333333333333	0\\
0.000854166666666667	0\\
0.000875	0\\
0.000895833333333333	0\\
0.000916666666666667	0\\
0.0009375	0\\
0.000958333333333333	0\\
0.000979166666666667	0\\
0.001	0\\
0.00102083333333333	0\\
0.00104166666666667	0\\
0.0010625	0\\
0.00108333333333333	0\\
0.00110416666666667	0\\
0.001125	0\\
0.00114583333333333	0\\
0.00116666666666667	0\\
0.0011875	0\\
0.00120833333333333	0\\
0.00122916666666667	0\\
0.00125	0\\
0.00127083333333333	0\\
0.00129166666666667	0\\
0.0013125	0\\
0.00133333333333333	0\\
0.00135416666666667	0\\
0.001375	0\\
0.00139583333333333	0\\
0.00141666666666667	0\\
0.0014375	0\\
0.00145833333333333	0\\
0.00147916666666667	0\\
0.0015	0\\
0.00152083333333333	0\\
0.00154166666666667	0\\
0.0015625	0\\
0.00158333333333333	0\\
0.00160416666666667	0\\
0.001625	0\\
0.00164583333333333	0\\
0.00166666666666667	0\\
0.0016875	0\\
0.00170833333333333	0\\
0.00172916666666667	0\\
0.00175	0\\
0.00177083333333333	0\\
0.00179166666666667	0\\
0.0018125	0\\
0.00183333333333333	0\\
0.00185416666666667	0\\
0.001875	0\\
0.00189583333333333	0\\
0.00191666666666667	0\\
0.0019375	0\\
0.00195833333333333	0\\
0.00197916666666667	0\\
0.002	0\\
0.00202083333333333	0\\
0.00204166666666667	0\\
0.0020625	0\\
0.00208333333333333	0\\
0.00210416666666667	0\\
0.002125	0\\
0.00214583333333333	0\\
0.00216666666666667	0\\
0.0021875	0\\
0.00220833333333333	0\\
0.00222916666666667	0\\
0.00225	0\\
0.00227083333333333	0\\
0.00229166666666667	0\\
0.0023125	0\\
0.00233333333333333	0\\
0.00235416666666667	0\\
0.002375	0\\
0.00239583333333333	0\\
0.00241666666666667	0\\
0.0024375	0\\
0.00245833333333333	0\\
0.00247916666666667	0\\
0.0025	0\\
0.00252083333333333	0\\
0.00254166666666667	0\\
0.0025625	0\\
0.00258333333333333	0\\
0.00260416666666667	0\\
0.002625	0\\
0.00264583333333333	0\\
0.00266666666666667	0\\
0.0026875	0\\
0.00270833333333333	0\\
0.00272916666666667	0\\
0.00275	0\\
0.00277083333333333	0\\
0.00279166666666667	0\\
0.0028125	0\\
0.00283333333333333	0\\
0.00285416666666667	0\\
0.002875	0\\
0.00289583333333333	0\\
0.00291666666666667	0\\
0.0029375	0\\
0.00295833333333333	0\\
0.00297916666666667	0\\
0.003	0\\
0.00302083333333333	0\\
0.00304166666666667	0\\
0.0030625	0\\
0.00308333333333333	0\\
0.00310416666666667	0\\
0.003125	0\\
0.00314583333333333	0\\
0.00316666666666667	0\\
0.0031875	0\\
0.00320833333333333	0\\
0.00322916666666667	0\\
0.00325	0\\
0.00327083333333333	0\\
0.00329166666666667	0\\
0.0033125	0\\
0.00333333333333333	0\\
0.00335416666666667	0\\
0.003375	0\\
0.00339583333333333	0\\
0.00341666666666667	0\\
0.0034375	0\\
0.00345833333333333	0\\
0.00347916666666667	0\\
0.0035	0\\
0.00352083333333333	0\\
0.00354166666666667	0\\
0.0035625	0\\
0.00358333333333333	0\\
0.00360416666666667	0\\
0.003625	0\\
0.00364583333333333	0\\
0.00366666666666667	0\\
0.0036875	0\\
0.00370833333333333	0\\
0.00372916666666667	0\\
0.00375	0\\
0.00377083333333333	0\\
0.00379166666666667	0\\
0.0038125	0\\
0.00383333333333333	0\\
0.00385416666666667	0\\
0.003875	0\\
0.00389583333333333	0\\
0.00391666666666667	0\\
0.0039375	0\\
0.00395833333333333	0\\
0.00397916666666667	0\\
0.004	0\\
0.00402083333333333	0\\
0.00404166666666667	0\\
0.0040625	0\\
0.00408333333333333	0\\
0.00410416666666667	0\\
0.004125	0\\
0.00414583333333333	0\\
0.00416666666666667	0\\
0.0041875	0\\
0.00420833333333333	0\\
0.00422916666666667	0\\
0.00425	0\\
0.00427083333333333	0\\
0.00429166666666667	0\\
0.0043125	0\\
0.00433333333333333	0\\
0.00435416666666667	0\\
0.004375	0\\
0.00439583333333333	0\\
0.00441666666666667	0\\
0.0044375	0\\
0.00445833333333333	0\\
0.00447916666666667	0\\
0.0045	0\\
0.00452083333333333	0\\
0.00454166666666667	0\\
0.0045625	0\\
0.00458333333333333	0\\
0.00460416666666667	0\\
0.004625	0\\
0.00464583333333333	0\\
0.00466666666666667	0\\
0.0046875	0\\
0.00470833333333333	0\\
0.00472916666666667	0\\
0.00475	0\\
0.00477083333333333	0\\
0.00479166666666667	0\\
0.0048125	0\\
0.00483333333333333	0\\
0.00485416666666667	0\\
0.004875	0\\
0.00489583333333333	0\\
0.00491666666666667	0\\
0.0049375	0\\
0.00495833333333333	0\\
0.00497916666666667	0\\
0.005	0\\
0.00502083333333333	0\\
0.00504166666666667	0\\
0.0050625	0\\
0.00508333333333333	0\\
0.00510416666666667	0\\
0.005125	0\\
0.00514583333333333	0\\
0.00516666666666667	0\\
0.0051875	0\\
0.00520833333333333	0\\
0.00522916666666667	0\\
0.00525	0\\
0.00527083333333333	0\\
0.00529166666666667	0\\
0.0053125	0\\
0.00533333333333333	0\\
0.00535416666666667	0\\
0.005375	0\\
0.00539583333333333	0\\
0.00541666666666667	0\\
0.0054375	0\\
0.00545833333333333	0\\
0.00547916666666667	0\\
0.0055	0\\
0.00552083333333333	0\\
0.00554166666666667	0\\
0.0055625	0\\
0.00558333333333333	0\\
0.00560416666666667	0\\
0.005625	0\\
0.00564583333333333	0\\
0.00566666666666667	0\\
0.0056875	0\\
0.00570833333333333	0\\
0.00572916666666667	0\\
0.00575	0\\
0.00577083333333333	0\\
0.00579166666666667	0\\
0.0058125	0\\
0.00583333333333333	0\\
0.00585416666666667	0\\
0.005875	0\\
0.00589583333333333	0\\
0.00591666666666667	0\\
0.0059375	0\\
0.00595833333333333	0\\
0.00597916666666667	0\\
0.006	0\\
0.00602083333333333	0\\
0.00604166666666667	0\\
0.0060625	0\\
0.00608333333333333	0\\
0.00610416666666667	0\\
0.006125	0\\
0.00614583333333333	0\\
0.00616666666666667	0\\
0.0061875	0\\
0.00620833333333333	0\\
0.00622916666666667	0\\
0.00625	0\\
0.00627083333333333	0\\
0.00629166666666667	0\\
0.0063125	0\\
0.00633333333333333	0\\
0.00635416666666667	0\\
0.006375	0\\
0.00639583333333333	0\\
0.00641666666666667	0\\
0.0064375	0\\
0.00645833333333333	0\\
0.00647916666666667	0\\
0.0065	0\\
0.00652083333333333	0\\
0.00654166666666667	0\\
0.0065625	0\\
0.00658333333333333	0\\
0.00660416666666667	0\\
0.006625	0\\
0.00664583333333333	0\\
0.00666666666666667	1\\
0.0066875	1\\
0.00670833333333333	1\\
0.00672916666666667	1\\
0.00675	1\\
0.00677083333333333	1\\
0.00679166666666667	1\\
0.0068125	1\\
0.00683333333333333	1\\
0.00685416666666667	1\\
0.006875	1\\
0.00689583333333333	1\\
0.00691666666666667	1\\
0.0069375	1\\
0.00695833333333333	1\\
0.00697916666666667	1\\
0.007	1\\
0.00702083333333333	1\\
0.00704166666666667	1\\
0.0070625	1\\
0.00708333333333333	1\\
0.00710416666666667	1\\
0.007125	1\\
0.00714583333333333	1\\
0.00716666666666667	1\\
0.0071875	1\\
0.00720833333333333	1\\
0.00722916666666667	1\\
0.00725	1\\
0.00727083333333333	1\\
0.00729166666666667	1\\
0.0073125	1\\
0.00733333333333333	1\\
0.00735416666666667	1\\
0.007375	1\\
0.00739583333333333	1\\
0.00741666666666667	1\\
0.0074375	1\\
0.00745833333333333	1\\
0.00747916666666667	1\\
0.0075	1\\
0.00752083333333333	1\\
0.00754166666666667	1\\
0.0075625	1\\
0.00758333333333333	1\\
0.00760416666666667	1\\
0.007625	1\\
0.00764583333333333	1\\
0.00766666666666667	1\\
0.0076875	1\\
0.00770833333333333	1\\
0.00772916666666667	1\\
0.00775	1\\
0.00777083333333333	1\\
0.00779166666666667	1\\
0.0078125	1\\
0.00783333333333333	1\\
0.00785416666666667	1\\
0.007875	1\\
0.00789583333333333	1\\
0.00791666666666667	1\\
0.0079375	1\\
0.00795833333333333	1\\
0.00797916666666667	1\\
0.008	1\\
0.00802083333333333	1\\
0.00804166666666667	1\\
0.0080625	1\\
0.00808333333333333	1\\
0.00810416666666667	1\\
0.008125	1\\
0.00814583333333333	1\\
0.00816666666666667	1\\
0.0081875	1\\
0.00820833333333333	1\\
0.00822916666666667	1\\
0.00825	1\\
0.00827083333333333	1\\
0.00829166666666667	1\\
0.0083125	1\\
0.00833333333333333	1\\
0.00835416666666667	1\\
0.008375	1\\
0.00839583333333333	1\\
0.00841666666666667	1\\
0.0084375	1\\
0.00845833333333333	1\\
0.00847916666666667	1\\
0.0085	1\\
0.00852083333333333	1\\
0.00854166666666667	1\\
0.0085625	1\\
0.00858333333333333	1\\
0.00860416666666667	1\\
0.008625	1\\
0.00864583333333333	1\\
0.00866666666666667	1\\
0.0086875	1\\
0.00870833333333333	1\\
0.00872916666666667	1\\
0.00875	1\\
0.00877083333333333	1\\
0.00879166666666667	1\\
0.0088125	1\\
0.00883333333333333	1\\
0.00885416666666667	1\\
0.008875	1\\
0.00889583333333333	1\\
0.00891666666666667	1\\
0.0089375	1\\
0.00895833333333333	1\\
0.00897916666666667	1\\
0.009	1\\
0.00902083333333333	1\\
0.00904166666666667	1\\
0.0090625	1\\
0.00908333333333333	1\\
0.00910416666666667	1\\
0.009125	1\\
0.00914583333333333	1\\
0.00916666666666667	1\\
0.0091875	1\\
0.00920833333333333	1\\
0.00922916666666667	1\\
0.00925	1\\
0.00927083333333333	1\\
0.00929166666666667	1\\
0.0093125	1\\
0.00933333333333333	1\\
0.00935416666666667	1\\
0.009375	1\\
0.00939583333333333	1\\
0.00941666666666667	1\\
0.0094375	1\\
0.00945833333333333	1\\
0.00947916666666667	1\\
0.0095	1\\
0.00952083333333333	1\\
0.00954166666666667	1\\
0.0095625	1\\
0.00958333333333333	1\\
0.00960416666666667	1\\
0.009625	1\\
0.00964583333333333	1\\
0.00966666666666667	1\\
0.0096875	1\\
0.00970833333333333	1\\
0.00972916666666667	1\\
0.00975	1\\
0.00977083333333333	1\\
0.00979166666666667	1\\
0.0098125	1\\
0.00983333333333333	1\\
0.00985416666666667	1\\
0.009875	1\\
0.00989583333333333	1\\
0.00991666666666667	1\\
0.0099375	1\\
0.00995833333333333	1\\
0.00997916666666667	1\\
0.01	1\\
0.0100208333333333	1\\
0.0100416666666667	1\\
0.0100625	1\\
0.0100833333333333	1\\
0.0101041666666667	1\\
0.010125	1\\
0.0101458333333333	1\\
0.0101666666666667	1\\
0.0101875	1\\
0.0102083333333333	1\\
0.0102291666666667	1\\
0.01025	1\\
0.0102708333333333	1\\
0.0102916666666667	1\\
0.0103125	1\\
0.0103333333333333	1\\
0.0103541666666667	1\\
0.010375	1\\
0.0103958333333333	1\\
0.0104166666666667	1\\
0.0104375	1\\
0.0104583333333333	1\\
0.0104791666666667	1\\
0.0105	1\\
0.0105208333333333	1\\
0.0105416666666667	1\\
0.0105625	1\\
0.0105833333333333	1\\
0.0106041666666667	1\\
0.010625	1\\
0.0106458333333333	1\\
0.0106666666666667	1\\
0.0106875	1\\
0.0107083333333333	1\\
0.0107291666666667	1\\
0.01075	1\\
0.0107708333333333	1\\
0.0107916666666667	1\\
0.0108125	1\\
0.0108333333333333	1\\
0.0108541666666667	1\\
0.010875	1\\
0.0108958333333333	1\\
0.0109166666666667	1\\
0.0109375	1\\
0.0109583333333333	1\\
0.0109791666666667	1\\
0.011	1\\
0.0110208333333333	1\\
0.0110416666666667	1\\
0.0110625	1\\
0.0110833333333333	1\\
0.0111041666666667	1\\
0.011125	1\\
0.0111458333333333	1\\
0.0111666666666667	1\\
0.0111875	1\\
0.0112083333333333	1\\
0.0112291666666667	1\\
0.01125	1\\
0.0112708333333333	1\\
0.0112916666666667	1\\
0.0113125	1\\
0.0113333333333333	1\\
0.0113541666666667	1\\
0.011375	1\\
0.0113958333333333	1\\
0.0114166666666667	1\\
0.0114375	1\\
0.0114583333333333	1\\
0.0114791666666667	1\\
0.0115	1\\
0.0115208333333333	1\\
0.0115416666666667	1\\
0.0115625	1\\
0.0115833333333333	1\\
0.0116041666666667	1\\
0.011625	1\\
0.0116458333333333	1\\
0.0116666666666667	1\\
0.0116875	1\\
0.0117083333333333	1\\
0.0117291666666667	1\\
0.01175	1\\
0.0117708333333333	1\\
0.0117916666666667	1\\
0.0118125	1\\
0.0118333333333333	1\\
0.0118541666666667	1\\
0.011875	1\\
0.0118958333333333	1\\
0.0119166666666667	1\\
0.0119375	1\\
0.0119583333333333	1\\
0.0119791666666667	1\\
0.012	1\\
0.0120208333333333	1\\
0.0120416666666667	1\\
0.0120625	1\\
0.0120833333333333	1\\
0.0121041666666667	1\\
0.012125	1\\
0.0121458333333333	1\\
0.0121666666666667	1\\
0.0121875	1\\
0.0122083333333333	1\\
0.0122291666666667	1\\
0.01225	1\\
0.0122708333333333	1\\
0.0122916666666667	1\\
0.0123125	1\\
0.0123333333333333	1\\
0.0123541666666667	1\\
0.012375	1\\
0.0123958333333333	1\\
0.0124166666666667	1\\
0.0124375	1\\
0.0124583333333333	1\\
0.0124791666666667	1\\
0.0125	1\\
0.0125208333333333	1\\
0.0125416666666667	1\\
0.0125625	1\\
0.0125833333333333	1\\
0.0126041666666667	1\\
0.012625	1\\
0.0126458333333333	1\\
0.0126666666666667	1\\
0.0126875	1\\
0.0127083333333333	1\\
0.0127291666666667	1\\
0.01275	1\\
0.0127708333333333	1\\
0.0127916666666667	1\\
0.0128125	1\\
0.0128333333333333	1\\
0.0128541666666667	1\\
0.012875	1\\
0.0128958333333333	1\\
0.0129166666666667	1\\
0.0129375	1\\
0.0129583333333333	1\\
0.0129791666666667	1\\
0.013	1\\
0.0130208333333333	1\\
0.0130416666666667	1\\
0.0130625	1\\
0.0130833333333333	1\\
0.0131041666666667	1\\
0.013125	1\\
0.0131458333333333	1\\
0.0131666666666667	1\\
0.0131875	1\\
0.0132083333333333	1\\
0.0132291666666667	1\\
0.01325	1\\
0.0132708333333333	1\\
0.0132916666666667	1\\
0.0133125	1\\
0.0133333333333333	1\\
0.0133541666666667	1\\
0.013375	1\\
0.0133958333333333	1\\
0.0134166666666667	1\\
0.0134375	1\\
0.0134583333333333	1\\
0.0134791666666667	1\\
0.0135	1\\
0.0135208333333333	1\\
0.0135416666666667	1\\
0.0135625	1\\
0.0135833333333333	1\\
0.0136041666666667	1\\
0.013625	1\\
0.0136458333333333	1\\
0.0136666666666667	1\\
0.0136875	1\\
0.0137083333333333	1\\
0.0137291666666667	1\\
0.01375	1\\
0.0137708333333333	1\\
0.0137916666666667	1\\
0.0138125	1\\
0.0138333333333333	1\\
0.0138541666666667	1\\
0.013875	1\\
0.0138958333333333	1\\
0.0139166666666667	1\\
0.0139375	1\\
0.0139583333333333	1\\
0.0139791666666667	1\\
0.014	1\\
0.0140208333333333	1\\
0.0140416666666667	1\\
0.0140625	1\\
0.0140833333333333	1\\
0.0141041666666667	1\\
0.014125	1\\
0.0141458333333333	1\\
0.0141666666666667	1\\
0.0141875	1\\
0.0142083333333333	1\\
0.0142291666666667	1\\
0.01425	1\\
0.0142708333333333	1\\
0.0142916666666667	1\\
0.0143125	1\\
0.0143333333333333	1\\
0.0143541666666667	1\\
0.014375	1\\
0.0143958333333333	1\\
0.0144166666666667	1\\
0.0144375	1\\
0.0144583333333333	1\\
0.0144791666666667	1\\
0.0145	1\\
0.0145208333333333	1\\
0.0145416666666667	1\\
0.0145625	1\\
0.0145833333333333	1\\
0.0146041666666667	1\\
0.014625	1\\
0.0146458333333333	1\\
0.0146666666666667	1\\
0.0146875	1\\
0.0147083333333333	1\\
0.0147291666666667	1\\
0.01475	1\\
0.0147708333333333	1\\
0.0147916666666667	1\\
0.0148125	1\\
0.0148333333333333	1\\
0.0148541666666667	1\\
0.014875	1\\
0.0148958333333333	1\\
0.0149166666666667	1\\
0.0149375	1\\
0.0149583333333333	1\\
0.0149791666666667	1\\
0.015	1\\
0.0150208333333333	1\\
0.0150416666666667	1\\
0.0150625	1\\
0.0150833333333333	1\\
0.0151041666666667	1\\
0.015125	1\\
0.0151458333333333	1\\
0.0151666666666667	1\\
0.0151875	1\\
0.0152083333333333	1\\
0.0152291666666667	1\\
0.01525	1\\
0.0152708333333333	1\\
0.0152916666666667	1\\
0.0153125	1\\
0.0153333333333333	1\\
0.0153541666666667	1\\
0.015375	1\\
0.0153958333333333	1\\
0.0154166666666667	1\\
0.0154375	1\\
0.0154583333333333	1\\
0.0154791666666667	1\\
0.0155	1\\
0.0155208333333333	1\\
0.0155416666666667	1\\
0.0155625	1\\
0.0155833333333333	1\\
0.0156041666666667	1\\
0.015625	1\\
0.0156458333333333	1\\
0.0156666666666667	1\\
0.0156875	1\\
0.0157083333333333	1\\
0.0157291666666667	1\\
0.01575	1\\
0.0157708333333333	1\\
0.0157916666666667	1\\
0.0158125	1\\
0.0158333333333333	1\\
0.0158541666666667	1\\
0.015875	1\\
0.0158958333333333	1\\
0.0159166666666667	1\\
0.0159375	1\\
0.0159583333333333	1\\
0.0159791666666667	1\\
0.016	1\\
0.0160208333333333	1\\
0.0160416666666667	1\\
0.0160625	1\\
0.0160833333333333	1\\
0.0161041666666667	1\\
0.016125	1\\
0.0161458333333333	1\\
0.0161666666666667	1\\
0.0161875	1\\
0.0162083333333333	1\\
0.0162291666666667	1\\
0.01625	1\\
0.0162708333333333	1\\
0.0162916666666667	1\\
0.0163125	1\\
0.0163333333333333	1\\
0.0163541666666667	1\\
0.016375	1\\
0.0163958333333333	1\\
0.0164166666666667	1\\
0.0164375	1\\
0.0164583333333333	1\\
0.0164791666666667	1\\
0.0165	1\\
0.0165208333333333	1\\
0.0165416666666667	1\\
0.0165625	1\\
0.0165833333333333	1\\
0.0166041666666667	1\\
0.016625	1\\
0.0166458333333333	1\\
0.0166666666666667	1\\
0.0166875	1\\
0.0167083333333333	1\\
0.0167291666666667	1\\
0.01675	1\\
0.0167708333333333	1\\
0.0167916666666667	1\\
0.0168125	1\\
0.0168333333333333	1\\
0.0168541666666667	1\\
0.016875	1\\
0.0168958333333333	1\\
0.0169166666666667	1\\
0.0169375	1\\
0.0169583333333333	1\\
0.0169791666666667	1\\
0.017	1\\
0.0170208333333333	1\\
0.0170416666666667	1\\
0.0170625	1\\
0.0170833333333333	1\\
0.0171041666666667	1\\
0.017125	1\\
0.0171458333333333	1\\
0.0171666666666667	1\\
0.0171875	1\\
0.0172083333333333	1\\
0.0172291666666667	1\\
0.01725	1\\
0.0172708333333333	1\\
0.0172916666666667	1\\
0.0173125	1\\
0.0173333333333333	1\\
0.0173541666666667	1\\
0.017375	1\\
0.0173958333333333	1\\
0.0174166666666667	1\\
0.0174375	1\\
0.0174583333333333	1\\
0.0174791666666667	1\\
0.0175	1\\
0.0175208333333333	1\\
0.0175416666666667	1\\
0.0175625	1\\
0.0175833333333333	1\\
0.0176041666666667	1\\
0.017625	1\\
0.0176458333333333	1\\
0.0176666666666667	1\\
0.0176875	1\\
0.0177083333333333	1\\
0.0177291666666667	1\\
0.01775	1\\
0.0177708333333333	1\\
0.0177916666666667	1\\
0.0178125	1\\
0.0178333333333333	1\\
0.0178541666666667	1\\
0.017875	1\\
0.0178958333333333	1\\
0.0179166666666667	1\\
0.0179375	1\\
0.0179583333333333	1\\
0.0179791666666667	1\\
0.018	1\\
0.0180208333333333	1\\
0.0180416666666667	1\\
0.0180625	1\\
0.0180833333333333	1\\
0.0181041666666667	1\\
0.018125	1\\
0.0181458333333333	1\\
0.0181666666666667	1\\
0.0181875	1\\
0.0182083333333333	1\\
0.0182291666666667	1\\
0.01825	1\\
0.0182708333333333	1\\
0.0182916666666667	1\\
0.0183125	1\\
0.0183333333333333	1\\
0.0183541666666667	1\\
0.018375	1\\
0.0183958333333333	1\\
0.0184166666666667	1\\
0.0184375	1\\
0.0184583333333333	1\\
0.0184791666666667	1\\
0.0185	1\\
0.0185208333333333	1\\
0.0185416666666667	1\\
0.0185625	1\\
0.0185833333333333	1\\
0.0186041666666667	1\\
0.018625	1\\
0.0186458333333333	1\\
0.0186666666666667	1\\
0.0186875	1\\
0.0187083333333333	1\\
0.0187291666666667	1\\
0.01875	1\\
0.0187708333333333	1\\
0.0187916666666667	1\\
0.0188125	1\\
0.0188333333333333	1\\
0.0188541666666667	1\\
0.018875	1\\
0.0188958333333333	1\\
0.0189166666666667	1\\
0.0189375	1\\
0.0189583333333333	1\\
0.0189791666666667	1\\
0.019	1\\
0.0190208333333333	1\\
0.0190416666666667	1\\
0.0190625	1\\
0.0190833333333333	1\\
0.0191041666666667	1\\
0.019125	1\\
0.0191458333333333	1\\
0.0191666666666667	1\\
0.0191875	1\\
0.0192083333333333	1\\
0.0192291666666667	1\\
0.01925	1\\
0.0192708333333333	1\\
0.0192916666666667	1\\
0.0193125	1\\
0.0193333333333333	1\\
0.0193541666666667	1\\
0.019375	1\\
0.0193958333333333	1\\
0.0194166666666667	1\\
0.0194375	1\\
0.0194583333333333	1\\
0.0194791666666667	1\\
0.0195	1\\
0.0195208333333333	1\\
0.0195416666666667	1\\
0.0195625	1\\
0.0195833333333333	1\\
0.0196041666666667	1\\
0.019625	1\\
0.0196458333333333	1\\
0.0196666666666667	1\\
0.0196875	1\\
0.0197083333333333	1\\
0.0197291666666667	1\\
0.01975	1\\
0.0197708333333333	1\\
0.0197916666666667	1\\
0.0198125	1\\
0.0198333333333333	1\\
0.0198541666666667	1\\
0.019875	1\\
0.0198958333333333	1\\
0.0199166666666667	1\\
0.0199375	1\\
0.0199583333333333	1\\
0.0199791666666667	1\\
0.02	1\\
0.0200208333333333	1\\
0.0200416666666667	1\\
0.0200625	1\\
0.0200833333333333	1\\
0.0201041666666667	1\\
0.020125	1\\
0.0201458333333333	1\\
0.0201666666666667	1\\
0.0201875	1\\
0.0202083333333333	1\\
0.0202291666666667	1\\
0.02025	1\\
0.0202708333333333	1\\
0.0202916666666667	1\\
0.0203125	1\\
0.0203333333333333	1\\
0.0203541666666667	1\\
0.020375	1\\
0.0203958333333333	1\\
0.0204166666666667	1\\
0.0204375	1\\
0.0204583333333333	1\\
0.0204791666666667	1\\
0.0205	1\\
0.0205208333333333	1\\
0.0205416666666667	1\\
0.0205625	1\\
0.0205833333333333	1\\
0.0206041666666667	1\\
0.020625	1\\
0.0206458333333333	1\\
0.0206666666666667	1\\
0.0206875	1\\
0.0207083333333333	1\\
0.0207291666666667	1\\
0.02075	1\\
0.0207708333333333	1\\
0.0207916666666667	1\\
0.0208125	1\\
0.0208333333333333	1\\
0.0208541666666667	1\\
0.020875	1\\
0.0208958333333333	1\\
0.0209166666666667	1\\
0.0209375	1\\
0.0209583333333333	1\\
0.0209791666666667	1\\
0.021	1\\
0.0210208333333333	1\\
0.0210416666666667	1\\
0.0210625	1\\
0.0210833333333333	1\\
0.0211041666666667	1\\
0.021125	1\\
0.0211458333333333	1\\
0.0211666666666667	1\\
0.0211875	1\\
0.0212083333333333	1\\
0.0212291666666667	1\\
0.02125	1\\
0.0212708333333333	1\\
0.0212916666666667	1\\
0.0213125	1\\
0.0213333333333333	1\\
0.0213541666666667	1\\
0.021375	1\\
0.0213958333333333	1\\
0.0214166666666667	1\\
0.0214375	1\\
0.0214583333333333	1\\
0.0214791666666667	1\\
0.0215	1\\
0.0215208333333333	1\\
0.0215416666666667	1\\
0.0215625	1\\
0.0215833333333333	1\\
0.0216041666666667	1\\
0.021625	1\\
0.0216458333333333	1\\
0.0216666666666667	1\\
0.0216875	1\\
0.0217083333333333	1\\
0.0217291666666667	1\\
0.02175	1\\
0.0217708333333333	1\\
0.0217916666666667	1\\
0.0218125	1\\
0.0218333333333333	1\\
0.0218541666666667	1\\
0.021875	1\\
0.0218958333333333	1\\
0.0219166666666667	1\\
0.0219375	1\\
0.0219583333333333	1\\
0.0219791666666667	1\\
0.022	1\\
0.0220208333333333	1\\
0.0220416666666667	1\\
0.0220625	1\\
0.0220833333333333	1\\
0.0221041666666667	1\\
0.022125	1\\
0.0221458333333333	1\\
0.0221666666666667	1\\
0.0221875	1\\
0.0222083333333333	1\\
0.0222291666666667	1\\
0.02225	1\\
0.0222708333333333	1\\
0.0222916666666667	1\\
0.0223125	1\\
0.0223333333333333	1\\
0.0223541666666667	1\\
0.022375	1\\
0.0223958333333333	1\\
0.0224166666666667	1\\
0.0224375	1\\
0.0224583333333333	1\\
0.0224791666666667	1\\
0.0225	1\\
0.0225208333333333	1\\
0.0225416666666667	1\\
0.0225625	1\\
0.0225833333333333	1\\
0.0226041666666667	1\\
0.022625	1\\
0.0226458333333333	1\\
0.0226666666666667	1\\
0.0226875	1\\
0.0227083333333333	1\\
0.0227291666666667	1\\
0.02275	1\\
0.0227708333333333	1\\
0.0227916666666667	1\\
0.0228125	1\\
0.0228333333333333	1\\
0.0228541666666667	1\\
0.022875	1\\
0.0228958333333333	1\\
0.0229166666666667	1\\
0.0229375	1\\
0.0229583333333333	1\\
0.0229791666666667	1\\
0.023	1\\
0.0230208333333333	1\\
0.0230416666666667	1\\
0.0230625	1\\
0.0230833333333333	1\\
0.0231041666666667	1\\
0.023125	1\\
0.0231458333333333	1\\
0.0231666666666667	1\\
0.0231875	1\\
0.0232083333333333	1\\
0.0232291666666667	1\\
0.02325	1\\
0.0232708333333333	1\\
0.0232916666666667	1\\
0.0233125	1\\
0.0233333333333333	0\\
0.0233541666666667	0\\
0.023375	0\\
0.0233958333333333	0\\
0.0234166666666667	0\\
0.0234375	0\\
0.0234583333333333	0\\
0.0234791666666667	0\\
0.0235	0\\
0.0235208333333333	0\\
0.0235416666666667	0\\
0.0235625	0\\
0.0235833333333333	0\\
0.0236041666666667	0\\
0.023625	0\\
0.0236458333333333	0\\
0.0236666666666667	0\\
0.0236875	0\\
0.0237083333333333	0\\
0.0237291666666667	0\\
0.02375	0\\
0.0237708333333333	0\\
0.0237916666666667	0\\
0.0238125	0\\
0.0238333333333333	0\\
0.0238541666666667	0\\
0.023875	0\\
0.0238958333333333	0\\
0.0239166666666667	0\\
0.0239375	0\\
0.0239583333333333	0\\
0.0239791666666667	0\\
0.024	0\\
0.0240208333333333	0\\
0.0240416666666667	0\\
0.0240625	0\\
0.0240833333333333	0\\
0.0241041666666667	0\\
0.024125	0\\
0.0241458333333333	0\\
0.0241666666666667	0\\
0.0241875	0\\
0.0242083333333333	0\\
0.0242291666666667	0\\
0.02425	0\\
0.0242708333333333	0\\
0.0242916666666667	0\\
0.0243125	0\\
0.0243333333333333	0\\
0.0243541666666667	0\\
0.024375	0\\
0.0243958333333333	0\\
0.0244166666666667	0\\
0.0244375	0\\
0.0244583333333333	0\\
0.0244791666666667	0\\
0.0245	0\\
0.0245208333333333	0\\
0.0245416666666667	0\\
0.0245625	0\\
0.0245833333333333	0\\
0.0246041666666667	0\\
0.024625	0\\
0.0246458333333333	0\\
0.0246666666666667	0\\
0.0246875	0\\
0.0247083333333333	0\\
0.0247291666666667	0\\
0.02475	0\\
0.0247708333333333	0\\
0.0247916666666667	0\\
0.0248125	0\\
0.0248333333333333	0\\
0.0248541666666667	0\\
0.024875	0\\
0.0248958333333333	0\\
0.0249166666666667	0\\
0.0249375	0\\
0.0249583333333333	0\\
0.0249791666666667	0\\
0.025	0\\
0.0250208333333333	0\\
0.0250416666666667	0\\
0.0250625	0\\
0.0250833333333333	0\\
0.0251041666666667	0\\
0.025125	0\\
0.0251458333333333	0\\
0.0251666666666667	0\\
0.0251875	0\\
0.0252083333333333	0\\
0.0252291666666667	0\\
0.02525	0\\
0.0252708333333333	0\\
0.0252916666666667	0\\
0.0253125	0\\
0.0253333333333333	0\\
0.0253541666666667	0\\
0.025375	0\\
0.0253958333333333	0\\
0.0254166666666667	0\\
0.0254375	0\\
0.0254583333333333	0\\
0.0254791666666667	0\\
0.0255	0\\
0.0255208333333333	0\\
0.0255416666666667	0\\
0.0255625	0\\
0.0255833333333333	0\\
0.0256041666666667	0\\
0.025625	0\\
0.0256458333333333	0\\
0.0256666666666667	0\\
0.0256875	0\\
0.0257083333333333	0\\
0.0257291666666667	0\\
0.02575	0\\
0.0257708333333333	0\\
0.0257916666666667	0\\
0.0258125	0\\
0.0258333333333333	0\\
0.0258541666666667	0\\
0.025875	0\\
0.0258958333333333	0\\
0.0259166666666667	0\\
0.0259375	0\\
0.0259583333333333	0\\
0.0259791666666667	0\\
0.026	0\\
0.0260208333333333	0\\
0.0260416666666667	0\\
0.0260625	0\\
0.0260833333333333	0\\
0.0261041666666667	0\\
0.026125	0\\
0.0261458333333333	0\\
0.0261666666666667	0\\
0.0261875	0\\
0.0262083333333333	0\\
0.0262291666666667	0\\
0.02625	0\\
0.0262708333333333	0\\
0.0262916666666667	0\\
0.0263125	0\\
0.0263333333333333	0\\
0.0263541666666667	0\\
0.026375	0\\
0.0263958333333333	0\\
0.0264166666666667	0\\
0.0264375	0\\
0.0264583333333333	0\\
0.0264791666666667	0\\
0.0265	0\\
0.0265208333333333	0\\
0.0265416666666667	0\\
0.0265625	0\\
0.0265833333333333	0\\
0.0266041666666667	0\\
0.026625	0\\
0.0266458333333333	0\\
0.0266666666666667	0\\
0.0266875	0\\
0.0267083333333333	0\\
0.0267291666666667	0\\
0.02675	0\\
0.0267708333333333	0\\
0.0267916666666667	0\\
0.0268125	0\\
0.0268333333333333	0\\
0.0268541666666667	0\\
0.026875	0\\
0.0268958333333333	0\\
0.0269166666666667	0\\
0.0269375	0\\
0.0269583333333333	0\\
0.0269791666666667	0\\
0.027	0\\
0.0270208333333333	0\\
0.0270416666666667	0\\
0.0270625	0\\
0.0270833333333333	0\\
0.0271041666666667	0\\
0.027125	0\\
0.0271458333333333	0\\
0.0271666666666667	0\\
0.0271875	0\\
0.0272083333333333	0\\
0.0272291666666667	0\\
0.02725	0\\
0.0272708333333333	0\\
0.0272916666666667	0\\
0.0273125	0\\
0.0273333333333333	0\\
0.0273541666666667	0\\
0.027375	0\\
0.0273958333333333	0\\
0.0274166666666667	0\\
0.0274375	0\\
0.0274583333333333	0\\
0.0274791666666667	0\\
0.0275	0\\
0.0275208333333333	0\\
0.0275416666666667	0\\
0.0275625	0\\
0.0275833333333333	0\\
0.0276041666666667	0\\
0.027625	0\\
0.0276458333333333	0\\
0.0276666666666667	0\\
0.0276875	0\\
0.0277083333333333	0\\
0.0277291666666667	0\\
0.02775	0\\
0.0277708333333333	0\\
0.0277916666666667	0\\
0.0278125	0\\
0.0278333333333333	0\\
0.0278541666666667	0\\
0.027875	0\\
0.0278958333333333	0\\
0.0279166666666667	0\\
0.0279375	0\\
0.0279583333333333	0\\
0.0279791666666667	0\\
0.028	0\\
0.0280208333333333	0\\
0.0280416666666667	0\\
0.0280625	0\\
0.0280833333333333	0\\
0.0281041666666667	0\\
0.028125	0\\
0.0281458333333333	0\\
0.0281666666666667	0\\
0.0281875	0\\
0.0282083333333333	0\\
0.0282291666666667	0\\
0.02825	0\\
0.0282708333333333	0\\
0.0282916666666667	0\\
0.0283125	0\\
0.0283333333333333	0\\
0.0283541666666667	0\\
0.028375	0\\
0.0283958333333333	0\\
0.0284166666666667	0\\
0.0284375	0\\
0.0284583333333333	0\\
0.0284791666666667	0\\
0.0285	0\\
0.0285208333333333	0\\
0.0285416666666667	0\\
0.0285625	0\\
0.0285833333333333	0\\
0.0286041666666667	0\\
0.028625	0\\
0.0286458333333333	0\\
0.0286666666666667	0\\
0.0286875	0\\
0.0287083333333333	0\\
0.0287291666666667	0\\
0.02875	0\\
0.0287708333333333	0\\
0.0287916666666667	0\\
0.0288125	0\\
0.0288333333333333	0\\
0.0288541666666667	0\\
0.028875	0\\
0.0288958333333333	0\\
0.0289166666666667	0\\
0.0289375	0\\
0.0289583333333333	0\\
0.0289791666666667	0\\
0.029	0\\
0.0290208333333333	0\\
0.0290416666666667	0\\
0.0290625	0\\
0.0290833333333333	0\\
0.0291041666666667	0\\
0.029125	0\\
0.0291458333333333	0\\
0.0291666666666667	0\\
0.0291875	0\\
0.0292083333333333	0\\
0.0292291666666667	0\\
0.02925	0\\
0.0292708333333333	0\\
0.0292916666666667	0\\
0.0293125	0\\
0.0293333333333333	0\\
0.0293541666666667	0\\
0.029375	0\\
0.0293958333333333	0\\
0.0294166666666667	0\\
0.0294375	0\\
0.0294583333333333	0\\
0.0294791666666667	0\\
0.0295	0\\
0.0295208333333333	0\\
0.0295416666666667	0\\
0.0295625	0\\
0.0295833333333333	0\\
0.0296041666666667	0\\
0.029625	0\\
0.0296458333333333	0\\
0.0296666666666667	0\\
0.0296875	0\\
0.0297083333333333	0\\
0.0297291666666667	0\\
0.02975	0\\
0.0297708333333333	0\\
0.0297916666666667	0\\
0.0298125	0\\
0.0298333333333333	0\\
0.0298541666666667	0\\
0.029875	0\\
0.0298958333333333	0\\
0.0299166666666667	0\\
0.0299375	0\\
0.0299583333333333	0\\
0.0299791666666667	0\\
0.03	0\\
0.0300208333333333	0\\
0.0300416666666667	0\\
0.0300625	0\\
0.0300833333333333	0\\
0.0301041666666667	0\\
0.030125	0\\
0.0301458333333333	0\\
0.0301666666666667	0\\
0.0301875	0\\
0.0302083333333333	0\\
0.0302291666666667	0\\
0.03025	0\\
0.0302708333333333	0\\
0.0302916666666667	0\\
0.0303125	0\\
0.0303333333333333	0\\
0.0303541666666667	0\\
0.030375	0\\
0.0303958333333333	0\\
0.0304166666666667	0\\
0.0304375	0\\
0.0304583333333333	0\\
0.0304791666666667	0\\
0.0305	0\\
0.0305208333333333	0\\
0.0305416666666667	0\\
0.0305625	0\\
0.0305833333333333	0\\
0.0306041666666667	0\\
0.030625	0\\
0.0306458333333333	0\\
0.0306666666666667	0\\
0.0306875	0\\
0.0307083333333333	0\\
0.0307291666666667	0\\
0.03075	0\\
0.0307708333333333	0\\
0.0307916666666667	0\\
0.0308125	0\\
0.0308333333333333	0\\
0.0308541666666667	0\\
0.030875	0\\
0.0308958333333333	0\\
0.0309166666666667	0\\
0.0309375	0\\
0.0309583333333333	0\\
0.0309791666666667	0\\
0.031	0\\
0.0310208333333333	0\\
0.0310416666666667	0\\
0.0310625	0\\
0.0310833333333333	0\\
0.0311041666666667	0\\
0.031125	0\\
0.0311458333333333	0\\
0.0311666666666667	0\\
0.0311875	0\\
0.0312083333333333	0\\
0.0312291666666667	0\\
0.03125	0\\
0.0312708333333333	0\\
0.0312916666666667	0\\
0.0313125	0\\
0.0313333333333333	0\\
0.0313541666666667	0\\
0.031375	0\\
0.0313958333333333	0\\
0.0314166666666667	0\\
0.0314375	0\\
0.0314583333333333	0\\
0.0314791666666667	0\\
0.0315	0\\
0.0315208333333333	0\\
0.0315416666666667	0\\
0.0315625	0\\
0.0315833333333333	0\\
0.0316041666666667	0\\
0.031625	0\\
0.0316458333333333	0\\
0.0316666666666667	0\\
0.0316875	0\\
0.0317083333333333	0\\
0.0317291666666667	0\\
0.03175	0\\
0.0317708333333333	0\\
0.0317916666666667	0\\
0.0318125	0\\
0.0318333333333333	0\\
0.0318541666666667	0\\
0.031875	0\\
0.0318958333333333	0\\
0.0319166666666667	0\\
0.0319375	0\\
0.0319583333333333	0\\
0.0319791666666667	0\\
0.032	0\\
0.0320208333333333	0\\
0.0320416666666667	0\\
0.0320625	0\\
0.0320833333333333	0\\
0.0321041666666667	0\\
0.032125	0\\
0.0321458333333333	0\\
0.0321666666666667	0\\
0.0321875	0\\
0.0322083333333333	0\\
0.0322291666666667	0\\
0.03225	0\\
0.0322708333333333	0\\
0.0322916666666667	0\\
0.0323125	0\\
0.0323333333333333	0\\
0.0323541666666667	0\\
0.032375	0\\
0.0323958333333333	0\\
0.0324166666666667	0\\
0.0324375	0\\
0.0324583333333333	0\\
0.0324791666666667	0\\
0.0325	0\\
0.0325208333333333	0\\
0.0325416666666667	0\\
0.0325625	0\\
0.0325833333333333	0\\
0.0326041666666667	0\\
0.032625	0\\
0.0326458333333333	0\\
0.0326666666666667	0\\
0.0326875	0\\
0.0327083333333333	0\\
0.0327291666666667	0\\
0.03275	0\\
0.0327708333333333	0\\
0.0327916666666667	0\\
0.0328125	0\\
0.0328333333333333	0\\
0.0328541666666667	0\\
0.032875	0\\
0.0328958333333333	0\\
0.0329166666666667	0\\
0.0329375	0\\
0.0329583333333333	0\\
0.0329791666666667	0\\
0.033	0\\
0.0330208333333333	0\\
0.0330416666666667	0\\
0.0330625	0\\
0.0330833333333333	0\\
0.0331041666666667	0\\
0.033125	0\\
0.0331458333333333	0\\
0.0331666666666667	0\\
0.0331875	0\\
0.0332083333333333	0\\
0.0332291666666667	0\\
0.03325	0\\
0.0332708333333333	0\\
0.0332916666666667	0\\
0.0333125	0\\
0.0333333333333333	0\\
0.0333541666666667	0\\
0.033375	0\\
0.0333958333333333	0\\
0.0334166666666667	0\\
0.0334375	0\\
0.0334583333333333	0\\
0.0334791666666667	0\\
0.0335	0\\
0.0335208333333333	0\\
0.0335416666666667	0\\
0.0335625	0\\
0.0335833333333333	0\\
0.0336041666666667	0\\
0.033625	0\\
0.0336458333333333	0\\
0.0336666666666667	0\\
0.0336875	0\\
0.0337083333333333	0\\
0.0337291666666667	0\\
0.03375	0\\
0.0337708333333333	0\\
0.0337916666666667	0\\
0.0338125	0\\
0.0338333333333333	0\\
0.0338541666666667	0\\
0.033875	0\\
0.0338958333333333	0\\
0.0339166666666667	0\\
0.0339375	0\\
0.0339583333333333	0\\
0.0339791666666667	0\\
0.034	0\\
0.0340208333333333	0\\
0.0340416666666667	0\\
0.0340625	0\\
0.0340833333333333	0\\
0.0341041666666667	0\\
0.034125	0\\
0.0341458333333333	0\\
0.0341666666666667	0\\
0.0341875	0\\
0.0342083333333333	0\\
0.0342291666666667	0\\
0.03425	0\\
0.0342708333333333	0\\
0.0342916666666667	0\\
0.0343125	0\\
0.0343333333333333	0\\
0.0343541666666667	0\\
0.034375	0\\
0.0343958333333333	0\\
0.0344166666666667	0\\
0.0344375	0\\
0.0344583333333333	0\\
0.0344791666666667	0\\
0.0345	0\\
0.0345208333333333	0\\
0.0345416666666667	0\\
0.0345625	0\\
0.0345833333333333	0\\
0.0346041666666667	0\\
0.034625	0\\
0.0346458333333333	0\\
0.0346666666666667	0\\
0.0346875	0\\
0.0347083333333333	0\\
0.0347291666666667	0\\
0.03475	0\\
0.0347708333333333	0\\
0.0347916666666667	0\\
0.0348125	0\\
0.0348333333333333	0\\
0.0348541666666667	0\\
0.034875	0\\
0.0348958333333333	0\\
0.0349166666666667	0\\
0.0349375	0\\
0.0349583333333333	0\\
0.0349791666666667	0\\
0.035	0\\
0.0350208333333333	0\\
0.0350416666666667	0\\
0.0350625	0\\
0.0350833333333333	0\\
0.0351041666666667	0\\
0.035125	0\\
0.0351458333333333	0\\
0.0351666666666667	0\\
0.0351875	0\\
0.0352083333333333	0\\
0.0352291666666667	0\\
0.03525	0\\
0.0352708333333333	0\\
0.0352916666666667	0\\
0.0353125	0\\
0.0353333333333333	0\\
0.0353541666666667	0\\
0.035375	0\\
0.0353958333333333	0\\
0.0354166666666667	0\\
0.0354375	0\\
0.0354583333333333	0\\
0.0354791666666667	0\\
0.0355	0\\
0.0355208333333333	0\\
0.0355416666666667	0\\
0.0355625	0\\
0.0355833333333333	0\\
0.0356041666666667	0\\
0.035625	0\\
0.0356458333333333	0\\
0.0356666666666667	0\\
0.0356875	0\\
0.0357083333333333	0\\
0.0357291666666667	0\\
0.03575	0\\
0.0357708333333333	0\\
0.0357916666666667	0\\
0.0358125	0\\
0.0358333333333333	0\\
0.0358541666666667	0\\
0.035875	0\\
0.0358958333333333	0\\
0.0359166666666667	0\\
0.0359375	0\\
0.0359583333333333	0\\
0.0359791666666667	0\\
0.036	0\\
0.0360208333333333	0\\
0.0360416666666667	0\\
0.0360625	0\\
0.0360833333333333	0\\
0.0361041666666667	0\\
0.036125	0\\
0.0361458333333333	0\\
0.0361666666666667	0\\
0.0361875	0\\
0.0362083333333333	0\\
0.0362291666666667	0\\
0.03625	0\\
0.0362708333333333	0\\
0.0362916666666667	0\\
0.0363125	0\\
0.0363333333333333	0\\
0.0363541666666667	0\\
0.036375	0\\
0.0363958333333333	0\\
0.0364166666666667	0\\
0.0364375	0\\
0.0364583333333333	0\\
0.0364791666666667	0\\
0.0365	0\\
0.0365208333333333	0\\
0.0365416666666667	0\\
0.0365625	0\\
0.0365833333333333	0\\
0.0366041666666667	0\\
0.036625	0\\
0.0366458333333333	0\\
0.0366666666666667	1\\
0.0366875	1\\
0.0367083333333333	1\\
0.0367291666666667	1\\
0.03675	1\\
0.0367708333333333	1\\
0.0367916666666667	1\\
0.0368125	1\\
0.0368333333333333	1\\
0.0368541666666667	1\\
0.036875	1\\
0.0368958333333333	1\\
0.0369166666666667	1\\
0.0369375	1\\
0.0369583333333333	1\\
0.0369791666666667	1\\
0.037	1\\
0.0370208333333333	1\\
0.0370416666666667	1\\
0.0370625	1\\
0.0370833333333333	1\\
0.0371041666666667	1\\
0.037125	1\\
0.0371458333333333	1\\
0.0371666666666667	1\\
0.0371875	1\\
0.0372083333333333	1\\
0.0372291666666667	1\\
0.03725	1\\
0.0372708333333333	1\\
0.0372916666666667	1\\
0.0373125	1\\
0.0373333333333333	1\\
0.0373541666666667	1\\
0.037375	1\\
0.0373958333333333	1\\
0.0374166666666667	1\\
0.0374375	1\\
0.0374583333333333	1\\
0.0374791666666667	1\\
0.0375	1\\
0.0375208333333333	1\\
0.0375416666666667	1\\
0.0375625	1\\
0.0375833333333333	1\\
0.0376041666666667	1\\
0.037625	1\\
0.0376458333333333	1\\
0.0376666666666667	1\\
0.0376875	1\\
0.0377083333333333	1\\
0.0377291666666667	1\\
0.03775	1\\
0.0377708333333333	1\\
0.0377916666666667	1\\
0.0378125	1\\
0.0378333333333333	1\\
0.0378541666666667	1\\
0.037875	1\\
0.0378958333333333	1\\
0.0379166666666667	1\\
0.0379375	1\\
0.0379583333333333	1\\
0.0379791666666667	1\\
0.038	1\\
0.0380208333333333	1\\
0.0380416666666667	1\\
0.0380625	1\\
0.0380833333333333	1\\
0.0381041666666667	1\\
0.038125	1\\
0.0381458333333333	1\\
0.0381666666666667	1\\
0.0381875	1\\
0.0382083333333333	1\\
0.0382291666666667	1\\
0.03825	1\\
0.0382708333333333	1\\
0.0382916666666667	1\\
0.0383125	1\\
0.0383333333333333	1\\
0.0383541666666667	1\\
0.038375	1\\
0.0383958333333333	1\\
0.0384166666666667	1\\
0.0384375	1\\
0.0384583333333333	1\\
0.0384791666666667	1\\
0.0385	1\\
0.0385208333333333	1\\
0.0385416666666667	1\\
0.0385625	1\\
0.0385833333333333	1\\
0.0386041666666667	1\\
0.038625	1\\
0.0386458333333333	1\\
0.0386666666666667	1\\
0.0386875	1\\
0.0387083333333333	1\\
0.0387291666666667	1\\
0.03875	1\\
0.0387708333333333	1\\
0.0387916666666667	1\\
0.0388125	1\\
0.0388333333333333	1\\
0.0388541666666667	1\\
0.038875	1\\
0.0388958333333333	1\\
0.0389166666666667	1\\
0.0389375	1\\
0.0389583333333333	1\\
0.0389791666666667	1\\
0.039	1\\
0.0390208333333333	1\\
0.0390416666666667	1\\
0.0390625	1\\
0.0390833333333333	1\\
0.0391041666666667	1\\
0.039125	1\\
0.0391458333333333	1\\
0.0391666666666667	1\\
0.0391875	1\\
0.0392083333333333	1\\
0.0392291666666667	1\\
0.03925	1\\
0.0392708333333333	1\\
0.0392916666666667	1\\
0.0393125	1\\
0.0393333333333333	1\\
0.0393541666666667	1\\
0.039375	1\\
0.0393958333333333	1\\
0.0394166666666667	1\\
0.0394375	1\\
0.0394583333333333	1\\
0.0394791666666667	1\\
0.0395	1\\
0.0395208333333333	1\\
0.0395416666666667	1\\
0.0395625	1\\
0.0395833333333333	1\\
0.0396041666666667	1\\
0.039625	1\\
0.0396458333333333	1\\
0.0396666666666667	1\\
0.0396875	1\\
0.0397083333333333	1\\
0.0397291666666667	1\\
0.03975	1\\
0.0397708333333333	1\\
0.0397916666666667	1\\
0.0398125	1\\
0.0398333333333333	1\\
0.0398541666666667	1\\
0.039875	1\\
0.0398958333333333	1\\
0.0399166666666667	1\\
0.0399375	1\\
0.0399583333333333	1\\
0.0399791666666667	1\\
0.04	0\\
0.0400208333333333	0\\
0.0400416666666667	0\\
0.0400625	0\\
0.0400833333333333	0\\
0.0401041666666667	0\\
0.040125	0\\
0.0401458333333333	0\\
0.0401666666666667	0\\
0.0401875	0\\
0.0402083333333333	0\\
0.0402291666666667	0\\
0.04025	0\\
0.0402708333333333	0\\
0.0402916666666667	0\\
0.0403125	0\\
0.0403333333333333	0\\
0.0403541666666667	0\\
0.040375	0\\
0.0403958333333333	0\\
0.0404166666666667	0\\
0.0404375	0\\
0.0404583333333333	0\\
0.0404791666666667	0\\
0.0405	0\\
0.0405208333333333	0\\
0.0405416666666667	0\\
0.0405625	0\\
0.0405833333333333	0\\
0.0406041666666667	0\\
0.040625	0\\
0.0406458333333333	0\\
0.0406666666666667	0\\
0.0406875	0\\
0.0407083333333333	0\\
0.0407291666666667	0\\
0.04075	0\\
0.0407708333333333	0\\
0.0407916666666667	0\\
0.0408125	0\\
0.0408333333333333	0\\
0.0408541666666667	0\\
0.040875	0\\
0.0408958333333333	0\\
0.0409166666666667	0\\
0.0409375	0\\
0.0409583333333333	0\\
0.0409791666666667	0\\
0.041	0\\
0.0410208333333333	0\\
0.0410416666666667	0\\
0.0410625	0\\
0.0410833333333333	0\\
0.0411041666666667	0\\
0.041125	0\\
0.0411458333333333	0\\
0.0411666666666667	0\\
0.0411875	0\\
0.0412083333333333	0\\
0.0412291666666667	0\\
0.04125	0\\
0.0412708333333333	0\\
0.0412916666666667	0\\
0.0413125	0\\
0.0413333333333333	0\\
0.0413541666666667	0\\
0.041375	0\\
0.0413958333333333	0\\
0.0414166666666667	0\\
0.0414375	0\\
0.0414583333333333	0\\
0.0414791666666667	0\\
0.0415	0\\
0.0415208333333333	0\\
0.0415416666666667	0\\
0.0415625	0\\
0.0415833333333333	0\\
0.0416041666666667	0\\
0.041625	0\\
0.0416458333333333	0\\
0.0416666666666667	0\\
0.0416875	0\\
0.0417083333333333	0\\
0.0417291666666667	0\\
0.04175	0\\
0.0417708333333333	0\\
0.0417916666666667	0\\
0.0418125	0\\
0.0418333333333333	0\\
0.0418541666666667	0\\
0.041875	0\\
0.0418958333333333	0\\
0.0419166666666667	0\\
0.0419375	0\\
0.0419583333333333	0\\
0.0419791666666667	0\\
0.042	0\\
0.0420208333333333	0\\
0.0420416666666667	0\\
0.0420625	0\\
0.0420833333333333	0\\
0.0421041666666667	0\\
0.042125	0\\
0.0421458333333333	0\\
0.0421666666666667	0\\
0.0421875	0\\
0.0422083333333333	0\\
0.0422291666666667	0\\
0.04225	0\\
0.0422708333333333	0\\
0.0422916666666667	0\\
0.0423125	0\\
0.0423333333333333	0\\
0.0423541666666667	0\\
0.042375	0\\
0.0423958333333333	0\\
0.0424166666666667	0\\
0.0424375	0\\
0.0424583333333333	0\\
0.0424791666666667	0\\
0.0425	0\\
0.0425208333333333	0\\
0.0425416666666667	0\\
0.0425625	0\\
0.0425833333333333	0\\
0.0426041666666667	0\\
0.042625	0\\
0.0426458333333333	0\\
0.0426666666666667	0\\
0.0426875	0\\
0.0427083333333333	0\\
0.0427291666666667	0\\
0.04275	0\\
0.0427708333333333	0\\
0.0427916666666667	0\\
0.0428125	0\\
0.0428333333333333	0\\
0.0428541666666667	0\\
0.042875	0\\
0.0428958333333333	0\\
0.0429166666666667	0\\
0.0429375	0\\
0.0429583333333333	0\\
0.0429791666666667	0\\
0.043	0\\
0.0430208333333333	0\\
0.0430416666666667	0\\
0.0430625	0\\
0.0430833333333333	0\\
0.0431041666666667	0\\
0.043125	0\\
0.0431458333333333	0\\
0.0431666666666667	0\\
0.0431875	0\\
0.0432083333333333	0\\
0.0432291666666667	0\\
0.04325	0\\
0.0432708333333333	0\\
0.0432916666666667	0\\
0.0433125	0\\
0.0433333333333333	0\\
0.0433541666666667	0\\
0.043375	0\\
0.0433958333333333	0\\
0.0434166666666667	0\\
0.0434375	0\\
0.0434583333333333	0\\
0.0434791666666667	0\\
0.0435	0\\
0.0435208333333333	0\\
0.0435416666666667	0\\
0.0435625	0\\
0.0435833333333333	0\\
0.0436041666666667	0\\
0.043625	0\\
0.0436458333333333	0\\
0.0436666666666667	0\\
0.0436875	0\\
0.0437083333333333	0\\
0.0437291666666667	0\\
0.04375	0\\
0.0437708333333333	0\\
0.0437916666666667	0\\
0.0438125	0\\
0.0438333333333333	0\\
0.0438541666666667	0\\
0.043875	0\\
0.0438958333333333	0\\
0.0439166666666667	0\\
0.0439375	0\\
0.0439583333333333	0\\
0.0439791666666667	0\\
0.044	0\\
0.0440208333333333	0\\
0.0440416666666667	0\\
0.0440625	0\\
0.0440833333333333	0\\
0.0441041666666667	0\\
0.044125	0\\
0.0441458333333333	0\\
0.0441666666666667	0\\
0.0441875	0\\
0.0442083333333333	0\\
0.0442291666666667	0\\
0.04425	0\\
0.0442708333333333	0\\
0.0442916666666667	0\\
0.0443125	0\\
0.0443333333333333	0\\
0.0443541666666667	0\\
0.044375	0\\
0.0443958333333333	0\\
0.0444166666666667	0\\
0.0444375	0\\
0.0444583333333333	0\\
0.0444791666666667	0\\
0.0445	0\\
0.0445208333333333	0\\
0.0445416666666667	0\\
0.0445625	0\\
0.0445833333333333	0\\
0.0446041666666667	0\\
0.044625	0\\
0.0446458333333333	0\\
0.0446666666666667	0\\
0.0446875	0\\
0.0447083333333333	0\\
0.0447291666666667	0\\
0.04475	0\\
0.0447708333333333	0\\
0.0447916666666667	0\\
0.0448125	0\\
0.0448333333333333	0\\
0.0448541666666667	0\\
0.044875	0\\
0.0448958333333333	0\\
0.0449166666666667	0\\
0.0449375	0\\
0.0449583333333333	0\\
0.0449791666666667	0\\
0.045	0\\
0.0450208333333333	0\\
0.0450416666666667	0\\
0.0450625	0\\
0.0450833333333333	0\\
0.0451041666666667	0\\
0.045125	0\\
0.0451458333333333	0\\
0.0451666666666667	0\\
0.0451875	0\\
0.0452083333333333	0\\
0.0452291666666667	0\\
0.04525	0\\
0.0452708333333333	0\\
0.0452916666666667	0\\
0.0453125	0\\
0.0453333333333333	0\\
0.0453541666666667	0\\
0.045375	0\\
0.0453958333333333	0\\
0.0454166666666667	0\\
0.0454375	0\\
0.0454583333333333	0\\
0.0454791666666667	0\\
0.0455	0\\
0.0455208333333333	0\\
0.0455416666666667	0\\
0.0455625	0\\
0.0455833333333333	0\\
0.0456041666666667	0\\
0.045625	0\\
0.0456458333333333	0\\
0.0456666666666667	0\\
0.0456875	0\\
0.0457083333333333	0\\
0.0457291666666667	0\\
0.04575	0\\
0.0457708333333333	0\\
0.0457916666666667	0\\
0.0458125	0\\
0.0458333333333333	0\\
0.0458541666666667	0\\
0.045875	0\\
0.0458958333333333	0\\
0.0459166666666667	0\\
0.0459375	0\\
0.0459583333333333	0\\
0.0459791666666667	0\\
0.046	0\\
0.0460208333333333	0\\
0.0460416666666667	0\\
0.0460625	0\\
0.0460833333333333	0\\
0.0461041666666667	0\\
0.046125	0\\
0.0461458333333333	0\\
0.0461666666666667	0\\
0.0461875	0\\
0.0462083333333333	0\\
0.0462291666666667	0\\
0.04625	0\\
0.0462708333333333	0\\
0.0462916666666667	0\\
0.0463125	0\\
0.0463333333333333	0\\
0.0463541666666667	0\\
0.046375	0\\
0.0463958333333333	0\\
0.0464166666666667	0\\
0.0464375	0\\
0.0464583333333333	0\\
0.0464791666666667	0\\
0.0465	0\\
0.0465208333333333	0\\
0.0465416666666667	0\\
0.0465625	0\\
0.0465833333333333	0\\
0.0466041666666667	0\\
0.046625	0\\
0.0466458333333333	0\\
0.0466666666666667	0\\
0.0466875	0\\
0.0467083333333333	0\\
0.0467291666666667	0\\
0.04675	0\\
0.0467708333333333	0\\
0.0467916666666667	0\\
0.0468125	0\\
0.0468333333333333	0\\
0.0468541666666667	0\\
0.046875	0\\
0.0468958333333333	0\\
0.0469166666666667	0\\
0.0469375	0\\
0.0469583333333333	0\\
0.0469791666666667	0\\
0.047	0\\
0.0470208333333333	0\\
0.0470416666666667	0\\
0.0470625	0\\
0.0470833333333333	0\\
0.0471041666666667	0\\
0.047125	0\\
0.0471458333333333	0\\
0.0471666666666667	0\\
0.0471875	0\\
0.0472083333333333	0\\
0.0472291666666667	0\\
0.04725	0\\
0.0472708333333333	0\\
0.0472916666666667	0\\
0.0473125	0\\
0.0473333333333333	0\\
0.0473541666666667	0\\
0.047375	0\\
0.0473958333333333	0\\
0.0474166666666667	0\\
0.0474375	0\\
0.0474583333333333	0\\
0.0474791666666667	0\\
0.0475	0\\
0.0475208333333333	0\\
0.0475416666666667	0\\
0.0475625	0\\
0.0475833333333333	0\\
0.0476041666666667	0\\
0.047625	0\\
0.0476458333333333	0\\
0.0476666666666667	0\\
0.0476875	0\\
0.0477083333333333	0\\
0.0477291666666667	0\\
0.04775	0\\
0.0477708333333333	0\\
0.0477916666666667	0\\
0.0478125	0\\
0.0478333333333333	0\\
0.0478541666666667	0\\
0.047875	0\\
0.0478958333333333	0\\
0.0479166666666667	0\\
0.0479375	0\\
0.0479583333333333	0\\
0.0479791666666667	0\\
0.048	0\\
0.0480208333333333	0\\
0.0480416666666667	0\\
0.0480625	0\\
0.0480833333333333	0\\
0.0481041666666667	0\\
0.048125	0\\
0.0481458333333333	0\\
0.0481666666666667	0\\
0.0481875	0\\
0.0482083333333333	0\\
0.0482291666666667	0\\
0.04825	0\\
0.0482708333333333	0\\
0.0482916666666667	0\\
0.0483125	0\\
0.0483333333333333	0\\
0.0483541666666667	0\\
0.048375	0\\
0.0483958333333333	0\\
0.0484166666666667	0\\
0.0484375	0\\
0.0484583333333333	0\\
0.0484791666666667	0\\
0.0485	0\\
0.0485208333333333	0\\
0.0485416666666667	0\\
0.0485625	0\\
0.0485833333333333	0\\
0.0486041666666667	0\\
0.048625	0\\
0.0486458333333333	0\\
0.0486666666666667	0\\
0.0486875	0\\
0.0487083333333333	0\\
0.0487291666666667	0\\
0.04875	0\\
0.0487708333333333	0\\
0.0487916666666667	0\\
0.0488125	0\\
0.0488333333333333	0\\
0.0488541666666667	0\\
0.048875	0\\
0.0488958333333333	0\\
0.0489166666666667	0\\
0.0489375	0\\
0.0489583333333333	0\\
0.0489791666666667	0\\
0.049	0\\
0.0490208333333333	0\\
0.0490416666666667	0\\
0.0490625	0\\
0.0490833333333333	0\\
0.0491041666666667	0\\
0.049125	0\\
0.0491458333333333	0\\
0.0491666666666667	0\\
0.0491875	0\\
0.0492083333333333	0\\
0.0492291666666667	0\\
0.04925	0\\
0.0492708333333333	0\\
0.0492916666666667	0\\
0.0493125	0\\
0.0493333333333333	0\\
0.0493541666666667	0\\
0.049375	0\\
0.0493958333333333	0\\
0.0494166666666667	0\\
0.0494375	0\\
0.0494583333333333	0\\
0.0494791666666667	0\\
0.0495	0\\
0.0495208333333333	0\\
0.0495416666666667	0\\
0.0495625	0\\
0.0495833333333333	0\\
0.0496041666666667	0\\
0.049625	0\\
0.0496458333333333	0\\
0.0496666666666667	0\\
0.0496875	0\\
0.0497083333333333	0\\
0.0497291666666667	0\\
0.04975	0\\
0.0497708333333333	0\\
0.0497916666666667	0\\
0.0498125	0\\
0.0498333333333333	0\\
0.0498541666666667	0\\
0.049875	0\\
0.0498958333333333	0\\
0.0499166666666667	0\\
0.0499375	0\\
0.0499583333333333	0\\
0.0499791666666667	0\\
0.05	1\\
0.0500208333333333	1\\
0.0500416666666667	1\\
0.0500625	1\\
0.0500833333333333	1\\
0.0501041666666667	1\\
0.050125	1\\
0.0501458333333333	1\\
0.0501666666666667	1\\
0.0501875	1\\
0.0502083333333333	1\\
0.0502291666666667	1\\
0.05025	1\\
0.0502708333333333	1\\
0.0502916666666667	1\\
0.0503125	1\\
0.0503333333333333	1\\
0.0503541666666667	1\\
0.050375	1\\
0.0503958333333333	1\\
0.0504166666666667	1\\
0.0504375	1\\
0.0504583333333333	1\\
0.0504791666666667	1\\
0.0505	1\\
0.0505208333333333	1\\
0.0505416666666667	1\\
0.0505625	1\\
0.0505833333333333	1\\
0.0506041666666667	1\\
0.050625	1\\
0.0506458333333333	1\\
0.0506666666666667	1\\
0.0506875	1\\
0.0507083333333333	1\\
0.0507291666666667	1\\
0.05075	1\\
0.0507708333333333	1\\
0.0507916666666667	1\\
0.0508125	1\\
0.0508333333333333	1\\
0.0508541666666667	1\\
0.050875	1\\
0.0508958333333333	1\\
0.0509166666666667	1\\
0.0509375	1\\
0.0509583333333333	1\\
0.0509791666666667	1\\
0.051	1\\
0.0510208333333333	1\\
0.0510416666666667	1\\
0.0510625	1\\
0.0510833333333333	1\\
0.0511041666666667	1\\
0.051125	1\\
0.0511458333333333	1\\
0.0511666666666667	1\\
0.0511875	1\\
0.0512083333333333	1\\
0.0512291666666667	1\\
0.05125	1\\
0.0512708333333333	1\\
0.0512916666666667	1\\
0.0513125	1\\
0.0513333333333333	1\\
0.0513541666666667	1\\
0.051375	1\\
0.0513958333333333	1\\
0.0514166666666667	1\\
0.0514375	1\\
0.0514583333333333	1\\
0.0514791666666667	1\\
0.0515	1\\
0.0515208333333333	1\\
0.0515416666666667	1\\
0.0515625	1\\
0.0515833333333333	1\\
0.0516041666666667	1\\
0.051625	1\\
0.0516458333333333	1\\
0.0516666666666667	1\\
0.0516875	1\\
0.0517083333333333	1\\
0.0517291666666667	1\\
0.05175	1\\
0.0517708333333333	1\\
0.0517916666666667	1\\
0.0518125	1\\
0.0518333333333333	1\\
0.0518541666666667	1\\
0.051875	1\\
0.0518958333333333	1\\
0.0519166666666667	1\\
0.0519375	1\\
0.0519583333333333	1\\
0.0519791666666667	1\\
0.052	1\\
0.0520208333333333	1\\
0.0520416666666667	1\\
0.0520625	1\\
0.0520833333333333	1\\
0.0521041666666667	1\\
0.052125	1\\
0.0521458333333333	1\\
0.0521666666666667	1\\
0.0521875	1\\
0.0522083333333333	1\\
0.0522291666666667	1\\
0.05225	1\\
0.0522708333333333	1\\
0.0522916666666667	1\\
0.0523125	1\\
0.0523333333333333	1\\
0.0523541666666667	1\\
0.052375	1\\
0.0523958333333333	1\\
0.0524166666666667	1\\
0.0524375	1\\
0.0524583333333333	1\\
0.0524791666666667	1\\
0.0525	1\\
0.0525208333333333	1\\
0.0525416666666667	1\\
0.0525625	1\\
0.0525833333333333	1\\
0.0526041666666667	1\\
0.052625	1\\
0.0526458333333333	1\\
0.0526666666666667	1\\
0.0526875	1\\
0.0527083333333333	1\\
0.0527291666666667	1\\
0.05275	1\\
0.0527708333333333	1\\
0.0527916666666667	1\\
0.0528125	1\\
0.0528333333333333	1\\
0.0528541666666667	1\\
0.052875	1\\
0.0528958333333333	1\\
0.0529166666666667	1\\
0.0529375	1\\
0.0529583333333333	1\\
0.0529791666666667	1\\
0.053	1\\
0.0530208333333333	1\\
0.0530416666666667	1\\
0.0530625	1\\
0.0530833333333333	1\\
0.0531041666666667	1\\
0.053125	1\\
0.0531458333333333	1\\
0.0531666666666667	1\\
0.0531875	1\\
0.0532083333333333	1\\
0.0532291666666667	1\\
0.05325	1\\
0.0532708333333333	1\\
0.0532916666666667	1\\
0.0533125	1\\
0.0533333333333333	1\\
0.0533541666666667	1\\
0.053375	1\\
0.0533958333333333	1\\
0.0534166666666667	1\\
0.0534375	1\\
0.0534583333333333	1\\
0.0534791666666667	1\\
0.0535	1\\
0.0535208333333333	1\\
0.0535416666666667	1\\
0.0535625	1\\
0.0535833333333333	1\\
0.0536041666666667	1\\
0.053625	1\\
0.0536458333333333	1\\
0.0536666666666667	1\\
0.0536875	1\\
0.0537083333333333	1\\
0.0537291666666667	1\\
0.05375	1\\
0.0537708333333333	1\\
0.0537916666666667	1\\
0.0538125	1\\
0.0538333333333333	1\\
0.0538541666666667	1\\
0.053875	1\\
0.0538958333333333	1\\
0.0539166666666667	1\\
0.0539375	1\\
0.0539583333333333	1\\
0.0539791666666667	1\\
0.054	1\\
0.0540208333333333	1\\
0.0540416666666667	1\\
0.0540625	1\\
0.0540833333333333	1\\
0.0541041666666667	1\\
0.054125	1\\
0.0541458333333333	1\\
0.0541666666666667	1\\
0.0541875	1\\
0.0542083333333333	1\\
0.0542291666666667	1\\
0.05425	1\\
0.0542708333333333	1\\
0.0542916666666667	1\\
0.0543125	1\\
0.0543333333333333	1\\
0.0543541666666667	1\\
0.054375	1\\
0.0543958333333333	1\\
0.0544166666666667	1\\
0.0544375	1\\
0.0544583333333333	1\\
0.0544791666666667	1\\
0.0545	1\\
0.0545208333333333	1\\
0.0545416666666667	1\\
0.0545625	1\\
0.0545833333333333	1\\
0.0546041666666667	1\\
0.054625	1\\
0.0546458333333333	1\\
0.0546666666666667	1\\
0.0546875	1\\
0.0547083333333333	1\\
0.0547291666666667	1\\
0.05475	1\\
0.0547708333333333	1\\
0.0547916666666667	1\\
0.0548125	1\\
0.0548333333333333	1\\
0.0548541666666667	1\\
0.054875	1\\
0.0548958333333333	1\\
0.0549166666666667	1\\
0.0549375	1\\
0.0549583333333333	1\\
0.0549791666666667	1\\
0.055	1\\
0.0550208333333333	1\\
0.0550416666666667	1\\
0.0550625	1\\
0.0550833333333333	1\\
0.0551041666666667	1\\
0.055125	1\\
0.0551458333333333	1\\
0.0551666666666667	1\\
0.0551875	1\\
0.0552083333333333	1\\
0.0552291666666667	1\\
0.05525	1\\
0.0552708333333333	1\\
0.0552916666666667	1\\
0.0553125	1\\
0.0553333333333333	1\\
0.0553541666666667	1\\
0.055375	1\\
0.0553958333333333	1\\
0.0554166666666667	1\\
0.0554375	1\\
0.0554583333333333	1\\
0.0554791666666667	1\\
0.0555	1\\
0.0555208333333333	1\\
0.0555416666666667	1\\
0.0555625	1\\
0.0555833333333333	1\\
0.0556041666666667	1\\
0.055625	1\\
0.0556458333333333	1\\
0.0556666666666667	1\\
0.0556875	1\\
0.0557083333333333	1\\
0.0557291666666667	1\\
0.05575	1\\
0.0557708333333333	1\\
0.0557916666666667	1\\
0.0558125	1\\
0.0558333333333333	1\\
0.0558541666666667	1\\
0.055875	1\\
0.0558958333333333	1\\
0.0559166666666667	1\\
0.0559375	1\\
0.0559583333333333	1\\
0.0559791666666667	1\\
0.056	1\\
0.0560208333333333	1\\
0.0560416666666667	1\\
0.0560625	1\\
0.0560833333333333	1\\
0.0561041666666667	1\\
0.056125	1\\
0.0561458333333333	1\\
0.0561666666666667	1\\
0.0561875	1\\
0.0562083333333333	1\\
0.0562291666666667	1\\
0.05625	1\\
0.0562708333333333	1\\
0.0562916666666667	1\\
0.0563125	1\\
0.0563333333333333	1\\
0.0563541666666667	1\\
0.056375	1\\
0.0563958333333333	1\\
0.0564166666666667	1\\
0.0564375	1\\
0.0564583333333333	1\\
0.0564791666666667	1\\
0.0565	1\\
0.0565208333333333	1\\
0.0565416666666667	1\\
0.0565625	1\\
0.0565833333333333	1\\
0.0566041666666667	1\\
0.056625	1\\
0.0566458333333333	1\\
0.0566666666666667	1\\
0.0566875	1\\
0.0567083333333333	1\\
0.0567291666666667	1\\
0.05675	1\\
0.0567708333333333	1\\
0.0567916666666667	1\\
0.0568125	1\\
0.0568333333333333	1\\
0.0568541666666667	1\\
0.056875	1\\
0.0568958333333333	1\\
0.0569166666666667	1\\
0.0569375	1\\
0.0569583333333333	1\\
0.0569791666666667	1\\
0.057	1\\
0.0570208333333333	1\\
0.0570416666666667	1\\
0.0570625	1\\
0.0570833333333333	1\\
0.0571041666666667	1\\
0.057125	1\\
0.0571458333333333	1\\
0.0571666666666667	1\\
0.0571875	1\\
0.0572083333333333	1\\
0.0572291666666667	1\\
0.05725	1\\
0.0572708333333333	1\\
0.0572916666666667	1\\
0.0573125	1\\
0.0573333333333333	1\\
0.0573541666666667	1\\
0.057375	1\\
0.0573958333333333	1\\
0.0574166666666667	1\\
0.0574375	1\\
0.0574583333333333	1\\
0.0574791666666667	1\\
0.0575	1\\
0.0575208333333333	1\\
0.0575416666666667	1\\
0.0575625	1\\
0.0575833333333333	1\\
0.0576041666666667	1\\
0.057625	1\\
0.0576458333333333	1\\
0.0576666666666667	1\\
0.0576875	1\\
0.0577083333333333	1\\
0.0577291666666667	1\\
0.05775	1\\
0.0577708333333333	1\\
0.0577916666666667	1\\
0.0578125	1\\
0.0578333333333333	1\\
0.0578541666666667	1\\
0.057875	1\\
0.0578958333333333	1\\
0.0579166666666667	1\\
0.0579375	1\\
0.0579583333333333	1\\
0.0579791666666667	1\\
0.058	1\\
0.0580208333333333	1\\
0.0580416666666667	1\\
0.0580625	1\\
0.0580833333333333	1\\
0.0581041666666667	1\\
0.058125	1\\
0.0581458333333333	1\\
0.0581666666666667	1\\
0.0581875	1\\
0.0582083333333333	1\\
0.0582291666666667	1\\
0.05825	1\\
0.0582708333333333	1\\
0.0582916666666667	1\\
0.0583125	1\\
0.0583333333333333	1\\
0.0583541666666667	1\\
0.058375	1\\
0.0583958333333333	1\\
0.0584166666666667	1\\
0.0584375	1\\
0.0584583333333333	1\\
0.0584791666666667	1\\
0.0585	1\\
0.0585208333333333	1\\
0.0585416666666667	1\\
0.0585625	1\\
0.0585833333333333	1\\
0.0586041666666667	1\\
0.058625	1\\
0.0586458333333333	1\\
0.0586666666666667	1\\
0.0586875	1\\
0.0587083333333333	1\\
0.0587291666666667	1\\
0.05875	1\\
0.0587708333333333	1\\
0.0587916666666667	1\\
0.0588125	1\\
0.0588333333333333	1\\
0.0588541666666667	1\\
0.058875	1\\
0.0588958333333333	1\\
0.0589166666666667	1\\
0.0589375	1\\
0.0589583333333333	1\\
0.0589791666666667	1\\
0.059	1\\
0.0590208333333333	1\\
0.0590416666666667	1\\
0.0590625	1\\
0.0590833333333333	1\\
0.0591041666666667	1\\
0.059125	1\\
0.0591458333333333	1\\
0.0591666666666667	1\\
0.0591875	1\\
0.0592083333333333	1\\
0.0592291666666667	1\\
0.05925	1\\
0.0592708333333333	1\\
0.0592916666666667	1\\
0.0593125	1\\
0.0593333333333333	1\\
0.0593541666666667	1\\
0.059375	1\\
0.0593958333333333	1\\
0.0594166666666667	1\\
0.0594375	1\\
0.0594583333333333	1\\
0.0594791666666667	1\\
0.0595	1\\
0.0595208333333333	1\\
0.0595416666666667	1\\
0.0595625	1\\
0.0595833333333333	1\\
0.0596041666666667	1\\
0.059625	1\\
0.0596458333333333	1\\
0.0596666666666667	1\\
0.0596875	1\\
0.0597083333333333	1\\
0.0597291666666667	1\\
0.05975	1\\
0.0597708333333333	1\\
0.0597916666666667	1\\
0.0598125	1\\
0.0598333333333333	1\\
0.0598541666666667	1\\
0.059875	1\\
0.0598958333333333	1\\
0.0599166666666667	1\\
0.0599375	1\\
0.0599583333333333	1\\
0.0599791666666667	1\\
0.06	1\\
0.0600208333333333	1\\
0.0600416666666667	1\\
0.0600625	1\\
0.0600833333333333	1\\
0.0601041666666667	1\\
0.060125	1\\
0.0601458333333333	1\\
0.0601666666666667	1\\
0.0601875	1\\
0.0602083333333333	1\\
0.0602291666666667	1\\
0.06025	1\\
0.0602708333333333	1\\
0.0602916666666667	1\\
0.0603125	1\\
0.0603333333333333	1\\
0.0603541666666667	1\\
0.060375	1\\
0.0603958333333333	1\\
0.0604166666666667	1\\
0.0604375	1\\
0.0604583333333333	1\\
0.0604791666666667	1\\
0.0605	1\\
0.0605208333333333	1\\
0.0605416666666667	1\\
0.0605625	1\\
0.0605833333333333	1\\
0.0606041666666667	1\\
0.060625	1\\
0.0606458333333333	1\\
0.0606666666666667	1\\
0.0606875	1\\
0.0607083333333333	1\\
0.0607291666666667	1\\
0.06075	1\\
0.0607708333333333	1\\
0.0607916666666667	1\\
0.0608125	1\\
0.0608333333333333	1\\
0.0608541666666667	1\\
0.060875	1\\
0.0608958333333333	1\\
0.0609166666666667	1\\
0.0609375	1\\
0.0609583333333333	1\\
0.0609791666666667	1\\
0.061	1\\
0.0610208333333333	1\\
0.0610416666666667	1\\
0.0610625	1\\
0.0610833333333333	1\\
0.0611041666666667	1\\
0.061125	1\\
0.0611458333333333	1\\
0.0611666666666667	1\\
0.0611875	1\\
0.0612083333333333	1\\
0.0612291666666667	1\\
0.06125	1\\
0.0612708333333333	1\\
0.0612916666666667	1\\
0.0613125	1\\
0.0613333333333333	1\\
0.0613541666666667	1\\
0.061375	1\\
0.0613958333333333	1\\
0.0614166666666667	1\\
0.0614375	1\\
0.0614583333333333	1\\
0.0614791666666667	1\\
0.0615	1\\
0.0615208333333333	1\\
0.0615416666666667	1\\
0.0615625	1\\
0.0615833333333333	1\\
0.0616041666666667	1\\
0.061625	1\\
0.0616458333333333	1\\
0.0616666666666667	1\\
0.0616875	1\\
0.0617083333333333	1\\
0.0617291666666667	1\\
0.06175	1\\
0.0617708333333333	1\\
0.0617916666666667	1\\
0.0618125	1\\
0.0618333333333333	1\\
0.0618541666666667	1\\
0.061875	1\\
0.0618958333333333	1\\
0.0619166666666667	1\\
0.0619375	1\\
0.0619583333333333	1\\
0.0619791666666667	1\\
0.062	1\\
0.0620208333333333	1\\
0.0620416666666667	1\\
0.0620625	1\\
0.0620833333333333	1\\
0.0621041666666667	1\\
0.062125	1\\
0.0621458333333333	1\\
0.0621666666666667	1\\
0.0621875	1\\
0.0622083333333333	1\\
0.0622291666666667	1\\
0.06225	1\\
0.0622708333333333	1\\
0.0622916666666667	1\\
0.0623125	1\\
0.0623333333333333	1\\
0.0623541666666667	1\\
0.062375	1\\
0.0623958333333333	1\\
0.0624166666666667	1\\
0.0624375	1\\
0.0624583333333333	1\\
0.0624791666666667	1\\
0.0625	1\\
0.0625208333333333	1\\
0.0625416666666667	1\\
0.0625625	1\\
0.0625833333333333	1\\
0.0626041666666667	1\\
0.062625	1\\
0.0626458333333333	1\\
0.0626666666666667	1\\
0.0626875	1\\
0.0627083333333333	1\\
0.0627291666666667	1\\
0.06275	1\\
0.0627708333333333	1\\
0.0627916666666667	1\\
0.0628125	1\\
0.0628333333333333	1\\
0.0628541666666667	1\\
0.062875	1\\
0.0628958333333333	1\\
0.0629166666666667	1\\
0.0629375	1\\
0.0629583333333333	1\\
0.0629791666666667	1\\
0.063	1\\
0.0630208333333333	1\\
0.0630416666666667	1\\
0.0630625	1\\
0.0630833333333333	1\\
0.0631041666666667	1\\
0.063125	1\\
0.0631458333333333	1\\
0.0631666666666667	1\\
0.0631875	1\\
0.0632083333333333	1\\
0.0632291666666667	1\\
0.06325	1\\
0.0632708333333333	1\\
0.0632916666666667	1\\
0.0633125	1\\
0.0633333333333333	1\\
0.0633541666666667	1\\
0.063375	1\\
0.0633958333333333	1\\
0.0634166666666667	1\\
0.0634375	1\\
0.0634583333333333	1\\
0.0634791666666667	1\\
0.0635	1\\
0.0635208333333333	1\\
0.0635416666666667	1\\
0.0635625	1\\
0.0635833333333333	1\\
0.0636041666666667	1\\
0.063625	1\\
0.0636458333333333	1\\
0.0636666666666667	1\\
0.0636875	1\\
0.0637083333333333	1\\
0.0637291666666667	1\\
0.06375	1\\
0.0637708333333333	1\\
0.0637916666666667	1\\
0.0638125	1\\
0.0638333333333333	1\\
0.0638541666666667	1\\
0.063875	1\\
0.0638958333333333	1\\
0.0639166666666667	1\\
0.0639375	1\\
0.0639583333333333	1\\
0.0639791666666667	1\\
0.064	1\\
0.0640208333333333	1\\
0.0640416666666667	1\\
0.0640625	1\\
0.0640833333333333	1\\
0.0641041666666667	1\\
0.064125	1\\
0.0641458333333333	1\\
0.0641666666666667	1\\
0.0641875	1\\
0.0642083333333333	1\\
0.0642291666666667	1\\
0.06425	1\\
0.0642708333333333	1\\
0.0642916666666667	1\\
0.0643125	1\\
0.0643333333333333	1\\
0.0643541666666667	1\\
0.064375	1\\
0.0643958333333333	1\\
0.0644166666666667	1\\
0.0644375	1\\
0.0644583333333333	1\\
0.0644791666666667	1\\
0.0645	1\\
0.0645208333333333	1\\
0.0645416666666667	1\\
0.0645625	1\\
0.0645833333333333	1\\
0.0646041666666667	1\\
0.064625	1\\
0.0646458333333333	1\\
0.0646666666666667	1\\
0.0646875	1\\
0.0647083333333333	1\\
0.0647291666666667	1\\
0.06475	1\\
0.0647708333333333	1\\
0.0647916666666667	1\\
0.0648125	1\\
0.0648333333333333	1\\
0.0648541666666667	1\\
0.064875	1\\
0.0648958333333333	1\\
0.0649166666666667	1\\
0.0649375	1\\
0.0649583333333333	1\\
0.0649791666666667	1\\
0.065	1\\
0.0650208333333333	1\\
0.0650416666666667	1\\
0.0650625	1\\
0.0650833333333333	1\\
0.0651041666666667	1\\
0.065125	1\\
0.0651458333333333	1\\
0.0651666666666667	1\\
0.0651875	1\\
0.0652083333333333	1\\
0.0652291666666667	1\\
0.06525	1\\
0.0652708333333333	1\\
0.0652916666666667	1\\
0.0653125	1\\
0.0653333333333333	1\\
0.0653541666666667	1\\
0.065375	1\\
0.0653958333333333	1\\
0.0654166666666667	1\\
0.0654375	1\\
0.0654583333333333	1\\
0.0654791666666667	1\\
0.0655	1\\
0.0655208333333333	1\\
0.0655416666666667	1\\
0.0655625	1\\
0.0655833333333333	1\\
0.0656041666666667	1\\
0.065625	1\\
0.0656458333333333	1\\
0.0656666666666667	1\\
0.0656875	1\\
0.0657083333333333	1\\
0.0657291666666667	1\\
0.06575	1\\
0.0657708333333333	1\\
0.0657916666666667	1\\
0.0658125	1\\
0.0658333333333333	1\\
0.0658541666666667	1\\
0.065875	1\\
0.0658958333333333	1\\
0.0659166666666667	1\\
0.0659375	1\\
0.0659583333333333	1\\
0.0659791666666667	1\\
0.066	1\\
0.0660208333333333	1\\
0.0660416666666667	1\\
0.0660625	1\\
0.0660833333333333	1\\
0.0661041666666667	1\\
0.066125	1\\
0.0661458333333333	1\\
0.0661666666666667	1\\
0.0661875	1\\
0.0662083333333333	1\\
0.0662291666666667	1\\
0.06625	1\\
0.0662708333333333	1\\
0.0662916666666667	1\\
0.0663125	1\\
0.0663333333333333	1\\
0.0663541666666667	1\\
0.066375	1\\
0.0663958333333333	1\\
0.0664166666666667	1\\
0.0664375	1\\
0.0664583333333333	1\\
0.0664791666666667	1\\
0.0665	1\\
0.0665208333333333	1\\
0.0665416666666667	1\\
0.0665625	1\\
0.0665833333333333	1\\
0.0666041666666667	1\\
0.066625	1\\
0.0666458333333333	1\\
0.0666666666666667	1\\
0.0666875	1\\
0.0667083333333333	1\\
0.0667291666666667	1\\
0.06675	1\\
0.0667708333333333	1\\
0.0667916666666667	1\\
0.0668125	1\\
0.0668333333333333	1\\
0.0668541666666667	1\\
0.066875	1\\
0.0668958333333333	1\\
0.0669166666666667	1\\
0.0669375	1\\
0.0669583333333333	1\\
0.0669791666666667	1\\
0.067	1\\
0.0670208333333333	1\\
0.0670416666666667	1\\
0.0670625	1\\
0.0670833333333333	1\\
0.0671041666666667	1\\
0.067125	1\\
0.0671458333333333	1\\
0.0671666666666667	1\\
0.0671875	1\\
0.0672083333333333	1\\
0.0672291666666667	1\\
0.06725	1\\
0.0672708333333333	1\\
0.0672916666666667	1\\
0.0673125	1\\
0.0673333333333333	1\\
0.0673541666666667	1\\
0.067375	1\\
0.0673958333333333	1\\
0.0674166666666667	1\\
0.0674375	1\\
0.0674583333333333	1\\
0.0674791666666667	1\\
0.0675	1\\
0.0675208333333333	1\\
0.0675416666666667	1\\
0.0675625	1\\
0.0675833333333333	1\\
0.0676041666666667	1\\
0.067625	1\\
0.0676458333333333	1\\
0.0676666666666667	1\\
0.0676875	1\\
0.0677083333333333	1\\
0.0677291666666667	1\\
0.06775	1\\
0.0677708333333333	1\\
0.0677916666666667	1\\
0.0678125	1\\
0.0678333333333333	1\\
0.0678541666666667	1\\
0.067875	1\\
0.0678958333333333	1\\
0.0679166666666667	1\\
0.0679375	1\\
0.0679583333333333	1\\
0.0679791666666667	1\\
0.068	1\\
0.0680208333333333	1\\
0.0680416666666667	1\\
0.0680625	1\\
0.0680833333333333	1\\
0.0681041666666667	1\\
0.068125	1\\
0.0681458333333333	1\\
0.0681666666666667	1\\
0.0681875	1\\
0.0682083333333333	1\\
0.0682291666666667	1\\
0.06825	1\\
0.0682708333333333	1\\
0.0682916666666667	1\\
0.0683125	1\\
0.0683333333333333	1\\
0.0683541666666667	1\\
0.068375	1\\
0.0683958333333333	1\\
0.0684166666666667	1\\
0.0684375	1\\
0.0684583333333333	1\\
0.0684791666666667	1\\
0.0685	1\\
0.0685208333333333	1\\
0.0685416666666667	1\\
0.0685625	1\\
0.0685833333333333	1\\
0.0686041666666667	1\\
0.068625	1\\
0.0686458333333333	1\\
0.0686666666666667	1\\
0.0686875	1\\
0.0687083333333333	1\\
0.0687291666666667	1\\
0.06875	1\\
0.0687708333333333	1\\
0.0687916666666667	1\\
0.0688125	1\\
0.0688333333333333	1\\
0.0688541666666667	1\\
0.068875	1\\
0.0688958333333333	1\\
0.0689166666666667	1\\
0.0689375	1\\
0.0689583333333333	1\\
0.0689791666666667	1\\
0.069	1\\
0.0690208333333333	1\\
0.0690416666666667	1\\
0.0690625	1\\
0.0690833333333333	1\\
0.0691041666666667	1\\
0.069125	1\\
0.0691458333333333	1\\
0.0691666666666667	1\\
0.0691875	1\\
0.0692083333333333	1\\
0.0692291666666667	1\\
0.06925	1\\
0.0692708333333333	1\\
0.0692916666666667	1\\
0.0693125	1\\
0.0693333333333333	1\\
0.0693541666666667	1\\
0.069375	1\\
0.0693958333333333	1\\
0.0694166666666667	1\\
0.0694375	1\\
0.0694583333333333	1\\
0.0694791666666667	1\\
0.0695	1\\
0.0695208333333333	1\\
0.0695416666666667	1\\
0.0695625	1\\
0.0695833333333333	1\\
0.0696041666666667	1\\
0.069625	1\\
0.0696458333333333	1\\
0.0696666666666667	1\\
0.0696875	1\\
0.0697083333333333	1\\
0.0697291666666667	1\\
0.06975	1\\
0.0697708333333333	1\\
0.0697916666666667	1\\
0.0698125	1\\
0.0698333333333333	1\\
0.0698541666666667	1\\
0.069875	1\\
0.0698958333333333	1\\
0.0699166666666667	1\\
0.0699375	1\\
0.0699583333333333	1\\
0.0699791666666667	1\\
0.07	0\\
0.0700208333333333	0\\
0.0700416666666667	0\\
0.0700625	0\\
0.0700833333333333	0\\
0.0701041666666667	0\\
0.070125	0\\
0.0701458333333333	0\\
0.0701666666666667	0\\
0.0701875	0\\
0.0702083333333333	0\\
0.0702291666666667	0\\
0.07025	0\\
0.0702708333333333	0\\
0.0702916666666667	0\\
0.0703125	0\\
0.0703333333333333	0\\
0.0703541666666667	0\\
0.070375	0\\
0.0703958333333333	0\\
0.0704166666666667	0\\
0.0704375	0\\
0.0704583333333333	0\\
0.0704791666666667	0\\
0.0705	0\\
0.0705208333333333	0\\
0.0705416666666667	0\\
0.0705625	0\\
0.0705833333333333	0\\
0.0706041666666667	0\\
0.070625	0\\
0.0706458333333333	0\\
0.0706666666666667	0\\
0.0706875	0\\
0.0707083333333333	0\\
0.0707291666666667	0\\
0.07075	0\\
0.0707708333333333	0\\
0.0707916666666667	0\\
0.0708125	0\\
0.0708333333333333	0\\
0.0708541666666667	0\\
0.070875	0\\
0.0708958333333333	0\\
0.0709166666666667	0\\
0.0709375	0\\
0.0709583333333333	0\\
0.0709791666666667	0\\
0.071	0\\
0.0710208333333333	0\\
0.0710416666666667	0\\
0.0710625	0\\
0.0710833333333333	0\\
0.0711041666666667	0\\
0.071125	0\\
0.0711458333333333	0\\
0.0711666666666667	0\\
0.0711875	0\\
0.0712083333333333	0\\
0.0712291666666667	0\\
0.07125	0\\
0.0712708333333333	0\\
0.0712916666666667	0\\
0.0713125	0\\
0.0713333333333333	0\\
0.0713541666666667	0\\
0.071375	0\\
0.0713958333333333	0\\
0.0714166666666667	0\\
0.0714375	0\\
0.0714583333333333	0\\
0.0714791666666667	0\\
0.0715	0\\
0.0715208333333333	0\\
0.0715416666666667	0\\
0.0715625	0\\
0.0715833333333333	0\\
0.0716041666666667	0\\
0.071625	0\\
0.0716458333333333	0\\
0.0716666666666667	0\\
0.0716875	0\\
0.0717083333333333	0\\
0.0717291666666667	0\\
0.07175	0\\
0.0717708333333333	0\\
0.0717916666666667	0\\
0.0718125	0\\
0.0718333333333333	0\\
0.0718541666666667	0\\
0.071875	0\\
0.0718958333333333	0\\
0.0719166666666667	0\\
0.0719375	0\\
0.0719583333333333	0\\
0.0719791666666667	0\\
0.072	0\\
0.0720208333333333	0\\
0.0720416666666667	0\\
0.0720625	0\\
0.0720833333333333	0\\
0.0721041666666667	0\\
0.072125	0\\
0.0721458333333333	0\\
0.0721666666666667	0\\
0.0721875	0\\
0.0722083333333333	0\\
0.0722291666666667	0\\
0.07225	0\\
0.0722708333333333	0\\
0.0722916666666667	0\\
0.0723125	0\\
0.0723333333333333	0\\
0.0723541666666667	0\\
0.072375	0\\
0.0723958333333333	0\\
0.0724166666666667	0\\
0.0724375	0\\
0.0724583333333333	0\\
0.0724791666666667	0\\
0.0725	0\\
0.0725208333333333	0\\
0.0725416666666667	0\\
0.0725625	0\\
0.0725833333333333	0\\
0.0726041666666667	0\\
0.072625	0\\
0.0726458333333333	0\\
0.0726666666666667	0\\
0.0726875	0\\
0.0727083333333333	0\\
0.0727291666666667	0\\
0.07275	0\\
0.0727708333333333	0\\
0.0727916666666667	0\\
0.0728125	0\\
0.0728333333333333	0\\
0.0728541666666667	0\\
0.072875	0\\
0.0728958333333333	0\\
0.0729166666666667	0\\
0.0729375	0\\
0.0729583333333333	0\\
0.0729791666666667	0\\
0.073	0\\
0.0730208333333333	0\\
0.0730416666666667	0\\
0.0730625	0\\
0.0730833333333333	0\\
0.0731041666666667	0\\
0.073125	0\\
0.0731458333333333	0\\
0.0731666666666667	0\\
0.0731875	0\\
0.0732083333333333	0\\
0.0732291666666667	0\\
0.07325	0\\
0.0732708333333333	0\\
0.0732916666666667	0\\
0.0733125	0\\
0.0733333333333333	1\\
0.0733541666666667	1\\
0.073375	1\\
0.0733958333333333	1\\
0.0734166666666667	1\\
0.0734375	1\\
0.0734583333333333	1\\
0.0734791666666667	1\\
0.0735	1\\
0.0735208333333333	1\\
0.0735416666666667	1\\
0.0735625	1\\
0.0735833333333333	1\\
0.0736041666666667	1\\
0.073625	1\\
0.0736458333333333	1\\
0.0736666666666667	1\\
0.0736875	1\\
0.0737083333333333	1\\
0.0737291666666667	1\\
0.07375	1\\
0.0737708333333333	1\\
0.0737916666666667	1\\
0.0738125	1\\
0.0738333333333333	1\\
0.0738541666666667	1\\
0.073875	1\\
0.0738958333333333	1\\
0.0739166666666667	1\\
0.0739375	1\\
0.0739583333333333	1\\
0.0739791666666667	1\\
0.074	1\\
0.0740208333333333	1\\
0.0740416666666667	1\\
0.0740625	1\\
0.0740833333333333	1\\
0.0741041666666667	1\\
0.074125	1\\
0.0741458333333333	1\\
0.0741666666666667	1\\
0.0741875	1\\
0.0742083333333333	1\\
0.0742291666666667	1\\
0.07425	1\\
0.0742708333333333	1\\
0.0742916666666667	1\\
0.0743125	1\\
0.0743333333333333	1\\
0.0743541666666667	1\\
0.074375	1\\
0.0743958333333333	1\\
0.0744166666666667	1\\
0.0744375	1\\
0.0744583333333333	1\\
0.0744791666666667	1\\
0.0745	1\\
0.0745208333333333	1\\
0.0745416666666667	1\\
0.0745625	1\\
0.0745833333333333	1\\
0.0746041666666667	1\\
0.074625	1\\
0.0746458333333333	1\\
0.0746666666666667	1\\
0.0746875	1\\
0.0747083333333333	1\\
0.0747291666666667	1\\
0.07475	1\\
0.0747708333333333	1\\
0.0747916666666667	1\\
0.0748125	1\\
0.0748333333333333	1\\
0.0748541666666667	1\\
0.074875	1\\
0.0748958333333333	1\\
0.0749166666666667	1\\
0.0749375	1\\
0.0749583333333333	1\\
0.0749791666666667	1\\
0.075	1\\
0.0750208333333333	1\\
0.0750416666666667	1\\
0.0750625	1\\
0.0750833333333333	1\\
0.0751041666666667	1\\
0.075125	1\\
0.0751458333333333	1\\
0.0751666666666667	1\\
0.0751875	1\\
0.0752083333333333	1\\
0.0752291666666667	1\\
0.07525	1\\
0.0752708333333333	1\\
0.0752916666666667	1\\
0.0753125	1\\
0.0753333333333333	1\\
0.0753541666666667	1\\
0.075375	1\\
0.0753958333333333	1\\
0.0754166666666667	1\\
0.0754375	1\\
0.0754583333333333	1\\
0.0754791666666667	1\\
0.0755	1\\
0.0755208333333333	1\\
0.0755416666666667	1\\
0.0755625	1\\
0.0755833333333333	1\\
0.0756041666666667	1\\
0.075625	1\\
0.0756458333333333	1\\
0.0756666666666667	1\\
0.0756875	1\\
0.0757083333333333	1\\
0.0757291666666667	1\\
0.07575	1\\
0.0757708333333333	1\\
0.0757916666666667	1\\
0.0758125	1\\
0.0758333333333333	1\\
0.0758541666666667	1\\
0.075875	1\\
0.0758958333333333	1\\
0.0759166666666667	1\\
0.0759375	1\\
0.0759583333333333	1\\
0.0759791666666667	1\\
0.076	1\\
0.0760208333333333	1\\
0.0760416666666667	1\\
0.0760625	1\\
0.0760833333333333	1\\
0.0761041666666667	1\\
0.076125	1\\
0.0761458333333333	1\\
0.0761666666666667	1\\
0.0761875	1\\
0.0762083333333333	1\\
0.0762291666666667	1\\
0.07625	1\\
0.0762708333333333	1\\
0.0762916666666667	1\\
0.0763125	1\\
0.0763333333333333	1\\
0.0763541666666667	1\\
0.076375	1\\
0.0763958333333333	1\\
0.0764166666666667	1\\
0.0764375	1\\
0.0764583333333333	1\\
0.0764791666666667	1\\
0.0765	1\\
0.0765208333333333	1\\
0.0765416666666667	1\\
0.0765625	1\\
0.0765833333333333	1\\
0.0766041666666667	1\\
0.076625	1\\
0.0766458333333333	1\\
0.0766666666666667	1\\
0.0766875	1\\
0.0767083333333333	1\\
0.0767291666666667	1\\
0.07675	1\\
0.0767708333333333	1\\
0.0767916666666667	1\\
0.0768125	1\\
0.0768333333333333	1\\
0.0768541666666667	1\\
0.076875	1\\
0.0768958333333333	1\\
0.0769166666666667	1\\
0.0769375	1\\
0.0769583333333333	1\\
0.0769791666666667	1\\
0.077	1\\
0.0770208333333333	1\\
0.0770416666666667	1\\
0.0770625	1\\
0.0770833333333333	1\\
0.0771041666666667	1\\
0.077125	1\\
0.0771458333333333	1\\
0.0771666666666667	1\\
0.0771875	1\\
0.0772083333333333	1\\
0.0772291666666667	1\\
0.07725	1\\
0.0772708333333333	1\\
0.0772916666666667	1\\
0.0773125	1\\
0.0773333333333333	1\\
0.0773541666666667	1\\
0.077375	1\\
0.0773958333333333	1\\
0.0774166666666667	1\\
0.0774375	1\\
0.0774583333333333	1\\
0.0774791666666667	1\\
0.0775	1\\
0.0775208333333333	1\\
0.0775416666666667	1\\
0.0775625	1\\
0.0775833333333333	1\\
0.0776041666666667	1\\
0.077625	1\\
0.0776458333333333	1\\
0.0776666666666667	1\\
0.0776875	1\\
0.0777083333333333	1\\
0.0777291666666667	1\\
0.07775	1\\
0.0777708333333333	1\\
0.0777916666666667	1\\
0.0778125	1\\
0.0778333333333333	1\\
0.0778541666666667	1\\
0.077875	1\\
0.0778958333333333	1\\
0.0779166666666667	1\\
0.0779375	1\\
0.0779583333333333	1\\
0.0779791666666667	1\\
0.078	1\\
0.0780208333333333	1\\
0.0780416666666667	1\\
0.0780625	1\\
0.0780833333333333	1\\
0.0781041666666667	1\\
0.078125	1\\
0.0781458333333333	1\\
0.0781666666666667	1\\
0.0781875	1\\
0.0782083333333333	1\\
0.0782291666666667	1\\
0.07825	1\\
0.0782708333333333	1\\
0.0782916666666667	1\\
0.0783125	1\\
0.0783333333333333	1\\
0.0783541666666667	1\\
0.078375	1\\
0.0783958333333333	1\\
0.0784166666666667	1\\
0.0784375	1\\
0.0784583333333333	1\\
0.0784791666666667	1\\
0.0785	1\\
0.0785208333333333	1\\
0.0785416666666667	1\\
0.0785625	1\\
0.0785833333333333	1\\
0.0786041666666667	1\\
0.078625	1\\
0.0786458333333333	1\\
0.0786666666666667	1\\
0.0786875	1\\
0.0787083333333333	1\\
0.0787291666666667	1\\
0.07875	1\\
0.0787708333333333	1\\
0.0787916666666667	1\\
0.0788125	1\\
0.0788333333333333	1\\
0.0788541666666667	1\\
0.078875	1\\
0.0788958333333333	1\\
0.0789166666666667	1\\
0.0789375	1\\
0.0789583333333333	1\\
0.0789791666666667	1\\
0.079	1\\
0.0790208333333333	1\\
0.0790416666666667	1\\
0.0790625	1\\
0.0790833333333333	1\\
0.0791041666666667	1\\
0.079125	1\\
0.0791458333333333	1\\
0.0791666666666667	1\\
0.0791875	1\\
0.0792083333333333	1\\
0.0792291666666667	1\\
0.07925	1\\
0.0792708333333333	1\\
0.0792916666666667	1\\
0.0793125	1\\
0.0793333333333333	1\\
0.0793541666666667	1\\
0.079375	1\\
0.0793958333333333	1\\
0.0794166666666667	1\\
0.0794375	1\\
0.0794583333333333	1\\
0.0794791666666667	1\\
0.0795	1\\
0.0795208333333333	1\\
0.0795416666666667	1\\
0.0795625	1\\
0.0795833333333333	1\\
0.0796041666666667	1\\
0.079625	1\\
0.0796458333333333	1\\
0.0796666666666667	1\\
0.0796875	1\\
0.0797083333333333	1\\
0.0797291666666667	1\\
0.07975	1\\
0.0797708333333333	1\\
0.0797916666666667	1\\
0.0798125	1\\
0.0798333333333333	1\\
0.0798541666666667	1\\
0.079875	1\\
0.0798958333333333	1\\
0.0799166666666667	1\\
0.0799375	1\\
0.0799583333333333	1\\
0.0799791666666667	1\\
0.08	0\\
0.0800208333333333	0\\
0.0800416666666667	0\\
0.0800625	0\\
0.0800833333333333	0\\
0.0801041666666667	0\\
0.080125	0\\
0.0801458333333333	0\\
0.0801666666666667	0\\
0.0801875	0\\
0.0802083333333333	0\\
0.0802291666666667	0\\
0.08025	0\\
0.0802708333333333	0\\
0.0802916666666667	0\\
0.0803125	0\\
0.0803333333333333	0\\
0.0803541666666667	0\\
0.080375	0\\
0.0803958333333333	0\\
0.0804166666666667	0\\
0.0804375	0\\
0.0804583333333333	0\\
0.0804791666666667	0\\
0.0805	0\\
0.0805208333333333	0\\
0.0805416666666667	0\\
0.0805625	0\\
0.0805833333333333	0\\
0.0806041666666667	0\\
0.080625	0\\
0.0806458333333333	0\\
0.0806666666666667	0\\
0.0806875	0\\
0.0807083333333333	0\\
0.0807291666666667	0\\
0.08075	0\\
0.0807708333333333	0\\
0.0807916666666667	0\\
0.0808125	0\\
0.0808333333333333	0\\
0.0808541666666667	0\\
0.080875	0\\
0.0808958333333333	0\\
0.0809166666666667	0\\
0.0809375	0\\
0.0809583333333333	0\\
0.0809791666666667	0\\
0.081	0\\
0.0810208333333333	0\\
0.0810416666666667	0\\
0.0810625	0\\
0.0810833333333333	0\\
0.0811041666666667	0\\
0.081125	0\\
0.0811458333333333	0\\
0.0811666666666667	0\\
0.0811875	0\\
0.0812083333333333	0\\
0.0812291666666667	0\\
0.08125	0\\
0.0812708333333333	0\\
0.0812916666666667	0\\
0.0813125	0\\
0.0813333333333333	0\\
0.0813541666666667	0\\
0.081375	0\\
0.0813958333333333	0\\
0.0814166666666667	0\\
0.0814375	0\\
0.0814583333333333	0\\
0.0814791666666667	0\\
0.0815	0\\
0.0815208333333333	0\\
0.0815416666666667	0\\
0.0815625	0\\
0.0815833333333333	0\\
0.0816041666666667	0\\
0.081625	0\\
0.0816458333333333	0\\
0.0816666666666667	0\\
0.0816875	0\\
0.0817083333333333	0\\
0.0817291666666667	0\\
0.08175	0\\
0.0817708333333333	0\\
0.0817916666666667	0\\
0.0818125	0\\
0.0818333333333333	0\\
0.0818541666666667	0\\
0.081875	0\\
0.0818958333333333	0\\
0.0819166666666667	0\\
0.0819375	0\\
0.0819583333333333	0\\
0.0819791666666667	0\\
0.082	0\\
0.0820208333333333	0\\
0.0820416666666667	0\\
0.0820625	0\\
0.0820833333333333	0\\
0.0821041666666667	0\\
0.082125	0\\
0.0821458333333333	0\\
0.0821666666666667	0\\
0.0821875	0\\
0.0822083333333333	0\\
0.0822291666666667	0\\
0.08225	0\\
0.0822708333333333	0\\
0.0822916666666667	0\\
0.0823125	0\\
0.0823333333333333	0\\
0.0823541666666667	0\\
0.082375	0\\
0.0823958333333333	0\\
0.0824166666666667	0\\
0.0824375	0\\
0.0824583333333333	0\\
0.0824791666666667	0\\
0.0825	0\\
0.0825208333333333	0\\
0.0825416666666667	0\\
0.0825625	0\\
0.0825833333333333	0\\
0.0826041666666667	0\\
0.082625	0\\
0.0826458333333333	0\\
0.0826666666666667	0\\
0.0826875	0\\
0.0827083333333333	0\\
0.0827291666666667	0\\
0.08275	0\\
0.0827708333333333	0\\
0.0827916666666667	0\\
0.0828125	0\\
0.0828333333333333	0\\
0.0828541666666667	0\\
0.082875	0\\
0.0828958333333333	0\\
0.0829166666666667	0\\
0.0829375	0\\
0.0829583333333333	0\\
0.0829791666666667	0\\
0.083	0\\
0.0830208333333333	0\\
0.0830416666666667	0\\
0.0830625	0\\
0.0830833333333333	0\\
0.0831041666666667	0\\
0.083125	0\\
0.0831458333333333	0\\
0.0831666666666667	0\\
0.0831875	0\\
0.0832083333333333	0\\
0.0832291666666667	0\\
0.08325	0\\
0.0832708333333333	0\\
0.0832916666666667	0\\
0.0833125	0\\
};
\end{axis}
\end{tikzpicture}%

\subsection{Densité spectrale de puissance}


On calcule ensuite la densité spectrale de puissance de ce signal NRZ en utilisant la fonction $pwelch$ de Matlab utilisant un périodogramme de Welch.

Puis la DSP théorique: 
\[
S_\text{NRZ}(f)=\frac{1}{4} T_s \sinc^2(\pi f T_s)+\frac{1}{4} \delta(f)
\]
On peut déormais comparer les densités spectrales de puissance théoriques et expérimentales:

% This file was created by matlab2tikz.
%
%The latest updates can be retrieved from
%  http://www.mathworks.com/matlabcentral/fileexchange/22022-matlab2tikz-matlab2tikz
%where you can also make suggestions and rate matlab2tikz.
%
\begin{tikzpicture}

\begin{axis}[%
width=3.809in,
height=3.548in,
at={(1.47in,0.499in)},
scale only axis,
xmin=-25000,
xmax=25000,
xlabel style={font=\color{white!15!black}},
xlabel={fréquence [Hz]},
ymode=log,
ymin=0.00083333333333324,
ymax=0.000833333333333423,
yminorticks=true,
ylabel style={font=\color{white!15!black}},
ylabel={Densité spectrale de puissance},
axis background/.style={fill=white},
title style={font=\bfseries},
title={Densité spectrale de puissance du signal NRZ aléatoire},
legend style={legend cell align=left, align=left, draw=white!15!black}
]
\addplot [color=blue]
  table[row sep=crcr]{%
-24000	0.00083333333333333\\
-23953.0791788856	0.00083333333333333\\
-23906.1583577713	0.00083333333333333\\
-23859.2375366569	0.00083333333333333\\
-23812.3167155425	0.00083333333333333\\
-23765.3958944282	0.00083333333333333\\
-23718.4750733138	0.00083333333333333\\
-23671.5542521994	0.00083333333333333\\
-23624.633431085	0.00083333333333333\\
-23577.7126099707	0.00083333333333333\\
-23530.7917888563	0.00083333333333333\\
-23483.8709677419	0.00083333333333333\\
-23436.9501466276	0.00083333333333333\\
-23390.0293255132	0.00083333333333333\\
-23343.1085043988	0.00083333333333333\\
-23296.1876832845	0.00083333333333333\\
-23249.2668621701	0.00083333333333333\\
-23202.3460410557	0.00083333333333333\\
-23155.4252199413	0.00083333333333333\\
-23108.504398827	0.00083333333333333\\
-23061.5835777126	0.00083333333333333\\
-23014.6627565982	0.00083333333333333\\
-22967.7419354839	0.00083333333333333\\
-22920.8211143695	0.00083333333333333\\
-22873.9002932551	0.00083333333333333\\
-22826.9794721408	0.00083333333333333\\
-22780.0586510264	0.00083333333333333\\
-22733.137829912	0.00083333333333333\\
-22686.2170087977	0.00083333333333333\\
-22639.2961876833	0.00083333333333333\\
-22592.3753665689	0.00083333333333333\\
-22545.4545454545	0.00083333333333333\\
-22498.5337243402	0.000833333333333331\\
-22451.6129032258	0.000833333333333331\\
-22404.6920821114	0.000833333333333331\\
-22357.7712609971	0.000833333333333331\\
-22310.8504398827	0.000833333333333331\\
-22263.9296187683	0.000833333333333331\\
-22217.008797654	0.000833333333333331\\
-22170.0879765396	0.000833333333333331\\
-22123.1671554252	0.000833333333333331\\
-22076.2463343109	0.000833333333333331\\
-22029.3255131965	0.000833333333333331\\
-21982.4046920821	0.00083333333333333\\
-21935.4838709677	0.000833333333333331\\
-21888.5630498534	0.000833333333333331\\
-21841.642228739	0.000833333333333331\\
-21794.7214076246	0.000833333333333331\\
-21747.8005865103	0.000833333333333331\\
-21700.8797653959	0.000833333333333331\\
-21653.9589442815	0.000833333333333331\\
-21607.0381231672	0.000833333333333331\\
-21560.1173020528	0.000833333333333331\\
-21513.1964809384	0.000833333333333331\\
-21466.275659824	0.000833333333333331\\
-21419.3548387097	0.000833333333333331\\
-21372.4340175953	0.000833333333333331\\
-21325.5131964809	0.000833333333333331\\
-21278.5923753666	0.000833333333333331\\
-21231.6715542522	0.000833333333333331\\
-21184.7507331378	0.000833333333333331\\
-21137.8299120235	0.000833333333333331\\
-21090.9090909091	0.000833333333333331\\
-21043.9882697947	0.000833333333333331\\
-20997.0674486804	0.000833333333333331\\
-20950.146627566	0.000833333333333331\\
-20903.2258064516	0.000833333333333331\\
-20856.3049853372	0.000833333333333331\\
-20809.3841642229	0.000833333333333331\\
-20762.4633431085	0.000833333333333331\\
-20715.5425219941	0.000833333333333331\\
-20668.6217008798	0.000833333333333331\\
-20621.7008797654	0.000833333333333331\\
-20574.780058651	0.000833333333333331\\
-20527.8592375367	0.000833333333333331\\
-20480.9384164223	0.000833333333333331\\
-20434.0175953079	0.000833333333333331\\
-20387.0967741935	0.000833333333333331\\
-20340.1759530792	0.000833333333333331\\
-20293.2551319648	0.000833333333333331\\
-20246.3343108504	0.000833333333333331\\
-20199.4134897361	0.000833333333333331\\
-20152.4926686217	0.000833333333333331\\
-20105.5718475073	0.000833333333333331\\
-20058.651026393	0.000833333333333331\\
-20011.7302052786	0.000833333333333331\\
-19964.8093841642	0.000833333333333331\\
-19917.8885630499	0.000833333333333331\\
-19870.9677419355	0.000833333333333331\\
-19824.0469208211	0.000833333333333331\\
-19777.1260997067	0.000833333333333331\\
-19730.2052785924	0.000833333333333331\\
-19683.284457478	0.000833333333333331\\
-19636.3636363636	0.000833333333333331\\
-19589.4428152493	0.000833333333333331\\
-19542.5219941349	0.000833333333333331\\
-19495.6011730205	0.000833333333333331\\
-19448.6803519062	0.000833333333333331\\
-19401.7595307918	0.000833333333333331\\
-19354.8387096774	0.000833333333333331\\
-19307.917888563	0.000833333333333331\\
-19260.9970674487	0.000833333333333331\\
-19214.0762463343	0.000833333333333331\\
-19167.1554252199	0.000833333333333331\\
-19120.2346041056	0.000833333333333331\\
-19073.3137829912	0.000833333333333331\\
-19026.3929618768	0.000833333333333331\\
-18979.4721407625	0.000833333333333331\\
-18932.5513196481	0.000833333333333331\\
-18885.6304985337	0.000833333333333331\\
-18838.7096774194	0.000833333333333331\\
-18791.788856305	0.000833333333333331\\
-18744.8680351906	0.000833333333333331\\
-18697.9472140762	0.000833333333333331\\
-18651.0263929619	0.000833333333333331\\
-18604.1055718475	0.000833333333333331\\
-18557.1847507331	0.000833333333333331\\
-18510.2639296188	0.000833333333333332\\
-18463.3431085044	0.000833333333333332\\
-18416.42228739	0.000833333333333332\\
-18369.5014662757	0.000833333333333332\\
-18322.5806451613	0.000833333333333332\\
-18275.6598240469	0.000833333333333332\\
-18228.7390029326	0.000833333333333332\\
-18181.8181818182	0.000833333333333332\\
-18134.8973607038	0.000833333333333332\\
-18087.9765395894	0.000833333333333332\\
-18041.0557184751	0.000833333333333332\\
-17994.1348973607	0.000833333333333332\\
-17947.2140762463	0.000833333333333332\\
-17900.293255132	0.000833333333333332\\
-17853.3724340176	0.000833333333333332\\
-17806.4516129032	0.000833333333333332\\
-17759.5307917889	0.000833333333333332\\
-17712.6099706745	0.000833333333333332\\
-17665.6891495601	0.000833333333333332\\
-17618.7683284457	0.000833333333333332\\
-17571.8475073314	0.000833333333333332\\
-17524.926686217	0.000833333333333332\\
-17478.0058651026	0.000833333333333332\\
-17431.0850439883	0.000833333333333332\\
-17384.1642228739	0.000833333333333332\\
-17337.2434017595	0.000833333333333332\\
-17290.3225806452	0.000833333333333332\\
-17243.4017595308	0.000833333333333332\\
-17196.4809384164	0.000833333333333332\\
-17149.5601173021	0.000833333333333332\\
-17102.6392961877	0.000833333333333332\\
-17055.7184750733	0.000833333333333332\\
-17008.7976539589	0.000833333333333332\\
-16961.8768328446	0.000833333333333332\\
-16914.9560117302	0.000833333333333332\\
-16868.0351906158	0.000833333333333332\\
-16821.1143695015	0.000833333333333332\\
-16774.1935483871	0.000833333333333332\\
-16727.2727272727	0.000833333333333332\\
-16680.3519061584	0.000833333333333332\\
-16633.431085044	0.000833333333333332\\
-16586.5102639296	0.000833333333333332\\
-16539.5894428152	0.000833333333333332\\
-16492.6686217009	0.000833333333333332\\
-16445.7478005865	0.000833333333333332\\
-16398.8269794721	0.000833333333333332\\
-16351.9061583578	0.000833333333333332\\
-16304.9853372434	0.000833333333333332\\
-16258.064516129	0.000833333333333332\\
-16211.1436950147	0.000833333333333332\\
-16164.2228739003	0.000833333333333332\\
-16117.3020527859	0.000833333333333332\\
-16070.3812316716	0.000833333333333332\\
-16023.4604105572	0.000833333333333332\\
-15976.5395894428	0.000833333333333332\\
-15929.6187683284	0.000833333333333332\\
-15882.6979472141	0.000833333333333332\\
-15835.7771260997	0.000833333333333332\\
-15788.8563049853	0.000833333333333332\\
-15741.935483871	0.000833333333333332\\
-15695.0146627566	0.000833333333333332\\
-15648.0938416422	0.000833333333333332\\
-15601.1730205279	0.000833333333333332\\
-15554.2521994135	0.000833333333333332\\
-15507.3313782991	0.000833333333333332\\
-15460.4105571848	0.000833333333333332\\
-15413.4897360704	0.000833333333333332\\
-15366.568914956	0.000833333333333332\\
-15319.6480938416	0.000833333333333332\\
-15272.7272727273	0.000833333333333332\\
-15225.8064516129	0.000833333333333332\\
-15178.8856304985	0.000833333333333332\\
-15131.9648093842	0.000833333333333332\\
-15085.0439882698	0.000833333333333332\\
-15038.1231671554	0.000833333333333332\\
-14991.2023460411	0.000833333333333332\\
-14944.2815249267	0.000833333333333332\\
-14897.3607038123	0.000833333333333332\\
-14850.4398826979	0.000833333333333332\\
-14803.5190615836	0.000833333333333332\\
-14756.5982404692	0.000833333333333332\\
-14709.6774193548	0.000833333333333332\\
-14662.7565982405	0.000833333333333332\\
-14615.8357771261	0.000833333333333332\\
-14568.9149560117	0.000833333333333332\\
-14521.9941348974	0.000833333333333332\\
-14475.073313783	0.000833333333333332\\
-14428.1524926686	0.000833333333333332\\
-14381.2316715543	0.000833333333333332\\
-14334.3108504399	0.000833333333333332\\
-14287.3900293255	0.000833333333333332\\
-14240.4692082111	0.000833333333333332\\
-14193.5483870968	0.000833333333333332\\
-14146.6275659824	0.000833333333333332\\
-14099.706744868	0.000833333333333332\\
-14052.7859237537	0.000833333333333332\\
-14005.8651026393	0.000833333333333332\\
-13958.9442815249	0.000833333333333332\\
-13912.0234604106	0.000833333333333332\\
-13865.1026392962	0.000833333333333332\\
-13818.1818181818	0.000833333333333332\\
-13771.2609970674	0.000833333333333332\\
-13724.3401759531	0.000833333333333332\\
-13677.4193548387	0.000833333333333332\\
-13630.4985337243	0.000833333333333332\\
-13583.57771261	0.000833333333333332\\
-13536.6568914956	0.000833333333333332\\
-13489.7360703812	0.000833333333333332\\
-13442.8152492669	0.000833333333333332\\
-13395.8944281525	0.000833333333333332\\
-13348.9736070381	0.000833333333333332\\
-13302.0527859238	0.000833333333333332\\
-13255.1319648094	0.000833333333333332\\
-13208.211143695	0.000833333333333332\\
-13161.2903225806	0.000833333333333332\\
-13114.3695014663	0.000833333333333332\\
-13067.4486803519	0.000833333333333332\\
-13020.5278592375	0.000833333333333332\\
-12973.6070381232	0.000833333333333332\\
-12926.6862170088	0.000833333333333332\\
-12879.7653958944	0.000833333333333332\\
-12832.8445747801	0.000833333333333332\\
-12785.9237536657	0.000833333333333332\\
-12739.0029325513	0.000833333333333332\\
-12692.082111437	0.000833333333333332\\
-12645.1612903226	0.000833333333333332\\
-12598.2404692082	0.000833333333333332\\
-12551.3196480938	0.000833333333333332\\
-12504.3988269795	0.000833333333333332\\
-12457.4780058651	0.000833333333333332\\
-12410.5571847507	0.000833333333333332\\
-12363.6363636364	0.000833333333333333\\
-12316.715542522	0.000833333333333333\\
-12269.7947214076	0.000833333333333333\\
-12222.8739002933	0.000833333333333333\\
-12175.9530791789	0.000833333333333333\\
-12129.0322580645	0.000833333333333333\\
-12082.1114369501	0.000833333333333333\\
-12035.1906158358	0.000833333333333333\\
-11988.2697947214	0.000833333333333333\\
-11941.348973607	0.000833333333333333\\
-11894.4281524927	0.000833333333333333\\
-11847.5073313783	0.000833333333333333\\
-11800.5865102639	0.000833333333333333\\
-11753.6656891496	0.000833333333333333\\
-11706.7448680352	0.000833333333333333\\
-11659.8240469208	0.000833333333333333\\
-11612.9032258065	0.000833333333333332\\
-11565.9824046921	0.000833333333333332\\
-11519.0615835777	0.000833333333333332\\
-11472.1407624633	0.000833333333333332\\
-11425.219941349	0.000833333333333332\\
-11378.2991202346	0.000833333333333333\\
-11331.3782991202	0.000833333333333333\\
-11284.4574780059	0.000833333333333333\\
-11237.5366568915	0.000833333333333333\\
-11190.6158357771	0.000833333333333333\\
-11143.6950146628	0.000833333333333333\\
-11096.7741935484	0.000833333333333333\\
-11049.853372434	0.000833333333333333\\
-11002.9325513196	0.000833333333333333\\
-10956.0117302053	0.000833333333333333\\
-10909.0909090909	0.000833333333333333\\
-10862.1700879765	0.000833333333333333\\
-10815.2492668622	0.000833333333333333\\
-10768.3284457478	0.000833333333333333\\
-10721.4076246334	0.000833333333333333\\
-10674.4868035191	0.000833333333333333\\
-10627.5659824047	0.000833333333333333\\
-10580.6451612903	0.000833333333333333\\
-10533.724340176	0.000833333333333333\\
-10486.8035190616	0.000833333333333333\\
-10439.8826979472	0.000833333333333333\\
-10392.9618768328	0.000833333333333333\\
-10346.0410557185	0.000833333333333333\\
-10299.1202346041	0.000833333333333333\\
-10252.1994134897	0.000833333333333333\\
-10205.2785923754	0.000833333333333333\\
-10158.357771261	0.000833333333333333\\
-10111.4369501466	0.000833333333333333\\
-10064.5161290323	0.000833333333333333\\
-10017.5953079179	0.000833333333333333\\
-9970.67448680352	0.000833333333333333\\
-9923.75366568915	0.000833333333333333\\
-9876.83284457478	0.000833333333333333\\
-9829.91202346041	0.000833333333333333\\
-9782.99120234604	0.000833333333333333\\
-9736.07038123167	0.000833333333333333\\
-9689.1495601173	0.000833333333333333\\
-9642.22873900293	0.000833333333333333\\
-9595.30791788856	0.000833333333333333\\
-9548.38709677419	0.000833333333333333\\
-9501.46627565982	0.000833333333333333\\
-9454.54545454545	0.000833333333333333\\
-9407.62463343109	0.000833333333333333\\
-9360.70381231671	0.000833333333333333\\
-9313.78299120235	0.000833333333333333\\
-9266.86217008798	0.000833333333333333\\
-9219.94134897361	0.000833333333333333\\
-9173.02052785924	0.000833333333333333\\
-9126.09970674487	0.000833333333333333\\
-9079.1788856305	0.000833333333333333\\
-9032.25806451613	0.000833333333333333\\
-8985.33724340176	0.000833333333333333\\
-8938.41642228739	0.000833333333333333\\
-8891.49560117302	0.000833333333333333\\
-8844.57478005865	0.000833333333333333\\
-8797.65395894428	0.000833333333333333\\
-8750.73313782991	0.000833333333333333\\
-8703.81231671554	0.000833333333333333\\
-8656.89149560117	0.000833333333333333\\
-8609.9706744868	0.000833333333333333\\
-8563.04985337243	0.000833333333333333\\
-8516.12903225806	0.000833333333333333\\
-8469.2082111437	0.000833333333333333\\
-8422.28739002933	0.000833333333333333\\
-8375.36656891496	0.000833333333333333\\
-8328.44574780059	0.000833333333333333\\
-8281.52492668622	0.000833333333333333\\
-8234.60410557185	0.000833333333333333\\
-8187.68328445748	0.000833333333333333\\
-8140.76246334311	0.000833333333333333\\
-8093.84164222874	0.000833333333333333\\
-8046.92082111437	0.000833333333333333\\
-8000	0.000833333333333333\\
-7953.07917888563	0.000833333333333333\\
-7906.15835777126	0.000833333333333333\\
-7859.23753665689	0.000833333333333333\\
-7812.31671554252	0.000833333333333333\\
-7765.39589442815	0.000833333333333333\\
-7718.47507331378	0.000833333333333333\\
-7671.55425219941	0.000833333333333333\\
-7624.63343108504	0.000833333333333333\\
-7577.71260997067	0.000833333333333333\\
-7530.79178885631	0.000833333333333333\\
-7483.87096774194	0.000833333333333333\\
-7436.95014662757	0.000833333333333333\\
-7390.0293255132	0.000833333333333333\\
-7343.10850439883	0.000833333333333333\\
-7296.18768328446	0.000833333333333333\\
-7249.26686217009	0.000833333333333333\\
-7202.34604105572	0.000833333333333333\\
-7155.42521994135	0.000833333333333333\\
-7108.50439882698	0.000833333333333333\\
-7061.58357771261	0.000833333333333333\\
-7014.66275659824	0.000833333333333333\\
-6967.74193548387	0.000833333333333333\\
-6920.8211143695	0.000833333333333333\\
-6873.90029325513	0.000833333333333333\\
-6826.97947214076	0.000833333333333333\\
-6780.05865102639	0.000833333333333333\\
-6733.13782991202	0.000833333333333333\\
-6686.21700879765	0.000833333333333333\\
-6639.29618768328	0.000833333333333333\\
-6592.37536656892	0.000833333333333333\\
-6545.45454545455	0.000833333333333333\\
-6498.53372434018	0.000833333333333333\\
-6451.61290322581	0.000833333333333333\\
-6404.69208211144	0.000833333333333333\\
-6357.77126099707	0.000833333333333333\\
-6310.8504398827	0.000833333333333333\\
-6263.92961876833	0.000833333333333333\\
-6217.00879765396	0.000833333333333333\\
-6170.08797653959	0.000833333333333333\\
-6123.16715542522	0.000833333333333333\\
-6076.24633431085	0.000833333333333333\\
-6029.32551319648	0.000833333333333333\\
-5982.40469208211	0.000833333333333333\\
-5935.48387096774	0.000833333333333333\\
-5888.56304985337	0.000833333333333333\\
-5841.642228739	0.000833333333333333\\
-5794.72140762463	0.000833333333333333\\
-5747.80058651026	0.000833333333333333\\
-5700.87976539589	0.000833333333333333\\
-5653.95894428152	0.000833333333333333\\
-5607.03812316716	0.000833333333333333\\
-5560.11730205279	0.000833333333333333\\
-5513.19648093842	0.000833333333333333\\
-5466.27565982405	0.000833333333333333\\
-5419.35483870968	0.000833333333333333\\
-5372.43401759531	0.000833333333333333\\
-5325.51319648094	0.000833333333333333\\
-5278.59237536657	0.000833333333333333\\
-5231.6715542522	0.000833333333333333\\
-5184.75073313783	0.000833333333333333\\
-5137.82991202346	0.000833333333333333\\
-5090.90909090909	0.000833333333333333\\
-5043.98826979472	0.000833333333333333\\
-4997.06744868035	0.000833333333333333\\
-4950.14662756598	0.000833333333333333\\
-4903.22580645161	0.000833333333333333\\
-4856.30498533724	0.000833333333333333\\
-4809.38416422287	0.000833333333333333\\
-4762.4633431085	0.000833333333333333\\
-4715.54252199414	0.000833333333333333\\
-4668.62170087977	0.000833333333333333\\
-4621.7008797654	0.000833333333333333\\
-4574.78005865103	0.000833333333333333\\
-4527.85923753666	0.000833333333333333\\
-4480.93841642229	0.000833333333333333\\
-4434.01759530792	0.000833333333333333\\
-4387.09677419355	0.000833333333333333\\
-4340.17595307918	0.000833333333333333\\
-4293.25513196481	0.000833333333333333\\
-4246.33431085044	0.000833333333333333\\
-4199.41348973607	0.000833333333333333\\
-4152.4926686217	0.000833333333333333\\
-4105.57184750733	0.000833333333333333\\
-4058.65102639296	0.000833333333333333\\
-4011.73020527859	0.000833333333333333\\
-3964.80938416422	0.000833333333333333\\
-3917.88856304985	0.000833333333333333\\
-3870.96774193548	0.000833333333333333\\
-3824.04692082111	0.000833333333333333\\
-3777.12609970674	0.000833333333333333\\
-3730.20527859238	0.000833333333333333\\
-3683.28445747801	0.000833333333333333\\
-3636.36363636364	0.000833333333333333\\
-3589.44281524927	0.000833333333333333\\
-3542.5219941349	0.000833333333333333\\
-3495.60117302053	0.000833333333333333\\
-3448.68035190616	0.000833333333333333\\
-3401.75953079179	0.000833333333333333\\
-3354.83870967742	0.000833333333333333\\
-3307.91788856305	0.000833333333333333\\
-3260.99706744868	0.000833333333333333\\
-3214.07624633431	0.000833333333333333\\
-3167.15542521994	0.000833333333333333\\
-3120.23460410557	0.000833333333333333\\
-3073.3137829912	0.000833333333333333\\
-3026.39296187683	0.000833333333333333\\
-2979.47214076246	0.000833333333333333\\
-2932.55131964809	0.000833333333333333\\
-2885.63049853372	0.000833333333333333\\
-2838.70967741936	0.000833333333333333\\
-2791.78885630499	0.000833333333333333\\
-2744.86803519062	0.000833333333333333\\
-2697.94721407625	0.000833333333333333\\
-2651.02639296188	0.000833333333333333\\
-2604.10557184751	0.000833333333333333\\
-2557.18475073314	0.000833333333333333\\
-2510.26392961877	0.000833333333333333\\
-2463.3431085044	0.000833333333333333\\
-2416.42228739003	0.000833333333333333\\
-2369.50146627566	0.000833333333333333\\
-2322.58064516129	0.000833333333333333\\
-2275.65982404692	0.000833333333333333\\
-2228.73900293255	0.000833333333333333\\
-2181.81818181818	0.000833333333333333\\
-2134.89736070381	0.000833333333333333\\
-2087.97653958944	0.000833333333333333\\
-2041.05571847507	0.000833333333333333\\
-1994.1348973607	0.000833333333333333\\
-1947.21407624633	0.000833333333333333\\
-1900.29325513196	0.000833333333333333\\
-1853.3724340176	0.000833333333333333\\
-1806.45161290323	0.000833333333333333\\
-1759.53079178886	0.000833333333333333\\
-1712.60997067449	0.000833333333333333\\
-1665.68914956012	0.000833333333333333\\
-1618.76832844575	0.000833333333333333\\
-1571.84750733138	0.000833333333333333\\
-1524.92668621701	0.000833333333333333\\
-1478.00586510264	0.000833333333333333\\
-1431.08504398827	0.000833333333333333\\
-1384.1642228739	0.000833333333333333\\
-1337.24340175953	0.000833333333333333\\
-1290.32258064516	0.000833333333333333\\
-1243.40175953079	0.000833333333333333\\
-1196.48093841642	0.000833333333333333\\
-1149.56011730205	0.000833333333333333\\
-1102.63929618768	0.000833333333333333\\
-1055.71847507331	0.000833333333333333\\
-1008.79765395894	0.000833333333333333\\
-961.876832844575	0.000833333333333333\\
-914.956011730205	0.000833333333333333\\
-868.035190615836	0.000833333333333333\\
-821.114369501466	0.000833333333333333\\
-774.193548387097	0.000833333333333333\\
-727.272727272727	0.000833333333333333\\
-680.351906158358	0.000833333333333333\\
-633.431085043988	0.000833333333333333\\
-586.510263929619	0.000833333333333333\\
-539.589442815249	0.000833333333333333\\
-492.66862170088	0.000833333333333333\\
-445.74780058651	0.000833333333333333\\
-398.826979472141	0.000833333333333333\\
-351.906158357771	0.000833333333333333\\
-304.985337243402	0.000833333333333333\\
-258.064516129032	0.000833333333333333\\
-211.143695014663	0.000833333333333333\\
-164.222873900293	0.000833333333333333\\
-117.302052785924	0.000833333333333333\\
-70.3812316715542	0.000833333333333333\\
-23.4604105571848	0.000833333333333333\\
23.4604105571848	0.000833333333333333\\
70.3812316715542	0.000833333333333333\\
117.302052785924	0.000833333333333333\\
164.222873900293	0.000833333333333333\\
211.143695014663	0.000833333333333333\\
258.064516129032	0.000833333333333333\\
304.985337243402	0.000833333333333333\\
351.906158357771	0.000833333333333333\\
398.826979472141	0.000833333333333333\\
445.74780058651	0.000833333333333333\\
492.66862170088	0.000833333333333333\\
539.589442815249	0.000833333333333333\\
586.510263929619	0.000833333333333333\\
633.431085043988	0.000833333333333333\\
680.351906158358	0.000833333333333333\\
727.272727272727	0.000833333333333333\\
774.193548387097	0.000833333333333333\\
821.114369501466	0.000833333333333333\\
868.035190615836	0.000833333333333333\\
914.956011730205	0.000833333333333333\\
961.876832844575	0.000833333333333333\\
1008.79765395894	0.000833333333333333\\
1055.71847507331	0.000833333333333333\\
1102.63929618768	0.000833333333333333\\
1149.56011730205	0.000833333333333333\\
1196.48093841642	0.000833333333333333\\
1243.40175953079	0.000833333333333333\\
1290.32258064516	0.000833333333333333\\
1337.24340175953	0.000833333333333333\\
1384.1642228739	0.000833333333333333\\
1431.08504398827	0.000833333333333333\\
1478.00586510264	0.000833333333333333\\
1524.92668621701	0.000833333333333333\\
1571.84750733138	0.000833333333333333\\
1618.76832844575	0.000833333333333333\\
1665.68914956012	0.000833333333333333\\
1712.60997067449	0.000833333333333333\\
1759.53079178886	0.000833333333333333\\
1806.45161290323	0.000833333333333333\\
1853.3724340176	0.000833333333333333\\
1900.29325513196	0.000833333333333333\\
1947.21407624633	0.000833333333333333\\
1994.1348973607	0.000833333333333333\\
2041.05571847507	0.000833333333333333\\
2087.97653958944	0.000833333333333333\\
2134.89736070381	0.000833333333333333\\
2181.81818181818	0.000833333333333333\\
2228.73900293255	0.000833333333333333\\
2275.65982404692	0.000833333333333333\\
2322.58064516129	0.000833333333333333\\
2369.50146627566	0.000833333333333333\\
2416.42228739003	0.000833333333333333\\
2463.3431085044	0.000833333333333333\\
2510.26392961877	0.000833333333333333\\
2557.18475073314	0.000833333333333333\\
2604.10557184751	0.000833333333333333\\
2651.02639296188	0.000833333333333333\\
2697.94721407625	0.000833333333333333\\
2744.86803519062	0.000833333333333333\\
2791.78885630499	0.000833333333333333\\
2838.70967741936	0.000833333333333333\\
2885.63049853372	0.000833333333333333\\
2932.55131964809	0.000833333333333333\\
2979.47214076246	0.000833333333333333\\
3026.39296187683	0.000833333333333333\\
3073.3137829912	0.000833333333333333\\
3120.23460410557	0.000833333333333333\\
3167.15542521994	0.000833333333333333\\
3214.07624633431	0.000833333333333333\\
3260.99706744868	0.000833333333333333\\
3307.91788856305	0.000833333333333333\\
3354.83870967742	0.000833333333333333\\
3401.75953079179	0.000833333333333333\\
3448.68035190616	0.000833333333333333\\
3495.60117302053	0.000833333333333333\\
3542.5219941349	0.000833333333333333\\
3589.44281524927	0.000833333333333333\\
3636.36363636364	0.000833333333333333\\
3683.28445747801	0.000833333333333333\\
3730.20527859238	0.000833333333333333\\
3777.12609970674	0.000833333333333333\\
3824.04692082111	0.000833333333333333\\
3870.96774193548	0.000833333333333333\\
3917.88856304985	0.000833333333333333\\
3964.80938416422	0.000833333333333333\\
4011.73020527859	0.000833333333333333\\
4058.65102639296	0.000833333333333333\\
4105.57184750733	0.000833333333333333\\
4152.4926686217	0.000833333333333333\\
4199.41348973607	0.000833333333333333\\
4246.33431085044	0.000833333333333333\\
4293.25513196481	0.000833333333333333\\
4340.17595307918	0.000833333333333333\\
4387.09677419355	0.000833333333333333\\
4434.01759530792	0.000833333333333333\\
4480.93841642229	0.000833333333333333\\
4527.85923753666	0.000833333333333333\\
4574.78005865103	0.000833333333333333\\
4621.7008797654	0.000833333333333333\\
4668.62170087977	0.000833333333333333\\
4715.54252199414	0.000833333333333333\\
4762.4633431085	0.000833333333333333\\
4809.38416422287	0.000833333333333333\\
4856.30498533724	0.000833333333333333\\
4903.22580645161	0.000833333333333333\\
4950.14662756598	0.000833333333333333\\
4997.06744868035	0.000833333333333333\\
5043.98826979472	0.000833333333333333\\
5090.90909090909	0.000833333333333333\\
5137.82991202346	0.000833333333333333\\
5184.75073313783	0.000833333333333333\\
5231.6715542522	0.000833333333333333\\
5278.59237536657	0.000833333333333333\\
5325.51319648094	0.000833333333333333\\
5372.43401759531	0.000833333333333333\\
5419.35483870968	0.000833333333333333\\
5466.27565982405	0.000833333333333333\\
5513.19648093842	0.000833333333333333\\
5560.11730205279	0.000833333333333333\\
5607.03812316716	0.000833333333333333\\
5653.95894428152	0.000833333333333333\\
5700.87976539589	0.000833333333333333\\
5747.80058651026	0.000833333333333333\\
5794.72140762463	0.000833333333333333\\
5841.642228739	0.000833333333333333\\
5888.56304985337	0.000833333333333333\\
5935.48387096774	0.000833333333333333\\
5982.40469208211	0.000833333333333333\\
6029.32551319648	0.000833333333333333\\
6076.24633431085	0.000833333333333333\\
6123.16715542522	0.000833333333333333\\
6170.08797653959	0.000833333333333333\\
6217.00879765396	0.000833333333333333\\
6263.92961876833	0.000833333333333333\\
6310.8504398827	0.000833333333333333\\
6357.77126099707	0.000833333333333333\\
6404.69208211144	0.000833333333333333\\
6451.61290322581	0.000833333333333333\\
6498.53372434018	0.000833333333333333\\
6545.45454545455	0.000833333333333333\\
6592.37536656892	0.000833333333333333\\
6639.29618768328	0.000833333333333333\\
6686.21700879765	0.000833333333333333\\
6733.13782991202	0.000833333333333333\\
6780.05865102639	0.000833333333333333\\
6826.97947214076	0.000833333333333333\\
6873.90029325513	0.000833333333333333\\
6920.8211143695	0.000833333333333333\\
6967.74193548387	0.000833333333333333\\
7014.66275659824	0.000833333333333333\\
7061.58357771261	0.000833333333333333\\
7108.50439882698	0.000833333333333333\\
7155.42521994135	0.000833333333333333\\
7202.34604105572	0.000833333333333333\\
7249.26686217009	0.000833333333333333\\
7296.18768328446	0.000833333333333333\\
7343.10850439883	0.000833333333333333\\
7390.0293255132	0.000833333333333333\\
7436.95014662757	0.000833333333333333\\
7483.87096774194	0.000833333333333333\\
7530.79178885631	0.000833333333333333\\
7577.71260997067	0.000833333333333333\\
7624.63343108504	0.000833333333333333\\
7671.55425219941	0.000833333333333333\\
7718.47507331378	0.000833333333333333\\
7765.39589442815	0.000833333333333333\\
7812.31671554252	0.000833333333333333\\
7859.23753665689	0.000833333333333333\\
7906.15835777126	0.000833333333333333\\
7953.07917888563	0.000833333333333333\\
8000	0.000833333333333333\\
8046.92082111437	0.000833333333333333\\
8093.84164222874	0.000833333333333333\\
8140.76246334311	0.000833333333333333\\
8187.68328445748	0.000833333333333333\\
8234.60410557185	0.000833333333333333\\
8281.52492668622	0.000833333333333333\\
8328.44574780059	0.000833333333333333\\
8375.36656891496	0.000833333333333333\\
8422.28739002933	0.000833333333333333\\
8469.2082111437	0.000833333333333333\\
8516.12903225806	0.000833333333333333\\
8563.04985337243	0.000833333333333333\\
8609.9706744868	0.000833333333333333\\
8656.89149560117	0.000833333333333333\\
8703.81231671554	0.000833333333333333\\
8750.73313782991	0.000833333333333333\\
8797.65395894428	0.000833333333333333\\
8844.57478005865	0.000833333333333333\\
8891.49560117302	0.000833333333333333\\
8938.41642228739	0.000833333333333333\\
8985.33724340176	0.000833333333333333\\
9032.25806451613	0.000833333333333333\\
9079.1788856305	0.000833333333333333\\
9126.09970674487	0.000833333333333333\\
9173.02052785924	0.000833333333333333\\
9219.94134897361	0.000833333333333333\\
9266.86217008798	0.000833333333333333\\
9313.78299120235	0.000833333333333333\\
9360.70381231671	0.000833333333333333\\
9407.62463343109	0.000833333333333333\\
9454.54545454545	0.000833333333333333\\
9501.46627565982	0.000833333333333333\\
9548.38709677419	0.000833333333333333\\
9595.30791788856	0.000833333333333333\\
9642.22873900293	0.000833333333333333\\
9689.1495601173	0.000833333333333333\\
9736.07038123167	0.000833333333333333\\
9782.99120234604	0.000833333333333333\\
9829.91202346041	0.000833333333333333\\
9876.83284457478	0.000833333333333333\\
9923.75366568915	0.000833333333333333\\
9970.67448680352	0.000833333333333333\\
10017.5953079179	0.000833333333333333\\
10064.5161290323	0.000833333333333333\\
10111.4369501466	0.000833333333333333\\
10158.357771261	0.000833333333333333\\
10205.2785923754	0.000833333333333333\\
10252.1994134897	0.000833333333333333\\
10299.1202346041	0.000833333333333333\\
10346.0410557185	0.000833333333333333\\
10392.9618768328	0.000833333333333333\\
10439.8826979472	0.000833333333333333\\
10486.8035190616	0.000833333333333333\\
10533.724340176	0.000833333333333333\\
10580.6451612903	0.000833333333333333\\
10627.5659824047	0.000833333333333333\\
10674.4868035191	0.000833333333333333\\
10721.4076246334	0.000833333333333333\\
10768.3284457478	0.000833333333333333\\
10815.2492668622	0.000833333333333333\\
10862.1700879765	0.000833333333333333\\
10909.0909090909	0.000833333333333333\\
10956.0117302053	0.000833333333333333\\
11002.9325513196	0.000833333333333333\\
11049.853372434	0.000833333333333333\\
11096.7741935484	0.000833333333333333\\
11143.6950146628	0.000833333333333333\\
11190.6158357771	0.000833333333333333\\
11237.5366568915	0.000833333333333333\\
11284.4574780059	0.000833333333333333\\
11331.3782991202	0.000833333333333333\\
11378.2991202346	0.000833333333333333\\
11425.219941349	0.000833333333333332\\
11472.1407624633	0.000833333333333332\\
11519.0615835777	0.000833333333333332\\
11565.9824046921	0.000833333333333332\\
11612.9032258065	0.000833333333333332\\
11659.8240469208	0.000833333333333333\\
11706.7448680352	0.000833333333333333\\
11753.6656891496	0.000833333333333333\\
11800.5865102639	0.000833333333333333\\
11847.5073313783	0.000833333333333333\\
11894.4281524927	0.000833333333333333\\
11941.348973607	0.000833333333333333\\
11988.2697947214	0.000833333333333333\\
12035.1906158358	0.000833333333333333\\
12082.1114369501	0.000833333333333333\\
12129.0322580645	0.000833333333333333\\
12175.9530791789	0.000833333333333333\\
12222.8739002933	0.000833333333333333\\
12269.7947214076	0.000833333333333333\\
12316.715542522	0.000833333333333333\\
12363.6363636364	0.000833333333333333\\
12410.5571847507	0.000833333333333332\\
12457.4780058651	0.000833333333333332\\
12504.3988269795	0.000833333333333332\\
12551.3196480938	0.000833333333333332\\
12598.2404692082	0.000833333333333332\\
12645.1612903226	0.000833333333333332\\
12692.082111437	0.000833333333333332\\
12739.0029325513	0.000833333333333332\\
12785.9237536657	0.000833333333333332\\
12832.8445747801	0.000833333333333332\\
12879.7653958944	0.000833333333333332\\
12926.6862170088	0.000833333333333332\\
12973.6070381232	0.000833333333333332\\
13020.5278592375	0.000833333333333332\\
13067.4486803519	0.000833333333333332\\
13114.3695014663	0.000833333333333332\\
13161.2903225806	0.000833333333333332\\
13208.211143695	0.000833333333333332\\
13255.1319648094	0.000833333333333332\\
13302.0527859238	0.000833333333333332\\
13348.9736070381	0.000833333333333332\\
13395.8944281525	0.000833333333333332\\
13442.8152492669	0.000833333333333332\\
13489.7360703812	0.000833333333333332\\
13536.6568914956	0.000833333333333332\\
13583.57771261	0.000833333333333332\\
13630.4985337243	0.000833333333333332\\
13677.4193548387	0.000833333333333332\\
13724.3401759531	0.000833333333333332\\
13771.2609970674	0.000833333333333332\\
13818.1818181818	0.000833333333333332\\
13865.1026392962	0.000833333333333332\\
13912.0234604106	0.000833333333333332\\
13958.9442815249	0.000833333333333332\\
14005.8651026393	0.000833333333333332\\
14052.7859237537	0.000833333333333332\\
14099.706744868	0.000833333333333332\\
14146.6275659824	0.000833333333333332\\
14193.5483870968	0.000833333333333332\\
14240.4692082111	0.000833333333333332\\
14287.3900293255	0.000833333333333332\\
14334.3108504399	0.000833333333333332\\
14381.2316715543	0.000833333333333332\\
14428.1524926686	0.000833333333333332\\
14475.073313783	0.000833333333333332\\
14521.9941348974	0.000833333333333332\\
14568.9149560117	0.000833333333333332\\
14615.8357771261	0.000833333333333332\\
14662.7565982405	0.000833333333333332\\
14709.6774193548	0.000833333333333332\\
14756.5982404692	0.000833333333333332\\
14803.5190615836	0.000833333333333332\\
14850.4398826979	0.000833333333333332\\
14897.3607038123	0.000833333333333332\\
14944.2815249267	0.000833333333333332\\
14991.2023460411	0.000833333333333332\\
15038.1231671554	0.000833333333333332\\
15085.0439882698	0.000833333333333332\\
15131.9648093842	0.000833333333333332\\
15178.8856304985	0.000833333333333332\\
15225.8064516129	0.000833333333333332\\
15272.7272727273	0.000833333333333332\\
15319.6480938416	0.000833333333333332\\
15366.568914956	0.000833333333333332\\
15413.4897360704	0.000833333333333332\\
15460.4105571848	0.000833333333333332\\
15507.3313782991	0.000833333333333332\\
15554.2521994135	0.000833333333333332\\
15601.1730205279	0.000833333333333332\\
15648.0938416422	0.000833333333333332\\
15695.0146627566	0.000833333333333332\\
15741.935483871	0.000833333333333332\\
15788.8563049853	0.000833333333333332\\
15835.7771260997	0.000833333333333332\\
15882.6979472141	0.000833333333333332\\
15929.6187683284	0.000833333333333332\\
15976.5395894428	0.000833333333333332\\
16023.4604105572	0.000833333333333332\\
16070.3812316716	0.000833333333333332\\
16117.3020527859	0.000833333333333332\\
16164.2228739003	0.000833333333333332\\
16211.1436950147	0.000833333333333332\\
16258.064516129	0.000833333333333332\\
16304.9853372434	0.000833333333333332\\
16351.9061583578	0.000833333333333332\\
16398.8269794721	0.000833333333333332\\
16445.7478005865	0.000833333333333332\\
16492.6686217009	0.000833333333333332\\
16539.5894428152	0.000833333333333332\\
16586.5102639296	0.000833333333333332\\
16633.431085044	0.000833333333333332\\
16680.3519061584	0.000833333333333332\\
16727.2727272727	0.000833333333333332\\
16774.1935483871	0.000833333333333332\\
16821.1143695015	0.000833333333333332\\
16868.0351906158	0.000833333333333332\\
16914.9560117302	0.000833333333333332\\
16961.8768328446	0.000833333333333332\\
17008.7976539589	0.000833333333333332\\
17055.7184750733	0.000833333333333332\\
17102.6392961877	0.000833333333333332\\
17149.5601173021	0.000833333333333332\\
17196.4809384164	0.000833333333333332\\
17243.4017595308	0.000833333333333332\\
17290.3225806452	0.000833333333333332\\
17337.2434017595	0.000833333333333332\\
17384.1642228739	0.000833333333333332\\
17431.0850439883	0.000833333333333332\\
17478.0058651026	0.000833333333333332\\
17524.926686217	0.000833333333333332\\
17571.8475073314	0.000833333333333332\\
17618.7683284457	0.000833333333333332\\
17665.6891495601	0.000833333333333332\\
17712.6099706745	0.000833333333333332\\
17759.5307917889	0.000833333333333332\\
17806.4516129032	0.000833333333333332\\
17853.3724340176	0.000833333333333332\\
17900.293255132	0.000833333333333332\\
17947.2140762463	0.000833333333333332\\
17994.1348973607	0.000833333333333332\\
18041.0557184751	0.000833333333333332\\
18087.9765395894	0.000833333333333332\\
18134.8973607038	0.000833333333333332\\
18181.8181818182	0.000833333333333332\\
18228.7390029326	0.000833333333333332\\
18275.6598240469	0.000833333333333332\\
18322.5806451613	0.000833333333333332\\
18369.5014662757	0.000833333333333332\\
18416.42228739	0.000833333333333332\\
18463.3431085044	0.000833333333333332\\
18510.2639296188	0.000833333333333332\\
18557.1847507331	0.000833333333333331\\
18604.1055718475	0.000833333333333331\\
18651.0263929619	0.000833333333333331\\
18697.9472140762	0.000833333333333331\\
18744.8680351906	0.000833333333333331\\
18791.788856305	0.000833333333333331\\
18838.7096774194	0.000833333333333331\\
18885.6304985337	0.000833333333333331\\
18932.5513196481	0.000833333333333331\\
18979.4721407625	0.000833333333333331\\
19026.3929618768	0.000833333333333331\\
19073.3137829912	0.000833333333333331\\
19120.2346041056	0.000833333333333331\\
19167.1554252199	0.000833333333333331\\
19214.0762463343	0.000833333333333331\\
19260.9970674487	0.000833333333333331\\
19307.917888563	0.000833333333333331\\
19354.8387096774	0.000833333333333331\\
19401.7595307918	0.000833333333333331\\
19448.6803519062	0.000833333333333331\\
19495.6011730205	0.000833333333333331\\
19542.5219941349	0.000833333333333331\\
19589.4428152493	0.000833333333333331\\
19636.3636363636	0.000833333333333331\\
19683.284457478	0.000833333333333331\\
19730.2052785924	0.000833333333333331\\
19777.1260997067	0.000833333333333331\\
19824.0469208211	0.000833333333333331\\
19870.9677419355	0.000833333333333331\\
19917.8885630499	0.000833333333333331\\
19964.8093841642	0.000833333333333331\\
20011.7302052786	0.000833333333333331\\
20058.651026393	0.000833333333333331\\
20105.5718475073	0.000833333333333331\\
20152.4926686217	0.000833333333333331\\
20199.4134897361	0.000833333333333331\\
20246.3343108504	0.000833333333333331\\
20293.2551319648	0.000833333333333331\\
20340.1759530792	0.000833333333333331\\
20387.0967741935	0.000833333333333331\\
20434.0175953079	0.000833333333333331\\
20480.9384164223	0.000833333333333331\\
20527.8592375367	0.000833333333333331\\
20574.780058651	0.000833333333333331\\
20621.7008797654	0.000833333333333331\\
20668.6217008798	0.000833333333333331\\
20715.5425219941	0.000833333333333331\\
20762.4633431085	0.000833333333333331\\
20809.3841642229	0.000833333333333331\\
20856.3049853372	0.000833333333333331\\
20903.2258064516	0.000833333333333331\\
20950.146627566	0.000833333333333331\\
20997.0674486804	0.000833333333333331\\
21043.9882697947	0.000833333333333331\\
21090.9090909091	0.000833333333333331\\
21137.8299120235	0.000833333333333331\\
21184.7507331378	0.000833333333333331\\
21231.6715542522	0.000833333333333331\\
21278.5923753666	0.000833333333333331\\
21325.5131964809	0.000833333333333331\\
21372.4340175953	0.000833333333333331\\
21419.3548387097	0.000833333333333331\\
21466.275659824	0.000833333333333331\\
21513.1964809384	0.000833333333333331\\
21560.1173020528	0.000833333333333331\\
21607.0381231672	0.000833333333333331\\
21653.9589442815	0.000833333333333331\\
21700.8797653959	0.000833333333333331\\
21747.8005865103	0.000833333333333331\\
21794.7214076246	0.000833333333333331\\
21841.642228739	0.000833333333333331\\
21888.5630498534	0.000833333333333331\\
21935.4838709677	0.000833333333333331\\
21982.4046920821	0.00083333333333333\\
22029.3255131965	0.000833333333333331\\
22076.2463343109	0.000833333333333331\\
22123.1671554252	0.000833333333333331\\
22170.0879765396	0.000833333333333331\\
22217.008797654	0.000833333333333331\\
22263.9296187683	0.000833333333333331\\
22310.8504398827	0.000833333333333331\\
22357.7712609971	0.000833333333333331\\
22404.6920821114	0.000833333333333331\\
22451.6129032258	0.000833333333333331\\
22498.5337243402	0.000833333333333331\\
22545.4545454545	0.00083333333333333\\
22592.3753665689	0.00083333333333333\\
22639.2961876833	0.00083333333333333\\
22686.2170087977	0.00083333333333333\\
22733.137829912	0.00083333333333333\\
22780.0586510264	0.00083333333333333\\
22826.9794721408	0.00083333333333333\\
22873.9002932551	0.00083333333333333\\
22920.8211143695	0.00083333333333333\\
22967.7419354839	0.00083333333333333\\
23014.6627565982	0.00083333333333333\\
23061.5835777126	0.00083333333333333\\
23108.504398827	0.00083333333333333\\
23155.4252199413	0.00083333333333333\\
23202.3460410557	0.00083333333333333\\
23249.2668621701	0.00083333333333333\\
23296.1876832845	0.00083333333333333\\
23343.1085043988	0.00083333333333333\\
23390.0293255132	0.00083333333333333\\
23436.9501466276	0.00083333333333333\\
23483.8709677419	0.00083333333333333\\
23530.7917888563	0.00083333333333333\\
23577.7126099707	0.00083333333333333\\
23624.633431085	0.00083333333333333\\
23671.5542521994	0.00083333333333333\\
23718.4750733138	0.00083333333333333\\
23765.3958944282	0.00083333333333333\\
23812.3167155425	0.00083333333333333\\
23859.2375366569	0.00083333333333333\\
23906.1583577713	0.00083333333333333\\
23953.0791788856	0.00083333333333333\\
24000	0.00083333333333333\\
};
\addlegendentry{Expérimentale}

\end{axis}
\end{tikzpicture}%


\subsection{Génération d'un signal modulé en fréquence}

Pour construire notre signal modulé en fréquence nous allons nous baser sur le signal NRZ et sur une simple sinusoïdale. Lorsque le signal NRZ vaut 0, la fréquence de la sinusoïdale sera $F_0=\SI{1180}{Hz}$ et lorsqu'il vaut 1 la sinusoïdale sera de fréquence $F_1=\SI{980}{Hz}$. Au final, notre signal modulé suit la formule suivante:
\[
\operatorname{modulé}(t)=\operatorname{NRZ}(t) \cos(2\pi F_1 t + \phi_1) + (1-\operatorname{NRZ}(t)) \cos(2 \pi F_0 t + \phi_1)
\]
$\phi_1$ et $\phi_2$ étant des déphasages tirés aléatoirement dans $[0, 2\pi]$ et $\text{NRZ}(t)$ le signal NRZ.


On obtient ainsi le signal suivant:

% This file was created by matlab2tikz.
%
%The latest updates can be retrieved from
%  http://www.mathworks.com/matlabcentral/fileexchange/22022-matlab2tikz-matlab2tikz
%where you can also make suggestions and rate matlab2tikz.
%
\definecolor{mycolor1}{rgb}{0.00000,0.44700,0.74100}%
%
\begin{tikzpicture}

\begin{axis}[%
width=4.521in,
height=3.559in,
at={(0.758in,0.488in)},
scale only axis,
xmin=0,
xmax=4000,
xlabel style={font=\color{white!15!black}},
xlabel={temps [s]},
ymin=-1.1,
ymax=1.1,
ylabel style={font=\color{white!15!black}},
ylabel={amplitude},
axis background/.style={fill=white},
title style={font=\bfseries},
title={Modulation du signal NRZ aléatoire (0 à 6000Hz, 1 à 2000Hz)}
]
\addplot [color=mycolor1, forget plot]
  table[row sep=crcr]{%
1	-0.378737871562718\\
2	-0.922238260469037\\
3	-0.925503984131966\\
4	-0.386622025920723\\
5	0.378737871562718\\
6	0.922238260469037\\
7	0.925503984131966\\
8	0.386622025920724\\
9	-0.378737871562718\\
10	-0.922238260469038\\
11	-0.925503984131966\\
12	-0.386622025920724\\
13	0.378737871562717\\
14	0.922238260469037\\
15	0.925503984131967\\
16	0.386622025920722\\
17	-0.378737871562717\\
18	-0.922238260469037\\
19	-0.925503984131965\\
20	-0.386622025920724\\
21	0.378737871562717\\
22	0.922238260469038\\
23	0.925503984131967\\
24	0.386622025920724\\
25	-0.378737871562717\\
26	-0.922238260469037\\
27	-0.925503984131967\\
28	-0.386622025920724\\
29	0.378737871562714\\
30	0.922238260469037\\
31	0.925503984131966\\
32	0.386622025920724\\
33	-0.378737871562717\\
34	-0.922238260469036\\
35	-0.925503984131967\\
36	-0.386622025920728\\
37	0.378737871562717\\
38	0.922238260469037\\
39	0.925503984131967\\
40	0.386622025920724\\
41	-0.378737871562713\\
42	-0.922238260469037\\
43	-0.925503984131964\\
44	-0.386622025920721\\
45	0.378737871562717\\
46	0.922238260469039\\
47	0.925503984131966\\
48	0.386622025920725\\
49	-0.37873787156272\\
50	-0.922238260469037\\
51	-0.925503984131965\\
52	-0.386622025920728\\
53	0.378737871562716\\
54	0.922238260469039\\
55	0.925503984131966\\
56	0.386622025920718\\
57	-0.378737871562713\\
58	-0.922238260469037\\
59	-0.925503984131965\\
60	-0.386622025920722\\
61	0.378737871562723\\
62	0.922238260469036\\
63	0.925503984131966\\
64	0.386622025920725\\
65	-0.378737871562719\\
66	-0.92223826046904\\
67	-0.925503984131967\\
68	-0.386622025920722\\
69	0.378737871562716\\
70	0.922238260469038\\
71	0.925503984131969\\
72	0.386622025920725\\
73	-0.378737871562719\\
74	-0.922238260469037\\
75	-0.925503984131965\\
76	-0.386622025920729\\
77	0.378737871562716\\
78	0.922238260469038\\
79	0.925503984131966\\
80	0.386622025920726\\
81	-0.378737871562712\\
82	-0.922238260469034\\
83	-0.92550398413197\\
84	-0.386622025920736\\
85	0.378737871562715\\
86	0.922238260469035\\
87	0.925503984131969\\
88	0.386622025920732\\
89	-0.378737871562705\\
90	-0.922238260469031\\
91	-0.925503984131968\\
92	-0.386622025920729\\
93	0.378737871562709\\
94	0.922238260469033\\
95	0.925503984131972\\
96	0.386622025920739\\
97	-0.378737871562712\\
98	-0.922238260469034\\
99	-0.92550398413197\\
100	-0.386622025920723\\
101	0.378737871562715\\
102	0.92223826046903\\
103	0.925503984131975\\
104	0.386622025920733\\
105	-0.378737871562705\\
106	-0.922238260469031\\
107	-0.925503984131968\\
108	-0.38662202592073\\
109	0.378737871562708\\
110	0.922238260469038\\
111	0.925503984131967\\
112	0.38662202592074\\
113	-0.378737871562698\\
114	-0.922238260469034\\
115	-0.925503984131971\\
116	-0.386622025920736\\
117	0.378737871562715\\
118	0.922238260469035\\
119	0.925503984131969\\
120	0.386622025920733\\
121	-0.378737871562718\\
122	-0.922238260469031\\
123	-0.925503984131973\\
124	-0.38662202592073\\
125	0.378737871562708\\
126	0.922238260469032\\
127	0.925503984131972\\
128	0.386622025920727\\
129	-0.378737871562711\\
130	-0.922238260469034\\
131	-0.925503984131965\\
132	-0.386622025920737\\
133	0.378737871562701\\
134	0.922238260469029\\
135	0.92550398413197\\
136	0.386622025920734\\
137	-0.378737871562704\\
138	-0.922238260469036\\
139	-0.925503984131968\\
140	-0.386622025920744\\
141	0.378737871562694\\
142	0.922238260469032\\
143	0.925503984131972\\
144	0.386622025920741\\
145	-0.37873787156271\\
146	-0.922238260469033\\
147	-0.925503984131971\\
148	-0.386622025920737\\
149	0.378737871562714\\
150	0.922238260469029\\
151	0.925503984131975\\
152	0.386622025920734\\
153	-0.378737871562704\\
154	-0.92223826046903\\
155	-0.925503984131968\\
156	-0.386622025920731\\
157	0.378737871562707\\
158	0.922238260469032\\
159	0.925503984131967\\
160	0.386622025920741\\
161	-0.378737871562697\\
162	-0.922238260469033\\
163	-0.925503984131971\\
164	-0.386622025920738\\
165	0.3787378715627\\
166	0.922238260469029\\
167	0.925503984131975\\
168	0.386622025920748\\
169	-0.378737871562716\\
170	-0.922238260469036\\
171	-0.925503984131969\\
172	-0.386622025920731\\
173	0.378737871562706\\
174	0.922238260469032\\
175	0.925503984131973\\
176	0.386622025920741\\
177	-0.378737871562696\\
178	-0.922238260469027\\
179	-0.925503984131977\\
180	-0.386622025920725\\
181	0.378737871562713\\
182	0.922238260469034\\
183	0.92550398413197\\
184	0.386622025920735\\
185	-0.378737871562703\\
186	-0.92223826046903\\
187	-0.925503984131974\\
188	-0.386622025920745\\
189	0.378737871562693\\
190	0.922238260469037\\
191	0.925503984131978\\
192	0.386622025920729\\
193	-0.378737871562709\\
194	-0.922238260469022\\
195	-0.925503984131972\\
196	-0.386622025920739\\
197	0.378737871562699\\
198	0.922238260469029\\
199	0.925503984131965\\
200	0.386622025920749\\
201	-0.378737871562715\\
202	-0.922238260469035\\
203	-0.92550398413198\\
204	-0.386622025920732\\
205	0.378737871562679\\
206	0.922238260469031\\
207	0.925503984131973\\
208	0.386622025920742\\
209	-0.378737871562696\\
210	-0.922238260469038\\
211	-0.925503984131977\\
212	-0.386622025920726\\
213	0.378737871562712\\
214	0.922238260469023\\
215	0.92550398413197\\
216	0.386622025920736\\
217	-0.378737871562702\\
218	-0.92223826046903\\
219	-0.925503984131964\\
220	-0.386622025920746\\
221	0.378737871562718\\
222	0.922238260469026\\
223	0.925503984131979\\
224	0.38662202592073\\
225	-0.378737871562682\\
226	-0.922238260469032\\
227	-0.925503984131972\\
228	-0.38662202592074\\
229	0.378737871562698\\
230	0.922238260469039\\
231	0.925503984131976\\
232	0.38662202592075\\
233	-0.378737871562715\\
234	-0.922238260469024\\
235	-0.925503984131969\\
236	-0.38662202592076\\
237	0.378737871562705\\
238	0.922238260469031\\
239	0.925503984131973\\
240	0.386622025920743\\
241	-0.378737871562721\\
242	-0.922238260469027\\
243	-0.925503984131978\\
244	-0.386622025920727\\
245	0.378737871562685\\
246	0.922238260469034\\
247	0.925503984131971\\
248	0.386622025920737\\
249	-0.378737871562701\\
250	-0.922238260469018\\
251	-0.925503984131975\\
252	-0.386622025920747\\
253	0.378737871562691\\
254	0.922238260469025\\
255	0.925503984131968\\
256	0.386622025920757\\
257	-0.378737871562707\\
258	-0.922238260469032\\
259	-0.925503984131972\\
260	-0.386622025920741\\
261	0.378737871562724\\
262	0.922238260469028\\
263	0.925503984131976\\
264	0.386622025920724\\
265	-0.378737871562687\\
266	-0.922238260469035\\
267	-0.92550398413198\\
268	-0.386622025920734\\
269	0.378737871562704\\
270	0.922238260469019\\
271	0.925503984131974\\
272	0.386622025920744\\
273	-0.378737871562694\\
274	-0.922238260469026\\
275	-0.925503984131967\\
276	-0.386622025920754\\
277	0.37873787156271\\
278	0.922238260469033\\
279	0.925503984131982\\
280	0.386622025920738\\
281	-0.378737871562674\\
282	-0.922238260469029\\
283	-0.925503984131975\\
284	-0.386622025920748\\
285	0.37873787156269\\
286	0.922238260469036\\
287	0.925503984131979\\
288	0.386622025920731\\
289	-0.378737871562706\\
290	-0.922238260469021\\
291	-0.925503984131973\\
292	-0.386622025920741\\
293	0.378737871562696\\
294	0.922238260469027\\
295	0.925503984131977\\
296	0.386622025920751\\
297	-0.378737871562713\\
298	-0.922238260469023\\
299	-0.925503984131981\\
300	-0.386622025920735\\
301	0.378737871562676\\
302	0.92223826046903\\
303	0.925503984131974\\
304	0.386622025920745\\
305	-0.378737871562693\\
306	-0.922238260469037\\
307	-0.925503984131978\\
308	-0.386622025920729\\
309	0.378737871562709\\
310	0.922238260469022\\
311	0.925503984131972\\
312	0.386622025920765\\
313	-0.378737871562699\\
314	-0.922238260469029\\
315	-0.925503984131976\\
316	-0.386622025920749\\
317	0.378737871562715\\
318	0.922238260469024\\
319	0.92550398413198\\
320	0.386622025920732\\
321	0.134652380105467\\
322	-0.126397742304071\\
323	-0.378834067457718\\
324	-0.60545347696702\\
325	-0.790812232580209\\
326	-0.922278441422054\\
327	-0.990892898618176\\
328	-0.991979642301411\\
329	-0.925464612685674\\
330	-0.795880699117988\\
331	-0.612058831160461\\
332	-0.386526165334394\\
333	-0.134652380105464\\
334	0.126397742304074\\
335	0.378834067457708\\
336	0.605453476967023\\
337	0.790812232580211\\
338	0.922278441422055\\
339	0.990892898618177\\
340	0.991979642301412\\
341	0.925464612685673\\
342	0.795880699117986\\
343	0.61205883116047\\
344	0.38652616533439\\
345	0.134652380105461\\
346	-0.126397742304063\\
347	-0.378834067457711\\
348	-0.605453476967026\\
349	-0.790812232580205\\
350	-0.922278441422057\\
351	-0.990892898618177\\
352	-0.991979642301412\\
353	-0.925464612685672\\
354	-0.795880699117992\\
355	-0.612058831160467\\
356	-0.386526165334387\\
357	-0.134652380105471\\
358	0.126397742304067\\
359	0.378834067457714\\
360	0.605453476967017\\
361	0.790812232580207\\
362	0.922278441422058\\
363	0.990892898618176\\
364	0.991979642301411\\
365	0.92546461268567\\
366	0.79588069911799\\
367	0.612058831160464\\
368	0.386526165334384\\
369	0.134652380105468\\
370	-0.12639774230407\\
371	-0.378834067457704\\
372	-0.60545347696702\\
373	-0.790812232580209\\
374	-0.922278441422054\\
375	-0.990892898618176\\
376	-0.991979642301411\\
377	-0.925464612685674\\
378	-0.795880699117997\\
379	-0.612058831160462\\
380	-0.386526165334394\\
381	-0.134652380105479\\
382	0.126397742304059\\
383	0.378834067457708\\
384	0.605453476967011\\
385	0.790812232580211\\
386	0.922278441422055\\
387	0.990892898618175\\
388	0.991979642301411\\
389	0.925464612685673\\
390	0.795880699117995\\
391	0.612058831160459\\
392	0.386526165334391\\
393	0.134652380105475\\
394	-0.126397742304063\\
395	-0.378834067457711\\
396	-0.605453476967014\\
397	-0.790812232580213\\
398	-0.922278441422056\\
399	-0.990892898618175\\
400	-0.99197964230141\\
401	-0.925464612685672\\
402	-0.795880699117993\\
403	-0.612058831160456\\
404	-0.386526165334388\\
405	-0.134652380105472\\
406	0.126397742304066\\
407	0.378834067457714\\
408	0.605453476967017\\
409	0.790812232580198\\
410	0.922278441422052\\
411	0.990892898618176\\
412	0.991979642301413\\
413	0.92546461268567\\
414	0.795880699117991\\
415	0.612058831160476\\
416	0.386526165334385\\
417	0.134652380105468\\
418	-0.126397742304056\\
419	-0.378834067457717\\
420	-0.605453476967019\\
421	-0.7908122325802\\
422	-0.922278441422054\\
423	-0.990892898618176\\
424	-0.991979642301413\\
425	-0.925464612685669\\
426	-0.795880699117989\\
427	-0.612058831160473\\
428	-0.386526165334381\\
429	-0.134652380105465\\
430	0.126397742304059\\
431	0.37883406745772\\
432	0.605453476967022\\
433	0.790812232580202\\
434	0.922278441422055\\
435	0.990892898618177\\
436	0.991979642301412\\
437	0.925464612685668\\
438	0.795880699117995\\
439	0.612058831160471\\
440	0.386526165334404\\
441	0.134652380105462\\
442	-0.126397742304062\\
443	-0.378834067457697\\
444	-0.605453476967025\\
445	-0.790812232580204\\
446	-0.922278441422051\\
447	-0.990892898618177\\
448	-0.991979642301412\\
449	-0.925464612685677\\
450	-0.795880699117993\\
451	-0.612058831160468\\
452	-0.386526165334401\\
453	-0.134652380105458\\
454	0.126397742304066\\
455	0.3788340674577\\
456	0.605453476967028\\
457	0.790812232580206\\
458	0.922278441422052\\
459	0.990892898618178\\
460	0.991979642301412\\
461	0.925464612685676\\
462	0.795880699117991\\
463	0.612058831160465\\
464	0.386526165334398\\
465	0.134652380105455\\
466	-0.126397742304055\\
467	-0.378834067457703\\
468	-0.605453476967008\\
469	-0.790812232580217\\
470	-0.922278441422053\\
471	-0.990892898618174\\
472	-0.991979642301413\\
473	-0.925464612685675\\
474	-0.795880699117989\\
475	-0.612058831160451\\
476	-0.386526165334395\\
477	-0.134652380105494\\
478	0.126397742304044\\
479	0.378834067457707\\
480	0.605453476967022\\
481	0.790812232580211\\
482	0.92227844142206\\
483	0.990892898618173\\
484	0.991979642301411\\
485	0.925464612685679\\
486	0.795880699117995\\
487	0.61205883116046\\
488	0.386526165334379\\
489	0.134652380105476\\
490	-0.126397742304062\\
491	-0.378834067457723\\
492	-0.605453476967013\\
493	-0.790812232580204\\
494	-0.922278441422056\\
495	-0.990892898618175\\
496	-0.991979642301412\\
497	-0.925464612685672\\
498	-0.795880699118002\\
499	-0.612058831160468\\
500	-0.386526165334389\\
501	-0.134652380105487\\
502	0.126397742304051\\
503	0.378834067457713\\
504	0.605453476967027\\
505	0.790812232580197\\
506	0.922278441422052\\
507	0.990892898618177\\
508	0.991979642301413\\
509	0.925464612685676\\
510	0.795880699117991\\
511	0.612058831160477\\
512	0.386526165334399\\
513	0.134652380105469\\
514	-0.12639774230404\\
515	-0.378834067457729\\
516	-0.605453476967019\\
517	-0.790812232580191\\
518	-0.922278441422059\\
519	-0.990892898618176\\
520	-0.991979642301411\\
521	-0.92546461268567\\
522	-0.795880699117998\\
523	-0.612058831160463\\
524	-0.386526165334382\\
525	-0.13465238010548\\
526	0.126397742304058\\
527	0.378834067457719\\
528	0.60545347696701\\
529	0.790812232580202\\
530	0.922278441422043\\
531	0.990892898618178\\
532	0.991979642301413\\
533	0.925464612685684\\
534	0.795880699117987\\
535	0.612058831160471\\
536	0.386526165334392\\
537	0.134652380105463\\
538	-0.126397742304047\\
539	-0.378834067457709\\
540	-0.605453476967024\\
541	-0.790812232580195\\
542	-0.92227844142205\\
543	-0.990892898618177\\
544	-0.991979642301414\\
545	-0.925464612685678\\
546	-0.795880699117994\\
547	-0.612058831160457\\
548	-0.386526165334402\\
549	-0.134652380105473\\
550	0.126397742304065\\
551	0.378834067457699\\
552	0.605453476967016\\
553	0.790812232580206\\
554	0.922278441422046\\
555	0.990892898618179\\
556	0.991979642301412\\
557	0.925464612685682\\
558	0.795880699117983\\
559	0.612058831160466\\
560	0.386526165334386\\
561	0.134652380105484\\
562	-0.126397742304054\\
563	-0.378834067457716\\
564	-0.605453476967007\\
565	-0.790812232580199\\
566	-0.922278441422053\\
567	-0.990892898618174\\
568	-0.991979642301413\\
569	-0.925464612685675\\
570	-0.795880699118007\\
571	-0.612058831160452\\
572	-0.386526165334396\\
573	-0.134652380105495\\
574	0.126397742304072\\
575	0.378834067457706\\
576	0.605453476967021\\
577	0.79081223258021\\
578	0.922278441422049\\
579	0.990892898618176\\
580	0.991979642301411\\
581	0.925464612685679\\
582	0.795880699117996\\
583	0.612058831160461\\
584	0.386526165334406\\
585	0.134652380105477\\
586	-0.126397742304061\\
587	-0.378834067457722\\
588	-0.605453476967012\\
589	-0.790812232580186\\
590	-0.922278441422056\\
591	-0.990892898618175\\
592	-0.991979642301412\\
593	-0.925464612685672\\
594	-0.795880699118003\\
595	-0.612058831160469\\
596	-0.38652616533439\\
597	-0.134652380105488\\
598	0.12639774230405\\
599	0.378834067457712\\
600	0.605453476967004\\
601	0.790812232580197\\
602	0.922278441422051\\
603	0.990892898618177\\
604	0.991979642301414\\
605	0.925464612685677\\
606	0.795880699117992\\
607	0.612058831160478\\
608	0.386526165334399\\
609	0.13465238010547\\
610	-0.126397742304039\\
611	-0.378834067457728\\
612	-0.605453476967018\\
613	-0.79081223258019\\
614	-0.922278441422058\\
615	-0.990892898618176\\
616	-0.991979642301411\\
617	-0.92546461268567\\
618	-0.795880699117998\\
619	-0.612058831160464\\
620	-0.386526165334409\\
621	-0.134652380105481\\
622	0.126397742304057\\
623	0.378834067457692\\
624	0.605453476967009\\
625	0.790812232580201\\
626	0.922278441422043\\
627	0.990892898618178\\
628	0.991979642301413\\
629	0.925464612685685\\
630	0.795880699117988\\
631	0.612058831160472\\
632	0.386526165334393\\
633	0.134652380105464\\
634	-0.126397742304046\\
635	-0.378834067457708\\
636	-0.605453476967023\\
637	-0.790812232580194\\
638	-0.92227844142205\\
639	-0.990892898618177\\
640	-0.991979642301414\\
641	-0.925464612685678\\
642	-0.795880699117994\\
643	-0.612058831160458\\
644	-0.386526165334403\\
645	-0.134652380105474\\
646	0.126397742304064\\
647	0.378834067457698\\
648	0.605453476967015\\
649	0.790812232580205\\
650	0.922278441422046\\
651	0.990892898618175\\
652	0.991979642301412\\
653	0.925464612685682\\
654	0.795880699118001\\
655	0.612058831160467\\
656	0.386526165334413\\
657	0.134652380105485\\
658	-0.126397742304053\\
659	-0.378834067457715\\
660	-0.605453476967006\\
661	-0.790812232580199\\
662	-0.922278441422053\\
663	-0.990892898618174\\
664	-0.991979642301413\\
665	-0.925464612685675\\
666	-0.795880699118007\\
667	-0.612058831160453\\
668	-0.386526165334397\\
669	-0.134652380105496\\
670	0.126397742304071\\
671	0.378834067457705\\
672	0.60545347696702\\
673	0.790812232580209\\
674	0.922278441422048\\
675	0.990892898618176\\
676	0.991979642301411\\
677	0.92546461268568\\
678	0.795880699117997\\
679	0.612058831160484\\
680	0.386526165334407\\
681	0.134652380105478\\
682	-0.126397742304032\\
683	-0.378834067457721\\
684	-0.605453476967012\\
685	-0.790812232580185\\
686	-0.922278441422055\\
687	-0.990892898618175\\
688	-0.991979642301412\\
689	-0.925464612685673\\
690	-0.795880699118003\\
691	-0.61205883116047\\
692	-0.38652616533439\\
693	-0.134652380105489\\
694	0.126397742304049\\
695	0.378834067457711\\
696	0.605453476967003\\
697	0.790812232580196\\
698	0.922278441422051\\
699	0.990892898618177\\
700	0.991979642301414\\
701	0.925464612685677\\
702	0.795880699117992\\
703	0.612058831160478\\
704	0.3865261653344\\
705	0.134652380105471\\
706	-0.126397742304038\\
707	-0.378834067457701\\
708	-0.605453476967017\\
709	-0.79081223258019\\
710	-0.922278441422047\\
711	-0.990892898618176\\
712	-0.991979642301415\\
713	-0.925464612685681\\
714	-0.795880699117999\\
715	-0.612058831160464\\
716	-0.38652616533441\\
717	-0.134652380105482\\
718	0.126397742304056\\
719	0.378834067457691\\
720	0.605453476967009\\
721	0.7908122325802\\
722	0.922278441422043\\
723	0.990892898618178\\
724	0.991979642301413\\
725	0.925464612685685\\
726	0.795880699117988\\
727	0.612058831160473\\
728	0.386526165334394\\
729	0.134652380105465\\
730	-0.126397742304045\\
731	-0.378834067457708\\
732	-0.605453476967023\\
733	-0.790812232580194\\
734	-0.92227844142205\\
735	-0.990892898618177\\
736	-0.991979642301414\\
737	-0.925464612685678\\
738	-0.795880699117995\\
739	-0.612058831160459\\
740	-0.386526165334404\\
741	-0.134652380105503\\
742	0.126397742304063\\
743	0.378834067457698\\
744	0.605453476967014\\
745	0.790812232580205\\
746	0.922278441422045\\
747	0.990892898618175\\
748	0.991979642301412\\
749	0.925464612685683\\
750	0.795880699118001\\
751	0.612058831160467\\
752	0.386526165334388\\
753	0.134652380105486\\
754	-0.126397742304052\\
755	-0.378834067457688\\
756	-0.605453476967005\\
757	-0.790812232580198\\
758	-0.922278441422041\\
759	-0.990892898618174\\
760	-0.991979642301413\\
761	-0.925464612685687\\
762	-0.795880699118008\\
763	-0.612058831160476\\
764	-0.386526165334424\\
765	-0.134652380105497\\
766	0.12639774230407\\
767	0.378834067457678\\
768	0.605453476966997\\
769	0.790812232580209\\
770	0.922278441422037\\
771	0.990892898618176\\
772	0.991979642301411\\
773	0.925464612685691\\
774	0.795880699117997\\
775	0.612058831160462\\
776	0.386526165334408\\
777	0.134652380105479\\
778	-0.126397742304059\\
779	-0.378834067457694\\
780	-0.605453476967011\\
781	-0.790812232580202\\
782	-0.922278441422044\\
783	-0.990892898618175\\
784	-0.991979642301412\\
785	-0.925464612685684\\
786	-0.795880699118004\\
787	-0.612058831160471\\
788	-0.386526165334418\\
789	-0.13465238010549\\
790	0.126397742304076\\
791	0.378834067457684\\
792	0.605453476967002\\
793	0.790812232580213\\
794	0.92227844142204\\
795	0.990892898618177\\
796	0.99197964230141\\
797	0.925464612685688\\
798	0.795880699117993\\
799	0.612058831160457\\
800	0.386526165334401\\
801	0.134652380105472\\
802	-0.126397742304066\\
803	-0.3788340674577\\
804	-0.605453476967016\\
805	-0.790812232580206\\
806	-0.922278441422047\\
807	-0.990892898618176\\
808	-0.991979642301412\\
809	-0.925464612685681\\
810	-0.795880699118\\
811	-0.612058831160465\\
812	-0.386526165334411\\
813	-0.134652380105483\\
814	0.126397742304055\\
815	0.37883406745769\\
816	0.605453476967008\\
817	0.790812232580182\\
818	0.922278441422042\\
819	0.990892898618174\\
820	0.991979642301416\\
821	0.925464612685685\\
822	0.795880699117989\\
823	0.612058831160496\\
824	0.386526165334421\\
825	0.134652380105466\\
826	-0.126397742304016\\
827	-0.378834067457707\\
828	-0.605453476967022\\
829	-0.790812232580176\\
830	-0.922278441422049\\
831	-0.990892898618177\\
832	-0.991979642301414\\
833	-0.925464612685679\\
834	-0.795880699117995\\
835	-0.612058831160482\\
836	-0.386526165334405\\
837	-0.134652380105476\\
838	0.126397742304034\\
839	0.378834067457697\\
840	0.605453476967013\\
841	0.790812232580187\\
842	0.922278441422045\\
843	0.990892898618175\\
844	0.991979642301416\\
845	0.925464612685683\\
846	0.795880699117985\\
847	0.612058831160491\\
848	0.386526165334415\\
849	0.134652380105459\\
850	-0.126397742304023\\
851	-0.378834067457713\\
852	-0.605453476967027\\
853	-0.79081223258018\\
854	-0.922278441422052\\
855	-0.990892898618177\\
856	-0.991979642301413\\
857	-0.925464612685676\\
858	-0.795880699117991\\
859	-0.612058831160477\\
860	-0.386526165334399\\
861	-0.134652380105469\\
862	0.12639774230404\\
863	0.378834067457703\\
864	0.605453476967019\\
865	0.790812232580191\\
866	0.922278441422048\\
867	0.990892898618176\\
868	0.991979642301415\\
869	0.92546461268568\\
870	0.795880699117998\\
871	0.612058831160485\\
872	0.386526165334409\\
873	0.134652380105452\\
874	-0.12639774230403\\
875	-0.378834067457693\\
876	-0.605453476966988\\
877	-0.790812232580184\\
878	-0.922278441422054\\
879	-0.990892898618171\\
880	-0.991979642301416\\
881	-0.925464612685674\\
882	-0.795880699118022\\
883	-0.612058831160471\\
884	-0.386526165334392\\
885	-0.134652380105519\\
886	0.126397742304047\\
887	0.378834067457709\\
888	0.605453476967002\\
889	0.790812232580195\\
890	0.92227844142205\\
891	0.990892898618173\\
892	0.991979642301414\\
893	0.925464612685678\\
894	0.795880699118011\\
895	0.61205883116048\\
896	0.386526165334402\\
897	0.134652380105501\\
898	-0.126397742304037\\
899	-0.378834067457699\\
900	-0.605453476966993\\
901	-0.790812232580188\\
902	-0.922278441422057\\
903	-0.990892898618172\\
904	-0.991979642301415\\
905	-0.925464612685671\\
906	-0.795880699118017\\
907	-0.612058831160466\\
908	-0.386526165334386\\
909	-0.134652380105512\\
910	0.126397742304054\\
911	0.378834067457716\\
912	0.605453476967007\\
913	0.790812232580199\\
914	0.922278441422053\\
915	0.990892898618174\\
916	0.991979642301413\\
917	0.925464612685675\\
918	0.795880699118007\\
919	0.612058831160474\\
920	0.386526165334396\\
921	0.134652380105495\\
922	-0.126397742304043\\
923	-0.378834067457706\\
924	-0.605453476966998\\
925	-0.790812232580193\\
926	-0.922278441422049\\
927	-0.990892898618173\\
928	-0.991979642301414\\
929	-0.925464612685668\\
930	-0.795880699118013\\
931	-0.612058831160483\\
932	-0.38652616533438\\
933	-0.134652380105505\\
934	0.126397742304061\\
935	0.378834067457669\\
936	0.60545347696699\\
937	0.790812232580203\\
938	0.922278441422034\\
939	0.990892898618175\\
940	0.991979642301412\\
941	0.925464612685694\\
942	0.795880699118003\\
943	0.612058831160469\\
944	0.386526165334416\\
945	0.134652380105488\\
946	-0.12639774230405\\
947	-0.378834067457686\\
948	-0.605453476967004\\
949	-0.790812232580197\\
950	-0.92227844142204\\
951	-0.990892898618173\\
952	-0.991979642301414\\
953	-0.925464612685687\\
954	-0.795880699118009\\
955	-0.612058831160478\\
956	-0.386526165334426\\
957	-0.134652380105499\\
958	0.126397742304068\\
959	0.378834067457676\\
960	0.605453476966995\\
961	0.790812232580225\\
962	0.922278441422047\\
963	0.990892898618176\\
964	0.991979642301411\\
965	0.925464612685691\\
966	0.795880699117981\\
967	0.612058831160486\\
968	0.386526165334409\\
969	0.134652380105481\\
970	-0.126397742304057\\
971	-0.378834067457718\\
972	-0.605453476966987\\
973	-0.790812232580218\\
974	-0.922278441422043\\
975	-0.990892898618174\\
976	-0.991979642301413\\
977	-0.925464612685696\\
978	-0.795880699117988\\
979	-0.612058831160495\\
980	-0.386526165334367\\
981	-0.134652380105492\\
982	0.126397742304103\\
983	0.378834067457708\\
984	0.605453476966978\\
985	0.790812232580212\\
986	0.922278441422039\\
987	0.990892898618181\\
988	0.991979642301414\\
989	0.9254646126857\\
990	0.795880699117994\\
991	0.612058831160503\\
992	0.386526165334377\\
993	0.134652380105502\\
994	-0.126397742304092\\
995	-0.378834067457698\\
996	-0.60545347696697\\
997	-0.790812232580205\\
998	-0.922278441422035\\
999	-0.990892898618179\\
1000	-0.991979642301415\\
1001	-0.925464612685704\\
1002	-0.795880699118001\\
1003	-0.612058831160512\\
1004	-0.386526165334387\\
1005	-0.134652380105513\\
1006	0.126397742303968\\
1007	0.378834067457688\\
1008	0.605453476967052\\
1009	0.790812232580199\\
1010	0.922278441422031\\
1011	0.990892898618178\\
1012	0.991979642301417\\
1013	0.925464612685665\\
1014	0.795880699118007\\
1015	0.61205883116052\\
1016	0.386526165334397\\
1017	0.134652380105524\\
1018	-0.126397742304071\\
1019	-0.378834067457679\\
1020	-0.605453476967043\\
1021	-0.790812232580192\\
1022	-0.922278441422026\\
1023	-0.990892898618176\\
1024	-0.991979642301418\\
1025	-0.925464612685669\\
1026	-0.795880699118014\\
1027	-0.612058831160529\\
1028	-0.386526165334407\\
1029	-0.134652380105422\\
1030	0.12639774230406\\
1031	0.378834067457669\\
1032	0.605453476967034\\
1033	0.790812232580185\\
1034	0.922278441422066\\
1035	0.990892898618175\\
1036	0.991979642301419\\
1037	0.925464612685673\\
1038	0.79588069911802\\
1039	0.612058831160447\\
1040	0.386526165334417\\
1041	0.134652380105433\\
1042	-0.126397742304049\\
1043	-0.378834067457659\\
1044	-0.605453476967026\\
1045	-0.790812232580179\\
1046	-0.922278441422062\\
1047	-0.990892898618173\\
1048	-0.991979642301421\\
1049	-0.925464612685677\\
1050	-0.795880699118027\\
1051	-0.612058831160456\\
1052	-0.386526165334427\\
1053	-0.134652380105443\\
1054	0.126397742304038\\
1055	0.378834067457649\\
1056	0.605453476967017\\
1057	0.790812232580172\\
1058	0.922278441422058\\
1059	0.990892898618172\\
1060	0.991979642301422\\
1061	0.925464612685681\\
1062	0.795880699118033\\
1063	0.612058831160464\\
1064	0.386526165334437\\
1065	0.134652380105567\\
1066	-0.126397742304028\\
1067	-0.378834067457744\\
1068	-0.605453476967009\\
1069	-0.790812232580166\\
1070	-0.922278441422054\\
1071	-0.99089289861817\\
1072	-0.991979642301409\\
1073	-0.925464612685685\\
1074	-0.79588069911804\\
1075	-0.612058831160473\\
1076	-0.386526165334447\\
1077	-0.134652380105465\\
1078	0.126397742304017\\
1079	0.378834067457734\\
1080	0.605453476967\\
1081	0.790812232580159\\
1082	0.92227844142205\\
1083	0.990892898618169\\
1084	0.991979642301411\\
1085	0.925464612685689\\
1086	0.795880699118046\\
1087	0.612058831160481\\
1088	0.386526165334352\\
1089	0.134652380105475\\
1090	-0.126397742304006\\
1091	-0.378834067457724\\
1092	-0.605453476966991\\
1093	-0.790812232580222\\
1094	-0.922278441422045\\
1095	-0.990892898618168\\
1096	-0.991979642301412\\
1097	-0.925464612685693\\
1098	-0.795880699117984\\
1099	-0.61205883116049\\
1100	-0.386526165334362\\
1101	-0.134652380105486\\
1102	0.126397742303996\\
1103	0.378834067457714\\
1104	0.605453476966983\\
1105	0.790812232580215\\
1106	0.922278441422041\\
1107	0.990892898618166\\
1108	0.991979642301413\\
1109	0.925464612685654\\
1110	0.795880699117991\\
1111	0.612058831160499\\
1112	0.386526165334371\\
1113	0.134652380105497\\
1114	-0.126397742304098\\
1115	-0.378834067457704\\
1116	-0.605453476966974\\
1117	-0.790812232580209\\
1118	-0.922278441422037\\
1119	-0.99089289861818\\
1120	-0.991979642301415\\
1121	-0.378737871562551\\
1122	-0.922238260469\\
1123	-0.925503984131972\\
1124	-0.38662202592087\\
1125	0.378737871562646\\
1126	0.922238260468996\\
1127	0.925503984132019\\
1128	0.386622025920775\\
1129	-0.378737871562637\\
1130	-0.922238260468991\\
1131	-0.92550398413198\\
1132	-0.386622025920785\\
1133	0.378737871562627\\
1134	0.922238260468987\\
1135	0.925503984131984\\
1136	0.386622025920795\\
1137	-0.378737871562617\\
1138	-0.922238260469027\\
1139	-0.925503984132031\\
1140	-0.386622025920805\\
1141	0.378737871562712\\
1142	0.922238260468979\\
1143	0.925503984131992\\
1144	0.38662202592071\\
1145	-0.378737871562597\\
1146	-0.922238260469019\\
1147	-0.925503984131996\\
1148	-0.386622025920825\\
1149	0.378737871562692\\
1150	0.922238260469015\\
1151	0.925503984132\\
1152	0.386622025920835\\
1153	-0.378737871562682\\
1154	-0.92223826046901\\
1155	-0.925503984132004\\
1156	-0.38662202592074\\
1157	0.378737871562567\\
1158	0.922238260469006\\
1159	0.925503984131965\\
1160	0.386622025920854\\
1161	-0.378737871562662\\
1162	-0.922238260469046\\
1163	-0.925503984132012\\
1164	-0.386622025920759\\
1165	0.378737871562652\\
1166	0.922238260468998\\
1167	0.925503984131973\\
1168	0.386622025920769\\
1169	-0.378737871562642\\
1170	-0.922238260468994\\
1171	-0.925503984131978\\
1172	-0.386622025920779\\
1173	0.378737871562632\\
1174	0.922238260469034\\
1175	0.925503984132025\\
1176	0.386622025920789\\
1177	-0.378737871562622\\
1178	-0.922238260468985\\
1179	-0.925503984131986\\
1180	-0.386622025920904\\
1181	0.378737871562612\\
1182	0.922238260469025\\
1183	0.925503984132033\\
1184	0.386622025920809\\
1185	-0.378737871562707\\
1186	-0.922238260468977\\
1187	-0.925503984131994\\
1188	-0.386622025920819\\
1189	0.378737871562592\\
1190	0.922238260469017\\
1191	0.925503984131998\\
1192	0.386622025920829\\
1193	-0.378737871562582\\
1194	-0.922238260469013\\
1195	-0.925503984132002\\
1196	-0.386622025920839\\
1197	0.378737871562677\\
1198	0.922238260468965\\
1199	0.925503984132006\\
1200	0.386622025920744\\
1201	-0.378737871562562\\
1202	-0.922238260469004\\
1203	-0.925503984131967\\
1204	-0.386622025920859\\
1205	0.378737871562657\\
1206	0.922238260469\\
1207	0.925503984132014\\
1208	0.386622025920764\\
1209	-0.378737871562647\\
1210	-0.922238260468996\\
1211	-0.925503984132018\\
1212	-0.386622025920774\\
1213	0.378737871562637\\
1214	0.922238260468992\\
1215	0.925503984131979\\
1216	0.386622025920784\\
1217	-0.378737871562627\\
1218	-0.922238260469032\\
1219	-0.925503984132027\\
1220	-0.386622025920794\\
1221	0.378737871562723\\
1222	0.922238260468983\\
1223	0.925503984131988\\
1224	0.386622025920699\\
1225	-0.378737871562608\\
1226	-0.922238260469023\\
1227	-0.925503984131992\\
1228	-0.386622025920814\\
1229	0.378737871562703\\
1230	0.922238260469019\\
1231	0.925503984131996\\
1232	0.386622025920824\\
1233	-0.378737871562693\\
1234	-0.922238260469015\\
1235	-0.925503984132\\
1236	-0.386622025920834\\
1237	0.378737871562578\\
1238	0.922238260469011\\
1239	0.925503984132004\\
1240	0.386622025920844\\
1241	-0.378737871562673\\
1242	-0.922238260468963\\
1243	-0.925503984132008\\
1244	-0.386622025920749\\
1245	0.378737871562558\\
1246	0.922238260469002\\
1247	0.925503984131969\\
1248	0.386622025920863\\
1249	-0.378737871562653\\
1250	-0.922238260468998\\
1251	-0.925503984132016\\
1252	-0.386622025920768\\
1253	0.378737871562643\\
1254	0.922238260468994\\
1255	0.92550398413202\\
1256	0.386622025920778\\
1257	-0.378737871562633\\
1258	-0.92223826046899\\
1259	-0.925503984131981\\
1260	-0.386622025920893\\
1261	0.378737871562623\\
1262	0.92223826046903\\
1263	0.925503984132028\\
1264	0.386622025920798\\
1265	-0.378737871562718\\
1266	-0.922238260468982\\
1267	-0.925503984131989\\
1268	-0.386622025920808\\
1269	0.378737871562603\\
1270	0.922238260469021\\
1271	0.925503984131993\\
1272	0.386622025920818\\
1273	-0.378737871562593\\
1274	-0.922238260469017\\
1275	-0.925503984131998\\
1276	-0.386622025920828\\
1277	0.378737871562688\\
1278	0.922238260468969\\
1279	0.925503984132002\\
1280	0.386622025920733\\
1281	-0.378737871562573\\
1282	-0.922238260469009\\
1283	-0.925503984131963\\
1284	-0.386622025920848\\
1285	0.378737871562668\\
1286	0.922238260469005\\
1287	0.92550398413201\\
1288	0.386622025920753\\
1289	-0.378737871562658\\
1290	-0.922238260469001\\
1291	-0.925503984132014\\
1292	-0.386622025920763\\
1293	0.378737871562648\\
1294	0.922238260468996\\
1295	0.925503984131975\\
1296	0.386622025920878\\
1297	-0.378737871562638\\
1298	-0.922238260468992\\
1299	-0.925503984132022\\
1300	-0.386622025920783\\
1301	0.378737871562523\\
1302	0.922238260468988\\
1303	0.92550398413194\\
1304	0.386622025920793\\
1305	-0.378737871562724\\
1306	-0.922238260469072\\
1307	-0.925503984131987\\
1308	-0.386622025920698\\
1309	0.378737871562819\\
1310	0.922238260469024\\
1311	0.925503984131948\\
1312	0.386622025920603\\
1313	-0.378737871562704\\
1314	-0.922238260469063\\
1315	-0.925503984131909\\
1316	-0.386622025920718\\
1317	0.378737871562799\\
1318	0.922238260469103\\
1319	0.925503984131956\\
1320	0.386622025920623\\
1321	-0.378737871562684\\
1322	-0.922238260469055\\
1323	-0.925503984131917\\
1324	-0.386622025920738\\
1325	0.378737871562779\\
1326	0.922238260469007\\
1327	0.925503984131964\\
1328	0.386622025920643\\
1329	-0.378737871562664\\
1330	-0.922238260469047\\
1331	-0.925503984132012\\
1332	-0.386622025920758\\
1333	0.378737871562759\\
1334	0.922238260468999\\
1335	0.925503984131973\\
1336	0.386622025920663\\
1337	-0.378737871562644\\
1338	-0.922238260469038\\
1339	-0.925503984131934\\
1340	-0.386622025920778\\
1341	0.378737871562739\\
1342	0.922238260469078\\
1343	0.925503984131981\\
1344	0.386622025920683\\
1345	-0.378737871562834\\
1346	-0.92223826046903\\
1347	-0.925503984131942\\
1348	-0.386622025920588\\
1349	0.378737871562719\\
1350	0.92223826046907\\
1351	0.925503984131903\\
1352	0.386622025920703\\
1353	-0.378737871562814\\
1354	-0.92223826046911\\
1355	-0.92550398413195\\
1356	-0.386622025920608\\
1357	0.378737871562699\\
1358	0.922238260469062\\
1359	0.925503984131911\\
1360	0.386622025920722\\
1361	-0.378737871562794\\
1362	-0.922238260469013\\
1363	-0.925503984131958\\
1364	-0.386622025920627\\
1365	0.378737871562679\\
1366	0.922238260469053\\
1367	0.925503984132005\\
1368	0.386622025920742\\
1369	-0.378737871562774\\
1370	-0.922238260469005\\
1371	-0.925503984131966\\
1372	-0.386622025920857\\
1373	0.378737871562659\\
1374	0.922238260469045\\
1375	0.925503984132014\\
1376	0.386622025920762\\
1377	-0.378737871562754\\
1378	-0.922238260468997\\
1379	-0.925503984131975\\
1380	-0.386622025920667\\
1381	0.378737871562639\\
1382	0.922238260469037\\
1383	0.925503984131936\\
1384	0.386622025920782\\
1385	-0.378737871562735\\
1386	-0.922238260469076\\
1387	-0.925503984131983\\
1388	-0.386622025920687\\
1389	0.37873787156283\\
1390	0.922238260469028\\
1391	0.925503984131944\\
1392	0.386622025920592\\
1393	-0.378737871562715\\
1394	-0.922238260469068\\
1395	-0.925503984131905\\
1396	-0.386622025920707\\
1397	0.37873787156281\\
1398	0.92223826046902\\
1399	0.925503984131952\\
1400	0.386622025920612\\
1401	-0.378737871562695\\
1402	-0.92223826046906\\
1403	-0.925503984131999\\
1404	-0.386622025920727\\
1405	0.37873787156279\\
1406	0.922238260469012\\
1407	0.92550398413196\\
1408	0.386622025920632\\
1409	-0.378737871562675\\
1410	-0.922238260469051\\
1411	-0.925503984132007\\
1412	-0.386622025920747\\
1413	0.37873787156277\\
1414	0.922238260469003\\
1415	0.925503984131968\\
1416	0.386622025920652\\
1417	-0.378737871562655\\
1418	-0.922238260469043\\
1419	-0.925503984132015\\
1420	-0.386622025920767\\
1421	0.37873787156275\\
1422	0.922238260468995\\
1423	0.925503984131976\\
1424	0.386622025920672\\
1425	-0.378737871562635\\
1426	-0.922238260469035\\
1427	-0.925503984131937\\
1428	-0.386622025920787\\
1429	0.37873787156273\\
1430	0.922238260469074\\
1431	0.925503984131985\\
1432	0.386622025920692\\
1433	-0.378737871562825\\
1434	-0.922238260469026\\
1435	-0.925503984131946\\
1436	-0.386622025920597\\
1437	0.37873787156271\\
1438	0.922238260469066\\
1439	0.925503984131907\\
1440	0.386622025920712\\
1441	-0.378737871562805\\
1442	-0.922238260469018\\
1443	-0.925503984131954\\
1444	-0.386622025920617\\
1445	0.37873787156269\\
1446	0.922238260469058\\
1447	0.925503984132001\\
1448	0.386622025920731\\
1449	-0.378737871562785\\
1450	-0.92223826046901\\
1451	-0.925503984131962\\
1452	-0.386622025920846\\
1453	0.37873787156267\\
1454	0.922238260469049\\
1455	0.925503984132009\\
1456	0.386622025920751\\
1457	-0.378737871562765\\
1458	-0.922238260469001\\
1459	-0.92550398413197\\
1460	-0.386622025920656\\
1461	0.37873787156265\\
1462	0.922238260469041\\
1463	0.925503984131931\\
1464	0.386622025920771\\
1465	-0.378737871562745\\
1466	-0.922238260469081\\
1467	-0.925503984131978\\
1468	-0.386622025920676\\
1469	0.378737871562841\\
1470	0.922238260469033\\
1471	0.925503984131939\\
1472	0.386622025920581\\
1473	-0.378737871562725\\
1474	-0.922238260469073\\
1475	-0.9255039841319\\
1476	-0.386622025920696\\
1477	0.378737871562821\\
1478	0.922238260469024\\
1479	0.925503984131947\\
1480	0.386622025920601\\
1481	-0.378737871562705\\
1482	-0.922238260469064\\
1483	-0.925503984131995\\
1484	-0.386622025920716\\
1485	0.378737871562801\\
1486	0.922238260469016\\
1487	0.925503984131956\\
1488	0.386622025920831\\
1489	-0.378737871562686\\
1490	-0.922238260469056\\
1491	-0.925503984132003\\
1492	-0.386622025920736\\
1493	0.37873787156257\\
1494	0.922238260469008\\
1495	0.925503984131964\\
1496	0.386622025920851\\
1497	-0.378737871562666\\
1498	-0.922238260469048\\
1499	-0.925503984132011\\
1500	-0.386622025920756\\
1501	0.378737871562761\\
1502	0.922238260468999\\
1503	0.925503984131886\\
1504	0.386622025920661\\
1505	-0.378737871562646\\
1506	-0.922238260469127\\
1507	-0.925503984131933\\
1508	-0.386622025920776\\
1509	0.37873787156253\\
1510	0.922238260469079\\
1511	0.92550398413198\\
1512	0.386622025920891\\
1513	-0.378737871562836\\
1514	-0.922238260469031\\
1515	-0.925503984132027\\
1516	-0.386622025920586\\
1517	0.378737871562721\\
1518	0.922238260468983\\
1519	0.925503984131902\\
1520	0.386622025920701\\
1521	-0.378737871562606\\
1522	-0.922238260469023\\
1523	-0.925503984131949\\
1524	-0.386622025920816\\
1525	0.378737871562701\\
1526	0.922238260469062\\
1527	0.925503984131996\\
1528	0.386622025920721\\
1529	-0.378737871562796\\
1530	-0.922238260469014\\
1531	-0.925503984131957\\
1532	-0.386622025920835\\
1533	0.378737871562681\\
1534	0.922238260469054\\
1535	0.925503984132005\\
1536	0.38662202592074\\
1537	-0.378737871562776\\
1538	-0.922238260469006\\
1539	-0.925503984131966\\
1540	-0.386622025920646\\
1541	0.378737871562661\\
1542	0.922238260468958\\
1543	0.925503984131927\\
1544	0.38662202592076\\
1545	-0.378737871562546\\
1546	-0.922238260469085\\
1547	-0.925503984131974\\
1548	-0.386622025920875\\
1549	0.378737871562851\\
1550	0.922238260469037\\
1551	0.925503984132021\\
1552	0.386622025920571\\
1553	-0.378737871562736\\
1554	-0.922238260468989\\
1555	-0.925503984131896\\
1556	-0.386622025920685\\
1557	0.378737871562621\\
1558	0.922238260469029\\
1559	0.925503984131943\\
1560	0.3866220259208\\
1561	-0.378737871562716\\
1562	-0.922238260469069\\
1563	-0.92550398413199\\
1564	-0.386622025920705\\
1565	0.378737871562812\\
1566	0.922238260469021\\
1567	0.925503984131951\\
1568	0.38662202592082\\
1569	-0.378737871562696\\
1570	-0.92223826046906\\
1571	-0.925503984131998\\
1572	-0.386622025920725\\
1573	0.378737871562792\\
1574	0.922238260469012\\
1575	0.925503984131959\\
1576	0.38662202592063\\
1577	-0.378737871562676\\
1578	-0.922238260468964\\
1579	-0.92550398413192\\
1580	-0.386622025920745\\
1581	0.378737871562561\\
1582	0.922238260469092\\
1583	0.925503984131967\\
1584	0.38662202592086\\
1585	-0.378737871562867\\
1586	-0.922238260469044\\
1587	-0.925503984132015\\
1588	-0.386622025920555\\
1589	0.378737871562752\\
1590	0.922238260468996\\
1591	0.92550398413189\\
1592	0.38662202592067\\
1593	-0.378737871562637\\
1594	-0.922238260469035\\
1595	-0.925503984131937\\
1596	-0.386622025920785\\
1597	0.378737871562732\\
1598	0.922238260469075\\
1599	0.925503984131984\\
1600	0.38662202592069\\
1601	-0.378737871562827\\
1602	-0.922238260469027\\
1603	-0.925503984131945\\
1604	-0.386622025920595\\
1605	0.378737871562712\\
1606	0.922238260469067\\
1607	0.925503984131992\\
1608	0.38662202592071\\
1609	-0.378737871562807\\
1610	-0.922238260469019\\
1611	-0.925503984131953\\
1612	-0.386622025920615\\
1613	0.378737871562692\\
1614	0.922238260469059\\
1615	0.925503984131914\\
1616	0.38662202592073\\
1617	-0.378737871562577\\
1618	-0.922238260469098\\
1619	-0.925503984131961\\
1620	-0.386622025920844\\
1621	0.378737871562882\\
1622	0.92223826046905\\
1623	0.925503984132008\\
1624	0.38662202592054\\
1625	-0.378737871562767\\
1626	-0.922238260469002\\
1627	-0.925503984131883\\
1628	-0.386622025920655\\
1629	0.378737871562652\\
1630	0.922238260468954\\
1631	0.92550398413193\\
1632	0.386622025920769\\
1633	-0.378737871562537\\
1634	-0.922238260469082\\
1635	-0.925503984131978\\
1636	-0.386622025920884\\
1637	0.378737871562842\\
1638	0.922238260469034\\
1639	0.925503984132025\\
1640	0.38662202592058\\
1641	-0.378737871562727\\
1642	-0.922238260468985\\
1643	-0.925503984131986\\
1644	-0.386622025920694\\
1645	0.378737871562612\\
1646	0.922238260469025\\
1647	0.925503984131947\\
1648	0.386622025920809\\
1649	-0.378737871562707\\
1650	-0.922238260469065\\
1651	-0.925503984131994\\
1652	-0.386622025920714\\
1653	0.378737871562592\\
1654	0.922238260469017\\
1655	0.925503984131955\\
1656	0.386622025920829\\
1657	-0.378737871562687\\
1658	-0.922238260469057\\
1659	-0.925503984132002\\
1660	-0.386622025920734\\
1661	0.378737871562783\\
1662	0.922238260469008\\
1663	0.925503984132049\\
1664	0.386622025920639\\
1665	-0.378737871562667\\
1666	-0.92223826046896\\
1667	-0.925503984131924\\
1668	-0.386622025920754\\
1669	0.378737871562552\\
1670	0.922238260469088\\
1671	0.925503984131971\\
1672	0.386622025920869\\
1673	-0.378737871562858\\
1674	-0.92223826046904\\
1675	-0.925503984132018\\
1676	-0.386622025920564\\
1677	0.378737871562743\\
1678	0.922238260468992\\
1679	0.925503984131979\\
1680	0.386622025920679\\
1681	-0.378737871562627\\
1682	-0.922238260469032\\
1683	-0.92550398413194\\
1684	-0.386622025920794\\
1685	0.378737871562723\\
1686	0.922238260469071\\
1687	0.925503984131988\\
1688	0.386622025920699\\
1689	-0.378737871562818\\
1690	-0.922238260469023\\
1691	-0.925503984131949\\
1692	-0.386622025920814\\
1693	0.378737871562703\\
1694	0.922238260469063\\
1695	0.925503984131996\\
1696	0.386622025920719\\
1697	-0.378737871562798\\
1698	-0.922238260469015\\
1699	-0.925503984131957\\
1700	-0.386622025920624\\
1701	0.378737871562683\\
1702	0.922238260468967\\
1703	0.925503984131918\\
1704	0.386622025920739\\
1705	-0.378737871562568\\
1706	-0.922238260469095\\
1707	-0.925503984131965\\
1708	-0.386622025920853\\
1709	0.378737871562873\\
1710	0.922238260469046\\
1711	0.925503984132012\\
1712	0.386622025920549\\
1713	-0.378737871562758\\
1714	-0.922238260468998\\
1715	-0.925503984131887\\
1716	-0.386622025920664\\
1717	0.378737871562643\\
1718	0.922238260469038\\
1719	0.925503984131934\\
1720	0.386622025920778\\
1721	-0.378737871562738\\
1722	-0.922238260469078\\
1723	-0.925503984131981\\
1724	-0.386622025920684\\
1725	0.378737871562833\\
1726	0.92223826046903\\
1727	0.925503984131942\\
1728	0.386622025920798\\
1729	-0.378737871562718\\
1730	-0.92223826046907\\
1731	-0.925503984131989\\
1732	-0.386622025920703\\
1733	0.378737871562813\\
1734	0.922238260469021\\
1735	0.92550398413195\\
1736	0.386622025920609\\
1737	-0.378737871562698\\
1738	-0.922238260468973\\
1739	-0.925503984131911\\
1740	-0.386622025920723\\
1741	0.378737871562583\\
1742	0.922238260469101\\
1743	0.925503984131959\\
1744	0.386622025920838\\
1745	-0.378737871562889\\
1746	-0.922238260469053\\
1747	-0.925503984132006\\
1748	-0.386622025920953\\
1749	0.378737871562773\\
1750	0.922238260469005\\
1751	0.925503984132053\\
1752	0.386622025920648\\
1753	-0.378737871562658\\
1754	-0.922238260468957\\
1755	-0.925503984131928\\
1756	-0.386622025920763\\
1757	0.378737871562543\\
1758	0.922238260469084\\
1759	0.925503984131975\\
1760	0.386622025920878\\
1761	0.13465238010551\\
1762	-0.126397742303971\\
1763	-0.378834067457586\\
1764	-0.605453476966963\\
1765	-0.7908122325802\\
1766	-0.922278441422032\\
1767	-0.990892898618178\\
1768	-0.991979642301416\\
1769	-0.925464612685707\\
1770	-0.795880699118006\\
1771	-0.612058831160518\\
1772	-0.386526165334499\\
1773	-0.134652380105521\\
1774	0.126397742303961\\
1775	0.378834067457681\\
1776	0.605453476966955\\
1777	0.790812232580194\\
1778	0.922278441422028\\
1779	0.990892898618177\\
1780	0.991979642301418\\
1781	0.925464612685711\\
1782	0.795880699118012\\
1783	0.612058831160526\\
1784	0.386526165334509\\
1785	0.134652380105532\\
1786	-0.126397742304063\\
1787	-0.378834067457671\\
1788	-0.605453476967037\\
1789	-0.790812232580187\\
1790	-0.922278441422023\\
1791	-0.990892898618175\\
1792	-0.991979642301419\\
1793	-0.925464612685715\\
1794	-0.795880699118019\\
1795	-0.612058831160535\\
1796	-0.386526165334414\\
1797	-0.134652380105542\\
1798	0.126397742304052\\
1799	0.378834067457661\\
1800	0.605453476967028\\
1801	0.790812232580181\\
1802	0.922278441422019\\
1803	0.990892898618174\\
1804	0.99197964230142\\
1805	0.925464612685719\\
1806	0.795880699118025\\
1807	0.612058831160543\\
1808	0.386526165334424\\
1809	0.13465238010544\\
1810	-0.126397742304041\\
1811	-0.378834067457651\\
1812	-0.60545347696702\\
1813	-0.790812232580174\\
1814	-0.922278441422015\\
1815	-0.990892898618172\\
1816	-0.991979642301422\\
1817	-0.925464612685723\\
1818	-0.795880699118032\\
1819	-0.612058831160462\\
1820	-0.386526165334434\\
1821	-0.134652380105451\\
1822	0.126397742304031\\
1823	0.378834067457641\\
1824	0.605453476967011\\
1825	0.790812232580167\\
1826	0.922278441422011\\
1827	0.990892898618171\\
1828	0.991979642301423\\
1829	0.925464612685684\\
1830	0.795880699117969\\
1831	0.612058831160471\\
1832	0.386526165334444\\
1833	0.134652380105462\\
1834	-0.12639774230402\\
1835	-0.378834067457631\\
1836	-0.605453476967002\\
1837	-0.790812232580161\\
1838	-0.922278441422007\\
1839	-0.990892898618169\\
1840	-0.99197964230141\\
1841	-0.925464612685688\\
1842	-0.795880699117976\\
1843	-0.612058831160479\\
1844	-0.386526165334454\\
1845	-0.134652380105472\\
1846	0.126397742304009\\
1847	0.378834067457621\\
1848	0.605453476966994\\
1849	0.790812232580154\\
1850	0.922278441422047\\
1851	0.990892898618168\\
1852	0.991979642301412\\
1853	0.925464612685692\\
1854	0.795880699117982\\
1855	0.612058831160488\\
1856	0.386526165334464\\
1857	0.134652380105483\\
1858	-0.126397742303999\\
1859	-0.378834067457611\\
1860	-0.605453476966985\\
1861	-0.790812232580148\\
1862	-0.922278441422042\\
1863	-0.990892898618182\\
1864	-0.991979642301413\\
1865	-0.925464612685696\\
1866	-0.795880699117989\\
1867	-0.612058831160496\\
1868	-0.386526165334474\\
1869	-0.134652380105606\\
1870	0.126397742303988\\
1871	0.378834067457601\\
1872	0.605453476966977\\
1873	0.790812232580211\\
1874	0.922278441422038\\
1875	0.990892898618165\\
1876	0.991979642301414\\
1877	0.9254646126857\\
1878	0.795880699118064\\
1879	0.612058831160505\\
1880	0.386526165334484\\
1881	0.134652380105617\\
1882	-0.126397742303977\\
1883	-0.378834067457697\\
1884	-0.605453476966968\\
1885	-0.790812232580204\\
1886	-0.922278441422034\\
1887	-0.990892898618164\\
1888	-0.991979642301416\\
1889	-0.925464612685704\\
1890	-0.795880699118071\\
1891	-0.612058831160513\\
1892	-0.386526165334493\\
1893	-0.134652380105515\\
1894	0.126397742304079\\
1895	0.378834067457687\\
1896	0.60545347696696\\
1897	0.790812232580197\\
1898	0.92227844142203\\
1899	0.990892898618162\\
1900	0.991979642301417\\
1901	0.925464612685709\\
1902	0.795880699118077\\
1903	0.612058831160522\\
1904	0.386526165334399\\
1905	0.134652380105526\\
1906	-0.126397742304069\\
1907	-0.378834067457677\\
1908	-0.605453476966951\\
1909	-0.790812232580191\\
1910	-0.922278441422026\\
1911	-0.990892898618161\\
1912	-0.991979642301418\\
1913	-0.925464612685713\\
1914	-0.795880699118015\\
1915	-0.61205883116044\\
1916	-0.386526165334409\\
1917	-0.134652380105536\\
1918	0.126397742304058\\
1919	0.378834067457667\\
1920	0.605453476966942\\
1921	-0.378737871562871\\
1922	-0.922238260469045\\
1923	-0.925503984132013\\
1924	-0.386622025920761\\
1925	0.378737871562755\\
1926	0.922238260468997\\
1927	0.925503984131974\\
1928	0.386622025920666\\
1929	-0.37873787156264\\
1930	-0.922238260469037\\
1931	-0.925503984131935\\
1932	-0.386622025920781\\
1933	0.378737871562735\\
1934	0.922238260468989\\
1935	0.925503984131982\\
1936	0.386622025920686\\
1937	-0.37873787156262\\
1938	-0.922238260469029\\
1939	-0.925503984131943\\
1940	-0.386622025920801\\
1941	0.378737871562715\\
1942	0.922238260469068\\
1943	0.925503984131991\\
1944	0.386622025920916\\
1945	-0.378737871562811\\
1946	-0.92223826046902\\
1947	-0.925503984132038\\
1948	-0.386622025920611\\
1949	0.378737871562696\\
1950	0.922238260468972\\
1951	0.925503984131912\\
1952	0.386622025920726\\
1953	-0.37873787156258\\
1954	-0.9222382604691\\
1955	-0.92550398413196\\
1956	-0.386622025920841\\
1957	0.378737871562886\\
1958	0.922238260469052\\
1959	0.925503984132007\\
1960	0.386622025920746\\
1961	-0.378737871562771\\
1962	-0.922238260469004\\
1963	-0.925503984131968\\
1964	-0.386622025920651\\
1965	0.378737871562656\\
1966	0.922238260469043\\
1967	0.925503984131929\\
1968	0.386622025920766\\
1969	-0.378737871562751\\
1970	-0.922238260469083\\
1971	-0.925503984131976\\
1972	-0.386622025920671\\
1973	0.378737871562636\\
1974	0.922238260469035\\
1975	0.925503984131937\\
1976	0.386622025920786\\
1977	-0.378737871562731\\
1978	-0.922238260469075\\
1979	-0.925503984131984\\
1980	-0.386622025920691\\
1981	0.378737871562826\\
1982	0.922238260469027\\
1983	0.925503984132031\\
1984	0.386622025920596\\
1985	-0.378737871562711\\
1986	-0.922238260468979\\
1987	-0.925503984131906\\
1988	-0.386622025920711\\
1989	0.378737871562596\\
1990	0.922238260469018\\
1991	0.925503984131953\\
1992	0.386622025920825\\
1993	-0.378737871562481\\
1994	-0.922238260469058\\
1995	-0.925503984132001\\
1996	-0.38662202592094\\
1997	0.378737871562786\\
1998	0.92223826046901\\
1999	0.925503984132048\\
2000	0.386622025920636\\
2001	-0.378737871562671\\
2002	-0.92223826046905\\
2003	-0.925503984131923\\
2004	-0.38662202592075\\
2005	0.378737871562766\\
2006	0.92223826046909\\
2007	0.92550398413197\\
2008	0.386622025920656\\
2009	-0.378737871562651\\
2010	-0.922238260469041\\
2011	-0.925503984131931\\
2012	-0.38662202592077\\
2013	0.378737871562746\\
2014	0.922238260468993\\
2015	0.925503984131978\\
2016	0.386622025920675\\
2017	-0.378737871562631\\
2018	-0.922238260469033\\
2019	-0.925503984132025\\
2020	-0.38662202592079\\
2021	0.378737871562726\\
2022	0.922238260468985\\
2023	0.925503984131986\\
2024	0.386622025920695\\
2025	-0.378737871562611\\
2026	-0.922238260469025\\
2027	-0.925503984131947\\
2028	-0.38662202592081\\
2029	0.378737871562496\\
2030	0.922238260469065\\
2031	0.925503984131994\\
2032	0.386622025920925\\
2033	-0.378737871562802\\
2034	-0.922238260469016\\
2035	-0.925503984132041\\
2036	-0.38662202592062\\
2037	0.378737871562686\\
2038	0.922238260468968\\
2039	0.925503984131916\\
2040	0.386622025920735\\
2041	-0.378737871562571\\
2042	-0.922238260469096\\
2043	-0.925503984131963\\
2044	-0.38662202592085\\
2045	0.378737871562666\\
2046	0.922238260469048\\
2047	0.925503984132011\\
2048	0.386622025920755\\
2049	-0.378737871562762\\
2050	-0.922238260469\\
2051	-0.925503984131972\\
2052	-0.38662202592066\\
2053	0.378737871562647\\
2054	0.92223826046904\\
2055	0.925503984132019\\
2056	0.386622025920775\\
2057	-0.378737871562742\\
2058	-0.922238260468991\\
2059	-0.92550398413198\\
2060	-0.38662202592068\\
2061	0.378737871562627\\
2062	0.922238260469031\\
2063	0.925503984131941\\
2064	0.386622025920795\\
2065	-0.378737871562722\\
2066	-0.922238260469071\\
2067	-0.925503984131988\\
2068	-0.386622025920909\\
2069	0.378737871562817\\
2070	0.922238260469023\\
2071	0.925503984132035\\
2072	0.386622025920605\\
2073	-0.378737871562702\\
2074	-0.922238260468975\\
2075	-0.92550398413191\\
2076	-0.38662202592072\\
2077	0.378737871562587\\
2078	0.922238260469103\\
2079	0.925503984131957\\
2080	0.386622025920835\\
2081	-0.378737871562892\\
2082	-0.922238260469054\\
2083	-0.925503984132004\\
2084	-0.38662202592074\\
2085	0.378737871562777\\
2086	0.922238260469006\\
2087	0.925503984131965\\
2088	0.386622025920645\\
2089	-0.378737871562662\\
2090	-0.922238260469046\\
2091	-0.925503984131926\\
2092	-0.386622025920759\\
2093	0.378737871562757\\
2094	0.922238260468998\\
2095	0.925503984131973\\
2096	0.386622025920665\\
2097	-0.378737871562642\\
2098	-0.922238260469038\\
2099	-0.925503984131934\\
2100	-0.386622025920779\\
2101	0.378737871562737\\
2102	0.922238260469077\\
2103	0.925503984131982\\
2104	0.386622025920894\\
2105	-0.378737871562832\\
2106	-0.922238260469029\\
2107	-0.925503984132029\\
2108	-0.38662202592059\\
2109	0.378737871562717\\
2110	0.922238260468981\\
2111	0.925503984131904\\
2112	0.386622025920704\\
2113	-0.378737871562602\\
2114	-0.922238260469109\\
2115	-0.925503984131951\\
2116	-0.386622025920819\\
2117	0.378737871562908\\
2118	0.922238260469061\\
2119	0.925503984131998\\
2120	0.386622025920724\\
2121	-0.378737871562793\\
2122	-0.922238260469013\\
2123	-0.925503984131959\\
2124	-0.386622025920629\\
2125	0.378737871562677\\
2126	0.922238260469052\\
2127	0.92550398413192\\
2128	0.386622025920744\\
2129	-0.378737871562773\\
2130	-0.922238260469004\\
2131	-0.925503984131967\\
2132	-0.386622025920649\\
2133	0.378737871562657\\
2134	0.922238260469044\\
2135	0.925503984132014\\
2136	0.386622025920764\\
2137	-0.378737871562753\\
2138	-0.922238260468996\\
2139	-0.925503984131975\\
2140	-0.386622025920879\\
2141	0.378737871562637\\
2142	0.922238260469036\\
2143	0.925503984132022\\
2144	0.386622025920784\\
2145	-0.378737871562733\\
2146	-0.922238260468988\\
2147	-0.925503984131983\\
2148	-0.386622025920689\\
2149	0.378737871562618\\
2150	0.922238260469027\\
2151	0.925503984131944\\
2152	0.386622025920804\\
2153	-0.378737871562502\\
2154	-0.922238260469067\\
2155	-0.925503984131992\\
2156	-0.386622025920919\\
2157	0.378737871562808\\
2158	0.922238260469019\\
2159	0.925503984132039\\
2160	0.386622025920614\\
2161	-0.378737871562693\\
2162	-0.922238260468971\\
2163	-0.925503984131914\\
2164	-0.386622025920729\\
2165	0.378737871562578\\
2166	0.922238260469099\\
2167	0.925503984131961\\
2168	0.386622025920844\\
2169	-0.378737871562673\\
2170	-0.922238260469051\\
2171	-0.925503984132008\\
2172	-0.386622025920749\\
2173	0.378737871562768\\
2174	0.922238260469002\\
2175	0.925503984131969\\
2176	0.386622025920654\\
2177	-0.378737871562653\\
2178	-0.922238260469042\\
2179	-0.925503984132016\\
2180	-0.386622025920768\\
2181	0.378737871562748\\
2182	0.922238260468994\\
2183	0.925503984131977\\
2184	0.386622025920674\\
2185	-0.378737871562633\\
2186	-0.922238260469034\\
2187	-0.925503984131938\\
2188	-0.386622025920788\\
2189	0.378737871562518\\
2190	0.922238260469074\\
2191	0.925503984131985\\
2192	0.386622025920903\\
2193	-0.378737871562823\\
2194	-0.922238260469026\\
2195	-0.925503984132032\\
2196	-0.386622025920599\\
2197	0.378737871562708\\
2198	0.922238260468977\\
2199	0.925503984131907\\
2200	0.386622025920713\\
2201	-0.378737871562593\\
2202	-0.922238260469105\\
2203	-0.925503984131954\\
2204	-0.386622025920828\\
2205	0.378737871562688\\
2206	0.922238260469057\\
2207	0.925503984132002\\
2208	0.386622025920733\\
2209	-0.378737871562783\\
2210	-0.922238260469009\\
2211	-0.925503984131963\\
2212	-0.386622025920638\\
2213	0.378737871562668\\
2214	0.922238260469049\\
2215	0.92550398413201\\
2216	0.386622025920753\\
2217	-0.378737871562764\\
2218	-0.922238260469001\\
2219	-0.925503984131971\\
2220	-0.386622025920658\\
2221	0.378737871562648\\
2222	0.92223826046904\\
2223	0.925503984131932\\
2224	0.386622025920773\\
2225	-0.378737871562533\\
2226	-0.92223826046908\\
2227	-0.925503984131979\\
2228	-0.386622025920888\\
2229	0.378737871562839\\
2230	0.922238260469032\\
2231	0.925503984132026\\
2232	0.386622025920583\\
2233	-0.378737871562724\\
2234	-0.922238260468984\\
2235	-0.925503984131901\\
2236	-0.386622025920698\\
2237	0.378737871562608\\
2238	0.922238260469112\\
2239	0.925503984131948\\
2240	0.386622025920813\\
2241	-0.378737871562704\\
2242	-0.922238260469063\\
2243	-0.925503984131995\\
2244	-0.386622025920718\\
2245	0.378737871562799\\
2246	0.922238260469015\\
2247	0.925503984131956\\
2248	0.386622025920623\\
2249	-0.378737871562684\\
2250	-0.922238260469055\\
2251	-0.925503984132003\\
2252	-0.386622025920738\\
2253	0.378737871562779\\
2254	0.922238260469007\\
2255	0.925503984131964\\
2256	0.386622025920853\\
2257	-0.378737871562664\\
2258	-0.922238260469047\\
2259	-0.925503984132012\\
2260	-0.386622025920758\\
2261	0.378737871562759\\
2262	0.922238260468999\\
2263	0.925503984131973\\
2264	0.386622025920872\\
2265	-0.378737871562644\\
2266	-0.922238260469038\\
2267	-0.92550398413202\\
2268	-0.386622025920778\\
2269	0.378737871562739\\
2270	0.92223826046899\\
2271	0.925503984131981\\
2272	0.386622025920683\\
2273	-0.378737871562624\\
2274	-0.922238260468942\\
2275	-0.925503984131942\\
2276	-0.386622025920797\\
2277	0.378737871562509\\
2278	0.92223826046907\\
2279	0.925503984131989\\
2280	0.386622025920912\\
2281	-0.378737871562814\\
2282	-0.922238260469022\\
2283	-0.925503984132036\\
2284	-0.386622025920608\\
2285	0.378737871562699\\
2286	0.922238260468974\\
2287	0.925503984131911\\
2288	0.386622025920722\\
2289	-0.378737871562584\\
2290	-0.922238260469013\\
2291	-0.925503984131958\\
2292	-0.386622025920837\\
2293	0.378737871562679\\
2294	0.922238260469053\\
2295	0.925503984132005\\
2296	0.386622025920742\\
2297	-0.378737871562774\\
2298	-0.922238260469005\\
2299	-0.925503984131966\\
2300	-0.386622025920857\\
2301	0.378737871562659\\
2302	0.922238260469045\\
2303	0.925503984132014\\
2304	0.386622025920762\\
2305	-0.378737871562754\\
2306	-0.922238260468997\\
2307	-0.925503984131975\\
2308	-0.386622025920667\\
2309	0.378737871562639\\
2310	0.922238260468949\\
2311	0.925503984131936\\
2312	0.386622025920782\\
2313	-0.378737871562524\\
2314	-0.922238260469076\\
2315	-0.925503984131983\\
2316	-0.386622025920897\\
2317	0.37873787156283\\
2318	0.922238260469028\\
2319	0.92550398413203\\
2320	0.386622025920592\\
2321	-0.378737871562715\\
2322	-0.92223826046898\\
2323	-0.925503984131905\\
2324	-0.386622025920707\\
2325	0.378737871562599\\
2326	0.92223826046902\\
2327	0.925503984131952\\
2328	0.386622025920822\\
2329	-0.378737871562695\\
2330	-0.92223826046906\\
2331	-0.925503984131999\\
2332	-0.386622025920727\\
2333	0.37873787156279\\
2334	0.922238260469012\\
2335	0.92550398413196\\
2336	0.386622025920842\\
2337	-0.378737871562675\\
2338	-0.922238260469051\\
2339	-0.925503984132007\\
2340	-0.386622025920747\\
2341	0.37873787156277\\
2342	0.922238260469003\\
2343	0.925503984131968\\
2344	0.386622025920652\\
2345	-0.378737871562655\\
2346	-0.922238260469043\\
2347	-0.925503984131929\\
2348	-0.386622025920767\\
2349	0.37873787156254\\
2350	0.922238260469083\\
2351	0.925503984131976\\
2352	0.386622025920881\\
2353	-0.378737871562845\\
2354	-0.922238260469035\\
2355	-0.925503984132024\\
2356	-0.386622025920577\\
2357	0.37873787156273\\
2358	0.922238260468986\\
2359	0.925503984131898\\
2360	0.386622025920692\\
2361	-0.378737871562615\\
2362	-0.922238260469114\\
2363	-0.925503984131946\\
2364	-0.386622025920806\\
2365	0.37873787156271\\
2366	0.922238260469066\\
2367	0.925503984131993\\
2368	0.386622025920712\\
2369	-0.378737871562805\\
2370	-0.922238260469018\\
2371	-0.925503984131954\\
2372	-0.386622025920617\\
2373	0.37873787156269\\
2374	0.92223826046897\\
2375	0.925503984132001\\
2376	0.386622025920731\\
2377	-0.378737871562575\\
2378	-0.92223826046901\\
2379	-0.925503984131962\\
2380	-0.386622025920846\\
2381	0.37873787156267\\
2382	0.922238260469049\\
2383	0.925503984132009\\
2384	0.386622025920751\\
2385	-0.378737871562555\\
2386	-0.922238260469001\\
2387	-0.92550398413197\\
2388	-0.386622025920866\\
2389	0.37873787156265\\
2390	0.922238260469041\\
2391	0.925503984132017\\
2392	0.386622025920771\\
2393	-0.378737871562745\\
2394	-0.922238260468993\\
2395	-0.925503984132064\\
2396	-0.386622025920676\\
2397	0.37873787156263\\
2398	0.922238260468945\\
2399	0.925503984131939\\
2400	0.386622025920791\\
2401	0.790812232580129\\
2402	0.922278441422031\\
2403	0.990892898618162\\
2404	0.991979642301431\\
2405	0.925464612685708\\
2406	0.795880699118076\\
2407	0.61205883116061\\
2408	0.386526165334502\\
2409	0.134652380105524\\
2410	-0.126397742303958\\
2411	-0.378834067457679\\
2412	-0.605453476966952\\
2413	-0.790812232580122\\
2414	-0.922278441422026\\
2415	-0.990892898618161\\
2416	-0.991979642301432\\
2417	-0.925464612685712\\
2418	-0.795880699118083\\
2419	-0.612058831160529\\
2420	-0.386526165334407\\
2421	-0.134652380105534\\
2422	0.126397742303947\\
2423	0.378834067457669\\
2424	0.605453476966944\\
2425	0.790812232580116\\
2426	0.922278441422022\\
2427	0.990892898618159\\
2428	0.991979642301434\\
2429	0.925464612685716\\
2430	0.79588069911802\\
2431	0.612058831160537\\
2432	0.386526165334417\\
2433	0.134652380105545\\
2434	-0.126397742303936\\
2435	-0.378834067457659\\
2436	-0.605453476966935\\
2437	-0.790812232580109\\
2438	-0.922278441422018\\
2439	-0.990892898618158\\
2440	-0.991979642301421\\
2441	-0.925464612685677\\
2442	-0.795880699118027\\
2443	-0.612058831160546\\
2444	-0.386526165334427\\
2445	-0.134652380105556\\
2446	0.126397742303926\\
2447	0.378834067457649\\
2448	0.605453476966927\\
2449	0.790812232580103\\
2450	0.922278441422014\\
2451	0.990892898618172\\
2452	0.991979642301422\\
2453	0.925464612685681\\
2454	0.795880699118033\\
2455	0.612058831160554\\
2456	0.386526165334437\\
2457	0.134652380105567\\
2458	-0.126397742303915\\
2459	-0.378834067457639\\
2460	-0.605453476966918\\
2461	-0.790812232580166\\
2462	-0.92227844142201\\
2463	-0.99089289861817\\
2464	-0.991979642301424\\
2465	-0.925464612685685\\
2466	-0.79588069911804\\
2467	-0.612058831160563\\
2468	-0.386526165334447\\
2469	-0.134652380105577\\
2470	0.126397742303904\\
2471	0.378834067457629\\
2472	0.60545347696691\\
2473	0.790812232580159\\
2474	0.92227844142205\\
2475	0.990892898618169\\
2476	0.991979642301425\\
2477	0.925464612685689\\
2478	0.795880699118046\\
2479	0.612058831160571\\
2480	0.386526165334456\\
2481	0.134652380105588\\
2482	-0.126397742303894\\
2483	-0.378834067457619\\
2484	-0.605453476966991\\
2485	-0.790812232580152\\
2486	-0.922278441422045\\
2487	-0.990892898618168\\
2488	-0.991979642301426\\
2489	-0.925464612685693\\
2490	-0.795880699118053\\
2491	-0.61205883116058\\
2492	-0.386526165334466\\
2493	-0.134652380105599\\
2494	0.126397742303996\\
2495	0.378834067457609\\
2496	0.605453476966983\\
2497	0.790812232580146\\
2498	0.922278441421997\\
2499	0.990892898618166\\
2500	0.991979642301428\\
2501	0.92546461268574\\
2502	0.795880699118059\\
2503	0.612058831160588\\
2504	0.386526165334476\\
2505	0.134652380105497\\
2506	-0.126397742303985\\
2507	-0.378834067457599\\
2508	-0.605453476966974\\
2509	-0.790812232580139\\
2510	-0.922278441422037\\
2511	-0.990892898618165\\
2512	-0.991979642301429\\
2513	-0.925464612685745\\
2514	-0.795880699118066\\
2515	-0.612058831160507\\
2516	-0.386526165334486\\
2517	-0.134652380105507\\
2518	0.126397742303974\\
2519	0.378834067457589\\
2520	0.605453476966966\\
2521	0.790812232580133\\
2522	0.922278441422033\\
2523	0.990892898618163\\
2524	0.99197964230143\\
2525	0.925464612685706\\
2526	0.795880699118073\\
2527	0.612058831160516\\
2528	0.386526165334391\\
2529	0.134652380105518\\
2530	-0.126397742303964\\
2531	-0.378834067457579\\
2532	-0.605453476966867\\
2533	-0.790812232580126\\
2534	-0.922278441422029\\
2535	-0.990892898618162\\
2536	-0.991979642301417\\
2537	-0.92546461268571\\
2538	-0.795880699118079\\
2539	-0.612058831160524\\
2540	-0.386526165334401\\
2541	-0.134652380105529\\
2542	0.126397742303953\\
2543	0.378834067457569\\
2544	0.605453476966858\\
2545	0.790812232580119\\
2546	0.922278441422025\\
2547	0.99089289861816\\
2548	0.991979642301419\\
2549	0.925464612685714\\
2550	0.795880699118086\\
2551	0.612058831160533\\
2552	0.386526165334411\\
2553	0.134652380105539\\
2554	-0.126397742303942\\
2555	-0.378834067457559\\
2556	-0.60545347696694\\
2557	-0.790812232580113\\
2558	-0.92227844142202\\
2559	-0.990892898618174\\
2560	-0.99197964230142\\
2561	-0.925464612685718\\
2562	-0.795880699118092\\
2563	-0.612058831160541\\
2564	-0.386526165334421\\
2565	-0.13465238010555\\
2566	0.126397742303932\\
2567	0.378834067457549\\
2568	0.605453476966931\\
2569	0.790812232580176\\
2570	0.92227844142206\\
2571	0.990892898618173\\
2572	0.991979642301421\\
2573	0.925464612685722\\
2574	0.795880699118099\\
2575	0.61205883116055\\
2576	0.386526165334431\\
2577	0.134652380105561\\
2578	-0.126397742303921\\
2579	-0.378834067457644\\
2580	-0.605453476966923\\
2581	-0.790812232580169\\
2582	-0.922278441422056\\
2583	-0.990892898618171\\
2584	-0.991979642301423\\
2585	-0.925464612685726\\
2586	-0.795880699118105\\
2587	-0.612058831160558\\
2588	-0.386526165334546\\
2589	-0.134652380105571\\
2590	0.126397742304023\\
2591	0.378834067457634\\
2592	0.605453476966914\\
2593	0.790812232580163\\
2594	0.922278441422008\\
2595	0.99089289861817\\
2596	0.991979642301424\\
2597	0.92546461268573\\
2598	0.795880699118112\\
2599	0.612058831160567\\
2600	0.386526165334451\\
2601	0.134652380105582\\
2602	-0.126397742304012\\
2603	-0.378834067457624\\
2604	-0.605453476966906\\
2605	-0.790812232580156\\
2606	-0.922278441422004\\
2607	-0.990892898618168\\
2608	-0.991979642301426\\
2609	-0.925464612685734\\
2610	-0.795880699118049\\
2611	-0.612058831160575\\
2612	-0.386526165334461\\
2613	-0.13465238010548\\
2614	0.126397742304002\\
2615	0.378834067457614\\
2616	0.605453476966897\\
2617	0.790812232580149\\
2618	0.922278441422\\
2619	0.990892898618167\\
2620	0.991979642301427\\
2621	0.925464612685738\\
2622	0.795880699118056\\
2623	0.612058831160494\\
2624	0.386526165334471\\
2625	0.134652380105491\\
2626	-0.126397742303991\\
2627	-0.378834067457604\\
2628	-0.605453476966888\\
2629	-0.790812232580143\\
2630	-0.922278441421995\\
2631	-0.990892898618165\\
2632	-0.991979642301428\\
2633	-0.925464612685699\\
2634	-0.795880699118062\\
2635	-0.612058831160502\\
2636	-0.386526165334481\\
2637	-0.134652380105501\\
2638	0.12639774230398\\
2639	0.378834067457594\\
2640	0.60545347696688\\
2641	0.790812232580136\\
2642	0.922278441421991\\
2643	0.990892898618164\\
2644	0.991979642301415\\
2645	0.925464612685703\\
2646	0.795880699118069\\
2647	0.612058831160601\\
2648	0.386526165334491\\
2649	0.134652380105512\\
2650	-0.126397742303969\\
2651	-0.378834067457584\\
2652	-0.605453476966871\\
2653	-0.79081223258013\\
2654	-0.922278441422031\\
2655	-0.990892898618163\\
2656	-0.991979642301417\\
2657	-0.925464612685707\\
2658	-0.795880699118075\\
2659	-0.612058831160609\\
2660	-0.386526165334501\\
2661	-0.134652380105523\\
2662	0.126397742303959\\
2663	0.378834067457574\\
2664	0.605453476966953\\
2665	0.790812232580123\\
2666	0.922278441422027\\
2667	0.990892898618176\\
2668	0.991979642301418\\
2669	0.925464612685711\\
2670	0.795880699118082\\
2671	0.612058831160618\\
2672	0.386526165334511\\
2673	0.134652380105534\\
2674	-0.126397742303948\\
2675	-0.378834067457669\\
2676	-0.605453476966945\\
2677	-0.790812232580116\\
2678	-0.922278441422023\\
2679	-0.990892898618175\\
2680	-0.991979642301419\\
2681	-0.925464612685716\\
2682	-0.795880699118089\\
2683	-0.612058831160626\\
2684	-0.386526165334521\\
2685	-0.134652380105544\\
2686	0.126397742303937\\
2687	0.37883406745766\\
2688	0.605453476966936\\
2689	0.79081223258011\\
2690	0.922278441422018\\
2691	0.990892898618173\\
2692	0.991979642301421\\
2693	0.92546461268572\\
2694	0.795880699118095\\
2695	0.612058831160545\\
2696	0.386526165334531\\
2697	0.134652380105555\\
2698	-0.126397742304039\\
2699	-0.37883406745765\\
2700	-0.605453476966928\\
2701	-0.790812232580103\\
2702	-0.922278441422014\\
2703	-0.990892898618172\\
2704	-0.991979642301422\\
2705	-0.925464612685724\\
2706	-0.795880699118102\\
2707	-0.612058831160553\\
2708	-0.386526165334436\\
2709	-0.134652380105566\\
2710	0.126397742304029\\
2711	0.37883406745764\\
2712	0.605453476966919\\
2713	0.790812232580097\\
2714	0.92227844142201\\
2715	0.990892898618155\\
2716	0.991979642301423\\
2717	0.925464612685728\\
2718	0.795880699118039\\
2719	0.612058831160562\\
2720	0.386526165334446\\
2721	0.134652380105576\\
2722	-0.126397742304018\\
2723	-0.37883406745763\\
2724	-0.60545347696691\\
2725	-0.79081223258009\\
2726	-0.922278441422006\\
2727	-0.990892898618154\\
2728	-0.991979642301425\\
2729	-0.925464612685689\\
2730	-0.795880699118046\\
2731	-0.612058831160571\\
2732	-0.386526165334455\\
2733	-0.134652380105587\\
2734	0.126397742304007\\
2735	0.37883406745762\\
2736	0.605453476966902\\
2737	0.790812232580083\\
2738	0.922278441422002\\
2739	0.990892898618168\\
2740	0.991979642301426\\
2741	0.925464612685693\\
2742	0.795880699118052\\
2743	0.612058831160579\\
2744	0.386526165334465\\
2745	0.134652380105598\\
2746	-0.126397742303997\\
2747	-0.37883406745761\\
2748	-0.605453476966893\\
2749	-0.790812232580146\\
2750	-0.922278441421998\\
2751	-0.990892898618166\\
2752	-0.991979642301413\\
2753	-0.925464612685697\\
2754	-0.795880699118059\\
2755	-0.612058831160588\\
2756	-0.386526165334475\\
2757	-0.134652380105608\\
2758	0.126397742303986\\
2759	0.3788340674576\\
2760	0.605453476966885\\
2761	0.79081223258014\\
2762	0.922278441422037\\
2763	0.990892898618165\\
2764	0.991979642301414\\
2765	0.925464612685701\\
2766	0.795880699118065\\
2767	0.612058831160596\\
2768	0.38652616533459\\
2769	0.134652380105619\\
2770	-0.126397742303975\\
2771	-0.37883406745759\\
2772	-0.605453476966966\\
2773	-0.790812232580133\\
2774	-0.922278441421989\\
2775	-0.990892898618163\\
2776	-0.991979642301416\\
2777	-0.925464612685705\\
2778	-0.795880699118072\\
2779	-0.612058831160605\\
2780	-0.386526165334495\\
2781	-0.13465238010563\\
2782	0.126397742303965\\
2783	0.378834067457685\\
2784	0.605453476966958\\
2785	0.790812232580127\\
2786	0.922278441421985\\
2787	0.990892898618162\\
2788	0.991979642301417\\
2789	0.925464612685709\\
2790	0.795880699118078\\
2791	0.612058831160613\\
2792	0.386526165334505\\
2793	0.134652380105528\\
2794	-0.126397742304067\\
2795	-0.378834067457675\\
2796	-0.605453476966949\\
2797	-0.79081223258012\\
2798	-0.922278441421981\\
2799	-0.99089289861816\\
2800	-0.991979642301419\\
2801	-0.925464612685713\\
2802	-0.795880699118085\\
2803	-0.612058831160532\\
2804	-0.386526165334515\\
2805	-0.134652380105538\\
2806	0.126397742304056\\
2807	0.378834067457665\\
2808	0.605453476966941\\
2809	0.790812232580113\\
2810	0.922278441421977\\
2811	0.990892898618159\\
2812	0.99197964230142\\
2813	0.925464612685717\\
2814	0.795880699118023\\
2815	0.61205883116054\\
2816	0.386526165334525\\
2817	0.134652380105549\\
2818	-0.126397742304045\\
2819	-0.378834067457655\\
2820	-0.605453476966932\\
2821	-0.790812232580107\\
2822	-0.922278441421973\\
2823	-0.990892898618157\\
2824	-0.991979642301421\\
2825	-0.925464612685722\\
2826	-0.795880699118029\\
2827	-0.612058831160549\\
2828	-0.386526165334535\\
2829	-0.13465238010556\\
2830	0.126397742303922\\
2831	0.378834067457645\\
2832	0.605453476966924\\
2833	0.7908122325801\\
2834	0.922278441422012\\
2835	0.990892898618156\\
2836	0.991979642301423\\
2837	0.925464612685683\\
2838	0.795880699118036\\
2839	0.612058831160557\\
2840	0.386526165334545\\
2841	0.13465238010557\\
2842	-0.126397742303911\\
2843	-0.378834067457635\\
2844	-0.605453476966915\\
2845	-0.790812232580094\\
2846	-0.922278441422008\\
2847	-0.99089289861817\\
2848	-0.991979642301424\\
2849	-0.925464612685687\\
2850	-0.795880699118042\\
2851	-0.612058831160566\\
2852	-0.386526165334555\\
2853	-0.134652380105581\\
2854	0.1263977423039\\
2855	0.378834067457625\\
2856	0.605453476966906\\
2857	0.790812232580157\\
2858	0.922278441422004\\
2859	0.990892898618168\\
2860	0.991979642301425\\
2861	0.925464612685691\\
2862	0.795880699118049\\
2863	0.612058831160574\\
2864	0.386526165334565\\
2865	0.134652380105592\\
2866	-0.12639774230389\\
2867	-0.378834067457615\\
2868	-0.605453476966988\\
2869	-0.79081223258015\\
2870	-0.922278441422\\
2871	-0.990892898618167\\
2872	-0.991979642301427\\
2873	-0.925464612685695\\
2874	-0.795880699118055\\
2875	-0.612058831160583\\
2876	-0.386526165334575\\
2877	-0.134652380105602\\
2878	0.126397742303992\\
2879	0.378834067457605\\
2880	0.60545347696698\\
2881	0.790812232580143\\
2882	0.922278441421996\\
2883	0.990892898618166\\
2884	0.991979642301428\\
2885	0.925464612685699\\
2886	0.795880699118062\\
2887	0.612058831160591\\
2888	0.38652616533448\\
2889	0.134652380105613\\
2890	-0.126397742303981\\
2891	-0.3788340674577\\
2892	-0.605453476966971\\
2893	-0.790812232580137\\
2894	-0.922278441421992\\
2895	-0.990892898618149\\
2896	-0.991979642301429\\
2897	-0.925464612685703\\
2898	-0.795880699118068\\
2899	-0.61205883116051\\
2900	-0.38652616533449\\
2901	-0.134652380105624\\
2902	0.12639774230397\\
2903	0.37883406745769\\
2904	0.605453476966963\\
2905	0.79081223258013\\
2906	0.922278441421987\\
2907	0.990892898618147\\
2908	0.991979642301431\\
2909	0.925464612685707\\
2910	0.795880699118075\\
2911	0.612058831160519\\
2912	0.3865261653345\\
2913	0.134652380105635\\
2914	-0.12639774230396\\
2915	-0.37883406745768\\
2916	-0.605453476966954\\
2917	-0.790812232580124\\
2918	-0.922278441421983\\
2919	-0.990892898618161\\
2920	-0.991979642301432\\
2921	-0.925464612685711\\
2922	-0.795880699118013\\
2923	-0.612058831160527\\
2924	-0.38652616533451\\
2925	-0.134652380105645\\
2926	0.126397742303949\\
2927	0.37883406745767\\
2928	0.605453476966945\\
2929	0.790812232580117\\
2930	0.922278441421979\\
2931	0.99089289861816\\
2932	0.991979642301419\\
2933	0.925464612685672\\
2934	0.795880699118019\\
2935	0.612058831160536\\
2936	0.38652616533452\\
2937	0.134652380105656\\
2938	-0.126397742303938\\
2939	-0.37883406745766\\
2940	-0.605453476966937\\
2941	-0.79081223258011\\
2942	-0.922278441422019\\
2943	-0.990892898618158\\
2944	-0.991979642301421\\
2945	-0.925464612685676\\
2946	-0.795880699118026\\
2947	-0.612058831160544\\
2948	-0.38652616533453\\
2949	-0.134652380105667\\
2950	0.126397742303928\\
2951	0.378834067457545\\
2952	0.605453476966928\\
2953	0.790812232580173\\
2954	0.922278441422015\\
2955	0.990892898618157\\
2956	0.991979642301422\\
2957	0.925464612685723\\
2958	0.795880699118032\\
2959	0.612058831160553\\
2960	0.38652616533454\\
2961	0.134652380105677\\
2962	-0.126397742303917\\
2963	-0.37883406745764\\
2964	-0.60545347696692\\
2965	-0.790812232580167\\
2966	-0.922278441422011\\
2967	-0.990892898618155\\
2968	-0.991979642301423\\
2969	-0.925464612685727\\
2970	-0.795880699118039\\
2971	-0.612058831160561\\
2972	-0.38652616533455\\
2973	-0.134652380105575\\
2974	0.126397742303906\\
2975	0.37883406745763\\
2976	0.605453476967002\\
2977	0.79081223258016\\
2978	0.922278441422006\\
2979	0.990892898618154\\
2980	0.991979642301425\\
2981	0.925464612685732\\
2982	0.795880699118045\\
2983	0.61205883116057\\
2984	0.386526165334559\\
2985	0.134652380105586\\
2986	-0.126397742304008\\
2987	-0.37883406745762\\
2988	-0.605453476966993\\
2989	-0.790812232580154\\
2990	-0.922278441422002\\
2991	-0.990892898618152\\
2992	-0.991979642301426\\
2993	-0.925464612685736\\
2994	-0.795880699118052\\
2995	-0.612058831160578\\
2996	-0.386526165334465\\
2997	-0.134652380105597\\
2998	0.126397742303998\\
2999	0.378834067457611\\
3000	0.605453476966984\\
3001	0.790812232580147\\
3002	0.922278441421998\\
3003	0.990892898618151\\
3004	0.991979642301427\\
3005	0.92546461268574\\
3006	0.795880699118058\\
3007	0.612058831160497\\
3008	0.386526165334474\\
3009	0.134652380105607\\
3010	-0.126397742303874\\
3011	-0.378834067457706\\
3012	-0.605453476966976\\
3013	-0.79081223258014\\
3014	-0.922278441421994\\
3015	-0.99089289861815\\
3016	-0.991979642301429\\
3017	-0.925464612685744\\
3018	-0.795880699117996\\
3019	-0.612058831160505\\
3020	-0.386526165334484\\
3021	-0.134652380105618\\
3022	0.126397742303864\\
3023	0.378834067457696\\
3024	0.605453476966967\\
3025	0.790812232580134\\
3026	0.92227844142199\\
3027	0.990892898618163\\
3028	0.99197964230143\\
3029	0.925464612685748\\
3030	0.795880699118003\\
3031	0.612058831160514\\
3032	0.386526165334494\\
3033	0.134652380105629\\
3034	-0.126397742303853\\
3035	-0.378834067457686\\
3036	-0.605453476966959\\
3037	-0.790812232580127\\
3038	-0.922278441422029\\
3039	-0.990892898618162\\
3040	-0.991979642301431\\
3041	-0.925464612685752\\
3042	-0.795880699118009\\
3043	-0.612058831160523\\
3044	-0.386526165334504\\
3045	-0.134652380105639\\
3046	0.126397742303842\\
3047	0.378834067457676\\
3048	0.60545347696695\\
3049	0.790812232580121\\
3050	0.922278441422025\\
3051	0.990892898618161\\
3052	0.991979642301433\\
3053	0.925464612685756\\
3054	0.795880699118016\\
3055	0.612058831160531\\
3056	0.386526165334514\\
3057	0.13465238010565\\
3058	-0.126397742303944\\
3059	-0.378834067457666\\
3060	-0.605453476966942\\
3061	-0.790812232580184\\
3062	-0.922278441422021\\
3063	-0.990892898618159\\
3064	-0.991979642301434\\
3065	-0.92546461268576\\
3066	-0.795880699118022\\
3067	-0.61205883116054\\
3068	-0.386526165334524\\
3069	-0.134652380105661\\
3070	0.126397742303934\\
3071	0.378834067457656\\
3072	0.605453476967023\\
3073	0.790812232580177\\
3074	0.922278441422017\\
3075	0.990892898618158\\
3076	0.991979642301436\\
3077	0.925464612685764\\
3078	0.795880699118097\\
3079	0.612058831160548\\
3080	0.386526165334534\\
3081	0.134652380105559\\
3082	-0.126397742303923\\
3083	-0.378834067457646\\
3084	-0.605453476967015\\
3085	-0.79081223258017\\
3086	-0.922278441422013\\
3087	-0.990892898618156\\
3088	-0.991979642301437\\
3089	-0.925464612685768\\
3090	-0.795880699118104\\
3091	-0.612058831160557\\
3092	-0.386526165334439\\
3093	-0.134652380105569\\
3094	0.126397742303912\\
3095	0.378834067457741\\
3096	0.605453476967006\\
3097	0.790812232580164\\
3098	0.922278441422009\\
3099	0.990892898618155\\
3100	0.991979642301438\\
3101	0.925464612685729\\
3102	0.795880699118111\\
3103	0.612058831160565\\
3104	0.386526165334449\\
3105	0.13465238010558\\
3106	-0.126397742303901\\
3107	-0.378834067457731\\
3108	-0.605453476966998\\
3109	-0.790812232580157\\
3110	-0.922278441422004\\
3111	-0.990892898618153\\
3112	-0.991979642301425\\
3113	-0.925464612685733\\
3114	-0.795880699118117\\
3115	-0.612058831160484\\
3116	-0.386526165334459\\
3117	-0.134652380105591\\
3118	0.126397742303891\\
3119	0.378834067457721\\
3120	0.605453476966989\\
3121	0.790812232580151\\
3122	0.922278441422\\
3123	0.990892898618167\\
3124	0.991979642301427\\
3125	0.925464612685738\\
3126	0.795880699118124\\
3127	0.612058831160492\\
3128	0.386526165334469\\
3129	0.134652380105602\\
3130	-0.12639774230388\\
3131	-0.378834067457501\\
3132	-0.60545347696698\\
3133	-0.790812232580144\\
3134	-0.922278441421996\\
3135	-0.990892898618166\\
3136	-0.991979642301428\\
3137	-0.925464612685742\\
3138	-0.79588069911813\\
3139	-0.612058831160501\\
3140	-0.386526165334479\\
3141	-0.134652380105612\\
3142	0.126397742303869\\
3143	0.378834067457596\\
3144	0.605453476966972\\
3145	0.790812232580137\\
3146	0.922278441422036\\
3147	0.990892898618164\\
3148	0.991979642301429\\
3149	0.925464612685746\\
3150	0.795880699118137\\
3151	0.612058831160509\\
3152	0.386526165334489\\
3153	0.134652380105623\\
3154	-0.126397742303859\\
3155	-0.378834067457586\\
3156	-0.605453476966963\\
3157	-0.7908122325802\\
3158	-0.922278441422032\\
3159	-0.990892898618163\\
3160	-0.991979642301431\\
3161	-0.92546461268575\\
3162	-0.795880699118143\\
3163	-0.612058831160518\\
3164	-0.386526165334499\\
3165	-0.134652380105634\\
3166	0.126397742303961\\
3167	0.378834067457576\\
3168	0.605453476966955\\
3169	0.790812232580194\\
3170	0.922278441422028\\
3171	0.990892898618161\\
3172	0.991979642301432\\
3173	0.925464612685754\\
3174	0.79588069911815\\
3175	0.612058831160526\\
3176	0.386526165334509\\
3177	0.134652380105532\\
3178	-0.12639774230395\\
3179	-0.378834067457566\\
3180	-0.605453476967037\\
3181	-0.790812232580187\\
3182	-0.922278441422023\\
3183	-0.99089289861816\\
3184	-0.991979642301433\\
3185	-0.925464612685758\\
3186	-0.795880699118087\\
3187	-0.612058831160625\\
3188	-0.386526165334519\\
3189	-0.134652380105542\\
3190	0.126397742303939\\
3191	0.378834067457556\\
3192	0.605453476967028\\
3193	0.790812232580181\\
3194	0.922278441422019\\
3195	0.990892898618158\\
3196	0.991979642301435\\
3197	0.925464612685719\\
3198	0.795880699118094\\
3199	0.612058831160633\\
3200	0.386526165334424\\
3201	0.134652380105553\\
3202	-0.126397742303929\\
3203	-0.378834067457546\\
3204	-0.60545347696702\\
3205	-0.790812232580174\\
3206	-0.922278441422015\\
3207	-0.990892898618157\\
3208	-0.991979642301436\\
3209	-0.925464612685723\\
3210	-0.7958806991181\\
3211	-0.612058831160642\\
3212	-0.386526165334434\\
3213	-0.134652380105564\\
3214	0.126397742303918\\
3215	0.378834067457536\\
3216	0.605453476967011\\
3217	0.790812232580167\\
3218	0.922278441422011\\
3219	0.990892898618156\\
3220	0.991979642301423\\
3221	0.925464612685727\\
3222	0.795880699118107\\
3223	0.61205883116065\\
3224	0.386526165334444\\
3225	0.134652380105574\\
3226	-0.126397742303907\\
3227	-0.378834067457526\\
3228	-0.605453476967002\\
3229	-0.790812232580161\\
3230	-0.922278441422007\\
3231	-0.990892898618169\\
3232	-0.991979642301425\\
3233	-0.925464612685731\\
3234	-0.795880699118113\\
3235	-0.612058831160659\\
3236	-0.386526165334454\\
3237	-0.134652380105585\\
3238	0.126397742303897\\
3239	0.378834067457516\\
3240	0.605453476966994\\
3241	0.790812232580154\\
3242	0.922278441422047\\
3243	0.990892898618168\\
3244	0.991979642301426\\
3245	0.925464612685735\\
3246	0.79588069911812\\
3247	0.612058831160667\\
3248	0.386526165334464\\
3249	0.134652380105596\\
3250	-0.126397742303886\\
3251	-0.378834067457611\\
3252	-0.605453476966895\\
3253	-0.790812232580148\\
3254	-0.922278441422042\\
3255	-0.990892898618166\\
3256	-0.991979642301427\\
3257	-0.925464612685739\\
3258	-0.795880699118127\\
3259	-0.612058831160676\\
3260	-0.386526165334474\\
3261	-0.134652380105606\\
3262	0.126397742303988\\
3263	0.378834067457601\\
3264	0.605453476966886\\
3265	0.790812232580211\\
3266	0.922278441422038\\
3267	0.990892898618165\\
3268	0.991979642301429\\
3269	0.925464612685743\\
3270	0.795880699118133\\
3271	0.612058831160595\\
3272	0.386526165334484\\
3273	0.134652380105617\\
3274	-0.126397742303977\\
3275	-0.378834067457591\\
3276	-0.605453476966878\\
3277	-0.790812232580204\\
3278	-0.922278441422034\\
3279	-0.990892898618164\\
3280	-0.99197964230143\\
3281	-0.925464612685748\\
3282	-0.795880699118071\\
3283	-0.612058831160603\\
3284	-0.386526165334493\\
3285	-0.134652380105515\\
3286	0.126397742303967\\
3287	0.378834067457582\\
3288	0.605453476966869\\
3289	0.790812232580197\\
3290	0.92227844142203\\
3291	0.990892898618162\\
3292	0.991979642301431\\
3293	0.925464612685752\\
3294	0.795880699118077\\
3295	0.612058831160612\\
3296	0.386526165334399\\
3297	0.134652380105526\\
3298	-0.126397742303956\\
3299	-0.378834067457572\\
3300	-0.60545347696686\\
3301	-0.790812232580191\\
3302	-0.922278441422026\\
3303	-0.990892898618161\\
3304	-0.991979642301433\\
3305	-0.925464612685713\\
3306	-0.795880699118084\\
3307	-0.61205883116062\\
3308	-0.386526165334409\\
3309	-0.134652380105536\\
3310	0.126397742303945\\
3311	0.378834067457562\\
3312	0.605453476966852\\
3313	0.790812232580184\\
3314	0.922278441422022\\
3315	0.990892898618159\\
3316	0.99197964230142\\
3317	0.925464612685717\\
3318	0.79588069911809\\
3319	0.612058831160629\\
3320	0.386526165334418\\
3321	0.134652380105547\\
3322	-0.126397742303934\\
3323	-0.378834067457552\\
3324	-0.605453476966843\\
3325	-0.790812232580108\\
3326	-0.922278441422017\\
3327	-0.990892898618158\\
3328	-0.991979642301421\\
3329	-0.925464612685721\\
3330	-0.795880699118097\\
3331	-0.612058831160637\\
3332	-0.386526165334428\\
3333	-0.134652380105558\\
3334	0.126397742303924\\
3335	0.378834067457542\\
3336	0.605453476966925\\
3337	0.790812232580101\\
3338	0.922278441422013\\
3339	0.990892898618172\\
3340	0.991979642301422\\
3341	0.925464612685725\\
3342	0.795880699118103\\
3343	0.612058831160646\\
3344	0.386526165334438\\
3345	0.134652380105568\\
3346	-0.126397742303913\\
3347	-0.378834067457637\\
3348	-0.605453476966917\\
3349	-0.790812232580095\\
3350	-0.922278441422053\\
3351	-0.99089289861817\\
3352	-0.991979642301424\\
3353	-0.925464612685729\\
3354	-0.79588069911811\\
3355	-0.612058831160654\\
3356	-0.386526165334553\\
3357	-0.134652380105579\\
3358	0.126397742303902\\
3359	0.378834067457627\\
3360	0.605453476966908\\
3361	-0.378737871562225\\
3362	-0.922238260469039\\
3363	-0.925503984132105\\
3364	-0.386622025920985\\
3365	0.37873787156253\\
3366	0.922238260468815\\
3367	0.925503984132152\\
3368	0.3866220259211\\
3369	-0.378737871562415\\
3370	-0.922238260468943\\
3371	-0.925503984132027\\
3372	-0.386622025921215\\
3373	0.3787378715623\\
3374	0.922238260469071\\
3375	0.925503984132074\\
3376	0.38662202592091\\
3377	-0.378737871562606\\
3378	-0.922238260468847\\
3379	-0.925503984132122\\
3380	-0.386622025921445\\
3381	0.378737871562491\\
3382	0.922238260468974\\
3383	0.925503984132169\\
3384	0.38662202592114\\
3385	-0.378737871562375\\
3386	-0.922238260468926\\
3387	-0.925503984132044\\
3388	-0.386622025920835\\
3389	0.37873787156226\\
3390	0.922238260468878\\
3391	0.925503984132091\\
3392	0.38662202592137\\
3393	-0.378737871562566\\
3394	-0.922238260469006\\
3395	-0.925503984132138\\
3396	-0.386622025921065\\
3397	0.378737871562451\\
3398	0.922238260468958\\
3399	0.925503984132013\\
3400	0.38662202592076\\
3401	-0.378737871562335\\
3402	-0.92223826046891\\
3403	-0.925503984132232\\
3404	-0.386622025921295\\
3405	0.378737871562641\\
3406	0.922238260468861\\
3407	0.925503984132107\\
3408	0.38662202592099\\
3409	-0.378737871562105\\
3410	-0.922238260468989\\
3411	-0.925503984131982\\
3412	-0.386622025921105\\
3413	0.378737871562411\\
3414	0.922238260468941\\
3415	0.925503984132201\\
3416	0.38662202592122\\
3417	-0.378737871562716\\
3418	-0.922238260468893\\
3419	-0.925503984132076\\
3420	-0.386622025920915\\
3421	0.37873787156218\\
3422	0.922238260469021\\
3423	0.925503984131951\\
3424	0.38662202592103\\
3425	-0.378737871562486\\
3426	-0.922238260468797\\
3427	-0.92550398413217\\
3428	-0.386622025921144\\
3429	0.378737871562792\\
3430	0.922238260468924\\
3431	0.925503984132045\\
3432	0.386622025921259\\
3433	-0.378737871562256\\
3434	-0.922238260469052\\
3435	-0.925503984132093\\
3436	-0.386622025920955\\
3437	0.378737871562561\\
3438	0.922238260468828\\
3439	0.92550398413214\\
3440	0.386622025921069\\
3441	-0.378737871562446\\
3442	-0.922238260468956\\
3443	-0.925503984132015\\
3444	-0.386622025921184\\
3445	0.378737871562331\\
3446	0.922238260468732\\
3447	0.925503984132062\\
3448	0.38662202592088\\
3449	-0.378737871562637\\
3450	-0.92223826046886\\
3451	-0.925503984132109\\
3452	-0.386622025921414\\
3453	0.378737871562521\\
3454	0.922238260468987\\
3455	0.925503984132156\\
3456	0.386622025921109\\
3457	-0.378737871562406\\
3458	-0.922238260468763\\
3459	-0.925503984132031\\
3460	-0.386622025920805\\
3461	0.378737871562291\\
3462	0.922238260468891\\
3463	0.925503984132078\\
3464	0.386622025921339\\
3465	-0.378737871562597\\
3466	-0.922238260469019\\
3467	-0.925503984132125\\
3468	-0.386622025921034\\
3469	0.378737871562481\\
3470	0.922238260468795\\
3471	0.925503984132\\
3472	0.38662202592073\\
3473	-0.378737871562366\\
3474	-0.922238260468923\\
3475	-0.925503984132219\\
3476	-0.386622025921264\\
3477	0.378737871562251\\
3478	0.92223826046905\\
3479	0.925503984132094\\
3480	0.386622025920959\\
3481	-0.378737871562136\\
3482	-0.922238260468826\\
3483	-0.925503984131969\\
3484	-0.386622025921074\\
3485	0.378737871562442\\
3486	0.922238260468954\\
3487	0.925503984132189\\
3488	0.386622025921189\\
3489	-0.378737871562326\\
3490	-0.922238260468906\\
3491	-0.925503984132064\\
3492	-0.386622025920884\\
3493	0.378737871562211\\
3494	0.922238260468858\\
3495	0.925503984131938\\
3496	0.386622025920999\\
3497	-0.378737871562517\\
3498	-0.922238260468985\\
3499	-0.925503984132158\\
3500	-0.386622025921114\\
3501	0.378737871561981\\
3502	0.922238260468937\\
3503	0.925503984132033\\
3504	0.386622025921229\\
3505	-0.378737871562286\\
3506	-0.922238260468889\\
3507	-0.92550398413208\\
3508	-0.386622025920924\\
3509	0.378737871562592\\
3510	0.922238260468841\\
3511	0.925503984132127\\
3512	0.386622025921039\\
3513	-0.378737871562056\\
3514	-0.922238260468969\\
3515	-0.925503984132002\\
3516	-0.386622025921154\\
3517	0.378737871562362\\
3518	0.922238260468921\\
3519	0.925503984132049\\
3520	0.386622025920849\\
3521	-0.9254646126857\\
3522	-0.795880699118064\\
3523	-0.612058831160594\\
3524	-0.386526165334588\\
3525	-0.134652380105729\\
3526	0.126397742303978\\
3527	0.378834067457592\\
3528	0.605453476966878\\
3529	0.790812232580135\\
3530	0.92227844142199\\
3531	0.990892898618164\\
3532	0.991979642301415\\
3533	0.925464612685704\\
3534	0.79588069911807\\
3535	0.612058831160602\\
3536	0.386526165334597\\
3537	0.134652380105739\\
3538	-0.126397742303968\\
3539	-0.378834067457582\\
3540	-0.60545347696696\\
3541	-0.790812232580128\\
3542	-0.922278441421986\\
3543	-0.990892898618178\\
3544	-0.991979642301417\\
3545	-0.925464612685708\\
3546	-0.795880699118077\\
3547	-0.612058831160611\\
3548	-0.386526165334607\\
3549	-0.134652380105637\\
3550	0.126397742303957\\
3551	0.378834067457572\\
3552	0.605453476966952\\
3553	0.790812232580122\\
3554	0.922278441421982\\
3555	0.990892898618176\\
3556	0.991979642301418\\
3557	0.925464612685712\\
3558	0.795880699118083\\
3559	0.612058831160619\\
3560	0.386526165334512\\
3561	0.134652380105648\\
3562	-0.126397742303946\\
3563	-0.378834067457668\\
3564	-0.605453476966943\\
3565	-0.790812232580115\\
3566	-0.922278441421978\\
3567	-0.990892898618144\\
3568	-0.99197964230142\\
3569	-0.925464612685716\\
3570	-0.79588069911809\\
3571	-0.612058831160538\\
3572	-0.386526165334522\\
3573	-0.134652380105659\\
3574	0.126397742304048\\
3575	0.378834067457658\\
3576	0.605453476966935\\
3577	0.790812232580109\\
3578	0.922278441421974\\
3579	0.990892898618143\\
3580	0.991979642301421\\
3581	0.92546461268572\\
3582	0.795880699118096\\
3583	0.612058831160546\\
3584	0.386526165334532\\
3585	0.134652380105669\\
3586	-0.126397742304038\\
3587	-0.378834067457648\\
3588	-0.605453476966926\\
3589	-0.790812232580102\\
3590	-0.92227844142197\\
3591	-0.990892898618156\\
3592	-0.991979642301422\\
3593	-0.925464612685725\\
3594	-0.795880699118034\\
3595	-0.612058831160555\\
3596	-0.386526165334542\\
3597	-0.13465238010568\\
3598	0.126397742303801\\
3599	0.378834067457638\\
3600	0.605453476966917\\
3601	0.790812232580095\\
3602	0.922278441421965\\
3603	0.990892898618155\\
3604	0.991979642301424\\
3605	0.925464612685685\\
3606	0.79588069911804\\
3607	0.612058831160564\\
3608	0.386526165334552\\
3609	0.134652380105691\\
3610	-0.126397742303791\\
3611	-0.378834067457628\\
3612	-0.605453476966909\\
3613	-0.790812232580089\\
3614	-0.922278441422005\\
3615	-0.990892898618154\\
3616	-0.991979642301425\\
3617	-0.92546461268569\\
3618	-0.795880699118047\\
3619	-0.612058831160572\\
3620	-0.386526165334562\\
3621	-0.134652380105702\\
3622	0.12639774230378\\
3623	0.378834067457618\\
3624	0.6054534769669\\
3625	0.790812232580152\\
3626	0.922278441422001\\
3627	0.990892898618152\\
3628	0.991979642301412\\
3629	0.925464612685694\\
3630	0.795880699118054\\
3631	0.612058831160581\\
3632	0.386526165334572\\
3633	0.134652380105712\\
3634	-0.126397742303882\\
3635	-0.378834067457608\\
3636	-0.605453476966892\\
3637	-0.790812232580145\\
3638	-0.922278441421997\\
3639	-0.990892898618151\\
3640	-0.991979642301413\\
3641	-0.925464612685698\\
3642	-0.79588069911806\\
3643	-0.612058831160589\\
3644	-0.386526165334582\\
3645	-0.13465238010561\\
3646	0.126397742303871\\
3647	0.378834067457598\\
3648	0.605453476966973\\
3649	0.790812232580139\\
3650	0.922278441421993\\
3651	0.990892898618149\\
3652	0.991979642301415\\
3653	0.925464612685702\\
3654	0.795880699118067\\
3655	0.612058831160598\\
3656	0.386526165334592\\
3657	0.134652380105621\\
3658	-0.126397742303861\\
3659	-0.378834067457693\\
3660	-0.605453476966965\\
3661	-0.790812232580132\\
3662	-0.922278441421989\\
3663	-0.990892898618148\\
3664	-0.991979642301445\\
3665	-0.925464612685706\\
3666	-0.795880699118073\\
3667	-0.612058831160606\\
3668	-0.386526165334497\\
3669	-0.134652380105632\\
3670	0.12639774230385\\
3671	0.378834067457683\\
3672	0.605453476966956\\
3673	0.790812232580125\\
3674	0.922278441421984\\
3675	0.990892898618146\\
3676	0.991979642301432\\
3677	0.92546461268571\\
3678	0.79588069911808\\
3679	0.612058831160525\\
3680	0.386526165334507\\
3681	0.134652380105642\\
3682	-0.126397742303839\\
3683	-0.378834067457673\\
3684	-0.605453476966948\\
3685	-0.790812232580119\\
3686	-0.92227844142198\\
3687	-0.990892898618145\\
3688	-0.991979642301433\\
3689	-0.925464612685714\\
3690	-0.795880699118086\\
3691	-0.612058831160533\\
3692	-0.386526165334517\\
3693	-0.134652380105653\\
3694	0.126397742303829\\
3695	0.378834067457663\\
3696	0.605453476966939\\
3697	0.790812232580112\\
3698	0.922278441421976\\
3699	0.990892898618159\\
3700	0.991979642301435\\
3701	0.925464612685718\\
3702	0.795880699118024\\
3703	0.612058831160542\\
3704	0.386526165334527\\
3705	0.134652380105664\\
3706	-0.126397742303818\\
3707	-0.378834067457653\\
3708	-0.605453476966931\\
3709	-0.790812232580106\\
3710	-0.922278441422016\\
3711	-0.990892898618157\\
3712	-0.991979642301436\\
3713	-0.925464612685679\\
3714	-0.79588069911803\\
3715	-0.61205883116055\\
3716	-0.386526165334537\\
3717	-0.134652380105674\\
3718	0.126397742303807\\
3719	0.378834067457643\\
3720	0.605453476966922\\
3721	0.790812232580099\\
3722	0.922278441422012\\
3723	0.990892898618156\\
3724	0.991979642301437\\
3725	0.925464612685683\\
3726	0.795880699118037\\
3727	0.612058831160559\\
3728	0.386526165334547\\
3729	0.134652380105685\\
3730	-0.126397742303909\\
3731	-0.378834067457633\\
3732	-0.605453476966913\\
3733	-0.790812232580162\\
3734	-0.922278441422007\\
3735	-0.990892898618154\\
3736	-0.991979642301439\\
3737	-0.925464612685687\\
3738	-0.795880699118043\\
3739	-0.612058831160567\\
3740	-0.386526165334557\\
3741	-0.134652380105696\\
3742	0.126397742303899\\
3743	0.378834067457518\\
3744	0.605453476966995\\
3745	0.790812232580155\\
3746	0.922278441422003\\
3747	0.990892898618153\\
3748	0.99197964230144\\
3749	0.925464612685691\\
3750	0.79588069911805\\
3751	0.612058831160576\\
3752	0.386526165334567\\
3753	0.134652380105594\\
3754	-0.126397742303888\\
3755	-0.378834067457508\\
3756	-0.605453476966987\\
3757	-0.790812232580149\\
3758	-0.922278441421999\\
3759	-0.990892898618151\\
3760	-0.991979642301441\\
3761	-0.925464612685696\\
3762	-0.795880699118056\\
3763	-0.612058831160584\\
3764	-0.386526165334472\\
3765	-0.134652380105604\\
3766	0.126397742303877\\
3767	0.378834067457498\\
3768	0.605453476966978\\
3769	0.790812232580142\\
3770	0.922278441421995\\
3771	0.99089289861815\\
3772	0.991979642301443\\
3773	0.9254646126857\\
3774	0.795880699118063\\
3775	0.612058831160593\\
3776	0.386526165334482\\
3777	0.134652380105615\\
3778	-0.126397742303866\\
3779	-0.378834067457488\\
3780	-0.60545347696697\\
3781	-0.790812232580136\\
3782	-0.922278441421991\\
3783	-0.990892898618149\\
3784	-0.99197964230143\\
3785	-0.925464612685747\\
3786	-0.79588069911807\\
3787	-0.612058831160512\\
3788	-0.386526165334492\\
3789	-0.134652380105626\\
3790	0.126397742303856\\
3791	0.378834067457478\\
3792	0.605453476966961\\
3793	0.790812232580129\\
3794	0.922278441421987\\
3795	0.990892898618147\\
3796	0.991979642301431\\
3797	0.925464612685751\\
3798	0.795880699118007\\
3799	0.61205883116052\\
3800	0.386526165334502\\
3801	0.134652380105636\\
3802	-0.126397742303845\\
3803	-0.378834067457468\\
3804	-0.605453476966952\\
3805	-0.790812232580122\\
3806	-0.922278441421982\\
3807	-0.990892898618161\\
3808	-0.991979642301432\\
3809	-0.925464612685755\\
3810	-0.795880699118014\\
3811	-0.612058831160529\\
3812	-0.386526165334512\\
3813	-0.134652380105647\\
3814	0.126397742303834\\
3815	0.378834067457563\\
3816	0.605453476966944\\
3817	0.790812232580116\\
3818	0.922278441422022\\
3819	0.990892898618159\\
3820	0.991979642301434\\
3821	0.925464612685759\\
3822	0.79588069911802\\
3823	0.612058831160537\\
3824	0.386526165334521\\
3825	0.134652380105658\\
3826	-0.126397742303824\\
3827	-0.378834067457553\\
3828	-0.605453476966935\\
3829	-0.790812232580179\\
3830	-0.922278441422018\\
3831	-0.990892898618158\\
3832	-0.991979642301435\\
3833	-0.925464612685763\\
3834	-0.795880699118027\\
3835	-0.612058831160546\\
3836	-0.386526165334531\\
3837	-0.134652380105669\\
3838	0.126397742303926\\
3839	0.378834067457543\\
3840	0.605453476966927\\
3841	-0.378737871562711\\
3842	-0.922238260468891\\
3843	-0.925503984132078\\
3844	-0.38662202592134\\
3845	0.378737871562175\\
3846	0.922238260469018\\
3847	0.925503984132126\\
3848	0.386622025921035\\
3849	-0.378737871562481\\
3850	-0.922238260468794\\
3851	-0.925503984132173\\
3852	-0.386622025920731\\
3853	0.378737871562365\\
3854	0.922238260468922\\
3855	0.925503984132048\\
3856	0.386622025921265\\
3857	-0.37873787156225\\
3858	-0.92223826046905\\
3859	-0.925503984132095\\
3860	-0.38662202592096\\
3861	0.378737871562556\\
3862	0.922238260468826\\
3863	0.925503984132142\\
3864	0.386622025920656\\
3865	-0.378737871562441\\
3866	-0.922238260468954\\
3867	-0.925503984132189\\
3868	-0.38662202592119\\
3869	0.378737871562325\\
3870	0.922238260469081\\
3871	0.925503984132064\\
3872	0.386622025920885\\
3873	-0.37873787156221\\
3874	-0.922238260468857\\
3875	-0.925503984132111\\
3876	-0.386622025921419\\
3877	0.378737871562516\\
3878	0.922238260468985\\
3879	0.925503984132158\\
3880	0.386622025921115\\
3881	-0.378737871562401\\
3882	-0.922238260468937\\
3883	-0.925503984132033\\
3884	-0.38662202592081\\
3885	0.378737871562286\\
3886	0.922238260468889\\
3887	0.925503984132253\\
3888	0.386622025921344\\
3889	-0.378737871562591\\
3890	-0.922238260469016\\
3891	-0.925503984132127\\
3892	-0.38662202592104\\
3893	0.378737871562055\\
3894	0.922238260468793\\
3895	0.925503984132002\\
3896	0.386622025921154\\
3897	-0.378737871562361\\
3898	-0.92223826046892\\
3899	-0.925503984132222\\
3900	-0.386622025921269\\
3901	0.378737871562666\\
3902	0.922238260468872\\
3903	0.925503984132097\\
3904	0.386622025920965\\
3905	-0.37873787156213\\
3906	-0.922238260468824\\
3907	-0.925503984131972\\
3908	-0.386622025921079\\
3909	0.378737871562436\\
3910	0.922238260468952\\
3911	0.925503984132191\\
3912	0.386622025921194\\
3913	-0.378737871562742\\
3914	-0.922238260468904\\
3915	-0.925503984132066\\
3916	-0.386622025921309\\
3917	0.378737871562206\\
3918	0.922238260468855\\
3919	0.925503984132113\\
3920	0.386622025921004\\
3921	-0.378737871562511\\
3922	-0.922238260468807\\
3923	-0.92550398413216\\
3924	-0.386622025921119\\
3925	0.378737871562396\\
3926	0.922238260468935\\
3927	0.925503984132035\\
3928	0.386622025921234\\
3929	-0.378737871562281\\
3930	-0.922238260468887\\
3931	-0.925503984132082\\
3932	-0.386622025920929\\
3933	0.378737871562587\\
3934	0.922238260468839\\
3935	0.925503984132129\\
3936	0.386622025921464\\
3937	-0.378737871562471\\
3938	-0.922238260468967\\
3939	-0.925503984132004\\
3940	-0.386622025921159\\
3941	0.378737871562356\\
3942	0.922238260468743\\
3943	0.925503984132051\\
3944	0.386622025920854\\
3945	-0.378737871562241\\
3946	-0.92223826046887\\
3947	-0.925503984132098\\
3948	-0.386622025921389\\
3949	0.378737871562547\\
3950	0.922238260468998\\
3951	0.925503984132146\\
3952	0.386622025921084\\
3953	-0.378737871562432\\
3954	-0.922238260468774\\
3955	-0.925503984132021\\
3956	-0.386622025920779\\
3957	0.378737871562316\\
3958	0.922238260468902\\
3959	0.925503984132068\\
3960	0.386622025921314\\
3961	-0.378737871562622\\
3962	-0.922238260469029\\
3963	-0.925503984132115\\
3964	-0.386622025921009\\
3965	0.378737871562086\\
3966	0.922238260468805\\
3967	0.92550398413199\\
3968	0.386622025921124\\
3969	-0.378737871562392\\
3970	-0.922238260468933\\
3971	-0.925503984132209\\
3972	-0.386622025921239\\
3973	0.378737871562697\\
3974	0.922238260468885\\
3975	0.925503984132084\\
3976	0.386622025920934\\
3977	-0.378737871562161\\
3978	-0.922238260468837\\
3979	-0.925503984131959\\
3980	-0.386622025921049\\
3981	0.378737871562467\\
3982	0.922238260468965\\
3983	0.925503984132178\\
3984	0.386622025921163\\
3985	-0.378737871562773\\
3986	-0.922238260468916\\
3987	-0.925503984132053\\
3988	-0.386622025921278\\
3989	0.378737871562237\\
3990	0.922238260468868\\
3991	0.925503984131928\\
3992	0.386622025920974\\
3993	-0.378737871562542\\
3994	-0.92223826046882\\
3995	-0.925503984132148\\
3996	-0.386622025921088\\
3997	0.378737871562427\\
3998	0.922238260468948\\
3999	0.925503984132022\\
4000	0.386622025921203\\
};
\end{axis}
\end{tikzpicture}%

\begin{align*}
	s_x(f) &= | \operatorname{TF}(x)(f) |^2 \\
	       &= \left| \operatorname{TF}(\operatorname{NRZ}(t) \cos(2\pi F_1 t + \phi_1)) + \operatorname{TF}((1 - \operatorname{NRZ}(t)) \cos(2 \pi F_0 t + \phi_0)) \right|^2 \\
	       &= | \TF \NRZ (f) \ast \TF (\cos (2\pi F_1 t) \cos \phi_1 - \sin(2 \pi F_1 t) \sin \phi_1) + \TF (1-\NRZ)(f) \ast \TF(\cos (2 \pi F_0 t) \cos \phi_0 - \sin (2 \pi F_0 t) \sin \phi_0) |^2 \\
	       &= | \TF \NRZ (f) \ast \frac{1}{2} \left( \cos \phi_1 (\delta(f-F_1) + \delta(f+F_1)) - \frac{1}{i} \sin \phi_1 (\delta(f-F_1) - \delta(f+F_1)) \right) + \TF(1-\NRZ)(f) \ast \frac{1}{2} \left( \cos \phi_0 (\delta(f-F_0) + \delta(f+F_0)) - \frac{1}{i} \sin \phi_0 (\delta(f-F_0) - \delta(f+F_0)) \right) |^2  \\
	       &= | \frac{1}{4} \left( \delta(f) + \sinc^2(\pi f T_s) \right)  \ast \frac{1}{2} \left( \cos \phi_1 (\delta(f-F_1) + \delta(f+F_1)) - \frac{1}{i} \sin \phi_1 (\delta(f-F_1) - \delta(f+F_1)) \right) + \left( \delta(f) - \frac{1}{4} \delta(f) - \sinc^2(\pi f T_s) \right)  \ast \frac{1}{2} \left( \cos \phi_0 (\delta(f-F_0) + \delta(f+F_0)) - \frac{1}{i} \sin \phi_0 (\delta(f-F_0) - \delta(f+F_0)) \right) |^2  \\
\end{align*}

On calcule ensuite la densité spectrale de puissance de ce signal NRZ en utilisant la fonction \verb|pwelch| de Matlab utilisant un périodogramme de Welch. 
On obtient ceci:

% This file was created by matlab2tikz.
%
%The latest updates can be retrieved from
%  http://www.mathworks.com/matlabcentral/fileexchange/22022-matlab2tikz-matlab2tikz
%where you can also make suggestions and rate matlab2tikz.
%
\begin{tikzpicture}

\begin{axis}[%
width=4.521in,
height=3.548in,
at={(0.758in,0.499in)},
scale only axis,
xmin=-25000,
xmax=25000,
xlabel style={font=\color{white!15!black}},
xlabel={fréquence [Hz]},
ymin=0,
ymax=1,
ylabel style={font=\color{white!15!black}},
ylabel={Densité spectrale de puissance},
axis background/.style={fill=white},
title style={font=\bfseries},
title={Densité spectrale de puissance du signal modulé},
axis x line*=bottom,
axis y line*=left
]
\addplot [color=red, forget plot]
  table[row sep=crcr]{%
-24000	1\\
24000	1\\
};
\addplot [color=blue, forget plot]
  table[row sep=crcr]{%
-24000	8.14827217254788e-07\\
-22076.2463343109	0.000176826575625455\\
-21982.4046920821	0.00124771342962049\\
-21935.4838709677	0.000277809660474304\\
-21700.8797653959	9.12488758331165e-07\\
-18041.0557184751	0.000259531043411698\\
-17994.1348973607	0.000684993883623974\\
-17900.293255132	0.000142605411383556\\
-17008.7976539589	1.06670631794259e-06\\
11612.9032258065	3.49209585692734e-08\\
17994.1348973607	0.000257560655882116\\
18041.0557184751	0.000684993883623974\\
18134.8973607038	0.000146692083944799\\
19073.3137829912	2.38798020291142e-06\\
21982.4046920821	0.000277809660474304\\
22029.3255131965	0.00124771342962049\\
22217.008797654	7.91521597420797e-05\\
22639.2961876833	7.95349478721619e-07\\
24000	4.76087734568864e-07\\
};
\end{axis}
\end{tikzpicture}%


\section{Canal de transmission à bruit additif, blanc et Gaussien}

Dans cette section, nous allons tenter de simuler un bruit blanc Gaussion que nous additionnerons à notre signal modulé en fréquence afin de modéliser le signal reçu par le modem.
Le bruit simulé sera généré aléatoirement grâce au module \verb|rand| de Matlab et sera de puissance $\sigma^2$ avec:

\[
\sigma=\sqrt{\frac{S_\text{module}}{10^{\operatorname{SNR}/10}}} \huge
\]

avec $S_\text{module}$ représentant la densité spectrale de puissance du signal modulé en fréquence et $\operatorname{SNR}$ le rapport signal sur bruit (signal to noise ratio) que nous fixerons à 10 par la suite.

\begin{figure}[H]
	\centering
	% This file was created by matlab2tikz.
%
%The latest updates can be retrieved from
%  http://www.mathworks.com/matlabcentral/fileexchange/22022-matlab2tikz-matlab2tikz
%where you can also make suggestions and rate matlab2tikz.
%
\definecolor{mycolor1}{rgb}{0.00000,0.44700,0.74100}%
%
\begin{tikzpicture}

\begin{axis}[%
width=4.521in,
height=3.559in,
at={(0.758in,0.488in)},
scale only axis,
xmin=0,
xmax=4000,
xlabel style={font=\color{white!15!black}},
xlabel={Temps [s]},
ymin=-2,
ymax=2,
ylabel style={font=\color{white!15!black}},
ylabel={Amplitude},
axis background/.style={fill=white},
title style={font=\bfseries},
title={Signal bruité}
]
\addplot [color=mycolor1, forget plot]
  table[row sep=crcr]{%
1	0.327650206150346\\
2	0.994898273921492\\
3	0.742265282484324\\
4	0.568542610052729\\
5	-0.652244301137382\\
6	-1.16832687295688\\
7	-1.09959165501646\\
8	-1.0286075542394\\
9	0.717433440733757\\
10	1.00190682864863\\
11	0.749505540858098\\
12	0.676382423042075\\
13	-0.77855560505986\\
14	-0.952015698548656\\
15	-0.972473551041009\\
16	-0.298306403562325\\
17	0.465565176128426\\
18	0.735593898631065\\
19	0.911717889539272\\
20	0.332841726282392\\
21	-0.255086161999748\\
22	-0.684485934084136\\
23	-0.670210530290491\\
24	-0.563005685638814\\
25	0.412865301955499\\
26	0.657441918768412\\
27	0.669264516583724\\
28	0.368205472881039\\
29	-0.0525832767925879\\
30	-1.10137114067771\\
31	-0.835335915250436\\
32	-0.420219290565774\\
33	0.645594935230822\\
34	0.685426113894724\\
35	0.925728423387559\\
36	0.493382224974462\\
37	-0.149260356243119\\
38	-0.583573580394599\\
39	-0.899213148303127\\
40	-0.703524989783497\\
41	0.229442293496784\\
42	0.691576137616095\\
43	1.44442601226737\\
44	0.231979356840414\\
45	-0.228149981012709\\
46	-0.972195219700834\\
47	-0.71959027558638\\
48	-0.540895505371972\\
49	0.0817554236498745\\
50	0.61083792770883\\
51	1.02769035025033\\
52	0.330045351539592\\
53	-0.439426630636155\\
54	-0.611523984044084\\
55	-0.853192245242021\\
56	-0.325472224347438\\
57	0.750847871074014\\
58	0.749113282184807\\
59	1.07433274729404\\
60	0.556613511599946\\
61	-0.450092313725125\\
62	-0.880873508996663\\
63	-1.17933428171372\\
64	-0.626626082311257\\
65	0.419022724520626\\
66	1.09076135113491\\
67	1.49702172242676\\
68	0.220501996268806\\
69	-0.353633178462622\\
70	-0.947596503677823\\
71	-1.35101292690967\\
72	-0.467969673052807\\
73	-0.0060577156610786\\
74	1.11719447694334\\
75	0.719719696699584\\
76	0.392136896554895\\
77	-0.517408161492757\\
78	-0.861214354041843\\
79	-1.05278329748512\\
80	-0.260094196846996\\
81	0.561007987598265\\
82	1.31222072970078\\
83	0.875001946711404\\
84	-0.10878106968281\\
85	-0.583436390359521\\
86	-0.626006034452946\\
87	-1.15836871878375\\
88	-0.15469678977851\\
89	0.423313708109638\\
90	1.25063869160639\\
91	0.479634248918094\\
92	0.325497472784912\\
93	-0.665844562343159\\
94	-0.278384469406136\\
95	-0.733775977209273\\
96	-0.0611530617122698\\
97	0.158755295702694\\
98	0.824269586561937\\
99	0.857469775397058\\
100	0.615542713589853\\
101	-0.457735890972832\\
102	-0.772145636720111\\
103	-1.37759640645316\\
104	-0.448922453781266\\
105	0.211252476855186\\
106	0.576223526423451\\
107	1.03211686724844\\
108	0.432840357657789\\
109	-0.388061853057634\\
110	-1.22758526130556\\
111	-0.666133502655912\\
112	-0.29137536613168\\
113	0.328629216219575\\
114	0.934258230918394\\
115	0.859813632237327\\
116	-0.0219227894523835\\
117	-0.459476677500043\\
118	-1.11517844670236\\
119	-1.1375686762975\\
120	-0.628516764593085\\
121	0.276155006855656\\
122	0.480987892777805\\
123	1.13421716257168\\
124	0.486116805979165\\
125	-0.400035767130413\\
126	-0.93691701637679\\
127	-1.09705509400709\\
128	-0.141777704610477\\
129	0.365742660434412\\
130	0.769239035491489\\
131	1.22085467233936\\
132	0.319439138456077\\
133	-0.52736635526525\\
134	-0.994871897218514\\
135	-1.10819093850091\\
136	-0.620399544942612\\
137	0.960819962365353\\
138	1.29960178208918\\
139	0.987262692427851\\
140	0.0884213583146095\\
141	-0.589227633811786\\
142	-0.968640623322739\\
143	-0.741340345639117\\
144	-0.667812998503713\\
145	-0.125821690385122\\
146	0.60485824331426\\
147	0.993075506086728\\
148	0.4573149782058\\
149	-0.294477525077993\\
150	-0.958290959213946\\
151	-0.877336956580824\\
152	-0.476291316786757\\
153	0.588456397171003\\
154	0.624417170668076\\
155	1.02026886013609\\
156	0.179814763919022\\
157	-0.47049465938296\\
158	-0.805434597472931\\
159	-0.68591594124551\\
160	-0.619842420400352\\
161	-0.61603286608914\\
162	-0.606008093366085\\
163	-0.573147871050279\\
164	-0.367849913966483\\
165	-0.11697054585428\\
166	0.124437566007243\\
167	0.444862707439232\\
168	0.66865693259875\\
169	1.01472279883799\\
170	1.28770658843005\\
171	1.10215225992375\\
172	0.934413237670486\\
173	1.03804652715547\\
174	0.794736787631649\\
175	0.327520492847653\\
176	0.536579518778607\\
177	0.136988835151332\\
178	-0.162017394665753\\
179	-0.324404201606751\\
180	-0.598681941994149\\
181	-1.04055205291123\\
182	-0.982328858779502\\
183	-1.03035053311308\\
184	-1.1003957482397\\
185	-0.521722296769318\\
186	-0.949704300486416\\
187	-0.666228687012945\\
188	-0.48349524205544\\
189	-0.331038998948329\\
190	0.14924753368904\\
191	0.378374380401242\\
192	0.999577515715522\\
193	0.774140647391298\\
194	0.707852325406168\\
195	1.35648683151248\\
196	1.25763812965774\\
197	0.846756366574474\\
198	0.41668724397075\\
199	0.458462551203131\\
200	0.289267135262124\\
201	0.130053099751461\\
202	-0.250709698566654\\
203	-0.340477008326697\\
204	-0.569481476686768\\
205	-1.10974302348441\\
206	-1.15813494511627\\
207	-1.16351063032981\\
208	-1.09498167481657\\
209	-0.969880081296787\\
210	-0.750944039133782\\
211	-1.23582764301467\\
212	-0.426433822557982\\
213	0.209759235549169\\
214	-0.0464387843012501\\
215	0.648654452452581\\
216	0.735573749932017\\
217	0.823376302785628\\
218	0.986830062880011\\
219	0.647439682931745\\
220	0.96242453468884\\
221	1.2570716906827\\
222	0.775742529673981\\
223	0.567219527785239\\
224	0.159871759955182\\
225	0.0613793167430422\\
226	-0.140272313380911\\
227	-0.344288941558919\\
228	-0.740606036879889\\
229	-0.882780788069948\\
230	-0.493141292244501\\
231	-1.50303903830455\\
232	-0.482439293772678\\
233	-0.822602262883777\\
234	-0.529941611401468\\
235	-0.930366252980151\\
236	-0.456200905113357\\
237	-0.13049974271521\\
238	0.286861674550422\\
239	0.0659495206128534\\
240	0.762721140124648\\
241	0.558457297619243\\
242	0.960813027160656\\
243	1.1436501867125\\
244	1.05453190034795\\
245	1.14041275609565\\
246	0.978873409701584\\
247	0.412301410384846\\
248	0.381687287340067\\
249	-0.14305725636615\\
250	-0.48805540920544\\
251	-0.23274883600295\\
252	-0.657168066268933\\
253	-0.842156894183672\\
254	-0.742109810064403\\
255	-0.864611132487397\\
256	-0.902974992239826\\
257	-0.618361843079025\\
258	-0.545666875351988\\
259	-0.504306963823113\\
260	-0.478651890584292\\
261	-0.214078892711062\\
262	0.459034243459959\\
263	0.0790115775467308\\
264	0.651705150582654\\
265	0.393755863926199\\
266	1.17436723239615\\
267	1.19050067770065\\
268	0.981602755860367\\
269	0.882290083871251\\
270	0.197355900645979\\
271	0.68801521264914\\
272	-0.166457691940342\\
273	-0.450731740041388\\
274	-0.174379229391499\\
275	-0.651955825860616\\
276	-0.565096208684015\\
277	-0.678373683751859\\
278	-0.754058434088378\\
279	-1.15233354122031\\
280	-0.880781372820665\\
281	-0.875622330257481\\
282	-0.56887712576905\\
283	-0.437980212362998\\
284	-0.123234794661365\\
285	-0.0977997275386044\\
286	0.159326111828863\\
287	0.665224232340323\\
288	0.181949452221189\\
289	0.716951281092865\\
290	0.661668556765502\\
291	0.912052060448205\\
292	1.12650358595648\\
293	1.08292247579415\\
294	0.526611146216218\\
295	0.452545361944025\\
296	0.354826791334723\\
297	0.00296185444801557\\
298	-0.124726032093976\\
299	-0.350209379691967\\
300	-0.865285208919747\\
301	-0.869438110171682\\
302	-1.42311934768869\\
303	-0.741358508556863\\
304	-1.12212008825621\\
305	-1.16753847209481\\
306	-0.810561870363843\\
307	-0.877662327320139\\
308	-0.32883098389719\\
309	-0.193739910866322\\
310	0.679608147682576\\
311	0.694470370217585\\
312	0.098428135080418\\
313	0.928626943612011\\
314	0.633127000011741\\
315	0.940590528339364\\
316	1.0181329061277\\
317	1.06546924325186\\
318	0.69263214189028\\
319	0.91070405010989\\
320	0.216539787551273\\
321	0.141565163467546\\
322	-0.0435131030517717\\
323	-0.420642148620925\\
324	-0.46004037463351\\
325	-0.757538832917121\\
326	-1.12147662561427\\
327	-1.40167143000115\\
328	-0.565428395027067\\
329	-1.03342319031178\\
330	-0.730604601735843\\
331	-0.431935966017516\\
332	-0.298742847416949\\
333	-0.270733630093761\\
334	0.0876019636594287\\
335	0.41175781409867\\
336	0.988139264197496\\
337	0.637234602159405\\
338	1.12159350178857\\
339	1.06672997166881\\
340	0.929009709344298\\
341	0.661613533472092\\
342	0.690149535066619\\
343	0.538561500720298\\
344	-0.00465646523817548\\
345	0.111281161726144\\
346	-0.37627879842148\\
347	-0.460794654353262\\
348	-0.581941692925113\\
349	-1.03230988455322\\
350	-1.0105068331685\\
351	-0.919329815383606\\
352	-1.39216966532235\\
353	-0.666312187523228\\
354	-0.211190547545835\\
355	-0.343267166447249\\
356	-0.394826666318896\\
357	0.0276420234787556\\
358	-0.039565045732779\\
359	0.859932674418289\\
360	0.867689233282988\\
361	1.00605874000897\\
362	0.749998696519005\\
363	1.06926449283589\\
364	0.85640809715825\\
365	0.828450569756119\\
366	0.626178027798463\\
367	0.328423201727208\\
368	0.120805188656423\\
369	0.0213041438495955\\
370	-0.572437050132515\\
371	-0.303737250167551\\
372	-0.683540019996941\\
373	-0.673409696970571\\
374	-0.885659337666745\\
375	-0.887055427661392\\
376	-1.31323423614018\\
377	-1.12645584908226\\
378	-0.853762504424613\\
379	-0.533421672571953\\
380	-0.071575824510434\\
381	-0.13316246305164\\
382	0.474575446073586\\
383	0.546095727852129\\
384	0.919922045199027\\
385	0.858275291607572\\
386	0.799019379433942\\
387	0.666160360657761\\
388	1.01613789787071\\
389	1.08131668583607\\
390	0.688259233334012\\
391	0.436944263460138\\
392	0.255095887847616\\
393	-0.177119907252768\\
394	-0.302592032399691\\
395	-0.480151273399634\\
396	-0.64692093875215\\
397	-0.844140518348403\\
398	-0.809197018662774\\
399	-0.973203536012252\\
400	-0.575403908754775\\
401	-0.828317553252875\\
402	-0.3499263062024\\
403	-0.719780471311514\\
404	-0.206333247004522\\
405	-0.126513497499533\\
406	0.326566093214344\\
407	0.572614893695009\\
408	0.167993209433768\\
409	0.532902627451338\\
410	0.623532308334842\\
411	1.08759096574336\\
412	1.31034306760566\\
413	0.825007770324228\\
414	0.935515284621458\\
415	0.68004146569094\\
416	0.0888301223468688\\
417	0.157219604441996\\
418	-0.360525075007372\\
419	-0.10039669253779\\
420	-0.664466926506842\\
421	-0.46376460931113\\
422	-1.04112029450328\\
423	-0.865763669002373\\
424	-0.995394071617775\\
425	-1.3506141815352\\
426	-0.973514887542959\\
427	-0.420893731925561\\
428	-0.33644585473082\\
429	-0.318623659258791\\
430	0.0520962817223639\\
431	0.49554244028384\\
432	0.434962293293581\\
433	1.04804069924054\\
434	0.802623660566717\\
435	1.40245225560789\\
436	0.739691177030355\\
437	0.942716413977648\\
438	0.41336035148683\\
439	0.39602743382083\\
440	0.191406445731553\\
441	0.158085482876556\\
442	0.0185630479401916\\
443	-0.372463089640613\\
444	-0.740664648417269\\
445	-0.647377879438741\\
446	-0.767226840682784\\
447	-0.970767121014372\\
448	-0.853509408929759\\
449	-0.805766842162616\\
450	-0.974595820752279\\
451	-0.38798543837364\\
452	-0.471231418531446\\
453	-0.203419138022754\\
454	0.231863731186168\\
455	0.370892719097018\\
456	0.627680577719203\\
457	0.963095808898729\\
458	1.18026040425065\\
459	0.953367543944153\\
460	1.05467023987364\\
461	0.844815070631719\\
462	0.805113343084365\\
463	0.656423324414439\\
464	0.186121764231982\\
465	0.129777523134616\\
466	-0.0577526363031583\\
467	-0.41904898477425\\
468	-0.270076821899908\\
469	-0.96604964212072\\
470	-1.11093966936659\\
471	-1.38925302095834\\
472	-0.777595707378577\\
473	-0.704106995430334\\
474	-0.771612700267547\\
475	-0.356900802461717\\
476	-0.283054492404104\\
477	-0.00320202335470898\\
478	0.217541395819505\\
479	0.538157695218252\\
480	0.679929154533514\\
481	1.01930140086523\\
482	0.668060688991065\\
483	0.503489309179967\\
484	0.114477576774999\\
485	-0.640221398044269\\
486	-1.0261686836388\\
487	-0.956142785008992\\
488	-0.418641459231029\\
489	0.516693300347801\\
490	1.01624562500185\\
491	1.08655211013674\\
492	0.767674765466557\\
493	-0.121858079655163\\
494	-1.21630190614165\\
495	-1.43961415506745\\
496	-0.167904771653681\\
497	-0.01522367348191\\
498	0.944074776520311\\
499	0.926382310851786\\
500	0.868131941758989\\
501	-0.41104265740323\\
502	-1.04266428645294\\
503	-0.865673459331643\\
504	-0.314732213947546\\
505	0.411228607877523\\
506	0.792948356693736\\
507	0.644851860938034\\
508	0.440564384558026\\
509	-0.696060059789016\\
510	-1.1601172863058\\
511	-0.620544411429331\\
512	-0.463480001588118\\
513	0.364152879226179\\
514	1.13049742986501\\
515	0.851284163044007\\
516	0.600088771817975\\
517	-0.472772514430859\\
518	-0.702492054216191\\
519	-0.777610797287478\\
520	-0.417406539169474\\
521	0.201828962477241\\
522	0.695710106130881\\
523	0.858006978813253\\
524	0.271691318713108\\
525	-0.487006734062427\\
526	-0.709039077177997\\
527	-0.985061149950892\\
528	-0.113806739288532\\
529	0.276588465195099\\
530	1.1467758843434\\
531	0.801573983878868\\
532	0.409252634605608\\
533	-0.178323087499314\\
534	-1.02177432807535\\
535	-1.0165185371595\\
536	0.0785787959687947\\
537	0.608366448752742\\
538	0.832462529254727\\
539	1.06366213526494\\
540	0.289160648586482\\
541	-0.237591617282097\\
542	-0.612298502004791\\
543	-1.2774998608703\\
544	-0.139502366122195\\
545	0.721816711699306\\
546	0.939759064403215\\
547	1.3092185476558\\
548	0.404510692109707\\
549	-0.672395517181594\\
550	-1.41999421176258\\
551	-0.993052285465951\\
552	-0.210062110954808\\
553	0.466583177952878\\
554	1.02169334583669\\
555	0.789303033277094\\
556	0.401962807219367\\
557	-0.762253141361555\\
558	-1.09922825073515\\
559	-1.10167154815957\\
560	-0.253433731763093\\
561	0.392385226249861\\
562	0.670552617711212\\
563	0.916311264994851\\
564	0.215373008415186\\
565	-0.544747307696462\\
566	-0.735757368425187\\
567	-0.893061839794358\\
568	-0.280592918899315\\
569	0.593367963851947\\
570	0.969473882756411\\
571	1.04171248745736\\
572	0.522571360260111\\
573	-0.133596165841012\\
574	-0.822648305391404\\
575	-0.602414542199213\\
576	-0.364678902634752\\
577	0.384841777023531\\
578	1.3098591597528\\
579	0.80437989940376\\
580	0.369099329062955\\
581	-0.189706898216264\\
582	-0.895611905364479\\
583	-0.60404795993901\\
584	-0.138323392368495\\
585	0.460801297140596\\
586	0.755103271673311\\
587	1.04525750141737\\
588	0.060336451041169\\
589	-0.340845622023469\\
590	-0.748224241985025\\
591	-0.870768456753324\\
592	-0.172884820127885\\
593	0.851811903191203\\
594	1.13589277146703\\
595	0.978173296079285\\
596	0.513328662442224\\
597	-0.300339209743367\\
598	-1.22334365324879\\
599	-1.01162688238398\\
600	-0.0956786350861595\\
601	0.38580071639208\\
602	1.05947014665281\\
603	0.69320903259531\\
604	0.384175709943616\\
605	-0.261221142529616\\
606	-1.23381460347292\\
607	-0.840658766067117\\
608	-0.410430926560072\\
609	0.185305680547613\\
610	0.920736828743545\\
611	0.494089374916931\\
612	-0.10645857771332\\
613	-0.658924829538771\\
614	-1.15079641037701\\
615	-1.18094289465996\\
616	-0.755852937088296\\
617	0.460053382736482\\
618	0.572390928985093\\
619	0.943107420476604\\
620	0.545867304692174\\
621	-0.396052260239627\\
622	-0.908300175829521\\
623	-1.00306639909072\\
624	-0.701530177967909\\
625	0.385748267081998\\
626	1.14414860182831\\
627	1.30742567906564\\
628	0.273467031426545\\
629	-0.759714816166312\\
630	-0.891910876099066\\
631	-0.834242284786917\\
632	-0.420524986827397\\
633	0.13845136071086\\
634	1.38213935854618\\
635	0.390430626778352\\
636	0.255617109440249\\
637	-0.691306175236857\\
638	-1.07148819854613\\
639	-0.84731422877912\\
640	-0.338845954103233\\
641	0.236506276096229\\
642	1.10301314531468\\
643	1.05772151253675\\
644	0.514608539889397\\
645	-0.490801438845854\\
646	-0.694485483979011\\
647	-0.77059028624821\\
648	0.191672708825854\\
649	0.633534023466599\\
650	1.18803089269856\\
651	0.930298298244412\\
652	0.0814243122731326\\
653	-0.478625519291741\\
654	-1.09871398791672\\
655	-1.04464721533837\\
656	-0.245509786241697\\
657	0.270958641857807\\
658	0.728828335411137\\
659	0.826843741484081\\
660	0.333735109062244\\
661	-0.303953449706979\\
662	-1.14231602652812\\
663	-0.847433714914344\\
664	-0.352278943807456\\
665	0.691923773933963\\
666	0.881432867302942\\
667	0.888349228501594\\
668	0.107613225119127\\
669	-0.705546828329479\\
670	-0.859650692019637\\
671	-0.974273187911836\\
672	-0.257010803630665\\
673	0.195794960763383\\
674	1.35622119961152\\
675	0.945795422091554\\
676	0.604042401837798\\
677	-0.446333956447935\\
678	-0.965500515944691\\
679	-0.764023529898352\\
680	-0.245371443430855\\
681	0.144864263682106\\
682	0.586151265677938\\
683	0.672762849426268\\
684	0.0529176890917725\\
685	-0.38222329752742\\
686	-1.02116534686806\\
687	-1.00079586364656\\
688	-0.674293352968525\\
689	0.570004871973935\\
690	1.02746700876252\\
691	0.898387101089507\\
692	0.598257006704144\\
693	-0.591132315203109\\
694	-0.836334702750883\\
695	-0.840468874261103\\
696	-0.291582360469656\\
697	0.232363633715321\\
698	1.00227599916006\\
699	0.803222979174645\\
700	0.169132270989917\\
701	-0.664820258395378\\
702	-0.696894493389138\\
703	-1.10774740126297\\
704	-0.408432718980997\\
705	0.125082848117722\\
706	0.862645196594941\\
707	0.195180140228019\\
708	0.12649942015735\\
709	-0.714760592981698\\
710	-1.15614888203604\\
711	-0.966167488020122\\
712	-0.442544252399863\\
713	0.830669595846384\\
714	0.801185060370302\\
715	0.862490737072966\\
716	0.0185581608067733\\
717	-0.502382228519385\\
718	-1.22854722291428\\
719	-0.911662453274613\\
720	-0.17883519593889\\
721	0.486017440170949\\
722	0.772351792066548\\
723	0.553561274426778\\
724	0.69645868815065\\
725	0.0632028248216609\\
726	-0.902170453818765\\
727	-1.13996206100198\\
728	-0.101701952544765\\
729	0.262929859764127\\
730	0.82400244303827\\
731	1.11679549964199\\
732	0.0597549292810514\\
733	-0.833434647073436\\
734	-0.834995741110849\\
735	-0.828765942322568\\
736	-0.348447380358774\\
737	0.506701552382449\\
738	1.17131883078869\\
739	1.13560870303566\\
740	0.242504206438348\\
741	-0.21429910702645\\
742	-0.890366827104387\\
743	-1.03157261796995\\
744	-0.636775139822825\\
745	0.540332544993747\\
746	0.850002615846391\\
747	0.928833820349157\\
748	0.192293108337642\\
749	-0.742526730281282\\
750	-0.890738527943857\\
751	-0.932348181633639\\
752	-0.101420795780991\\
753	0.574958590343753\\
754	1.16484355411125\\
755	0.750859755165678\\
756	0.160207881768916\\
757	-0.679549084330222\\
758	-0.817698303058284\\
759	-0.294304597551226\\
760	-0.206922773003553\\
761	0.222558340659416\\
762	1.11635714573154\\
763	0.665945968642823\\
764	0.0509715335117867\\
765	-0.235005335616667\\
766	-1.10321499262655\\
767	-0.847731697326099\\
768	-0.0549850017450266\\
769	0.485317273458705\\
770	1.13717454155668\\
771	0.559097724061407\\
772	0.517776213930051\\
773	0.0829982193399048\\
774	-0.808040266549575\\
775	-1.26325889575027\\
776	-0.415197327202233\\
777	0.283672170182961\\
778	1.01484873054623\\
779	1.01064766036582\\
780	0.460478979242427\\
781	-0.476960491441103\\
782	-1.06324054806723\\
783	-1.05038045326505\\
784	-0.17873358532289\\
785	-0.0191105799256349\\
786	0.882745852576754\\
787	0.978947741062978\\
788	0.223661681178706\\
789	-0.288760445864187\\
790	-0.945095822187015\\
791	-1.12841498725287\\
792	-0.333628418354232\\
793	0.33554022026723\\
794	0.837415029008171\\
795	0.759263565970758\\
796	0.383488426341349\\
797	-0.808679144026823\\
798	-1.01827461148341\\
799	-1.04007746753121\\
800	-0.573802062391896\\
801	0.21433510845561\\
802	-0.356581307986382\\
803	-0.318723175849003\\
804	-0.438806398461789\\
805	-0.864971512058881\\
806	-0.883835385595778\\
807	-0.854555929350032\\
808	-0.999463895492281\\
809	-0.777106524158019\\
810	-1.03627001131877\\
811	-0.30947129695036\\
812	-0.545606958584707\\
813	-0.477530098456698\\
814	0.502088596125529\\
815	0.425668630360565\\
816	0.757638398585057\\
817	0.748577522823704\\
818	0.717615813216785\\
819	0.309995856182025\\
820	1.12149087431569\\
821	0.833987958638565\\
822	0.709573126578465\\
823	0.648726993525858\\
824	0.00654292243083382\\
825	-0.0949598334561264\\
826	0.0644812993838626\\
827	-0.305915951002137\\
828	-0.943903225346762\\
829	-1.32291173992607\\
830	-1.07383407955717\\
831	-0.949778608715188\\
832	-0.77045841670427\\
833	-0.877168165071352\\
834	-1.00488428962134\\
835	-0.48944924669207\\
836	-0.586507559690972\\
837	-0.283319046785142\\
838	0.0459744730226675\\
839	0.164592937969456\\
840	0.596542880735794\\
841	0.628477145293418\\
842	0.882070480809565\\
843	0.894186263957177\\
844	0.889641327485107\\
845	0.785460458537219\\
846	1.02972050375374\\
847	0.694534546488579\\
848	0.337382599772753\\
849	-0.259997274917199\\
850	-0.556087563981771\\
851	-0.879376161487963\\
852	-0.0741907859925575\\
853	-0.612269999209383\\
854	-0.888494098273472\\
855	-1.21568123551159\\
856	-1.23787512938107\\
857	-0.7755916825144\\
858	-0.403501219680978\\
859	-0.936895945876748\\
860	-0.424729234049514\\
861	-0.0871375930046681\\
862	-0.253500667513944\\
863	0.627958799998025\\
864	0.564387210158208\\
865	1.25777334941702\\
866	0.858526098045054\\
867	1.08923296423566\\
868	0.725691125663017\\
869	0.758310502393582\\
870	0.492198569195226\\
871	0.645811131037048\\
872	0.615487994487575\\
873	-0.064569849878289\\
874	-0.0946150204936975\\
875	-0.552571374169566\\
876	-0.634329646325718\\
877	-0.562171283477979\\
878	-0.91908429193574\\
879	-1.22969046637648\\
880	-1.17314452191258\\
881	-0.936156297481096\\
882	-0.796625711072081\\
883	-0.751712349485147\\
884	-0.283734621961686\\
885	0.215147852451861\\
886	0.136060281121193\\
887	0.393927069968946\\
888	0.164524567305772\\
889	0.656547319498154\\
890	0.634215750762191\\
891	0.911235300580784\\
892	1.09895796136435\\
893	1.23901784507834\\
894	1.15619987192895\\
895	0.53180471717183\\
896	0.252510260636011\\
897	0.251217868034793\\
898	-0.082579329199408\\
899	-0.268458339788799\\
900	-0.483016614735951\\
901	-1.1611285893698\\
902	-0.825078899678403\\
903	-1.01815107685057\\
904	-1.15147132040747\\
905	-1.05335415771788\\
906	-0.466970850873849\\
907	-0.739163353689599\\
908	-0.600937617845943\\
909	-0.0202324400739786\\
910	0.64223547977807\\
911	0.445424126528695\\
912	0.726170200701449\\
913	0.619907043933959\\
914	1.32065566288573\\
915	1.02563629893871\\
916	1.09996830319246\\
917	0.685088976136434\\
918	0.944851308157625\\
919	0.645043670046513\\
920	0.0654745687080291\\
921	0.0771677973956698\\
922	-0.293101288414909\\
923	-0.41525816847486\\
924	-0.713247718250106\\
925	-0.872363214151397\\
926	-1.17714556213567\\
927	-1.07001253880678\\
928	-0.809809994224683\\
929	-0.507720529547919\\
930	-1.01343448836873\\
931	-0.0259458235011355\\
932	0.0173409125367999\\
933	-0.0305663248066962\\
934	0.124862990725034\\
935	0.283361333658808\\
936	0.843535895381426\\
937	0.674429028675057\\
938	0.842641301982931\\
939	1.19540162732622\\
940	1.07889791511669\\
941	1.09881531736768\\
942	0.866682843832663\\
943	0.46824856488332\\
944	0.209175023901058\\
945	0.246485233282117\\
946	-0.342465075597302\\
947	-0.174156050899039\\
948	-0.480236623314651\\
949	-0.765481121905919\\
950	-0.945279138683288\\
951	-0.915849058603082\\
952	-0.192145816118548\\
953	-0.923302952099299\\
954	-1.10206758498746\\
955	-0.129401053381819\\
956	-0.187692665889376\\
957	-0.213263884948622\\
958	0.777971651814489\\
959	0.562996782844031\\
960	0.723015865643952\\
961	0.655801979810099\\
962	0.707691946663875\\
963	0.601932402795283\\
964	0.886702140539051\\
965	0.66247962958703\\
966	0.898688454735743\\
967	0.486882417298813\\
968	0.720026876569427\\
969	0.406311598762335\\
970	-0.155564639612883\\
971	-0.502982195368421\\
972	-0.399348560726759\\
973	-1.08643166638487\\
974	-0.795241368410249\\
975	-1.14740007745342\\
976	-1.07092657101389\\
977	-1.04847733560062\\
978	-0.624920590509113\\
979	-0.732082065601978\\
980	-0.562166283776104\\
981	0.0554702161020068\\
982	0.0975121890072768\\
983	0.520621238414387\\
984	0.578433753458114\\
985	0.890184772122006\\
986	0.372130833049713\\
987	1.10191722300275\\
988	1.39613095870203\\
989	1.13071323675043\\
990	0.95757399331595\\
991	0.504314118144211\\
992	0.364667076988001\\
993	0.122932737193345\\
994	-0.170696009719155\\
995	-0.625547094568\\
996	-0.736005908673536\\
997	-0.868978193401726\\
998	-1.05356127925118\\
999	-0.810399964065696\\
1000	-0.413313101649137\\
1001	-1.19427655961306\\
1002	-0.725014478026846\\
1003	-0.880735326928257\\
1004	-0.0326893319910943\\
1005	0.247233980745809\\
1006	-0.179775692040355\\
1007	0.874661596900948\\
1008	0.414446346809526\\
1009	0.880624539941659\\
1010	1.19191849532395\\
1011	1.03060432508255\\
1012	1.49506480146077\\
1013	1.51410697751173\\
1014	0.784686970302043\\
1015	0.131199361948327\\
1016	0.242485670307616\\
1017	-0.121514533108604\\
1018	-0.363402567095093\\
1019	-0.692060140821447\\
1020	-0.639682178628712\\
1021	-0.358447438280369\\
1022	-1.10619278177195\\
1023	-1.06044010468314\\
1024	-0.72030994556448\\
1025	-0.626737853248858\\
1026	-0.645080411270693\\
1027	-0.32835990119816\\
1028	-0.129313173966106\\
1029	-0.153704231861574\\
1030	0.288244033392401\\
1031	0.372766410517565\\
1032	0.455807428020136\\
1033	0.97190997364199\\
1034	0.961095781578997\\
1035	0.955792881783179\\
1036	1.04662522155553\\
1037	1.11916743138517\\
1038	0.841664906065393\\
1039	0.601497513588171\\
1040	0.386773518189579\\
1041	0.0797368231609138\\
1042	-0.365576106789341\\
1043	-0.26364025198072\\
1044	-0.341895251411467\\
1045	-0.94938730308738\\
1046	-0.514608973847983\\
1047	-1.03710707686332\\
1048	-1.03588894959042\\
1049	-1.09900626712607\\
1050	-0.931042803728742\\
1051	-0.771216822704503\\
1052	-0.244966830156112\\
1053	0.289106373020173\\
1054	0.310303558706387\\
1055	0.630858224445407\\
1056	0.957456467125394\\
1057	0.270523771300742\\
1058	0.908504796653744\\
1059	1.07666401382112\\
1060	1.14184911964271\\
1061	0.606143410802164\\
1062	0.528641872647974\\
1063	0.734899434612691\\
1064	0.298077254010102\\
1065	0.192044759652252\\
1066	-0.407238982920115\\
1067	-0.805316454429111\\
1068	-0.486838485540347\\
1069	-0.562829952530769\\
1070	-0.580780733874726\\
1071	-1.34053860877995\\
1072	-1.28050237228049\\
1073	-1.22795805901338\\
1074	-0.763057676815373\\
1075	-0.69569867182281\\
1076	-0.0300830661757573\\
1077	-0.393888318559923\\
1078	-0.197634665190938\\
1079	0.485648480846639\\
1080	0.924132146741177\\
1081	0.650212133806517\\
1082	0.662778059855771\\
1083	0.964249314170058\\
1084	0.615140093859251\\
1085	0.902023276365349\\
1086	0.939110131031139\\
1087	0.60668627450875\\
1088	-0.103085676272799\\
1089	-0.0518547330833591\\
1090	-0.259855250055908\\
1091	-0.0338637484393043\\
1092	-0.452454493228395\\
1093	-0.84264066019633\\
1094	-0.652993170741742\\
1095	-1.23145663225977\\
1096	-1.05927754923429\\
1097	-0.582040267807107\\
1098	-0.417532407478733\\
1099	-0.394518072856668\\
1100	-0.214343368520852\\
1101	-0.199437583275732\\
1102	0.358959720251226\\
1103	0.254381780036354\\
1104	0.785744970252065\\
1105	0.892937406383688\\
1106	1.20095404009868\\
1107	0.902366218824683\\
1108	1.12369753651783\\
1109	1.07563867374724\\
1110	0.552696302427869\\
1111	0.592925311734075\\
1112	0.681595573102062\\
1113	0.0934364468781667\\
1114	-0.261330544418981\\
1115	-0.337514688201392\\
1116	-0.718741045645473\\
1117	-0.730700896615574\\
1118	-0.97615869843894\\
1119	-1.00176802956976\\
1120	-0.878227649746903\\
1121	-0.593284205970064\\
1122	-0.652614253335581\\
1123	-0.189087117917536\\
1124	-0.778068556850537\\
1125	-0.168809084268304\\
1126	0.245077078522599\\
1127	0.252811299458165\\
1128	0.371647086608412\\
1129	0.967946760240321\\
1130	1.08311088449777\\
1131	1.06242401710682\\
1132	1.06980620329126\\
1133	0.703328179365466\\
1134	0.642362078662933\\
1135	0.534090050316356\\
1136	0.170241679696628\\
1137	0.142542799913116\\
1138	0.337021206849252\\
1139	-0.547618747738006\\
1140	-0.512293825819104\\
1141	-1.06134984934018\\
1142	-0.646236714903426\\
1143	-1.21453993878956\\
1144	-0.934636510413194\\
1145	-1.0365702558564\\
1146	-0.639155935664152\\
1147	-0.555420150086554\\
1148	-0.334007220841987\\
1149	0.591670573217839\\
1150	0.051275537157763\\
1151	0.429214946496042\\
1152	1.25758433277446\\
1153	0.573261744573045\\
1154	1.06975065327221\\
1155	0.756777629320746\\
1156	1.2119783340547\\
1157	0.971436309807072\\
1158	0.899666272475148\\
1159	0.495557678924856\\
1160	0.37903224560964\\
1161	0.397792569822855\\
1162	-0.701386966499208\\
1163	-0.805201953167223\\
1164	-0.564205944694955\\
1165	-0.976390790546128\\
1166	-1.01231764362977\\
1167	-1.3326457563279\\
1168	-1.18382577015338\\
1169	-0.988589367738508\\
1170	-0.916152831622672\\
1171	-0.751862240688552\\
1172	-0.418563989389698\\
1173	-0.27922383518756\\
1174	0.492552294629063\\
1175	0.218514518899362\\
1176	1.06399685228326\\
1177	0.746076089726628\\
1178	0.621131802256932\\
1179	0.859218478390651\\
1180	1.19046451702715\\
1181	1.13443690280051\\
1182	0.789589870745844\\
1183	0.622275169881755\\
1184	0.465794543016189\\
1185	-0.258228331932242\\
1186	-0.322441221016937\\
1187	-0.849254798539538\\
1188	-0.757679117397308\\
1189	-0.617439321163572\\
1190	-0.785452662225877\\
1191	-0.485698262087811\\
1192	-0.944032534581204\\
1193	-1.3807195953782\\
1194	-0.375682045151536\\
1195	-0.27101330815459\\
1196	-0.454544133987264\\
1197	-0.0184583118677024\\
1198	0.366691696974521\\
1199	0.52579501024684\\
1200	0.813011604772833\\
1201	0.804643797685199\\
1202	0.937313141819486\\
1203	1.0173789509658\\
1204	0.804634180100371\\
1205	1.21657086129618\\
1206	0.755151249104851\\
1207	0.711581062663251\\
1208	0.132847180667891\\
1209	-0.172340436189887\\
1210	-0.212623297663592\\
1211	-0.496270638240758\\
1212	-0.38980258825958\\
1213	-0.694242778445215\\
1214	-0.825045227733048\\
1215	-1.32101383867338\\
1216	-1.19789238305545\\
1217	-0.852927101281013\\
1218	-0.831582180088912\\
1219	-0.269266210103759\\
1220	-0.0240412512164787\\
1221	-0.19824518733749\\
1222	0.388103808030175\\
1223	0.751978137651387\\
1224	0.729009014774794\\
1225	1.19314629955703\\
1226	1.18375147609985\\
1227	1.04557883792341\\
1228	1.17755296248744\\
1229	0.941646205335729\\
1230	0.660890144291786\\
1231	0.638176559893911\\
1232	0.33226936147668\\
1233	-0.0133220587248164\\
1234	0.204050490335098\\
1235	-0.390189283364743\\
1236	-0.046177704240923\\
1237	-0.89614281878155\\
1238	-0.819723508797282\\
1239	-0.643509081104171\\
1240	-0.370597966590321\\
1241	-0.830210623419778\\
1242	-0.930577704206432\\
1243	-0.378181248085896\\
1244	-0.619530161201234\\
1245	-0.129555751912236\\
1246	0.253104868306146\\
1247	0.773036563941944\\
1248	0.978781499817495\\
1249	0.823699522007686\\
1250	1.15275762055077\\
1251	0.925770701809205\\
1252	1.12928865687525\\
1253	1.32674478266208\\
1254	0.788813879260986\\
1255	0.490707189875201\\
1256	0.212266137512219\\
1257	0.228606127747604\\
1258	0.106990580835703\\
1259	0.0359772289119383\\
1260	-0.645084902895272\\
1261	-0.793383151619454\\
1262	-1.08758763895177\\
1263	-0.636936502429192\\
1264	-0.998226555597783\\
1265	-1.00404256936421\\
1266	-0.26494684999522\\
1267	-0.376726488218107\\
1268	-0.163860915722894\\
1269	-0.293300618807066\\
1270	0.289383871658832\\
1271	0.556102857686437\\
1272	0.412757694934701\\
1273	0.779339101608149\\
1274	0.855388033405144\\
1275	1.11579042963723\\
1276	0.756005433374871\\
1277	1.14184343567352\\
1278	1.15315205831324\\
1279	0.489987054725171\\
1280	0.322217359760209\\
1281	0.508928296790883\\
1282	1.19839865478151\\
1283	1.03525941518623\\
1284	0.458588838180833\\
1285	-0.503596941155395\\
1286	-0.980940213537728\\
1287	-0.781179933992255\\
1288	0.0068486173479792\\
1289	0.522748570702096\\
1290	0.659251797226644\\
1291	1.01535261897698\\
1292	0.349123508714081\\
1293	-0.45016835333311\\
1294	-0.978185821095505\\
1295	-1.11531617175336\\
1296	-0.441527521937618\\
1297	0.220018180891706\\
1298	0.847464976494844\\
1299	0.9446854489441\\
1300	0.408751793223661\\
1301	-0.443813356167062\\
1302	-0.96328723699204\\
1303	-0.910904021929518\\
1304	-0.267184216870018\\
1305	0.682356269977574\\
1306	1.06789918454757\\
1307	0.854291115232747\\
1308	0.503561645094917\\
1309	-0.450499040118482\\
1310	-1.32762778669283\\
1311	-1.44370589213704\\
1312	-0.753205072626511\\
1313	0.3424898200666\\
1314	0.79047664801913\\
1315	0.757285880071975\\
1316	0.378836436451319\\
1317	-0.54302024191585\\
1318	-1.0702320777739\\
1319	-0.782021370627786\\
1320	-0.194667971436237\\
1321	0.940810676812772\\
1322	0.996808441250815\\
1323	0.931493508515439\\
1324	0.241259011643405\\
1325	-0.439238412974646\\
1326	-0.940443818299541\\
1327	-1.31134855138728\\
1328	-0.427329474349492\\
1329	0.563285782038497\\
1330	0.801410833225624\\
1331	1.02904185671108\\
1332	0.591601887204756\\
1333	-0.154512475768834\\
1334	-0.755294992227468\\
1335	-1.42414784749211\\
1336	-0.49603393380755\\
1337	0.597288919895315\\
1338	1.01745611964748\\
1339	0.919529061131671\\
1340	0.46751147948933\\
1341	-0.142667263187655\\
1342	-0.89472529056353\\
1343	-1.08820766663388\\
1344	-0.410054950471221\\
1345	0.34905497938744\\
1346	1.12980885231896\\
1347	1.01070860077728\\
1348	0.4922616881605\\
1349	-0.36247155468821\\
1350	-1.0102038014549\\
1351	-0.904760596688214\\
1352	-0.321244030257319\\
1353	0.0826833363885593\\
1354	0.969163429271583\\
1355	1.12601673065087\\
1356	0.345082687091821\\
1357	-0.0436841245145719\\
1358	-0.803709871743005\\
1359	-1.01250709206011\\
1360	-0.404187974737545\\
1361	0.333970250458142\\
1362	0.98309369834222\\
1363	1.08732580612324\\
1364	0.304414866497576\\
1365	-0.292963444208966\\
1366	-0.536339012250785\\
1367	-0.709994589369523\\
1368	-0.185061251831987\\
1369	0.213217121371419\\
1370	0.809592227268819\\
1371	0.972720741372739\\
1372	0.347215299199084\\
1373	-0.759205878204656\\
1374	-1.26803226668042\\
1375	-0.688802974181877\\
1376	-0.539391315639024\\
1377	0.860644138172064\\
1378	0.431901746340296\\
1379	1.01886667848109\\
1380	0.369878321117103\\
1381	-0.564784525688945\\
1382	-0.838656515697485\\
1383	-1.09609044723939\\
1384	-0.177336675067146\\
1385	0.410517885633083\\
1386	0.5622752903266\\
1387	0.375834319272094\\
1388	0.306222995924114\\
1389	-0.139146416592641\\
1390	-0.888594067260069\\
1391	-0.906302673540689\\
1392	-0.215833295980682\\
1393	0.0837384798500017\\
1394	1.24810307568988\\
1395	0.718417509047095\\
1396	0.378189161211598\\
1397	-0.476913170504293\\
1398	-0.895668353688236\\
1399	-1.35358531013527\\
1400	-0.0287223376892019\\
1401	0.517694875832448\\
1402	1.37891579669419\\
1403	1.23546406869088\\
1404	0.372307342892568\\
1405	-0.605693590900976\\
1406	-1.3182673771544\\
1407	-0.914644993188649\\
1408	-0.320694978528123\\
1409	0.62958677694554\\
1410	0.716325903238078\\
1411	1.09630181727399\\
1412	0.385722710183127\\
1413	-0.568688932885397\\
1414	-0.755894710135493\\
1415	-0.858994761072746\\
1416	-0.422349920440574\\
1417	0.815652770524586\\
1418	1.06547103574132\\
1419	0.893765749849494\\
1420	0.431827400823\\
1421	-0.376074893032931\\
1422	-0.890289471223459\\
1423	-0.816938509292703\\
1424	-0.504792948740641\\
1425	0.217196631706191\\
1426	1.00984166911859\\
1427	1.34028691566659\\
1428	0.30107788759954\\
1429	-0.622180437621806\\
1430	-0.982517478991135\\
1431	-1.02675515521132\\
1432	-0.442991081434523\\
1433	0.501964952118524\\
1434	0.900043591759897\\
1435	0.785438882858016\\
1436	0.270432437027398\\
1437	-0.691508484309642\\
1438	-1.12147887079234\\
1439	-1.23464291377755\\
1440	-0.485205084807943\\
1441	0.587044880581732\\
1442	0.787753742432463\\
1443	0.930162562841888\\
1444	0.712709175725688\\
1445	0.893809102647043\\
1446	0.829902178666773\\
1447	0.342409986583245\\
1448	0.573436110210006\\
1449	-0.286662504005328\\
1450	-0.279942192904477\\
1451	-0.771947367188524\\
1452	-0.734299588902517\\
1453	-1.12613193529424\\
1454	-1.07463714028571\\
1455	-1.16342126048015\\
1456	-1.2465213929189\\
1457	-1.36021455375602\\
1458	-0.627285757846185\\
1459	-0.110370860350724\\
1460	0.173370855870148\\
1461	-0.178413857779203\\
1462	0.181972228839843\\
1463	0.335481118480415\\
1464	0.674174671646452\\
1465	0.624265716789121\\
1466	0.516464528003403\\
1467	0.989511339493335\\
1468	0.707185140124436\\
1469	0.4724168795013\\
1470	1.28504162258432\\
1471	0.505743076071982\\
1472	0.414543564353358\\
1473	0.335124370669837\\
1474	-0.569280592791439\\
1475	-0.347251559589679\\
1476	-0.544899113404251\\
1477	-0.811279120523856\\
1478	-0.910689165194156\\
1479	-1.11580907018792\\
1480	-0.819536812360588\\
1481	-1.08834117614208\\
1482	-0.931948479424589\\
1483	-0.395645566857681\\
1484	0.0532499693458477\\
1485	-0.154741737019832\\
1486	0.0792407180650159\\
1487	0.531023998304239\\
1488	0.897150144630307\\
1489	1.04648155538331\\
1490	1.00645400775559\\
1491	0.855699612453566\\
1492	1.12227056166901\\
1493	0.880411113288191\\
1494	1.06186683501596\\
1495	0.242832412454635\\
1496	0.355768187427578\\
1497	0.356887068983896\\
1498	-0.303023139163717\\
1499	-0.579528361165523\\
1500	-0.866107090378417\\
1501	-0.892240005701969\\
1502	-0.990874376503198\\
1503	-0.967066735898816\\
1504	-1.17901774382973\\
1505	-0.879671709278872\\
1506	-0.888868948931552\\
1507	-0.640474666317577\\
1508	-0.511728531320685\\
1509	-0.131486805071437\\
1510	0.991128626448898\\
1511	1.20224020195793\\
1512	0.913835500917293\\
1513	1.00603899218981\\
1514	0.659980608837592\\
1515	0.86684681938497\\
1516	0.842732645920857\\
1517	1.01793117539795\\
1518	0.861816772260108\\
1519	0.58223543628038\\
1520	0.248014864928753\\
1521	-0.138910766395274\\
1522	-0.191075842337043\\
1523	-0.184787697035878\\
1524	-0.561544693797794\\
1525	-0.789355970662545\\
1526	-0.798870832664095\\
1527	-0.866924581592909\\
1528	-1.3430707684654\\
1529	-0.893940665570475\\
1530	-0.84905547773701\\
1531	-1.01313382593804\\
1532	-0.621465816981552\\
1533	-0.13985851150142\\
1534	0.376980835678167\\
1535	0.388394632586743\\
1536	0.715865479975212\\
1537	0.628918107128587\\
1538	0.463241755690333\\
1539	1.01868316423248\\
1540	0.768488146304795\\
1541	1.16054312678038\\
1542	1.14201259169066\\
1543	0.474783159400196\\
1544	0.59110157986157\\
1545	0.281681310878935\\
1546	-0.508924158685093\\
1547	-0.447170025475258\\
1548	-0.598746515814911\\
1549	-0.712831926062189\\
1550	-0.794550305739217\\
1551	-0.846750881570185\\
1552	-0.829531177724379\\
1553	-0.968090791017464\\
1554	-0.66886751087331\\
1555	-1.0621125808595\\
1556	-0.585615985395838\\
1557	0.200700033257801\\
1558	0.0444072191581059\\
1559	0.513006454291452\\
1560	0.994666585327886\\
1561	0.630947915154078\\
1562	0.731333196913475\\
1563	0.887406257642231\\
1564	1.08004196286581\\
1565	0.852320008625467\\
1566	0.53492156771114\\
1567	0.205968359628392\\
1568	0.241306673823104\\
1569	0.018787046634366\\
1570	-0.312036184498294\\
1571	-0.498329482336555\\
1572	-0.902310217613226\\
1573	-0.706960195462122\\
1574	-1.43263577156508\\
1575	-0.96530481357339\\
1576	-0.831626986547251\\
1577	-1.08809222358079\\
1578	-0.766509927482119\\
1579	-0.492665300573527\\
1580	-0.461702285828643\\
1581	-0.071327847251537\\
1582	0.284489124573929\\
1583	0.229991057686578\\
1584	0.844069935461936\\
1585	1.07480205052821\\
1586	1.01787462039762\\
1587	1.0010484921295\\
1588	0.720753579474489\\
1589	1.00887700506642\\
1590	0.481652723781927\\
1591	0.913946124132913\\
1592	0.441977612083935\\
1593	0.127535660152448\\
1594	-0.28034571849062\\
1595	-0.926718342039604\\
1596	-0.924741728584566\\
1597	-0.748114317937461\\
1598	-0.406146670199489\\
1599	-0.832014676175793\\
1600	-0.912027928842323\\
1601	-1.01942739337092\\
1602	-0.726369925934368\\
1603	-0.342622945863351\\
1604	-0.380997753552908\\
1605	0.355060720761396\\
1606	-0.0808006822886164\\
1607	0.355511557382532\\
1608	0.66891177068808\\
1609	0.392617606081871\\
1610	1.07253039860398\\
1611	1.01507533130722\\
1612	0.999401947290213\\
1613	1.07598099495711\\
1614	0.987290151279991\\
1615	0.737473158268287\\
1616	0.219250376460677\\
1617	0.0578483854649711\\
1618	-0.11317131325112\\
1619	-0.720951312775561\\
1620	-0.889021434019631\\
1621	-0.925412525611976\\
1622	-0.928620446371991\\
1623	-0.596893741211995\\
1624	-0.802931855972786\\
1625	-0.997164403989132\\
1626	-0.894511806735561\\
1627	-0.214729383264339\\
1628	0.28754580066434\\
1629	-0.0308054155149913\\
1630	0.0851066958762942\\
1631	0.123583122422086\\
1632	0.875361198995899\\
1633	0.89470024920067\\
1634	1.20679511890413\\
1635	0.794317203139781\\
1636	1.03517837807078\\
1637	0.751780054640542\\
1638	1.03904256617622\\
1639	0.434616021891023\\
1640	0.304603203986673\\
1641	-0.00956014139569045\\
1642	0.0301613445006604\\
1643	-0.343830504141366\\
1644	-0.8818949285903\\
1645	-0.971449529154113\\
1646	-1.18000914659332\\
1647	-0.884499880286685\\
1648	-0.907058391257964\\
1649	-1.26411316325831\\
1650	-1.18044539023475\\
1651	-0.567001367202862\\
1652	-0.168146168472583\\
1653	-0.0305002931195979\\
1654	-0.153202671358988\\
1655	0.09244458986703\\
1656	0.851213136938941\\
1657	0.797335746171592\\
1658	0.859568252758055\\
1659	1.29220556386146\\
1660	0.803110635090964\\
1661	0.928450925920997\\
1662	0.847230709301702\\
1663	0.741432929490426\\
1664	0.132977029520736\\
1665	0.14823522876472\\
1666	0.42278272594035\\
1667	-0.778278251174567\\
1668	-0.560063771584117\\
1669	-0.881284410536944\\
1670	-1.13108574451222\\
1671	-1.18384689912807\\
1672	-0.869923189365158\\
1673	-0.379953181707526\\
1674	-0.931375974273798\\
1675	-0.436905515478945\\
1676	-0.449269747814116\\
1677	0.374039444128968\\
1678	0.314150349711868\\
1679	0.408845192107997\\
1680	0.795894451178207\\
1681	0.828617876902688\\
1682	1.19376915832801\\
1683	0.956135361508385\\
1684	0.730392804736907\\
1685	0.95329223178888\\
1686	1.10306190172679\\
1687	0.290182011827185\\
1688	0.26993166920375\\
1689	0.29313435739928\\
1690	-0.621950805282363\\
1691	-0.299743866049569\\
1692	-0.488682395146475\\
1693	-0.782093994608532\\
1694	-1.1183633855808\\
1695	-0.999263140613668\\
1696	-0.960489506501159\\
1697	-0.688852273328186\\
1698	-0.605252887862286\\
1699	-0.636335737639657\\
1700	0.0383312777467554\\
1701	-0.0796501724443882\\
1702	0.0104020829763199\\
1703	0.341596346652108\\
1704	0.849318218021773\\
1705	0.873552439231841\\
1706	1.14490987420022\\
1707	1.01315695309202\\
1708	1.53784141434439\\
1709	0.525405807923009\\
1710	0.660658357208019\\
1711	0.539113267544905\\
1712	0.344136207435406\\
1713	0.394130138312979\\
1714	-0.143157608232152\\
1715	-0.465412709035261\\
1716	-0.641827420597706\\
1717	-0.661698322934993\\
1718	-1.10028046217618\\
1719	-0.896783883094815\\
1720	-1.33156256030154\\
1721	-0.915774856867674\\
1722	-0.964487736684867\\
1723	-0.704057755596026\\
1724	-0.260939337212853\\
1725	-0.320043994459639\\
1726	-0.0290450341742239\\
1727	0.100488024766077\\
1728	0.158485088436585\\
1729	0.79595979815465\\
1730	1.20619899693093\\
1731	0.95695301067536\\
1732	0.928988786398174\\
1733	0.663904735484536\\
1734	1.10804806941514\\
1735	0.502147458426288\\
1736	0.613737930707213\\
1737	0.149278386303282\\
1738	-0.251132386666837\\
1739	-0.197544235007649\\
1740	-0.437330978891967\\
1741	-0.854806020555725\\
1742	-0.655994338057533\\
1743	-0.599049906255817\\
1744	-0.996025945559429\\
1745	-0.955358015019094\\
1746	-0.663184776088925\\
1747	-0.479245156798025\\
1748	-0.143459362837342\\
1749	-0.125470604095802\\
1750	0.352294151956101\\
1751	0.674359770807411\\
1752	0.69691088241112\\
1753	0.864172009674602\\
1754	0.66530943681146\\
1755	0.735218199969582\\
1756	0.654302529674888\\
1757	0.619696291999269\\
1758	0.719026347258228\\
1759	0.19657697034674\\
1760	0.826785986804703\\
1761	0.560989639657866\\
1762	0.466440095140645\\
1763	0.660078419170061\\
1764	0.34575090445614\\
1765	-0.778860541645027\\
1766	-1.03205077301069\\
1767	-0.674299689778095\\
1768	-0.406411190639767\\
1769	0.380506938087454\\
1770	0.705000485834946\\
1771	0.794097307878881\\
1772	0.0672881087162032\\
1773	-0.314051059441087\\
1774	-1.03033059691267\\
1775	-0.743141788155984\\
1776	-0.0928704703867329\\
1777	0.636608422513951\\
1778	0.921339256752094\\
1779	0.805660895905524\\
1780	0.0958283097513059\\
1781	-0.369034092638327\\
1782	-0.714425828946218\\
1783	-0.994735858947852\\
1784	-0.408523234667881\\
1785	0.533469142618914\\
1786	1.1224630969097\\
1787	0.60136692456447\\
1788	0.414483284655551\\
1789	-0.351731989420461\\
1790	-0.96295625049967\\
1791	-0.718656280219352\\
1792	-0.451236372560225\\
1793	0.210670166995608\\
1794	0.990901542467633\\
1795	1.02133578185295\\
1796	0.321064756352064\\
1797	-0.217398247876604\\
1798	-1.26883774076389\\
1799	-1.15898673945045\\
1800	-1.05722478100465\\
1801	0.512233998354032\\
1802	0.70736556929479\\
1803	0.861798485408195\\
1804	0.595590710685888\\
1805	-0.3841385191383\\
1806	-1.02769563930135\\
1807	-1.10831248808807\\
1808	-0.423528186551738\\
1809	0.530476975884613\\
1810	0.58983005198633\\
1811	0.903147774126461\\
1812	0.544821403305263\\
1813	-0.71348276125381\\
1814	-0.779346027219975\\
1815	-0.765548642873433\\
1816	-0.567141885065099\\
1817	0.0588762933794552\\
1818	1.02606353035681\\
1819	1.09931350181405\\
1820	0.499292447647723\\
1821	-0.224844239166723\\
1822	-1.1936713308141\\
1823	-0.787789814983376\\
1824	-0.49953227085872\\
1825	0.269931691409279\\
1826	0.580958813839763\\
1827	1.16464643225863\\
1828	0.408912603843817\\
1829	-0.170966687267617\\
1830	-0.591003880202971\\
1831	-1.17293310185161\\
1832	-0.225844113739105\\
1833	0.392698520396261\\
1834	1.13372821778899\\
1835	1.1663216696119\\
1836	0.553342598400966\\
1837	-0.578518415449067\\
1838	-0.957440725621985\\
1839	-0.859333073087562\\
1840	0.337074450863608\\
1841	0.670045286693863\\
1842	1.44844572349095\\
1843	1.01119871256036\\
1844	0.417141693021282\\
1845	-0.258339999110008\\
1846	-1.04724144152537\\
1847	-0.640598296088381\\
1848	-0.405012666427211\\
1849	0.0881705412769035\\
1850	1.12401137661844\\
1851	0.567437726589196\\
1852	-0.0429670967602522\\
1853	-0.331013855988197\\
1854	-1.14193003671027\\
1855	-1.12224557699312\\
1856	-0.40597905379545\\
1857	0.286147940741714\\
1858	0.879282009392236\\
1859	0.979325545967142\\
1860	0.108245384170669\\
1861	-0.669512038578214\\
1862	-1.39895548735441\\
1863	-1.00577089819265\\
1864	-0.221085816534156\\
1865	0.238401924230523\\
1866	1.04132261190449\\
1867	1.03935368170682\\
1868	0.591447602171299\\
1869	-0.174151778938396\\
1870	-1.08328425282663\\
1871	-1.11018511554107\\
1872	-0.35890953356301\\
1873	0.246773636671307\\
1874	1.25422930342577\\
1875	1.22722342270121\\
1876	0.39102775451003\\
1877	-0.491136074199362\\
1878	-0.814827846372469\\
1879	-1.06530308233304\\
1880	-0.397714354762842\\
1881	0.276846575891042\\
1882	0.952764005478354\\
1883	1.17094401338649\\
1884	0.535889298739368\\
1885	-0.13963943433823\\
1886	-1.13382992988891\\
1887	-0.878206880922856\\
1888	-0.589790928059374\\
1889	0.481670554234841\\
1890	1.00202030542988\\
1891	1.20852994733937\\
1892	0.615722825713356\\
1893	-0.249376267183295\\
1894	-1.04216085561704\\
1895	-1.02496022185745\\
1896	-0.828834837194301\\
1897	0.295232753005417\\
1898	0.582010563884307\\
1899	1.12651834505388\\
1900	0.571573550680655\\
1901	-0.364608107066285\\
1902	-1.01381573156323\\
1903	-0.886427947542199\\
1904	-0.0105640551105204\\
1905	0.697457967204728\\
1906	0.828783988047214\\
1907	0.95612968109543\\
1908	0.118994435334572\\
1909	-0.305878738564509\\
1910	-0.763749178285048\\
1911	-0.717031522546757\\
1912	-0.712392900488438\\
1913	0.508477551992755\\
1914	0.735742772482043\\
1915	0.834177603323674\\
1916	0.546078491475989\\
1917	-0.328821753283564\\
1918	-0.965777978625018\\
1919	-0.782668661972622\\
1920	-0.00397151918144761\\
1921	0.690342304884384\\
1922	0.643876836875029\\
1923	0.737581280551357\\
1924	0.770032728799607\\
1925	0.74793897435869\\
1926	0.882783560128108\\
1927	0.0912091635071816\\
1928	0.376967276672449\\
1929	-0.105919665134803\\
1930	0.0537922887486532\\
1931	-0.631163010384505\\
1932	-0.655487983034798\\
1933	-1.03967931382706\\
1934	-1.09851990776014\\
1935	-1.05588032938351\\
1936	-1.03254233312716\\
1937	-1.01558416829119\\
1938	-0.501240216116543\\
1939	-0.434851294632139\\
1940	0.0909797757757713\\
1941	-0.130328469054751\\
1942	0.430956430089935\\
1943	-0.0300602635146957\\
1944	0.800058563118728\\
1945	0.912993403584695\\
1946	0.862267558043387\\
1947	1.15327099691856\\
1948	1.17773770086443\\
1949	1.12944134527181\\
1950	0.84767488629267\\
1951	0.692467724761605\\
1952	0.173332253346938\\
1953	-0.176803294555556\\
1954	-0.252153763166754\\
1955	-0.39795949798328\\
1956	-0.44437765126146\\
1957	-1.00676878155344\\
1958	-1.0555360321485\\
1959	-0.332041382913446\\
1960	-1.12066431458923\\
1961	-0.46841824357725\\
1962	-0.538649280880347\\
1963	-0.682785682819739\\
1964	-0.348008303364821\\
1965	-0.116423382691955\\
1966	0.298223836232053\\
1967	0.745307521592426\\
1968	0.290672932739831\\
1969	1.28273546962853\\
1970	1.12005408450104\\
1971	0.874833760114823\\
1972	0.953143993446292\\
1973	0.965181816266512\\
1974	0.820011279876998\\
1975	0.826671131759015\\
1976	0.568121409777018\\
1977	-0.011995508862137\\
1978	-0.221341817399929\\
1979	-0.275510433851484\\
1980	-0.630251527479969\\
1981	-0.575579962413411\\
1982	-1.09968656597325\\
1983	-0.892113774039981\\
1984	-0.916817680117838\\
1985	-0.586673911328785\\
1986	-1.05482961796618\\
1987	-0.557793722697207\\
1988	-0.312192462945685\\
1989	-0.592344290403371\\
1990	0.471564369538753\\
1991	1.06835303444389\\
1992	0.605221095365062\\
1993	0.894098991084858\\
1994	0.842032287809149\\
1995	1.0836688515379\\
1996	0.896345233564247\\
1997	0.875366130628308\\
1998	1.11584280308812\\
1999	0.36836861157917\\
2000	0.19529284880952\\
2001	0.0512870126700457\\
2002	-0.221911102930005\\
2003	-0.117618114535244\\
2004	-0.202624975983039\\
2005	-1.04876829139668\\
2006	-1.08944879374695\\
2007	-0.881194962405665\\
2008	-1.0665354193007\\
2009	-0.452183302498574\\
2010	-0.387660731401595\\
2011	-0.432864216124572\\
2012	-0.198033156787244\\
2013	-0.144000887505572\\
2014	0.363155068959434\\
2015	0.692003643640565\\
2016	0.688859893923481\\
2017	1.26134335707193\\
2018	1.09607584292325\\
2019	0.973025957600804\\
2020	1.0643563167429\\
2021	1.14353697883195\\
2022	0.664633757110978\\
2023	0.538836241332126\\
2024	0.85525567361643\\
2025	-0.0377788465565485\\
2026	0.0195046073926211\\
2027	-0.256618093025629\\
2028	-0.301600701258473\\
2029	-0.712156680848786\\
2030	-1.49766083642053\\
2031	-1.18844451454312\\
2032	-0.866827000992137\\
2033	-0.840457328626897\\
2034	-0.314579581996165\\
2035	-0.243257051479441\\
2036	-0.21298246469552\\
2037	-0.0497390353367962\\
2038	-0.154250302132686\\
2039	0.856641817182241\\
2040	0.687171582116854\\
2041	0.483904078239647\\
2042	1.04296063292956\\
2043	1.02070555994065\\
2044	0.853711152491478\\
2045	1.00847899358273\\
2046	0.595407659820049\\
2047	0.256301108466485\\
2048	-0.0684125838492715\\
2049	-0.0131380645170927\\
2050	-0.332597275691648\\
2051	-0.34120913759368\\
2052	-0.993438944163403\\
2053	-0.876465933638105\\
2054	-1.28281673985536\\
2055	-0.59268432965196\\
2056	-1.00751872513475\\
2057	-0.623741995301925\\
2058	-0.642976404677614\\
2059	-0.35463641167209\\
2060	-0.582453569657291\\
2061	0.213139891696294\\
2062	0.517313164200666\\
2063	-0.0250414871288009\\
2064	0.617684889659512\\
2065	0.683944802610003\\
2066	0.981030074006159\\
2067	0.807403766761344\\
2068	1.04378735841189\\
2069	1.03175908853711\\
2070	0.808496842019934\\
2071	0.531477497459943\\
2072	0.035620911902154\\
2073	0.0811563279961456\\
2074	-0.308840633115031\\
2075	-0.725925101224828\\
2076	-0.669761702330863\\
2077	-0.876789258411037\\
2078	-0.903012756384821\\
2079	-0.775603299185584\\
2080	-0.852391282176536\\
2081	0.687761106726386\\
2082	0.76593254936163\\
2083	0.724968972060655\\
2084	0.348496872795967\\
2085	-0.0860195469817563\\
2086	-0.637399291049513\\
2087	-0.946596882577157\\
2088	-0.503306105152043\\
2089	0.0552650419942619\\
2090	1.07948010111987\\
2091	0.923554817747659\\
2092	0.106173705825685\\
2093	-0.470425562012858\\
2094	-1.33852986578153\\
2095	-0.822220045026215\\
2096	-0.408674314968553\\
2097	0.760591959372755\\
2098	0.624924354749256\\
2099	1.13562188219741\\
2100	0.401881801207357\\
2101	-0.377068992273948\\
2102	-0.768761524225062\\
2103	-0.651365963556197\\
2104	-0.609419512614578\\
2105	0.690696739298095\\
2106	0.658362524268043\\
2107	0.678078245481551\\
2108	0.219233300382687\\
2109	-0.230749086558253\\
2110	-1.168570844713\\
2111	-0.843540484580508\\
2112	-0.27756748122669\\
2113	0.430042833233631\\
2114	1.05323732061565\\
2115	0.655984356052796\\
2116	0.595215080322359\\
2117	-0.37015048255537\\
2118	-0.765765825948895\\
2119	-1.13853215182074\\
2120	-0.358094633124862\\
2121	0.59194237496461\\
2122	1.15607957165083\\
2123	0.899567328540376\\
2124	-0.0450434717835119\\
2125	-0.640999534829835\\
2126	-0.880211673021154\\
2127	-0.740707097901475\\
2128	-0.266100680693331\\
2129	0.258460138847457\\
2130	1.43138580732041\\
2131	0.934632627189835\\
2132	0.56342226711694\\
2133	-0.488578145074259\\
2134	-1.17863585103763\\
2135	-0.76509755109682\\
2136	-0.1375286014725\\
2137	0.803326860242447\\
2138	0.810759388917332\\
2139	0.55413776521161\\
2140	0.731656738715162\\
2141	-0.336445661950877\\
2142	-1.18136776448446\\
2143	-1.04357462899669\\
2144	-0.550740874013162\\
2145	0.655362479491019\\
2146	1.06163677432\\
2147	0.972764603295378\\
2148	0.423552051572057\\
2149	-0.579388595493139\\
2150	-0.706898206643054\\
2151	-0.840926832750832\\
2152	-0.428174442808967\\
2153	0.354219058961141\\
2154	0.906369803465037\\
2155	0.719941000495797\\
2156	0.535642930497179\\
2157	-0.0840068874959378\\
2158	-0.375528721567277\\
2159	-0.805676133311103\\
2160	-0.553792191339976\\
2161	0.440529499854165\\
2162	0.703778635786019\\
2163	0.781227744872821\\
2164	0.222275956047767\\
2165	-0.582013284931384\\
2166	-0.844507339454322\\
2167	-1.16803313685863\\
2168	-0.220423521170169\\
2169	0.573532833841068\\
2170	1.1613049963313\\
2171	0.913870667940375\\
2172	0.508246991874806\\
2173	0.00793402964355605\\
2174	-0.917277176005433\\
2175	-0.958250831073312\\
2176	0.0269063939537935\\
2177	-0.165863934903227\\
2178	0.826960842923431\\
2179	1.46231051026876\\
2180	0.264236624949792\\
2181	-0.520943773523928\\
2182	-1.20359377342887\\
2183	-0.740997000881079\\
2184	-0.841898841194537\\
2185	0.216666007081298\\
2186	1.05601316537308\\
2187	0.918119851770682\\
2188	0.509359527802229\\
2189	-0.367259132925614\\
2190	-0.77677197123354\\
2191	-1.20653955401578\\
2192	-0.32093537125665\\
2193	0.0449889369950179\\
2194	1.10441812111816\\
2195	0.848943678818949\\
2196	0.516386135453565\\
2197	-0.515827929438673\\
2198	-0.855592438575462\\
2199	-0.68256343497229\\
2200	-0.812753598464096\\
2201	-0.0223285491222457\\
2202	0.519093017342097\\
2203	1.10833994664556\\
2204	0.460410661610513\\
2205	-0.552748951813213\\
2206	-0.59368076037846\\
2207	-0.887607127416213\\
2208	-0.724835719973203\\
2209	0.16749045651734\\
2210	0.619149573300358\\
2211	1.13212639588901\\
2212	0.235220028189709\\
2213	-0.657798655127349\\
2214	-1.05828121477631\\
2215	-1.10561843647696\\
2216	-0.178861431229201\\
2217	0.502370707247071\\
2218	0.996788896025664\\
2219	1.01148477612938\\
2220	0.37935507941849\\
2221	-0.607887501110369\\
2222	-0.807935324691999\\
2223	-1.10219410878236\\
2224	-0.609608187444855\\
2225	0.155194661308321\\
2226	1.12372374269608\\
2227	1.13798157326217\\
2228	-0.0238501992433811\\
2229	-0.428694736300138\\
2230	-0.872756668573168\\
2231	-1.05503408223052\\
2232	-0.19944681059894\\
2233	0.45248996429046\\
2234	0.894934985963302\\
2235	0.793705822388434\\
2236	0.521240630981849\\
2237	-0.205498910636613\\
2238	-0.824114820583249\\
2239	-0.972100883351572\\
2240	-0.163640887074714\\
2241	0.192612827162097\\
2242	0.961458766072541\\
2243	1.23092993818271\\
2244	0.327287586399829\\
2245	-0.584386812429481\\
2246	-0.494235099385267\\
2247	-0.973698629650527\\
2248	-0.270913059599909\\
2249	0.21572247411346\\
2250	0.99349696132076\\
2251	1.02295164457075\\
2252	0.370759002570497\\
2253	-0.322581735583159\\
2254	-0.659574607059852\\
2255	-1.23265456004273\\
2256	-0.323737376206338\\
2257	-0.0500341070697515\\
2258	0.935493299520338\\
2259	1.35045179870819\\
2260	0.580344980121204\\
2261	-0.397103499618012\\
2262	-1.00482739418542\\
2263	-0.80022788557912\\
2264	-0.133226139116616\\
2265	0.398852594279341\\
2266	1.29048584255235\\
2267	0.735871429837663\\
2268	-0.0773185868269239\\
2269	-0.562983573459445\\
2270	-0.97373605314109\\
2271	-1.00185866791719\\
2272	-0.559362189540582\\
2273	0.694799916214794\\
2274	0.567437905205684\\
2275	0.808384229503505\\
2276	0.152828749356817\\
2277	-0.501649409346217\\
2278	-1.07820953347188\\
2279	-0.811392472598272\\
2280	-0.602705291582645\\
2281	0.211907575597109\\
2282	0.951766668166749\\
2283	1.16101541014474\\
2284	-0.0452662799429108\\
2285	0.00719315850348734\\
2286	-1.12569780989241\\
2287	-0.773493677210795\\
2288	-0.134867461030504\\
2289	0.691634635117035\\
2290	0.813361834435843\\
2291	0.923883920684021\\
2292	0.333403892048717\\
2293	-0.531088262120655\\
2294	-0.868332998764467\\
2295	-0.838185832097301\\
2296	-0.448578980792399\\
2297	0.425173341357636\\
2298	1.52742777388978\\
2299	0.916749612649766\\
2300	0.528689181900319\\
2301	-0.499060607579765\\
2302	-1.04726402974224\\
2303	-0.995822671000033\\
2304	-0.24340196279875\\
2305	0.216186220602135\\
2306	0.514485155758277\\
2307	0.930351376444485\\
2308	0.272077592493362\\
2309	-0.332894123739684\\
2310	-0.910202941314258\\
2311	-0.755281114486441\\
2312	-0.168168466558015\\
2313	0.098903769296563\\
2314	0.587967051027607\\
2315	1.03087683596919\\
2316	0.438302215857704\\
2317	-0.605718462661794\\
2318	-0.652231203222299\\
2319	-0.911055364942346\\
2320	-0.342086198959061\\
2321	0.456855867635536\\
2322	1.09790254057913\\
2323	1.25959346609729\\
2324	0.311113501715036\\
2325	-0.782844799519512\\
2326	-0.702343814926264\\
2327	-1.12605071827881\\
2328	-0.0914285186120258\\
2329	0.263638684760705\\
2330	1.26289845625558\\
2331	0.738081666561836\\
2332	0.120761762838703\\
2333	-0.181410849084018\\
2334	-1.23415847010697\\
2335	-0.766142032394844\\
2336	-0.271762420735339\\
2337	0.363493665160036\\
2338	0.852708126705132\\
2339	1.37893781332765\\
2340	0.285925085627793\\
2341	-0.438117457757483\\
2342	-1.1392046940522\\
2343	-1.10047864162655\\
2344	-0.500339905319574\\
2345	0.778110049999141\\
2346	0.84176816635493\\
2347	0.764020516617885\\
2348	0.269435771373365\\
2349	-0.235691434684352\\
2350	-1.22356916550138\\
2351	-0.778105906297919\\
2352	-0.667654209745022\\
2353	0.366272866533801\\
2354	0.520314739564447\\
2355	1.03837100476272\\
2356	0.524786871946707\\
2357	-0.54941323713032\\
2358	-0.858069146936995\\
2359	-0.882002025417837\\
2360	-0.110352261982927\\
2361	0.424284519853016\\
2362	1.16174500852266\\
2363	0.892311798152097\\
2364	0.224662891816282\\
2365	-0.404053352878444\\
2366	-0.97374070732939\\
2367	-0.72114342658072\\
2368	-0.382302080831123\\
2369	0.20730225064224\\
2370	0.894485153852759\\
2371	1.1844254222625\\
2372	0.276890620080534\\
2373	-0.542938637259813\\
2374	-0.708939398593169\\
2375	-0.895790141879193\\
2376	-0.450904202885652\\
2377	0.230219703406219\\
2378	0.597978127729865\\
2379	0.952764138899095\\
2380	0.298796315892573\\
2381	-0.324186703260128\\
2382	-1.25508069953163\\
2383	-0.622643769184282\\
2384	-0.260712136868427\\
2385	0.0449756947639322\\
2386	1.16568385588403\\
2387	0.786955095006013\\
2388	0.437369099919514\\
2389	-0.346442849855111\\
2390	-1.12565905043744\\
2391	-1.55876428767185\\
2392	-0.249892111867952\\
2393	0.319013140714628\\
2394	0.795347639385734\\
2395	1.01218666334849\\
2396	0.215402058718512\\
2397	-0.245355405466191\\
2398	-0.895368270786516\\
2399	-1.14027921882296\\
2400	-0.182473132904066\\
2401	0.936052391280682\\
2402	1.22620965987108\\
2403	0.797916234064247\\
2404	0.780999018754321\\
2405	0.968257776807517\\
2406	0.903002450049678\\
2407	0.743276177215165\\
2408	0.325620124990151\\
2409	0.0154870360363162\\
2410	-0.046178989558428\\
2411	-8.42037653515093e-05\\
2412	-0.459227745769321\\
2413	-0.810949580713104\\
2414	-1.07456533189584\\
2415	-0.88583279352635\\
2416	-1.08967518519328\\
2417	-0.844782817108985\\
2418	-0.579578648075665\\
2419	-0.351127985139472\\
2420	-0.192550494310072\\
2421	0.24007814568324\\
2422	0.606527752516758\\
2423	0.0150478092879028\\
2424	0.259146825457481\\
2425	0.623399335204238\\
2426	0.498906693628658\\
2427	0.917755014478377\\
2428	0.906063236858527\\
2429	0.954182086896042\\
2430	0.817771411283431\\
2431	0.404410772228137\\
2432	0.142160676879227\\
2433	0.245821719799994\\
2434	-0.277738101617806\\
2435	-0.746552452275737\\
2436	-0.654872628280092\\
2437	-0.7841866959625\\
2438	-1.03798000550551\\
2439	-0.84917541167376\\
2440	-0.930817375371785\\
2441	-0.945771820143866\\
2442	-0.763588368284369\\
2443	-0.455429447457228\\
2444	-0.422943480842003\\
2445	-0.30434984483157\\
2446	0.157450215397225\\
2447	0.468606174006564\\
2448	0.771178648782408\\
2449	0.823126170131848\\
2450	0.843652345810907\\
2451	1.13111402108727\\
2452	0.955937409161727\\
2453	1.07873719431824\\
2454	0.733638371520065\\
2455	0.55655293030552\\
2456	0.303606983353954\\
2457	-0.137879113795235\\
2458	-0.3996991319806\\
2459	-0.654806702406678\\
2460	-0.25777155507074\\
2461	-0.874665304247149\\
2462	-0.73555657508905\\
2463	-0.919532653346282\\
2464	-0.565274579966092\\
2465	-0.690682615238133\\
2466	-1.02830325037298\\
2467	-0.631190266369637\\
2468	-0.124631629639798\\
2469	-0.00378486879607304\\
2470	0.699170186107727\\
2471	0.428988721254206\\
2472	0.309904432194192\\
2473	0.929243763995023\\
2474	0.741955041922523\\
2475	1.00872899664484\\
2476	1.22258687249403\\
2477	0.967105948138542\\
2478	0.820787202384495\\
2479	0.513827602441201\\
2480	0.29139719990002\\
2481	0.0452096769572956\\
2482	-0.818620466781345\\
2483	-0.35169880141254\\
2484	-0.43558257634393\\
2485	-1.12025034955816\\
2486	-1.28658582989288\\
2487	-0.858697184888557\\
2488	-1.31869479656171\\
2489	-1.27395712112874\\
2490	-0.577177943510792\\
2491	-0.704375713227111\\
2492	-0.0455801733020876\\
2493	-0.35745659199676\\
2494	0.0547874641255156\\
2495	0.49379192151272\\
2496	0.780112095190884\\
2497	0.934513987494896\\
2498	0.988845157123741\\
2499	0.946241001493392\\
2500	0.85172606601214\\
2501	1.0056636329449\\
2502	0.66716927876198\\
2503	0.652309735135879\\
2504	0.567569353051619\\
2505	-0.435078746194258\\
2506	0.211538268109399\\
2507	-0.581147498735045\\
2508	-0.362583350756097\\
2509	-0.482653268231068\\
2510	-1.27471205146549\\
2511	-0.958071443111818\\
2512	-0.205666082027206\\
2513	-0.94617095512977\\
2514	-0.644914461958584\\
2515	-0.483915792737591\\
2516	-0.0404334613173568\\
2517	0.175759532450319\\
2518	-0.019529167581102\\
2519	0.341578621350295\\
2520	0.733984844473951\\
2521	0.816797729387803\\
2522	1.51327550291666\\
2523	1.09582624819643\\
2524	1.07925003959494\\
2525	0.710687569306687\\
2526	0.461142554041431\\
2527	0.735673541595971\\
2528	0.280019965435243\\
2529	0.213541638073283\\
2530	-0.378333313678499\\
2531	-0.239209758699543\\
2532	-1.06299963268808\\
2533	-0.479276128903388\\
2534	-0.756571894554144\\
2535	-0.973016200997249\\
2536	-1.24117299429704\\
2537	-0.987102166900672\\
2538	-0.69682859096563\\
2539	-0.287682908836733\\
2540	-0.555448526037896\\
2541	0.221552025929778\\
2542	0.811714886228926\\
2543	0.328857224142982\\
2544	0.762079162113514\\
2545	0.68277786255378\\
2546	0.561781879943064\\
2547	1.32673803827313\\
2548	1.13667613838785\\
2549	0.783858646964354\\
2550	0.779108080202001\\
2551	0.506534933309629\\
2552	-0.00294181708454566\\
2553	-0.576731883876741\\
2554	-0.202888985301642\\
2555	-0.543195805150013\\
2556	-0.78622053739946\\
2557	-1.01917724979197\\
2558	-1.35266045594192\\
2559	-1.11441897664821\\
2560	-0.945216725685909\\
2561	-1.13575308623641\\
2562	-0.65292325385821\\
2563	-0.619007491765421\\
2564	-0.346866850598747\\
2565	-0.388078870376708\\
2566	0.0212078472134283\\
2567	0.531485799809387\\
2568	0.480408759063721\\
2569	0.866038203308695\\
2570	1.38861470915066\\
2571	1.1756305825462\\
2572	1.21348372961564\\
2573	1.42633455693122\\
2574	1.01676324174035\\
2575	0.646966982649268\\
2576	0.760812222272496\\
2577	-0.134168426636463\\
2578	-0.344918148574158\\
2579	0.0119459529982566\\
2580	-0.555506700825947\\
2581	-0.839289223996575\\
2582	-0.843472795606177\\
2583	-1.05677649892279\\
2584	-1.04668288772495\\
2585	-0.984699325308919\\
2586	-0.877706406247789\\
2587	-0.769147590242856\\
2588	-0.337556258552288\\
2589	-0.196877243583537\\
2590	0.514923848191678\\
2591	0.492411544078042\\
2592	0.885741976069324\\
2593	0.798576032478076\\
2594	0.927751640995072\\
2595	1.27923591624316\\
2596	1.30996996539661\\
2597	0.905513449784377\\
2598	1.13247563519769\\
2599	0.664419098940563\\
2600	0.406632855589783\\
2601	0.558798798776884\\
2602	0.1513875494191\\
2603	-0.236679677111717\\
2604	-0.567161922564405\\
2605	-0.950345842451072\\
2606	-0.971783897385831\\
2607	-0.589940969254801\\
2608	-1.43371370461558\\
2609	-1.04068423493808\\
2610	-0.795984571361962\\
2611	-0.524780116895994\\
2612	-0.72335860648422\\
2613	0.0501194885519856\\
2614	0.355565485654246\\
2615	0.229323003579174\\
2616	0.599156213764155\\
2617	0.558301181713374\\
2618	1.30954138491014\\
2619	1.08125083560609\\
2620	0.925198358593898\\
2621	0.691044258344691\\
2622	0.775812472933411\\
2623	0.934867651849717\\
2624	0.385760793045916\\
2625	0.147333575932358\\
2626	-0.460885723184491\\
2627	-0.402529447324662\\
2628	-0.483490418120998\\
2629	-1.1390035982929\\
2630	-0.693515779597782\\
2631	-0.445952631862492\\
2632	-1.32997206369551\\
2633	-1.36061462244334\\
2634	-0.769980416458736\\
2635	-0.388087679859497\\
2636	-0.342438216370627\\
2637	-0.103737801316796\\
2638	0.193720850390098\\
2639	0.201468571329521\\
2640	0.289305338509461\\
2641	1.18378502996735\\
2642	1.0114820131899\\
2643	0.73190454128558\\
2644	1.22211489137309\\
2645	0.607817287069185\\
2646	0.72648471440248\\
2647	0.723730186206745\\
2648	0.13602842655887\\
2649	-0.0099294855955071\\
2650	-0.185635711983197\\
2651	-0.00280514562684286\\
2652	-0.554459335379414\\
2653	-0.919224863234008\\
2654	-0.996437454920444\\
2655	-1.06038802705754\\
2656	-0.688925021648097\\
2657	-0.961779193648523\\
2658	-0.930902634016472\\
2659	-0.504115491421361\\
2660	-0.345343280185477\\
2661	0.109234578726128\\
2662	0.40696926724952\\
2663	0.6829764993149\\
2664	0.802002072595228\\
2665	0.971418713310395\\
2666	1.37191459312225\\
2667	0.738077050190428\\
2668	0.90362214155357\\
2669	0.518821881438454\\
2670	0.959473761659804\\
2671	0.562779213553447\\
2672	0.25717736528262\\
2673	-0.31685263967626\\
2674	-0.151218501130886\\
2675	-0.497828244759233\\
2676	-0.960352711589405\\
2677	-1.08887854981773\\
2678	-1.1165177195324\\
2679	-1.05367837784105\\
2680	-1.35030147265468\\
2681	-0.938867675498889\\
2682	-0.61595639847248\\
2683	-0.642252816776251\\
2684	-0.480369208872338\\
2685	-0.000218533962268869\\
2686	0.438665430463707\\
2687	0.339764035013208\\
2688	0.552738761026341\\
2689	0.50856049362924\\
2690	0.727208616813644\\
2691	1.13325311990482\\
2692	0.955909195300149\\
2693	1.06923489150543\\
2694	0.903643419302794\\
2695	0.64205352309152\\
2696	0.495724383219793\\
2697	0.409195988523012\\
2698	-0.0631855098884721\\
2699	-0.344381707247168\\
2700	-0.666786276638757\\
2701	-1.10785972170228\\
2702	-1.05899872091353\\
2703	-0.92474783183713\\
2704	-1.23005605988266\\
2705	-0.793372093555829\\
2706	-0.681540321975993\\
2707	-0.535578266432467\\
2708	-0.256708716495904\\
2709	-0.0629435535373388\\
2710	0.187331336238879\\
2711	0.437745736016739\\
2712	0.865171880949965\\
2713	0.809630323316491\\
2714	0.354967772456274\\
2715	0.888876332548296\\
2716	1.02519661712786\\
2717	1.11290777313747\\
2718	1.06262213749082\\
2719	0.347281063914035\\
2720	0.494429885905688\\
2721	0.121516416955332\\
2722	-0.238896356659788\\
2723	-0.434169461229762\\
2724	-0.663465424143155\\
2725	-0.798457743887295\\
2726	-0.892834246818194\\
2727	-1.14794514303152\\
2728	-1.21519189953161\\
2729	-0.682023555986553\\
2730	-0.80291127307061\\
2731	-0.241417374725827\\
2732	-0.530977169082374\\
2733	-0.201231101457785\\
2734	0.510318507739588\\
2735	0.754698790542258\\
2736	0.42691032886471\\
2737	0.876080314545664\\
2738	1.24611379543537\\
2739	1.29590070816902\\
2740	0.693810088401073\\
2741	1.26030072064893\\
2742	0.901796416345399\\
2743	0.608836365388555\\
2744	0.273684254370791\\
2745	-0.147894383417543\\
2746	-0.171003711286226\\
2747	-0.497156953065338\\
2748	-0.140220028143464\\
2749	-0.787335788299029\\
2750	-0.984805270501286\\
2751	-1.00080591209241\\
2752	-1.11846075182769\\
2753	-0.434516387184504\\
2754	-0.611181655512116\\
2755	-0.541201277570866\\
2756	-0.0727997304883231\\
2757	-0.0756881274655712\\
2758	0.170392829084147\\
2759	0.315145662792883\\
2760	0.519443390099099\\
2761	1.18893270665982\\
2762	1.11797431452107\\
2763	1.01712045119978\\
2764	1.3627427091236\\
2765	0.903430737275798\\
2766	0.818639879857381\\
2767	0.240217169249123\\
2768	0.430519786889404\\
2769	-0.25560500693064\\
2770	-0.413423161315958\\
2771	-0.227254255216806\\
2772	-0.578582626417211\\
2773	-1.02506559771642\\
2774	-1.03509362738109\\
2775	-0.940281555740227\\
2776	-0.932105601706024\\
2777	-1.2811266562584\\
2778	-1.02355720772847\\
2779	-0.954702048731899\\
2780	-0.340385600909365\\
2781	-0.453424367492425\\
2782	0.349333095478197\\
2783	0.214702825790366\\
2784	0.409583991200893\\
2785	1.23055712255745\\
2786	1.42989751832408\\
2787	0.814329345142066\\
2788	0.973051780918752\\
2789	1.33742129694788\\
2790	0.632834012557123\\
2791	0.94230247903883\\
2792	0.507619467391529\\
2793	0.0806991902045883\\
2794	-0.271288805311294\\
2795	-0.0608141053156296\\
2796	-0.498060981298958\\
2797	-0.971466549794802\\
2798	-0.858117106973166\\
2799	-1.19413602866604\\
2800	-0.948011827445006\\
2801	-0.555339628103195\\
2802	-0.634717921696216\\
2803	-0.727958621950656\\
2804	-0.246291896539616\\
2805	-0.224428466679243\\
2806	0.644147226176031\\
2807	0.0383130586379294\\
2808	0.509595195942448\\
2809	1.00250234318881\\
2810	0.762425859984919\\
2811	1.00166955603002\\
2812	1.12781402790783\\
2813	1.17958870795478\\
2814	0.546273197743408\\
2815	0.519945900460232\\
2816	0.21861302602918\\
2817	0.0427117157012974\\
2818	-0.109583270591555\\
2819	-0.578138765770953\\
2820	-0.478130156085956\\
2821	-0.734381516468496\\
2822	-1.05169884068901\\
2823	-0.93591430527062\\
2824	-0.761093223052008\\
2825	-0.675905256691228\\
2826	-0.604203095057338\\
2827	-0.57250626736071\\
2828	-0.462928616482189\\
2829	-0.0181249287450512\\
2830	0.0879551738318081\\
2831	0.365391464907103\\
2832	0.864133473191875\\
2833	1.15368129763194\\
2834	1.07933586570872\\
2835	1.45714565053774\\
2836	0.639128396669057\\
2837	0.903218979531239\\
2838	0.775057102139943\\
2839	0.919258273288899\\
2840	0.436337958083874\\
2841	-0.0041938939051003\\
2842	-0.0682446152744103\\
2843	-0.295940140958724\\
2844	-0.616311532438698\\
2845	-1.28357408418893\\
2846	-0.717056913905588\\
2847	-1.77900074030977\\
2848	-0.956841210661321\\
2849	-1.24529757974639\\
2850	-0.833200761045093\\
2851	-0.244993002005957\\
2852	-0.439068462874402\\
2853	0.360750003872754\\
2854	0.366458763193472\\
2855	0.38455406784895\\
2856	0.454987695109236\\
2857	0.718894612022122\\
2858	1.02980519277105\\
2859	0.467190946362134\\
2860	1.11898074606044\\
2861	0.960680403878763\\
2862	0.936979266393903\\
2863	1.01615020677656\\
2864	0.251716255523708\\
2865	-0.15115011365501\\
2866	0.0717107815185018\\
2867	-0.239516628543711\\
2868	-0.68916411175627\\
2869	-0.962673505333217\\
2870	-0.890298602006863\\
2871	-1.25048826299694\\
2872	-0.966552110576413\\
2873	-1.04089618367491\\
2874	-0.937078301707617\\
2875	-0.34135318081511\\
2876	-0.317372009395449\\
2877	0.23407241661246\\
2878	-0.0508411606071334\\
2879	0.493471878804987\\
2880	0.423065686325951\\
2881	0.968323282813297\\
2882	1.23804049387288\\
2883	1.22670249679992\\
2884	0.507964263973236\\
2885	1.05069866571833\\
2886	0.755683304413496\\
2887	0.632764370794714\\
2888	0.20166393314443\\
2889	-0.301181051522768\\
2890	0.00752177660518508\\
2891	-0.553174026895268\\
2892	-0.730979597552387\\
2893	-0.7921931210175\\
2894	-0.378331601877341\\
2895	-0.48440077822681\\
2896	-1.27643149710984\\
2897	-1.00424343143106\\
2898	-0.532778868165932\\
2899	-1.02186102559202\\
2900	-0.0457764030569855\\
2901	-0.138818825185326\\
2902	0.505399517446142\\
2903	0.242806485573802\\
2904	0.450749139059685\\
2905	1.34778225162152\\
2906	1.0775861583189\\
2907	0.928347022850788\\
2908	0.750186888865199\\
2909	1.14504883969849\\
2910	0.830291188826091\\
2911	0.0513265456898812\\
2912	0.12202697127293\\
2913	0.732905277363642\\
2914	-0.19489977689946\\
2915	-0.887160688482047\\
2916	-0.855898514177853\\
2917	-0.923832579999704\\
2918	-0.79056893634377\\
2919	-1.03408721059\\
2920	-0.958611973454126\\
2921	-0.902138324661074\\
2922	-0.968635510313813\\
2923	-0.295511551067958\\
2924	-0.405903887216399\\
2925	-0.17197641670814\\
2926	0.235901321983277\\
2927	0.424724347643916\\
2928	0.540471661829934\\
2929	0.858309461113705\\
2930	1.41885050900479\\
2931	1.11879396641019\\
2932	0.834748294062673\\
2933	0.703559592700143\\
2934	0.700273988277356\\
2935	0.205287907014406\\
2936	0.732728503574966\\
2937	0.113985584849687\\
2938	-0.452610422308071\\
2939	-0.412715617542951\\
2940	-0.980208505612585\\
2941	-0.699550296479332\\
2942	-0.986609259861227\\
2943	-0.988265641951415\\
2944	-0.900370030383475\\
2945	-0.707098459303244\\
2946	-0.934944427091923\\
2947	-0.886256297129713\\
2948	-0.347413293869773\\
2949	-0.252425628308463\\
2950	0.287368696371141\\
2951	0.0246659246399138\\
2952	0.849310406822584\\
2953	0.898431691527187\\
2954	0.697256755206635\\
2955	0.926674701924747\\
2956	1.16883888055491\\
2957	1.44113426695804\\
2958	0.675167860933026\\
2959	0.867215192945106\\
2960	0.131566754859543\\
2961	0.114572556358902\\
2962	-0.238700276567867\\
2963	-0.363563154785715\\
2964	-0.768646323327577\\
2965	-1.14794564733458\\
2966	-1.00659467997506\\
2967	-0.899279126611921\\
2968	-1.0946071181896\\
2969	-0.939395260581988\\
2970	-0.663775362122004\\
2971	-0.437292950552076\\
2972	-0.488326335555552\\
2973	-0.128330836366477\\
2974	0.387088483123182\\
2975	0.513722878440367\\
2976	0.355548076950814\\
2977	1.17624858689567\\
2978	0.808010642080645\\
2979	0.841340952943336\\
2980	0.662771838154059\\
2981	0.873608539567254\\
2982	0.730749524093557\\
2983	0.485743736034431\\
2984	0.136281350408889\\
2985	-0.303321870891777\\
2986	0.481864895352593\\
2987	-0.103848806438563\\
2988	-0.737514054629038\\
2989	-0.855574783578925\\
2990	-0.865686489966699\\
2991	-1.0003826409059\\
2992	-1.10897731624403\\
2993	-0.689124486738109\\
2994	-1.00846589069\\
2995	-0.119971918597277\\
2996	-0.718737439118264\\
2997	-0.14063026104118\\
2998	0.387534646104798\\
2999	0.46609258107801\\
3000	0.320826928922343\\
3001	1.03615466968951\\
3002	0.982531880233366\\
3003	0.815360223723735\\
3004	1.07168459774328\\
3005	1.06735421960489\\
3006	1.24256635932278\\
3007	0.536974400688359\\
3008	0.25832630210184\\
3009	-0.0446901697168998\\
3010	-0.62348888557284\\
3011	-0.32303146621379\\
3012	-1.11691356411084\\
3013	-0.890251190666473\\
3014	-1.00754751011536\\
3015	-0.808540797794659\\
3016	-1.60147527368762\\
3017	-0.692695970480025\\
3018	-0.869932635025422\\
3019	-0.847310112682075\\
3020	-0.32611441156288\\
3021	-0.0160553267753921\\
3022	0.0973405814176081\\
3023	0.110873633180723\\
3024	0.92715090875079\\
3025	0.874028257713194\\
3026	0.787817085708044\\
3027	1.26986415995513\\
3028	1.06350778272973\\
3029	0.614498445336408\\
3030	0.892712394093011\\
3031	0.156175711240519\\
3032	0.0874410877599442\\
3033	0.102187278431762\\
3034	-0.100234410557302\\
3035	-0.423929644503217\\
3036	-1.08975023076158\\
3037	-0.646654288343653\\
3038	-0.663646169178665\\
3039	-0.736439585050204\\
3040	-1.10897065520771\\
3041	0.307407896537563\\
3042	0.70145547484566\\
3043	0.974433594396438\\
3044	0.426158479653523\\
3045	-0.568194829240244\\
3046	-0.723360891482964\\
3047	-0.727747668636796\\
3048	-0.222687083317775\\
3049	0.226944358263614\\
3050	0.62960720059392\\
3051	1.06439710346787\\
3052	0.693595096315298\\
3053	-0.684257218437068\\
3054	-1.42328262842897\\
3055	-0.597071719259793\\
3056	-0.383539161360158\\
3057	0.659122454177412\\
3058	1.14967330708766\\
3059	0.645750330703629\\
3060	0.273092862061368\\
3061	-0.582396387367197\\
3062	-0.888927315982833\\
3063	-0.657448601840911\\
3064	-0.357209724167777\\
3065	-0.074391582720541\\
3066	1.20556967950597\\
3067	0.919040574407847\\
3068	0.266903466636091\\
3069	-0.866200977359213\\
3070	-0.845460104140691\\
3071	-0.863576461958897\\
3072	-0.293971251646032\\
3073	0.154305469135065\\
3074	0.765740882406573\\
3075	0.713387824275362\\
3076	0.769755301815074\\
3077	-0.579143138958618\\
3078	-0.973155864846212\\
3079	-1.117638333497\\
3080	-0.165931678251947\\
3081	0.392809245021121\\
3082	0.94543033446807\\
3083	1.12866450544988\\
3084	0.520842232866526\\
3085	-0.21967060036308\\
3086	-1.40639882724857\\
3087	-0.869227079811421\\
3088	-0.153503673537274\\
3089	0.473875328586799\\
3090	1.13094341110131\\
3091	1.39312043867369\\
3092	0.649145845721417\\
3093	-0.646899491160784\\
3094	-1.1148916875648\\
3095	-0.945694282666554\\
3096	-0.513536676357386\\
3097	0.377886597204265\\
3098	1.20396197869657\\
3099	0.799534158754718\\
3100	0.305713926548205\\
3101	-0.446487347107308\\
3102	-0.778191766022696\\
3103	-0.686419590497869\\
3104	-0.403184802719343\\
3105	0.324597765042012\\
3106	1.13805497576186\\
3107	1.0802986257259\\
3108	0.478983781777988\\
3109	0.335353153788812\\
3110	-0.92204200507705\\
3111	-0.842072534892787\\
3112	-0.388430701476378\\
3113	0.533496126650359\\
3114	0.811652110807491\\
3115	1.14394647903443\\
3116	0.779053114545594\\
3117	-0.376468262242387\\
3118	-0.944425620410641\\
3119	-1.04453454026842\\
3120	-0.47571419973404\\
3121	0.0143384812096307\\
3122	0.917114547405003\\
3123	0.721216335167177\\
3124	0.650780500447763\\
3125	-0.360679865528887\\
3126	-0.916843185607569\\
3127	-0.605475895216315\\
3128	-0.767091256291162\\
3129	0.469323650722185\\
3130	0.692399412564995\\
3131	0.774470152094167\\
3132	0.427571864395028\\
3133	-0.196011377438447\\
3134	-1.11635460723243\\
3135	-0.794659709979373\\
3136	-0.0433766427548598\\
3137	0.204204406931792\\
3138	0.707119638234424\\
3139	0.91582465703326\\
3140	0.510027747615593\\
3141	-0.395218458461555\\
3142	-1.11181285917942\\
3143	-0.892675127088371\\
3144	-0.323400564133472\\
3145	0.296070297597156\\
3146	0.678393754601831\\
3147	1.01586676800878\\
3148	0.373562289176071\\
3149	-0.476782559728367\\
3150	-1.07039374549504\\
3151	-1.03040503159304\\
3152	-0.563797942341656\\
3153	0.162802811489951\\
3154	1.21231312535252\\
3155	0.864409513421221\\
3156	-0.0171829345689012\\
3157	-0.504812071521\\
3158	-0.69184409276162\\
3159	-1.03887036936553\\
3160	0.02799286679269\\
3161	0.22111391578337\\
3162	0.760636572072709\\
3163	0.687249056362262\\
3164	0.630180979278872\\
3165	-0.525369581408595\\
3166	-0.835752408242307\\
3167	-1.28725872380236\\
3168	-0.497891023980591\\
3169	0.543076323895867\\
3170	0.642608818987061\\
3171	0.929659458701019\\
3172	0.492459341163591\\
3173	0.00242530402405283\\
3174	-0.785674043648146\\
3175	-0.706590285164351\\
3176	-0.44901308356381\\
3177	0.266026748862745\\
3178	0.941773522741935\\
3179	0.718535768370382\\
3180	0.62083581785016\\
3181	-0.221939666042107\\
3182	-0.713749541876248\\
3183	-0.636305584498314\\
3184	-0.393888121017856\\
3185	0.266398138758715\\
3186	1.04676157916559\\
3187	1.17622727949491\\
3188	0.56182548375851\\
3189	-0.183953733845032\\
3190	-0.592095416825519\\
3191	-0.852352274486407\\
3192	-0.100286446397066\\
3193	0.392903107772734\\
3194	1.10580967432027\\
3195	0.60435033929697\\
3196	0.714829358487141\\
3197	-0.792741045491052\\
3198	-0.883114332034395\\
3199	-0.995917294212169\\
3200	-0.629094380697888\\
3201	0.143425711286557\\
3202	-0.117917662820142\\
3203	-0.123906376328669\\
3204	-0.626552155414181\\
3205	-0.700606066713868\\
3206	-0.891603058071626\\
3207	-0.669232388470788\\
3208	-0.728348317013209\\
3209	-0.895758018620001\\
3210	-0.725911065153235\\
3211	-0.15976142705382\\
3212	-0.377752380268937\\
3213	0.123726021431957\\
3214	-0.00429074228505588\\
3215	0.269197961294826\\
3216	0.850251982958416\\
3217	0.974944347643873\\
3218	1.18278004224406\\
3219	0.855971537237622\\
3220	1.20914749097433\\
3221	0.93661535137483\\
3222	0.743110640185628\\
3223	0.935956478122248\\
3224	0.645744048945146\\
3225	-0.435469554926834\\
3226	-0.112491898594501\\
3227	-0.6299718901604\\
3228	-0.724212491990273\\
3229	-0.971798784950067\\
3230	-0.582476946775067\\
3231	-0.719940961355777\\
3232	-0.857082509803716\\
3233	-0.74073179254254\\
3234	-0.651206957481895\\
3235	-0.404899635171829\\
3236	-0.267930121190912\\
3237	-0.108220232575891\\
3238	0.305885169395619\\
3239	0.370356623005186\\
3240	0.568927391355125\\
3241	0.76950533131419\\
3242	0.926279382644053\\
3243	0.999463338321975\\
3244	1.54790572283162\\
3245	0.624405707974732\\
3246	0.513898933332982\\
3247	0.613023765905738\\
3248	0.312837692082834\\
3249	-0.0951176844153402\\
3250	-0.119093831306087\\
3251	-0.271201927121477\\
3252	-0.915358023995976\\
3253	-0.921139320379935\\
3254	-1.23420765492287\\
3255	-0.978963612284879\\
3256	-0.944687841067309\\
3257	-0.745395998989427\\
3258	-0.996840748829945\\
3259	-0.49138839132573\\
3260	-0.356232880963678\\
3261	0.243215390998786\\
3262	0.260910385405219\\
3263	0.319494965007819\\
3264	0.605948963639011\\
3265	0.709872252838408\\
3266	1.26802400237494\\
3267	0.883293111388739\\
3268	0.621766469333468\\
3269	0.853107985588962\\
3270	1.00962654726752\\
3271	0.706391676819387\\
3272	0.360882140799714\\
3273	0.467811275778111\\
3274	-0.323784287827913\\
3275	-0.381738658408723\\
3276	-0.852277365995352\\
3277	-1.00618646269247\\
3278	-1.02305712283007\\
3279	-0.8527433658535\\
3280	-0.521672012593088\\
3281	-0.719355485613653\\
3282	-0.993419131339929\\
3283	-0.603870493264814\\
3284	-0.448236279357784\\
3285	-0.133829337816107\\
3286	0.456378009948408\\
3287	0.524158716433541\\
3288	0.879072867333973\\
3289	0.964907124286952\\
3290	0.768569710198529\\
3291	1.20596420716432\\
3292	0.679153980808631\\
3293	1.07711816598191\\
3294	0.887920690107098\\
3295	0.80421399238318\\
3296	0.0064548960319275\\
3297	0.0227052787777786\\
3298	-0.188258154871695\\
3299	-0.417249020007483\\
3300	-0.461068903907703\\
3301	-0.673066447988857\\
3302	-0.800093554802108\\
3303	-0.957715179361532\\
3304	-0.60628655855095\\
3305	-0.970891995758633\\
3306	-0.731099576168955\\
3307	-0.687012661148833\\
3308	-0.337932191181886\\
3309	0.0940966110455287\\
3310	-0.0823867598037447\\
3311	0.482431860136557\\
3312	0.492742205204816\\
3313	1.00629118862753\\
3314	0.506189551758558\\
3315	0.560061049241879\\
3316	1.34482569641628\\
3317	0.86470488243006\\
3318	0.313277027060633\\
3319	0.180638595245461\\
3320	0.304712224601463\\
3321	0.137061378062249\\
3322	-0.11673106225799\\
3323	-0.200935091204685\\
3324	-0.58267385826946\\
3325	-0.846311294787715\\
3326	-0.711534964523263\\
3327	-0.536936011181624\\
3328	-0.975369291166978\\
3329	-0.951966289155432\\
3330	-0.52134527843419\\
3331	-0.671467410936858\\
3332	-0.656045768386732\\
3333	-0.0921045586472413\\
3334	0.247151675489566\\
3335	0.472331015447154\\
3336	0.406975041274407\\
3337	0.69628421878605\\
3338	0.79394580900885\\
3339	1.21138780688005\\
3340	0.952072274528807\\
3341	1.04346507487503\\
3342	0.887669524214183\\
3343	0.514883369982468\\
3344	-0.119097975637918\\
3345	0.149492796305196\\
3346	-0.0347374974693068\\
3347	-0.608121066358048\\
3348	-0.660964998281895\\
3349	-0.928702413133135\\
3350	-0.565277856371541\\
3351	-1.00450730778088\\
3352	-0.935710168250705\\
3353	-1.13262003364767\\
3354	-0.918294888742807\\
3355	-0.861531372843671\\
3356	-0.161121238397686\\
3357	0.0676009588500546\\
3358	0.00691374361905142\\
3359	1.08590010066862\\
3360	0.862862542539\\
3361	0.920674304768681\\
3362	1.37480303767406\\
3363	0.448134927725852\\
3364	0.960964615187397\\
3365	0.89261909445803\\
3366	0.798475864492516\\
3367	0.325607825865333\\
3368	0.442624733358787\\
3369	-0.427870126576769\\
3370	-0.168229180484272\\
3371	-0.341486811504386\\
3372	-0.666352690448074\\
3373	-0.74834239718825\\
3374	-1.11690066421289\\
3375	-0.874694540703165\\
3376	-1.06517878617135\\
3377	-1.26942711666019\\
3378	-0.823157036354215\\
3379	-0.696691288205687\\
3380	-0.553445607305218\\
3381	0.0447333403331365\\
3382	0.201323763496278\\
3383	0.382702268693341\\
3384	0.534607354795276\\
3385	1.2050539207238\\
3386	1.2084946376763\\
3387	0.955358578143802\\
3388	1.00071585648784\\
3389	1.23662135542684\\
3390	1.15503090979398\\
3391	0.804704441374503\\
3392	0.251245624655338\\
3393	-0.0621470288262616\\
3394	-0.0240771884135555\\
3395	-0.632991290795675\\
3396	-0.896818825775148\\
3397	-0.909361768641534\\
3398	-0.732095177964014\\
3399	-0.962022644956748\\
3400	-0.926529841514518\\
3401	-0.646446470854545\\
3402	-0.701229519199938\\
3403	-0.812747496278986\\
3404	-0.135720503598599\\
3405	0.0383205747009995\\
3406	0.172486957718017\\
3407	0.347276824490949\\
3408	0.535312843442521\\
3409	0.824494992050012\\
3410	0.766862826995079\\
3411	0.883049040533745\\
3412	0.648655785938652\\
3413	0.782080743268715\\
3414	0.598604157229066\\
3415	0.687227387977648\\
3416	0.649754723604316\\
3417	0.299994683972644\\
3418	0.0419905222169101\\
3419	-0.288383771300757\\
3420	-0.534862755868711\\
3421	-0.662035375204947\\
3422	-1.10932380594829\\
3423	-0.811213236935505\\
3424	-1.24955327966455\\
3425	-0.759673384309851\\
3426	-0.940010925023865\\
3427	-0.775137459950099\\
3428	-0.425649735230038\\
3429	-0.359793536986104\\
3430	0.19584155187717\\
3431	0.688247920860225\\
3432	0.601374549158872\\
3433	0.940794866977457\\
3434	0.982261684827\\
3435	1.26884679836179\\
3436	0.945914982958375\\
3437	0.963702776946435\\
3438	1.30595732851255\\
3439	0.814464170543608\\
3440	0.115188885790266\\
3441	0.318860876930587\\
3442	-0.399089379844554\\
3443	-0.43748860679991\\
3444	-0.592424958712671\\
3445	-0.635178663846729\\
3446	-1.034899972989\\
3447	-0.877305209439735\\
3448	-0.585312571186575\\
3449	-1.32454974578619\\
3450	-0.994688107162793\\
3451	-0.290558144405291\\
3452	-0.389943013949958\\
3453	-0.076818183528877\\
3454	0.399023997758514\\
3455	0.937180264048764\\
3456	0.365842201207278\\
3457	0.626454008875957\\
3458	0.928658155362937\\
3459	0.997780781743117\\
3460	1.00525472362247\\
3461	0.828218932558317\\
3462	1.02666582215609\\
3463	0.354420120566975\\
3464	0.369809197865279\\
3465	0.289759607727803\\
3466	-0.108304891129952\\
3467	-0.578924438748861\\
3468	-0.630201529987814\\
3469	-0.900893797225009\\
3470	-1.08019780598706\\
3471	-0.694858950856317\\
3472	-1.02557660134366\\
3473	-1.02748674890363\\
3474	-0.922904547353392\\
3475	-0.530839185528484\\
3476	-0.658901565755306\\
3477	-0.436650788871861\\
3478	0.235433158873034\\
3479	0.151082063976244\\
3480	0.567701369626053\\
3481	1.00338620528261\\
3482	1.15235899570237\\
3483	1.34330144036108\\
3484	0.755423742053094\\
3485	1.08985498071623\\
3486	0.826526688950871\\
3487	0.838230630625107\\
3488	0.373430380523695\\
3489	0.250698789995585\\
3490	0.0627121813723014\\
3491	-0.531837896706279\\
3492	-0.760355967366505\\
3493	-1.12728052851184\\
3494	-0.965581790282398\\
3495	-0.918125161370533\\
3496	-1.14764164290264\\
3497	-0.882023009938543\\
3498	-0.833489983883861\\
3499	-0.654564587868614\\
3500	-0.149766103131781\\
3501	0.114401616163382\\
3502	0.449233199989943\\
3503	0.196914187984889\\
3504	0.626717897734378\\
3505	0.976463043418977\\
3506	0.656551488853507\\
3507	1.42883075010781\\
3508	1.033539578758\\
3509	0.776177347141776\\
3510	0.984856231122886\\
3511	0.642356266274738\\
3512	0.348612547801622\\
3513	-0.0839371577238804\\
3514	-0.587324693375193\\
3515	-0.655331923374602\\
3516	-0.80165833046802\\
3517	-0.728600915369099\\
3518	-0.808489199055266\\
3519	-1.17340045065434\\
3520	-0.796014158872175\\
3521	-0.7487879769778\\
3522	-0.781818828648171\\
3523	-0.770826144660578\\
3524	-0.124543551092451\\
3525	-0.0973037686014694\\
3526	-0.0345300943047692\\
3527	0.49616214639278\\
3528	0.774346689662232\\
3529	0.502784981072466\\
3530	1.36366369416858\\
3531	0.516753563161898\\
3532	0.615410870927551\\
3533	1.17205190758499\\
3534	0.826501507339716\\
3535	0.461979189240398\\
3536	0.326758447018275\\
3537	0.231928597247073\\
3538	-0.386512103065293\\
3539	-0.232002835523183\\
3540	-0.501134283650726\\
3541	-0.839970891841147\\
3542	-0.903212013277603\\
3543	-1.13837058131981\\
3544	-1.17303486292598\\
3545	-0.985127771950862\\
3546	-0.523575625392985\\
3547	-0.611605381273951\\
3548	-0.425227233666117\\
3549	-0.287414351975151\\
3550	-0.0613751207653362\\
3551	0.444655771771293\\
3552	0.889963109194865\\
3553	1.10544762370244\\
3554	1.07135753867022\\
3555	0.928510149984397\\
3556	0.899080584154083\\
3557	0.979963089531379\\
3558	0.963513565894559\\
3559	0.775976383423402\\
3560	-0.150007478730077\\
3561	0.0578769468098985\\
3562	-0.408612114062625\\
3563	-0.545161273697501\\
3564	-0.26887153702487\\
3565	-1.00386220323557\\
3566	-1.3227997933059\\
3567	-0.958051583471147\\
3568	-1.00919818105337\\
3569	-0.991134477360906\\
3570	-0.556929946325251\\
3571	-0.431273322418363\\
3572	-0.240041107495033\\
3573	-0.130251272656507\\
3574	0.270611468602868\\
3575	0.374517115774528\\
3576	0.698819608077341\\
3577	0.958898109982804\\
3578	1.02200928304597\\
3579	0.84710189206239\\
3580	0.861170251211237\\
3581	0.665376377847749\\
3582	0.976905767692601\\
3583	0.552130646685398\\
3584	0.177140776671625\\
3585	0.219940109963394\\
3586	-0.306580742796621\\
3587	-0.339899050451634\\
3588	-0.317788049811851\\
3589	-0.620016904585641\\
3590	-0.981842140847085\\
3591	-1.02314114965485\\
3592	-0.547854300169929\\
3593	-0.936363181724419\\
3594	-0.686339117889716\\
3595	-0.485011512521186\\
3596	-0.117204341145712\\
3597	-0.0200971025263423\\
3598	0.274225676104146\\
3599	0.511796226844539\\
3600	1.25892208188837\\
3601	1.02591311496193\\
3602	0.74986090418207\\
3603	0.725227326619643\\
3604	0.619975283053793\\
3605	1.30991358305565\\
3606	0.570475857996595\\
3607	0.779980211173875\\
3608	0.443620763909011\\
3609	-0.273374877591321\\
3610	0.260406940658009\\
3611	-0.316113271058692\\
3612	-0.249869339198263\\
3613	-0.7531905026666\\
3614	-0.905818962844453\\
3615	-1.22576870609372\\
3616	-0.972928730284381\\
3617	-0.867456640697195\\
3618	-1.0941141003688\\
3619	-0.562194531583012\\
3620	-0.287632285048392\\
3621	-0.2296532569827\\
3622	0.284151187405036\\
3623	0.168344048117025\\
3624	0.528787134214489\\
3625	0.85345176028068\\
3626	0.810593208090162\\
3627	1.09205538748503\\
3628	0.900153636674189\\
3629	0.702533241261481\\
3630	0.962205619822859\\
3631	0.692237866296321\\
3632	0.452022081379299\\
3633	0.221588148250421\\
3634	0.0336205236084486\\
3635	-0.217969391917477\\
3636	-0.649639983248753\\
3637	-0.930628222523626\\
3638	-0.970763641173183\\
3639	-0.72065960472143\\
3640	-1.2493987958542\\
3641	-1.02444155466814\\
3642	-0.520263658488432\\
3643	-0.368643447690071\\
3644	-0.435054142688453\\
3645	-0.114556273936317\\
3646	0.157402795800631\\
3647	0.378077236298858\\
3648	0.110431060359505\\
3649	0.734241253914575\\
3650	1.01516233988295\\
3651	1.54627791990319\\
3652	0.654037085363309\\
3653	0.755921038026671\\
3654	0.667473572034337\\
3655	0.347098927470075\\
3656	0.173387126335661\\
3657	-0.362330841052422\\
3658	-0.217423445517908\\
3659	-0.555737172144423\\
3660	-0.537461302802906\\
3661	-0.817020499501813\\
3662	-0.707493440845261\\
3663	-0.634979527729056\\
3664	-0.954074392186328\\
3665	-1.1751896831675\\
3666	-0.699104140692161\\
3667	-0.245059965143674\\
3668	-0.345528423714753\\
3669	0.0184670937682718\\
3670	-0.0239743092024576\\
3671	0.777429966649632\\
3672	0.747548112666574\\
3673	0.735400782165933\\
3674	0.57138941315192\\
3675	0.843789277788336\\
3676	0.75148004829772\\
3677	0.78791788499962\\
3678	0.831342813100923\\
3679	0.743552751104556\\
3680	0.358990804124105\\
3681	0.0870082024388582\\
3682	-0.108435303922096\\
3683	0.154025692472\\
3684	-0.582724779494643\\
3685	-0.798365613971612\\
3686	-0.592914532355304\\
3687	-0.977627638788774\\
3688	-1.13194241963055\\
3689	-0.689603667847379\\
3690	-0.833822751240361\\
3691	-0.512734845126472\\
3692	-0.128076940988007\\
3693	0.112527187326492\\
3694	-0.166525943155596\\
3695	-0.088872332591302\\
3696	0.500153382609152\\
3697	1.19952866323741\\
3698	0.998605472025058\\
3699	0.963707449326988\\
3700	0.946458088458412\\
3701	1.13046156849361\\
3702	0.827689725210813\\
3703	0.664436569139514\\
3704	-0.143647190748759\\
3705	0.0293294262927658\\
3706	-0.187972812702666\\
3707	-0.632161747333492\\
3708	-0.662314827667089\\
3709	-0.964653219658043\\
3710	-0.751335211825772\\
3711	-1.12540246900784\\
3712	-0.689382663329519\\
3713	-0.90768486179984\\
3714	-0.553333254909554\\
3715	-0.0466578836517842\\
3716	-0.309208911030615\\
3717	0.266183137302036\\
3718	-0.0477283886845118\\
3719	0.847732051093087\\
3720	0.386757998497464\\
3721	0.815233913078526\\
3722	0.87286192700796\\
3723	0.740510611579128\\
3724	0.688630272190117\\
3725	1.0427889900566\\
3726	0.442859591210438\\
3727	0.257944192277803\\
3728	0.576781482162027\\
3729	0.068654344792636\\
3730	-0.107301394843978\\
3731	-0.641404784256925\\
3732	-0.701217688033891\\
3733	-0.761180083865689\\
3734	-0.908527766359728\\
3735	-0.597531839786378\\
3736	-1.12099933207015\\
3737	-0.931700097543301\\
3738	-0.567501991492847\\
3739	-0.345797975235164\\
3740	-0.765844759541289\\
3741	-0.15596905030249\\
3742	0.0408328768592567\\
3743	0.43612041485526\\
3744	0.772464586554368\\
3745	0.929410779436893\\
3746	1.20131030684457\\
3747	1.09784822068728\\
3748	1.05193164898629\\
3749	1.10974562289249\\
3750	0.8496602068931\\
3751	0.261541592298808\\
3752	0.348739248473731\\
3753	-0.30201743060173\\
3754	0.0562785252337104\\
3755	-0.911401042109152\\
3756	-0.780221251477393\\
3757	-0.808773600016498\\
3758	-0.95455386927604\\
3759	-0.576306711821912\\
3760	-0.968931762272019\\
3761	-1.03551777474893\\
3762	-0.622372284921926\\
3763	-0.827997244659154\\
3764	-0.202129892107843\\
3765	-0.0121218037252161\\
3766	0.104378039363991\\
3767	0.300505576965336\\
3768	0.390775381416514\\
3769	0.409505556774942\\
3770	0.85111684310366\\
3771	1.17159707720026\\
3772	0.821580967473622\\
3773	1.25275125868862\\
3774	0.554948816707082\\
3775	1.03703394539718\\
3776	0.169268467488888\\
3777	0.0904954604600937\\
3778	0.128851498310046\\
3779	-0.163696138520053\\
3780	-0.826724256213424\\
3781	-0.664588813771115\\
3782	-1.19537186205926\\
3783	-1.37940445172732\\
3784	-0.833388371303965\\
3785	-1.285199756766\\
3786	-1.23219057811204\\
3787	-0.571398390779661\\
3788	-0.0140748250884938\\
3789	0.204246984649557\\
3790	-0.142294445960661\\
3791	0.448048757323822\\
3792	0.836844965632523\\
3793	1.04779529480692\\
3794	1.29526073825599\\
3795	1.35263609480338\\
3796	1.17296427644704\\
3797	0.579372405207527\\
3798	0.762618904812912\\
3799	0.249165556330129\\
3800	0.600099016519906\\
3801	0.458115121434777\\
3802	-0.640159552846608\\
3803	-0.252738788689669\\
3804	-0.733894807771839\\
3805	-0.936825129938529\\
3806	-1.14496341534807\\
3807	-0.714937487114998\\
3808	-0.895590283011635\\
3809	-0.924742007971877\\
3810	-0.660363143320622\\
3811	-0.331223082148296\\
3812	-0.518743350883494\\
3813	-0.246107691143567\\
3814	0.346339166161612\\
3815	0.794546083475773\\
3816	0.9369431552022\\
3817	0.803997535114886\\
3818	0.647978805471315\\
3819	0.473395563153904\\
3820	0.773997170010747\\
3821	1.15091560829062\\
3822	0.630851070296985\\
3823	0.621454209897041\\
3824	0.371779054754635\\
3825	-0.424657649618646\\
3826	0.08769730243883\\
3827	0.013381147172229\\
3828	-0.665883494530543\\
3829	-0.606405013235634\\
3830	-1.1154765503691\\
3831	-0.864261713772117\\
3832	-0.506211048138964\\
3833	-0.604605215571565\\
3834	-0.910366652296057\\
3835	-0.786129149072071\\
3836	-0.312277503521606\\
3837	-0.121314729008499\\
3838	0.178227721877245\\
3839	0.724153263512968\\
3840	0.706665373410634\\
3841	1.20224085422292\\
3842	0.9363875522922\\
3843	0.967851242222832\\
3844	1.15884427304625\\
3845	0.993532872326992\\
3846	0.706762695817758\\
3847	0.202564121821219\\
3848	0.0743407915481958\\
3849	0.206314070330485\\
3850	-0.0706507685411102\\
3851	-0.768190059930071\\
3852	-0.630346239079773\\
3853	-0.711408475708644\\
3854	-1.20899622412861\\
3855	-1.22104629258025\\
3856	-0.714140964652167\\
3857	-1.29058286412064\\
3858	-0.653810400644066\\
3859	-0.339091573714888\\
3860	-0.725009479772758\\
3861	-0.097408368505322\\
3862	-0.0707268619077012\\
3863	0.459837214147396\\
3864	0.562707305474319\\
3865	0.460422708255168\\
3866	1.08428956277511\\
3867	1.1310747824162\\
3868	0.842949338159132\\
3869	1.01626946645484\\
3870	0.452904010115969\\
3871	0.281646267741222\\
3872	0.0660477098374058\\
3873	0.0438094520306064\\
3874	-0.30957057182978\\
3875	-0.0119968766912575\\
3876	-0.673091267350949\\
3877	-0.704140158571098\\
3878	-0.817535809663922\\
3879	-0.914480342684939\\
3880	-0.938445617208983\\
3881	-0.619741676698046\\
3882	-1.01424901553506\\
3883	-0.192021749458784\\
3884	-0.170577226604356\\
3885	-0.325505901929608\\
3886	0.440696758941267\\
3887	0.704338667524901\\
3888	1.03924361429407\\
3889	0.73580936937642\\
3890	1.31285959624334\\
3891	1.05662530827647\\
3892	0.653845739708169\\
3893	0.724451107462656\\
3894	0.961734367570485\\
3895	0.959187497566207\\
3896	0.059398813949924\\
3897	0.0433959490952832\\
3898	-0.339237205743847\\
3899	-0.0190603096289872\\
3900	-0.857703419410567\\
3901	-0.531903997280574\\
3902	-1.08576750690493\\
3903	-0.823862030737449\\
3904	-1.27768304468189\\
3905	-0.562411371597838\\
3906	-1.24669803292423\\
3907	-0.485788917417279\\
3908	-0.229951475792856\\
3909	-0.251930571910819\\
3910	-0.0240065723951809\\
3911	0.576064987166213\\
3912	0.425296735462487\\
3913	0.823548220099869\\
3914	0.314996119180591\\
3915	0.531971099121265\\
3916	0.971327889718868\\
3917	1.14831696836164\\
3918	0.384609874449614\\
3919	0.709832534907712\\
3920	0.434782710079572\\
3921	-0.0634121989892618\\
3922	-0.197620565056376\\
3923	0.0506645385783735\\
3924	-0.970600020557219\\
3925	-0.722430941625917\\
3926	-0.899555566091831\\
3927	-0.925590704798476\\
3928	-0.982658635152935\\
3929	-0.835172970071838\\
3930	-0.809978827616885\\
3931	-0.936921859922774\\
3932	-0.454854117479344\\
3933	-0.0158113963398902\\
3934	0.250177680439135\\
3935	0.57479128494339\\
3936	1.15418491390922\\
3937	0.459707276165927\\
3938	1.09826274377469\\
3939	1.02806231835199\\
3940	0.972841906906306\\
3941	0.743032327875297\\
3942	0.725196913786225\\
3943	0.710211757308945\\
3944	0.419541415531549\\
3945	-0.291040529759236\\
3946	0.0105243271505312\\
3947	-0.383869520111095\\
3948	-0.233290482442737\\
3949	-1.07879431480671\\
3950	-1.10757895476186\\
3951	-1.38976938828971\\
3952	-0.980534345454834\\
3953	-0.675449451457971\\
3954	-0.860917695053573\\
3955	-0.674125026997399\\
3956	-0.390102184560673\\
3957	-0.104202505656488\\
3958	0.421308256004077\\
3959	0.443262487974671\\
3960	0.549170643347197\\
3961	0.97189409725446\\
3962	0.640668318207104\\
3963	1.13162629063832\\
3964	0.935132737803017\\
3965	1.37688750938218\\
3966	0.608543369177656\\
3967	0.652752079488746\\
3968	0.350539289362622\\
3969	-0.17358087962511\\
3970	0.00365787447055849\\
3971	-0.937900946902105\\
3972	-1.09267831383152\\
3973	-1.1229271398898\\
3974	-0.653829875152954\\
3975	-0.777474709815097\\
3976	-1.26137685446836\\
3977	-0.938367120376742\\
3978	-0.920093197683717\\
3979	-0.505766302569896\\
3980	0.146073625286726\\
3981	-0.264454409818372\\
3982	0.628364928933876\\
3983	0.335587373681165\\
3984	0.511293758506191\\
3985	0.966289789295847\\
3986	1.10468823028562\\
3987	1.06027906945087\\
3988	0.652956669017052\\
3989	0.743439588435373\\
3990	0.559496649844134\\
3991	0.443272441804632\\
3992	0.149148850939095\\
3993	0.163557513824954\\
3994	0.0340949254372442\\
3995	-0.495799702820903\\
3996	-0.533152757742212\\
3997	-0.660045503983731\\
3998	-0.916049882015202\\
3999	-1.05866012850738\\
4000	-1.10277144906389\\
};
\end{axis}
\end{tikzpicture}%
	\caption{Signal bruité}
	\label{fig:bruite}
\end{figure}

\section{Démodulation par filtrage}

On souhaite désormais reconstituer le signal de départ. Pour cela, nous allons procéder à un filtrage passe bas d'une part et passe haut d'autre part, avec une fréquence de coupure $F_c=\frac{F_0+F_1}{2}$. Nous ferons ensuite passer chacun des signaux filtrés par un détecteur d'énergie qui permettra de reproduire de signal binaire initial de manière fidèle.

\paragraph{Filtre passe haut}
Pour le passe haut, nous allons utiliser un filtre de réponse impulsionnelle suivante:
\[
h_\text{haut}(t) = \frac{2 F_c}{F_e} \sinc(2 F_c t)
\]
\\
et donc \[H_\text{haut}(f) = \TF(h_\text{haut}(t))
\]

\paragraph{Filtre passe bas}

\[
H_\text{bas}(f) = 1-H_\text{haut}(f)
\]

Les réponses des filtres sont les suivantes:

\begin{figure}[H]
	\centering
	\begin{minipage}{0.45\textwidth}
		\scalebox{0.5}{
		% This file was created by matlab2tikz.
%
%The latest updates can be retrieved from
%  http://www.mathworks.com/matlabcentral/fileexchange/22022-matlab2tikz-matlab2tikz
%where you can also make suggestions and rate matlab2tikz.
%
\definecolor{mycolor1}{rgb}{0.00000,0.44700,0.74100}%
%
\begin{tikzpicture}

\begin{axis}[%
width=4.521in,
height=3.548in,
at={(0.758in,0.499in)},
scale only axis,
xmin=-25000,
xmax=25000,
xlabel style={font=\color{white!15!black}},
xlabel={fréquence [Hz]},
ymin=-1,
ymax=1.5,
ylabel style={font=\color{white!15!black}},
ylabel={amplitude},
axis background/.style={fill=white},
title style={font=\bfseries},
title={Réponse fréquentielle du filtre passe-haut}
]
\addplot [color=mycolor1, forget plot]
  table[row sep=crcr]{%
-24000	-0.00284134051617002\\
-23812.3167155425	-0.00155955278751208\\
-23624.633431085	-2.70725868176669e-05\\
-23483.8709677419	-0.000548645708477125\\
-23202.3460410557	-0.00284926158565213\\
-23014.6627565982	-0.00159841876666178\\
-22826.9794721408	-3.38786740030628e-05\\
-22686.2170087977	-0.000526930340129184\\
-22357.7712609971	-0.00280235687750974\\
-22123.1671554252	-0.000636388762359275\\
-21982.4046920821	-1.17250419862103e-05\\
-21794.7214076246	-0.00149594479808002\\
-21607.0381231672	-0.00291377263056347\\
-21466.275659824	-0.00220800224269624\\
-21184.7507331378	-8.05644958745688e-06\\
-20997.0674486804	-0.0014915046558599\\
-20809.3841642229	-0.00297193384903949\\
-20668.6217008798	-0.00228540270109079\\
-20387.0967741935	-5.04584022564813e-06\\
-20199.4134897361	-0.00149549735579058\\
-20011.7302052786	-0.00304916612367379\\
-19870.9677419355	-0.00237929897048161\\
-19589.4428152493	-2.68563599092886e-06\\
-19448.6803519062	-0.000946292362641543\\
-19214.0762463343	-0.00314749735844089\\
-19073.3137829912	-0.00249197738230578\\
-18791.788856305	-9.90523403743282e-07\\
-18651.0263929619	-0.000949365567066707\\
-18416.42228739	-0.00326963812767644\\
-18275.6598240469	-0.00262637966443435\\
-17994.1348973607	0\\
-17853.3724340176	-0.000958257169259014\\
-17618.7683284457	-0.00341915822355077\\
-17478.0058651026	-0.00278628941669012\\
-17196.4809384164	2.17394699575379e-07\\
-17055.7184750733	-0.000973495734797325\\
-16821.1143695015	-0.00360073762931279\\
-16680.3519061584	-0.00297659391799243\\
-16398.8269794721	-4.42349119111896e-07\\
-16258.064516129	-0.000995847902231617\\
-16023.4604105572	-0.00382052583518089\\
-15882.6979472141	-0.00320365634615882\\
-15601.1730205279	-2.12810846278444e-06\\
-15460.4105571848	-0.00102639131364413\\
-15225.8064516129	-0.0040866629278753\\
-15085.0439882698	-0.00347585372946924\\
-14803.5190615836	-5.04711715620942e-06\\
-14662.7565982405	-0.00106662478719954\\
-14428.1524926686	-0.00441004937601974\\
-14287.3900293255	-0.00380437034618808\\
-14005.8651026393	-9.48428350966424e-06\\
-13865.1026392962	-0.00111863720303518\\
-13630.4985337243	-0.00480550939028035\\
-13536.6568914956	-0.00471772119635716\\
-13395.8944281525	-0.00256852132588392\\
-13255.1319648094	-0.000293328674160875\\
-13161.2903225806	-8.14525883470196e-05\\
-13020.5278592375	-0.00208657428447623\\
-12832.8445747801	-0.00529359817664954\\
-12739.0029325513	-0.00524460818633088\\
-12598.2404692082	-0.00290595429760288\\
-12457.4780058651	-0.000353494651790243\\
-12363.6363636364	-7.59781651140656e-05\\
-12222.8739002933	-0.00226834615023108\\
-12035.1906158358	-0.0059035022786702\\
-11941.348973607	-0.00590254162307247\\
-11800.5865102639	-0.00332739876830601\\
-11659.8240469208	-0.000429218944191234\\
-11565.9824046921	-7.07726976543199e-05\\
-11472.1407624633	-0.0013819257983414\\
-11237.5366568915	-0.00667787798010977\\
-11143.6950146628	-0.00673804868347361\\
-11002.9325513196	-0.00386319812969305\\
-10862.1700879765	-0.000526078558323206\\
-10768.3284457478	-6.57911659800448e-05\\
-10674.4868035191	-0.00152747882384574\\
-10439.8826979472	-0.00768130552023649\\
-10346.0410557185	-0.00782163147596293\\
-10252.1994134897	-0.00596938660601154\\
-10064.5161290323	-0.000652343453111826\\
-9970.67448680352	-6.09614253335167e-05\\
-9876.83284457478	-0.00172285567896324\\
-9642.22873900293	-0.00901591725778417\\
-9548.38709677419	-0.00926489897392457\\
-9454.54545454546	-0.00714169262573705\\
-9266.86217008798	-0.000820679717435269\\
-9173.02052785924	-5.61288979952224e-05\\
-9079.1788856305	-0.00199355611403007\\
-8844.57478005865	-0.0108523768176383\\
-8750.73313782991	-0.0112545511656208\\
-8656.89149560118	-0.00876042005620548\\
-8469.2082111437	-0.00105120532316505\\
-8375.36656891496	-5.08975972479675e-05\\
-8281.52492668622	-0.00238578128119116\\
-8046.92082111437	-0.0134970002036425\\
-8000	-0.0142886932881083\\
-7906.15835777126	-0.0130173601792194\\
-7812.31671554252	-0.00862627023161622\\
-7671.55425219941	-0.00137709997216007\\
-7624.63343108504	-0.000192498438991606\\
-7577.71260997068	-4.41264928667806e-05\\
-7530.79178885631	-0.00100295104130055\\
-7436.95014662757	-0.00579141289927065\\
-7296.18768328446	-0.0153626103710849\\
-7249.26686217009	-0.0175553090775793\\
-7202.34604105572	-0.0186720486417471\\
-7155.42521994135	-0.0185435081839387\\
-7108.50439882698	-0.0171649116418848\\
-7014.66275659824	-0.0114780104304373\\
-6920.8211143695	-0.0045559476930066\\
-6873.90029325513	-0.00185444155795267\\
-6826.97947214076	-0.000249052271101391\\
-6780.05865102639	-3.20851650030818e-05\\
-6733.13782991202	-0.00131866893207189\\
-6686.21700879766	-0.00402393187323469\\
-6592.37536656891	-0.0123980622156523\\
-6498.53372434018	-0.0212469569632958\\
-6451.61290322581	-0.0244012596158427\\
-6404.69208211144	-0.026077666792844\\
-6357.77126099707	-0.0260161226869968\\
-6310.8504398827	-0.0241844528973161\\
-6263.92961876833	-0.0207935577745957\\
-6076.24633431085	-0.0025713571048982\\
-6029.32551319648	-0.000269645413936814\\
-5982.40469208211	-0\\
-5935.48387096774	-0.00196774065625505\\
-5888.56304985337	-0.00608394503797172\\
-5841.64222873901	-0.0119613011738693\\
-5700.87976539589	-0.0328243405965623\\
-5653.95894428153	-0.0378992881851445\\
-5607.03812316716	-0.0407103125398862\\
-5560.11730205279	-0.0408008691847499\\
-5513.19648093842	-0.0380653591673763\\
-5466.27565982405	-0.0327877527997771\\
-5372.43401759531	-0.0175703652348602\\
-5325.51319648094	-0.00979060947429389\\
-5278.59237536657	-0.00353315567554091\\
-5231.6715542522	6.49135872663464e-05\\
-5184.75073313783	0.000134797119244467\\
-5137.82991202346	-0.0037837612144358\\
-5090.90909090909	-0.0116453659065883\\
-5043.98826979472	-0.0228611888487649\\
-4903.22580645161	-0.0637433610427252\\
-4856.30498533724	-0.0741516930283979\\
-4809.38416422287	-0.0801984987483593\\
-4762.46334310851	-0.0807672688897583\\
-4715.54252199414	-0.0754016910177597\\
-4668.62170087976	-0.0644537962907634\\
-4621.7008797654	-0.0491488528896298\\
-4527.85923753666	-0.014426898687816\\
-4480.93841642229	-0.00100546083558584\\
-4434.01759530792	0.00533527493826114\\
-4387.09677419355	0.0014395615326066\\
-4340.17595307918	-0.0152401761006331\\
-4293.25513196481	-0.0462804555172625\\
-4246.33431085044	-0.0919875972904265\\
-4199.41348973607	-0.151198593510344\\
-4152.4926686217	-0.221210061459715\\
-4105.57184750733	-0.297851811217697\\
-4058.65102639296	-0.375707748957211\\
-4011.73020527859	-0.448472158775985\\
-3964.80938416422	-0.50941537509425\\
-3917.88856304986	-0.551920965364843\\
-3870.96774193548	-0.570048086581664\\
-3824.04692082111	-0.559068617545563\\
-3777.12609970674	-0.515929512002913\\
-3730.20527859238	-0.439596565753163\\
-3683.28445747801	-0.331245926976408\\
-3636.36363636364	-0.194283220345824\\
-3589.44281524927	-0.0341857566781982\\
-3542.5219941349	0.141820606528199\\
-3495.60117302053	0.325223420804832\\
-3448.68035190616	0.506785543351725\\
-3401.75953079179	0.677159597824357\\
-3354.83870967742	0.827503768388851\\
-3307.91788856305	0.950048800230434\\
-3260.99706744868	1.03857193115255\\
-3214.07624633431	1.08874337239467\\
-3167.15542521994	1.09832377809289\\
-3120.23460410557	1.06720549494275\\
-3073.3137829912	0.997304746193549\\
-3026.39296187683	0.892324800373899\\
-2979.47214076246	0.757420334743074\\
-2932.55131964809	0.598799696403148\\
-2885.63049853372	0.423304094249033\\
-2838.70967741936	0.238000904770161\\
-2791.78885630499	0.0498226896379492\\
-2744.86803519062	-0.134724951087264\\
-2697.94721407625	-0.3097737947719\\
-2651.02639296188	-0.470219484359404\\
-2604.10557184751	-0.611771782150754\\
-2557.18475073314	-0.730944239243399\\
-2510.26392961877	-0.825004112215538\\
-2463.3431085044	-0.89190529501866\\
-2416.42228739003	-0.930225673757377\\
-2369.50146627566	-0.939126011850021\\
-2322.58064516129	-0.918340945438104\\
-2275.65982404692	-0.868204881917336\\
-2228.73900293255	-0.789707646559691\\
-2181.81818181818	-0.684567702923232\\
-2134.89736070381	-0.555305627251073\\
-2087.97653958944	-0.405297930567031\\
-2041.05571847507	-0.238791641222633\\
-1994.13489736071	-0.0608632529620081\\
-1900.29325513196	0.305520663005154\\
-1853.37243401759	0.480985937887453\\
-1806.45161290323	0.642428013168683\\
-1759.53079178886	0.783505841398437\\
-1712.60997067449	0.898526153785497\\
-1665.68914956012	0.98275605753588\\
-1618.76832844575	1.03268985293835\\
-1571.84750733138	1.04624763565153\\
-1524.92668621701	1.02288928178314\\
-1478.00586510264	0.963635151703784\\
-1431.08504398827	0.870993390686635\\
-1384.1642228739	0.748802068985242\\
-1337.24340175953	0.602001650429884\\
-1290.32258064516	0.436358621358522\\
-1243.40175953079	0.258164014034264\\
-1149.56011730205	-0.109888274149853\\
-1102.63929618768	-0.287145095044252\\
-1055.71847507331	-0.452191986380058\\
-1008.79765395894	-0.60002414306291\\
-961.876832844573	-0.726370878339367\\
-914.956011730206	-0.827743097313942\\
-868.035190615836	-0.901448280397744\\
-821.114369501465	-0.945585980800388\\
-774.193548387098	-0.95903637466472\\
-727.272727272728	-0.941451922135457\\
-680.351906158357	-0.893258152973431\\
-633.431085043987	-0.815664637873851\\
-586.51026392962	-0.710682120818092\\
-539.58944281525	-0.58113735326333\\
-492.668621700879	-0.430674080835161\\
-445.747800586509	-0.26372739301587\\
-398.826979472142	-0.0854595096789126\\
-304.985337243401	0.281477819997235\\
-258.064516129034	0.457450955709646\\
-211.143695014664	0.619823099190398\\
-164.222873900293	0.762449812162231\\
-117.302052785923	0.879785753681062\\
-70.3812316715557	0.967168451319594\\
-23.4604105571852	1.02106831050696\\
23.4604105571852	1.03928424894548\\
70.3812316715557	1.02106831050696\\
117.302052785923	0.967168451319594\\
164.222873900293	0.879785753681062\\
211.143695014664	0.762449812162231\\
258.064516129034	0.619823099190398\\
304.985337243401	0.457450955709646\\
351.906158357771	0.281477819997235\\
445.747800586509	-0.0854595096789126\\
492.668621700879	-0.26372739301587\\
539.58944281525	-0.430674080835161\\
586.51026392962	-0.58113735326333\\
633.431085043987	-0.710682120818092\\
680.351906158357	-0.815664637873851\\
727.272727272728	-0.893258152973431\\
774.193548387098	-0.941451922135457\\
821.114369501465	-0.95903637466472\\
868.035190615836	-0.945585980800388\\
914.956011730206	-0.901448280397744\\
961.876832844573	-0.827743097313942\\
1008.79765395894	-0.726370878339367\\
1055.71847507331	-0.60002414306291\\
1102.63929618768	-0.452191986380058\\
1149.56011730205	-0.287145095044252\\
1196.48093841642	-0.109888274149853\\
1290.32258064516	0.258164014034264\\
1337.24340175953	0.436358621358522\\
1384.1642228739	0.602001650429884\\
1431.08504398827	0.748802068985242\\
1478.00586510264	0.870993390686635\\
1524.92668621701	0.963635151703784\\
1571.84750733138	1.02288928178314\\
1618.76832844575	1.04624763565153\\
1665.68914956012	1.03268985293835\\
1712.60997067449	0.98275605753588\\
1759.53079178886	0.898526153785497\\
1806.45161290323	0.783505841398437\\
1853.37243401759	0.642428013168683\\
1900.29325513196	0.480985937887453\\
1947.21407624634	0.305520663005154\\
2041.05571847507	-0.0608632529620081\\
2087.97653958944	-0.238791641222633\\
2134.89736070381	-0.405297930567031\\
2181.81818181818	-0.555305627251073\\
2228.73900293255	-0.684567702923232\\
2275.65982404692	-0.789707646559691\\
2322.58064516129	-0.868204881917336\\
2369.50146627566	-0.918340945438104\\
2416.42228739003	-0.939126011850021\\
2463.3431085044	-0.930225673757377\\
2510.26392961877	-0.89190529501866\\
2557.18475073314	-0.825004112215538\\
2604.10557184751	-0.730944239243399\\
2651.02639296188	-0.611771782150754\\
2697.94721407625	-0.470219484359404\\
2744.86803519062	-0.3097737947719\\
2791.78885630499	-0.134724951087264\\
2838.70967741936	0.0498226896379492\\
2885.63049853372	0.238000904770161\\
2932.55131964809	0.423304094249033\\
2979.47214076246	0.598799696403148\\
3026.39296187683	0.757420334743074\\
3073.3137829912	0.892324800373899\\
3120.23460410557	0.997304746193549\\
3167.15542521994	1.06720549494275\\
3214.07624633431	1.09832377809289\\
3260.99706744868	1.08874337239467\\
3307.91788856305	1.03857193115255\\
3354.83870967742	0.950048800230434\\
3401.75953079179	0.827503768388851\\
3448.68035190616	0.677159597824357\\
3495.60117302053	0.506785543351725\\
3542.5219941349	0.325223420804832\\
3589.44281524927	0.141820606528199\\
3636.36363636364	-0.0341857566781982\\
3683.28445747801	-0.194283220345824\\
3730.20527859238	-0.331245926976408\\
3777.12609970674	-0.439596565753163\\
3824.04692082111	-0.515929512002913\\
3870.96774193548	-0.559068617545563\\
3917.88856304986	-0.570048086581664\\
3964.80938416422	-0.551920965364843\\
4011.73020527859	-0.50941537509425\\
4058.65102639296	-0.448472158775985\\
4105.57184750733	-0.375707748957211\\
4152.4926686217	-0.297851811217697\\
4199.41348973607	-0.221210061459715\\
4246.33431085044	-0.151198593510344\\
4293.25513196481	-0.0919875972904265\\
4340.17595307918	-0.0462804555172625\\
4387.09677419355	-0.0152401761006331\\
4434.01759530792	0.0014395615326066\\
4480.93841642229	0.00533527493826114\\
4527.85923753666	-0.00100546083558584\\
4574.78005865103	-0.014426898687816\\
4668.62170087976	-0.0491488528896298\\
4715.54252199414	-0.0644537962907634\\
4762.46334310851	-0.0754016910177597\\
4809.38416422287	-0.0807672688897583\\
4856.30498533724	-0.0801984987483593\\
4903.22580645161	-0.0741516930283979\\
4950.14662756598	-0.0637433610427252\\
4997.06744868035	-0.0505423156901088\\
5090.90909090909	-0.0228611888487649\\
5137.82991202346	-0.0116453659065883\\
5184.75073313783	-0.0037837612144358\\
5231.6715542522	0.000134797119244467\\
5278.59237536657	6.49135872663464e-05\\
5325.51319648094	-0.00353315567554091\\
5372.43401759531	-0.00979060947429389\\
5513.19648093842	-0.0327877527997771\\
5560.11730205279	-0.0380653591673763\\
5607.03812316716	-0.0408008691847499\\
5653.95894428153	-0.0407103125398862\\
5700.87976539589	-0.0378992881851445\\
5747.80058651026	-0.0328243405965623\\
5841.64222873901	-0.0189501734239457\\
5888.56304985337	-0.0119613011738693\\
5935.48387096774	-0.00608394503797172\\
5982.40469208211	-0.00196774065625505\\
6029.32551319648	-0\\
6076.24633431085	-0.000269645413936814\\
6123.16715542522	-0.0025713571048982\\
6217.00879765396	-0.011259289120062\\
6310.8504398827	-0.0207935577745957\\
6357.77126099707	-0.0241844528973161\\
6404.69208211144	-0.0260161226869968\\
6451.61290322581	-0.026077666792844\\
6498.53372434018	-0.0244012596158427\\
6545.45454545455	-0.0212469569632958\\
6733.13782991202	-0.00402393187323469\\
6780.05865102639	-0.00131866893207189\\
6826.97947214076	-3.20851650030818e-05\\
6873.90029325513	-0.000249052271101391\\
6920.8211143695	-0.00185444155795267\\
7014.66275659824	-0.00792913526674965\\
7108.50439882698	-0.0147026617487427\\
7155.42521994135	-0.0171649116418848\\
7202.34604105572	-0.0185435081839387\\
7249.26686217009	-0.0186720486417471\\
7296.18768328446	-0.0175553090775793\\
7390.0293255132	-0.012399882034515\\
7530.79178885631	-0.00299187833661563\\
7577.71260997068	-0.00100295104130055\\
7624.63343108504	-4.41264928667806e-05\\
7671.55425219941	-0.000192498438991606\\
7765.39589442815	-0.00339285389054567\\
7953.07917888563	-0.0130173601792194\\
8046.92082111437	-0.0142886932881083\\
8140.76246334311	-0.0118694593002147\\
8375.36656891496	-0.000822044759843266\\
8469.2082111437	-0.000142476936161984\\
8563.04985337243	-0.0026248142348777\\
8750.73313782991	-0.0103235307251452\\
8844.57478005865	-0.0114359155340935\\
8938.41642228739	-0.00959054988197749\\
9173.02052785924	-0.000707572002283996\\
9266.86217008798	-0.000103662288893247\\
9360.70381231671	-0.00208974762426806\\
9548.38709677419	-0.00845874003789504\\
9642.22873900293	-0.00945708349536289\\
9736.07038123167	-0.00800656017963775\\
10017.5953079179	-6.09614253335167e-05\\
10111.4369501466	-0.000652343453111826\\
10252.1994134897	-0.0045593090908369\\
10346.0410557185	-0.00710726022953168\\
10439.8826979472	-0.008020297929761\\
10533.724340176	-0.00685458801308414\\
10815.2492668622	-6.57911659800448e-05\\
10909.0909090909	-0.000526078558323206\\
11049.853372434	-0.00386319812969305\\
11190.6158357771	-0.00673804868347361\\
11284.4574780059	-0.00667787798010977\\
11425.219941349	-0.00375005009482265\\
11565.9824046921	-0.000536908737558406\\
11659.8240469208	-3.65871492249426e-05\\
11753.6656891496	-0.00118688253496657\\
11988.2697947214	-0.00590254162307247\\
12082.1114369501	-0.0059035022786702\\
12222.8739002933	-0.00337143889555591\\
12363.6363636364	-0.000508325570990564\\
12457.4780058651	-2.46188574237749e-05\\
12598.2404692082	-0.00190355304584955\\
12785.9237536657	-0.00524460818633088\\
12879.7653958944	-0.00529359817664954\\
13020.5278592375	-0.00307414636699832\\
13161.2903225806	-0.00048749904090073\\
13255.1319648094	-1.58298498718068e-05\\
13395.8944281525	-0.00166615676425863\\
13583.57771261	-0.00471772119635716\\
13677.4193548387	-0.00480550939028035\\
13818.1818181818	-0.00283749097798136\\
14005.8651026393	-8.72344353410881e-05\\
14146.6275659824	-0.000757054673158564\\
14428.1524926686	-0.00450092817845871\\
14568.9149560117	-0.00341582870532875\\
14803.5190615836	-9.33636983972974e-05\\
14944.2815249267	-0.000663445651298389\\
15225.8064516129	-0.00415198862538091\\
15366.568914956	-0.00319724591827253\\
15601.1730205279	-9.98853247438092e-05\\
15741.935483871	-0.000585612084250897\\
16070.3812316716	-0.00382052583518089\\
16211.1436950147	-0.00236954064166639\\
16398.8269794721	-0.000106850937299896\\
16539.5894428152	-0.000520238765602699\\
16868.0351906158	-0.00360073762931279\\
17055.7184750733	-0.00160676927771419\\
17243.4017595308	2.17394699575379e-07\\
17384.1642228739	-0.000977913896349492\\
17618.7683284457	-0.00342688743330655\\
17759.5307917889	-0.00275573623002856\\
18041.0557184751	0\\
18181.8181818182	-0.000897936151886825\\
18416.42228739	-0.00326221221257583\\
18557.1847507331	-0.00266116777129355\\
18838.7096774194	-9.90523403743282e-07\\
18979.4721407625	-0.000829367305414053\\
19214.0762463343	-0.00312614663198474\\
19354.8387096774	-0.00258683256106451\\
19636.3636363636	-2.68563599092886e-06\\
19777.1260997067	-0.000770294122048654\\
20058.651026393	-0.00304916612367379\\
20246.3343108504	-0.00149549735579058\\
20434.0175953079	-5.04584022564813e-06\\
20574.780058651	-0.000719204213964986\\
20856.3049853372	-0.00297193384903949\\
21043.9882697947	-0.0014915046558599\\
21231.6715542522	-8.05644958745688e-06\\
21372.4340175953	-0.000674891754897544\\
21653.9589442815	-0.00291377263056347\\
21841.642228739	-0.00149594479808002\\
22029.3255131965	-1.17250419862103e-05\\
22170.0879765396	-0.000636388762359275\\
22451.6129032258	-0.00287321353243897\\
22639.2961876833	-0.00150868630589684\\
22826.9794721408	-1.60803137987386e-05\\
22967.7419354839	-0.000602914318733383\\
23249.2668621701	-0.00284926158565213\\
23436.9501466276	-0.0015297995560104\\
23624.633431085	-2.11721016967203e-05\\
23765.3958944282	-0.000573836819967255\\
24000	-0.0027461170902825\\
};
\end{axis}
\end{tikzpicture}%
	}
	\end{minipage}\hfill
	\begin{minipage}{0.45\textwidth}
		\scalebox{0.5}{
		% This file was created by matlab2tikz.
%
%The latest updates can be retrieved from
%  http://www.mathworks.com/matlabcentral/fileexchange/22022-matlab2tikz-matlab2tikz
%where you can also make suggestions and rate matlab2tikz.
%
\definecolor{mycolor1}{rgb}{0.00000,0.44700,0.74100}%
%
\begin{tikzpicture}

\begin{axis}[%
width=4.521in,
height=3.548in,
at={(0.758in,0.499in)},
scale only axis,
xmin=-0.0008,
xmax=0.0008,
xlabel style={font=\color{white!15!black}},
xlabel={temps [s]},
ymin=-0.05,
ymax=0.2,
ylabel style={font=\color{white!15!black}},
ylabel={amplitude},
axis background/.style={fill=white},
title style={font=\bfseries},
title={Réponse impulsionnelle du filtre passe-haut}
]
\addplot [color=mycolor1, line width=3.0pt, forget plot]
  table[row sep=crcr]{%
-0.000624999999999987	0\\
-0.000604166666666656	0.0054881014859274\\
-0.000583333333333325	0.00984515884681772\\
-0.000562499999999994	0.0117892550438441\\
-0.000541666666666663	0.0106024787581114\\
-0.000520833333333331	0.00636619772367583\\
-0.0005	-0\\
-0.000479166666666669	-0.00691978013443023\\
-0.000458333333333338	-0.0125302021686771\\
-0.000437500000000007	-0.0151576136277995\\
-0.000416666666666676	-0.0137832223855448\\
-0.000395833333333345	-0.00837657595220501\\
-0.000374999999999986	0\\
-0.000354166666666655	0.00936205547599384\\
-0.000333333333333324	0.017229027981931\\
-0.000312499999999993	0.0212206590789194\\
-0.000291666666666662	0.0196903176936354\\
-0.000270833333333331	0.0122426879301458\\
-0.00025	-0\\
-0.000229166666666669	-0.0144686311901723\\
-0.000208333333333338	-0.0275664447710896\\
-0.000187500000000007	-0.0353677651315323\\
-0.000166666666666676	-0.034458055963862\\
-0.000145833333333345	-0.0227364204416994\\
-0.000124999999999986	0\\
-0.000104166666666655	0.031830988618379\\
-6.24999999999931e-05	0.106103295394597\\
-4.16666666666621e-05	0.137832223855448\\
-2.0833333333331e-05	0.159154943091895\\
0	0.166666666666667\\
2.0833333333331e-05	0.159154943091895\\
4.16666666666621e-05	0.137832223855448\\
6.24999999999931e-05	0.106103295394597\\
0.000104166666666655	0.031830988618379\\
0.000124999999999986	0\\
0.000145833333333345	-0.0227364204416994\\
0.000166666666666676	-0.034458055963862\\
0.000187500000000007	-0.0353677651315323\\
0.000208333333333338	-0.0275664447710896\\
0.000229166666666669	-0.0144686311901723\\
0.00025	-0\\
0.000270833333333331	0.0122426879301458\\
0.000291666666666662	0.0196903176936354\\
0.000312499999999993	0.0212206590789194\\
0.000333333333333324	0.017229027981931\\
0.000354166666666655	0.00936205547599384\\
0.000374999999999986	0\\
0.000395833333333345	-0.00837657595220501\\
0.000416666666666676	-0.0137832223855448\\
0.000437500000000007	-0.0151576136277995\\
0.000458333333333338	-0.0125302021686771\\
0.000479166666666669	-0.00691978013443023\\
0.0005	-0\\
0.000520833333333331	0.00636619772367583\\
0.000541666666666663	0.0106024787581114\\
0.000562499999999994	0.0117892550438441\\
0.000583333333333325	0.00984515884681772\\
0.000604166666666656	0.0054881014859274\\
0.000624999999999987	0\\
};
\end{axis}
\end{tikzpicture}%
	}
	\end{minipage}
	\begin{minipage}{0.45\textwidth}
		\scalebox{0.5}{
		% This file was created by matlab2tikz.
%
%The latest updates can be retrieved from
%  http://www.mathworks.com/matlabcentral/fileexchange/22022-matlab2tikz-matlab2tikz
%where you can also make suggestions and rate matlab2tikz.
%
\definecolor{mycolor1}{rgb}{0.00000,0.44700,0.74100}%
%
\begin{tikzpicture}

\begin{axis}[%
width=4.521in,
height=3.548in,
at={(0.758in,0.499in)},
scale only axis,
xmin=-25000,
xmax=25000,
xlabel style={font=\color{white!15!black}},
xlabel={fréquence [Hz]},
ymin=-0.5,
ymax=2,
ylabel style={font=\color{white!15!black}},
ylabel={amplitude},
axis background/.style={fill=white},
title style={font=\bfseries},
title={Réponse fréquentielle du filtre passe-bas}
]
\addplot [color=mycolor1, forget plot]
  table[row sep=crcr]{%
-24000	1.00284134051617\\
-23812.3167155425	1.00155955278751\\
-23624.633431085	1.00002707258682\\
-23483.8709677419	1.00054864570848\\
-23202.3460410557	1.00284926158565\\
-23014.6627565982	1.00159841876666\\
-22826.9794721408	1.000033878674\\
-22686.2170087977	1.00052693034013\\
-22357.7712609971	1.00280235687751\\
-22123.1671554252	1.00063638876236\\
-21982.4046920821	1.00001172504199\\
-21794.7214076246	1.00149594479808\\
-21607.0381231672	1.00291377263056\\
-21466.275659824	1.0022080022427\\
-21184.7507331378	1.00000805644959\\
-20997.0674486804	1.00149150465586\\
-20809.3841642229	1.00297193384904\\
-20668.6217008798	1.00228540270109\\
-20387.0967741935	1.00000504584023\\
-20199.4134897361	1.00149549735579\\
-20011.7302052786	1.00304916612367\\
-19870.9677419355	1.00237929897048\\
-19589.4428152493	1.00000268563599\\
-19448.6803519062	1.00094629236264\\
-19214.0762463343	1.00314749735844\\
-19073.3137829912	1.00249197738231\\
-18791.788856305	1.0000009905234\\
-18651.0263929619	1.00094936556707\\
-18416.42228739	1.00326963812768\\
-18275.6598240469	1.00262637966443\\
-17994.1348973607	1\\
-17853.3724340176	1.00095825716926\\
-17618.7683284457	1.00341915822355\\
-17478.0058651026	1.00278628941669\\
-17196.4809384164	0.9999997826053\\
-17055.7184750733	1.0009734957348\\
-16821.1143695015	1.00360073762931\\
-16680.3519061584	1.00297659391799\\
-16398.8269794721	1.00000044234912\\
-16258.064516129	1.00099584790223\\
-16023.4604105572	1.00382052583518\\
-15882.6979472141	1.00320365634616\\
-15601.1730205279	1.00000212810846\\
-15460.4105571848	1.00102639131364\\
-15225.8064516129	1.00408666292788\\
-15085.0439882698	1.00347585372947\\
-14803.5190615836	1.00000504711716\\
-14662.7565982405	1.0010666247872\\
-14428.1524926686	1.00441004937602\\
-14287.3900293255	1.00380437034619\\
-14005.8651026393	1.00000948428351\\
-13865.1026392962	1.00111863720304\\
-13630.4985337243	1.00480550939028\\
-13536.6568914956	1.00471772119636\\
-13395.8944281525	1.00256852132588\\
-13255.1319648094	1.00029332867416\\
-13161.2903225806	1.00008145258835\\
-13020.5278592375	1.00208657428448\\
-12832.8445747801	1.00529359817665\\
-12739.0029325513	1.00524460818633\\
-12598.2404692082	1.0029059542976\\
-12457.4780058651	1.00035349465179\\
-12363.6363636364	1.00007597816511\\
-12222.8739002933	1.00226834615023\\
-12035.1906158358	1.00590350227867\\
-11941.348973607	1.00590254162307\\
-11800.5865102639	1.00332739876831\\
-11659.8240469208	1.00042921894419\\
-11565.9824046921	1.00007077269765\\
-11472.1407624633	1.00138192579834\\
-11237.5366568915	1.00667787798011\\
-11143.6950146628	1.00673804868347\\
-11002.9325513196	1.00386319812969\\
-10862.1700879765	1.00052607855832\\
-10768.3284457478	1.00006579116598\\
-10674.4868035191	1.00152747882385\\
-10439.8826979472	1.00768130552024\\
-10346.0410557185	1.00782163147596\\
-10252.1994134897	1.00596938660601\\
-10064.5161290323	1.00065234345311\\
-9970.67448680352	1.00006096142533\\
-9876.83284457478	1.00172285567896\\
-9642.22873900293	1.00901591725778\\
-9548.38709677419	1.00926489897392\\
-9454.54545454546	1.00714169262574\\
-9266.86217008798	1.00082067971744\\
-9173.02052785924	1.000056128898\\
-9079.1788856305	1.00199355611403\\
-8844.57478005865	1.01085237681764\\
-8750.73313782991	1.01125455116562\\
-8656.89149560118	1.00876042005621\\
-8469.2082111437	1.00105120532317\\
-8375.36656891496	1.00005089759725\\
-8281.52492668622	1.00238578128119\\
-8046.92082111437	1.01349700020364\\
-8000	1.01428869328811\\
-7906.15835777126	1.01301736017922\\
-7812.31671554252	1.00862627023162\\
-7671.55425219941	1.00137709997216\\
-7624.63343108504	1.00019249843899\\
-7577.71260997068	1.00004412649287\\
-7530.79178885631	1.0010029510413\\
-7436.95014662757	1.00579141289927\\
-7296.18768328446	1.01536261037108\\
-7249.26686217009	1.01755530907758\\
-7202.34604105572	1.01867204864175\\
-7155.42521994135	1.01854350818394\\
-7108.50439882698	1.01716491164188\\
-7014.66275659824	1.01147801043044\\
-6920.8211143695	1.00455594769301\\
-6873.90029325513	1.00185444155795\\
-6826.97947214076	1.0002490522711\\
-6780.05865102639	1.000032085165\\
-6733.13782991202	1.00131866893207\\
-6686.21700879766	1.00402393187323\\
-6592.37536656891	1.01239806221565\\
-6498.53372434018	1.0212469569633\\
-6451.61290322581	1.02440125961584\\
-6404.69208211144	1.02607766679284\\
-6357.77126099707	1.026016122687\\
-6310.8504398827	1.02418445289732\\
-6263.92961876833	1.0207935577746\\
-6076.24633431085	1.0025713571049\\
-6029.32551319648	1.00026964541394\\
-5982.40469208211	1\\
-5935.48387096774	1.00196774065626\\
-5888.56304985337	1.00608394503797\\
-5841.64222873901	1.01196130117387\\
-5700.87976539589	1.03282434059656\\
-5653.95894428153	1.03789928818514\\
-5607.03812316716	1.04071031253989\\
-5560.11730205279	1.04080086918475\\
-5513.19648093842	1.03806535916738\\
-5466.27565982405	1.03278775279978\\
-5372.43401759531	1.01757036523486\\
-5325.51319648094	1.00979060947429\\
-5278.59237536657	1.00353315567554\\
-5231.6715542522	0.999935086412734\\
-5184.75073313783	0.999865202880756\\
-5137.82991202346	1.00378376121444\\
-5090.90909090909	1.01164536590659\\
-5043.98826979472	1.02286118884876\\
-4903.22580645161	1.06374336104273\\
-4856.30498533724	1.0741516930284\\
-4809.38416422287	1.08019849874836\\
-4762.46334310851	1.08076726888976\\
-4715.54252199414	1.07540169101776\\
-4668.62170087976	1.06445379629076\\
-4621.7008797654	1.04914885288963\\
-4527.85923753666	1.01442689868782\\
-4480.93841642229	1.00100546083559\\
-4434.01759530792	0.994664725061739\\
-4387.09677419355	0.998560438467393\\
-4340.17595307918	1.01524017610063\\
-4293.25513196481	1.04628045551726\\
-4246.33431085044	1.09198759729043\\
-4199.41348973607	1.15119859351034\\
-4152.4926686217	1.22121006145971\\
-4105.57184750733	1.2978518112177\\
-4058.65102639296	1.37570774895721\\
-4011.73020527859	1.44847215877599\\
-3964.80938416422	1.50941537509425\\
-3917.88856304986	1.55192096536484\\
-3870.96774193548	1.57004808658166\\
-3824.04692082111	1.55906861754556\\
-3777.12609970674	1.51592951200291\\
-3730.20527859238	1.43959656575316\\
-3683.28445747801	1.33124592697641\\
-3636.36363636364	1.19428322034582\\
-3589.44281524927	1.0341857566782\\
-3542.5219941349	0.858179393471801\\
-3495.60117302053	0.674776579195168\\
-3448.68035190616	0.493214456648275\\
-3401.75953079179	0.322840402175643\\
-3354.83870967742	0.172496231611149\\
-3307.91788856305	0.0499511997695663\\
-3260.99706744868	-0.0385719311525463\\
-3214.07624633431	-0.0887433723946742\\
-3167.15542521994	-0.0983237780928903\\
-3120.23460410557	-0.0672054949427547\\
-3073.3137829912	0.00269525380645064\\
-3026.39296187683	0.107675199626101\\
-2979.47214076246	0.242579665256926\\
-2932.55131964809	0.401200303596852\\
-2885.63049853372	0.576695905750967\\
-2838.70967741936	0.761999095229839\\
-2791.78885630499	0.950177310362051\\
-2744.86803519062	1.13472495108726\\
-2697.94721407625	1.3097737947719\\
-2651.02639296188	1.4702194843594\\
-2604.10557184751	1.61177178215075\\
-2557.18475073314	1.7309442392434\\
-2510.26392961877	1.82500411221554\\
-2463.3431085044	1.89190529501866\\
-2416.42228739003	1.93022567375738\\
-2369.50146627566	1.93912601185002\\
-2322.58064516129	1.9183409454381\\
-2275.65982404692	1.86820488191734\\
-2228.73900293255	1.78970764655969\\
-2181.81818181818	1.68456770292323\\
-2134.89736070381	1.55530562725107\\
-2087.97653958944	1.40529793056703\\
-2041.05571847507	1.23879164122263\\
-1994.13489736071	1.06086325296201\\
-1900.29325513196	0.694479336994846\\
-1853.37243401759	0.519014062112547\\
-1806.45161290323	0.357571986831317\\
-1759.53079178886	0.216494158601563\\
-1712.60997067449	0.101473846214503\\
-1665.68914956012	0.0172439424641198\\
-1618.76832844575	-0.032689852938347\\
-1571.84750733138	-0.0462476356515253\\
-1524.92668621701	-0.0228892817831365\\
-1478.00586510264	0.0363648482962162\\
-1431.08504398827	0.129006609313365\\
-1384.1642228739	0.251197931014758\\
-1337.24340175953	0.397998349570116\\
-1290.32258064516	0.563641378641478\\
-1243.40175953079	0.741835985965736\\
-1149.56011730205	1.10988827414985\\
-1102.63929618768	1.28714509504425\\
-1055.71847507331	1.45219198638006\\
-1008.79765395894	1.60002414306291\\
-961.876832844573	1.72637087833937\\
-914.956011730206	1.82774309731394\\
-868.035190615836	1.90144828039774\\
-821.114369501465	1.94558598080039\\
-774.193548387098	1.95903637466472\\
-727.272727272728	1.94145192213546\\
-680.351906158357	1.89325815297343\\
-633.431085043987	1.81566463787385\\
-586.51026392962	1.71068212081809\\
-539.58944281525	1.58113735326333\\
-492.668621700879	1.43067408083516\\
-445.747800586509	1.26372739301587\\
-398.826979472142	1.08545950967891\\
-304.985337243401	0.718522180002765\\
-258.064516129034	0.542549044290354\\
-211.143695014664	0.380176900809602\\
-164.222873900293	0.237550187837769\\
-117.302052785923	0.120214246318938\\
-70.3812316715557	0.0328315486804058\\
-23.4604105571852	-0.021068310506962\\
23.4604105571852	-0.0392842489454779\\
70.3812316715557	-0.021068310506962\\
117.302052785923	0.0328315486804058\\
164.222873900293	0.120214246318938\\
211.143695014664	0.237550187837769\\
258.064516129034	0.380176900809602\\
304.985337243401	0.542549044290354\\
351.906158357771	0.718522180002765\\
445.747800586509	1.08545950967891\\
492.668621700879	1.26372739301587\\
539.58944281525	1.43067408083516\\
586.51026392962	1.58113735326333\\
633.431085043987	1.71068212081809\\
680.351906158357	1.81566463787385\\
727.272727272728	1.89325815297343\\
774.193548387098	1.94145192213546\\
821.114369501465	1.95903637466472\\
868.035190615836	1.94558598080039\\
914.956011730206	1.90144828039774\\
961.876832844573	1.82774309731394\\
1008.79765395894	1.72637087833937\\
1055.71847507331	1.60002414306291\\
1102.63929618768	1.45219198638006\\
1149.56011730205	1.28714509504425\\
1196.48093841642	1.10988827414985\\
1290.32258064516	0.741835985965736\\
1337.24340175953	0.563641378641478\\
1384.1642228739	0.397998349570116\\
1431.08504398827	0.251197931014758\\
1478.00586510264	0.129006609313365\\
1524.92668621701	0.0363648482962162\\
1571.84750733138	-0.0228892817831365\\
1618.76832844575	-0.0462476356515253\\
1665.68914956012	-0.032689852938347\\
1712.60997067449	0.0172439424641198\\
1759.53079178886	0.101473846214503\\
1806.45161290323	0.216494158601563\\
1853.37243401759	0.357571986831317\\
1900.29325513196	0.519014062112547\\
1947.21407624634	0.694479336994846\\
2041.05571847507	1.06086325296201\\
2087.97653958944	1.23879164122263\\
2134.89736070381	1.40529793056703\\
2181.81818181818	1.55530562725107\\
2228.73900293255	1.68456770292323\\
2275.65982404692	1.78970764655969\\
2322.58064516129	1.86820488191734\\
2369.50146627566	1.9183409454381\\
2416.42228739003	1.93912601185002\\
2463.3431085044	1.93022567375738\\
2510.26392961877	1.89190529501866\\
2557.18475073314	1.82500411221554\\
2604.10557184751	1.7309442392434\\
2651.02639296188	1.61177178215075\\
2697.94721407625	1.4702194843594\\
2744.86803519062	1.3097737947719\\
2791.78885630499	1.13472495108726\\
2838.70967741936	0.950177310362051\\
2885.63049853372	0.761999095229839\\
2932.55131964809	0.576695905750967\\
2979.47214076246	0.401200303596852\\
3026.39296187683	0.242579665256926\\
3073.3137829912	0.107675199626101\\
3120.23460410557	0.00269525380645064\\
3167.15542521994	-0.0672054949427547\\
3214.07624633431	-0.0983237780928903\\
3260.99706744868	-0.0887433723946742\\
3307.91788856305	-0.0385719311525463\\
3354.83870967742	0.0499511997695663\\
3401.75953079179	0.172496231611149\\
3448.68035190616	0.322840402175643\\
3495.60117302053	0.493214456648275\\
3542.5219941349	0.674776579195168\\
3589.44281524927	0.858179393471801\\
3636.36363636364	1.0341857566782\\
3683.28445747801	1.19428322034582\\
3730.20527859238	1.33124592697641\\
3777.12609970674	1.43959656575316\\
3824.04692082111	1.51592951200291\\
3870.96774193548	1.55906861754556\\
3917.88856304986	1.57004808658166\\
3964.80938416422	1.55192096536484\\
4011.73020527859	1.50941537509425\\
4058.65102639296	1.44847215877599\\
4105.57184750733	1.37570774895721\\
4152.4926686217	1.2978518112177\\
4199.41348973607	1.22121006145971\\
4246.33431085044	1.15119859351034\\
4293.25513196481	1.09198759729043\\
4340.17595307918	1.04628045551726\\
4387.09677419355	1.01524017610063\\
4434.01759530792	0.998560438467393\\
4480.93841642229	0.994664725061739\\
4527.85923753666	1.00100546083559\\
4574.78005865103	1.01442689868782\\
4668.62170087976	1.04914885288963\\
4715.54252199414	1.06445379629076\\
4762.46334310851	1.07540169101776\\
4809.38416422287	1.08076726888976\\
4856.30498533724	1.08019849874836\\
4903.22580645161	1.0741516930284\\
4950.14662756598	1.06374336104273\\
4997.06744868035	1.05054231569011\\
5090.90909090909	1.02286118884876\\
5137.82991202346	1.01164536590659\\
5184.75073313783	1.00378376121444\\
5231.6715542522	0.999865202880756\\
5278.59237536657	0.999935086412734\\
5325.51319648094	1.00353315567554\\
5372.43401759531	1.00979060947429\\
5513.19648093842	1.03278775279978\\
5560.11730205279	1.03806535916738\\
5607.03812316716	1.04080086918475\\
5653.95894428153	1.04071031253989\\
5700.87976539589	1.03789928818514\\
5747.80058651026	1.03282434059656\\
5841.64222873901	1.01895017342395\\
5888.56304985337	1.01196130117387\\
5935.48387096774	1.00608394503797\\
5982.40469208211	1.00196774065626\\
6029.32551319648	1\\
6076.24633431085	1.00026964541394\\
6123.16715542522	1.0025713571049\\
6217.00879765396	1.01125928912006\\
6310.8504398827	1.0207935577746\\
6357.77126099707	1.02418445289732\\
6404.69208211144	1.026016122687\\
6451.61290322581	1.02607766679284\\
6498.53372434018	1.02440125961584\\
6545.45454545455	1.0212469569633\\
6733.13782991202	1.00402393187323\\
6780.05865102639	1.00131866893207\\
6826.97947214076	1.000032085165\\
6873.90029325513	1.0002490522711\\
6920.8211143695	1.00185444155795\\
7014.66275659824	1.00792913526675\\
7108.50439882698	1.01470266174874\\
7155.42521994135	1.01716491164188\\
7202.34604105572	1.01854350818394\\
7249.26686217009	1.01867204864175\\
7296.18768328446	1.01755530907758\\
7390.0293255132	1.01239988203452\\
7530.79178885631	1.00299187833662\\
7577.71260997068	1.0010029510413\\
7624.63343108504	1.00004412649287\\
7671.55425219941	1.00019249843899\\
7765.39589442815	1.00339285389055\\
7953.07917888563	1.01301736017922\\
8046.92082111437	1.01428869328811\\
8140.76246334311	1.01186945930021\\
8375.36656891496	1.00082204475984\\
8469.2082111437	1.00014247693616\\
8563.04985337243	1.00262481423488\\
8750.73313782991	1.01032353072515\\
8844.57478005865	1.01143591553409\\
8938.41642228739	1.00959054988198\\
9173.02052785924	1.00070757200228\\
9266.86217008798	1.00010366228889\\
9360.70381231671	1.00208974762427\\
9548.38709677419	1.0084587400379\\
9642.22873900293	1.00945708349536\\
9736.07038123167	1.00800656017964\\
10017.5953079179	1.00006096142533\\
10111.4369501466	1.00065234345311\\
10252.1994134897	1.00455930909084\\
10346.0410557185	1.00710726022953\\
10439.8826979472	1.00802029792976\\
10533.724340176	1.00685458801308\\
10815.2492668622	1.00006579116598\\
10909.0909090909	1.00052607855832\\
11049.853372434	1.00386319812969\\
11190.6158357771	1.00673804868347\\
11284.4574780059	1.00667787798011\\
11425.219941349	1.00375005009482\\
11565.9824046921	1.00053690873756\\
11659.8240469208	1.00003658714922\\
11753.6656891496	1.00118688253497\\
11988.2697947214	1.00590254162307\\
12082.1114369501	1.00590350227867\\
12222.8739002933	1.00337143889556\\
12363.6363636364	1.00050832557099\\
12457.4780058651	1.00002461885742\\
12598.2404692082	1.00190355304585\\
12785.9237536657	1.00524460818633\\
12879.7653958944	1.00529359817665\\
13020.5278592375	1.003074146367\\
13161.2903225806	1.0004874990409\\
13255.1319648094	1.00001582984987\\
13395.8944281525	1.00166615676426\\
13583.57771261	1.00471772119636\\
13677.4193548387	1.00480550939028\\
13818.1818181818	1.00283749097798\\
14005.8651026393	1.00008723443534\\
14146.6275659824	1.00075705467316\\
14428.1524926686	1.00450092817846\\
14568.9149560117	1.00341582870533\\
14803.5190615836	1.0000933636984\\
14944.2815249267	1.0006634456513\\
15225.8064516129	1.00415198862538\\
15366.568914956	1.00319724591827\\
15601.1730205279	1.00009988532474\\
15741.935483871	1.00058561208425\\
16070.3812316716	1.00382052583518\\
16211.1436950147	1.00236954064167\\
16398.8269794721	1.0001068509373\\
16539.5894428152	1.0005202387656\\
16868.0351906158	1.00360073762931\\
17055.7184750733	1.00160676927771\\
17243.4017595308	0.9999997826053\\
17384.1642228739	1.00097791389635\\
17618.7683284457	1.00342688743331\\
17759.5307917889	1.00275573623003\\
18041.0557184751	1\\
18181.8181818182	1.00089793615189\\
18416.42228739	1.00326221221258\\
18557.1847507331	1.00266116777129\\
18838.7096774194	1.0000009905234\\
18979.4721407625	1.00082936730541\\
19214.0762463343	1.00312614663198\\
19354.8387096774	1.00258683256106\\
19636.3636363636	1.00000268563599\\
19777.1260997067	1.00077029412205\\
20058.651026393	1.00304916612367\\
20246.3343108504	1.00149549735579\\
20434.0175953079	1.00000504584023\\
20574.780058651	1.00071920421396\\
20856.3049853372	1.00297193384904\\
21043.9882697947	1.00149150465586\\
21231.6715542522	1.00000805644959\\
21372.4340175953	1.0006748917549\\
21653.9589442815	1.00291377263056\\
21841.642228739	1.00149594479808\\
22029.3255131965	1.00001172504199\\
22170.0879765396	1.00063638876236\\
22451.6129032258	1.00287321353244\\
22639.2961876833	1.0015086863059\\
22826.9794721408	1.0000160803138\\
22967.7419354839	1.00060291431873\\
23249.2668621701	1.00284926158565\\
23436.9501466276	1.00152979955601\\
23624.633431085	1.0000211721017\\
23765.3958944282	1.00057383681997\\
24000	1.00274611709028\\
};
\end{axis}
\end{tikzpicture}%
	}
	\end{minipage}\hfill
	\begin{minipage}{0.45\textwidth}
		\scalebox{0.5}{
		% This file was created by matlab2tikz.
%
%The latest updates can be retrieved from
%  http://www.mathworks.com/matlabcentral/fileexchange/22022-matlab2tikz-matlab2tikz
%where you can also make suggestions and rate matlab2tikz.
%
\definecolor{mycolor1}{rgb}{0.00000,0.44700,0.74100}%
%
\begin{tikzpicture}

\begin{axis}[%
width=4.521in,
height=3.548in,
at={(0.758in,0.499in)},
scale only axis,
xmin=-0.0008,
xmax=0.0008,
xlabel style={font=\color{white!15!black}},
xlabel={temps [s]},
ymin=-0.2,
ymax=1,
ylabel style={font=\color{white!15!black}},
ylabel={amplitude},
axis background/.style={fill=white},
title style={font=\bfseries},
title={Réponse impulsionnelle du filtre passe-bas}
]
\addplot [color=mycolor1, forget plot]
  table[row sep=crcr]{%
-0.000624999999999987	-0\\
-0.000604166666666628	-0.00548810148592738\\
-0.00058333333333338	-0.00984515884681769\\
-0.000562500000000021	-0.0117892550438441\\
-0.000541666666666663	-0.0106024787581114\\
-0.000520833333333304	-0.00636619772367586\\
-0.000499999999999945	0\\
-0.000479166666666697	0.00691978013443018\\
-0.000458333333333338	0.0125302021686771\\
-0.00043749999999998	0.0151576136277995\\
-0.000416666666666621	0.0137832223855449\\
-0.000395833333333373	0.00837657595220498\\
-0.000375000000000014	-0\\
-0.000354166666666655	-0.00936205547599389\\
-0.000333333333333297	-0.017229027981931\\
-0.000312500000000049	-0.0212206590789193\\
-0.00029166666666669	-0.0196903176936354\\
-0.000270833333333331	-0.0122426879301458\\
-0.000249999999999972	0\\
-0.000229166666666614	0.0144686311901723\\
-0.000208333333333366	0.0275664447710896\\
-0.000187500000000007	0.0353677651315323\\
-0.000166666666666648	0.034458055963862\\
-0.00014583333333329	0.0227364204416993\\
-0.000125000000000042	-0\\
-0.000104166666666683	-0.0318309886183791\\
-6.24999999999654e-05	-0.106103295394597\\
-4.16666666667176e-05	-0.137832223855448\\
-2.08333333333588e-05	-0.159154943091895\\
0	0.833333333333333\\
2.08333333333588e-05	-0.159154943091895\\
4.16666666667176e-05	-0.137832223855448\\
6.24999999999654e-05	-0.106103295394597\\
0.000104166666666683	-0.0318309886183791\\
0.000125000000000042	-0\\
0.00014583333333329	0.0227364204416993\\
0.000166666666666648	0.034458055963862\\
0.000187500000000007	0.0353677651315323\\
0.000208333333333366	0.0275664447710896\\
0.000229166666666614	0.0144686311901723\\
0.000249999999999972	0\\
0.000270833333333331	-0.0122426879301458\\
0.00029166666666669	-0.0196903176936354\\
0.000312500000000049	-0.0212206590789193\\
0.000333333333333297	-0.017229027981931\\
0.000354166666666655	-0.00936205547599389\\
0.000375000000000014	-0\\
0.000395833333333373	0.00837657595220498\\
0.000416666666666621	0.0137832223855449\\
0.00043749999999998	0.0151576136277995\\
0.000458333333333338	0.0125302021686771\\
0.000479166666666697	0.00691978013443018\\
0.000499999999999945	0\\
0.000520833333333304	-0.00636619772367586\\
0.000541666666666663	-0.0106024787581114\\
0.000562500000000021	-0.0117892550438441\\
0.00058333333333338	-0.00984515884681769\\
0.000604166666666628	-0.00548810148592738\\
0.000624999999999987	-0\\
};
\end{axis}
\end{tikzpicture}%
	}
	\end{minipage}
	\caption{Réponses des filtres}
	\label{fig:reponses-filtres}
\end{figure}

Les sorties des filtres sont les suivantes

% This file was created by matlab2tikz.
%
%The latest updates can be retrieved from
%  http://www.mathworks.com/matlabcentral/fileexchange/22022-matlab2tikz-matlab2tikz
%where you can also make suggestions and rate matlab2tikz.
%
\definecolor{mycolor1}{rgb}{0.00000,0.44700,0.74100}%
\definecolor{mycolor2}{rgb}{0.85000,0.32500,0.09800}%
%
\begin{tikzpicture}

\begin{axis}[%
width=4.521in,
height=3.559in,
at={(0.758in,0.488in)},
scale only axis,
xmin=0,
xmax=4000,
xlabel style={font=\color{white!15!black}},
xlabel={Temps [s]},
ymin=-1.5,
ymax=1.5,
ylabel style={font=\color{white!15!black}},
ylabel={Amplitude},
axis background/.style={fill=white},
title style={font=\bfseries},
title={Signal démodulé},
legend style={legend cell align=left, align=left, draw=white!15!black}
]
\addplot [color=mycolor1]
  table[row sep=crcr]{%
1	-0.0037331097405513\\
3	-0.0142427619216505\\
7	-0.0381347271045342\\
8	-0.0423112335402038\\
9	-0.0450159428287407\\
10	-0.0457074914238547\\
11	-0.043993347050673\\
12	-0.0395898552928884\\
13	-0.0320865960025003\\
14	-0.0213516150829491\\
15	-0.0071884278017933\\
16	0.0104635209931985\\
17	0.0316027244630277\\
18	0.0561464217275898\\
19	0.0838477803672504\\
20	0.114429365314663\\
21	0.147480861340227\\
22	0.182532521439498\\
24	0.256414529294943\\
26	0.330863627591498\\
27	0.366472803108991\\
28	0.399955722209597\\
29	0.430543546280205\\
30	0.457581967224996\\
31	0.480300474701835\\
32	0.498098621252666\\
33	0.510371490871421\\
34	0.516721402921121\\
35	0.51670537803102\\
36	0.51008032604841\\
37	0.49671337237578\\
38	0.476604925274387\\
39	0.449927518028289\\
40	0.416949165145525\\
41	0.378018212678853\\
42	0.333684092299336\\
43	0.28451378337104\\
44	0.231400341564949\\
45	0.175101027343317\\
46	0.116595889903238\\
49	-0.0621510681089603\\
50	-0.119299367629992\\
51	-0.173546185672876\\
52	-0.223846554468309\\
53	-0.269290443138289\\
54	-0.309073430755234\\
55	-0.342400575179454\\
56	-0.368678904026183\\
57	-0.387458599237107\\
58	-0.398322735374677\\
59	-0.401173747006851\\
60	-0.395959271737411\\
61	-0.382791665147579\\
62	-0.358165948637179\\
63	-0.324968717777665\\
64	-0.284056380010497\\
65	-0.236369629347337\\
66	-0.183109011037686\\
67	-0.125523806878846\\
68	-0.0648720119443169\\
71	0.120465004425569\\
72	0.17844569236695\\
73	0.232254784275938\\
74	0.280414415665291\\
75	0.321971381833919\\
76	0.355840774361241\\
77	0.381266160304676\\
78	0.397598172271955\\
79	0.40445984342432\\
80	0.401687938582199\\
81	0.389346045969887\\
82	0.367732689977856\\
83	0.337416978330111\\
84	0.299110888256564\\
85	0.253613764194142\\
86	0.202052633882431\\
87	0.145705864016236\\
88	0.0858487071845957\\
91	-0.0999200173569079\\
92	-0.159040550459849\\
93	-0.214366168223023\\
94	-0.264646885224465\\
95	-0.30862879047163\\
96	-0.345250894141373\\
97	-0.373590352721749\\
98	-0.393036851441138\\
99	-0.403120685456543\\
100	-0.403667421890532\\
101	-0.394608724420777\\
102	-0.376161317522019\\
103	-0.348707885113981\\
104	-0.312916926128764\\
105	-0.269730201140646\\
106	-0.220078088334049\\
107	-0.165252350938317\\
108	-0.106465190241124\\
110	0.0172041697046552\\
111	0.0791175250428751\\
112	0.139202586380179\\
113	0.195970649540868\\
114	0.248062711702914\\
115	0.294286775466389\\
116	0.333508902288486\\
117	0.364753787419886\\
118	0.38735283864844\\
119	0.400678011920263\\
120	0.404495521086574\\
121	0.398668677444221\\
122	0.383354262726243\\
123	0.35885493339083\\
124	0.325787717114054\\
125	0.284970824333868\\
126	0.237391025639226\\
127	0.184183478862906\\
128	0.126587107000887\\
129	0.0659267407568223\\
132	-0.119598698627215\\
133	-0.177688389699142\\
134	-0.23159189440139\\
135	-0.279922525361599\\
136	-0.321629480944011\\
137	-0.355745295966699\\
138	-0.381395739969776\\
139	-0.397929087109787\\
140	-0.404985823710831\\
141	-0.402397762282817\\
142	-0.390250491425377\\
143	-0.36882316543506\\
144	-0.338622190556634\\
145	-0.300393687788983\\
146	-0.255019089970119\\
147	-0.203577359522569\\
148	-0.147355046755365\\
149	-0.0875957218204348\\
152	0.0981783935071689\\
153	0.157336329874852\\
154	0.212759459284371\\
155	0.263152256094145\\
156	0.307249956834312\\
157	0.344019745219157\\
158	0.372514303503067\\
159	0.392128629129729\\
160	0.402443836198017\\
161	0.398601925595358\\
162	0.387921739129524\\
163	0.370668597360691\\
164	0.347072746440062\\
165	0.317654085929917\\
166	0.282884811656004\\
167	0.243343954683041\\
168	0.199756146990239\\
169	0.152759277686982\\
170	0.10308791875832\\
172	-0.0011791187162089\\
174	-0.107239132772975\\
175	-0.159214336961213\\
176	-0.209754585496285\\
177	-0.258311580264035\\
178	-0.304390317698108\\
179	-0.347685846478726\\
180	-0.387901326410883\\
181	-0.424828415690627\\
182	-0.458387751883492\\
183	-0.488583131761516\\
184	-0.515359251099653\\
185	-0.538959558910847\\
186	-0.559455265442466\\
187	-0.5771159424703\\
188	-0.592194123734771\\
189	-0.604898914041769\\
190	-0.615601930921912\\
191	-0.624538894007856\\
192	-0.631944246886178\\
194	-0.643005041166816\\
196	-0.650138717338905\\
198	-0.653433578108434\\
199	-0.653517768100755\\
200	-0.652385223929741\\
201	-0.649805690514768\\
202	-0.645425042655461\\
203	-0.638928101118381\\
204	-0.629924074562496\\
205	-0.618057853160735\\
206	-0.602953210952819\\
207	-0.584180111497972\\
208	-0.56135498522508\\
209	-0.534188886016182\\
210	-0.502414736710307\\
211	-0.465773913938392\\
212	-0.424250331694111\\
213	-0.377821381527156\\
214	-0.326438582513674\\
215	-0.270465367070756\\
216	-0.210070419600015\\
217	-0.145757658014645\\
218	-0.0780536283468791\\
219	-0.00757796570542268\\
221	0.138171413222153\\
222	0.216368339002202\\
223	0.290994829807005\\
224	0.360836160359213\\
225	0.424725155917713\\
226	0.481623529230546\\
227	0.530596241625062\\
228	0.570868289254577\\
229	0.601751080874237\\
230	0.622708292295101\\
231	0.633533792376511\\
232	0.63382367575241\\
233	0.623718337003538\\
234	0.603361274037525\\
235	0.573130531389324\\
236	0.533425305115088\\
237	0.484983462857144\\
238	0.428537248983048\\
239	0.365083611372484\\
240	0.295586458536945\\
241	0.221256484431706\\
242	0.14323085578917\\
244	-0.0185623218148976\\
245	-0.0996988886786312\\
246	-0.179152096396137\\
247	-0.255685030636414\\
248	-0.328032327379788\\
249	-0.394972912064532\\
250	-0.455429840618308\\
251	-0.508434014089289\\
252	-0.553068363005877\\
253	-0.58857723814981\\
254	-0.614475392921122\\
255	-0.630289364400141\\
256	-0.635685482216559\\
257	-0.630663600461958\\
258	-0.615303619588303\\
259	-0.589846624977326\\
260	-0.554655757568526\\
261	-0.510346184085392\\
262	-0.457680639179216\\
263	-0.39744577147394\\
264	-0.330699035836915\\
265	-0.258478027123147\\
266	-0.182108617744689\\
267	-0.102736340170395\\
270	0.140314969451993\\
271	0.218471294087067\\
272	0.293031436286128\\
273	0.362832692386746\\
274	0.426656749630638\\
275	0.483442202000788\\
276	0.532296781873356\\
277	0.572386333978557\\
278	0.603057891953085\\
279	0.623856534303286\\
280	0.634340093661194\\
281	0.634437411390536\\
282	0.624129145451661\\
283	0.603537933676762\\
284	0.573018666254484\\
285	0.533086802226535\\
286	0.484408104122394\\
287	0.427748510679521\\
288	0.364128929618346\\
289	0.294470872913735\\
290	0.219990027846052\\
291	0.141920857502555\\
293	-0.0200346199094383\\
294	-0.101146270895697\\
295	-0.18059664985185\\
296	-0.257153218464282\\
297	-0.329455993173724\\
298	-0.396354657201755\\
299	-0.456730497659009\\
300	-0.509621465755117\\
301	-0.554112543225528\\
302	-0.589496984669495\\
303	-0.615241924146176\\
304	-0.630830105138557\\
305	-0.636051505709929\\
306	-0.630829962477947\\
307	-0.615223830533068\\
308	-0.589579574981599\\
309	-0.554222108847625\\
310	-0.509847538642589\\
311	-0.457086372751746\\
312	-0.396694603442938\\
313	-0.329873635304466\\
314	-0.257600752665439\\
315	-0.181105884686531\\
316	-0.101657716398222\\
319	0.141313405595611\\
320	0.219424707362577\\
321	0.293884828563932\\
322	0.363501223379899\\
323	0.427188820310221\\
324	0.483771861011974\\
325	0.532443597091515\\
326	0.572340380731021\\
327	0.602909423332221\\
328	0.623482478154074\\
329	0.633837891299663\\
330	0.633753817841352\\
331	0.62318660578967\\
332	0.602447448494331\\
333	0.571764074598832\\
334	0.531665498088842\\
335	0.482867936200364\\
336	0.426073928695132\\
337	0.362352570627991\\
338	0.292642511773465\\
339	0.218133540959116\\
340	0.140013926401025\\
342	-0.0217607776448858\\
343	-0.102787291380992\\
344	-0.182105250348286\\
345	-0.258462310102914\\
346	-0.330576029471104\\
347	-0.397277157275312\\
348	-0.457437268682952\\
349	-0.510124379265108\\
350	-0.554411031528161\\
351	-0.589620214151637\\
352	-0.615077608107185\\
353	-0.630469436979183\\
354	-0.635527644387821\\
355	-0.630151460333764\\
356	-0.614383084501242\\
357	-0.588523196030565\\
358	-0.552971697704379\\
359	-0.508384185925934\\
360	-0.455420509732903\\
361	-0.394998856969778\\
362	-0.328035592471224\\
363	-0.255750745473506\\
364	-0.179174937145945\\
365	-0.0996986443283276\\
368	0.143228040053145\\
369	0.221267332517527\\
370	0.295689489172219\\
371	0.36526105609164\\
372	0.42882343222027\\
373	0.485241953151217\\
374	0.533757067028546\\
375	0.573455391817788\\
376	0.603794949767689\\
377	0.624209194741525\\
378	0.634364108465434\\
379	0.634058746587016\\
380	0.623350265899717\\
381	0.602382874423711\\
382	0.571498530362987\\
383	0.531245333197603\\
384	0.482233252890182\\
385	0.425334660928911\\
386	0.361449516232824\\
387	0.291615811799147\\
388	0.216929908431666\\
389	0.138741953709086\\
391	-0.0231931182652261\\
392	-0.10424578787206\\
393	-0.183580091208114\\
394	-0.25993484645187\\
395	-0.332017761810675\\
396	-0.398644076764413\\
397	-0.458684138890021\\
398	-0.511252760359639\\
399	-0.555375562028075\\
400	-0.590419505783757\\
401	-0.615741519276071\\
402	-0.630995858204642\\
403	-0.635811543496857\\
404	-0.630202672915402\\
405	-0.614251720920493\\
406	-0.588190612739254\\
407	-0.552485819387584\\
408	-0.507628084524185\\
409	-0.454439257822287\\
410	-0.393808573012848\\
411	-0.326734318068247\\
412	-0.254300491056711\\
413	-0.177697125728628\\
414	-0.0981375251058125\\
417	0.144927852137243\\
418	0.222953857527045\\
419	0.297234946557182\\
420	0.366726875363383\\
421	0.430136879780548\\
422	0.486519380444861\\
423	0.534848943543693\\
424	0.57442860934043\\
425	0.604591062653981\\
426	0.624800110167598\\
427	0.634701784351364\\
428	0.634177882373024\\
429	0.623257363931771\\
430	0.602098926173767\\
431	0.570995350183694\\
432	0.530584672702389\\
433	0.481397771429783\\
434	0.424352860045929\\
435	0.360273820204839\\
436	0.290341911786072\\
437	0.215568828486539\\
438	0.137299354513743\\
440	-0.02468732811667\\
441	-0.105736315924787\\
442	-0.185089397455613\\
443	-0.26135364844049\\
444	-0.333325538073041\\
445	-0.399830475055296\\
446	-0.459787007948307\\
447	-0.512190731126339\\
448	-0.556206008082881\\
449	-0.591043282356623\\
450	-0.616124146198672\\
451	-0.631142045382148\\
452	-0.63576422336655\\
453	-0.629936372996326\\
454	-0.613791707585278\\
455	-0.587538592866167\\
456	-0.551628748082294\\
457	-0.50666564065159\\
458	-0.453389142263859\\
459	-0.392621684972255\\
460	-0.325425283259847\\
461	-0.252833265993559\\
462	-0.1761295300239\\
463	-0.0965047313693503\\
466	0.146433148353935\\
467	0.224379532702642\\
468	0.298603194177304\\
469	0.367909440837138\\
470	0.431189032297425\\
471	0.487400263430118\\
472	0.535583668323397\\
473	0.574990439787598\\
474	0.604930535716903\\
475	0.624933562936803\\
476	0.634677458209808\\
477	0.633955472058915\\
478	0.622851073445872\\
479	0.601480623754469\\
480	0.570283378790464\\
481	0.534136505053539\\
482	0.489323347469963\\
483	0.436269066231944\\
484	0.375725556972611\\
485	0.30851831466407\\
486	0.235571542559228\\
487	0.158030039785444\\
488	0.0770741234673551\\
490	-0.0902890703159756\\
491	-0.174065850409079\\
492	-0.256143285971575\\
493	-0.335224467415628\\
494	-0.409928058745209\\
495	-0.479153373808003\\
496	-0.541776086119171\\
497	-0.596757129545495\\
498	-0.643203878055374\\
499	-0.680374699647473\\
500	-0.707653016545464\\
501	-0.724513708199083\\
502	-0.730713684909006\\
503	-0.726206850000381\\
504	-0.711108671960119\\
505	-0.685708340583915\\
506	-0.650429580886339\\
507	-0.605949752180095\\
508	-0.553143653465668\\
509	-0.492882385159191\\
510	-0.426319057105047\\
511	-0.354734323178036\\
512	-0.279272071641117\\
514	-0.122545924438782\\
515	-0.0440331006120687\\
516	0.0326358006659575\\
517	0.106163601890785\\
518	0.175200563353883\\
519	0.238567508979941\\
520	0.295141334464915\\
521	0.34399639821504\\
522	0.384295367163759\\
523	0.415408472448235\\
524	0.436903260573672\\
525	0.448493324572155\\
526	0.45013505700399\\
527	0.442016321330357\\
528	0.424398028202631\\
529	0.397919361986624\\
530	0.363128845364372\\
531	0.321003471540735\\
532	0.272486670185117\\
533	0.218789741043565\\
534	0.161134353666512\\
535	0.100776574236988\\
537	-0.0226144491584819\\
538	-0.0828560874520008\\
539	-0.140408427921557\\
540	-0.193964810680882\\
541	-0.242427567066898\\
542	-0.289319003186847\\
543	-0.329341801479131\\
544	-0.361595030891749\\
545	-0.385232320848445\\
546	-0.399665458844993\\
547	-0.404603522041725\\
548	-0.399866256110272\\
549	-0.385577713065231\\
550	-0.362069976462863\\
551	-0.32998149155037\\
552	-0.290046442458333\\
553	-0.243169448572189\\
554	-0.190510383486526\\
555	-0.133342074423126\\
556	-0.0730316537546969\\
559	0.112653054613475\\
560	0.171121467752073\\
561	0.225582357150415\\
562	0.274655691427597\\
563	0.317163233934934\\
564	0.352157501718921\\
565	0.378774566246648\\
566	0.396344757390125\\
567	0.404487576850897\\
568	0.402987590418888\\
569	0.391922425549637\\
570	0.371512071903453\\
571	0.342260737101697\\
572	0.304914357649068\\
573	0.260266446915921\\
574	0.209447850144443\\
575	0.153649986101755\\
576	0.0942027006490207\\
579	-0.091646681983093\\
580	-0.151179866332768\\
581	-0.207147583016649\\
582	-0.25817748063082\\
583	-0.303087929275534\\
584	-0.340754183707304\\
585	-0.37029178368266\\
586	-0.390986733524642\\
587	-0.402368154893793\\
588	-0.404151727471799\\
589	-0.39631303419219\\
590	-0.379081798790139\\
591	-0.352775211033531\\
592	-0.318110391018308\\
593	-0.275875605572764\\
594	-0.227024847851681\\
595	-0.172777739390312\\
596	-0.11441868222073\\
597	-0.0532703249023143\\
599	0.0713519013543191\\
600	0.131833334763996\\
601	0.189192700535841\\
602	0.242068843016568\\
603	0.289243702496606\\
604	0.329457435875611\\
605	0.361886148330996\\
606	0.385765281822842\\
607	0.400408859557956\\
608	0.40558382150175\\
609	0.40110183870911\\
610	0.387038170119922\\
611	0.363795557410413\\
612	0.331952789847946\\
613	0.292226529112213\\
614	0.245548655156199\\
615	0.193048753006678\\
616	0.135960721710489\\
617	0.0756253827316868\\
620	-0.110209755823234\\
621	-0.168802579742987\\
622	-0.223376722745343\\
623	-0.272635643664671\\
624	-0.315345960590548\\
625	-0.350585596983365\\
626	-0.377492373974292\\
627	-0.395389378780237\\
628	-0.403838232263297\\
629	-0.402639202814953\\
630	-0.391881822749838\\
631	-0.371814974771041\\
632	-0.342877969571873\\
633	-0.305759998558642\\
634	-0.261425593388594\\
635	-0.210795880890146\\
636	-0.155176858087088\\
637	-0.095886788745247\\
641	0.149179870078569\\
642	0.205116612808979\\
643	0.25613897613357\\
644	0.301070899030492\\
645	0.33883058475476\\
646	0.368455850458304\\
647	0.389273577687163\\
648	0.400792112322961\\
649	0.402736987134631\\
650	0.395105034735025\\
651	0.37808142791755\\
652	0.352051537621719\\
653	0.317614361799315\\
654	0.275629747913172\\
655	0.227037172742257\\
656	0.172984514515974\\
657	0.11482956758573\\
658	0.0539285986337745\\
660	-0.0703652073548255\\
661	-0.130724844094402\\
662	-0.187979484926473\\
663	-0.240780681997421\\
664	-0.287885154507876\\
665	-0.328144985955532\\
666	-0.360552025714242\\
667	-0.384422428791822\\
668	-0.399084761113954\\
669	-0.404254461023811\\
670	-0.399853258342318\\
671	-0.385901095273312\\
672	-0.362816962421221\\
673	-0.331070028824797\\
674	-0.29147978277706\\
675	-0.244906274754157\\
676	-0.192524425737702\\
677	-0.135542552030074\\
678	-0.0752910119304033\\
681	0.110267550950084\\
682	0.168862580389487\\
683	0.223430198245296\\
684	0.272674987643768\\
685	0.315372218264656\\
686	0.350601415193523\\
687	0.377503749705284\\
688	0.395455954454519\\
689	0.403900177007472\\
690	0.402708380047443\\
691	0.391976430287286\\
692	0.371891184910965\\
693	0.34297702734284\\
694	0.30584642685244\\
695	0.261500626517318\\
696	0.210831597978995\\
697	0.155200740672171\\
698	0.095829846698507\\
702	-0.149637908684326\\
703	-0.205687254251188\\
704	-0.256863396280551\\
705	-0.301907862788084\\
706	-0.339813912918544\\
707	-0.369542950517371\\
708	-0.390535814952727\\
709	-0.40219100696595\\
710	-0.404317703521883\\
711	-0.396841039965238\\
712	-0.379945607133322\\
713	-0.354090310555421\\
714	-0.31974084221747\\
715	-0.277804753397504\\
716	-0.229233975310763\\
717	-0.175217107112985\\
718	-0.117046866512737\\
719	-0.0561404504637721\\
721	0.0681578263634037\\
722	0.128631603989561\\
723	0.18601588049205\\
724	0.238914442259556\\
725	0.286089528298362\\
726	0.326502774854816\\
727	0.359113345140486\\
728	0.383108212868137\\
729	0.397976680734246\\
730	0.403340146776372\\
731	0.39908585432886\\
732	0.385344232036914\\
733	0.362433279489778\\
734	0.330789664854819\\
735	0.291308924970508\\
736	0.244839292465713\\
737	0.192530014428485\\
738	0.135576015246443\\
739	0.0753770190908654\\
742	-0.110132110263748\\
743	-0.168735466980252\\
744	-0.223316741532926\\
745	-0.272645509818631\\
746	-0.315446866170078\\
747	-0.350772472361768\\
748	-0.377743181080405\\
749	-0.395740671863678\\
750	-0.404321037864975\\
751	-0.403289469916217\\
752	-0.39270570323788\\
753	-0.372749939204823\\
754	-0.343974934470225\\
755	-0.306953503098157\\
756	-0.26262995133493\\
757	-0.212054305554375\\
758	-0.156497814353315\\
759	-0.0972469070470652\\
763	0.148318554592151\\
764	0.204499544311602\\
765	0.255777448394838\\
766	0.300973191255252\\
767	0.338997214852043\\
768	0.368862321055076\\
769	0.390010952489774\\
770	0.401854145099605\\
771	0.404191177879056\\
772	0.396872447163787\\
773	0.380071017554656\\
774	0.354304439954376\\
775	0.320093390135753\\
776	0.278239509239938\\
777	0.22973951455424\\
778	0.175774871459453\\
779	0.117607491228227\\
780	0.0566711118299281\\
782	-0.0676964963422506\\
783	-0.128224546794627\\
784	-0.185747950119094\\
785	-0.238721839961272\\
786	-0.286030510862929\\
787	-0.326575963242249\\
788	-0.359346122128954\\
789	-0.383533589958461\\
790	-0.398611271828031\\
791	-0.404173242204706\\
792	-0.400102247961968\\
793	-0.386540227400474\\
794	-0.363781008761634\\
795	-0.332299826775397\\
796	-0.292919973195239\\
797	-0.246528550095718\\
798	-0.194287218502268\\
799	-0.137401544090608\\
800	-0.0772365417187757\\
801	-0.0253320103629449\\
803	0.0835265398227421\\
804	0.138178199015329\\
805	0.191508667836843\\
806	0.242419403878557\\
807	0.289767863331235\\
808	0.332581254878278\\
809	0.369855506486147\\
810	0.400791437973567\\
811	0.424612939689268\\
812	0.440760845940531\\
813	0.448763067171058\\
814	0.448167367583665\\
815	0.438911093217939\\
816	0.420885383922268\\
817	0.394286472901967\\
818	0.359393116973024\\
819	0.316711491874685\\
820	0.26669886489799\\
821	0.210056461954082\\
822	0.147582371258068\\
823	0.0802403490979486\\
824	0.00899463215455398\\
825	-0.065120624422434\\
829	-0.36766294414565\\
830	-0.439259122655585\\
831	-0.507101745706677\\
832	-0.570276924911013\\
833	-0.62774647822016\\
834	-0.678709804744813\\
835	-0.722583492645299\\
836	-0.758720662954602\\
837	-0.786606514930099\\
838	-0.805922628418557\\
839	-0.816394902265984\\
840	-0.817970868055909\\
841	-0.810638037648914\\
842	-0.794581815720449\\
843	-0.770031622580518\\
844	-0.737354183021125\\
845	-0.69698357113748\\
846	-0.649608877176888\\
847	-0.595727573786917\\
848	-0.536079372696349\\
849	-0.471447948225887\\
850	-0.402641443210541\\
851	-0.330551501941045\\
852	-0.256173663945901\\
856	0.0477788200587383\\
857	0.120895701952577\\
858	0.191132573928371\\
859	0.257883454583862\\
860	0.320296310613685\\
861	0.377720219531966\\
862	0.439952284378705\\
863	0.494850437918103\\
864	0.541657531485725\\
865	0.579510337854117\\
866	0.607852276426001\\
867	0.62618413605378\\
868	0.634253588464162\\
869	0.63186011629432\\
870	0.619094576640691\\
871	0.596069707644801\\
872	0.563268493269788\\
873	0.521194299901254\\
874	0.470502114671945\\
875	0.41215513340876\\
876	0.346979776313674\\
877	0.276079124857006\\
878	0.200636776290594\\
879	0.121896914742592\\
882	-0.121119242344321\\
883	-0.199911599874667\\
884	-0.275437224844154\\
885	-0.346507285158623\\
886	-0.411844166635547\\
887	-0.470381595232084\\
888	-0.52116584675332\\
889	-0.563465030936186\\
890	-0.596520047072318\\
891	-0.619767269076874\\
892	-0.632839399698241\\
893	-0.635523575630486\\
894	-0.62779346332627\\
895	-0.609737772295375\\
896	-0.581633508638788\\
897	-0.544033090396169\\
898	-0.497486622160977\\
899	-0.442769400491215\\
900	-0.380793573572191\\
901	-0.312496298612132\\
902	-0.239125478403366\\
903	-0.161802767316658\\
904	-0.0818150925365444\\
906	0.0807989980025923\\
907	0.160836718920564\\
908	0.238223317370284\\
909	0.311648258320474\\
910	0.379877155949089\\
911	0.441898969752856\\
912	0.496641745222405\\
913	0.543344987623641\\
914	0.581026781242144\\
915	0.609176494013809\\
916	0.627277975044308\\
917	0.635096515367877\\
918	0.632473959494746\\
919	0.619488959665432\\
920	0.59628539624282\\
921	0.563281920589816\\
922	0.52103911488939\\
923	0.470179774395092\\
924	0.411661595426722\\
925	0.346350101152893\\
926	0.275414879707114\\
927	0.199889087863085\\
928	0.121072407035172\\
931	-0.122090177486371\\
932	-0.200885879411089\\
933	-0.276329149057347\\
934	-0.347259951184242\\
935	-0.412451805500041\\
936	-0.47091641462066\\
937	-0.521594140785965\\
938	-0.563679181581392\\
939	-0.596549735087592\\
940	-0.619629069657549\\
941	-0.632528650948643\\
942	-0.635006718688146\\
943	-0.627026657142778\\
944	-0.608750890211468\\
945	-0.580472755547817\\
946	-0.542595895738032\\
947	-0.495869772749757\\
948	-0.440980226136162\\
949	-0.378871294795772\\
950	-0.310495014173284\\
951	-0.237032254618498\\
952	-0.159713981873665\\
953	-0.0796663349206028\\
955	0.0830748083262733\\
956	0.163029694825582\\
957	0.240334553177945\\
958	0.313651950534222\\
959	0.381860862853955\\
960	0.443814198714335\\
961	0.498505704591025\\
962	0.544970332342928\\
963	0.582545037198088\\
964	0.61051228264887\\
965	0.628453977850768\\
966	0.636041383861993\\
967	0.63324880635173\\
968	0.619969979481539\\
969	0.596509139366844\\
970	0.56331558445072\\
971	0.520923432652125\\
972	0.469908199927431\\
973	0.411236229035239\\
974	0.345750337709887\\
975	0.274677798979155\\
976	0.199062801713353\\
977	0.120201024119524\\
980	-0.122894145928512\\
981	-0.201688452340477\\
982	-0.277112335170386\\
983	-0.348000196063367\\
984	-0.413132626019433\\
985	-0.471487799367878\\
986	-0.522024718221019\\
987	-0.564058536917855\\
988	-0.596829092678945\\
989	-0.6197420592257\\
990	-0.63243799495649\\
991	-0.634770863718131\\
992	-0.62659829241511\\
993	-0.608140530509445\\
994	-0.579708295455021\\
995	-0.541734242502571\\
996	-0.494871655182578\\
997	-0.439842668890833\\
998	-0.377607200367038\\
999	-0.309185459395394\\
1000	-0.235685117327193\\
1001	-0.158233812495837\\
1002	-0.0781997769822738\\
1004	0.0843683584230348\\
1005	0.164200940322644\\
1006	0.241431692075821\\
1007	0.314588655843636\\
1008	0.382683205416924\\
1009	0.44445468584172\\
1010	0.498895278487907\\
1011	0.545153815252888\\
1012	0.582417936791444\\
1013	0.610173394817139\\
1014	0.627883552094318\\
1015	0.635289636109974\\
1016	0.632299004850211\\
1017	0.618902789219646\\
1018	0.595309231894589\\
1019	0.562013127758291\\
1020	0.519414317653627\\
1021	0.46823442314826\\
1022	0.409400751011162\\
1023	0.343826401394381\\
1024	0.272558991198366\\
1025	0.196844671701001\\
1026	0.117902337694886\\
1029	-0.125079747966993\\
1030	-0.203702937714752\\
1031	-0.278985497008762\\
1032	-0.349674392699853\\
1033	-0.414612630618194\\
1034	-0.472767743541226\\
1035	-0.523102553927401\\
1036	-0.564872959179866\\
1037	-0.597362331285694\\
1038	-0.620020317313447\\
1039	-0.632450190907093\\
1040	-0.634509381017779\\
1041	-0.626134715848821\\
1042	-0.607410012059518\\
1043	-0.578749260542736\\
1044	-0.540564881191131\\
1045	-0.493461723159726\\
1046	-0.43827940420897\\
1047	-0.375901278488527\\
1048	-0.307270759883522\\
1049	-0.233537307718962\\
1050	-0.155976309121343\\
1051	-0.0758424850300798\\
1053	0.0867312526770547\\
1054	0.166577253423839\\
1055	0.243684957351434\\
1056	0.31676710734564\\
1057	0.384744597765803\\
1058	0.446361205396443\\
1059	0.500633471747278\\
1060	0.546708131194464\\
1061	0.583887583194155\\
1062	0.611390416009272\\
1063	0.62884616376914\\
1064	0.635961287462578\\
1065	0.632677722799599\\
1066	0.61904001660605\\
1067	0.595184284177321\\
1068	0.561580275023971\\
1069	0.518680823909108\\
1070	0.467283463011427\\
1071	0.408300345424777\\
1072	0.342588436401002\\
1073	0.271301057364781\\
1074	0.195540371594689\\
1075	0.116527686902373\\
1078	-0.126720645164369\\
1079	-0.205418329569966\\
1080	-0.280766235934152\\
1081	-0.351428922934247\\
1082	-0.41626069854874\\
1083	-0.474334678625382\\
1084	-0.524572251969857\\
1085	-0.566193493781157\\
1086	-0.598525875775522\\
1087	-0.621024863578896\\
1088	-0.633262455500699\\
1089	-0.635128238553079\\
1090	-0.62659124975653\\
1091	-0.607787783651929\\
1092	-0.578995168466918\\
1093	-0.540683465805159\\
1094	-0.493486526472225\\
1095	-0.438145606453418\\
1096	-0.375629936908808\\
1097	-0.306973528959588\\
1098	-0.233263524751237\\
1099	-0.155720745588951\\
1100	-0.0756230268734726\\
1102	0.0869635437766192\\
1103	0.166771217898713\\
1104	0.243835927873079\\
1105	0.316908375862567\\
1106	0.384707946181152\\
1107	0.446266893302891\\
1108	0.500419521441927\\
1109	0.546338959663444\\
1110	0.583297376259907\\
1111	0.610644844542094\\
1112	0.627919099393239\\
1113	0.634931233241332\\
1114	0.631542392774008\\
1115	0.6177337064114\\
1116	0.593794271117076\\
1117	0.560090495412169\\
1118	0.517110825239342\\
1119	0.465679068649933\\
1120	0.406595399539128\\
1121	0.340819551479854\\
1122	0.269420287264438\\
1123	0.193575870426685\\
1124	0.114669804531786\\
1127	-0.128185209531694\\
1128	-0.20669985905397\\
1129	-0.281802859001345\\
1130	-0.352259716074059\\
1131	-0.41690515450091\\
1132	-0.47476375491442\\
1133	-0.524784029007151\\
1134	-0.566186056692914\\
1135	-0.598255344331392\\
1136	-0.62048668295256\\
1137	-0.63249480839977\\
1138	-0.634204068168856\\
1139	-0.625429389976034\\
1140	-0.606352113757112\\
1141	-0.577245388918072\\
1142	-0.538738593254038\\
1143	-0.491331975072171\\
1144	-0.435845616809729\\
1145	-0.373204268834797\\
1146	-0.304411968467321\\
1147	-0.230581616191103\\
1148	-0.152966895698228\\
1149	-0.0729412016034985\\
1151	0.0895936391102623\\
1152	0.169313408202015\\
1153	0.2463209547509\\
1154	0.319229323939453\\
1155	0.386961776972839\\
1156	0.448245081275218\\
1157	0.502197648661877\\
1158	0.547934213722328\\
1159	0.584694480407506\\
1160	0.611824220720791\\
1161	0.62887411970496\\
1162	0.635651616257292\\
1163	0.632001945919455\\
1164	0.617893240867943\\
1165	0.593688163489787\\
1166	0.559717549949255\\
1167	0.516597450606241\\
1168	0.464945503017589\\
1169	0.405668741251247\\
1170	0.33973968090595\\
1171	0.268197433716523\\
1172	0.192263673251091\\
1173	0.113215609915642\\
1176	-0.129946762911004\\
1177	-0.208481514181585\\
1178	-0.283550297077909\\
1179	-0.353988609656426\\
1180	-0.418639247595365\\
1181	-0.476400901557099\\
1182	-0.526293567413632\\
1183	-0.567558621148692\\
1184	-0.599477520998789\\
1185	-0.621579391651721\\
1186	-0.633453179400931\\
1187	-0.634877126936772\\
1188	-0.625922345189792\\
1189	-0.606722824644748\\
1190	-0.5775086079434\\
1191	-0.538838403547288\\
1192	-0.491275880206558\\
1193	-0.435590759744628\\
1194	-0.372860001997196\\
1195	-0.303987603685528\\
1196	-0.230074243267609\\
1197	-0.152413160429205\\
1198	-0.0722405833166704\\
1200	0.0903571616981935\\
1201	0.170098708830665\\
1202	0.247003543136998\\
1203	0.319893939225949\\
1204	0.387493044343955\\
1205	0.448698610395695\\
1206	0.502540153883729\\
1207	0.548129983300896\\
1208	0.584731659870158\\
1209	0.611724372055505\\
1210	0.628705157241257\\
1211	0.635278771243065\\
1212	0.631391655926564\\
1213	0.617198878728232\\
1214	0.592781684716101\\
1215	0.558700178213712\\
1216	0.515390208011013\\
1217	0.463629648904771\\
1218	0.404246594061533\\
1219	0.338192556625927\\
1220	0.266560881372243\\
1221	0.190602028832927\\
1222	0.111507697668003\\
1225	-0.131609600143747\\
1226	-0.210075614650123\\
1227	-0.285059438613189\\
1228	-0.35539438602018\\
1229	-0.419856980328404\\
1230	-0.477390984820431\\
1231	-0.527097700063223\\
1232	-0.56813254065537\\
1233	-0.599806710361918\\
1234	-0.621673940961955\\
1235	-0.633234732801156\\
1236	-0.634484388789588\\
1237	-0.625173233696842\\
1238	-0.605644857885181\\
1239	-0.576199652293326\\
1240	-0.537282122783836\\
1241	-0.489438878714736\\
1242	-0.433515514139799\\
1243	-0.37051021584648\\
1244	-0.301356801877773\\
1245	-0.227254819962127\\
1246	-0.149463253773774\\
1248	0.01216676530521\\
1249	0.0934183530043811\\
1250	0.173158556334783\\
1251	0.250089638303507\\
1252	0.32293875461437\\
1253	0.3904245526046\\
1254	0.45149732033633\\
1255	0.505258052279714\\
1256	0.55072763917633\\
1257	0.587097119853752\\
1258	0.613839427192033\\
1259	0.630507443569968\\
1260	0.636868291906922\\
1261	0.632785628807596\\
1262	0.618321880666826\\
1263	0.593669032794423\\
1264	0.55931823347737\\
1265	0.515784319799877\\
1266	0.463784769793165\\
1267	0.404183652448864\\
1268	0.337977589466391\\
1269	0.266241348655967\\
1270	0.190113997050958\\
1271	0.110902282472125\\
1274	-0.132103804616236\\
1275	-0.210515793181457\\
1276	-0.285457987982227\\
1277	-0.355724204569469\\
1278	-0.420113629513708\\
1279	-0.477545605657269\\
1280	-0.527080966138328\\
1281	-0.556921870156202\\
1282	-0.576983385053154\\
1283	-0.587174971330569\\
1284	-0.587742556677767\\
1285	-0.579112933839951\\
1286	-0.561849016095948\\
1287	-0.536757688222224\\
1288	-0.50474908753722\\
1289	-0.466765150991705\\
1290	-0.423940710460101\\
1291	-0.377538234441545\\
1293	-0.27874514814448\\
1294	-0.228870779230874\\
1295	-0.180210813113717\\
1296	-0.133992042824502\\
1297	-0.0911327189837721\\
1298	-0.0526832710497729\\
1299	-0.0193383441437618\\
1300	0.00828997263079145\\
1301	0.0297521190518637\\
1302	0.0446699885119415\\
1303	0.0529540518282374\\
1304	0.0547190980551022\\
1305	0.0501105517787437\\
1306	0.0396261952837449\\
1307	0.0238100193400896\\
1308	0.00332375478774338\\
1309	-0.0210194871774547\\
1310	-0.0483369378616771\\
1312	-0.10793835565164\\
1314	-0.167327663233209\\
1315	-0.194380060842832\\
1316	-0.218395352183961\\
1317	-0.238461938564342\\
1318	-0.253791803817421\\
1319	-0.263771177471426\\
1320	-0.267858585803879\\
1321	-0.2657837304846\\
1322	-0.25721522038566\\
1323	-0.242176790607573\\
1324	-0.220725780930479\\
1325	-0.193266082639184\\
1326	-0.160222439103563\\
1327	-0.122096009795769\\
1328	-0.0797858402161182\\
1329	-0.0341159861395681\\
1330	0.0139667542052848\\
1333	0.162094136659562\\
1334	0.209068988182935\\
1335	0.253064077966428\\
1336	0.292956875056916\\
1337	0.327731666608088\\
1338	0.35664095965376\\
1339	0.378904121803771\\
1340	0.39383320255547\\
1341	0.401018455668691\\
1342	0.388885767550619\\
1343	0.367519323734086\\
1344	0.337366664503406\\
1345	0.299170854055774\\
1346	0.253801343453233\\
1347	0.20239695561304\\
1348	0.146181731096476\\
1349	0.0865114365474255\\
1352	-0.0990960339463527\\
1353	-0.158185512221735\\
1354	-0.213549288302602\\
1355	-0.263846338876647\\
1356	-0.307855162148371\\
1357	-0.344559240962099\\
1358	-0.373048059638222\\
1359	-0.392625037882681\\
1360	-0.402848689845086\\
1361	-0.40347555057042\\
1362	-0.394513260504937\\
1363	-0.376164681485989\\
1364	-0.348830206145522\\
1365	-0.313193370121098\\
1366	-0.270104351987811\\
1367	-0.220574756216593\\
1368	-0.165804614701301\\
1369	-0.107026768192554\\
1371	0.0166350335803145\\
1372	0.0785939398374467\\
1373	0.138733640686041\\
1374	0.195605695019822\\
1375	0.247765455331319\\
1376	0.294069706641494\\
1377	0.333336023530137\\
1378	0.364750526714943\\
1379	0.387431023454155\\
1380	0.400931110668807\\
1381	0.40491314590372\\
1382	0.399273645847188\\
1383	0.384116189457473\\
1384	0.359780549730658\\
1385	0.326890112657111\\
1386	0.286268277352974\\
1387	0.238856847693114\\
1388	0.185682216358146\\
1389	0.128094777739989\\
1390	0.0675067484926331\\
1392	-0.0570334490457753\\
1393	-0.117940537739742\\
1394	-0.176092689795041\\
1395	-0.230004054347773\\
1396	-0.278520992610538\\
1397	-0.320359431534598\\
1398	-0.354547422575706\\
1399	-0.380258207281258\\
1400	-0.396996648354616\\
1401	-0.404251495822336\\
1402	-0.401896936129106\\
1403	-0.389981227214321\\
1404	-0.368771063843724\\
1405	-0.338750603802964\\
1406	-0.300651657697017\\
1407	-0.255411703692516\\
1408	-0.204108387857104\\
1409	-0.147963599165905\\
1410	-0.0882659846770366\\
1413	0.0974352397838629\\
1414	0.156600053036072\\
1415	0.212074897366165\\
1416	0.262521287514573\\
1417	0.306644323514774\\
1418	0.343524441494083\\
1419	0.372213968790675\\
1420	0.392006432885864\\
1421	0.402469836128603\\
1422	0.40334124123865\\
1423	0.394610080997609\\
1424	0.376513405617061\\
1425	0.349430352376658\\
1426	0.314016144036486\\
1427	0.271119859515238\\
1428	0.22179768937167\\
1429	0.16720730769066\\
1430	0.108584418819646\\
1432	-0.0149209413871176\\
1433	-0.0768964010421769\\
1434	-0.137057428953085\\
1435	-0.193939800712542\\
1436	-0.246149154746945\\
1437	-0.29251241960219\\
1438	-0.33185492294524\\
1439	-0.363370958717042\\
1440	-0.386192190484962\\
1441	-0.39448487726213\\
1442	-0.393878227285313\\
1443	-0.384427125703951\\
1444	-0.36645169532585\\
1445	-0.340426744732667\\
1446	-0.307061003689341\\
1447	-0.267180883752189\\
1448	-0.221829002354298\\
1449	-0.171954078697581\\
1450	-0.118870593065367\\
1453	0.0465073526784181\\
1454	0.0991744106927399\\
1455	0.148528200571491\\
1456	0.193257431215443\\
1457	0.232337153794106\\
1458	0.264688780425786\\
1459	0.289535077656637\\
1460	0.306192542048393\\
1461	0.314374899536688\\
1462	0.313644261342688\\
1463	0.303999647474029\\
1464	0.285508716604909\\
1465	0.258521411069978\\
1466	0.223577611939618\\
1467	0.181334757110562\\
1468	0.132801101519362\\
1469	0.0789694955128653\\
1470	0.0209141893669766\\
1471	-0.0400065215267205\\
1474	-0.226308355005131\\
1475	-0.28492125572393\\
1476	-0.339513958796942\\
1477	-0.388791820882489\\
1478	-0.431525646551108\\
1479	-0.466715366142125\\
1480	-0.493484344799981\\
1481	-0.5110105808958\\
1482	-0.518790653294673\\
1483	-0.516497221520694\\
1484	-0.503967314507008\\
1485	-0.481229940377943\\
1486	-0.448585471538536\\
1487	-0.406508587839653\\
1488	-0.355663976870801\\
1489	-0.296978046101231\\
1490	-0.231436483619746\\
1491	-0.160179220538339\\
1492	-0.0845737207073398\\
1493	-0.00596495971285549\\
1495	0.154324112502763\\
1496	0.232952329641194\\
1497	0.308560934470279\\
1498	0.379745008565351\\
1499	0.445129845612428\\
1500	0.503435052678469\\
1501	0.553539103344065\\
1502	0.588971444329673\\
1503	0.614721642098175\\
1504	0.630396726640356\\
1505	0.635668842914129\\
1506	0.630531908241664\\
1507	0.615033609775764\\
1508	0.589408001852462\\
1509	0.55412716192177\\
1510	0.509599239518593\\
1511	0.456726775587867\\
1512	0.39637470009302\\
1513	0.329541839815192\\
1514	0.257317483044517\\
1515	0.180866349116968\\
1516	0.101441446998706\\
1519	-0.141558792568958\\
1520	-0.21958999342587\\
1521	-0.293991201260724\\
1522	-0.363635870261533\\
1523	-0.427315931049179\\
1524	-0.483963438846331\\
1525	-0.53263677726045\\
1526	-0.572584270367315\\
1527	-0.603137885415435\\
1528	-0.623688206542738\\
1529	-0.634045412312389\\
1530	-0.63398382538071\\
1531	-0.623372899238348\\
1532	-0.602584214654144\\
1533	-0.571896709686825\\
1534	-0.531813527002669\\
1535	-0.483025472496138\\
1536	-0.426261248303945\\
1537	-0.362461066855758\\
1538	-0.292683244172167\\
1539	-0.218139335182968\\
1540	-0.140000166222762\\
1542	0.0217847038525179\\
1543	0.102836206834127\\
1544	0.182199379545182\\
1545	0.258597502443536\\
1546	0.330754342069213\\
1547	0.397444415540122\\
1548	0.457617661872973\\
1549	0.510280957936175\\
1550	0.554554735979764\\
1551	0.589701682050872\\
1552	0.615139109048414\\
1553	0.630516578151855\\
1554	0.635503002733458\\
1555	0.630133565766755\\
1556	0.614324829889483\\
1557	0.588402393466367\\
1558	0.552875187457175\\
1559	0.508202459762288\\
1560	0.455149707738201\\
1561	0.394661194799937\\
1562	0.327701074307697\\
1563	0.255346012797872\\
1564	0.178780940388151\\
1565	0.0992686158792822\\
1568	-0.143645302287041\\
1569	-0.221695301081127\\
1570	-0.296086191674476\\
1571	-0.365536505868022\\
1572	-0.428966927695001\\
1573	-0.485434147989508\\
1574	-0.533877229760947\\
1575	-0.573645778476475\\
1576	-0.60398830695749\\
1577	-0.624378145623723\\
1578	-0.634504262759947\\
1579	-0.634216195661338\\
1580	-0.62349916855419\\
1581	-0.602563718126476\\
1582	-0.571753195451493\\
1583	-0.531503657093253\\
1584	-0.482538604197089\\
1585	-0.425669323137754\\
1586	-0.361798779002129\\
1587	-0.29197143602687\\
1588	-0.217328318874934\\
1589	-0.139203158886176\\
1591	0.0226410583295547\\
1592	0.103617656539427\\
1593	0.182905370293611\\
1594	0.259216510778515\\
1595	0.331323802181942\\
1596	0.397935349712952\\
1597	0.457977484302319\\
1598	0.510418461441077\\
1599	0.554473670576044\\
1600	0.589467631413754\\
1601	0.614763672040226\\
1602	0.629981731868156\\
1603	0.634835626429322\\
1604	0.629231891098698\\
1605	0.613269492675499\\
1606	0.587285060246813\\
1607	0.551573939756508\\
1608	0.506812613576585\\
1609	0.453765803713395\\
1610	0.39320798311519\\
1611	0.326197347274956\\
1612	0.253821396607236\\
1613	0.177254900812841\\
1614	0.097745141094947\\
1617	-0.145091669857266\\
1618	-0.22299878408694\\
1619	-0.29724877413264\\
1620	-0.366599978312479\\
1621	-0.429915866083775\\
1622	-0.48623020621244\\
1623	-0.534593269933339\\
1624	-0.574139902071693\\
1625	-0.604204726284024\\
1626	-0.624350151006183\\
1627	-0.634297732564846\\
1628	-0.633858438537118\\
1629	-0.622919842131978\\
1630	-0.601725537884249\\
1631	-0.570627847401738\\
1632	-0.530195801949503\\
1633	-0.481051080564157\\
1634	-0.42398103198002\\
1635	-0.359954280599595\\
1636	-0.289984130739867\\
1637	-0.215214067288798\\
1638	-0.136976994619545\\
1640	0.0250488730121106\\
1641	0.106105104882772\\
1642	0.185406353664803\\
1643	0.261687217159761\\
1644	0.333673950889988\\
1645	0.400211645745458\\
1646	0.460191719038448\\
1647	0.512564308390665\\
1648	0.556515825961469\\
1649	0.591342712419191\\
1650	0.616496533968075\\
1651	0.631446117266478\\
1652	0.636067717689912\\
1653	0.630229950582816\\
1654	0.614070089804954\\
1655	0.587807606053957\\
1656	0.551797724425342\\
1657	0.506765066079424\\
1658	0.453445752366861\\
1659	0.392685850632461\\
1660	0.325485466944883\\
1661	0.252908954375016\\
1662	0.176153579419861\\
1663	0.0965105721465989\\
1666	-0.146533624363656\\
1667	-0.224493455399625\\
1668	-0.298788250067446\\
1669	-0.3681494590287\\
1670	-0.431491840944545\\
1671	-0.487690405582725\\
1672	-0.535914017231789\\
1673	-0.57536958593073\\
1674	-0.605288161844328\\
1675	-0.625283754367501\\
1676	-0.634986272507831\\
1677	-0.634336409061234\\
1678	-0.623223384536686\\
1679	-0.601846671100702\\
1680	-0.570631809493534\\
1681	-0.530023106886347\\
1682	-0.480716298242896\\
1683	-0.423471945470283\\
1684	-0.359212527915133\\
1685	-0.289101226831008\\
1686	-0.214294156121014\\
1687	-0.135911385499185\\
1689	0.0262794176824173\\
1690	0.107404676426995\\
1691	0.186695592386059\\
1692	0.26289546666294\\
1693	0.334838265329381\\
1694	0.401285697101684\\
1695	0.461141984971619\\
1696	0.513374596714129\\
1697	0.557178709384516\\
1698	0.591815178202978\\
1699	0.616791517667934\\
1700	0.631553468575021\\
1701	0.635985667730893\\
1702	0.629980021029041\\
1703	0.613645660638213\\
1704	0.587191278461432\\
1705	0.55107496975188\\
1706	0.505899949753257\\
1707	0.452424592205261\\
1708	0.39150160166264\\
1709	0.324236239385755\\
1710	0.251578450057877\\
1711	0.174778353109105\\
1712	0.0951481309643896\\
1715	-0.147780046306252\\
1716	-0.225670695156623\\
1717	-0.299808307836429\\
1718	-0.369002337911752\\
1719	-0.432154914482908\\
1720	-0.488118800745724\\
1721	-0.536131559931619\\
1722	-0.575288844212992\\
1723	-0.604985745833346\\
1724	-0.62474519901707\\
1725	-0.634237466838385\\
1726	-0.63327851315853\\
1727	-0.621844082400003\\
1728	-0.600262114720408\\
1729	-0.568814731960629\\
1730	-0.528057341251042\\
1731	-0.478601966628048\\
1732	-0.421277209427899\\
1733	-0.356980484089945\\
1734	-0.286830858923622\\
1735	-0.211991732560364\\
1736	-0.133671045957271\\
1738	0.0283039739524611\\
1739	0.109240258355385\\
1740	0.188370184463565\\
1741	0.264445014618559\\
1742	0.336131222040876\\
1743	0.402308320799875\\
1744	0.461886701238654\\
1745	0.513858131455891\\
1746	0.557399897552386\\
1747	0.591811448070985\\
1748	0.616428726986214\\
1749	0.630958660115539\\
1750	0.635125798995887\\
1751	0.628785871597756\\
1752	0.612186066544837\\
1753	0.585529823458273\\
1754	0.549270039589373\\
1755	0.503960337232002\\
1756	0.450390558086383\\
1757	0.389416154720038\\
1758	0.322036897620364\\
1759	0.249390756045159\\
1760	0.172541887118768\\
1762	0.00733138782516107\\
1763	-0.0732609938290807\\
1764	-0.15072705410239\\
1765	-0.223776617989188\\
1766	-0.291428110351717\\
1767	-0.352719365218036\\
1768	-0.406846510235482\\
1769	-0.453168267397814\\
1770	-0.491387626248979\\
1771	-0.521189087831772\\
1772	-0.542522639690105\\
1773	-0.555585775472537\\
1774	-0.560612669564762\\
1775	-0.558180924816497\\
1776	-0.548878286454965\\
1777	-0.533432195409659\\
1778	-0.512670594714564\\
1779	-0.487596841070626\\
1780	-0.459152651186287\\
1781	-0.428485723187805\\
1784	-0.332948148394735\\
1785	-0.303177080776095\\
1786	-0.275925160744009\\
1787	-0.251821027914502\\
1788	-0.23155465992113\\
1789	-0.215549466515768\\
1790	-0.204045850127386\\
1791	-0.197245810243203\\
1792	-0.195160668394692\\
1793	-0.197632643931684\\
1794	-0.204418001276281\\
1795	-0.214991730901147\\
1796	-0.228877712662779\\
1797	-0.245378504843302\\
1799	-0.2831546802704\\
1801	-0.321644261174242\\
1802	-0.338993992992982\\
1803	-0.353889306001747\\
1804	-0.365548162204959\\
1805	-0.373231959288205\\
1806	-0.376266008011953\\
1807	-0.374093784720117\\
1808	-0.366337891445255\\
1809	-0.352715642292878\\
1810	-0.333067696521994\\
1811	-0.307451844777916\\
1812	-0.2760476781541\\
1813	-0.239149146289037\\
1814	-0.197311855812131\\
1815	-0.151140104670503\\
1816	-0.101313872240553\\
1817	-0.0487179970955367\\
1819	0.0608424412521344\\
1820	0.115733480666677\\
1822	0.22382731598691\\
1823	0.272873425657508\\
1824	0.315456648484542\\
1825	0.350540507060487\\
1826	0.377253766188005\\
1827	0.3949434065471\\
1828	0.403272115122036\\
1829	0.401943753065098\\
1830	0.391025500916385\\
1831	0.370821703906131\\
1832	0.341747665571347\\
1833	0.304521291423498\\
1834	0.260049077534404\\
1835	0.209352011392184\\
1836	0.153691581188923\\
1837	0.0944054495735145\\
1841	-0.150884097104608\\
1842	-0.20686846806575\\
1843	-0.257875540279656\\
1844	-0.302772828394609\\
1845	-0.340472009011592\\
1846	-0.370027920091616\\
1847	-0.390808944989658\\
1848	-0.402305750940741\\
1849	-0.404164721039706\\
1850	-0.396448164035974\\
1851	-0.379245705836638\\
1852	-0.352982636628894\\
1853	-0.318383176831958\\
1854	-0.276180956717781\\
1855	-0.227373693192931\\
1856	-0.173156571689105\\
1857	-0.114814887568627\\
1858	-0.0537132366848709\\
1860	0.0707954306599277\\
1861	0.131255981603317\\
1862	0.188697925558245\\
1863	0.241590082399398\\
1864	0.288735835786156\\
1865	0.329076033719502\\
1866	0.361547538528157\\
1867	0.385411745264264\\
1868	0.400092868585489\\
1869	0.405272393916675\\
1870	0.400867249057228\\
1871	0.386885111081938\\
1872	0.363711536110713\\
1873	0.331920033027927\\
1874	0.292140781719809\\
1875	0.245459207933891\\
1876	0.192967544229305\\
1877	0.13586362957858\\
1878	0.0754945670832967\\
1881	-0.110267190925242\\
1882	-0.168897073214794\\
1883	-0.22356962064714\\
1884	-0.272874304911511\\
1885	-0.315717346717975\\
1886	-0.351002277943735\\
1887	-0.377979327858156\\
1888	-0.39587993062014\\
1889	-0.404408613756914\\
1890	-0.403294337616899\\
1891	-0.392592216245703\\
1892	-0.372600209314442\\
1893	-0.343694815575873\\
1894	-0.306601272755415\\
1895	-0.262194401519537\\
1896	-0.211512722640691\\
1897	-0.155840890141917\\
1898	-0.0964754624078523\\
1902	0.149181573478472\\
1903	0.205350729423117\\
1904	0.256557967664321\\
1905	0.301657698244981\\
1906	0.339622212267841\\
1907	0.369463156854636\\
1908	0.390573826656691\\
1909	0.402324604981459\\
1910	0.404465394800809\\
1911	0.397024082447388\\
1912	0.380132975644301\\
1913	0.354144594931313\\
1914	0.31980322511663\\
1915	0.277815041883059\\
1916	0.229193087771364\\
1917	0.175146217148722\\
1918	0.116937689088445\\
1919	0.0559399805811154\\
1920	-0.0063943320733415\\
1921	-0.0665803764791235\\
1922	-0.124652956341833\\
1923	-0.179392048497448\\
1924	-0.229570184457771\\
1925	-0.274141504133695\\
1926	-0.312253624141704\\
1927	-0.342997685075716\\
1928	-0.36590841401221\\
1929	-0.38049242778925\\
1930	-0.386602725268858\\
1931	-0.384172674436741\\
1932	-0.373408286925041\\
1933	-0.354617876318116\\
1934	-0.328417622621146\\
1935	-0.295476722958483\\
1936	-0.256715412266203\\
1937	-0.213169920907148\\
1938	-0.165993324028022\\
1940	-0.0654999529770066\\
1941	-0.0147274056653259\\
1942	0.0346062538314982\\
1943	0.0813518663521791\\
1944	0.124181532086368\\
1945	0.162007684818946\\
1946	0.193846968335038\\
1947	0.218732393524533\\
1948	0.236032652355789\\
1949	0.245154279895814\\
1950	0.245808170192959\\
1951	0.237780720522096\\
1952	0.221157866536487\\
1953	0.19612697215598\\
1954	0.163081763188984\\
1955	0.122627990393084\\
1956	0.0755724622986236\\
1957	0.0228799242795503\\
1958	-0.0343415662896405\\
1959	-0.0950176481442213\\
1963	-0.344341046050431\\
1964	-0.401539843182945\\
1965	-0.453970395734359\\
1966	-0.500459326635792\\
1967	-0.539925721221152\\
1968	-0.571248615503009\\
1969	-0.593780481900467\\
1970	-0.606694329605034\\
1971	-0.609541036774772\\
1972	-0.602087308844602\\
1973	-0.584318389980581\\
1974	-0.556263869324994\\
1975	-0.518355544940732\\
1976	-0.471069362127309\\
1977	-0.415088244747949\\
1978	-0.351385476853011\\
1979	-0.281032529774166\\
1980	-0.205165605852017\\
1982	-0.0443332681124957\\
1984	0.118038052575685\\
1985	0.196975797478444\\
1986	0.272749647982891\\
1987	0.344048102056149\\
1988	0.409643658339974\\
1989	0.468617128311053\\
1990	0.519800407825187\\
1991	0.56245978193374\\
1992	0.595910542234833\\
1993	0.619586245351002\\
1994	0.633086360893685\\
1995	0.6361804361336\\
1996	0.628856058903239\\
1997	0.611202900546687\\
1998	0.58347366515909\\
1999	0.546255720620593\\
2000	0.500053743521221\\
2001	0.445662653547515\\
2002	0.383914227268178\\
2003	0.315862930716776\\
2004	0.242551610625014\\
2005	0.165388090482338\\
2006	0.0855041764175439\\
2008	-0.0771508439124773\\
2009	-0.157254328896215\\
2010	-0.234750780167815\\
2011	-0.308368234756472\\
2012	-0.376903437377678\\
2013	-0.439242433142681\\
2014	-0.494385771114139\\
2015	-0.541383978488284\\
2016	-0.579443478785379\\
2017	-0.607990951569263\\
2018	-0.626545728770907\\
2019	-0.634770154228136\\
2020	-0.632487275508993\\
2021	-0.619890885251607\\
2022	-0.597009227621584\\
2023	-0.564331970689182\\
2024	-0.522453888268956\\
2025	-0.471907315140925\\
2026	-0.413626381750419\\
2027	-0.348517753576743\\
2028	-0.277665369270835\\
2029	-0.202278113466036\\
2030	-0.123411369702808\\
2033	0.119826753297275\\
2034	0.198734467284339\\
2035	0.27440339121631\\
2036	0.34561721648879\\
2037	0.411173342746224\\
2038	0.46999098346987\\
2039	0.52103179597907\\
2040	0.563581880525817\\
2041	0.596908558350151\\
2042	0.620420904506773\\
2043	0.633721514142508\\
2044	0.636662713156966\\
2045	0.62913063649512\\
2046	0.611325381834831\\
2047	0.58344414116209\\
2048	0.546026762071961\\
2049	0.499622352514962\\
2050	0.444975884430278\\
2051	0.383085505541658\\
2052	0.314990085011686\\
2053	0.241639387615123\\
2054	0.164369387836359\\
2055	0.0843238411471248\\
2057	-0.0784758681229505\\
2058	-0.158593494168599\\
2059	-0.236065062683338\\
2060	-0.309655324919277\\
2061	-0.378193247179297\\
2062	-0.440499555080351\\
2063	-0.495483762436379\\
2064	-0.542324498642756\\
2065	-0.580221873995924\\
2066	-0.608659175730736\\
2067	-0.627059873105281\\
2068	-0.635149831892249\\
2069	-0.632823060773262\\
2070	-0.620033439169674\\
2071	-0.59704976456851\\
2072	-0.564249433539317\\
2073	-0.522203192306279\\
2074	-0.471580791224824\\
2075	-0.413168235561443\\
2076	-0.347983828710767\\
2077	-0.277098702495095\\
2078	-0.201630486401882\\
2079	-0.122893193682103\\
2080	-0.0421453141807433\\
2082	0.0886736286520318\\
2083	0.151951212268159\\
2084	0.213105251051729\\
2085	0.271728536255523\\
2086	0.327363499308831\\
2087	0.379812989192942\\
2088	0.428761376509101\\
2089	0.474026862618757\\
2090	0.515338556305778\\
2091	0.55259187906222\\
2092	0.58580820844827\\
2093	0.614878138630502\\
2094	0.639778453748931\\
2095	0.660449414114282\\
2096	0.676895936277106\\
2097	0.68905919954932\\
2098	0.697059832915784\\
2099	0.700768633697407\\
2100	0.700379356700523\\
2101	0.695785349569178\\
2102	0.686997878427064\\
2103	0.67413385579539\\
2104	0.657247448640192\\
2105	0.636290047925286\\
2106	0.611482979350512\\
2107	0.582820317994447\\
2108	0.550436661500953\\
2109	0.514476984510111\\
2110	0.47524710832522\\
2111	0.43287968706818\\
2112	0.387712101556644\\
2113	0.340010429657923\\
2114	0.290199159981512\\
2115	0.238681491920033\\
2117	0.13239169358485\\
2119	0.0252298848981809\\
2120	-0.0272830087801594\\
2121	-0.0783547463061041\\
2122	-0.127272962924053\\
2123	-0.173463144444213\\
2124	-0.216413970464146\\
2125	-0.255511743291208\\
2126	-0.290304596927854\\
2127	-0.32028303028801\\
2128	-0.345060463080245\\
2129	-0.364238398939506\\
2130	-0.377659130980192\\
2131	-0.385054973363367\\
2132	-0.386408193772695\\
2133	-0.381705351640903\\
2134	-0.371016848486761\\
2135	-0.354647893877882\\
2136	-0.332878293565955\\
2137	-0.306074827281009\\
2138	-0.274688732792583\\
2139	-0.239306272679642\\
2140	-0.200653320161109\\
2141	-0.159358220896138\\
2142	-0.100185503120883\\
2144	0.0237464755800829\\
2145	0.0855856701014091\\
2146	0.145434174450656\\
2147	0.201829430482121\\
2148	0.253392202503619\\
2149	0.298937473738079\\
2150	0.337309706559608\\
2151	0.367715538773609\\
2152	0.38936973927548\\
2153	0.401729895999324\\
2154	0.404545488799613\\
2155	0.397717843315149\\
2156	0.3814388199512\\
2157	0.356056928341786\\
2158	0.322219823706291\\
2159	0.280695463324264\\
2160	0.232571755720983\\
2161	0.178877412772181\\
2162	0.120955081430566\\
2163	0.0601647856819909\\
2165	-0.0642320268129879\\
2166	-0.124862268194192\\
2167	-0.182533625845281\\
2168	-0.235886507920895\\
2169	-0.283608796394674\\
2170	-0.324544523602981\\
2171	-0.357760367694027\\
2172	-0.382433350592692\\
2173	-0.39801517226806\\
2174	-0.404079475294111\\
2175	-0.400500809159439\\
2176	-0.387456461009151\\
2177	-0.365040638789651\\
2178	-0.333988573603619\\
2179	-0.295023824742202\\
2180	-0.249000537335633\\
2181	-0.197016997333321\\
2182	-0.140302777781926\\
2183	-0.0802797927408392\\
2186	0.105360350823048\\
2187	0.164201266604778\\
2188	0.219139028796235\\
2189	0.268870034478368\\
2190	0.312171104457775\\
2191	0.348149228066632\\
2192	0.375765749403854\\
2193	0.394496308061662\\
2194	0.403761911479251\\
2195	0.403427581478354\\
2196	0.393510630833589\\
2197	0.37426264187161\\
2198	0.346107094165291\\
2199	0.309650064638845\\
2200	0.265864877374952\\
2201	0.215819185271812\\
2202	0.160608156939361\\
2203	0.101488550194063\\
2205	-0.0224672399899646\\
2206	-0.0843793731401092\\
2207	-0.144263674328158\\
2208	-0.200683506892346\\
2209	-0.252338071702525\\
2210	-0.298000976665662\\
2211	-0.336581203192054\\
2212	-0.367129576083244\\
2213	-0.388940618795004\\
2214	-0.401506157124459\\
2215	-0.404509203246562\\
2216	-0.397939132255033\\
2217	-0.381856803883693\\
2218	-0.356670667492381\\
2219	-0.322977587400601\\
2220	-0.281635630698929\\
2221	-0.233604072116123\\
2222	-0.180029940681379\\
2223	-0.122176188415324\\
2224	-0.0614111640829833\\
2226	0.0629403589136928\\
2227	0.123590521248843\\
2228	0.18131105151997\\
2229	0.234703846441789\\
2230	0.28248876033831\\
2231	0.323557041337153\\
2232	0.356854826449762\\
2233	0.381665627461189\\
2234	0.397408309712773\\
2235	0.403644757597249\\
2236	0.400223577012639\\
2237	0.387296542380227\\
2238	0.365049300189185\\
2239	0.33415325776059\\
2240	0.295323647048917\\
2241	0.249425230892939\\
2242	0.197548741490209\\
2243	0.140913527354314\\
2244	0.0809869345848711\\
2247	-0.10466461599799\\
2248	-0.163529941678462\\
2249	-0.218472373856457\\
2250	-0.268256994648709\\
2251	-0.311655251385673\\
2252	-0.347673757674329\\
2253	-0.375396734200876\\
2254	-0.394245343954935\\
2255	-0.403607260748231\\
2256	-0.40342256068061\\
2257	-0.393569456452042\\
2258	-0.374418193574456\\
2259	-0.34637638841923\\
2260	-0.310051403608213\\
2261	-0.266389897579302\\
2262	-0.21637782263906\\
2263	-0.161232426767128\\
2264	-0.102201134449388\\
2266	0.0216719030099739\\
2267	0.0836515827550102\\
2268	0.143638665357685\\
2269	0.200144881736378\\
2270	0.251889558919629\\
2271	0.297636689651881\\
2272	0.336362092687978\\
2273	0.367004482821358\\
2274	0.388964666807169\\
2275	0.401657108940526\\
2276	0.404795975824072\\
2277	0.398327846328357\\
2278	0.3823776942113\\
2279	0.35730103081687\\
2280	0.32375783927273\\
2281	0.282498536880212\\
2282	0.234476927863398\\
2283	0.180888509365559\\
2284	0.123031110656939\\
2285	0.0621642928740584\\
2287	-0.0623944392277735\\
2288	-0.123184003615279\\
2289	-0.181103103664555\\
2290	-0.234635847404661\\
2291	-0.28258494176589\\
2292	-0.323820812494887\\
2293	-0.357309024751885\\
2294	-0.38232089240546\\
2295	-0.398242438450325\\
2296	-0.404678458932722\\
2297	-0.401466626613455\\
2298	-0.388749401665791\\
2299	-0.366730232874033\\
2300	-0.336014389320098\\
2301	-0.297253020335575\\
2302	-0.251457396086153\\
2303	-0.199661937722794\\
2304	-0.143110304161382\\
2305	-0.083161245089741\\
2308	0.102646295474187\\
2309	0.161623315668294\\
2310	0.216724515116766\\
2311	0.266670184050781\\
2312	0.310262515509294\\
2313	0.34650074981846\\
2314	0.37449506477833\\
2315	0.393544449029832\\
2316	0.403168653194371\\
2317	0.403250832313915\\
2318	0.393634756860138\\
2319	0.374712382762027\\
2320	0.346905117497045\\
2321	0.310812620253728\\
2322	0.267288104299041\\
2323	0.217375237809392\\
2324	0.162343014500493\\
2325	0.103487649061663\\
2327	-0.0202380590435496\\
2328	-0.0821788811635997\\
2329	-0.142148436285879\\
2330	-0.198750264392856\\
2331	-0.250557359314371\\
2332	-0.296394687852626\\
2333	-0.335227676944669\\
2334	-0.365980397994008\\
2335	-0.388124136835359\\
2336	-0.4009955056531\\
2337	-0.40431519663116\\
2338	-0.397991852542418\\
2339	-0.382245718836657\\
2340	-0.357319620227827\\
2341	-0.323920868062032\\
2342	-0.282788129638902\\
2343	-0.234904424116394\\
2344	-0.181410241325466\\
2345	-0.123707285853015\\
2346	-0.0629343327805145\\
2348	0.061553523872135\\
2349	0.12233946896913\\
2350	0.180263130798721\\
2351	0.233814659629843\\
2352	0.281853026398494\\
2353	0.323151114230768\\
2354	0.356781700958436\\
2355	0.381885654213875\\
2356	0.397896357828813\\
2357	0.404447700434957\\
2358	0.401357880198248\\
2359	0.388770229414149\\
2360	0.366848188997665\\
2361	0.336229123251542\\
2362	0.297559797779741\\
2363	0.251828301456953\\
2364	0.200114852158094\\
2365	0.143642117436229\\
2366	0.0837223832832024\\
2369	-0.10210483425135\\
2370	-0.16112240652501\\
2371	-0.216338210151662\\
2372	-0.266356665469175\\
2373	-0.310026441816262\\
2374	-0.346399699745234\\
2375	-0.374513624407427\\
2376	-0.393660185589397\\
2377	-0.403432750631964\\
2378	-0.403614077461498\\
2379	-0.394176983384568\\
2380	-0.375372960752884\\
2381	-0.347646049232935\\
2382	-0.311603880951225\\
2383	-0.268194571274762\\
2384	-0.218367312936607\\
2385	-0.163338213244515\\
2386	-0.104511558192826\\
2388	0.0192619554095472\\
2389	0.0812163146142666\\
2390	0.141217742578192\\
2391	0.197939507937463\\
2392	0.249822643853349\\
2393	0.295765722104989\\
2394	0.334712377230971\\
2395	0.365614114601613\\
2396	0.387873663928985\\
2397	0.400859645052606\\
2398	0.404295654493126\\
2399	0.398120449037378\\
2400	0.382473707389636\\
2401	0.353565748056553\\
2402	0.317010604090228\\
2403	0.273724144012249\\
2404	0.224700169055268\\
2405	0.171062531701864\\
2406	0.114121943542614\\
2409	-0.0635215131060249\\
2410	-0.120414960860671\\
2411	-0.174019934068383\\
2412	-0.22297484268347\\
2413	-0.266251854684469\\
2414	-0.302799583648266\\
2415	-0.3318836409303\\
2416	-0.352709731971572\\
2417	-0.364885850056453\\
2418	-0.368162749830844\\
2419	-0.362405156165551\\
2420	-0.347757341087345\\
2421	-0.324550957977408\\
2422	-0.293350252253276\\
2423	-0.254735380571219\\
2424	-0.209716125996238\\
2425	-0.159325225055454\\
2426	-0.104652998169058\\
2427	-0.0470319370888319\\
2430	0.130241773558737\\
2431	0.186218184024256\\
2432	0.238416150804369\\
2433	0.285482666995904\\
2434	0.326357255172297\\
2435	0.36005519737455\\
2436	0.385600762941976\\
2437	0.402328444777595\\
2438	0.409743386504033\\
2439	0.407485322718912\\
2440	0.395527873646643\\
2441	0.373923521682173\\
2442	0.342987471340621\\
2443	0.303166642571341\\
2444	0.255287118146498\\
2445	0.200139855702218\\
2446	0.138699918626116\\
2447	0.0722556191080912\\
2448	0.00200236261071041\\
2450	-0.144161974871622\\
2451	-0.217243733591204\\
2452	-0.288364989851743\\
2453	-0.356002796610028\\
2454	-0.418826260909555\\
2455	-0.475547258333336\\
2456	-0.524950852864094\\
2457	-0.566020152442889\\
2458	-0.597830885331859\\
2459	-0.619709413688724\\
2460	-0.631204704711763\\
2461	-0.631826204213212\\
2462	-0.617351309617334\\
2463	-0.59270333558834\\
2464	-0.558395275526436\\
2465	-0.514893058046255\\
2466	-0.462859917801779\\
2467	-0.40323476743697\\
2468	-0.337007819477549\\
2469	-0.265249457201662\\
2470	-0.189181597971583\\
2471	-0.109937788749448\\
2474	0.133344833618594\\
2475	0.211775262007905\\
2476	0.286725013850628\\
2477	0.356957283377142\\
2478	0.421338227468823\\
2479	0.478816925356568\\
2480	0.528428946676286\\
2481	0.569349043680631\\
2482	0.600987155464736\\
2483	0.622690941642304\\
2484	0.634069255821487\\
2485	0.635078804879413\\
2486	0.625674504125982\\
2487	0.605919505504517\\
2488	0.576288615593512\\
2489	0.537191952966623\\
2490	0.489228292573443\\
2491	0.433258273766114\\
2492	0.370108253451235\\
2493	0.300928706485593\\
2494	0.226824515363205\\
2495	0.148950365472956\\
2497	-0.0128428792972954\\
2498	-0.0940649313733957\\
2499	-0.173744808384527\\
2500	-0.250538751846761\\
2501	-0.323247291828466\\
2502	-0.390634745342595\\
2503	-0.451616922087396\\
2504	-0.505181055831599\\
2505	-0.550392258815009\\
2506	-0.586660309870695\\
2507	-0.613217272206384\\
2508	-0.629733859389944\\
2509	-0.635899209920353\\
2510	-0.63156293302518\\
2511	-0.616894784842771\\
2512	-0.592137787657975\\
2513	-0.557587137129303\\
2514	-0.51387037597442\\
2515	-0.461726376239312\\
2516	-0.402014374854843\\
2517	-0.335694974640774\\
2518	-0.263834799371125\\
2519	-0.187652160305333\\
2520	-0.108404944538051\\
2523	0.134763773855411\\
2524	0.213163706034265\\
2525	0.288120947968309\\
2526	0.358333930424124\\
2527	0.422567361483743\\
2528	0.479897849114423\\
2529	0.529351750824844\\
2530	0.570127573478658\\
2531	0.601542592321039\\
2532	0.623083642295114\\
2533	0.634283083229093\\
2534	0.635121293439624\\
2535	0.625513376452545\\
2536	0.605678158810406\\
2537	0.575900206657934\\
2538	0.536633490539771\\
2539	0.488533771066614\\
2540	0.432449414071471\\
2541	0.369216162297562\\
2542	0.299892965935214\\
2543	0.225661211966781\\
2544	0.147776415540648\\
2546	-0.0138860752977052\\
2547	-0.0951179942426279\\
2548	-0.174714544517883\\
2549	-0.25145037952916\\
2550	-0.324062365268219\\
2551	-0.391279049666991\\
2552	-0.452060841208095\\
2553	-0.505334106027021\\
2554	-0.550409835643222\\
2555	-0.58640882254258\\
2556	-0.612742093795077\\
2557	-0.62898553355717\\
2558	-0.634866352640529\\
2559	-0.630341298492567\\
2560	-0.615474929334141\\
2561	-0.590470065019417\\
2562	-0.555766352298633\\
2563	-0.511927811159239\\
2564	-0.459658925128224\\
2565	-0.399790220940304\\
2566	-0.333435795985224\\
2567	-0.26153859495389\\
2568	-0.185368668698175\\
2569	-0.106130906322051\\
2572	0.136569818363569\\
2573	0.214796528495299\\
2574	0.289439273895823\\
2575	0.359354868704031\\
2576	0.423326208821891\\
2577	0.480419605128645\\
2578	0.529606119830987\\
2579	0.569984490373827\\
2580	0.601037073179668\\
2581	0.622236432262525\\
2582	0.633187606128104\\
2583	0.633801598126411\\
2584	0.623952802956865\\
2585	0.603848815334004\\
2586	0.573793968034352\\
2587	0.534303752568121\\
2588	0.4860557590805\\
2589	0.429807119472571\\
2590	0.366467902094428\\
2591	0.297098544778692\\
2592	0.222865399607144\\
2593	0.144935008542234\\
2595	-0.0166568190065846\\
2596	-0.0977523181695688\\
2597	-0.177237689577851\\
2598	-0.253828883804999\\
2599	-0.326206804486901\\
2600	-0.39319869810879\\
2601	-0.453812781911893\\
2602	-0.506900718204633\\
2603	-0.551607580347536\\
2604	-0.587303013036035\\
2605	-0.613300781817543\\
2606	-0.629247110523465\\
2607	-0.634910060479342\\
2608	-0.629980719386822\\
2609	-0.614736557871765\\
2610	-0.589416743101538\\
2611	-0.554395117040258\\
2612	-0.51021913186878\\
2613	-0.45772465139089\\
2614	-0.397732549873126\\
2615	-0.331096838914164\\
2616	-0.259031889114794\\
2617	-0.182680561157213\\
2618	-0.103381598992655\\
2621	0.139580175064566\\
2622	0.217758525128829\\
2623	0.292369726303605\\
2624	0.362204125812241\\
2625	0.426100737634897\\
2626	0.483011424581946\\
2627	0.531980584677967\\
2628	0.572231457500948\\
2629	0.603110218461097\\
2630	0.624055145516195\\
2631	0.63476726657882\\
2632	0.635120210863988\\
2633	0.625061285165884\\
2634	0.604725554890592\\
2635	0.574418659035473\\
2636	0.534682257977693\\
2637	0.486202542079354\\
2638	0.429675793847309\\
2639	0.366122114891368\\
2640	0.296628442094971\\
2641	0.222164408858134\\
2642	0.144072963984399\\
2644	-0.0178543936071947\\
2645	-0.0990222641958098\\
2646	-0.17858623494385\\
2647	-0.25523350284584\\
2648	-0.327657578870912\\
2649	-0.394685998443947\\
2650	-0.455244944231254\\
2651	-0.508317396395796\\
2652	-0.553027080531137\\
2653	-0.58861136775613\\
2654	-0.614550075084026\\
2655	-0.630386565763729\\
2656	-0.635867807035993\\
2657	-0.63085970219754\\
2658	-0.615504364439403\\
2659	-0.590020912994987\\
2660	-0.55486042574239\\
2661	-0.510603556064325\\
2662	-0.457920059175194\\
2663	-0.397727695233243\\
2664	-0.331003323840832\\
2665	-0.258850263030126\\
2666	-0.182490911449804\\
2667	-0.103058310367942\\
2670	0.13999756526573\\
2671	0.218176108247917\\
2672	0.29278384324607\\
2673	0.362568851606738\\
2674	0.426404378637471\\
2675	0.483247624333671\\
2676	0.532137485097337\\
2677	0.572292660400308\\
2678	0.603014023984088\\
2679	0.623876227559322\\
2680	0.634489723820025\\
2681	0.63463135820939\\
2682	0.62431224117654\\
2683	0.603776729245055\\
2684	0.573360133540064\\
2685	0.53346971431256\\
2686	0.484797249391704\\
2687	0.428157190791808\\
2688	0.364509599498888\\
2689	0.294904610242156\\
2690	0.220395255595577\\
2691	0.142282309552229\\
2693	-0.0196563611270903\\
2694	-0.100875997883122\\
2695	-0.180391238535321\\
2696	-0.256934256149179\\
2697	-0.329290699882222\\
2698	-0.396218993150342\\
2699	-0.456631475570703\\
2700	-0.509555521459788\\
2701	-0.554096206186841\\
2702	-0.589481063515905\\
2703	-0.615226624764091\\
2704	-0.630866462780432\\
2705	-0.636127874334761\\
2706	-0.630986555388517\\
2707	-0.615449779257233\\
2708	-0.589788160319586\\
2709	-0.554465845045797\\
2710	-0.510024279362824\\
2711	-0.457197249449109\\
2712	-0.396838756113539\\
2713	-0.329971915542501\\
2714	-0.257650931855096\\
2715	-0.181143439167954\\
2716	-0.101679250558846\\
2719	0.141385272898788\\
2720	0.219518228835113\\
2721	0.294044667237358\\
2722	0.363764477888253\\
2723	0.427502246318454\\
2724	0.484216085836579\\
2725	0.532957177930712\\
2726	0.572935343035624\\
2727	0.603543057475235\\
2728	0.624172250732954\\
2729	0.634516008196442\\
2730	0.634427048129965\\
2731	0.623932222562416\\
2732	0.603214479941926\\
2733	0.572568745720673\\
2734	0.53243938533933\\
2735	0.483601794429887\\
2736	0.426860238698282\\
2737	0.363053964545543\\
2738	0.293262145527024\\
2739	0.21866135824439\\
2740	0.140535801447641\\
2742	-0.0214472457419106\\
2743	-0.102522764167588\\
2744	-0.181922174310785\\
2745	-0.25832069905573\\
2746	-0.330472683573134\\
2747	-0.397165754494836\\
2748	-0.457416177391224\\
2749	-0.510100652392339\\
2750	-0.55441205658235\\
2751	-0.589602749541882\\
2752	-0.615053413393071\\
2753	-0.630464562989346\\
2754	-0.635459772642207\\
2755	-0.629996644222501\\
2756	-0.614203524970435\\
2757	-0.58827776732187\\
2758	-0.552660795367046\\
2759	-0.507966034372657\\
2760	-0.454919787365725\\
2761	-0.394448750419997\\
2762	-0.32749648304025\\
2763	-0.255125139606207\\
2764	-0.178570170218791\\
2765	-0.0990673407150098\\
2768	0.1440240401148\\
2769	0.222132422545656\\
2770	0.296582853938617\\
2771	0.366125798747817\\
2772	0.429674583852375\\
2773	0.486217969827976\\
2774	0.534748149253574\\
2775	0.574430291076624\\
2776	0.604728187302044\\
2777	0.625153239332121\\
2778	0.635319042254196\\
2779	0.635074067992264\\
2780	0.624314313224659\\
2781	0.603374101465306\\
2782	0.572462733980956\\
2783	0.532177876260448\\
2784	0.483160645240332\\
2785	0.426146563531347\\
2786	0.362132089416718\\
2787	0.292238585273481\\
2788	0.217511120891231\\
2789	0.139163183031997\\
2791	-0.0229359390427817\\
2792	-0.104065825707949\\
2793	-0.183515495946722\\
2794	-0.259930962651651\\
2795	-0.332072109065393\\
2796	-0.398731421061257\\
2797	-0.458859287175528\\
2798	-0.511440332039001\\
2799	-0.555563064915532\\
2800	-0.590576240015707\\
2801	-0.615968300320219\\
2802	-0.631152389279578\\
2803	-0.63595566492495\\
2804	-0.630340016552054\\
2805	-0.614352270877589\\
2806	-0.588341784242857\\
2807	-0.552557536113454\\
2808	-0.507724500574568\\
2809	-0.45452198646808\\
2810	-0.39385998042826\\
2811	-0.326750766112582\\
2812	-0.254281211780381\\
2813	-0.177657360412923\\
2814	-0.0980079793853292\\
2817	0.145132033695063\\
2818	0.223152852085605\\
2819	0.297527701859508\\
2820	0.366974494039368\\
2821	0.430399389287686\\
2822	0.486820682994221\\
2823	0.535210806520809\\
2824	0.574778842010346\\
2825	0.60493954606136\\
2826	0.625133905140046\\
2827	0.635080144168569\\
2828	0.634568173998559\\
2829	0.623654126442489\\
2830	0.602468726524876\\
2831	0.571393185657598\\
2832	0.530947112545618\\
2833	0.481757656615173\\
2834	0.424648850379072\\
2835	0.3605464117054\\
2836	0.290615681773943\\
2837	0.21588072994291\\
2838	0.137560836277316\\
2840	-0.0245495089779979\\
2841	-0.105599582857394\\
2842	-0.184961765993194\\
2843	-0.261227032875468\\
2844	-0.333218665658933\\
2845	-0.399682899051641\\
2846	-0.459575025892718\\
2847	-0.511802543790509\\
2848	-0.555754382753548\\
2849	-0.590515108187901\\
2850	-0.615548190675781\\
2851	-0.630550267038871\\
2852	-0.635094434318489\\
2853	-0.629266324501259\\
2854	-0.613082491203841\\
2855	-0.586804622465479\\
2856	-0.550819906485231\\
2857	-0.505807407947032\\
2858	-0.452524968795842\\
2859	-0.391720679961054\\
2860	-0.324566010663148\\
2861	-0.252043564595624\\
2862	-0.17535564893933\\
2863	-0.0958278831203643\\
2866	0.146969542066472\\
2867	0.224876812123966\\
2868	0.299033481620882\\
2869	0.368313115592628\\
2870	0.431556607829407\\
2871	0.487707071181831\\
2872	0.535838183589021\\
2873	0.575195710010576\\
2874	0.605120714893019\\
2875	0.625019175588932\\
2876	0.634685238533166\\
2877	0.633889107593404\\
2878	0.622738127586672\\
2879	0.6013372921675\\
2880	0.570062368964045\\
2881	0.529415631082429\\
2882	0.480048385655209\\
2883	0.422781061626665\\
2884	0.358650196588769\\
2885	0.288581687706028\\
2886	0.21378698209719\\
2887	0.135466314517998\\
2889	-0.0265108821663489\\
2890	-0.107545844142805\\
2891	-0.186797942267276\\
2892	-0.262981504553863\\
2893	-0.334826901036649\\
2894	-0.401208045084786\\
2895	-0.461006924572757\\
2896	-0.513119954191097\\
2897	-0.556895670094036\\
2898	-0.591517646239936\\
2899	-0.616351551608204\\
2900	-0.631094421582929\\
2901	-0.635455237408223\\
2902	-0.629426691464687\\
2903	-0.61298568228267\\
2904	-0.586467546904714\\
2905	-0.550389187923429\\
2906	-0.505274624797039\\
2907	-0.451770397071869\\
2908	-0.390922304409742\\
2909	-0.323617235203528\\
2910	-0.251010708649119\\
2911	-0.174196088356894\\
2912	-0.0945083961714772\\
2915	0.148434211672338\\
2916	0.226313282591491\\
2917	0.300450794607514\\
2918	0.369637804352351\\
2919	0.432789994791165\\
2920	0.488767345755605\\
2921	0.536781161984891\\
2922	0.575989228442722\\
2923	0.605700278238601\\
2924	0.625520034746387\\
2925	0.635013789856202\\
2926	0.634045267810052\\
2927	0.622711008425085\\
2928	0.601149186600196\\
2929	0.56967387711893\\
2930	0.528786578467134\\
2931	0.479263543360958\\
2932	0.421862090130617\\
2933	0.35755917753113\\
2934	0.287352856001689\\
2935	0.212449650593499\\
2936	0.134022015735809\\
2938	-0.0280234635210945\\
2939	-0.10907421278489\\
2940	-0.188270440980887\\
2941	-0.264443653317812\\
2942	-0.336218683160041\\
2943	-0.402457840100396\\
2944	-0.462095385422799\\
2945	-0.514217061485397\\
2946	-0.557793345666141\\
2947	-0.592204308066357\\
2948	-0.616906551167631\\
2949	-0.631475603707258\\
2950	-0.635719547442022\\
2951	-0.629414934227498\\
2952	-0.612856778671812\\
2953	-0.586213372023394\\
2954	-0.549897721446996\\
2955	-0.504513701585893\\
2956	-0.450870135146033\\
2957	-0.38993062053396\\
2958	-0.322508565428507\\
2959	-0.249794996325818\\
2960	-0.172997163263062\\
2961	-0.0933116433157011\\
2964	0.149532936964533\\
2965	0.227330242004882\\
2966	0.301432955015571\\
2967	0.370527151581427\\
2968	0.433529740087579\\
2969	0.489381331230561\\
2970	0.537220417948902\\
2971	0.576208979334751\\
2972	0.605695764938446\\
2973	0.625243526319082\\
2974	0.634572433146332\\
2975	0.633418604162671\\
2976	0.621844703064198\\
2977	0.600009587970362\\
2978	0.56836760959095\\
2979	0.527407830432367\\
2980	0.477774629161559\\
2981	0.420263882928793\\
2982	0.355840852449091\\
2983	0.285551087539716\\
2984	0.21062206745637\\
2985	0.132212586546302\\
2989	-0.190092519257178\\
2990	-0.266129216011905\\
2991	-0.337744145164834\\
2992	-0.403832872227667\\
2993	-0.463345978078451\\
2994	-0.515208140262075\\
2995	-0.558659122371409\\
2996	-0.59288112786453\\
2997	-0.617320795288833\\
2998	-0.631677700117507\\
2999	-0.635669290346414\\
3000	-0.629155138779424\\
3001	-0.612389835497197\\
3002	-0.585502980360161\\
3003	-0.548991584279065\\
3004	-0.503481658023702\\
3005	-0.449699803127714\\
3006	-0.388548334888583\\
3007	-0.320999299822688\\
3008	-0.248186862888815\\
3009	-0.17126378134617\\
3010	-0.0915130484140718\\
3013	0.151365633297701\\
3014	0.22911573064448\\
3015	0.303044967228288\\
3016	0.372086427526028\\
3017	0.434957066817333\\
3018	0.490744057701704\\
3019	0.53842156812334\\
3020	0.577258258463644\\
3021	0.606553175220142\\
3022	0.625914603084311\\
3023	0.635000476453115\\
3024	0.633602671167409\\
3025	0.621793732651895\\
3026	0.599762552139055\\
3027	0.567855335126751\\
3028	0.526647199686067\\
3029	0.476795753660554\\
3030	0.419071594295019\\
3031	0.354521173985177\\
3032	0.284127439508666\\
3033	0.209006768127892\\
3034	0.13045125424469\\
3036	-0.0316842050488049\\
3037	-0.112727913577601\\
3038	-0.191871260877178\\
3039	-0.26791037771909\\
3040	-0.339517047921618\\
3041	-0.387231241084464\\
3042	-0.429572710312186\\
3043	-0.466396705818624\\
3044	-0.497554079024212\\
3045	-0.522983269368979\\
3046	-0.542775623123362\\
3047	-0.556864148239583\\
3048	-0.565525320623692\\
3049	-0.568935581025016\\
3050	-0.567299494171493\\
3051	-0.560926763298994\\
3052	-0.550089044197648\\
3053	-0.535008157045468\\
3054	-0.515979890880317\\
3055	-0.493480280579661\\
3056	-0.467584147661\\
3057	-0.438788220400056\\
3058	-0.407232751235824\\
3059	-0.373152026572825\\
3060	-0.336916457134521\\
3061	-0.298774985463751\\
3062	-0.258919223139401\\
3063	-0.217660674740728\\
3065	-0.131723181950292\\
3068	0.00173431216398967\\
3070	0.0905368064777576\\
3072	0.176290466019054\\
3073	0.217204755071634\\
3074	0.256362716607782\\
3075	0.293366553763008\\
3076	0.32780292827465\\
3077	0.359307879677999\\
3078	0.387599401684383\\
3079	0.412239816343117\\
3080	0.432853972446537\\
3081	0.449238754083581\\
3082	0.461080054035847\\
3083	0.468115881285485\\
3084	0.470188215147118\\
3085	0.467254639223029\\
3086	0.459250041289124\\
3087	0.44608151818511\\
3088	0.427939069862077\\
3089	0.404962764994252\\
3090	0.37735763549199\\
3091	0.345510963945799\\
3092	0.309775239325063\\
3093	0.270719895282127\\
3094	0.228832594260439\\
3095	0.184676529740955\\
3097	0.0921394092938499\\
3099	-0.00126147690025391\\
3100	-0.0464922280884821\\
3101	-0.0898438534013621\\
3102	-0.149454153983697\\
3103	-0.205529604071216\\
3104	-0.256661090983016\\
3105	-0.301679582970337\\
3106	-0.339564940403761\\
3107	-0.369324777707789\\
3108	-0.390286692968857\\
3109	-0.402017495030577\\
3110	-0.404135176493583\\
3111	-0.396648178102623\\
3112	-0.379662832511258\\
3113	-0.353631528157621\\
3114	-0.319210093616221\\
3115	-0.277236556218668\\
3116	-0.228595311431491\\
3117	-0.174498516269978\\
3118	-0.116209426239038\\
3119	-0.0551381846257755\\
3121	0.0694300682789617\\
3122	0.129967391932496\\
3123	0.187460672601901\\
3124	0.240478542728852\\
3125	0.28778390025218\\
3126	0.328205903396793\\
3127	0.360869827648912\\
3128	0.384997394885886\\
3129	0.399917298956098\\
3130	0.405323660135764\\
3131	0.401137508376451\\
3132	0.387393513803545\\
3133	0.364429899766492\\
3134	0.332810935942689\\
3135	0.293260011377697\\
3136	0.246718271877398\\
3137	0.194424652042471\\
3138	0.137463183186355\\
3139	0.0772347625520524\\
3142	-0.108467014815233\\
3143	-0.167132437472446\\
3144	-0.221789505314064\\
3145	-0.271145646971036\\
3146	-0.314018115289855\\
3147	-0.349499843653575\\
3148	-0.376591576770807\\
3149	-0.394683550805894\\
3150	-0.403377428708609\\
3151	-0.402450565265553\\
3152	-0.391900400776649\\
3153	-0.372027815760703\\
3154	-0.343392178183421\\
3155	-0.30653800115806\\
3156	-0.262335823259036\\
3157	-0.211923727227713\\
3158	-0.156485615851125\\
3159	-0.0972587479236608\\
3161	0.0265622758611244\\
3162	0.0882930484208373\\
3163	0.147942855977362\\
3164	0.204026951234027\\
3165	0.255260228289444\\
3166	0.300384585599659\\
3167	0.33842077153713\\
3168	0.368364077965452\\
3169	0.389497621969895\\
3170	0.401450540681708\\
3171	0.403738938950482\\
3172	0.396401265922577\\
3173	0.379580932793488\\
3174	0.353752910618823\\
3175	0.319461286488149\\
3176	0.277613806325917\\
3177	0.22916883759035\\
3178	0.175206521905238\\
3179	0.117079113540967\\
3180	0.0561001074183878\\
3182	-0.0684226576668152\\
3183	-0.128981823040249\\
3184	-0.186467868218188\\
3185	-0.23946768434098\\
3186	-0.286821327462349\\
3187	-0.327369115104375\\
3188	-0.360094035623661\\
3189	-0.384321682716291\\
3190	-0.399371843227982\\
3191	-0.404922272212389\\
3192	-0.400847189082924\\
3193	-0.387186599496545\\
3194	-0.364313387071888\\
3195	-0.332758289013782\\
3196	-0.293314664673744\\
3197	-0.246796362394889\\
3198	-0.19449277798185\\
3199	-0.137547663366149\\
3200	-0.0772864477194162\\
3201	-0.0252336618864319\\
3203	0.0837836173691358\\
3204	0.13858823434839\\
3205	0.192054815660867\\
3206	0.243063127306414\\
3207	0.290494661379853\\
3208	0.333405023593514\\
3209	0.370819205787484\\
3210	0.401857346435918\\
3211	0.425748031696912\\
3212	0.44196453318682\\
3213	0.449952338663934\\
3214	0.449446469361646\\
3215	0.440282783605653\\
3216	0.422324772416232\\
3217	0.395745112765781\\
3218	0.360874933698597\\
3219	0.318198198952359\\
3220	0.268144111621041\\
3221	0.211546709228969\\
3222	0.149096310796267\\
3223	0.0817140041940547\\
3224	0.0104267257861466\\
3225	-0.0635849459499696\\
3229	-0.366211062285856\\
3230	-0.437855683173439\\
3231	-0.505775361027645\\
3232	-0.568914015708287\\
3233	-0.626406010669143\\
3234	-0.677417150064684\\
3235	-0.72130135478028\\
3236	-0.757417621457989\\
3237	-0.785323011712535\\
3238	-0.804663234607233\\
3239	-0.815157151006133\\
3240	-0.816750966923337\\
3241	-0.809405745164895\\
3242	-0.793315134370459\\
3243	-0.768724137308254\\
3244	-0.736049147045833\\
3245	-0.695660145711372\\
3246	-0.64818442446267\\
3247	-0.594255863955823\\
3248	-0.534563644960599\\
3249	-0.469894117418335\\
3250	-0.401099371142209\\
3251	-0.329014350870239\\
3252	-0.254511829188687\\
3256	0.0495005147936354\\
3257	0.122632932959732\\
3258	0.192907609548911\\
3259	0.259600253959434\\
3260	0.321975894977186\\
3261	0.379321637382873\\
3262	0.441451653711283\\
3263	0.496352528525222\\
3264	0.543120207976244\\
3265	0.580948294074915\\
3266	0.60920276389561\\
3267	0.627486485554527\\
3268	0.635513011516196\\
3269	0.633063102088272\\
3270	0.620161433048452\\
3271	0.597085251097269\\
3272	0.564247167387748\\
3273	0.522062564832595\\
3274	0.471376427065024\\
3275	0.412891146077527\\
3276	0.347649222560904\\
3277	0.276729069903013\\
3278	0.201249232450664\\
3279	0.122451874186936\\
3282	-0.120726941935118\\
3283	-0.19957810222013\\
3284	-0.275103450835559\\
3285	-0.346117381057411\\
3286	-0.411553219743382\\
3287	-0.470151880023423\\
3288	-0.521062011081085\\
3289	-0.563387809816504\\
3290	-0.596431410587229\\
3291	-0.61966040422476\\
3292	-0.632663532243896\\
3293	-0.635328987457797\\
3294	-0.627544251527524\\
3295	-0.609460474149728\\
3296	-0.581300403505793\\
3297	-0.543621344285384\\
3298	-0.49702082550948\\
3299	-0.442237407993616\\
3300	-0.380218049619089\\
3301	-0.311945626069246\\
3302	-0.238540049198946\\
3303	-0.16119705584606\\
3304	-0.081231051058694\\
3306	0.0814863914301895\\
3307	0.161483189986484\\
3308	0.23884762904936\\
3309	0.312268820005556\\
3310	0.380590338135335\\
3311	0.442654521830718\\
3312	0.497478820303968\\
3313	0.544059647787435\\
3314	0.581779502759218\\
3315	0.609923613703359\\
3316	0.628002336037298\\
3317	0.635806582796704\\
3318	0.633217490492825\\
3319	0.620182457114879\\
3320	0.596955181085377\\
3321	0.563904679539519\\
3322	0.521624427642564\\
3323	0.470738171337871\\
3324	0.412117955956091\\
3325	0.346744170003149\\
3326	0.275645253429502\\
3327	0.200041874534691\\
3328	0.121157173625306\\
3331	-0.122083278531591\\
3332	-0.200866436028264\\
3333	-0.276386567956251\\
3334	-0.347334681337543\\
3335	-0.412621180803399\\
3336	-0.471080234887268\\
3337	-0.521832319904206\\
3338	-0.564004825153006\\
3339	-0.596933396716395\\
3340	-0.620006469035161\\
3341	-0.632875662536208\\
3342	-0.635368770823789\\
3343	-0.627443446388952\\
3344	-0.609151084270707\\
3345	-0.580906813775073\\
3346	-0.543118186439642\\
3347	-0.496341688635766\\
3348	-0.441442077302327\\
3349	-0.379266540900517\\
3350	-0.310912996606476\\
3351	-0.237443736346904\\
3352	-0.160041740467022\\
3353	-0.0800335015237579\\
3355	0.0826732088576136\\
3356	0.162624996859904\\
3357	0.239851109168285\\
3358	0.313189897889515\\
3359	0.38130552546636\\
3360	0.443198195496734\\
3361	0.497834139772294\\
3362	0.544252292823785\\
3363	0.58181785726174\\
3364	0.60975928547532\\
3365	0.627709911252623\\
3366	0.635296392017153\\
3367	0.632476839987248\\
3368	0.619213978107837\\
3369	0.595843556765885\\
3370	0.562647057300182\\
3371	0.520154500958597\\
3372	0.469153818673021\\
3373	0.410414195129761\\
3374	0.344986343956407\\
3375	0.273801260933396\\
3376	0.198135464632287\\
3377	0.119317110503289\\
3380	-0.123833941896464\\
3381	-0.202590837132448\\
3382	-0.278000521181639\\
3383	-0.348835513254016\\
3384	-0.413917052047054\\
3385	-0.472260655092668\\
3386	-0.522841967786462\\
3387	-0.56477899266838\\
3388	-0.597419888322293\\
3389	-0.620318077491902\\
3390	-0.633034949036755\\
3391	-0.6353071530134\\
3392	-0.627142576548067\\
3393	-0.608689319316909\\
3394	-0.580233691729518\\
3395	-0.54220149494131\\
3396	-0.495258006332278\\
3397	-0.440219645024627\\
3398	-0.377964901902487\\
3399	-0.309484145859642\\
3400	-0.235884495693881\\
3401	-0.158450255675234\\
3402	-0.0783759542123335\\
3404	0.0842964423040939\\
3405	0.164162386212411\\
3406	0.241393087334927\\
3407	0.314671974575504\\
3408	0.382751775185625\\
3409	0.444546733835068\\
3410	0.499044406728444\\
3411	0.545381800590803\\
3412	0.582741290642844\\
3413	0.610508935354119\\
3414	0.62822097724802\\
3415	0.63558899802274\\
3416	0.632481029270821\\
3417	0.619034697901043\\
3418	0.595415017020969\\
3419	0.561989740401714\\
3420	0.519414043968936\\
3421	0.468259184038743\\
3422	0.409432085454682\\
3423	0.343878074155327\\
3424	0.272624624097261\\
3425	0.196904207593434\\
3426	0.117983008751253\\
3429	-0.125075748539984\\
3430	-0.203818420619882\\
3431	-0.279182739418502\\
3432	-0.349919581284666\\
3433	-0.414929281476361\\
3434	-0.473101366420451\\
3435	-0.523542383585209\\
3436	-0.56531922414888\\
3437	-0.597832514681613\\
3438	-0.620592409669371\\
3439	-0.633092987182863\\
3440	-0.635148630741242\\
3441	-0.626816358138058\\
3442	-0.608103456217123\\
3443	-0.579424100171764\\
3444	-0.541231871826312\\
3445	-0.49416070103598\\
3446	-0.438890194507167\\
3447	-0.376434002122096\\
3448	-0.307843153973408\\
3449	-0.234105435891706\\
3450	-0.156504848639543\\
3451	-0.0763708435106309\\
3453	0.0863345142565777\\
3454	0.166188025869815\\
3455	0.243316034806867\\
3456	0.316490852048446\\
3457	0.384454987918616\\
3458	0.446102614531355\\
3459	0.500450537640063\\
3460	0.546560132951981\\
3461	0.583703631722074\\
3462	0.61124653744173\\
3463	0.628776698089041\\
3464	0.635925359792054\\
3465	0.632654860934963\\
3466	0.618999357790472\\
3467	0.595186591938273\\
3468	0.561575527570767\\
3469	0.518748039963157\\
3470	0.46741856287781\\
3471	0.408355389172812\\
3472	0.342627159158383\\
3473	0.271261203163249\\
3474	0.195467444180849\\
3475	0.116444682672409\\
3478	-0.126565397694321\\
3479	-0.205163679057932\\
3480	-0.280412989796332\\
3481	-0.351077582375638\\
3482	-0.415966857607145\\
3483	-0.474061016564065\\
3484	-0.524272719635974\\
3485	-0.565949133937465\\
3486	-0.598271322660366\\
3487	-0.620809347190061\\
3488	-0.633123253568556\\
3489	-0.635031603639618\\
3490	-0.626526404665583\\
3491	-0.607661052177264\\
3492	-0.578782554958252\\
3493	-0.540401251651474\\
3494	-0.493148395033131\\
3495	-0.437806204321078\\
3496	-0.375221710472943\\
3497	-0.306516751703384\\
3498	-0.232755091545187\\
3499	-0.155119191869289\\
3500	-0.0749871458656344\\
3502	0.087585334549658\\
3503	0.167403197999192\\
3504	0.244469167508669\\
3505	0.317506958518607\\
3506	0.385381347112798\\
3507	0.446826656484518\\
3508	0.500986090237802\\
3509	0.546956327675616\\
3510	0.583825807590074\\
3511	0.611132743325015\\
3512	0.628451868348293\\
3513	0.6354291460284\\
3514	0.631992768069267\\
3515	0.618174095941413\\
3516	0.594218620935862\\
3517	0.560399570730624\\
3518	0.517376353892814\\
3519	0.465899042134424\\
3520	0.406733244415591\\
3521	0.340890451825544\\
3522	0.269454720423255\\
3523	0.193642392083348\\
3524	0.114563747770717\\
3527	-0.128465863709607\\
3528	-0.20706531010228\\
3529	-0.282202873295319\\
3530	-0.352782486778324\\
3531	-0.417483926190926\\
3532	-0.4752937722551\\
3533	-0.525386542803062\\
3534	-0.566833462814884\\
3535	-0.598952816879319\\
3536	-0.621219734079204\\
3537	-0.633327699149504\\
3538	-0.634997383882364\\
3539	-0.626231032194028\\
3540	-0.607200018567255\\
3541	-0.578155812003843\\
3542	-0.539590301538283\\
3543	-0.492136733397729\\
3544	-0.436577883269365\\
3545	-0.373910605071615\\
3546	-0.305090202850806\\
3547	-0.231241192313973\\
3548	-0.153569070721915\\
3549	-0.0733859965057491\\
3551	0.0892674220490335\\
3552	0.168994033524086\\
3553	0.245936083717424\\
3554	0.318829818975246\\
3555	0.386526561341725\\
3556	0.447879509996255\\
3557	0.501830347543546\\
3558	0.547537270121666\\
3559	0.584234347059464\\
3560	0.611392202008119\\
3561	0.628481336214918\\
3562	0.635260434507472\\
3563	0.631593883448204\\
3564	0.617450470383574\\
3565	0.59323165628166\\
3566	0.559297166799297\\
3567	0.516086489655208\\
3568	0.464467645439981\\
3569	0.405183563326318\\
3570	0.339193669520228\\
3571	0.267667996978616\\
3572	0.19173166377368\\
3573	0.1126503736632\\
3576	-0.130489510363532\\
3577	-0.209024561534989\\
3578	-0.284097460014891\\
3579	-0.354477894328284\\
3580	-0.419067974250993\\
3581	-0.476728048502082\\
3582	-0.526602236816416\\
3583	-0.56781884087377\\
3584	-0.599710498689547\\
3585	-0.621737484646019\\
3586	-0.633544831629479\\
3587	-0.634982082340684\\
3588	-0.62598113709646\\
3589	-0.606675101967539\\
3590	-0.57741926834251\\
3591	-0.538604833086993\\
3592	-0.49106945881249\\
3593	-0.435408776703298\\
3594	-0.372541719646961\\
3595	-0.303571558372823\\
3596	-0.229641044440086\\
3597	-0.151911324647699\\
3599	0.00971336512429843\\
3600	0.0909289570895453\\
3601	0.170687497616655\\
3602	0.247665110102844\\
3603	0.320587064289612\\
3604	0.38823341943089\\
3605	0.449423516861316\\
3606	0.503298461933809\\
3607	0.548898084224675\\
3608	0.585464852090809\\
3609	0.612454010491092\\
3610	0.629295583904877\\
3611	0.635829287986326\\
3612	0.631917875341514\\
3613	0.617682520029575\\
3614	0.59331145950182\\
3615	0.559208625362771\\
3616	0.515881635670212\\
3617	0.464080813354485\\
3618	0.404712134496094\\
3619	0.338680545453826\\
3620	0.267088807472192\\
3621	0.191068344172436\\
3622	0.11193601041532\\
3625	-0.131046861368759\\
3626	-0.209504599324646\\
3627	-0.284552831271412\\
3628	-0.354859743960333\\
3629	-0.419322206847937\\
3630	-0.476930314386664\\
3631	-0.526657007723315\\
3632	-0.567726079913427\\
3633	-0.599463357832519\\
3634	-0.621361972127488\\
3635	-0.633027274451706\\
3636	-0.634275013429487\\
3637	-0.625071471096362\\
3638	-0.605587442413253\\
3639	-0.576177897618436\\
3640	-0.537262506935804\\
3641	-0.489522953316737\\
3642	-0.433778018995326\\
3643	-0.370849964832814\\
3644	-0.301814381541135\\
3645	-0.227833354765608\\
3646	-0.150075405505959\\
3647	-0.0698690807967068\\
3649	0.0927754930698939\\
3650	0.17244616547805\\
3651	0.249217843564111\\
3652	0.321982264815688\\
3653	0.389484396082025\\
3654	0.450538033758221\\
3655	0.50421153081561\\
3656	0.549595279809182\\
3657	0.585987790217132\\
3658	0.612696973534639\\
3659	0.629349709945927\\
3660	0.635636733477895\\
3661	0.631539678859099\\
3662	0.617004366824403\\
3663	0.592279922012494\\
3664	0.557846089609939\\
3665	0.514263513062815\\
3666	0.462270797972451\\
3667	0.402599125795405\\
3668	0.336381218628958\\
3669	0.264614219496707\\
3670	0.188481437462087\\
3671	0.109272828701251\\
3674	-0.133739986757064\\
3675	-0.212136210361678\\
3676	-0.287067819411277\\
3677	-0.357273864011859\\
3678	-0.421627019417429\\
3679	-0.479099836149089\\
3680	-0.528655339259785\\
3681	-0.569520076860954\\
3682	-0.601053309606868\\
3683	-0.622735343682052\\
3684	-0.634170307360819\\
3685	-0.635168236924528\\
3686	-0.625745890276903\\
3687	-0.606025035428047\\
3688	-0.576308492505632\\
3689	-0.537173770725531\\
3690	-0.48919619706021\\
3691	-0.433153859348295\\
3692	-0.370015074128787\\
3693	-0.300794756217783\\
3694	-0.226577650558283\\
3695	-0.148615050605258\\
3697	0.0131715974471263\\
3698	0.0943940594879678\\
3699	0.174080934890753\\
3700	0.250936740589623\\
3701	0.32359196861762\\
3702	0.390959205679337\\
3703	0.451934835876273\\
3704	0.50553615617855\\
3705	0.550761176718424\\
3706	0.586939065924525\\
3707	0.613493305918837\\
3708	0.629947199273374\\
3709	0.636016412323897\\
3710	0.631647352485288\\
3711	0.616947765788609\\
3712	0.592119210349665\\
3713	0.557508604103987\\
3714	0.513736500834966\\
3715	0.461501421031244\\
3716	0.40172174448162\\
3717	0.335319349640486\\
3718	0.263438141958886\\
3719	0.187208371810812\\
3720	0.107962758453596\\
3723	-0.135037415060651\\
3724	-0.2133205990026\\
3725	-0.288168185561062\\
3726	-0.358262594274947\\
3727	-0.4224363163612\\
3728	-0.479692396246719\\
3729	-0.529069451635678\\
3730	-0.569746532748013\\
3731	-0.601059585327675\\
3732	-0.622454502846267\\
3733	-0.633651618791191\\
3734	-0.63444775666494\\
3735	-0.624876850959026\\
3736	-0.604961993129109\\
3737	-0.575121140008832\\
3738	-0.535836455215303\\
3739	-0.487727790432018\\
3740	-0.431528892906044\\
3741	-0.3682853951982\\
3742	-0.298983858799147\\
3743	-0.224775734775449\\
3744	-0.146834352572569\\
3746	0.014856919070553\\
3747	0.0960432159727134\\
3748	0.175623299324343\\
3749	0.252288150620188\\
3750	0.324853630170765\\
3751	0.39209117681412\\
3752	0.452866852513125\\
3753	0.506253116657263\\
3754	0.551260003976495\\
3755	0.587225061477966\\
3756	0.613488942477488\\
3757	0.629692409798281\\
3758	0.635601916003452\\
3759	0.630996157786285\\
3760	0.616052734837922\\
3761	0.590999767951871\\
3762	0.556234980242152\\
3763	0.512352957329313\\
3764	0.460016822127272\\
3765	0.40007210874046\\
3766	0.333607122332069\\
3767	0.261668625905713\\
3768	0.185422537312661\\
3769	0.106160609017934\\
3772	-0.136882511112162\\
3773	-0.215175580141022\\
3774	-0.2899092693483\\
3775	-0.359924496894109\\
3776	-0.42393476569714\\
3777	-0.481044180543449\\
3778	-0.530247465487719\\
3779	-0.570788622183954\\
3780	-0.601846381537598\\
3781	-0.62311226795191\\
3782	-0.634076081821604\\
3783	-0.634615209710773\\
3784	-0.62479605228873\\
3785	-0.604664023997884\\
3786	-0.574547565042849\\
3787	-0.535080213397578\\
3788	-0.486851990699051\\
3789	-0.430567877893736\\
3790	-0.367175932500686\\
3791	-0.29777743411023\\
3792	-0.22353173126703\\
3793	-0.145603558804396\\
3795	0.0160991257630485\\
3796	0.0972691173524254\\
3797	0.176835624785781\\
3798	0.253471912727946\\
3799	0.325991716956196\\
3800	0.39309891257335\\
3801	0.45367094273297\\
3802	0.506899072236592\\
3803	0.551727813755861\\
3804	0.587527531763953\\
3805	0.613684851498874\\
3806	0.629762931646383\\
3807	0.635459743822139\\
3808	0.630717395187276\\
3809	0.615617297853078\\
3810	0.590402036137675\\
3811	0.555460712659624\\
3812	0.511393463494187\\
3813	0.458950523083786\\
3814	0.398895590983557\\
3815	0.332326558728255\\
3816	0.260236012762562\\
3817	0.183932624078352\\
3818	0.104647744700287\\
3821	-0.138192812436955\\
3822	-0.216392408402953\\
3823	-0.291026025478459\\
3824	-0.36091178800234\\
3825	-0.42477560736188\\
3826	-0.481736951262519\\
3827	-0.530808743682883\\
3828	-0.571115248540536\\
3829	-0.602068125575897\\
3830	-0.623101757715176\\
3831	-0.633890960571989\\
3832	-0.634265888727896\\
3833	-0.624227686777431\\
3834	-0.603834819764415\\
3835	-0.573561693680858\\
3836	-0.533817418271155\\
3837	-0.485351345637355\\
3838	-0.428892815282325\\
3839	-0.365381730143781\\
3840	-0.295843209566101\\
3841	-0.221512189876648\\
3842	-0.143467907391823\\
3844	0.0182766137691033\\
3845	0.0994159611987016\\
3846	0.178888608397301\\
3847	0.25544573533989\\
3848	0.327843165498507\\
3849	0.394852109790918\\
3850	0.455376230272122\\
3851	0.508406396223563\\
3852	0.553084487497927\\
3853	0.588686297006006\\
3854	0.614659110065531\\
3855	0.630544229881707\\
3856	0.636025999536287\\
3857	0.631107724411777\\
3858	0.615725467083848\\
3859	0.590248640348364\\
3860	0.555104274712448\\
3861	0.510861294212191\\
3862	0.458260459351095\\
3863	0.398024376692319\\
3864	0.331321877231403\\
3865	0.259178143485315\\
3866	0.18272574481216\\
3867	0.1032645263208\\
3870	-0.139792951266827\\
3871	-0.217970398375655\\
3872	-0.292554630386348\\
3873	-0.362394312464403\\
3874	-0.426269271762521\\
3875	-0.483160073559702\\
3876	-0.532049263461886\\
3877	-0.572216598517116\\
3878	-0.60301148701592\\
3879	-0.623921107868227\\
3880	-0.634598816228845\\
3881	-0.634843266039752\\
3882	-0.624562181596957\\
3883	-0.604116331693149\\
3884	-0.573707500296678\\
3885	-0.533836589980638\\
3886	-0.485297677388189\\
3887	-0.428714868143743\\
3888	-0.365083880256861\\
3889	-0.295455310997568\\
3890	-0.220997018133403\\
3891	-0.142918741678386\\
3893	0.0190981128516796\\
3894	0.100255553102215\\
3895	0.179700801978925\\
3896	0.256246014213957\\
3897	0.328576411740869\\
3898	0.395509483088063\\
3899	0.455890070007172\\
3900	0.508865385457284\\
3901	0.553406811255627\\
3902	0.588926055218508\\
3903	0.614700117153461\\
3904	0.6304149051889\\
3905	0.635729338585406\\
3906	0.630668644882462\\
3907	0.615176499153222\\
3908	0.589544294004554\\
3909	0.554261648632291\\
3910	0.5099126077248\\
3911	0.457151541964777\\
3912	0.396872711114611\\
3913	0.330087007146176\\
3914	0.257948365167067\\
3915	0.181520303249727\\
3916	0.102071653422172\\
3919	-0.140925171599974\\
3920	-0.219041254997137\\
3921	-0.293590887139544\\
3922	-0.363289890786291\\
3923	-0.427085363469359\\
3924	-0.483755367591584\\
3925	-0.532524527190617\\
3926	-0.572560891046123\\
3927	-0.60313981830086\\
3928	-0.623798162289404\\
3929	-0.634235931308467\\
3930	-0.634213855693815\\
3931	-0.623778685508114\\
3932	-0.603115164695282\\
3933	-0.572557037101433\\
3934	-0.532568665230883\\
3935	-0.483845485122856\\
3936	-0.427153639303469\\
3937	-0.363402698383652\\
3938	-0.293718654105305\\
3939	-0.219197964526757\\
3940	-0.141078429881418\\
3942	0.0208142614283133\\
3943	0.101847422026822\\
3944	0.181268585091857\\
3945	0.257729984426533\\
3946	0.329857803793402\\
3947	0.396623013483349\\
3948	0.456833149543399\\
3949	0.509607880970634\\
3950	0.553949001322962\\
3951	0.589247747017453\\
3952	0.61479100153565\\
3953	0.630167844169137\\
3954	0.6352300141657\\
3955	0.629885843391548\\
3956	0.614200229959351\\
3957	0.588350108382656\\
3958	0.552824585774488\\
3959	0.508220270415677\\
3960	0.455325649560564\\
3961	0.394861711735757\\
3962	0.327995884039865\\
3963	0.255652623843162\\
3964	0.179152438631263\\
3965	0.099610436983312\\
3968	-0.143298150936062\\
3969	-0.221294834734181\\
3970	-0.295724989697646\\
3971	-0.365234081875315\\
3972	-0.42871577869937\\
3973	-0.485203412495139\\
3974	-0.533753518492631\\
3975	-0.573586442926626\\
3976	-0.603932018297201\\
3977	-0.624334141712552\\
3978	-0.634440354906474\\
3979	-0.634202422958424\\
3980	-0.623533556266011\\
3981	-0.60255885415927\\
3982	-0.571767180857933\\
3983	-0.531517008160336\\
3984	-0.482482242664901\\
3985	-0.425624503106064\\
3986	-0.361736091228977\\
3987	-0.291901353321009\\
3988	-0.217232123171925\\
3989	-0.139019544748862\\
3991	0.0229283960184148\\
3992	0.103966473600849\\
3993	0.183290306528761\\
3994	0.259604449246126\\
3995	0.331677349206984\\
3996	0.398287015872484\\
3997	0.458379682013401\\
3998	0.510867050037177\\
3999	0.55501428470734\\
4000	0.590031574483874\\
};
\addlegendentry{1}

\addplot [color=mycolor2]
  table[row sep=crcr]{%
1	0.0037331097405513\\
3	0.0142427619216505\\
7	0.0381347271045342\\
8	0.0423112335402038\\
9	0.0450159428287407\\
10	0.0457074914238547\\
11	0.043993347050673\\
12	0.0395898552928884\\
13	0.0320865960025003\\
14	0.0213516150829491\\
15	0.0071884278017933\\
16	-0.0104635209931985\\
17	-0.0316027244630277\\
18	-0.0561464217275898\\
19	-0.0838477803672504\\
20	-0.114429365314663\\
21	-0.147480861340227\\
22	-0.182532521439498\\
24	-0.256414529294943\\
26	-0.330863627591498\\
27	-0.366472803108991\\
28	-0.399955722209597\\
29	-0.430543546280205\\
30	-0.457581967224996\\
31	-0.0854242369550775\\
32	0.0347031395781414\\
33	0.143936139799735\\
34	0.249627704551585\\
35	0.335196330195231\\
36	0.41174907423283\\
37	0.473441832381468\\
38	0.513385485182425\\
39	0.550725129406601\\
40	0.558341064181604\\
41	0.548778117834445\\
42	0.529669161066977\\
43	0.483282395452079\\
44	0.432942969570377\\
45	0.366060194824513\\
46	0.290047898050943\\
47	0.204360803959389\\
51	-0.171996153593227\\
52	-0.260817452463016\\
53	-0.34390079686591\\
54	-0.422427499259811\\
55	-0.483530127380618\\
56	-0.536997327606969\\
57	-0.573353529848646\\
58	-0.59255970005006\\
59	-0.595218765312893\\
60	-0.590717048515216\\
61	-0.563025181457306\\
62	-0.528132548893154\\
63	-0.476387133452135\\
64	-0.421143570026288\\
65	-0.348507869274727\\
66	-0.268861534787447\\
67	-0.182685741981459\\
68	-0.0924125838055261\\
70	0.0840607340105635\\
71	0.175260773609352\\
72	0.259914275829942\\
73	0.346578770628639\\
75	0.475178451316424\\
76	0.522967032438373\\
77	0.562779028035493\\
78	0.583145703819355\\
79	0.591858807321842\\
80	0.587786295291153\\
81	0.57395978976183\\
82	0.540739807334376\\
83	0.496025894247396\\
84	0.443096343963589\\
85	0.373634110911553\\
86	0.297625252146645\\
87	0.217828376677062\\
88	0.12452325603499\\
89	0.0368633923617381\\
90	-0.0546865793808138\\
91	-0.147632230775798\\
92	-0.233625887075959\\
93	-0.318164729038472\\
94	-0.392004904158057\\
95	-0.453811259055328\\
96	-0.506236607307983\\
97	-0.543848884929503\\
98	-0.579796858869031\\
99	-0.592807005690247\\
100	-0.594140734314351\\
101	-0.584855863855864\\
102	-0.554270571337838\\
103	-0.51692489642619\\
104	-0.4584875065284\\
105	-0.398778768821103\\
106	-0.323597591000635\\
107	-0.244285879139625\\
109	-0.0676258504822727\\
110	0.0266468618178806\\
111	0.118488956600231\\
112	0.209108510394344\\
113	0.288413455476984\\
114	0.360765045317294\\
115	0.431617128407197\\
116	0.494141851419499\\
117	0.534688129754159\\
118	0.572367397025573\\
119	0.590109772315827\\
120	0.598575632818665\\
121	0.582358121424022\\
122	0.56369095473292\\
124	0.486113010135796\\
125	0.421499926771958\\
126	0.352760392958317\\
128	0.185521342270022\\
130	0.00832247174366785\\
131	-0.0863556151316516\\
132	-0.173717547850629\\
133	-0.265273614604212\\
134	-0.340635892778664\\
135	-0.412805959932939\\
136	-0.475782642874492\\
137	-0.521107557188316\\
138	-0.559153592179882\\
139	-0.583956590171056\\
140	-0.597353173146985\\
141	-0.588048480955422\\
142	-0.571892448752351\\
143	-0.541802233536146\\
144	-0.496649559504021\\
145	-0.440989877441098\\
146	-0.3775246967366\\
147	-0.298563400087914\\
148	-0.216931216614739\\
149	-0.129317609434111\\
150	-0.0387232668676916\\
151	0.0545107316297617\\
152	0.141819071362079\\
153	0.235556806249406\\
154	0.316107139272845\\
155	0.388908391471432\\
156	0.453650779249983\\
157	0.505929366550845\\
158	0.551965388431654\\
159	0.578292530452018\\
160	0.592079381203348\\
161	0.602217159038446\\
162	0.587300260109714\\
163	0.558430043542558\\
164	0.51522146538764\\
165	0.455390804023864\\
166	0.383088453103028\\
167	0.308391944130562\\
168	0.214651756387866\\
169	0.113474828820017\\
170	0.00816548081138535\\
171	-0.0935763847237467\\
172	-0.192602416449517\\
174	-0.378262076655574\\
175	-0.457523578359087\\
176	-0.519462996667698\\
177	-0.564085138963947\\
178	-0.595437857289653\\
179	-0.60812161734566\\
180	-0.602537183797722\\
181	-0.574663723014964\\
182	-0.528524289038614\\
183	-0.457812339022439\\
184	-0.375894419265933\\
185	-0.2669790429768\\
186	-0.148923202001697\\
187	-0.0128209283038814\\
188	0.135562990429662\\
190	0.447206595397347\\
191	1.09551415370152\\
192	0.984686816462272\\
193	0.86649316261628\\
194	0.744752226341916\\
196	0.495870586619276\\
197	0.373627771013616\\
198	0.254237379238475\\
199	0.142044670163159\\
200	0.0369059529534752\\
201	-0.0634421613635823\\
202	-0.152996620313843\\
203	-0.22817819580132\\
204	-0.294445990327858\\
205	-0.350100535684305\\
206	-0.385991857070167\\
207	-0.415118983617504\\
208	-0.430814423569245\\
209	-0.433308159838816\\
210	-0.425907334029034\\
211	-0.410227518504598\\
212	-0.380642377363074\\
213	-0.344445051704952\\
214	-0.302235843796097\\
215	-0.24849809064699\\
216	-0.201249685736002\\
220	0.0260023077344158\\
221	0.0791058677109504\\
222	0.12800444471668\\
224	0.205134855444612\\
225	0.247793345876744\\
226	0.279667686846551\\
227	0.304555435740895\\
229	0.344941290356473\\
230	0.357677928417161\\
231	0.364749341145398\\
232	0.363796345090577\\
233	0.359607970432535\\
234	0.347839238436791\\
236	0.304692104844435\\
237	0.277325214822667\\
238	0.245467026993083\\
239	0.209418439165347\\
240	0.170286080373444\\
241	0.120793881662848\\
242	0.0818030758860004\\
243	0.0395037750763549\\
244	-0.0123540404206324\\
245	-0.0543932561013207\\
246	-0.101320084675535\\
247	-0.146089350612328\\
248	-0.187123226706717\\
249	-0.229466451963162\\
250	-0.260879396075779\\
251	-0.287512145265737\\
252	-0.316509321574358\\
253	-0.337102633731774\\
254	-0.353697731070042\\
255	-0.361190333850118\\
256	-0.363791445187417\\
257	-0.360531359571269\\
258	-0.353523107698038\\
259	-0.338614716650682\\
260	-0.313447537767843\\
261	-0.297670218702933\\
262	-0.257466272359579\\
263	-0.227245928901993\\
264	-0.187554077312598\\
265	-0.152213011251206\\
267	-0.0597451939506755\\
268	-0.0116908448358117\\
269	0.0303244325341439\\
270	0.0811821336706089\\
271	0.122235320297932\\
272	0.167859936970672\\
273	0.209124266276376\\
274	0.244954717083147\\
275	0.27921280280907\\
276	0.307049795459989\\
278	0.346034071854319\\
279	0.355276819345363\\
281	0.365671149485934\\
282	0.357713661242087\\
283	0.345846523825003\\
284	0.330583950026721\\
285	0.30705335947232\\
286	0.278646032154484\\
287	0.248248707498078\\
288	0.211022235937889\\
289	0.169591142951504\\
291	0.0801275869671372\\
292	0.0382329978210691\\
293	-0.0147160771125527\\
294	-0.0576712550964658\\
295	-0.107521910455034\\
296	-0.144673262433571\\
297	-0.186423934727372\\
298	-0.226636664971011\\
300	-0.297029637526521\\
301	-0.315675546240982\\
302	-0.342158584734079\\
303	-0.357139744563028\\
304	-0.360741240384868\\
305	-0.366066826743463\\
306	-0.360067423111104\\
307	-0.350597909231965\\
308	-0.3354589490732\\
309	-0.318663012857542\\
310	-0.290558217733633\\
311	-0.260996150299889\\
312	-0.224861174868238\\
313	-0.187178367562865\\
315	-0.103453513811473\\
316	-0.0579475211411591\\
317	-0.00885607671352773\\
318	0.0308105534863898\\
319	0.0805562085042766\\
320	0.123610529814869\\
321	0.168326212482953\\
322	0.210595718418062\\
323	0.247475496255447\\
325	0.305033179806742\\
326	0.329308151097848\\
327	0.345782496103766\\
328	0.358935494630259\\
329	0.365065689466064\\
330	0.362045078587926\\
331	0.357761161946655\\
332	0.341473144884276\\
333	0.331225955796072\\
334	0.304313508393079\\
336	0.244625576424369\\
337	0.205383775589326\\
338	0.168516079684196\\
339	0.124518194493248\\
340	0.0858011052509937\\
341	0.0372791775726\\
342	-0.0175832442400861\\
343	-0.0575741697398371\\
344	-0.107292741245601\\
346	-0.189099210794211\\
348	-0.26293775092654\\
349	-0.289017433773552\\
350	-0.319031995136356\\
351	-0.337387518574815\\
352	-0.351301152083124\\
353	-0.361422368763215\\
354	-0.362502029035113\\
355	-0.360605443358963\\
356	-0.354034981714904\\
357	-0.341477809741718\\
358	-0.312931155857768\\
359	-0.292806790913346\\
360	-0.260832058260803\\
361	-0.225094947036268\\
362	-0.187719409909732\\
363	-0.148455824795292\\
364	-0.103623433274606\\
365	-0.0572716465153462\\
366	-0.00715879174140355\\
367	0.0342893526235457\\
368	0.084072338756414\\
369	0.127750203943378\\
370	0.169172519672429\\
372	0.245296571128165\\
373	0.278192549159485\\
374	0.302913577562322\\
375	0.329447111583249\\
376	0.344531713908964\\
377	0.357818805017814\\
378	0.364558807943013\\
379	0.361622707408515\\
380	0.356841923864067\\
381	0.346250305766716\\
382	0.323691247016995\\
383	0.307018581033844\\
384	0.282054136252555\\
385	0.246131969309317\\
387	0.16822215575985\\
388	0.122200748722662\\
389	0.0838491984445682\\
390	0.0356178282786459\\
391	-0.0114321815171934\\
392	-0.0616406867616206\\
393	-0.104488112689523\\
394	-0.150201618321717\\
395	-0.190954569157839\\
396	-0.22971008908462\\
397	-0.265186942765922\\
398	-0.294997936637628\\
400	-0.342087928269848\\
402	-0.361674701627635\\
403	-0.362614550845137\\
404	-0.360329424354404\\
406	-0.34009893839584\\
407	-0.318553224156858\\
408	-0.291636957582341\\
409	-0.259970465492643\\
410	-0.222968662243147\\
411	-0.187677711880951\\
412	-0.142669549611128\\
413	-0.100481915366345\\
414	-0.0533453352268225\\
417	0.0798611802833875\\
418	0.128174503758601\\
419	0.172293161645484\\
420	0.209552411160985\\
421	0.245339799060275\\
422	0.278107114037084\\
423	0.304072128401913\\
424	0.328068161243209\\
425	0.345981377264252\\
426	0.358028617667514\\
427	0.363482570908218\\
428	0.364708625023468\\
430	0.348966255094183\\
431	0.327843507043781\\
432	0.307956040396675\\
433	0.274152172039066\\
434	0.244199637669681\\
436	0.167551481391456\\
437	0.124740306827789\\
438	0.0736598476419204\\
439	0.0294214447412742\\
440	-0.0175309900882894\\
442	-0.102915259116799\\
443	-0.150653556767793\\
446	-0.265951959760059\\
448	-0.320389816574334\\
449	-0.335250946146061\\
450	-0.353115441196223\\
451	-0.357930123933784\\
452	-0.365156281818145\\
453	-0.359530275220095\\
454	-0.351661406244602\\
455	-0.340956349068165\\
456	-0.31802563643123\\
458	-0.259554405480685\\
459	-0.227117461458874\\
460	-0.187490290316873\\
461	-0.14399603899983\\
462	-0.102814371161912\\
463	-0.0528357078846966\\
464	-0.00992386841335247\\
465	0.0422009715794047\\
466	0.0818008741080121\\
467	0.129296865558445\\
468	0.168012035650463\\
469	0.20954351711498\\
470	0.246095709429937\\
471	0.28049437794698\\
472	0.309302791548362\\
473	0.330396951226703\\
474	0.34601873793963\\
475	0.360095248681773\\
476	0.365581571412804\\
477	0.36364649084453\\
478	0.358227163491392\\
479	0.345610905271769\\
480	0.324500061158687\\
481	0.300625087824301\\
483	0.227806656871053\\
484	0.189131103855743\\
488	0.011809397873094\\
489	-0.035927637063196\\
491	-0.120258092603308\\
492	-0.157006129897127\\
493	-0.190527548160844\\
494	-0.222600735758988\\
495	-0.245428102558435\\
496	-0.264203338967036\\
497	-0.279218053168279\\
498	-0.283574540524114\\
499	-0.290793089186991\\
500	-0.287031007012956\\
501	-0.279342384726988\\
503	-0.236482600756062\\
504	-0.212044401471303\\
506	-0.144262255093508\\
507	-0.104383191328907\\
508	-0.0617827449846118\\
510	0.0315632921469842\\
511	-0.390298419222745\\
512	-0.364106440211799\\
513	-0.317759999973987\\
514	-0.257025855330539\\
515	-0.188449590185883\\
516	-0.111184156401578\\
517	-0.0298116669737283\\
518	0.0535207028142395\\
519	0.138874310684514\\
520	0.220025813582197\\
521	0.297798186853015\\
522	0.370370208846452\\
523	0.430719541407143\\
524	0.476247376842366\\
525	0.513130856146745\\
526	0.546493032731178\\
527	0.552602799101805\\
528	0.554794667722945\\
529	0.538216177852973\\
530	0.512625941491024\\
531	0.46361261198399\\
532	0.40645485998175\\
533	0.340922200037312\\
534	0.264395825813608\\
535	0.180037745893515\\
536	0.0892353256022034\\
537	-0.00455750314813486\\
538	-0.094389629318357\\
539	-0.189817369249795\\
540	-0.277049494782659\\
541	-0.353630219827664\\
542	-0.426705997796944\\
543	-0.485092457913652\\
544	-0.530162557820859\\
545	-0.567570161194453\\
546	-0.585861434952676\\
547	-0.596165651722458\\
548	-0.586231832342037\\
549	-0.566204597983415\\
550	-0.533725082018009\\
551	-0.488084257103765\\
552	-0.429795585315787\\
553	-0.359241407282298\\
554	-0.282241565928871\\
555	-0.198076381276223\\
556	-0.106140023759053\\
557	-0.0177080356020269\\
558	0.0772818393852504\\
559	0.163483100975554\\
560	0.25288622939479\\
561	0.329554492980151\\
562	0.403263526920909\\
563	0.467618725188458\\
564	0.515986023934602\\
565	0.55507553026564\\
566	0.586460246398019\\
567	0.596173568748327\\
568	0.591015155569039\\
569	0.577233789921593\\
570	0.545087761060586\\
571	0.504572725299568\\
572	0.451231226336404\\
573	0.378930541296995\\
574	0.310122514445084\\
575	0.22905393138808\\
576	0.138491098139184\\
577	0.0515432563270224\\
578	-0.0438238734154766\\
579	-0.137776606423358\\
580	-0.227525013599916\\
581	-0.305647825454798\\
582	-0.378332099965064\\
583	-0.445164979273159\\
584	-0.500308919577947\\
585	-0.545960477103108\\
586	-0.574845342410299\\
587	-0.595630612675905\\
588	-0.596393825711402\\
589	-0.585089082924696\\
590	-0.556731654740361\\
591	-0.519321184717228\\
592	-0.470860977559369\\
593	-0.406137299494731\\
594	-0.335869582552277\\
595	-0.256060832898129\\
597	-0.0787236320866214\\
599	0.105998167063717\\
600	0.193320869277159\\
601	0.278432372871521\\
602	0.356389129266063\\
603	0.426627667740831\\
604	0.484203180839359\\
605	0.533857941441056\\
606	0.565613870152902\\
607	0.586899489154348\\
608	0.59820651188511\\
609	0.586514760978844\\
610	0.566931697632754\\
611	0.53474125594812\\
612	0.486024120297316\\
613	0.43024470166938\\
614	0.360657431920117\\
615	0.281682557882505\\
617	0.10958260150619\\
618	0.0139332061248751\\
621	-0.250182040117579\\
622	-0.328784436977458\\
623	-0.398407658383803\\
624	-0.463571983544171\\
625	-0.516588286795013\\
626	-0.554970615035472\\
627	-0.581445494420223\\
628	-0.597490303269296\\
629	-0.59351732487994\\
630	-0.573746396012211\\
631	-0.546725109940326\\
632	-0.502474199182871\\
633	-0.451191493995793\\
634	-0.383214185349516\\
635	-0.307374659939796\\
636	-0.229732538132339\\
637	-0.138470058682287\\
639	0.0413860119879246\\
640	0.134363618629777\\
641	0.217961243070476\\
642	0.299930974515064\\
643	0.377317766433407\\
644	0.443941350953992\\
645	0.499121116154583\\
646	0.541594315522616\\
647	0.576845187085837\\
648	0.589704728983179\\
649	0.596479331420596\\
650	0.586749310445157\\
651	0.559799341593134\\
652	0.521499415845028\\
653	0.469546140426701\\
654	0.404962786595661\\
655	0.336383410588951\\
656	0.258876393523224\\
657	0.174732232673705\\
659	-0.0145232475447301\\
662	-0.276604759298152\\
663	-0.356615515591329\\
664	-0.419013135558089\\
665	-0.488194045989985\\
666	-0.531993497047097\\
667	-0.569047919707828\\
668	-0.589333832363536\\
669	-0.595011334837181\\
670	-0.588981544015951\\
671	-0.570444345777105\\
672	-0.533086914246724\\
673	-0.486686665200523\\
674	-0.428223908246309\\
675	-0.362025426147284\\
676	-0.281517219493253\\
677	-0.198424707796221\\
678	-0.105620817110776\\
679	-0.0174665220952193\\
680	0.0745565278225513\\
681	0.16213987267929\\
683	0.327685361681688\\
684	0.399297042814851\\
685	0.462711925840267\\
686	0.51710444727405\\
687	0.554209571863794\\
688	0.579777931024637\\
689	0.593425991488402\\
690	0.592410877290604\\
691	0.577951248740192\\
692	0.545456320487119\\
693	0.505774900196684\\
694	0.450746358651486\\
695	0.388240059424788\\
696	0.31043874930856\\
697	0.228785740058356\\
698	0.139230967889944\\
699	0.0481091496121735\\
700	-0.0399898099112761\\
701	-0.132154948560128\\
703	-0.303852732512041\\
704	-0.37293712252449\\
705	-0.443301274969144\\
706	-0.496998150620129\\
707	-0.543801793652619\\
708	-0.574616865411826\\
709	-0.590036754730136\\
710	-0.593526565359298\\
711	-0.586273827498189\\
712	-0.562268381829199\\
713	-0.522976947735515\\
714	-0.473035393343707\\
715	-0.407873857476261\\
716	-0.337363774341156\\
717	-0.257682271485464\\
718	-0.174180208731286\\
719	-0.0795538736119852\\
722	0.193444119022843\\
724	0.354735429863922\\
725	0.424265909776295\\
726	0.483812287131059\\
727	0.529474927260708\\
728	0.567323769246741\\
729	0.587447031073225\\
730	0.594611718359374\\
731	0.587751190902509\\
732	0.57250699649876\\
733	0.534467376371595\\
734	0.48946974209548\\
735	0.428952424763793\\
736	0.362558657913723\\
737	0.27893523482453\\
738	0.199904396439706\\
740	0.0200827489802577\\
742	-0.160363625969694\\
743	-0.241623469818023\\
744	-0.327719947244077\\
745	-0.399640324319535\\
746	-0.465758725109481\\
747	-0.515449762860044\\
748	-0.557256300225163\\
749	-0.58086591115898\\
750	-0.591855365261836\\
751	-0.59153150646307\\
752	-0.578515886065361\\
753	-0.551393023710261\\
754	-0.502191886749642\\
755	-0.446596206188588\\
756	-0.385878139982651\\
757	-0.314134641151213\\
758	-0.227529524089732\\
759	-0.144295861586215\\
760	-0.0534061823877892\\
761	0.0414660128385549\\
763	0.213820957838379\\
764	0.301728848693983\\
765	0.377154003901978\\
766	0.442973920951317\\
767	0.49983583571202\\
768	0.545320610022827\\
769	0.576254322338173\\
770	0.590349732617142\\
771	0.596811864260417\\
772	0.584636992986361\\
773	0.558489362573255\\
774	0.51891189259004\\
775	0.472569481532901\\
776	0.408685325655824\\
777	0.338200366186356\\
778	0.256891106740568\\
779	0.169622933809478\\
780	0.0837402251140702\\
781	-0.00853206631472858\\
782	-0.0972637547702107\\
783	-0.187298283593464\\
784	-0.271332583234198\\
785	-0.353563815509915\\
786	-0.423790787584039\\
787	-0.484253758027535\\
788	-0.528563026531629\\
789	-0.559116106881447\\
790	-0.585906467709719\\
791	-0.597478824686277\\
792	-0.587928205768094\\
793	-0.572281886038127\\
794	-0.53933505267787\\
795	-0.48823661955339\\
796	-0.433584478375906\\
797	-0.362903592508701\\
798	-0.283552086450072\\
799	-0.202073471816675\\
800	-0.112343651549509\\
801	-0.0166330533347718\\
802	0.0886367228990821\\
803	0.188496897295863\\
804	0.275348330695124\\
805	0.352620811609995\\
806	0.426912589790845\\
807	0.485167708169683\\
808	0.532114342095156\\
809	0.562123918424732\\
810	0.576222956592119\\
811	0.572490793221732\\
812	0.553863007069594\\
813	0.520213627477915\\
814	0.475260013496154\\
815	0.407756617420091\\
816	0.338484563853854\\
817	0.257084630202826\\
819	0.0734698375931657\\
821	-0.121528674193542\\
822	-0.210892596156555\\
823	-0.29728791242178\\
824	-0.373982768529459\\
825	-0.439495962997171\\
826	-0.488889875447057\\
827	-0.528626309917399\\
828	-0.543774254434993\\
829	-0.544238493949251\\
830	-0.526176882235177\\
831	0.576834288993723\\
832	0.50871913268702\\
833	0.441828129623445\\
834	0.369630654180128\\
835	0.291956813996876\\
838	0.0680507287775072\\
840	-0.0697154836971094\\
841	-0.124073202042837\\
842	-0.181742926694369\\
843	-0.229082658423977\\
844	-0.259146459589374\\
845	-0.290885520637858\\
846	-0.309021674251198\\
847	-0.320850847861038\\
848	-0.323051675278748\\
849	-0.319270349425551\\
850	-0.293936552854575\\
851	-0.270716991295558\\
852	-0.237676240125893\\
854	-0.156045095775426\\
855	-0.105050126795049\\
856	-0.0480944334212836\\
857	0.00551476704231391\\
858	0.0569787832259863\\
859	0.109942147459151\\
860	0.166822138926818\\
861	0.218773768399842\\
862	0.255995530947075\\
863	0.285626646546007\\
865	0.334613078726306\\
866	0.347527737137625\\
867	0.358483261807123\\
868	0.363698357564317\\
869	0.360980527067568\\
870	0.355699932593325\\
871	0.341103922838101\\
872	0.323626763260563\\
873	0.299070537914758\\
874	0.270147680967511\\
876	0.2028252558448\\
877	0.160749207262143\\
878	0.116218844981177\\
879	0.0676039563927588\\
881	-0.026589432307901\\
882	-0.0627591545162431\\
883	-0.111662168086696\\
884	-0.156626924123429\\
885	-0.20008949119665\\
886	-0.23797299995249\\
887	-0.26785130920689\\
888	-0.294841890077805\\
889	-0.326399980095175\\
890	-0.342584929678651\\
892	-0.366896576443196\\
893	-0.362112257215358\\
894	-0.360449388032976\\
895	-0.34488507978358\\
896	-0.333951835254084\\
898	-0.287291591482244\\
899	-0.254725747229713\\
900	-0.220349674323643\\
901	-0.177757251844923\\
902	-0.133623587904822\\
903	-0.0935326129388159\\
904	-0.0453984035220856\\
905	-0.000898666176908591\\
907	0.0953413727393126\\
908	0.137219808726968\\
910	0.216324737536524\\
911	0.253447615956702\\
912	0.284832669127354\\
913	0.309950766572911\\
914	0.333878678477504\\
915	0.352398039151467\\
916	0.359395073194719\\
917	0.363944948265271\\
918	0.357946394448845\\
919	0.353600447861027\\
920	0.338884155744381\\
921	0.32217950391032\\
922	0.29993599155523\\
923	0.273028462292132\\
924	0.240008505125843\\
926	0.15697702006355\\
927	0.11617616912099\\
930	-0.022171157192588\\
931	-0.0734827065584795\\
932	-0.114137480513818\\
933	-0.158872208396133\\
934	-0.201053034675624\\
935	-0.23834753616029\\
936	-0.267437467959553\\
937	-0.301215090380083\\
938	-0.326376683591661\\
939	-0.341975611988801\\
940	-0.351164895549118\\
941	-0.362951728323878\\
942	-0.363734434724392\\
943	-0.361966568953449\\
944	-0.345657257149014\\
945	-0.332899311149049\\
947	-0.286849612156402\\
950	-0.180934995302778\\
951	-0.136108089071513\\
952	-0.0927689256818667\\
953	-0.0456809461147714\\
957	0.136543351580258\\
958	0.181016650340098\\
959	0.222247950046494\\
960	0.251197986630359\\
961	0.290335949413475\\
962	0.315021390388665\\
963	0.333389328185604\\
964	0.348301740359602\\
965	0.357633244975204\\
966	0.365396396337019\\
967	0.360281210550056\\
968	0.353230521999194\\
969	0.342836979524236\\
970	0.322766780575876\\
971	0.299378639005681\\
972	0.269261232492681\\
973	0.233524724616927\\
975	0.157846677632733\\
976	0.11119108277353\\
977	0.0700860977922275\\
980	-0.0719338615863307\\
981	-0.116207750720605\\
982	-0.152363818650883\\
983	-0.20105632903369\\
984	-0.241583115835056\\
985	-0.267205712144005\\
986	-0.299105840108496\\
987	-0.325886436293786\\
988	-0.337227225703828\\
989	-0.354887421138756\\
990	-0.362686015872896\\
991	-0.36630428729768\\
992	-0.362252083720705\\
993	-0.35320374160392\\
995	-0.313393826070751\\
996	-0.282635612183185\\
997	-0.253197612116765\\
998	-0.212772141247569\\
1000	-0.134747585711921\\
1001	-0.091339528161825\\
1002	-0.0422205273580403\\
1003	-0.000732623234853236\\
1004	0.0499998493651219\\
1005	0.092872441057807\\
1006	0.137697053742158\\
1008	0.220906547725008\\
1009	0.253284978681677\\
1011	0.314012527741852\\
1012	0.333244161495713\\
1013	0.350855272463377\\
1014	0.359384978228263\\
1015	0.36496284683335\\
1016	0.356792094652519\\
1017	0.355775949977669\\
1018	0.345288056493246\\
1019	0.324195361571583\\
1020	0.29941174968144\\
1022	0.234658441029751\\
1023	0.197178292429726\\
1024	0.156037997641761\\
1025	0.110564888755562\\
1026	0.0663755986597607\\
1027	0.0203806888825966\\
1028	-0.0270236924261553\\
1029	-0.0704536002790519\\
1030	-0.111820993222864\\
1031	-0.163675655644056\\
1032	-0.201031002995023\\
1033	-0.241832510061158\\
1034	-0.269043691924253\\
1035	-0.297727345123803\\
1036	-0.328536423121477\\
1037	-0.339089215952754\\
1038	-0.358830688318903\\
1039	-0.363061134975396\\
1040	-0.362275451007463\\
1041	-0.359575928642698\\
1042	-0.344098108862454\\
1043	-0.326613373804321\\
1044	-0.310531995116435\\
1045	-0.288107198145099\\
1047	-0.218215697214873\\
1048	-0.178711032083356\\
1049	-0.137347695608696\\
1050	-0.0902544434284209\\
1051	-0.0398475555266486\\
1052	0.000501285909649596\\
1053	0.0481001090897735\\
1054	0.0971238610982255\\
1055	0.141466912922169\\
1057	0.221938855359895\\
1058	0.256929545093499\\
1060	0.31352316645598\\
1061	0.333095625936039\\
1063	0.361026006430166\\
1064	0.363905633937065\\
1065	0.361453406101646\\
1066	0.354649306317697\\
1067	0.342557601311\\
1068	0.321957162712806\\
1069	0.297032132442382\\
1070	0.267840458132014\\
1071	0.233463201377617\\
1072	0.193936474853217\\
1073	0.15647078783968\\
1074	0.114379971486414\\
1075	0.0647707053531121\\
1076	0.0239965989367192\\
1077	-0.0273945173430548\\
1079	-0.120257729147397\\
1080	-0.16305879166066\\
1081	-0.203874367141452\\
1082	-0.238152577919209\\
1084	-0.299709101976532\\
1085	-0.3227871157751\\
1086	-0.340154064059334\\
1087	-0.361547802546738\\
1088	-0.363359676264736\\
1089	-0.363217014480142\\
1091	-0.3509959741491\\
1092	-0.333681806795084\\
1094	-0.282433759505693\\
1096	-0.216626838208867\\
1097	-0.178657040491998\\
1098	-0.130967604826765\\
1099	-0.0854813535224821\\
1100	-0.0385275778376126\\
1101	0.00105278739647474\\
1102	0.0481019603153072\\
1103	0.0935592078139962\\
1104	0.140931796664063\\
1105	0.181491660904612\\
1106	0.224702692313713\\
1107	0.253693116553677\\
1109	0.314729170096598\\
1110	0.338055456830716\\
1111	0.349232126056904\\
1112	0.358101985265421\\
1113	0.364510448097462\\
1115	0.354735637707108\\
1116	0.342663849383825\\
1117	0.32182214091381\\
1118	0.292384656879221\\
1119	0.26584190691392\\
1121	0.199277344863731\\
1122	0.156267594572455\\
1123	0.110577481997097\\
1124	0.0681711752422416\\
1125	0.0164340726614682\\
1126	-0.0288011382294826\\
1127	-0.0711925257014627\\
1129	-0.160968086233879\\
1130	-0.201998171559808\\
1131	-0.241583138899387\\
1132	-0.271777211195513\\
1133	-0.303999945704163\\
1134	-0.324605298837014\\
1135	-0.343669818475064\\
1137	-0.364927398705731\\
1138	-0.363396472114346\\
1139	-0.357953113622898\\
1140	-0.350751391737958\\
1141	-0.331750002167155\\
1142	-0.306341877201703\\
1143	-0.282421824798803\\
1144	-0.251485983748353\\
1146	-0.175888549076262\\
1148	-0.088661574591697\\
1149	-0.0422952564363186\\
1151	0.0540927193524112\\
1152	0.0978424602240011\\
1153	0.144636391859422\\
1154	0.178281074639472\\
1155	0.22058989613015\\
1156	0.257350728532856\\
1157	0.285932618942297\\
1159	0.336403043696009\\
1160	0.351928807594504\\
1161	0.361006496221762\\
1162	0.365058579000561\\
1163	0.36004317631523\\
1164	0.352811985344033\\
1165	0.339747198977875\\
1166	0.318933010375076\\
1167	0.296379060937852\\
1168	0.271272858471548\\
1169	0.230878608948842\\
1170	0.195540821585382\\
1171	0.150763972251298\\
1172	0.112514072247905\\
1173	0.0619606064915388\\
1174	0.0182177227861757\\
1175	-0.030338754211698\\
1176	-0.0740053183985765\\
1177	-0.120114130052116\\
1178	-0.163293882631024\\
1179	-0.19695400098044\\
1180	-0.241918155733856\\
1181	-0.273642837125863\\
1182	-0.296117688456889\\
1183	-0.328256922787205\\
1184	-0.342713345792617\\
1185	-0.358918293045008\\
1187	-0.363293451595382\\
1188	-0.357322377003584\\
1189	-0.348218069109407\\
1190	-0.330195810345685\\
1191	-0.305132949745712\\
1192	-0.286150016092506\\
1193	-0.252793769212985\\
1194	-0.212088121229954\\
1195	-0.174862101747294\\
1196	-0.131641024098371\\
1197	-0.0897161456032336\\
1200	0.0512570270607284\\
1202	0.141742453793995\\
1203	0.182303718391267\\
1204	0.226134337289295\\
1205	0.256016627994995\\
1206	0.293118078722728\\
1207	0.314298032179977\\
1208	0.332770006909868\\
1209	0.349986433253434\\
1210	0.362996016494435\\
1211	0.366952086458241\\
1212	0.362665651378848\\
1213	0.35463929958587\\
1214	0.341364704470834\\
1215	0.316982345393171\\
1216	0.294015926465818\\
1217	0.261429493638843\\
1218	0.230355796778895\\
1219	0.195546480627399\\
1220	0.153937405832494\\
1221	0.113796657554303\\
1222	0.0636282987247796\\
1223	0.0120584775454518\\
1224	-0.0260487657565136\\
1225	-0.0731837822781927\\
1226	-0.122318120473665\\
1227	-0.163529984635716\\
1228	-0.202565400669755\\
1229	-0.240400541802046\\
1230	-0.272722014078681\\
1231	-0.303021229613478\\
1232	-0.326382519660001\\
1233	-0.344287349775641\\
1234	-0.358724815532241\\
1235	-0.36048407644239\\
1236	-0.364277496160412\\
1238	-0.34979714709516\\
1239	-0.333355306145222\\
1240	-0.308787632663552\\
1241	-0.281804680504138\\
1243	-0.211888597861162\\
1244	-0.172521275339477\\
1245	-0.134583113711869\\
1246	-0.0889409884039196\\
1247	-0.0403309618432104\\
1248	0.00512114754383219\\
1249	0.0552886822897563\\
1250	0.101026923133304\\
1251	0.140711047904915\\
1252	0.185577814060252\\
1253	0.225393600940151\\
1255	0.291562633873582\\
1256	0.316228003519427\\
1257	0.335143396761396\\
1258	0.351912824834017\\
1259	0.359913726729701\\
1260	0.362115964465829\\
1261	0.361417591025202\\
1262	0.3523281644907\\
1264	0.322326796687776\\
1265	0.293879728361844\\
1266	0.269665851879836\\
1267	0.228737010091663\\
1268	0.192682406266158\\
1269	0.153810595286814\\
1270	0.112771514494398\\
1271	0.0619610289759294\\
1272	0.0130663651602845\\
1273	-0.0301028848934948\\
1274	-0.0810615848386078\\
1276	-0.165359038406677\\
1278	-0.239755322270867\\
1279	-0.275908334080668\\
1280	-0.302180577721174\\
1281	-0.339391767478901\\
1282	-0.366689215365113\\
1284	-0.409006892136404\\
1285	-0.420201407002423\\
1286	-0.423043102284282\\
1287	-0.414382650938933\\
1288	-0.398312907310356\\
1289	-0.372954556438344\\
1290	-0.344946706418796\\
1291	-0.303009016780834\\
1292	-0.254324285622261\\
1294	-0.127463402389822\\
1295	-0.0545443988380612\\
1296	0.0314800385131093\\
1297	0.113623907945112\\
1299	0.290801531154102\\
1300	0.387184171383069\\
1301	0.480227104693768\\
1302	0.567659008892406\\
1303	0.657158148592316\\
1304	0.739163065535195\\
1305	0.816963709361971\\
1306	0.880899131835577\\
1307	0.94311099853121\\
1308	0.991015774097832\\
1309	1.02028348441945\\
1310	1.04141650942256\\
1311	-0.11719790073721\\
1312	-0.233861979285393\\
1313	-0.345450671564322\\
1314	-0.445389186669672\\
1315	-0.534477330494155\\
1316	-0.606747042224015\\
1317	-0.66200460965274\\
1318	-0.700023678677553\\
1319	-0.725470082670654\\
1320	-0.734697388583754\\
1321	-0.718655391842731\\
1322	-0.690477287586873\\
1323	-0.64537487432608\\
1324	-0.585169655355912\\
1325	-0.513326497231901\\
1326	-0.427565635671272\\
1327	-0.335192644742619\\
1328	-0.234184347758855\\
1331	0.080442925278021\\
1332	0.18134356246901\\
1333	0.276360519002992\\
1334	0.363392041066618\\
1335	0.440687418284142\\
1336	0.502317929619494\\
1337	0.549615446459484\\
1338	0.585873790767437\\
1339	0.602507950318795\\
1340	0.601549964843343\\
1341	0.586703567628774\\
1342	0.570212078110671\\
1343	0.541908524830433\\
1344	0.496572317174014\\
1345	0.440019134502563\\
1346	0.374924270680822\\
1347	0.297634033261147\\
1348	0.214839521565864\\
1349	0.129576460782118\\
1350	0.0388948000127129\\
1351	-0.049309810638988\\
1352	-0.144698772568063\\
1353	-0.232423831244432\\
1354	-0.315438080460353\\
1355	-0.388695032816486\\
1356	-0.453236090829705\\
1357	-0.511093915018591\\
1358	-0.549731259946384\\
1359	-0.576419604024068\\
1360	-0.594545143327196\\
1361	-0.59322147212697\\
1362	-0.578996436014222\\
1363	-0.551843797277797\\
1364	-0.512385061801069\\
1365	-0.466792500277279\\
1366	-0.399631137459437\\
1367	-0.323506079657363\\
1368	-0.244024369658746\\
1369	-0.158509637917632\\
1370	-0.067366708366535\\
1371	0.0260298899838745\\
1372	0.11513675491824\\
1373	0.20160080650794\\
1374	0.286518193476695\\
1375	0.363304780231829\\
1376	0.433914235753036\\
1377	0.490726429500228\\
1378	0.537174556440732\\
1379	0.569715414206257\\
1380	0.588407319348335\\
1381	0.595125804444251\\
1382	0.587058937212532\\
1383	0.561081937489689\\
1384	0.528831365315909\\
1385	0.482136460890615\\
1386	0.419968581993544\\
1387	0.353843917655922\\
1388	0.273435336872353\\
1389	0.186606719431438\\
1390	0.0980485924851564\\
1391	0.00630104279434818\\
1392	-0.084202268736135\\
1394	-0.260584159655536\\
1395	-0.33825912664588\\
1396	-0.406552682066831\\
1397	-0.469855405492581\\
1398	-0.520340587320334\\
1399	-0.561852866102072\\
1400	-0.585658520155903\\
1401	-0.594470051447388\\
1402	-0.591598874442298\\
1403	-0.577276200088818\\
1404	-0.545651317365355\\
1405	-0.495716749622716\\
1406	-0.44359602147324\\
1407	-0.370615125880704\\
1408	-0.304620294647975\\
1409	-0.215902337401076\\
1410	-0.129015147486371\\
1411	-0.0396920913403846\\
1412	0.0547988598859774\\
1413	0.14272256703498\\
1414	0.23364860675747\\
1415	0.313553928583588\\
1416	0.383935625386584\\
1417	0.447217460476168\\
1418	0.506198710062563\\
1419	0.551534095183797\\
1420	0.578484669623776\\
1421	0.593536560344546\\
1422	0.59616577633642\\
1423	0.578571856658073\\
1424	0.558040576634539\\
1425	0.51284346756438\\
1426	0.462658884959183\\
1427	0.398487926784583\\
1428	0.326821870575714\\
1429	0.241545798252901\\
1430	0.162889431882377\\
1431	0.0704923301709641\\
1432	-0.0181149917707444\\
1434	-0.202484447346706\\
1435	-0.288389067472053\\
1436	-0.367195398321201\\
1437	-0.431627882584053\\
1438	-0.489307782151172\\
1439	-0.533855071064863\\
1440	-0.571983709873621\\
1441	-0.593517541001347\\
1442	-0.60589778505755\\
1443	-0.603584933271577\\
1444	-0.580967477833383\\
1445	-0.548392102049093\\
1446	-0.501962421196367\\
1447	-0.436959658607975\\
1448	-0.368112533156818\\
1449	-0.288483827504933\\
1450	-0.198633107249861\\
1451	-0.104415775652797\\
1452	-0.0061218123796607\\
1453	0.0936879226815108\\
1454	0.189353584099081\\
1455	0.283359076112902\\
1456	0.374680837041069\\
1457	0.458967262311489\\
1458	0.525321160264411\\
1459	0.583672046840093\\
1460	0.632671631264202\\
1461	0.665501029258394\\
1462	0.684784007681628\\
1463	0.690633203059406\\
1464	0.678525269725924\\
1465	0.652267487213976\\
1466	0.613621209729899\\
1467	0.560016867314971\\
1468	0.497977422152417\\
1469	0.423900493957717\\
1470	0.345237865002218\\
1471	-0.305832393780747\\
1472	-0.359887991653068\\
1473	-0.40614212905075\\
1474	-0.447253746365277\\
1475	-0.475359006760755\\
1476	-0.496825421343146\\
1477	-0.513565025676598\\
1478	-0.514504331628814\\
1479	-0.518069645728701\\
1480	-0.50523257941677\\
1481	-0.490348388980692\\
1482	-0.463821742864639\\
1483	-0.435966767608988\\
1484	-0.398884216289389\\
1485	-0.35923657694093\\
1486	-0.316333134015622\\
1487	-0.271308002428668\\
1489	-0.162081296411543\\
1490	-0.109958937890951\\
1491	-0.0644236680450376\\
1492	-0.0125022643910597\\
1494	0.0847831534792931\\
1495	0.127389825170212\\
1496	0.166866367629609\\
1497	0.209162183607987\\
1498	0.240512675997252\\
1499	0.268566466020729\\
1500	0.302965827400385\\
1501	0.31702583265087\\
1502	0.338689513763256\\
1503	0.355143371069062\\
1504	0.357590353604337\\
1505	0.365253861753445\\
1506	0.362405126052181\\
1507	0.352485631247873\\
1508	0.33789863912807\\
1510	0.293425076270978\\
1512	0.22524737971662\\
1513	0.190327623577559\\
1514	0.151030768189685\\
1515	0.102535390842149\\
1516	0.0567069688390802\\
1518	-0.0329034286219212\\
1520	-0.125679676793879\\
1521	-0.170491219175346\\
1523	-0.245681144460832\\
1524	-0.274896420730329\\
1525	-0.309037246118805\\
1526	-0.328442794802413\\
1528	-0.359161081284128\\
1529	-0.36536045357434\\
1530	-0.36597728271272\\
1531	-0.358592319293166\\
1532	-0.34654612219947\\
1533	-0.328229833150999\\
1534	-0.307543408130641\\
1536	-0.246354137161688\\
1537	-0.209289591645756\\
1538	-0.170393926640372\\
1539	-0.126395494876306\\
1540	-0.072977896937573\\
1541	-0.0271806968271449\\
1542	0.0145442570014893\\
1543	0.0602692122010922\\
1544	0.101221337511106\\
1545	0.146595865361633\\
1547	0.228790787058188\\
1548	0.263226244349426\\
1549	0.292623683348211\\
1550	0.317054148764328\\
1551	0.335969708185075\\
1552	0.352736174170332\\
1553	0.364109701507914\\
1554	0.365448699046738\\
1555	0.361749534687988\\
1556	0.353810335635444\\
1557	0.338871308455509\\
1558	0.313568472638053\\
1559	0.291680272397571\\
1560	0.260382353288605\\
1561	0.222147545049665\\
1562	0.185340409380387\\
1563	0.1461254721562\\
1564	0.104816363564169\\
1565	0.0569106324228414\\
1566	0.0114792237059191\\
1568	-0.0867207595838408\\
1569	-0.126423190530659\\
1570	-0.171422524917944\\
1572	-0.24191246881901\\
1573	-0.279024128288711\\
1575	-0.326697592743585\\
1576	-0.349335295041328\\
1577	-0.357934832947194\\
1578	-0.363084556190643\\
1579	-0.362317668326341\\
1580	-0.355825265413841\\
1581	-0.343781405478239\\
1582	-0.326021688708806\\
1583	-0.30514460208633\\
1584	-0.27548690511685\\
1585	-0.248686244265627\\
1586	-0.209586451574978\\
1587	-0.16422472602062\\
1588	-0.125591851863192\\
1589	-0.0784623311042196\\
1590	-0.0296667358747982\\
1591	0.0117574071623494\\
1594	0.150658111736902\\
1595	0.190495646149429\\
1596	0.226961874912377\\
1598	0.292975662077879\\
1599	0.318678113886563\\
1600	0.338050147322065\\
1601	0.353169675377103\\
1603	0.366378668651123\\
1604	0.357045583929903\\
1605	0.353046606627231\\
1606	0.339186282908031\\
1608	0.291326589468099\\
1609	0.261540518753463\\
1611	0.187536599286886\\
1612	0.146886656625611\\
1613	0.0998929101056092\\
1614	0.0582252924327804\\
1615	0.0122358009562049\\
1616	-0.0361881610351702\\
1618	-0.130383116407756\\
1619	-0.16934596416786\\
1620	-0.213048297549449\\
1621	-0.243215707333547\\
1622	-0.277909772900784\\
1623	-0.306185512949469\\
1624	-0.330337045130818\\
1625	-0.351632019448061\\
1626	-0.361046604886724\\
1627	-0.363213142311452\\
1628	-0.358274670536957\\
1629	-0.355749419416497\\
1630	-0.344596369092869\\
1631	-0.328724616977524\\
1632	-0.30404158888814\\
1634	-0.243962694705715\\
1635	-0.202441187311251\\
1636	-0.169294238611201\\
1637	-0.124608864402489\\
1638	-0.0787474378439583\\
1639	-0.0370887225208207\\
1640	0.0152138690937136\\
1643	0.151364995804215\\
1644	0.193164600841556\\
1645	0.230698211669733\\
1647	0.293105902932439\\
1648	0.319190106874885\\
1649	0.335531016084587\\
1650	0.350356414267026\\
1651	0.360326351993535\\
1652	0.364076916523118\\
1653	0.36455307416054\\
1654	0.353023415928419\\
1655	0.335183136688556\\
1657	0.293347708692636\\
1658	0.265480383261092\\
1660	0.184961621723232\\
1661	0.141322566977578\\
1662	0.102750050889881\\
1663	0.0556544177870819\\
1664	0.0110858178736635\\
1665	-0.0401591483587254\\
1666	-0.0835748966233041\\
1667	-0.130198834077873\\
1668	-0.168376564502069\\
1669	-0.212150724566072\\
1670	-0.24731310570678\\
1671	-0.280084711484051\\
1673	-0.328389808202701\\
1674	-0.348737117118617\\
1675	-0.359336784760671\\
1676	-0.365824019108913\\
1677	-0.361867622291811\\
1678	-0.355837615029486\\
1679	-0.347975486833548\\
1680	-0.330614736063126\\
1681	-0.303132270720198\\
1682	-0.273178825522336\\
1683	-0.241575578731954\\
1684	-0.208696896170295\\
1685	-0.168586851334567\\
1686	-0.120287769937477\\
1687	-0.0776978039298228\\
1688	-0.0321381127851055\\
1689	0.0182760127277106\\
1690	0.0599096307823856\\
1691	0.107398619677042\\
1692	0.15169987948093\\
1693	0.193727599090835\\
1694	0.227957231448727\\
1695	0.264851366438052\\
1696	0.300085856743408\\
1697	0.315623507877262\\
1698	0.339795982730266\\
1699	0.35250393021397\\
1700	0.359637143398231\\
1701	0.362104016245212\\
1702	0.361557086649555\\
1703	0.356152148441652\\
1704	0.33401040540457\\
1705	0.316288650472416\\
1706	0.287953550399379\\
1707	0.262972333593098\\
1708	0.224893512003291\\
1710	0.144788819993664\\
1711	0.0994185368144826\\
1712	0.0562543601058678\\
1714	-0.0419189551262207\\
1715	-0.0848619680450611\\
1716	-0.126522373364423\\
1717	-0.175185785042686\\
1718	-0.212740468210086\\
1719	-0.246131182997033\\
1720	-0.284845154388222\\
1722	-0.328857949589747\\
1723	-0.347119823135927\\
1724	-0.360656931039102\\
1725	-0.364394356285629\\
1726	-0.36365237837208\\
1727	-0.355342621651289\\
1728	-0.343574930967861\\
1729	-0.327794868039291\\
1730	-0.299938863395255\\
1731	-0.27524591590236\\
1732	-0.244005906657549\\
1733	-0.206297179117882\\
1734	-0.163139836558003\\
1735	-0.121624295561105\\
1736	-0.0750697690641573\\
1737	-0.0306273535461514\\
1738	0.0214815288127284\\
1739	0.0587125108181681\\
1740	0.106994033330921\\
1741	0.151411161127726\\
1743	0.234123457739315\\
1744	0.265598852809489\\
1745	0.294749382051577\\
1746	0.320193488164023\\
1747	0.341463677697448\\
1749	0.363391037461497\\
1750	0.361296525688431\\
1751	0.361098415878132\\
1752	0.349643059798836\\
1753	0.334975969143215\\
1754	0.31620502052283\\
1755	0.287038529530037\\
1756	0.256539606801653\\
1757	0.220310832734413\\
1758	0.179996343829316\\
1759	0.142847180060016\\
1760	0.101745797147487\\
1762	0.0107310806779424\\
1763	-0.0385525916631195\\
1764	-0.0814741708823021\\
1765	-0.134905542355682\\
1766	-0.180309257265435\\
1767	-0.229802145734538\\
1768	-0.276187232463144\\
1769	-0.314762260773023\\
1770	-0.352007175364179\\
1771	-0.385994286552886\\
1772	-0.40795865434302\\
1773	-0.424585340917474\\
1774	-0.438284037926678\\
1775	-0.439327827855323\\
1776	-0.428959920519446\\
1777	-0.410244056982265\\
1778	-0.380186754397982\\
1779	-0.343131329229436\\
1780	-0.291261471193138\\
1781	-0.230679254787447\\
1782	-0.163183848279004\\
1783	-0.084832432899475\\
1784	-0.00143658315255379\\
1785	0.0923916113415544\\
1786	0.191333143296561\\
1787	0.295888634780567\\
1788	0.405459659348253\\
1789	0.510735252775248\\
1790	0.627485795366738\\
1791	1.07096171345574\\
1792	0.976933300487417\\
1793	0.877045667374659\\
1794	0.765532382242327\\
1796	0.511020255701624\\
1797	0.380062449067736\\
1798	0.241535650485275\\
1799	0.10762872018222\\
1800	-0.0242823387529825\\
1801	-0.145991548165512\\
1802	-0.26095723314711\\
1803	-0.358554150600867\\
1804	-0.448059627219664\\
1805	-0.517608413660128\\
1806	-0.57229936752401\\
1807	-0.610915629804822\\
1808	-0.633730444599223\\
1809	-0.637164983680123\\
1810	-0.623642308795752\\
1811	-0.588765827674251\\
1812	-0.53945172730937\\
1813	-0.480945707812225\\
1814	-0.406971687194527\\
1815	-0.321563539966519\\
1817	-0.13838889004046\\
1818	-0.0357385483043799\\
1819	0.0632055302467052\\
1820	0.158736212993517\\
1821	0.252109840939738\\
1822	0.329485148028198\\
1823	0.400878329680836\\
1824	0.466141286529819\\
1825	0.51825644353994\\
1826	0.556156543133966\\
1827	0.584619843840755\\
1828	0.591729679169475\\
1829	0.590881996748976\\
1830	0.570476504487488\\
1831	0.548785565189974\\
1832	0.502691922735266\\
1833	0.449616914527269\\
1834	0.386988918594398\\
1835	0.31027317160806\\
1836	0.227193290123978\\
1838	0.0494196764643675\\
1839	-0.0406151334823335\\
1840	-0.136185993108029\\
1841	-0.220644931805509\\
1843	-0.381382871144524\\
1844	-0.44295672593671\\
1845	-0.498573616489011\\
1846	-0.545846058671486\\
1847	-0.578026298018813\\
1848	-0.590784202068789\\
1849	-0.593000248489261\\
1850	-0.582355304781231\\
1851	-0.55693626660468\\
1852	-0.523000563375717\\
1853	-0.468317929676232\\
1854	-0.409018442983324\\
1855	-0.337398530455175\\
1856	-0.26003338662531\\
1857	-0.168407424553152\\
1858	-0.0807375974200113\\
1860	0.105377067001427\\
1861	0.188478478220077\\
1862	0.276815577434263\\
1863	0.353211600864142\\
1864	0.424725595762538\\
1865	0.484467125405445\\
1866	0.531691463073003\\
1867	0.563280765304171\\
1868	0.58662248076962\\
1869	0.595293245772154\\
1870	0.595330226519309\\
1871	0.570612516862184\\
1872	0.539110701382924\\
1873	0.488152486649597\\
1874	0.429481900527662\\
1875	0.361891339809517\\
1877	0.202352303553198\\
1878	0.110668373548833\\
1879	0.0166812876937001\\
1880	-0.0700460958464646\\
1881	-0.165525118167807\\
1882	-0.252304390439349\\
1884	-0.402913994900246\\
1885	-0.465660439482235\\
1886	-0.515825423381102\\
1887	-0.556069446849961\\
1888	-0.581850695876255\\
1889	-0.593224118134003\\
1890	-0.594794081562213\\
1891	-0.57915548495339\\
1892	-0.551563981885465\\
1893	-0.50521998522936\\
1894	-0.448333633242783\\
1895	-0.386156819471125\\
1896	-0.309114045931437\\
1897	-0.227239186129282\\
1898	-0.138988720504585\\
1899	-0.0483417662353531\\
1900	0.0397585034952499\\
1902	0.219870848369283\\
1904	0.380749678716256\\
1905	0.446906222751295\\
1906	0.499740807788839\\
1907	0.542420818263054\\
1908	0.575356770628332\\
1909	0.589978598138259\\
1910	0.594340901888245\\
1911	0.582404012955976\\
1912	0.558893669995996\\
1913	0.522990744208528\\
1914	0.47146925228526\\
1915	0.410550773826344\\
1917	0.25733280960776\\
1918	0.169053530772544\\
1919	0.0823549409378757\\
1922	-0.192686979368318\\
1923	-0.280905796687421\\
1924	-0.364281529683012\\
1925	-0.436498813813614\\
1926	-0.501863422089173\\
1927	-0.548220567346561\\
1928	-0.587258082673998\\
1929	-0.604004535142394\\
1930	-0.611336869798834\\
1931	-0.605042931167645\\
1932	-0.58297019999145\\
1933	-0.543852613014224\\
1934	-0.488643233189578\\
1935	-0.424470910206765\\
1936	-0.352347714398093\\
1937	-0.265137724903525\\
1938	-0.174417286865264\\
1940	0.0313881754141221\\
1941	0.13515069613004\\
1942	0.231742971307995\\
1943	0.334476783033551\\
1944	0.423647977715518\\
1945	0.508878710340468\\
1946	0.585613632446893\\
1947	0.647089662285907\\
1948	0.695612845194773\\
1949	0.732875068811154\\
1950	0.755924081936882\\
1951	0.547922236221439\\
1953	0.406083515614682\\
1955	0.258796627492302\\
1956	0.187313509831256\\
1957	0.10843491507103\\
1958	0.0429825681608236\\
1959	-0.0265995200102225\\
1960	-0.0858160970205972\\
1961	-0.148782881373336\\
1962	-0.200085046693403\\
1963	-0.249556523065621\\
1964	-0.29004538907293\\
1965	-0.323592856294454\\
1966	-0.351184506824666\\
1967	-0.372542622858418\\
1968	-0.382715036596892\\
1969	-0.390943447473546\\
1970	-0.388381118798861\\
1971	-0.387142443944413\\
1972	-0.372071001460426\\
1973	-0.361338956904547\\
1974	-0.332243708679925\\
1975	-0.305088296017857\\
1976	-0.274833907172251\\
1978	-0.19808047355491\\
1980	-0.115582570737388\\
1983	0.0177084000306422\\
1984	0.065917671223815\\
1985	0.11221046358105\\
1986	0.157285812610553\\
1987	0.19420775430126\\
1988	0.232519867158771\\
1989	0.273947158648298\\
1990	0.295300502969894\\
1991	0.325472972658645\\
1992	0.342511002740139\\
1993	0.352687415892433\\
1994	0.361441253114208\\
1995	0.363003115988249\\
1996	0.36033592976446\\
1997	0.352213382388982\\
1998	0.329684057116083\\
1999	0.316462585944464\\
2000	0.287161325600664\\
2002	0.218523638315219\\
2003	0.180490744316103\\
2005	0.0963640486093027\\
2006	0.0503097681544205\\
2007	0.000510234604462312\\
2008	-0.0456166592184672\\
2009	-0.0895448693281651\\
2010	-0.135296308923444\\
2011	-0.17520194389499\\
2012	-0.218542247012465\\
2013	-0.251659516368818\\
2014	-0.283597906273371\\
2015	-0.308011644036469\\
2016	-0.33593265556101\\
2017	-0.349251667027602\\
2019	-0.369779157000721\\
2020	-0.360553884324872\\
2021	-0.34980835893839\\
2022	-0.343588787370209\\
2023	-0.323763460491591\\
2024	-0.301372146864651\\
2026	-0.238855812103338\\
2027	-0.20094975423126\\
2028	-0.156541098209345\\
2029	-0.11881992893359\\
2030	-0.0732202433177918\\
2032	0.0206850571266841\\
2033	0.0705136596761804\\
2034	0.117060840375416\\
2035	0.153688183045688\\
2036	0.195174111435335\\
2037	0.235248105717801\\
2038	0.266849790457854\\
2039	0.301419230844203\\
2040	0.324938222870514\\
2041	0.341531265519734\\
2042	0.354952563846382\\
2043	0.360552403750262\\
2044	0.364635663809622\\
2045	0.361116423379826\\
2047	0.336629321318469\\
2048	0.312322707015483\\
2049	0.283986713273862\\
2050	0.253829380588741\\
2051	0.219991865682005\\
2052	0.177529797814259\\
2053	0.136251331028689\\
2054	0.0974719619512143\\
2055	0.0454017521969945\\
2056	0.00202684669420705\\
2057	-0.0447720048655356\\
2058	-0.0888306596175426\\
2059	-0.135506565541618\\
2060	-0.184244810727705\\
2061	-0.219726856346824\\
2063	-0.28420827033824\\
2065	-0.330076853905211\\
2066	-0.348229522616293\\
2067	-0.359568780411337\\
2068	-0.367720603371254\\
2069	-0.358602110362426\\
2070	-0.355066912694383\\
2071	-0.345586730943069\\
2073	-0.298864490766391\\
2074	-0.271258059506181\\
2075	-0.23537635264347\\
2076	-0.200646901694654\\
2077	-0.161377275506311\\
2078	-0.11901131172317\\
2079	-0.0706965428880721\\
2080	-0.0249179271240791\\
2081	0.0397576750938242\\
2082	0.0976672472384053\\
2083	0.161329207798644\\
2084	0.216171254046458\\
2085	0.276734272025351\\
2087	0.3623816824188\\
2088	0.391847287574365\\
2089	0.414085466557026\\
2090	0.420184855257503\\
2091	0.424919877466891\\
2092	0.412722910687535\\
2094	0.347144112094611\\
2095	0.296987780611744\\
2096	0.238165969855345\\
2097	0.164540850536014\\
2098	0.0857900765499835\\
2099	-0.00333550843970443\\
2100	-0.101299812228717\\
2101	-0.204919328235064\\
2102	-0.314221046955026\\
2103	-0.419995768719218\\
2104	-0.530222805864469\\
2105	-0.638886565134726\\
2106	-0.739280742776373\\
2107	-0.83679497363164\\
2108	-0.92518209883383\\
2109	-1.0029378133795\\
2110	-1.07207628094511\\
2111	0.517249917361369\\
2112	0.595919605830204\\
2113	0.657906458606703\\
2114	0.700216613115572\\
2115	0.722227671299152\\
2116	0.719187763519585\\
2117	0.69243561440453\\
2118	0.648394613217533\\
2119	0.585651798113304\\
2120	0.514348163147133\\
2121	0.423695332608077\\
2122	0.321435251566527\\
2123	0.216326083857439\\
2124	0.101717202579948\\
2125	-0.00572596131451064\\
2126	-0.117626113288679\\
2127	-0.219255877417254\\
2128	-0.323864254258297\\
2129	-0.4063334546845\\
2130	-0.483210657193922\\
2131	-0.543896510301693\\
2132	-0.586948878459225\\
2133	-0.613193777151992\\
2134	-0.627793973060761\\
2135	-0.613953370936542\\
2136	-0.593365862715928\\
2137	-0.549870137741891\\
2138	-0.490039908982908\\
2139	-0.413782750379141\\
2140	-0.332387689014013\\
2141	-0.2338276700184\\
2142	-0.146730217441473\\
2143	-0.0568105940178611\\
2144	0.0359365259378137\\
2145	0.123128649000591\\
2146	0.216007301626178\\
2147	0.296919824050292\\
2148	0.374168962137901\\
2149	0.437327465080671\\
2150	0.496215435315662\\
2151	0.542756540081882\\
2152	0.574877804575408\\
2153	0.590628990450114\\
2154	0.590793249090439\\
2155	0.582443572381635\\
2156	0.561396221886753\\
2157	0.52536116174997\\
2158	0.474810965961296\\
2160	0.346724486621497\\
2162	0.179177150370833\\
2164	-0.0065687222431734\\
2166	-0.18267573776393\\
2168	-0.349552082885566\\
2169	-0.422186757386953\\
2171	-0.527047603376559\\
2172	-0.566454694097956\\
2173	-0.588043058578478\\
2174	-0.597525787169161\\
2175	-0.587888135600224\\
2176	-0.569790012154499\\
2177	-0.537748547961201\\
2178	-0.492058908146646\\
2179	-0.436966929942173\\
2180	-0.365123081245656\\
2181	-0.2900609826861\\
2182	-0.208025802160591\\
2183	-0.119491670116986\\
2184	-0.0281539032248475\\
2185	0.0619689590139387\\
2186	0.155966223176165\\
2187	0.244058130387657\\
2188	0.3273684639762\\
2189	0.396188526196511\\
2190	0.457067007000205\\
2191	0.51215661933702\\
2192	0.550149886526924\\
2193	0.578506350141197\\
2194	0.592145015207734\\
2195	0.591348091766577\\
2196	0.579497681508656\\
2197	0.547788845904506\\
2198	0.510290241066286\\
2199	0.457041731006484\\
2200	0.39316273811437\\
2201	0.317003674442276\\
2202	0.237112632996286\\
2203	0.152909564112633\\
2204	0.0585839348627815\\
2205	-0.0337607554447459\\
2206	-0.120436113685173\\
2207	-0.218086157638481\\
2208	-0.296560744068756\\
2209	-0.366120635584139\\
2210	-0.439745696789032\\
2211	-0.496629477977876\\
2212	-0.54302461144789\\
2213	-0.5705526134102\\
2214	-0.595431090441707\\
2215	-0.596835777301294\\
2216	-0.583888386579019\\
2217	-0.561366020869627\\
2218	-0.522822223969342\\
2219	-0.474318395285991\\
2220	-0.41212293460012\\
2221	-0.345696080437847\\
2223	-0.182019655760541\\
2224	-0.0872284743172713\\
2225	0.0020032364564031\\
2226	0.0958286737300114\\
2228	0.269529606772267\\
2229	0.349780800343069\\
2230	0.413382855295367\\
2231	0.474053419246502\\
2232	0.523230671638885\\
2233	0.565751214655847\\
2234	0.587848037798722\\
2235	0.59452917371982\\
2236	0.594453807454556\\
2237	0.572322589919168\\
2238	0.535857865758771\\
2239	0.49162572854766\\
2240	0.433524840378595\\
2241	0.37110860722305\\
2242	0.291243968591061\\
2243	0.206667900258708\\
2244	0.119647804694978\\
2245	0.0279023235034401\\
2246	-0.0601549624011568\\
2247	-0.151410981501158\\
2248	-0.238549957864052\\
2249	-0.31937755420995\\
2250	-0.39362012831316\\
2251	-0.459869377739778\\
2252	-0.509634043427468\\
2253	-0.55362449940867\\
2254	-0.581928580139902\\
2255	-0.595977471445622\\
2256	-0.591483109454202\\
2257	-0.577013179644382\\
2258	-0.555047041044872\\
2259	-0.510216706231404\\
2260	-0.455985084094664\\
2261	-0.393655892141851\\
2262	-0.317004585987888\\
2263	-0.236939954912941\\
2264	-0.151045056073144\\
2265	-0.061570465699333\\
2266	0.0330567603155032\\
2267	0.124458451317423\\
2268	0.211709140485254\\
2269	0.293269448130104\\
2270	0.372006047936793\\
2271	0.435258687847636\\
2272	0.494464323134252\\
2273	0.542425974519119\\
2274	0.571150998746816\\
2275	0.58835385381235\\
2276	0.599178409584965\\
2277	0.584690481649432\\
2278	0.562694618756268\\
2279	0.523026691682844\\
2280	0.476039374835182\\
2281	0.415703023028982\\
2282	0.344095375271536\\
2283	0.266029370761771\\
2284	0.182852074635321\\
2285	0.0876764570316482\\
2286	-0.000300429634535249\\
2287	-0.0967722332011363\\
2288	-0.181625607737715\\
2290	-0.343261968155275\\
2291	-0.415852393043679\\
2292	-0.477148463307458\\
2293	-0.524463042177558\\
2294	-0.55998461462832\\
2295	-0.585611322218938\\
2296	-0.591395104915591\\
2297	-0.592025506140544\\
2298	-0.575759734344047\\
2299	-0.540505697058961\\
2300	-0.493950849266184\\
2301	-0.437303127205723\\
2302	-0.370891576008034\\
2303	-0.289766987264557\\
2304	-0.213149399727172\\
2306	-0.0323899138320485\\
2307	0.0605233627493362\\
2309	0.239802563663034\\
2310	0.317521523153573\\
2311	0.39150551027933\\
2312	0.457693875384393\\
2313	0.513092288591452\\
2314	0.54756640672349\\
2315	0.583656162701118\\
2316	0.59185569526835\\
2317	0.595247880055922\\
2318	0.582086523462294\\
2319	0.554767823929524\\
2320	0.509551095694405\\
2321	0.457519320265419\\
2322	0.392999372689246\\
2323	0.31856390970961\\
2324	0.239403960680647\\
2325	0.152750219525387\\
2327	-0.0300684522176198\\
2328	-0.115500617985617\\
2329	-0.209718721834179\\
2330	-0.29133517839773\\
2331	-0.370261941610806\\
2332	-0.437927207994562\\
2333	-0.494587331311777\\
2334	-0.537960881633808\\
2335	-0.573473919042044\\
2336	-0.594702504289216\\
2337	-0.595267214067917\\
2338	-0.587021911062948\\
2339	-0.562074805782686\\
2340	-0.525844054316622\\
2341	-0.475176555355119\\
2342	-0.414233101597347\\
2343	-0.348751762875963\\
2344	-0.270525791088858\\
2346	-0.0919447605438108\\
2347	-0.00312079462082693\\
2348	0.0933426064466403\\
2349	0.1801129206101\\
2350	0.265437752841081\\
2351	0.344662297458399\\
2352	0.416377166300208\\
2353	0.478894791375751\\
2354	0.524353029370104\\
2355	0.558052976759427\\
2356	0.587782376378982\\
2357	0.593090637411024\\
2358	0.593423581833122\\
2359	0.570689791515179\\
2360	0.54315529687301\\
2361	0.492950132265833\\
2362	0.435455177882432\\
2363	0.37281639336652\\
2364	0.291533327731031\\
2365	0.212976284038632\\
2366	0.124304481964373\\
2368	-0.0603023343096538\\
2369	-0.14537744970994\\
2370	-0.237671162542938\\
2371	-0.318446720941211\\
2372	-0.393779458784593\\
2373	-0.457840726284303\\
2374	-0.510724529143772\\
2375	-0.546881478843716\\
2376	-0.579787199308157\\
2377	-0.594888078557233\\
2378	-0.594636495572558\\
2379	-0.57820138886882\\
2380	-0.555072601994652\\
2381	-0.509908503479437\\
2382	-0.461320295236874\\
2383	-0.394710827019026\\
2384	-0.325210944743503\\
2386	-0.151908105762686\\
2387	-0.0646845396845492\\
2388	0.0294346455270897\\
2390	0.210769529328445\\
2391	0.291737712324448\\
2392	0.370228518569547\\
2393	0.435333453415296\\
2394	0.491415630592201\\
2395	0.538387538188545\\
2396	0.570750982952177\\
2397	0.592326260180926\\
2398	0.595340777979345\\
2399	0.584502269102359\\
2400	0.562979799457025\\
2401	0.533665580015168\\
2402	0.48435410303955\\
2403	0.425707583887288\\
2404	0.360322342719428\\
2405	0.280032882050364\\
2406	0.193246149789047\\
2407	0.101392522755759\\
2408	0.00561398988293149\\
2409	-0.0856803814208433\\
2410	-0.180175261934437\\
2412	-0.356004778148417\\
2413	-0.425460908041259\\
2414	-0.493175062853425\\
2415	-0.552097024708473\\
2416	-0.587864895761868\\
2417	-0.619363505464662\\
2418	-0.630696130953311\\
2419	-0.62943549061265\\
2420	-0.615703400730126\\
2421	-0.589613398960864\\
2422	-0.537886522913595\\
2423	-0.483299657541011\\
2424	-0.416173329757385\\
2425	-0.336705140421145\\
2426	-0.254431874800957\\
2428	-0.0686155472044447\\
2429	0.023559482723158\\
2430	0.121171177149336\\
2431	0.644708786061528\\
2432	0.658808824784501\\
2433	0.656814522374589\\
2434	0.650278326956141\\
2435	0.637541487055387\\
2436	0.614665035536291\\
2437	0.583762730401304\\
2438	0.543803721714085\\
2439	0.499141323647109\\
2440	0.451796743265731\\
2441	0.401143135904022\\
2442	0.34181456005399\\
2443	0.280774579670833\\
2444	0.218515023258078\\
2445	0.159587064581501\\
2446	0.0964618330895064\\
2447	0.038269769604085\\
2448	-0.018329073523546\\
2449	-0.0731531198307493\\
2450	-0.125714565755061\\
2451	-0.1717766035149\\
2452	-0.214062486943931\\
2453	-0.260937018925233\\
2454	-0.293918402941927\\
2455	-0.319708371060187\\
2456	-0.344094150083492\\
2457	-0.356539733151749\\
2458	-0.366714415442402\\
2459	-0.369718253022256\\
2460	-0.368067061735928\\
2461	-0.363108932730484\\
2462	-0.35503431604775\\
2464	-0.320147353815628\\
2465	-0.297339281218228\\
2466	-0.264469574539362\\
2467	-0.22989144515941\\
2468	-0.193309858273096\\
2471	-0.0627229032829746\\
2473	0.0317560322650934\\
2476	0.164351721772618\\
2477	0.20519939427686\\
2478	0.242878252634455\\
2479	0.274508179112672\\
2480	0.301874640695587\\
2481	0.327578910617831\\
2482	0.343908463500156\\
2483	0.358281523685946\\
2484	0.362829736503045\\
2485	0.36355025567218\\
2486	0.357895656954952\\
2487	0.344763501030229\\
2488	0.327689838823972\\
2489	0.305128493226221\\
2490	0.2837661250569\\
2491	0.247205998663958\\
2492	0.213616875042135\\
2493	0.172631086123602\\
2494	0.133493729960719\\
2495	0.0868184549922262\\
2496	0.036048681286502\\
2498	-0.0523588575715621\\
2499	-0.0993206093485242\\
2500	-0.138906956881328\\
2501	-0.185677187117108\\
2502	-0.227599684604684\\
2503	-0.258007745352643\\
2504	-0.291641956385774\\
2505	-0.315392656651966\\
2506	-0.333706258290022\\
2507	-0.350583184333573\\
2508	-0.359950054187721\\
2509	-0.364483675603879\\
2510	-0.361863086172889\\
2512	-0.344945674472001\\
2514	-0.291546787605967\\
2515	-0.266725184319967\\
2516	-0.232939316390912\\
2517	-0.190094874878014\\
2518	-0.15366790255348\\
2519	-0.110387584902583\\
2520	-0.0594387736182398\\
2521	-0.0162873559838772\\
2522	0.0346706197042295\\
2523	0.075126292654204\\
2524	0.121492757975375\\
2525	0.166177594737746\\
2526	0.206924984148372\\
2527	0.243513198520759\\
2528	0.275641073699262\\
2529	0.302912270994057\\
2530	0.325336457762205\\
2531	0.345531604744338\\
2532	0.355750632871604\\
2533	0.363953959953506\\
2534	0.365811378139824\\
2535	0.352765171095598\\
2536	0.350293978118316\\
2537	0.327625897668895\\
2538	0.309368884320975\\
2539	0.282263568974486\\
2540	0.243261602006896\\
2542	0.178320117878684\\
2543	0.127577575325631\\
2544	0.0844616350359502\\
2545	0.038017111582576\\
2546	-0.00658188950546901\\
2547	-0.0534501256793192\\
2548	-0.103625425696919\\
2549	-0.146438146692617\\
2551	-0.225682562412658\\
2552	-0.25473349546246\\
2553	-0.290058844545911\\
2554	-0.315814941026019\\
2555	-0.339242033668143\\
2556	-0.355355043449435\\
2558	-0.365541303324335\\
2560	-0.355677012077194\\
2561	-0.337401087694616\\
2562	-0.323452264681691\\
2563	-0.290655348185282\\
2564	-0.262199763844819\\
2565	-0.229486261600869\\
2566	-0.194107819519559\\
2567	-0.151100674377631\\
2568	-0.105849234435027\\
2569	-0.0581967790076305\\
2570	-0.0166459934293925\\
2571	0.0353869823420609\\
2572	0.0849461531993256\\
2573	0.122590333478911\\
2574	0.167697815571046\\
2575	0.20536896668591\\
2576	0.239828083775137\\
2577	0.279689772082293\\
2578	0.306167949921473\\
2579	0.326788020491222\\
2580	0.34606266059609\\
2581	0.357470040181852\\
2582	0.361032107854953\\
2583	0.358117659932759\\
2584	0.358763664161415\\
2585	0.346255158699933\\
2586	0.328738729383076\\
2587	0.305458106145124\\
2589	0.246095037403848\\
2590	0.211250232979182\\
2591	0.168689640702723\\
2592	0.129408932617935\\
2594	0.0372639319821246\\
2595	-0.0124519258843065\\
2596	-0.0575460980603566\\
2597	-0.100587620630449\\
2598	-0.14724612833561\\
2599	-0.186750241931804\\
2600	-0.221273458095766\\
2601	-0.258701718640623\\
2602	-0.288735678199828\\
2604	-0.334851114028424\\
2605	-0.351669367306386\\
2606	-0.357460000173432\\
2607	-0.367095323107606\\
2608	-0.364014132689135\\
2609	-0.349403541022184\\
2610	-0.338474416598729\\
2611	-0.319597609200173\\
2612	-0.293287722498462\\
2613	-0.264805526526288\\
2614	-0.230405794554372\\
2615	-0.192487366876776\\
2616	-0.151569640329399\\
2617	-0.108707975813104\\
2618	-0.0611954793739642\\
2619	-0.0158219281765923\\
2620	0.0353907981025259\\
2621	0.078835255219019\\
2622	0.125478121550259\\
2625	0.245646760425643\\
2626	0.278802090042063\\
2627	0.30375751849806\\
2628	0.330610982823146\\
2629	0.345637704692308\\
2630	0.357502445248429\\
2631	0.367794968800354\\
2632	0.36651294663443\\
2633	0.359300972843812\\
2634	0.346498107923253\\
2635	0.327076132135517\\
2636	0.305295030964317\\
2637	0.281830654837449\\
2638	0.240947762832548\\
2639	0.207673273897854\\
2640	0.168794569215606\\
2641	0.126982576040518\\
2642	0.0779975061391269\\
2643	0.0370984234236857\\
2644	-0.00904975994126289\\
2645	-0.059255023399146\\
2646	-0.103247463067873\\
2647	-0.149186463992464\\
2648	-0.184277150794969\\
2649	-0.225423707902337\\
2650	-0.261435828782851\\
2651	-0.293278006887704\\
2652	-0.316549905979628\\
2654	-0.351368560563515\\
2655	-0.360235264918174\\
2656	-0.36680900654892\\
2657	-0.360916527938116\\
2659	-0.340997663623057\\
2660	-0.315242709692029\\
2661	-0.286862582510366\\
2662	-0.265685663864133\\
2663	-0.232333644283244\\
2666	-0.104465217880715\\
2667	-0.0591561258593174\\
2668	-0.0123645736030085\\
2670	0.0767746054775671\\
2671	0.128771597457217\\
2672	0.168613232465304\\
2673	0.205278674448437\\
2674	0.246879652231655\\
2675	0.274091804679301\\
2676	0.304690123678938\\
2677	0.329567338471406\\
2679	0.356585809484386\\
2680	0.363418335201004\\
2681	0.367772158569096\\
2682	0.358555651301685\\
2683	0.344838550419354\\
2684	0.327704162473765\\
2685	0.304716493620163\\
2686	0.280393215316963\\
2687	0.244412546186595\\
2689	0.16915931661606\\
2691	0.0829929690976314\\
2693	-0.00906645637496695\\
2694	-0.0564414877835588\\
2695	-0.101961950417262\\
2696	-0.142920600822436\\
2697	-0.191104647740303\\
2699	-0.265099599700989\\
2700	-0.289489254897035\\
2701	-0.316941802748261\\
2702	-0.337946410855238\\
2703	-0.355818861895386\\
2705	-0.364452185850951\\
2706	-0.363855692713059\\
2707	-0.354469009501827\\
2708	-0.338867893205588\\
2710	-0.295067528292748\\
2711	-0.261516494568696\\
2713	-0.189118832629902\\
2714	-0.148489552847423\\
2715	-0.102442048996636\\
2716	-0.0551380373872234\\
2717	-0.0121382902775622\\
2718	0.0345024074736102\\
2719	0.0784044943184199\\
2721	0.170374073773473\\
2723	0.247025461139401\\
2724	0.279215263243714\\
2725	0.306410987559957\\
2726	0.330119418986669\\
2727	0.349203286860302\\
2728	0.358859928009679\\
2729	0.364445735727259\\
2730	0.363352969396601\\
2731	0.354635975272686\\
2732	0.344340223945437\\
2733	0.328590965220428\\
2734	0.302462364816165\\
2735	0.278018724226968\\
2736	0.245126967612578\\
2737	0.208096555956217\\
2738	0.168616263130389\\
2739	0.125295791901863\\
2742	-0.0102442805155079\\
2743	-0.0590263442231844\\
2744	-0.110251988962773\\
2746	-0.189132057235383\\
2747	-0.225730417695104\\
2748	-0.259266152400414\\
2749	-0.294664410866972\\
2751	-0.337605079952027\\
2752	-0.353275371824111\\
2753	-0.361562252267049\\
2754	-0.364600187053384\\
2755	-0.361168651177195\\
2756	-0.35193257309993\\
2757	-0.339190920043166\\
2758	-0.319727031047933\\
2759	-0.289717793983982\\
2760	-0.262054438273026\\
2761	-0.223743580382688\\
2762	-0.190576336917729\\
2763	-0.148387759798879\\
2764	-0.100009272401167\\
2765	-0.0544802233880546\\
2766	-0.0134519326875306\\
2768	0.0845191150560822\\
2769	0.129172355341325\\
2770	0.165931742122666\\
2771	0.212281162755062\\
2772	0.24655776386453\\
2773	0.277917911656004\\
2774	0.304700478399809\\
2775	0.325885507526891\\
2776	0.345647226970868\\
2778	0.368012482801078\\
2779	0.363052353060993\\
2780	0.356134391309752\\
2781	0.344445905514476\\
2782	0.325458798662567\\
2783	0.308399478880347\\
2784	0.277134627604028\\
2786	0.209152450292549\\
2788	0.123715023481054\\
2790	0.0317560072680863\\
2791	-0.00986418480897555\\
2792	-0.0581480636287779\\
2793	-0.105073132216148\\
2794	-0.145152023140326\\
2795	-0.190151881395195\\
2797	-0.265892135900231\\
2798	-0.291719254504187\\
2799	-0.32131511892112\\
2800	-0.340344153914884\\
2802	-0.360470641962365\\
2803	-0.365980135570226\\
2804	-0.361683511531282\\
2805	-0.351068109961943\\
2806	-0.336143907748465\\
2807	-0.320025201373028\\
2808	-0.293235179710791\\
2809	-0.264092330955464\\
2810	-0.225600114191366\\
2811	-0.190857641813636\\
2812	-0.143938811747375\\
2813	-0.103829151721584\\
2814	-0.0585683163326394\\
2815	-0.00564448592240296\\
2816	0.0418024636433074\\
2817	0.081135801144228\\
2819	0.174556461644897\\
2820	0.208751620370094\\
2821	0.25012102359824\\
2822	0.280326126258842\\
2823	0.306283432179043\\
2825	0.349818093752219\\
2826	0.359180521378221\\
2828	0.363830085839254\\
2830	0.344877636710862\\
2831	0.330170131195246\\
2832	0.304751234242758\\
2834	0.243504839530942\\
2835	0.204515667799114\\
2836	0.170447333907305\\
2837	0.119095799167098\\
2839	0.0339117159956004\\
2840	-0.016282580382267\\
2842	-0.104864623962385\\
2843	-0.147241272812153\\
2844	-0.193359448105639\\
2845	-0.229814744716805\\
2846	-0.264868641907924\\
2847	-0.294226860908111\\
2848	-0.318336914059273\\
2849	-0.340547231102391\\
2850	-0.351831660231255\\
2851	-0.361222797732808\\
2852	-0.365931650179391\\
2853	-0.3609004630257\\
2854	-0.35003218005113\\
2855	-0.334956378533661\\
2856	-0.315148557024713\\
2857	-0.291024909036423\\
2858	-0.261680941747727\\
2860	-0.187991672788485\\
2861	-0.146084713849632\\
2863	-0.0524582088091847\\
2864	-0.00779175400930399\\
2865	0.0421390591313866\\
2866	0.0802608122971833\\
2867	0.128405381494304\\
2869	0.214361954352626\\
2870	0.24817267594608\\
2872	0.308181891305139\\
2873	0.330955419258316\\
2874	0.347069666605876\\
2875	0.353660071782997\\
2876	0.366074512676278\\
2877	0.355646269477347\\
2878	0.357308804999775\\
2880	0.326132266161949\\
2881	0.306773099610382\\
2882	0.274257060311811\\
2883	0.24692535871327\\
2884	0.207549791654401\\
2885	0.165090107450851\\
2888	0.0324585830899196\\
2889	-0.0203993164395797\\
2890	-0.0601736122998773\\
2891	-0.106369605325199\\
2892	-0.148851821803873\\
2893	-0.187334855852441\\
2894	-0.230666082771222\\
2895	-0.266378353615892\\
2897	-0.317287687673797\\
2898	-0.33944347832221\\
2899	-0.354782541650366\\
2901	-0.367015912100669\\
2902	-0.360850774602113\\
2903	-0.353065097215222\\
2904	-0.338336958048785\\
2906	-0.289759717594279\\
2907	-0.256194425895956\\
2908	-0.226682689530207\\
2909	-0.185111427388165\\
2910	-0.146308685393251\\
2912	-0.0515065393306031\\
2913	-0.00552832946732451\\
2914	0.0340767668126318\\
2915	0.0862808245137785\\
2917	0.172617303155221\\
2918	0.210341132137273\\
2919	0.244052957822532\\
2920	0.281902133664062\\
2921	0.306274892928286\\
2922	0.329102684475856\\
2923	0.347258098177008\\
2924	0.363827463447706\\
2925	0.368724754069717\\
2926	0.360148611935983\\
2927	0.355505567183627\\
2928	0.346375029174851\\
2929	0.321523542299929\\
2930	0.305605955501505\\
2931	0.273766741317104\\
2932	0.244731149377003\\
2933	0.202774151387075\\
2934	0.16247520711795\\
2935	0.126771436567651\\
2938	-0.0185149504113724\\
2939	-0.0601343973025905\\
2940	-0.107258646444734\\
2941	-0.156657103554608\\
2942	-0.194757028124513\\
2943	-0.224088381989986\\
2944	-0.264926313665001\\
2945	-0.299112853748284\\
2946	-0.321630141615969\\
2947	-0.340330229016672\\
2948	-0.351985893335041\\
2949	-0.362230729916973\\
2950	-0.36397128655608\\
2951	-0.360685224091867\\
2952	-0.353220598240114\\
2953	-0.333131161521578\\
2954	-0.31576214038887\\
2955	-0.290032946362771\\
2956	-0.257833338909677\\
2957	-0.223331452341426\\
2958	-0.185666700377169\\
2959	-0.142497484432624\\
2960	-0.0939743326366624\\
2961	-0.051819776726461\\
2962	-0.00792799352439033\\
2963	0.0382409380381432\\
2964	0.0856709587446858\\
2965	0.127275012798236\\
2966	0.177272202343829\\
2967	0.213259034022485\\
2968	0.246318129586598\\
2969	0.281235464913152\\
2970	0.305144583977381\\
2971	0.33202612208288\\
2972	0.347274654504872\\
2973	0.358999814478921\\
2974	0.36498587689448\\
2975	0.365425815292383\\
2976	0.355088665621679\\
2977	0.341175733066848\\
2978	0.32606121426943\\
2980	0.275185815386976\\
2981	0.237100755688516\\
2982	0.206208364506438\\
2983	0.164684570991994\\
2984	0.118478621818213\\
2987	-0.0114621494535641\\
2988	-0.0641043566379267\\
2989	-0.105613549028021\\
2990	-0.154216513089978\\
2992	-0.231834301476738\\
2993	-0.264684633327306\\
2994	-0.296305037752973\\
2995	-0.323208437483117\\
2996	-0.340268771937644\\
2997	-0.352769452171287\\
2998	-0.363093750283497\\
2999	-0.364623122742159\\
3000	-0.359635700229774\\
3001	-0.349638316105029\\
3002	-0.337082318369085\\
3003	-0.31513455510094\\
3005	-0.256852142261323\\
3006	-0.225504348772574\\
3007	-0.180295812488112\\
3008	-0.143357539210228\\
3010	-0.0554164779637176\\
3011	-0.00592661722885168\\
3012	0.0407328849960322\\
3013	0.0861944587604739\\
3014	0.129577343888741\\
3015	0.170176471122431\\
3016	0.220096145540992\\
3017	0.252752433698333\\
3019	0.30830553265605\\
3020	0.331577571832895\\
3021	0.347582232464902\\
3022	0.357421956724011\\
3023	0.365963213981104\\
3024	0.360584541931985\\
3025	0.360779268217811\\
3026	0.339897710744026\\
3027	0.324910746285695\\
3028	0.303996549702333\\
3029	0.273750653867864\\
3030	0.237112784004694\\
3031	0.205498657438511\\
3032	0.163448433575923\\
3033	0.11828598948432\\
3034	0.0760551512539678\\
3035	0.0306278100442796\\
3036	-0.0128985944620581\\
3037	-0.0643102879103026\\
3038	-0.110075783926277\\
3040	-0.198181294045753\\
3041	-0.248827848893143\\
3042	-0.305596362796223\\
3043	-0.346137677634033\\
3044	-0.382999944146377\\
3046	-0.435107634086762\\
3048	-0.435471214669633\\
3049	-0.423262567424899\\
3050	-0.39524578273722\\
3051	-0.35847557295665\\
3052	-0.313061896581985\\
3053	-0.258588157314989\\
3055	-0.115046729202732\\
3056	-0.036485038074261\\
3057	0.0537495417597711\\
3059	0.229162386083317\\
3060	0.324953856053071\\
3061	0.409443732550244\\
3062	0.497379235874178\\
3063	0.581008023761569\\
3064	0.655351068154232\\
3065	0.719455440652837\\
3066	0.76955631316514\\
3067	0.81850448110572\\
3068	0.849456281176572\\
3069	0.865357977931126\\
3070	0.863132182746995\\
3071	-1.08956401204296\\
3072	-1.07620209408196\\
3073	-1.03579270521504\\
3074	-0.976949184572732\\
3075	-0.901070680373323\\
3076	-0.802132764874841\\
3077	-0.692847548219106\\
3078	-0.572646753857953\\
3079	-0.447068610307269\\
3080	-0.314831153888917\\
3081	-0.17549295563731\\
3082	-0.0407653431539075\\
3083	0.0809473688741491\\
3085	0.315296537766244\\
3086	0.407078217258913\\
3087	0.489505877209467\\
3088	0.551495064165465\\
3089	0.590555998430773\\
3090	0.617156381135374\\
3091	0.621637711829862\\
3092	0.61010121529489\\
3093	0.579928181075047\\
3094	0.52771097952791\\
3095	0.457415936118196\\
3096	0.385599697704492\\
3097	0.292153386801601\\
3098	0.191479478963174\\
3099	0.0819719354244626\\
3100	-0.0215677300820971\\
3101	-0.131166021644731\\
3102	-0.218733999047799\\
3103	-0.304424469086825\\
3104	-0.379032724075842\\
3105	-0.445849101748536\\
3107	-0.545345470183747\\
3108	-0.574942391701825\\
3109	-0.593739523673321\\
3110	-0.592916242809224\\
3111	-0.583992071346984\\
3112	-0.558965367172732\\
3113	-0.5188855979045\\
3114	-0.468896015133396\\
3115	-0.406819695410832\\
3116	-0.341847265830893\\
3117	-0.257087738421887\\
3119	-0.081515561570086\\
3120	0.0113975409126397\\
3121	0.105526023030052\\
3122	0.192616227779581\\
3124	0.35039118789382\\
3125	0.421521334306817\\
3126	0.479893940739657\\
3127	0.529170838524806\\
3128	0.567447516691573\\
3129	0.585469596532675\\
3130	0.593991359935444\\
3131	0.587878611229826\\
3132	0.569646326277052\\
3133	0.536675823625046\\
3134	0.487500847902083\\
3135	0.428992599501271\\
3136	0.363428409276821\\
3137	0.285874533420156\\
3138	0.201535203117146\\
3139	0.119146474334229\\
3140	0.0206806017522467\\
3142	-0.161488640440894\\
3143	-0.246192442883512\\
3144	-0.329142716934257\\
3145	-0.39833111458438\\
3146	-0.4595973465116\\
3147	-0.515465705640054\\
3148	-0.555572225296601\\
3149	-0.583249163757046\\
3150	-0.595761979060626\\
3151	-0.597078008913286\\
3152	-0.577876382020804\\
3153	-0.550441863705601\\
3154	-0.503230534765407\\
3155	-0.451242274666129\\
3156	-0.386318709443003\\
3157	-0.308930773557677\\
3158	-0.234182648931437\\
3159	-0.142224104875368\\
3160	-0.054575059787112\\
3162	0.131287943698226\\
3163	0.220558467436149\\
3165	0.378011621688984\\
3166	0.446607288952237\\
3167	0.497393202451349\\
3168	0.541185907215549\\
3169	0.574574095414391\\
3170	0.593423359112421\\
3171	0.59545843531987\\
3172	0.582898029310854\\
3173	0.560365715494299\\
3174	0.522591060831019\\
3175	0.470763722739321\\
3176	0.407598284165488\\
3177	0.33963967658201\\
3178	0.259268615640394\\
3179	0.172803574543195\\
3181	-0.00887373249724988\\
3182	-0.101152354389797\\
3183	-0.19065453329722\\
3184	-0.270138894480624\\
3185	-0.35168468633583\\
3186	-0.424770422264828\\
3187	-0.481716641204912\\
3188	-0.526559023503523\\
3189	-0.565759162201175\\
3190	-0.582801317803842\\
3191	-0.596744210778979\\
3192	-0.59073353831991\\
3193	-0.571422898206492\\
3194	-0.533021867188381\\
3195	-0.490676141333097\\
3196	-0.430520373200125\\
3197	-0.36702248524125\\
3198	-0.287766956112137\\
3199	-0.201350887609806\\
3200	-0.117229764980948\\
3201	-0.0130330057445462\\
3202	0.0882074464698235\\
3203	0.187435660752271\\
3204	0.275161521731661\\
3205	0.357630501875519\\
3206	0.425931367780322\\
3207	0.484264799771154\\
3208	0.530561245346689\\
3209	0.558244792701316\\
3210	0.576757491580338\\
3211	0.573900939778468\\
3212	0.556147419219087\\
3213	0.52315703543627\\
3214	0.471833559161041\\
3215	0.409234450400618\\
3216	0.338682136388343\\
3217	0.257594941260322\\
3218	0.167248296444996\\
3220	-0.0220206524190871\\
3221	-0.120263674081798\\
3222	-0.210077663200536\\
3223	-0.298189056495175\\
3224	-0.372736146018269\\
3225	-0.442577754882223\\
3226	-0.487118157759596\\
3227	-0.53006423843226\\
3228	-0.543958420145373\\
3229	-0.54524963464246\\
3230	-0.528132167397871\\
3231	0.574800192181556\\
3232	0.509738209005263\\
3233	0.44243203359747\\
3234	0.366464200604696\\
3235	0.29231512753222\\
3236	0.215737531877494\\
3237	0.143341950775266\\
3238	0.0694972074170437\\
3239	-0.00281918657992719\\
3240	-0.0678378434527076\\
3241	-0.123811313282204\\
3242	-0.181334333308314\\
3243	-0.224389299407449\\
3244	-0.265505420939007\\
3245	-0.293770603550456\\
3246	-0.309521795918499\\
3247	-0.320063300741367\\
3248	-0.319924825559156\\
3249	-0.315375063061765\\
3250	-0.294603766954424\\
3251	-0.271229868378668\\
3252	-0.239003353305179\\
3253	-0.195807310786677\\
3254	-0.151435746643983\\
3255	-0.110220705463234\\
3256	-0.051582411390882\\
3257	0.000543291646863509\\
3258	0.057396373741085\\
3259	0.111729635359552\\
3260	0.170046550625102\\
3261	0.219466248087429\\
3262	0.253631608760315\\
3263	0.285486260876041\\
3264	0.311772856959124\\
3265	0.334018970969282\\
3266	0.349356815811461\\
3267	0.358928488230504\\
3268	0.365032976495968\\
3269	0.361831168327626\\
3270	0.354357459445964\\
3271	0.341495912975006\\
3272	0.32308931664511\\
3274	0.275843184675523\\
3275	0.234022633893346\\
3277	0.159285742842712\\
3278	0.115361418060274\\
3279	0.0686945797810949\\
3280	0.0248148182272416\\
3281	-0.0209350169475329\\
3282	-0.0715467351546977\\
3283	-0.115078339897536\\
3284	-0.160411096644566\\
3286	-0.235337787415574\\
3288	-0.300867558405116\\
3289	-0.322030807447391\\
3290	-0.341989937447124\\
3292	-0.361938357895269\\
3293	-0.365385472583966\\
3294	-0.36028379222671\\
3295	-0.350630711734539\\
3296	-0.330197941387269\\
3297	-0.313019131540386\\
3298	-0.288794609477009\\
3299	-0.254311612250604\\
3301	-0.177703299486893\\
3302	-0.136750114043934\\
3303	-0.0888248878691229\\
3304	-0.0482696717635918\\
3305	0.000146694391787605\\
3306	0.0442386128938779\\
3308	0.135557794897977\\
3309	0.179854721692664\\
3310	0.222113584947238\\
3311	0.254855847116232\\
3312	0.282031494832154\\
3313	0.310711643920058\\
3314	0.331484146093317\\
3315	0.348167948211994\\
3316	0.361867647334748\\
3317	0.364534659426226\\
3318	0.364334421773037\\
3319	0.355984539332894\\
3320	0.339555292288878\\
3321	0.3244982905303\\
3322	0.295160819878674\\
3323	0.270854223560491\\
3324	0.236874580838503\\
3325	0.200570716431685\\
3326	0.154290887671323\\
3327	0.113742709670078\\
3328	0.0690904838420465\\
3329	0.0230078636195685\\
3330	-0.0219058793845761\\
3331	-0.0686184951546238\\
3333	-0.15821160825999\\
3334	-0.195537081129714\\
3335	-0.237355146209211\\
3336	-0.269909787767574\\
3337	-0.300456420616683\\
3338	-0.323426111041954\\
3340	-0.358019597574184\\
3342	-0.365702113641873\\
3343	-0.357693467877198\\
3344	-0.35338585238469\\
3345	-0.337111932574317\\
3346	-0.30739449798466\\
3347	-0.284585037451961\\
3348	-0.257171510867011\\
3349	-0.22093299779408\\
3351	-0.135098838705744\\
3352	-0.0906809018219974\\
3353	-0.0431747253351205\\
3355	0.0474309831643041\\
3356	0.0957530465548189\\
3357	0.142347172597965\\
3359	0.218369487026393\\
3360	0.256725292415013\\
3361	0.284564877886169\\
3363	0.333502325002428\\
3364	0.35027524129373\\
3365	0.360263325866072\\
3366	0.361784287071259\\
3367	0.361271302914702\\
3368	0.35350051739124\\
3369	0.343662112392394\\
3371	0.299595146906086\\
3372	0.270225071575624\\
3373	0.234812197782958\\
3374	0.193249706397182\\
3375	0.157755181438006\\
3376	0.115189698829909\\
3377	0.0666388180979993\\
3378	0.0219287055192581\\
3379	-0.0257005077419308\\
3380	-0.0672014580268296\\
3381	-0.11619022070272\\
3383	-0.202310103616128\\
3384	-0.238964544094415\\
3385	-0.273739892307276\\
3387	-0.322362805426565\\
3388	-0.344331983239499\\
3389	-0.349092559712972\\
3390	-0.360693320269547\\
3391	-0.363124436422368\\
3392	-0.355049688929284\\
3393	-0.354189930436405\\
3395	-0.310629663799773\\
3396	-0.283249434054596\\
3397	-0.25443023907701\\
3398	-0.215183055301623\\
3399	-0.182030267141272\\
3400	-0.134674605955297\\
3402	-0.0447091882124369\\
3403	0.00277482063665957\\
3404	0.0468614408250687\\
3405	0.0956327362173397\\
3406	0.137793024322036\\
3407	0.176855129679552\\
3408	0.218859476608031\\
3409	0.253546148788701\\
3411	0.313677382880087\\
3412	0.333956900372868\\
3413	0.349143229909714\\
3415	0.367806040429969\\
3416	0.364957409733506\\
3417	0.354181310892272\\
3418	0.341235825951117\\
3419	0.325275766242157\\
3420	0.301382744611601\\
3421	0.27047656012428\\
3422	0.233495102350844\\
3424	0.157435695738513\\
3425	0.110431055968547\\
3427	0.0198973549663606\\
3428	-0.0238697644167587\\
3429	-0.071970871785652\\
3430	-0.116827676463345\\
3431	-0.158010788123192\\
3432	-0.200548428329057\\
3433	-0.240837305801506\\
3435	-0.299372578350358\\
3436	-0.324574396332991\\
3437	-0.34388260293008\\
3438	-0.357052286368798\\
3439	-0.362978540360018\\
3440	-0.365878474743113\\
3441	-0.360366963732304\\
3442	-0.351852260560463\\
3443	-0.33324453123987\\
3444	-0.311722206344257\\
3445	-0.281858775541423\\
3446	-0.248299655665505\\
3447	-0.213476095523674\\
3448	-0.174092607152488\\
3449	-0.132750670337828\\
3450	-0.0886797523394307\\
3451	-0.0423491602800823\\
3452	0.000924175244108483\\
3453	0.0509842594074144\\
3454	0.092230968880358\\
3455	0.140509070722146\\
3457	0.217769396146196\\
3458	0.254230545170685\\
3460	0.312748940566053\\
3461	0.336428244601393\\
3463	0.360784927395798\\
3464	0.364152808172094\\
3465	0.364600706529927\\
3466	0.353684825326127\\
3467	0.341003676605396\\
3468	0.326595786545113\\
3469	0.299090432794401\\
3470	0.264994804416347\\
3471	0.235794738264303\\
3472	0.193563272451229\\
3473	0.154775546342989\\
3475	0.0679896728706808\\
3476	0.0187467207051668\\
3477	-0.0259274655245463\\
3478	-0.0694501823854807\\
3479	-0.122763419192779\\
3480	-0.164047250552358\\
3481	-0.199428487698242\\
3482	-0.239893355491859\\
3483	-0.272489381296964\\
3484	-0.299123158849397\\
3485	-0.319974224638827\\
3486	-0.346313230817941\\
3487	-0.358210952709669\\
3488	-0.363297737345874\\
3489	-0.364100944486836\\
3490	-0.358835014725628\\
3491	-0.34890627807772\\
3492	-0.328923885993845\\
3493	-0.311659916057579\\
3494	-0.282055969374142\\
3496	-0.214051549170563\\
3497	-0.17685430938036\\
3500	-0.0441478972334153\\
3501	0.00690152931474586\\
3502	0.0500244594145443\\
3504	0.138703203320802\\
3505	0.182538460430806\\
3508	0.287832470854937\\
3510	0.33396966363398\\
3511	0.352269085549779\\
3512	0.362455485016199\\
3513	0.367795669837051\\
3514	0.360025108871923\\
3515	0.356169904871422\\
3516	0.341115859101592\\
3517	0.323823996962346\\
3518	0.296875002741217\\
3519	0.268641572682554\\
3520	0.235450532165032\\
3522	0.153160146867322\\
3523	0.107670108544426\\
3524	0.0651574373432595\\
3525	0.0197403023771585\\
3526	-0.0294679675871521\\
3527	-0.0739058530652983\\
3528	-0.119877976237149\\
3529	-0.163163469463143\\
3531	-0.237872091120607\\
3532	-0.270346218251689\\
3533	-0.303970987935372\\
3534	-0.325545064417838\\
3536	-0.359214282091216\\
3537	-0.358840122060883\\
3538	-0.363502979035275\\
3539	-0.360140248952575\\
3541	-0.330346232816282\\
3542	-0.308813510202071\\
3543	-0.283387345971278\\
3544	-0.254007306004041\\
3545	-0.216237114409068\\
3546	-0.176037871699918\\
3548	-0.0863859720147957\\
3549	-0.0435634410782768\\
3550	0.00666257583179686\\
3551	0.052866929065658\\
3552	0.0968723070081978\\
3553	0.139215209198028\\
3554	0.185203094167264\\
3555	0.22173877146588\\
3557	0.288728686257855\\
3558	0.315572019903811\\
3559	0.332220624039564\\
3560	0.355160887163038\\
3561	0.355953208862957\\
3563	0.365129341868396\\
3564	0.355092561719175\\
3565	0.339484002827703\\
3566	0.320915511095791\\
3567	0.29778213796726\\
3568	0.264529482722992\\
3569	0.234513795467137\\
3570	0.196198210922375\\
3572	0.110481336518205\\
3574	0.016383053620757\\
3575	-0.0292542839797534\\
3576	-0.0723090202013736\\
3577	-0.120131840107206\\
3578	-0.163657142182728\\
3579	-0.205238434492912\\
3580	-0.242613740390425\\
3581	-0.273161582829289\\
3582	-0.29947806731343\\
3583	-0.322685215181991\\
3584	-0.342464330562052\\
3585	-0.357046221461587\\
3586	-0.363963578944549\\
3587	-0.363103394558948\\
3588	-0.35662635637982\\
3589	-0.345467069369079\\
3590	-0.335565237588526\\
3591	-0.308759052368714\\
3592	-0.283434394379128\\
3593	-0.250380412972845\\
3594	-0.209458814799291\\
3595	-0.175552325840727\\
3596	-0.135172993937431\\
3597	-0.0864793395176093\\
3599	0.00480577053349407\\
3600	0.0542704602075901\\
3601	0.0992462027065812\\
3602	0.142862636702375\\
3604	0.22317887898862\\
3605	0.256848707414974\\
3606	0.288715114580555\\
3607	0.31565400411364\\
3608	0.336037777442925\\
3609	0.349135865074913\\
3610	0.359119064257357\\
3612	0.364008940635813\\
3613	0.353455579986075\\
3614	0.33795401712905\\
3615	0.321246264410547\\
3617	0.266061960658135\\
3618	0.234280594693701\\
3619	0.195032766195709\\
3621	0.107966386696262\\
3622	0.0671540478438146\\
3623	0.0159972539058799\\
3625	-0.0758823328310427\\
3627	-0.164052947948676\\
3628	-0.204022901017197\\
3629	-0.241075030317006\\
3630	-0.268732269370958\\
3631	-0.301253540741072\\
3632	-0.328659849824362\\
3633	-0.347546525276812\\
3634	-0.360879774818386\\
3635	-0.359759926579954\\
3636	-0.366330026591186\\
3638	-0.346752882899182\\
3639	-0.334385436337016\\
3640	-0.304086164251203\\
3641	-0.279922543026714\\
3642	-0.244983193781536\\
3643	-0.2121371773801\\
3645	-0.133053983661739\\
3647	-0.0398175008267572\\
3648	0.00311138117240262\\
3649	0.0530079674167609\\
3650	0.0991085403520628\\
3651	0.141269374008516\\
3652	0.185496803417209\\
3653	0.220508790070653\\
3654	0.256952643897876\\
3655	0.289218829703259\\
3656	0.313636052085712\\
3657	0.336716586186867\\
3658	0.350286691021211\\
3660	0.366354805798892\\
3661	0.363203100524515\\
3662	0.354840871864781\\
3663	0.341035901773921\\
3664	0.322097962122825\\
3665	0.297119377898071\\
3666	0.265156960812874\\
3667	0.229977354737912\\
3669	0.154667722672002\\
3670	0.105633191321886\\
3671	0.0616519584873458\\
3672	0.0188210862947926\\
3673	-0.0283557782258868\\
3674	-0.0775842374678177\\
3675	-0.121887369164142\\
3676	-0.164704362719476\\
3677	-0.205209453314637\\
3678	-0.246908514881397\\
3679	-0.275246942961985\\
3680	-0.301979476033921\\
3681	-0.320600580657356\\
3682	-0.347362545115629\\
3683	-0.357854126611528\\
3684	-0.363790179919761\\
3685	-0.365579387992511\\
3686	-0.359534260564487\\
3687	-0.351013157604029\\
3689	-0.308451549748042\\
3690	-0.278617081120501\\
3691	-0.247629304722977\\
3692	-0.209229281911576\\
3693	-0.168365417564473\\
3694	-0.129315869839047\\
3695	-0.0878482972748316\\
3697	0.0106336909084348\\
3698	0.0538079266034401\\
3699	0.100493853707576\\
3700	0.141409928147823\\
3701	0.188596213130495\\
3702	0.224711171711988\\
3703	0.257738837592569\\
3705	0.313598373494642\\
3706	0.333541095531928\\
3707	0.349895353529973\\
3708	0.361180508381494\\
3709	0.365778300548754\\
3710	0.361799236900879\\
3711	0.353174550135464\\
3713	0.32484909303048\\
3716	0.233356721030304\\
3717	0.192057484071938\\
3718	0.149194550825541\\
3719	0.109161864770158\\
3720	0.0608437598525597\\
3722	-0.0295543950146566\\
3723	-0.0759343377276309\\
3724	-0.126288913263579\\
3725	-0.170866026209296\\
3726	-0.207336585565827\\
3727	-0.238910403911177\\
3728	-0.274893901832456\\
3729	-0.304046752836712\\
3730	-0.327359305365462\\
3731	-0.342622821815894\\
3732	-0.35650562355795\\
3734	-0.368724589915928\\
3735	-0.3588141175278\\
3736	-0.346937107517533\\
3737	-0.33173743842417\\
3738	-0.307276957824797\\
3739	-0.280949410406265\\
3740	-0.245519962100389\\
3741	-0.212482642486975\\
3742	-0.168573663799634\\
3743	-0.129037649286602\\
3744	-0.0823175140749299\\
3745	-0.0331576695261901\\
3746	0.00859207111443538\\
3747	0.0582985530727456\\
3748	0.098207565044504\\
3749	0.14869333164097\\
3750	0.183516964271348\\
3751	0.22459394779753\\
3752	0.258859207536261\\
3753	0.287552969849003\\
3755	0.337998555056402\\
3756	0.348585289198127\\
3758	0.366886422127209\\
3759	0.361796186383799\\
3760	0.35409016419635\\
3762	0.318493814533667\\
3763	0.294414812998639\\
3765	0.233443946528041\\
3766	0.189949408981192\\
3767	0.149747847817707\\
3768	0.108222008876965\\
3769	0.0629820854878744\\
3770	0.0100023870922996\\
3771	-0.0331606592380922\\
3772	-0.0799624377041255\\
3773	-0.123353386986309\\
3774	-0.165030257897342\\
3775	-0.205273869280063\\
3777	-0.27478103018575\\
3778	-0.303271783144282\\
3779	-0.325012581975443\\
3780	-0.344042720185826\\
3781	-0.360065880170623\\
3782	-0.363163032025568\\
3783	-0.367437251699812\\
3784	-0.355545821417763\\
3785	-0.351180598089286\\
3786	-0.330501053388616\\
3787	-0.306360531968039\\
3788	-0.278883921263059\\
3789	-0.242294140783997\\
3790	-0.210057779193903\\
3791	-0.171753312991768\\
3792	-0.126425393787031\\
3793	-0.0856956321799771\\
3794	-0.0356790669807197\\
3796	0.0554448694456369\\
3797	0.100516411724584\\
3798	0.143179992696787\\
3799	0.183130910188083\\
3800	0.22484376130069\\
3801	0.262355645826574\\
3802	0.289459706539219\\
3803	0.320313984766017\\
3804	0.335267776943056\\
3805	0.356955016999109\\
3806	0.359756225617275\\
3807	0.364746926441512\\
3808	0.364958841917542\\
3809	0.355785230529364\\
3810	0.336811721079357\\
3811	0.320087539716496\\
3812	0.290647827388511\\
3813	0.259179767384921\\
3814	0.230067285036057\\
3815	0.18653072523739\\
3816	0.144352973597506\\
3817	0.105214049949154\\
3818	0.0628905106905222\\
3819	0.0159606929655638\\
3820	-0.036883036774725\\
3823	-0.165217349608156\\
3824	-0.203950283941595\\
3825	-0.240618906153031\\
3827	-0.308037272082402\\
3829	-0.348699227532052\\
3830	-0.35487876839079\\
3831	-0.359884787333158\\
3832	-0.368407622959239\\
3833	-0.356242244275109\\
3834	-0.347256085139179\\
3835	-0.330203031322526\\
3836	-0.308402942586781\\
3837	-0.275790759913889\\
3838	-0.245389782719485\\
3839	-0.210101785459301\\
3840	-0.169069588098864\\
3841	-0.12504067261716\\
3842	-0.0845287414954328\\
3843	-0.038253681574588\\
3844	0.0117950670537539\\
3845	0.06030309060543\\
3846	0.105158177724206\\
3847	0.146005671163948\\
3848	0.184745467780431\\
3849	0.220859317289523\\
3850	0.258674778640852\\
3851	0.293644876147482\\
3853	0.337719674978416\\
3854	0.352349526098806\\
3855	0.355946929650599\\
3856	0.366764043265448\\
3857	0.365561580334543\\
3859	0.339914863467584\\
3860	0.315827341293698\\
3861	0.293442680817861\\
3862	0.266607821487469\\
3863	0.230350896037635\\
3865	0.145518864007045\\
3866	0.104166621507375\\
3868	0.011966441536515\\
3869	-0.0317016872859313\\
3870	-0.0801568922415754\\
3871	-0.121728545756923\\
3872	-0.168284859895721\\
3873	-0.208403226812152\\
3874	-0.242840172641081\\
3875	-0.276124894583518\\
3876	-0.305519111840113\\
3877	-0.331535981459638\\
3878	-0.34799971828761\\
3879	-0.355943745084005\\
3880	-0.362029272287145\\
3881	-0.366478203984116\\
3882	-0.357012706843307\\
3884	-0.33048500303039\\
3885	-0.307205054697988\\
3886	-0.274307498586495\\
3887	-0.248406762168088\\
3889	-0.165886781480367\\
3890	-0.129364699649614\\
3891	-0.0808756705973792\\
3892	-0.037167764095102\\
3893	0.0122535972041078\\
3894	0.0575644607074537\\
3895	0.100377898480929\\
3896	0.149239867978849\\
3897	0.190559582064907\\
3898	0.226103197765497\\
3899	0.263234169230145\\
3901	0.314921523716748\\
3902	0.335256735602798\\
3903	0.352257423793617\\
3905	0.368539379673621\\
3907	0.353412043631579\\
3908	0.338692065216037\\
3909	0.31791005829291\\
3910	0.291924937238491\\
3911	0.263934316348696\\
3912	0.223927967156214\\
3913	0.191814726324537\\
3914	0.148165975914708\\
3915	0.100177122807509\\
3916	0.0596842789905168\\
3918	-0.0321685487997456\\
3919	-0.0830101463648134\\
3920	-0.123170330381981\\
3921	-0.168886191855108\\
3922	-0.212626858617568\\
3923	-0.247468272846163\\
3924	-0.276103865339337\\
3925	-0.301999905225784\\
3926	-0.331424092393718\\
3927	-0.34645226122575\\
3928	-0.359412560485453\\
3929	-0.359576176120299\\
3930	-0.365663379455327\\
3931	-0.354590194946013\\
3932	-0.346962254243408\\
3933	-0.326140249370383\\
3934	-0.307773000181896\\
3935	-0.273378612914257\\
3936	-0.249033063848401\\
3937	-0.206804117049614\\
3939	-0.126538956066724\\
3940	-0.0820312029359229\\
3941	-0.0323725896696487\\
3942	0.0106388321146369\\
3943	0.0594366455557065\\
3944	0.0987090099133638\\
3946	0.190084668531654\\
3947	0.230913506166416\\
3948	0.25924798721644\\
3949	0.294570244765509\\
3950	0.319523922255485\\
3951	0.337177157623046\\
3952	0.353018961995531\\
3953	0.366808369498358\\
3954	0.361639105765789\\
3955	0.36266520315985\\
3956	0.352930050313716\\
3957	0.338338075556749\\
3958	0.317216172163626\\
3959	0.292248993470366\\
3960	0.260596416465887\\
3961	0.222707662602261\\
3962	0.186708469817404\\
3963	0.147056914908717\\
3966	0.0154644961285157\\
3967	-0.039834910292484\\
3968	-0.0807300759529426\\
3969	-0.126730048111312\\
3970	-0.169744171609182\\
3971	-0.211106423462297\\
3972	-0.246323651945659\\
3973	-0.276905165691915\\
3975	-0.332472985848199\\
3976	-0.34420722153618\\
3977	-0.357347065365957\\
3978	-0.359501025492591\\
3979	-0.365983933266762\\
3980	-0.358915064082339\\
3981	-0.349205872501443\\
3982	-0.327514792322745\\
3984	-0.277460025934488\\
3985	-0.244858749621926\\
3986	-0.207697811393246\\
3987	-0.167337892362411\\
3988	-0.121926382013953\\
3989	-0.0793420257773505\\
3990	-0.0343232768464077\\
3991	0.0148728871445201\\
3992	0.056916419375284\\
3993	0.106829468335491\\
3994	0.148823904920846\\
3995	0.195377551682668\\
3996	0.227344992071266\\
3997	0.264146558783068\\
3998	0.293617270978757\\
3999	0.316217569240962\\
4000	0.340636992789769\\
};
\addlegendentry{0}

\end{axis}
\end{tikzpicture}%

\subsection{Détection d'énergie}


Avec ces signaux ainsi filtrés, nous désirons reconstituer le signal de base. Pour cela nous allons utiliser un détecteur d'énergie.
Nous divisons nos signaux en périodes $T_s$ et sur chaque période nous calculons l'énergie suivant la formule suivante:
\[
E=\sum_{i=1}^{N_s} x_n^2
\]
Enfin, on compare cette énergie à un seuil $K$ qu'on fixera à la moyenne des énergies du signal.
Pour le signal en sortie du passe bas par exemple, si $E>K$ alors le signal reconstitué sera égal à 1 sur cette période $T_s$, sinon il sera égal à 0.

Voici les figures obtenues grâce à cette méthode:

\begin{figure}
	\centering
	\input{figures/bits-reconstruits-filtrage.tex}
\end{figure}



\subsection{Modification du démodulateur} 

\subsubsection{Modification du nombre de coefficients}

Avec 201 coefficients pour le filtre, le taux d'erreur est grand.
Voici le signal démodulé qu'on obtient:

\begin{figure}[H]
	\centering
	\input{figures/bits-reconstruits-filtrage-201-coefs.tex}
	\caption{Avec 201 coefficients}
	\label{fig:201-coefficients}
\end{figure}

\subsubsection{Modification des fréquences}

Nous travaillons jusqu'alors avec des fréquences très éloignées: $F_1=2kHz$ et $F_2=6kHz$. Cependant, la norme V21 impose des fréquences proches: $F_1=980Hz$ et $F_2=1180kHz$. On observe en utilisant ces fréquences une augmentation du taux d'erreur, il devient difficile de démoduler en séparant le signal par filtrage haut et bas.
Voici le signal démodulé obtenu:

\begin{figure}[H]
	\centering
	\input{figures/bits-reconstruits-filtrage-v21.tex}
	\caption{Avec les fréquences de la norme V21}
	\label{fig:filtrage-avec-v21}
\end{figure}

\section{Démodulateur de fréquence adapté à la norme V21}

Nous avons vudans la partie précédente que la méthode par filtrage n'était pas optimale. Nous allons ici introduire une nouvelle méthode de démodulation plus adaptée à la norme V21.

\subsection{Contexte de synchronisation idéale}
\label{synchro-ideale}


La multiplication du signal avec le cosinus correspondant (par exemple) au bit 1 donne, en fonction du temps, une mesure de la synchronisation entre le signal et ce cosinus: plus le résultat est proche de 1, plus les signaux sont synchronisés à cet instant, et donc plus le signal est susceptible d'être un bit 1.
On a:

\begin{align*}
	\int_{0}^{T_s} \cos^2(2 \pi F_0 t + \phi_0) \, \mathrm{d}t  &= \frac{1} {8 \pi F_0} \sin(2(2 \pi F_0 T + phi_0))+4 \pi F_0 T - \sin(2 \phi_0) \\
	\int_{0}^{T_s} \cos^2(2 \pi F_0 t + \phi_1) \, \mathrm{d}t  &= \frac{1} {8 \pi F_0} \sin(2(2 \pi F_0 T + phi_1))+4 \pi F_0 T - \sin(2 \phi_1) \\
	\int_{0}^{T_s} \cos(2 \pi F_0 t + \phi_0) \cos(2 \pi F_0 t + \phi_1)\, \mathrm{d}t  &= \frac{1} {8 \pi F_0} \sin(4 \pi F_0 T + phi_0 + phi_1)\\
											    & + 4 \pi F_0 T \cos(phi_0 - phi_1) - \sin (\phi_0 + \phi_1) 
\end{align*}


\begin{center}
\begin{figure}[H]
	\centering
	\begin{tikzpicture}
        \begin{axis}[
            ymax=1.1, ymin=-1.1,
            ylabel={Amplitude},
            xlabel={t [s]},
            samples=2000,
        ]
        \addplot[domain=0:100, red] {cos(2*pi*x)};
        \addplot[domain=0:100, blue] {cos(4*pi*x)};
        \addplot[domain=0:100, black, ultra thick] {cos(2*pi*x)*cos(4*pi*x)};
        \legend{cos(2t), cos(4t), produit}
        \end{axis}
        \node[] at (axis cs:55,1) {synchronises};
        \node[] at (axis cs:29,1) {desynchronises};
    \end{tikzpicture}
	\caption{Synchronisation entre deux cosinus de fréquences différentes}
	\label{fig:synchronisation}
\end{figure}
\end{center}

On fait ensuite une accumulation (une moyenne en quelque sorte) de ces mesures sur une période $T_s$ en intégrant sur $T_s$, pour avoir une idée de la synchronisation avec le signal d'un bit 1 et d'un bit 0 à chaque instant échantilloné.

% TODO montrer l'intégrale pas juste le produit

\begin{center}
\begin{figure}[H]
	\centering
	\begin{tikzpicture}
        \begin{axis}[
            ymax=1.1, ymin=-1.1,
            ylabel={Amplitude},
            xlabel={t [s]},
            samples=1000,
        ]
        \addplot[domain=0:200, red, thick] {cos(2*pi*x) * (x<30*pi) + cos(4*pi*x) * (x>=30*pi)};
        \addplot[domain=0:200, blue, thick] {cos(4*pi*x)};
        \addplot[domain=0:200, black, ultra thick] {(cos(2*pi*x) * (x<30*pi) + cos(4*pi*x) * (x>=30*pi)) * cos(4*pi*x)};
        \legend{signal, cos(4t), produit}
        \end{axis}
    \end{tikzpicture}
	\caption{Synchronisation entre le signal obtenu et le signal théorique d'un bit}
\end{figure}
\end{center}

Finalement, pour chaque échantillon temporel (de durée $T_s$), on effectue une comparaison:

$$
\operatorname{reconstitué}(kT_s) = \begin{cases}
    1 & \text{si } \operatorname{sync}_1(kT_s) - \operatorname{sync}_0(kT_s) > 0 \\
    0 & \text{sinon}
\end{cases}
$$

En notant $\operatorname{sync}_b$ le produit pour le bit $b$.

Le signal démodulé avec cette méthode est le suivant:


\begin{figure}
\end{figure}

\begin{figure}[H]
	\centering
\input{figures/bits-reconstruits-fsk-synchro-parfaite.tex}
	\caption{Signal démodulé en supposant une synchronisation parfaite}
	\label{fig:synchro-parfaite}
\end{figure}



\subsection{Gestion d'une erreur de synchronisation de phase porteuse}

Un sinus est déphasé d'un quart de phase, comparé à un cosinus de même fréquence et même déphasage.

En rajoutant ces mesures de désynchronisation, on prend en compte les signaux d'entrée qui seraient déphasés: si le signal n'est pas synchronisé avec le cosinus à cause d'un léger déphasage, l'ajout d'une mesure de synchronisation avec ce même cosinus, mais déphasé de $\frac{\pi}{2}$ compensera la faible valeur de synchronisation.

\begin{figure}[H]
	\centering
	\begin{tikzpicture}
		\begin{axis}[ymin=-1.1, ymax=1.1, xlabel={Temps [s]}, ylabel={Amplitude}, samples=1000]
		\addplot[domain=0:100, blue, thick] {cos(2*pi*x)};
		\addplot[domain=0:100, blue, thick, dashed] {sin(2*pi*x)};
		\addplot[domain=0:100, red, thick] {cos(2*pi*x - 20*pi)};
		\legend{cosinus, sinus, signal};
	\end{axis}	
	\end{tikzpicture}
	\caption{Compensation d'un déphasage du signal d'entrée}
	\label{fig:compensation-dephasage}
\end{figure}

Le reste du processus reste le même qu'en \ref{synchro-ideale}

\begin{figure}
% This file was created by matlab2tikz.
%
%The latest updates can be retrieved from
%  http://www.mathworks.com/matlabcentral/fileexchange/22022-matlab2tikz-matlab2tikz
%where you can also make suggestions and rate matlab2tikz.
%
\definecolor{mycolor1}{rgb}{0.00000,0.44700,0.74100}%
\definecolor{mycolor2}{rgb}{0.85000,0.32500,0.09800}%
%
\begin{tikzpicture}

\begin{axis}[%
width=4.521in,
height=3.406in,
at={(0.758in,0.488in)},
scale only axis,
xmin=0,
xmax=0.09,
xlabel style={font=\color{white!15!black}},
xlabel={temps [s]},
ymin=-0.1,
ymax=1.1,
ylabel style={font=\color{white!15!black}},
ylabel={bit},
axis background/.style={fill=white},
title style={font=\bfseries, align=center},
title={Signal reconstruit par démodulation FSK\\[1ex]taux d'erreur: 0.00\%, SNR: 50},
axis x line*=bottom,
axis y line*=left,
legend style={legend cell align=left, align=left, draw=white!15!black}
]
\addplot [color=mycolor1, line width=2.0pt]
  table[row sep=crcr]{%
0	1\\
0.00331250000000005	1\\
0.00335416666666677	0\\
0.00997916666666665	0\\
0.0100208333333334	1\\
0.0133125000000001	1\\
0.0133541666666668	0\\
0.0233125000000001	0\\
0.0233541666666666	1\\
0.0399791666666667	1\\
0.0400208333333334	0\\
0.0466458333333333	0\\
0.0466875	1\\
0.0533125000000001	1\\
0.0533541666666666	0\\
0.0566458333333333	0\\
0.0566875	1\\
0.0699791666666667	1\\
0.0700208333333334	0\\
0.0799791666666667	0\\
0.0800208333333334	1\\
0.0833124999999999	1\\
};
\addlegendentry{Original}

\addplot [color=mycolor2, dashed, line width=4.0pt]
  table[row sep=crcr]{%
0	1\\
0.00331250000000005	1\\
0.00335416666666677	0\\
0.00997916666666665	0\\
0.0100208333333334	1\\
0.0133125000000001	1\\
0.0133541666666668	0\\
0.0233125000000001	0\\
0.0233541666666666	1\\
0.0399791666666667	1\\
0.0400208333333334	0\\
0.0466458333333333	0\\
0.0466875	1\\
0.0533125000000001	1\\
0.0533541666666666	0\\
0.0566458333333333	0\\
0.0566875	1\\
0.0699791666666667	1\\
0.0700208333333334	0\\
0.0799791666666667	0\\
0.0800208333333334	1\\
0.0833124999999999	1\\
};
\addlegendentry{Reconstitué}

\end{axis}
\end{tikzpicture}%
\end{figure}

% ne pas écrire en dessous

\begin{center}
    \includegraphics[width=0.125\textwidth]{frog.jpg}
\end{center}
\end{document}
