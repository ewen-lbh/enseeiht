\documentclass{article}
\usepackage{cancel}
\usepackage{amsmath, amssymb, amsfonts}
\usepackage[binary-units]{siunitx}
\usepackage{tikz}
\usepackage{float}
\usepackage{pgffor}
\usepackage{import}
\usepackage{vwcol}
\usepackage{fontawesome}
\usepackage{stmaryrd}
\usepackage{multicol}
\usepackage{pdfpages}
\usepackage{transparent}
\usepackage{xcolor}
\usepackage{scalerel}
\usepackage{stackengine}
\usepackage{algpseudocode}
\newcommand{\diag}{\operatorname{diag}}
\newcommand{\card}{\operatorname{card}}
\newcommand{\tr}{\operatorname{tr}}
\newcommand{\rg}{\operatorname{rg}}
\renewcommand{\epsilon}{\varepsilon}
\newcommand{\equivalent}[1]{\underset{#1}{\sim}}
\newcommand{\R}{\mathbb{R}}
\newcommand{\Q}{\mathbb{Q}}
\newcommand{\C}{\mathbb{C}}
\newcommand{\N}{\mathbb{N}}
\newcommand{\Z}{\mathbb{Z}}
\newcommand{\cM}{\mathcal{M}}
\newcommand{\cO}{\mathcal{O}}
\newcommand{\dx}{\mathrm{d}x}
\newcommand{\dy}{\mathrm{d}y}
\newcommand{\dz}{\mathrm{d}z}
\newcommand{\dt}{\mathrm{d}t}
\newcommand{\df}{\mathrm{d}f}
\newcommand{\Sp}{\operatorname{Sp}}
\newcommand{\dangersign}[1][2ex]{%
  \renewcommand\stacktype{L}%
  \scaleto{\stackon[1.3pt]{\color{red}$\triangle$}{\tiny !}}{#1}%
}

\usepackage[a4paper,top=4cm,bottom=4cm,left=3cm,right=3cm,marginparwidth=1.75cm]{geometry}
% \newcommand{\incfig}[2][1]{%
%     \def\svgwidth{#1\columnwidth}
%     \import{./figures/}{#2.pdf_tex}
% }
% 
\newenvironment{theorem}[1][\unskip]{
	\paragraph{Théorème #1}

}{}

\newenvironment{proof}[1][\unskip]{
	\def\temp{#1}\ifx\temp\empty
		\paragraph{Preuve}
	\else
		\paragraph{Preuve \emph{(#1)}}
	\fi

}{}

\newenvironment{definition}[1][\unskip]{
	\paragraph{Définition: #1}

}{}

\newenvironment{warning}[1][\unskip]
{
	\vspace{1cm}
	\begin{minipage}[c]{0.1\linewidth}
	\dangersign[8ex] 
\end{minipage}%
\begin{minipage}[l]{0.9\linewidth}
}
{
	\end{minipage}
	\vspace{1cm}
}

% \pdfsuppresswarningpagegroup=1

\begin{document}

\section{Décompositions LU et Schur par blocs}

\subsection{Exercice 1}

Soit $A$ une matrice non singulière d'ordre $n$ telle que il existe $P$ une matrice de permutation telle que $PA$ peut être factorisé sans pivotation.
Considérons la forme en blocs

\begin{align*}
    PA &= \begin{pmatrix} A_{11} & A_{12} \\ A_{21} & A_{22} \end{pmatrix}  \\
\end{align*}

On définit la \emph{matrice de Schur} $S = A_{21} A_{11}^{-1} A_{12}$

\begin{enumerate}
    \item Expliquez comment adapter l'algorithme de factorisation LU pour obtenir la décomposition de $PA$. On a $PA = \begin{pmatrix} L_{11} & 0 \\ L_{21} & I \end{pmatrix} \begin{pmatrix} U_{11} & U_{12} \\ 0 & S \end{pmatrix} = \begin{pmatrix} L_{11} & 0 \\ L_{21} & I \end{pmatrix} \begin{pmatrix} I & 0 \\ 0 & S \end{pmatrix} \begin{pmatrix} U_{11} & U_{12} \\ 0 & I \end{pmatrix} $
    \item Montrez que $\det A = \det P \det A_{11} \det S$
    \item On suppose savoir comment calculer $Y$ tel que $Y = S^{-1} Z$. Décrivez comment utiliser la factorisation par blocs incomplète précédente pour résoudre $AX=B$.
\end{enumerate}

\paragraph{Correction}

\begin{enumerate}
    \item On exécute l'algorithme de LU usr $PA$ et on l'arrête après $q$ étapes, $q$ étant la taille de $A_{11}$.

        On a 
        \begin{align*}
            L^{(1)}\cdots L^{(q)} PA &= \begin{pmatrix} U_{11} & | & U_{12} \\ \hline 0 & | & C \end{pmatrix} \quad&\text{avec $U_{11}$ triangulaire supérieure.} \\
            \implies PA &= \underbrace{(L^{(1)})^{-1} \cdots (L^{(q)})^{-1}}_{\begin{pmatrix} \cT_{n, s} & 0 \\ \cM_{n} & I \end{pmatrix} } \\
                        &= \begin{pmatrix} L_{11} & \\ L_{12} & I \end{pmatrix} \begin{pmatrix} U_{11} & U_{12} \\ & C \end{pmatrix}  \\
        \end{align*}

        Vérifions que $C=S$.


        On a donc, par blocs:

        \begin{itemize}
            \item $A_{11} = L_{11} U_{11}$ 
            \item $A_{12}= L_{11} U_{12}$ i.e. $U_{12} = L_{11}^{-1} A_{12}$
            \item $A_{21} = L_{21} U_{11}$ i.e. $L_{21} = A_{21} U_{11}^{-1}$
            \item $A_{22} = L_{21} U_{12} + C = A_{21} \underbrace{U_{11}^{-1} L_{11}^{-1}}_{A_{11}^{-1}} A_{12} + C$ i.e. $C = A_{12} - A_{21} A_{11}^{-1} A_{12}$
        \end{itemize}
    \item \begin{align*}
        \det(PA) &= \det A \det P \\
    \end{align*}

    Or $P$ est une permutation donc $P^\top = P^{-1}$.
    Ainsi 
    \begin{align*}
        1 = \det I = \det (P P^{-1}) = \det P \det (P^\top ) = (\det P)^2
    \end{align*}

    \begin{align*}
        \det P \det A &= \det \begin{pmatrix} L_{11} & 0 \\ L_{21} & I \end{pmatrix} \det \begin{pmatrix} U_{11} & U_{12} \\ 0 & S \end{pmatrix}  \\
        &= \det L_{11} \det I \det U_{11} \det S \\
        &= 1  \cdot 1  \cdot \det U_{11} \det S \quad&\text{car $L_{11}$ n'a que des 1 sur la diagonale} \\
    \iff \det A &= \det P \det A_{11} \det S \quad&\text{$ \times \det P$} \\
    \iff \det A &= \pm \underbrace{\det U_{11}}_\text{non nul car on a pu faire la LU} \det S \\
    \end{align*}

    Donc si $A$ inversible ssi $S$ inversible.

\item On pose $\tilde{ \cdot } = P \cdot $

    On a à résoudre:
    \begin{align*}
        PAX &= \tilde{B} \\
        \iff \begin{pmatrix} L_{11} \\ L_{21} & I \end{pmatrix} \begin{pmatrix} I & \\ & S \end{pmatrix}  \begin{pmatrix} U_{11} & U_{12} \\ & S \end{pmatrix} \begin{pmatrix} X_1 \\ X_2 \end{pmatrix} &= \begin{pmatrix} \tilde{B_1} \\ \tilde{B_2} \end{pmatrix}  \\
    \end{align*}

    \begin{enumerate}
        \item Résolution de $\begin{pmatrix} L_{11} \\ L_{12} & I \end{pmatrix} \begin{pmatrix} Z_1 \\ Z_2 \end{pmatrix} = \begin{pmatrix} \tilde{B_1} \\ \tilde{B_2} \end{pmatrix} $. On appelle ça la \emph{phase de condensation} .

            \begin{align*}
                \begin{pmatrix} L_{11} \\ L_{12} & I \end{pmatrix} \begin{pmatrix} Z_1 \\ Z_2 \end{pmatrix} &= \begin{pmatrix} \tilde{B_1} \\ \tilde{B_2} \end{pmatrix} \\
            \end{align*}

            \begin{itemize}
                \item $Z_1 &= L_{11}^{-1} \tilde{B_1}$: facile (forward substitution\footnote{dernière étape de l'algo de Gauss})
                \item $Z_2 = \tilde{B_2} - L_{12} Z_1$ : trivial
            \end{itemize}
        \item 2e phase

            \begin{align*}
                \begin{pmatrix} I \\ & S \end{pmatrix} \begin{pmatrix} Y_1 \\ Y_2 \end{pmatrix} = \begin{pmatrix} Z_1 \\ Z_2 \end{pmatrix} \iff \begin{cases}
                    Y_1 &=  Z_1 \\
                    Y_2 &= S^{-1} Z_2 \quad\text{méthode ``ad-hoc"} \\
                \end{cases}
            \end{align*}

        \item \emph{Phase d'expansion} 

            \begin{align*}
                
            \begin{pmatrix} U_{11} & U_{12} \\ & I \end{pmatrix} \begin{pmatrix} X_1 \\ X_2 \end{pmatrix} &= \begin{cases}
                U_{11}X_1 + U_{12}X_2 &= Y_1 \\
                X_2 &= Y_2 \\
            \end{cases} \iff \begin{cases}
                X_1 &= U_{11}^{-1} (Y_1 - U_{12} Y_2) \\
                X_2 &= Y_2 \quad\text{backward substitution} \\
            \end{cases} \\
            \end{align*}
    \end{enumerate}


\end{enumerate}


\subsection{Exercice 2 : Exhibition d'un presque-noyau \emph{(null space)} }

cf transparent 19/38

On fait l'hypothèse que $x=0$ et $\det U_{11} = \det A_{11} \neq 0$. On a $\ker(PA) = \ker A$ car  $A$ inversible.

\begin{align*}
    X \in \ker A &\iff PAX = 0 \\
                 &\iff \begin{pmatrix} L_{11} & \\ L_{21} & I \end{pmatrix} \begin{pmatrix} U_{11} & U_{12} \\ & 0 \end{pmatrix} \begin{pmatrix} X_1 \\X_2 \end{pmatrix} = \begin{pmatrix} 0 \\ 0 \end{pmatrix} 
                 &\iff \begin{pmatrix} U_{11} & U_{12} \\ & 0 \end{pmatrix} \begin{pmatrix} X_1 \\ X_2 \end{pmatrix} &= \begin{pmatrix} 0 \\ 0 \end{pmatrix} \\
                 &\iff U_{11} X_1 + U_{12} X_2 = 0 \\
                 &\iff X_1 = -U_{11}^{-1} U_{12} X_2 \\
                 &\iff X = \begin{pmatrix} -U_{11}^{-1} &U_{12} \\ I_r \end{pmatrix} X_2 \in \cM_n
\end{align*}

\paragraph{Vérification:}

\begin{align*}
    PA \begin{pmatrix} U_{11}^{-1} U_{12} \\ I_r \end{pmatrix} &= \begin{pmatrix} L_{11} \\ L_{12} I \end{pmatrix} \begin{pmatrix} -U_{11} U_{11}^{-1} U_{12} + U_{12} I_r \\ 0 \end{pmatrix}  \\
\end{align*}

j'ai décroché là…

\end{document}
