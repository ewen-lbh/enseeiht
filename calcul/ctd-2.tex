\documentclass{article}
\usepackage{cancel}
\usepackage{amsmath, amssymb, amsfonts}
\usepackage[binary-units]{siunitx}
\usepackage{tikz}
\usepackage{float}
\usepackage{pgffor}
\usepackage{import}
\usepackage{vwcol}
\usepackage{fontawesome}
\usepackage{stmaryrd}
\usepackage{multicol}
\usepackage{pdfpages}
\usepackage{transparent}
\usepackage{xcolor}
\usepackage{scalerel}
\usepackage{stackengine}
\usepackage{algpseudocode}
\newcommand{\diag}{\operatorname{diag}}
\newcommand{\card}{\operatorname{card}}
\newcommand{\tr}{\operatorname{tr}}
\newcommand{\rg}{\operatorname{rg}}
\renewcommand{\epsilon}{\varepsilon}
\newcommand{\equivalent}[1]{\underset{#1}{\sim}}
\newcommand{\R}{\mathbb{R}}
\newcommand{\Q}{\mathbb{Q}}
\newcommand{\C}{\mathbb{C}}
\newcommand{\N}{\mathbb{N}}
\newcommand{\Z}{\mathbb{Z}}
\newcommand{\cM}{\mathcal{M}}
\newcommand{\cO}{\mathcal{O}}
\newcommand{\dx}{\mathrm{d}x}
\newcommand{\dy}{\mathrm{d}y}
\newcommand{\dz}{\mathrm{d}z}
\newcommand{\dt}{\mathrm{d}t}
\newcommand{\df}{\mathrm{d}f}
\newcommand{\Sp}{\operatorname{Sp}}
\newcommand{\dangersign}[1][2ex]{%
  \renewcommand\stacktype{L}%
  \scaleto{\stackon[1.3pt]{\color{red}$\triangle$}{\tiny !}}{#1}%
}

\usepackage[a4paper,top=4cm,bottom=4cm,left=3cm,right=3cm,marginparwidth=1.75cm]{geometry}
% \newcommand{\incfig}[2][1]{%
%     \def\svgwidth{#1\columnwidth}
%     \import{./figures/}{#2.pdf_tex}
% }
% 
\newenvironment{theorem}[1][\unskip]{
	\paragraph{Théorème #1}

}{}

\newenvironment{proof}[1][\unskip]{
	\def\temp{#1}\ifx\temp\empty
		\paragraph{Preuve}
	\else
		\paragraph{Preuve \emph{(#1)}}
	\fi

}{}

\newenvironment{definition}[1][\unskip]{
	\paragraph{Définition: #1}

}{}

\newenvironment{warning}[1][\unskip]
{
	\vspace{1cm}
	\begin{minipage}[c]{0.1\linewidth}
	\dangersign[8ex] 
\end{minipage}%
\begin{minipage}[l]{0.9\linewidth}
}
{
	\end{minipage}
	\vspace{1cm}
}

% \pdfsuppresswarningpagegroup=1


\begin{document}
\section{Factorisation LU}

Quand on triangularise un système linéaire pour un pivot de Gauss, on fait des CLs des lignes et on peut donc écrire la matrice du système comme
\begin{align*}
    A = LU
\end{align*}

Avec $A$ la matrice du système initial, $U$ celle du système triangularisé et $L^{-1}$ les opérations effectuées sur $A$ pour obtenir $U$.

\subsection{Construction de $L$}

\begin{definition}[Matrice de transvection $T_{i, j}(\lambda)$]
    \begin{align*}
        T_{i, j}(\lambda) &= I + \lambda E_{ij} &i \neq j
    \end{align*}
\end{definition}

\subsubsection{Propriétés de la transvection}
\begin{description}
    \item[] 
    \item[Produit] 
        \begin{align*}
            \prod_{i>j} T_{i, j}(\lambda_{i, j}  ) &= \begin{pmatrix} 1 \\ & 1 \\ & | \\ & \lambda_{ \cdot , j} & 1 \\ & | & &  1 \end{pmatrix}  \\
        \end{align*}
    \item[Inverse] \begin{align*}
            T_{i, j}(\lambda)^{-1} &= T_{i, j}(-\lambda) \quad&\text{$T_{i, j} \in GL_n(\R)$ car $\operatorname{det} T_{i, j}(\lambda) = 1 $} \\
    \end{align*}
    \item[Produit de produits] 
        \begin{align*}
            \begin{pmatrix} 1 \\ | & 1 \\ \lambda_{ \cdot , 1} & & 1 \\ | & & & 1 \end{pmatrix}  \times 
            \begin{pmatrix} 1 \\ & 1 \\ & | &  1 \\ & \lambda_{ \cdot , 2} & & 1 \end{pmatrix}  \times \cdots &= \begin{pmatrix} 1 \\ | & 1  \\ \lambda_{ \cdot , 1} & | & 1 \\ | & \lambda_{ \cdot , 2} & \ddots & 1 \end{pmatrix}  \\
        \end{align*}

        \emph{{\bf Attention} l'ordre est important} 
\end{description}

\subsection{Existance et unicité}

\begin{theorem}
   \begin{align*}
       \exists L, U, \ A=LU &\iff \forall k\in \llbracket 1, n \llbracket, \det(A(1:k, 1:k)) \neq 0 
   \end{align*} 
   \begin{align*}
       \exists L, U, A=LU \implies (L, U) \text{ unique et }\ldots
   \end{align*}
\end{theorem}

\ldots



\end{document}
