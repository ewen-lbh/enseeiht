%! TEX root=cours.tex
\documentclass{article}
\usepackage{cancel}
\usepackage{amsmath, amssymb, amsfonts}
\usepackage{stmaryrd}
\usepackage{import}
\usepackage{pdfpages}
\usepackage{transparent}
\usepackage{xcolor}
\newcommand{\mult}{\operatorname{mult}} 
\newcommand{\trivial}{$\mathbb{TRIVIAL}$} 
\newcommand{\Sp}{\operatorname{Sp}} 
\newcommand{\image}{\operatorname{im}} 
\newcommand{\rg}{\operatorname{rg}} 
\newcommand{\dim}{\operatorname{dim}} 
\newcommand{\cM}{\mathcal{M}}
\newcommand{\cT}{\mathcal{T}}
\newcommand{\cF}{\mathcal{F}}
\newcommand{\cO}{\mathcal{O}}
\newcommand{\cE}{\mathcal{E}}
\newcommand{\R}{\mathbb{R}}
\newcommand{\Q}{\mathbb{Q}}
\newcommand{\C}{\mathbb{C}}
\newcommand{\N}{\mathbb{N}}
\newcommand{\Z}{\mathbb{Z}}
\newcommand{\dx}{\mathrm{d}x}
\newcommand{\dy}{\mathrm{d}y}
\newcommand{\dz}{\mathrm{d}z}
\newcommand{\dt}{\mathrm{d}t}
\newcommand{\df}{\mathrm{d}f}


\newenvironment{theorem}[1][\unskip]{
	\paragraph{Théorème #1}

}{}

\newenvironment{proof}[1][\unskip]{
	\def\temp{#1}\ifx\temp\empty
		\paragraph{Preuve}
	\else
		\paragraph{Preuve \emph{(#1)}}
	\fi

}{}

\newenvironment{definition}[1][\unskip]{
	\paragraph{Définition: #1}

}{}

\begin{document}
\section{Rappels d'algèbre linéaire}

\begin{description}
	\item[Matrice de permutation] \begin{align*}
		P_\sigma := \left(\sum_{j=1}^{n} E_{\sigma(i),j}\right)_i
	\end{align*} avec $E_{i, j}$ la matrice élémentaire $(i, j)$.
\end{description}

\begin{theorem}[de Schur]
	$\forall  A \in \cM_n(\C), \exists U \in \cM_n(\C), \underbrace{U^\ast = U^{-1}}_{\text{unitaire}} \land U^\ast AU \in \cT_{n, s}(\C)$
\end{theorem}

\subsection{Méthodes itératives}

\paragraph{Principe}

On cherche une suite $x^{(p)}$ de $\R^{n}$ convergeant vers la solution de $Ax = b$:

\begin{align*}
	\forall x^{(0)}\quad x^{(p+1)} = H(x^{(p)})
\end{align*}

\paragraph{Propriétés}

\begin{itemize}
	\item $A$ n'est jamais modifiée
	\item Problème du suivi de la convergence et du choix du test d'arrêt
	\item Solution obtenue inexacte
	\item Matrice doit vérifier conditions de convergence
	\item Vitesse de convergence dépend de la matrice
\end{itemize}

\section{Décomposition en valeurs singulières}

\paragraph{Objectif}
Soit $A \in \cM_{n}(\R)$, $n \ge 1$.

On pose $
	A = U \Sigma V^\top
$  avec $\begin{cases}
U \in \cO_n(\R) \\
V \in \cO_n(\R) \\
\Sigma \in \cM_n(\R)
\end{cases}$

On a 

\[
	\Sigma = \begin{pmatrix} \sigma_i & 0 & \cdots & | & 0 & \cdots & 0 \\ 0 & \sigma_i & \cdots & | & 0 & \cdots & 0 \\ 0 & 0 & \ddots & | & 0 & \cdots  & 0 \\ \hline 0  & 0  & \cdots 0 & | & 0 & \cdots & 0 \\ 0  & 0 & \cdots 0 & | & 0 & \cdots & 0\end{pmatrix} 
\] 


\subsection{SVD d'une matrice}

\paragraph{SVD} Singular value decomposition

Soit $A \in \cM_n(\R)$ telle que $\operatornam{rg}A \ge  1$


\subsubsection{Propriétés}

\begin{enumerate}
	\item $A$ est symétrique réelle semi-définie-positive i.e. $\forall x \in \R^n,\ x^\top (A^\top A)x \ge 0$ i.e. $\forall x\in \R^n,\ \|Ax\|^2\ge 0$
	\item $\operatorname{Sp}(A) \subset \R_+$
	\item $A^\top A$ est orthoDZ: $\begin{cases}
		\ker(A^\top A) &= \ker A \\
		\image(A^\top A) &= \image (A^\top) \\
	\end{cases}$

	\begin{proof}[de 3.]
		\begin{itemize}
			\item $\ker A \subset  \ker(A^\top A)$: \trivial
			\item $\ker A \supset \ker(A^\top A)$
			Soit $x\in \ker(A^\top A)$.

			On a \begin{align*}
				A^\top A x &= 0 \\
				\implies x^\top (A^\top A x) &= 0 \\
				\implies \|Ax\|^2 &= 0 \\
				\implies x &\in \ker A
			\end{align*}

			\item $\image A \subset  \image(A^\top A)$ 
				\begin{align*}
					\dim \ker (A^\top A) + \rg (A^\top A) &= n \\
					\implies \rg(A^\top A) &= n - \dim \ker(A^\top A) \\
					&= n - \dim \ker A \quad&\text{car $\ker A = \ker (A^\top A)$} \\
					&=  \rg A \\
					&= \rg (A^\top) \\
					\implies \image(A^\top A) &\subset   \image A \\
				\end{align*}
			\item $\image A \supset  \image(A^\top A)$ : \trivial
		\end{itemize}
	\end{proof}

	\item $A ^\top A \in GL_n(\R) \iff \rg A = n$
		\begin{proof}[de 4.]
		\begin{align*}
			A^\top A \in GL_n(\R) &\iff \ker(A^\top A) = \{0\}  \\
					      &\iff \ker A = \{0\} \\
					      &\iff \rg A = n \quad&\text{par théorème du rang}
		\end{align*}
	\end{proof}
\end{enumerate}

\subsection{Construction de la SVD de $A$}

Soit $A \in \cM_{m,n}(\R)$ tel que $\rg A \ge 1$. On note $r := \rg A$.

\begin{itemize}
	\item $\dim \ker (A^\top A) = n - r$ par théorème du rang. 

		Donc $0 \in \Sp(A^\top A)$ et $\mult_{A^\top A}(0) = n -r$
	\item On note $(\lambda_i)_{i\in \llbracket 1, n \rrbracket} = \Sp(A^\top A) \subset \R_+^\ast$, {\bf telles que $i < j \implies \lambda_i \le  \lambda_j$}

		On note $(v_i)_{i\in \llbracket 1, r\rrbracket}$ une bond DZante de $A^\top A$ associée à $(\lambda_i)_{i\in \llbracket 1, n \rrbracket}$

		Et on pose enfin

		\begin{align*}
		\cE = \left(\underbrace{v_1, \ldots, v_r}_{\text{bon de $(\ker A)^\top$}}, \underbrace{v_{r+1}, \ldots, v_n}_{\text{bon de $\ker A$}} \right)
		\end{align*}


		\item 
		On pose $V = (v_1, \ldots, v_r) \in \cO_n(\R)$
		Et \[
			\forall i\in \llbracket 1, n\rrbracket u_i = \frac{1}{\sqrt{\lambda_i} } A v_i \in \R^n 
		\] 

		\begin{align*}
			\forall (i, j) \in \llbracket 1, r \rrbracket^2\quad \left<U_i, U_j \right> &= \frac{1}{\sqrt{\lambda_i \lambda_j} } \left<Av_i, Av_j \right> \\
												    &= \frac{1}{\sqrt{\lambda_i \lambda_j} } \left<\underbrace{A^\top A v_i}_{\lambda_i v_i}, A^\top Av_j \right> \\
												    &= \sqrt{\frac{\lambda_i}{\lambda_j}} \underbrace{\left<v_i, v_j \right>}_{\delta_{ij}} \quad&\text{car $(v_i)_i$ orthonormée} \\
												    &= \delta_{ij} \quad&\text{car si $i=j$, ça fait 1, sinon ça fait 0} \\
		\end{align*}

		Donc $(U_i)_i$ est une famille orthonormée de $\R^m$
	
		Donc $\rg (U_i)_{i\in \llbracket 1, n \rrbracket} = r$

		Or $(u_i)_i \subset \image A$

		D'où $\Vect(u_i)_i \subset  \image A$ et $(\rg u_i)_i = \rg A$

		Donc  $(\Vect U_i)_i = \image A$ 

		 et $(U_i)_i$ bon de $\image A$.

		 D'où  $0\in \Sp(A^\top A)$ avec $\image A$

		 On la complète en  $(U_i)_{i\in \llbracket 1, m\rrbracket}$ bon de $\R^m$

		 \[
			 \cF = \left( \underbrace{u_1, \ldots, u_{r}}_{\text{bon de $\image A$ }}, \underbrace{u_{r+1}, \ldots, u_m}_\text{bon de $(\image A)^\top$} \right) 
		 \] 

		 Posons $U = (U_1, \ldots, U_m) \in \cO_n(\R)$

		 \paragraph{Remarque}
		 $\forall i\in \llbracket 1, r \rrbracket$
		 \begin{align*}
			 B u_i &= \mu_i u_i \\
		 \end{align*}


		 Avec $B, \mu_i$ à determiner.

		  \begin{align*}
			  A A ^\top U_i &= \frac{1}{\sqrt{\lambda_i} } \underbrace{A A^\top A v_i}_{=\lambda_i v_i\quad \text{par def de $v_i$}} \\
			  &= \sqrt{\lambda_i} A v_i  \\
			  &= \lambda_i \underbrace{\left( \frac{1}{\sqrt{\lambda_i} } A v_i \right) }_{U_i} \\
			  &= \lambda_i U_i \\
		 \end{align*}


		 D'où $U_i$ vecteur propre de $A A^\top$ associé à  $\lambda_i$

		 \item  On pose \begin{align*}
		 	\Sigma &= U^\top AV \\
			       &= U^\top \left(\underbrace{Av_1, \ldots, Av_r}_{\sqrt{\lambda_i} u_i }, \underbrace{A v_{r+1}, \ldots, A v_n}_{0\ \text{car $v_i \in \ker A$}}\right) \\
			       &= \begin{pmatrix} u_1^\top \\ \vdots \\ u_m^\top \end{pmatrix} \begin{pmatrix} \sqrt{\lambda_j}  \left<u_i, u_j \right> | (0) \\ \hline \sqrt{\lambda_j} \left<u_i, u_j \right> | (0) \end{pmatrix}   \\
			       &= \begin{pmatrix} \sqrt{\lambda_1} &  & (0) &  | & \\  & \ddots &  & |  & (0) \\  (0) &  & \sqrt{\lambda_r}  & | & \\ \hline & (0) & & | & (0) \end{pmatrix}  \\
			       \implies A &= U \sigma V^\top \\
		 \end{align*}


		 \begin{definition}[SVD de $A$]
		 	Décomposition de la forme \[
		 		A = U \Sigma V^\top
		 	\] 

			Avec $U, \Sigma, V$ comme définies précédemment

			On appelle valeurs singulières notées  $(\sigma_i)_i$ les valeurs propres racines carrées
		 \end{definition}


\end{itemize}


%\begin{align*}
%	A\quad {}^\top A^\top,\  A \\
%	A\quad{}^\top k\tau,\ B
%\end{align*}
%

\end{document}
