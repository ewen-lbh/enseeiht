\documentclass{article}

% Language setting
% Replace `english' with e.g. `spanish' to change the document language
\usepackage[french]{babel}
\usepackage[utf8]{inputenc}
\newcommand{\TF}{\operatorname{TF}}
\newcommand{\NRZ}{\operatorname{NRZ}}
\usepackage{stmaryrd}

\usepackage{float}

% Set page size and margins
% Replace `letterpaper' with `a4paper' for UK/EU standard size
\usepackage[a4paper,top=2cm,bottom=2cm,left=3cm,right=3cm,marginparwidth=1.75cm]{geometry}

% Useful packages
\usepackage{siunitx}
\usepackage{amsmath}
\usepackage{pgfplots}
\pgfplotsset{compat=newest}
\usetikzlibrary{plotmarks}
\usetikzlibrary{arrows.meta}
\usepgfplotslibrary{patchplots}
\usepackage{grffile}
\pgfplotsset{plot coordinates/math parser=false}
\newlength\figureheight
\newlength\figurewidth
  
\usepackage{graphicx}
\usepackage{hyperref}
%\newcommand{\TF}{\operatorname{TF}}
\newcommand{\sinc}{\operatorname{sinc}} 




\title{
\includegraphics[width=0.2\textwidth]{n7.png}
\\[1cm]
Rapport de projet -- Calcul Scientifique

}
\author{Ewen Le Bihan}

\date{ENSEEIHT, département Sciences du Numérique}

\begin{document}

\maketitle

\setcounter{tocdepth}{3}
\tableofcontents

\section{Test des fournitures}

On teste avec des matrices de types

\begin{description}
    \item[1] $\begin{pmatrix} 1 & & & & (0) \\ & 2 & & & \\ & & 3 & & \\ & & & \ddots & \\ (0) & & & & n \end{pmatrix}$
    \item[2] $\operatorname{diag}(\verb|random(1e-10, 1)|)$
    \item[3] $\operatorname{diag}\left( (10^{5})^{- \frac{i-1}{n-1}} \right)_{i\in \llbracket 1, n \rrbracket} $
    \item[4] $\operatorname{diag}\left( 1 - (1 - 10^{-2}) \frac{i-1}{n-1} \right)_{i\in \llbracket 1, n \rrbracket} $
\end{description}

\begin{table}[H]
    \centering
    \label{tab:vitesse-algos}
    \begin{tabular}{c|cccc}
        Type / Algo & 1 & 2 & 3 & 4 \\\hline
        \verb|eig| (10) & $\SI{20}{ms}$ & $\SI{0}{ms}$ & $\SI{10}{ms}$ & $\SI{10}{ms}$ \\
        \verb|power| (11) & $\SI{1,77}{s}$ & $\SI{30}{ms}$ & $\SI{60}{ms}$ & $\SI{1,74}{s}$ \\
        \verb|v0| (0) & 
    \end{tabular}
    \caption{Vitesse de calcul des différents algorithmes}
\end{table}


\end{document}
